\documentclass[10pt,a4paper,twoside]{article}
\usepackage{latexsym}
\textwidth=6.5in
\oddsidemargin=0.1cm
\evensidemargin=-0.9cm
\parskip=.5mm
\lineskip=0mm
\usepackage{amsmath}
\usepackage{amssymb}
\usepackage{amsfonts}
\def\Fq{{\mathbb{F}}_q}
\def\oFq{\overline{\mathbb{F}}_q}
\def\Cl{{\mathcal{C}}}
\def\Tl{{\mathcal{T}}}
\hfuzz=3pt
\newtheorem{theorem}{Theorem}[section]
\newtheorem{Proposition}{Proposition}[section]
\newcommand{\sopra}[2]{\genfrac{}{}{0pt}{}{#1}{#2}}
\pagenumbering{arabic} \pagestyle{myheadings}
\markboth{C. Malvenuto and F. Pappalardi}{Degrees of PP}
\title{Enumerating Permutation Polynomials I:\\  Permutations with non--maximal degree}
\author{Claudia Malvenuto and Francesco Pappalardi}
\date{\today}
\begin{document}
\maketitle
\begin{abstract}
Let $\Cl$ be a conjugation class of permutations of a finite field
$\Fq$. We
consider the function $N_{\Cl}(q)$ defined as the number of permutations in
$\Cl$ for which the associated permutation polynomial has degree $<q-2$. In 1969,
C. Wells proved a formula for $N_{[3]}(q)$ where $[k]$ denotes the conjugation class of
$k$--cycles. We will prove formulas for $N_{[k]}(q)$ where $k=4, 5, 6$ and for the
classes of permutations of type $[2\ 2], [3\ 2], [4\ 2], [3\ 3]$ and $[2\ 2\ 2]$.
Finally
in the case $q=2^n$, we will prove a formula for the classes of permutations
which are product
of $2$--cycles.
\end{abstract}
\section{Introduction.}
Let $q$ be a power of a prime and denote with $\Fq$ the finite field with $q$
elements. If $\sigma$ is a permutation of the elements of $\Fq$, then one can
associate to $\sigma$ the polynomial in $\Fq[x]$
\begin{equation}\label{definiz}
f_\sigma(x)=\sum_{c\in\Fq}\sigma(c)\left(1-\left(x-c\right)^{q-1}\right).
\end{equation}
The following properties are easy to verify:
\begin{enumerate}
\item $f_\sigma(b)=\sigma(b)$ for all $b\in\Fq$;
\item The degree $\partial(f_\sigma)\leq q-2$ (since the sum of all the
elements of $\Fq$ is zero).
\end{enumerate}
Notice that $f_\sigma$ is the unique polynomial in $\Fq$ with these
two properties. Such
polynomials are called \emph{permutation polynomials}.

Basic literature on permutation polynomials can be found in the book
of Lidl and Niederreiter \cite{LN}. Various applications of permutation polynomials to
cryptography have been described. See for example \cite{LB,LMu}.
Lidl and Mullen in \cite{LM1,LM2}, (see also \cite{M}),
describe a number of open problems regarding permutations polynomials: among these,
the problem of enumerating permutation polynomials by their degree.

We denote by
$$S_\sigma = \{x\in \Fq\ | \sigma(x)\neq x\}$$
the set of those elements of $\Fq$ which are moved by $\sigma$.
Our first remark is that
\begin{equation}\label{lb}
\partial(f_\sigma)\geq q-|S_\sigma| \quad \textrm{ if }\quad  \sigma\neq\textrm{id}.
\end{equation}
To see this it is enough to note that the polynomial $f_\sigma(x)-x$ has as roots all
the elements of $\Fq$ fixed by $\sigma$, that is which are not in $S_\sigma$.
Therefore, if not identically zero, it has to have degree at least $q-|S_\sigma|$.
An immediate consequence is that all transpositions give rise to permutation polynomials
of degree exactly $q-2$. This fact was noticed by C. Wells in \cite{W},
where he also proved the following:

\begin{theorem}[Wells -- 1969] If $q>3$,
the number of $3$--cycles permutations $\sigma$ of $\Fq$ such that
$\partial(f_\sigma)\leq q-3$ (which in view of (\ref{lb}) implies
$\partial(f_\sigma)= q-3$) is
$$
\left\{\begin{array}{ll}
\frac{2}{3}q(q-1) & \textrm{if\ } q\equiv1\pmod3\\
\\
0 & \textrm{if\ } q\equiv2\pmod3\\
\\
\frac{1}{3}q(q-1) & \textrm{if\ } q\equiv0\pmod3.\\
\end{array}\right.
$$
\end{theorem}

The goal of this note is to extend the previous result to the cases of $k$--cycles
($k=4,5, 6$) and to other classes of permutations.

Let $A_i(\sigma)$ denote the coefficient of $x^{q-1-i}$ in the polynomial $f_\sigma(x)$
of (\ref{definiz}), that is
$$f_\sigma(x)=A_1(\sigma)x^{q-2}+A_2(\sigma)x^{q-3}+
\cdots+A_{q-2}(\sigma)x+A_{q-1}(\sigma).$$

From (\ref{definiz}) it is very easy to derive the formula for the coefficient
of the leading term
$$A_1(\sigma)=-\sum_{c\in\Fq}\sigma(c)c.$$
Since the squares of all the elements of $\Fq$ add up to zero when $q>3$,
the previous formula can be written as
\begin{equation}\label{lavoro}
A_1(\sigma)=\sum_{c\in\Fq}(c-\sigma(c))c=
\sum_{c\in S_\sigma}(c-\sigma(c))c=\sum_{j=1}^t
\sum_{k=1}^{l_j}(c_{j,k}-c_{j,k+1})c_{j,k}.
\end{equation}
where we write $\sigma$ as the product of disjoint cycles (of length larger than $1$):
$$\sigma=(c_{1,1},\ldots, c_{1,l_1})(c_{2,1},\ldots, c_{2,l_2})
\ldots(c_{k,1},\ldots, c_{t,l_t}).$$
Note that in (\ref{lavoro})
we set $c_{j,l_j+1}=c_{j,1}$.

It is well known that a conjugation class of permutations is determined by a cycle
decomposition and we will denote by $[l_1\ l_2\ \cdots\ l_k]$ ($l_i>1$)
a conjugation class.
For example $[k]$ denotes the class of $k$--cycle permutations.
If $\sigma\in[l_1\ l_2\ \cdots\ l_k]$
then $|S_\sigma|=l_1+l_2+\cdots+l_k$.

For a given conjugation class $\Cl$ here we will consider the following function:
$$N_{\Cl}(q)=\left|\left\{\sigma\in\Cl \ |\
\partial(f_\sigma)< q-2\right\}\right|$$
that counts the number of permutations in  $\Cl$ whose degree is non--maximal.

If we denote by $\eta$ the quadratic character of $\Fq^*$ ($q$
odd) and set $\eta(0)=0$, for $q>3$, Wells' result can be written as
$$
    N_{[3]}(q)=
\left\{
    \begin{array}{ll}
\frac{1}{3}q(q-1)\left(1+\eta(-3)\right) & \textrm{ if } q>3 \textrm{ odd }\\
    \\
    \frac{1}{3}2^n(2^n-1)(1+(-1)^n)    & \textrm{ if } q=2^n, q>2.
    \\
    \\ 0 & q\leq 3
    \end{array}
\right.
$$

Sometimes it is also useful to denote the class of those permutations
that are the product
of $m_1$  cycles of length $1$, $m_2$ cycles of length $2$, $\ldots$, $m_t$
cycles of length $t$,
as $(m_1;m_2;\ldots;m_t)$ where $m_1+2m_2+\cdots+tm_t=q$. With this notation
we have that if
$\sigma\in\Cl$ then $|S_\sigma|=q-m_1$ and
\begin{equation}\label{conjusize}
|\Cl|=\frac{q!}{m_1!1^{m_1}m_2!2^{m_2}\cdots m_t!t^{m_t}}
\end{equation}

Now for $\Cl=[l_1\ \ldots \ l_k]$ and $c=l_1+\cdots+l_k$ consider the
polynomial in
$c$ indeterminates (cfr. (\ref{lavoro}))
\begin{equation}\label{polinomiaccio}
A_{\Cl}(x_1,\ldots,x_c)=
\sum_{\sopra{i=1}{i\not\in \{l_1,l_1+l_2,\ldots, c\}}}^c (x_i-x_{i+1})x_i
+\sum_{i=1}^k(x_{l_1+\cdots+l_i}-x_{l_1+\cdots+l_{i-1}+1})x_{l_1+\cdots+l_{i}}.
\end{equation}
From the above discussion we deduce that
\begin{equation}
\label{gen2}
N_{\Cl}(q)=\frac{1}{m_2!2^{m_2}\cdots m_t!t^{m_t}}
\left|\left\{{\underline x}\in \Fq^{c}: \quad {\underline x}
\textrm{ has coordinates all distinct and }
A_{\Cl}({\underline x})=0
\right\}\right|.
\end{equation}
Indeed by (\ref{lavoro}) every permutation counted by $N_{\Cl}(q)$ gives rise
to a root of
(\ref{polinomiaccio}); furthermore by cyclically permuting the elements of every cycle
and by permuting
different cycles of the same length we get different roots of (\ref{polinomiaccio})
that correspond
to the same element of $N_{\Cl}(q)$.

\section{4--cycle polynomials.}

Let us now consider the case of 4--cycles. We will show the following:

\begin{theorem} Suppose $q>3$ is odd. Then
$$N_{[4]}(q)=\frac{1}{4}q(q-1)\left(q-5-2\eta(-1)-4\eta(-3)\right).$$
Suppose $q=2^n$ with $n\geq2$. Then
$$N_{[4]}(2^n)=\frac{1}{4}2^n(2^n-1)(2^{n}-4)(1+(-1)^n).$$
\end{theorem}

\noindent\textbf{Proof.} By (\ref{gen2}), any $a,\,b,\,c,\,d\in\Fq$
(all distinct)
such that the 4--cycle $(a\ b\ c\ d)$ is counted by $N_{[4]}(q)$,
have to satisfy the equation:
\begin{equation}\label{4cicl}
(a-b)a+(b-c)b+(c-d)c+(d-a)d=0.
\end{equation}
For every of the $q(q-1)$ fixed choices of $a$ and $b$ distinct in $\Fq$,
substituting in (\ref{4cicl}) $c=x(b-a)+a$, $d=y(b-a)+a$, we obtain the equation
\begin{equation}\label{normal}
  (1-x)+(x-y)x+y^2=0
\end{equation}
Since the conditions that $a,\,b,\,c$ and $d$ are all distinct, are equivalent
to the conditions that $x,\,y\not\in\{0,1\}$ and $x\neq y$, taking into account
that every circular permutation of a solution of (\ref{4cicl}) gives rise to the
same $4$--cycle, we have
$$N_{[4]}(q)=\frac{1}{4}q(q-1)P_{[4]}(q)$$
where
$$P_{[4]}(q)=\left|\left\{(x,y)\
\left|\ x,y\in\Fq\setminus\{0,1\},\ x\neq y,\ (1-x)+(x-y)x+y^2=0\right.\right\}
\right|$$

Assume $q$ odd. The affine conic $(1-x)+(x-y)x+y^2=0$ has
\begin{equation}\label{cla1}
q-\eta(-3)
\end{equation}
rational points over $\Fq$. This can be seen by noticing that
the associated projective conic has $q+1$ points and its points at infinity
over $\oFq$ are $[1,\omega,0]$,
$[1,\omega^2,0]$ (where $\omega,\omega^2\in\oFq$ are
the roots of $T^2-T+1$, i.e. $\omega=\frac{-1+\sqrt{-3}}{2}$)  which
are rational if and only if $\eta(-3)\neq-1$.

From (\ref{cla1}) we have to subtract the number of rational points $(x,y)$ verifying
one of the conditions $x, y\in\{0,1\}$ or $x=y$. All these conditions give rise to
the following (at most) 10 points over $\oFq$:
$$(0,i),\ \ (0,-i),\ \ (1+i,1),\ \ (1-i,1)$$
$$(1,\omega),\ \ (1,{\omega}^2),\ \ (\omega,0),\ \ ({\omega}^2,0),\ \
(\omega,\omega),\ \ ({\omega}^2,{\omega}^2)$$
where $i$ is a root of $T^2+1$. The number of the above points which are
rational over $\Fq$ is
$$2\left[1+\eta(-1)\right]+
  3\left[1+\eta(-3)\right]$$
Subtracting the above quantity from (\ref{cla1}), we obtain the statement for $q$ odd.

If $q=2^n$, then first note that the affine transformation $y=Y+1, x=X+Y$ maps the
affine conic (\ref{normal}) to
$$X^2+Y^2+X Y=0.$$
Therefore the number of solutions of (\ref{normal}) is $2^{n+1}-1$ if $n$ is
even and $1$
if $n$ is odd. We can write this number with one formula
by:
$$(1+(-1)^n)2^n-(-1)^n$$
Finally the five conditions $x,y\neq0,1$ and $x\neq y$ are equivalent to
$Y\neq 0,1$, $X\neq Y$, $X\neq Y+1$ and $X\neq 1$.
So the total number of rational points over $\mathbb F_{2^n}$
that have to be removed is $1+3(1+(-1)^n)$.
This concludes the proof.
\bigskip\hfill $_\square$\bigskip

\section{Product of two disjoint transposition polynomials.}

Let us now consider the case of permutations which are product of two disjoint
transpositions,
that is whose conjugation class is $[2\ 2]$. In his paper of 1969, C. Wells announces
the following formula. For completeness we will prove it here.

\begin{theorem} Suppose $q>3$ is odd. Then
\begin{equation}\label{22}
N_{[2\ 2]}(q)=\frac{1}{8}q(q-1)(q-4)\left\{1+\eta(-1)\right\}
\end{equation}
and if $q=2^n$, then
$$N_{[2\ 2]}(2^n)=\frac{1}{8}2^{n}(2^n-1)(2^{n}-2).$$
\end{theorem}

\noindent\textbf{Proof.}  Following the same lines of previous section, if $q$ is odd,
 by (\ref{lavoro}), a permutation $(a\ b)\ (c\ d)$ with degree  $< q-2$
has to satisfy the equation:
\begin{equation}\label{toto}
(a-b)^2+(c-d)^2=0.
\end{equation}
It is clear that this equation has a (admissible) solution if and only if $-1$
is a square
in $\Fq$. In this case, if $q$ is odd, for any of the $q(q-1)$ fixed choices
of the first two variables $(a_0,b_0)$ we have the linear equations:
$$c=d\pm\sqrt{-1}(a_0-b_0)$$
where $d$ can assume all possible values except the ones in the set
$$\{a_0,b_0,a_0\mp\sqrt{-1}(a_0-b_0),b_0\mp\sqrt{-1}(a_0-b_0)\}.$$
This analysis yields $2q(q-1)(q-4)$ solutions.
If $q=2^n$ with $n\geq2$, then (\ref{toto}) becomes
$$a^2+b^2+c^2+d^2=(a+b+c+d)^2=0$$
and for any of the $q(q-1)(q-2)$ choices of $a, b, c$, the value $d=a+b+c$
is not equal to $a$ or $b$ or $c$.

Finally regardless of the characteristic, since the $8$ solutions
$$
(a,b,c,d), (b,a,c,d), (a,b,d,c), (b,a,d,c)$$
$$ (c,d,a,b,), (d,c,a,b), (c,d,b,a), (d,c,b,a)
$$
give rise to the same permutation, we deduce the formula for $N_{[2\ 2]}(q)$.
\bigskip\hfill $_\square$\bigskip


\section{5--cycle polynomials.}

In this section we will see the limits of the approach under consideration.

\begin{theorem} Let $q$ be a power of a prime $p$. Then
$$N_{[5]}(q)=\frac{1}{5}q(q-1)P_{[5]}(q)$$
where
\begin{equation}\label{laprima}
\left\{
    \begin{array}{ll}
    P_{[5]}(q)=q^2-\left(9-\eta(5))\right)q+26+5\eta(-7)+15\eta(-3)+15\eta(-1)&
\quad\textrm{ if } p>3\\
        &\\
    P_{[5]}(3^n)=3^{2n}-(9-6(-1)^n)3^n+26+35(-1)^n & \\
    &\\
    P_{[5]}(2^n)=(2^n-3-(-1)^n)(2^n-6-3(-1)^n).&
    \end{array}
\right.
\end{equation}
\end{theorem}

\noindent\textbf{Proof.} Using, as in the previous sections, identity
(\ref{gen2}) and a transformation
that eliminates two of the variables, we deduce that
$$N_{[5]}(q)=\frac{1}{5}q(q-1)P_{[5]}(q)$$
where $P_{[5]}(q)$ is the number of solutions $(x,y,z)$ of
\begin{equation}\label{quadrica5ciclo}
1-x+{x}^{2}-x y+{y}^{2}-y z+{z}^{2}=0
\end{equation}
with $x,y,z\not\in\{0,1\}$, $x\neq y$, $y\neq z$ and  $z\neq x$.

If $q$ is odd, the affine transformation
$$(x,y,z)\mapsto (x-y+z-2^{-1},x-z-2^{-1},y-2^{-1})$$
yields to the quadric
$$x^2+y^2+z^2=5/4$$
which has the same number of rational points of (\ref{quadrica5ciclo}).
%the matrix associated to the previous quadric surface is
%$$\left(\begin{array}{cccc}
%2 & -1 & 0 & 0 \\
%-1 & 2 & -1 & 0 \\
%0 & -1 & 2 & -1 \\
%0 & 0 & -1 & 2 \\
%\end{array}\right)$$
%and it has determinant equal to $5$.
Therefore the number of rational points
on (\ref{quadrica5ciclo}), which can be calculated with the standard
formulas that can be found in \cite[Theorem 6.27]{LN}, is
\begin{equation}\label{primo}
q^2+q\eta(5).
\end{equation}
If $q$ is even then the quadric (\ref{quadrica5ciclo}) is equivalent
via the transformation $x=X+Y+Z, y=Y+1, z=X+Y+1$ to the quadric
$$X+YZ+Y^2+Z^2=0$$
which has clearly $q^2$ points.

From (\ref{primo}) we have to subtract the number of points on the $9$ quadratic
curves obtained intersecting the previous surface with the following $9$ planes:
$$x=0,\ \ x=1,\ \ y=0,\ \ y=1,\ \ z=0,\ \ z=1,\ \ x=y,\ \  x=z,\ \ y=z.$$
Note that all these quadrics are non degenerate, except in characteristics
$2$ and $3$: hence
the cases $(6,q)=1, q=2^n$ and $q=3^n$ have to be considered separately.
The equations of the $9$  curves obtained in this way and
the number of their points (calculated again with
the formulas in \cite[Theorem 6.26 -- 6.32]{LN}) are listed in the following table.
\begin{center}
\begin{tabular}{|l|c|l|c|c|c|}
\hline
   & Plane &        \multicolumn{1}{|c|}{Curve}          &
\multicolumn{3}{c|}{Number of points}\\
\cline{4-6}
&&& $(6,q)=1$ & $q=2^n$ & $q=2^n$\\
\hline
 1 & $x=0$ & $1+{y}^{2}-yz+{z}^{2}=0$ &
$q-\eta(-3)$ & $3^n(1+(-1)^n)$ & $2^n-(-1)^n$\\
\hline
 2 & $x=1$ &    $1-y+y^2-yz+z^2=0$    &
$q-\eta(-3)$ & $3^n$ &$2^n(1+(-1)^n)-(-1)^n$\\
\hline
 3 & $y=0$ &     $1-x+x^2+z^2=0$      &
$q-\eta(-1)$  & $3^n(1+(-1)^n)-(-1)^n$ &$2^n$\\
\hline
 4 & $y=1$ &    $2-2x+x^2-z+z^2=0$    &
$q-\eta(-1)$  & $3^n(1+(-1)^n)-(-1)^n$ & $2^n$\\
\hline
 5 & $z=0$ &    $1-x+x^2-xy+y^2=0$    &
$q-\eta(-3)$ & $3^n$ &$2^n(1+(-1)^n)-(-1)^n$\\
\hline
 6 & $z=1$ &   $2-x+x^2-xy+y^2-y=0$   &
$q-\eta(-3)$ & $3^n(1+(-1)^n)$&$2^n-(-1)^n$ \\
\hline
 7 & $x=y$ &    $1-x+x^2-xz+z^2=0$    &
$q-\eta(-3)$  & $3^n$&$2^n(1+(-1)^n)-(-1)^n$\\
\hline
 8 & $x=z$ &   $1-z+2z^2-2yz+y^2=0$   &
$q-\eta(-1)$ & $3^n(1+(-1)^n)-(-1)^n$ &$2^n$\\
\hline
 9 & $y=z$ &    $1-x+x^2-xz+z^2=0$    &
$q-\eta(-3)$  & $3^n$& $2^n(1+(-1)^n)-(-1)^n$\\
\hline
\end{tabular}
\end{center}
Therefore the total number of points to subtract from (\ref{primo}) is
\begin{equation}\label{secondo}
\left\{
\begin{array}{ll}
9q-6\eta(-3)-3\eta(-1) &
\textrm{ if } (6,q)=1;\\
9\cdot3^n+5(-1)^n3^n-3(-1)^n&\textrm{ if } q=3^n;\\
9\cdot2^n+4(-1)^n2^n-6(-1)^n&\textrm{ if } q=2^n.\\
\end{array}
\right.
\end{equation}

The final step is to add back the number of points that we have
subtracted too many times,
that is the points that lie in the intersection of two or more of the previous curves.

If the characteristic is odd consider the following $19$ pairs of points of the quadric
(\ref{quadrica5ciclo}) over $\oFq$. Beside each pair of points
we have written the number of quadrics of the previous table to which the points belong.

\begin{center}
\begin{tabular}{|l|c|r|}
\hline
1&$(0,0,\pm i)$   &  3 \\
2&$\left(0,1,\frac{1\pm\sqrt {-7}}{2}\right)$   & 2 \\
3&$(0,\pm i,0)$   &  3 \\
4&$\left(0,\frac{1\pm\sqrt {-7}}{2},1\right)$   &  2  \\
6&$(1,1,\omega^{\pm1})$   &  3 \\
7&$(1,\omega^{\pm1},0)$   &  2  \\
8&$(1,1\pm i,1)$   &  3\\
9&$(\omega^{\pm1},\omega^{\pm1},\omega^{\pm1})$   & 3 \\
10&$(\omega^{\pm1},\omega^{\pm1},0)$   & 2  \\
\hline
\end{tabular}\hspace{2cm}
\begin{tabular}{|l|c|r|}
\hline
 11&$(1\pm i,1\pm i,1)$   &  2\\
 12&$(0,\pm i,\pm i)$   &  2\\
 13&$(1,\omega^{\pm1},\omega^{\pm1})$   &  2 \\
 14&$\left(\frac{1\pm\sqrt {-7}}{4},0,\frac{1\pm\sqrt {-7}}{4}\right)$   & 2 \\
 15&$\left(\frac{3\pm\sqrt {-7}}{4},1,\frac{3\pm\sqrt {-7}}{4}\right)$   &  2\\
 16&$(1\pm i,1,0)$   &  2 \\
 17&$\left(\frac{1\pm\sqrt{-7}}{2},0,1\right)$   & 2 \\
 18&$(\omega^{\pm1},0,0)$ & 3\\
 19&$(1\pm i,1,1)$ & 3\\
\hline
\end{tabular}
\end{center}
The total number of points that has to be subtracted from (\ref{secondo}) is therefore
\begin{equation}\label{terzo}
26+5\eta(-7)+12\eta(-1)+9\eta(-3) \quad \textrm{ if } (6,q)=1
\end{equation}
and
\begin{equation}
26+17(-1)^n \quad \textrm{ if } q=3^n.
\end{equation}
If the characteristic is even and $\xi$ is a root of $T^2+T+1$ in
$\overline{\mathbb F}_{2^n}$,
then the corresponding points lying on more than one of the quadrics 1-9 are:
$$(0,0,1),\ (0,1,0),\ (0,1,1),\ (1,0,1)$$
each lying on 4 quadrics,
$$(\xi,\xi,\xi),\ (\xi,0,0),\ (1,1,\xi)$$
each lying on 3 quadrics and
$$(1,\xi,0),\ (\xi,\xi,0),\ (1,\xi,\xi)$$
each lying on 2 quadrics. The total number of points to be subtracted when $q=2^n$ is
\begin{equation}\label{quarto}
12+9(1+(-1)^n),
\end{equation}
because $\xi\in \mathbb F_{2^n}$ if and only if $n$ is even.

Collecting together the various quantities, we obtain the
formulas. \hfill $_\square$\bigskip

The same argument applied to permutations that are the product of a
3--cycle and a 2--cycle
lead to the following result whose proof we omit.

\begin{theorem}
 Let $q$ be a power of a prime $p$. Then
$$N_{[2\ 3]}(q) =\frac{1}{12}q(q-1)P_{[2\ 3]}(q)$$
where
$$P_{[2\ 3]}(q)=\left\{\begin{array}{ll}
   \left(q^2-(9+\eta(-3)+3\eta(-1))q+(24+6\eta(-3)+
18\eta(-1)+6\eta(-7))\right) & \textrm{if } p>3;\\
\\
(1+(-1)^n)(3^{2n}-9\cdot3^n+24) & \textrm{if } q=3^n;\\
\\
(2^n-3-(-1)^n)(2^n-6) & \textrm{if } q=2^n.
\end{array}\right.$$
\end{theorem}

%\noindent\textbf{Proof.} We will only outline the main steps since it is
%analogous to the ones of the previous sections.
%
%First
%$$N_{[2\ 3]}(q) =\frac{1}{12}q(q-1)\left|\left\{
%(x,y,z)
%\ \left|\ 1+x(x-y)+y(y-z)+z(z-x)=0,
%\genfrac{}{}{0pt}{}{\textrm{with } x,y,z\not\in\{0,1\},}
%{ x\neq y, y\neq z \text{ and }  z\neq x
%}\right.\right\}\right|.$$
%Note that $\#\{(x,y,z)
%\ |\ 1+x(x-y)+y(y-z)+z(z-x)=0\}=q(q-\eta(-3))$ and each of the nine
%conditions
%$$x=0,\ x=1,\ y=0,\ y=1,\ z=0,\ z=1,\ x=y,\ x=z,\ z=t$$
%corresponding to points to be removes gives one of the curves
%
%Finally the intersections of any two of the previous curves are in
%the table:
%
%and this completes the calculation for $q$ odd.

\section{General conjugation classes of permutations.}

It is not difficult in principle to generalize the 
inclusion--exclusion argument of the
previous sections to any conjugation class of permutation.

For example, if we want to compute $N_{[k]}(q)$, the number
of $k$--cycles permutations $\sigma$ of $\Fq$ such that
$\partial f_\sigma<q-2$, then (see
 (\ref{polinomiaccio})), we want to count the number of
$\sigma=(a_1,\ a_2,\ \ldots,\ a_k)$ for which
$$(a_1-a_2)^2+(a_2-a_3)^2+\cdots+(a_{k-1}-a_k)^2+(a_k-a_1)^2=0.$$
We perform the transformation $a_{i+2}=x_i(a_1-a_2)+a_1$, $i=1,\ldots,k-2$ and
simplify $(a_1-a_2)^2$. Taking into account that there are $q$ possibilities for $a_1$
and $q-1$ for $a_2$ and that circular permutations of the $a_i$'s give rise to the same
$\sigma$, we have that $N_{[k]}(q)$ equals $q(q-1)/k$ times the number
of rational points
of the quadric hypersurface
\begin{equation}\label{genreduced}
1+(x_1-1)^2+(x_2-x_1)^2+\cdots+(x_{k-3}-x_{k-2})^2+x_{k-2}^2=0
\end{equation}
with the $(k+1)(k-2)/2$ conditions that $x_i\neq 0,1$, $x_i\neq x_j$,
for $i,j=1,\ldots,k-2$
and $i\neq j$.

Let us assume for simplicity that $q$ is odd. Then
we can associate to the quadratic hypersurface in (\ref{genreduced}) the
$(k-1)\times(k-1)$ matrix:

\begin{equation}\label{matrix}M_k=
\left(\begin{array}{cccccc}
2 & -1 & 0 & 0 & \cdots & 0 \\
-1 & 2 & -1 & 0 & & 0\\
0 & -1 & 2 & -1 & &0\\
\vdots& & -1&  \ddots & -1\\
0 & & & -1& 2& -1\\
0 & & & & -1 & 2
\end{array}\right)
\end{equation}

It is easy to see that the determinant of $M_k$ equals $k$.
If $(q,k-1)=1$ then the transformation of variables:
$$(x_1,x_2,\ldots,x_{k-2})\mapsto \left(x_1-\frac{k-2}{k-1},x_2-
\frac{k-3}{k-1},\ldots,x_{k-2}
-\frac1{k-1}\right)$$
brings (\ref{genreduced}) to
$$x_1^2+(x_2-x_1)^2+\cdots+(x_{k-3}-x_{k-2})+x_{k-2}^2+\frac k{k-1}=0$$
which has matrix
$$
\left(\begin{array}{cccc}
k/(k-1) & 0 & \cdots & 0\\
0 & & & \\
\vdots  & & M_{k-1} & \\
0 & & &
\end{array}
\right).$$

From
\cite[Theorem 6.26 -- 6.27]{LN}
we find that if $(q,(k-1))=1$, the number of rational points of
(\ref{genreduced}) equals
\begin{equation}\label{akappa}
a_k(q)=
\left\{
\begin{array}{ll}
q^{k-3}+q^{(k-3)/2}\eta((-1)^{(k-1)/2}k) & \textrm{if $k$ is odd}\\
&\\
q^{k-3} +v(k)q^{(k-4)/2}\eta((-1)^{(k-2)/2}(k-1)) & \textrm{if $k$ is even}
\end{array}
\right.
\end{equation}
where $v(k)=-1$ if $(q,k)=1$ and $v(0)=q-1$.

If $(q,(k-1))>1$, then $k=1$ in $\Fq$. In this case we can count the number of
point on (\ref{genreduced}) as follows:
$$a_k(q)=\frac 1{(q-1)}\left(a'-a''\right)$$
where $a'$ is the number of solutions of the non--degenerate
quadric
$$x_0^2+(x_1-x_0)^2+(x_2-x_1)^2+\cdots+(x_{k-3}-x_{k-2})^2+x_{k-2}^2=0$$
and $a''$ is the number of solutions of the degenerate quadric
$$x_1^2+(x_2-x_1)^2+\cdots+(x_{k-3}-x_{k-2})^2+x_{k-2}^2=0$$
which is equivalent to the quadric (non--degenerate is $k-3$ variables)
$$x_2^2+(x_3-x_2)^2+\cdots+(x_{k-3}-x_{k-2})^2+x_{k-2}^2=0$$
via the transformation
$(x_1,x_2,\ldots,x_{k-2})\mapsto \left(x_1,x_2-2x_1,x_3-3x_1
\ldots,x_{k-2}-(k-2)x_1\right)$. Note that $a'$ is the number
of projective solutions of the projective quadric associated to (\ref{genreduced})
and $a''$ is its number of solutions at infinity (i.e. on the hyperplane
$x_0=0$).

From
\cite[Theorem 6.26 -- 6.27]{LN} we have that
$$
a'=\left\{\begin{array}{ll}
q^{k-2}+(q-1)q^{(k-3)/2}\eta((-1)^{(k-1)/2})            & \textrm{if $k$ is odd}\\
q^{k-2}            & \textrm{if $k$ is even}\\
\end{array}
\right.
$$
and$$
a''=\left\{\begin{array}{ll}
q^{k-3}+(q-1)q^{(k-3)/2}\eta((-1)^{(k-1)/2}))            & \textrm{if $k$ is odd}\\
q^{k-3}            & \textrm{if $k$ is even}\\
\end{array}
\right.
$$
Therefore we obtain that $a_k(q)=q^{k-3}$ when $(q,k-1)>1$.%as in (\ref{akappa})

 In all cases, if $q$ is odd, we have the upper bound
$$N_{[k]}(q)\leq \frac{q(q-1)}{k}a_k(q).$$

To compute $N_{[k]}(q)$, we have to subtract from $a_k(q)$ the number of points in
the $(k+1)(k-2)/2$
quadratic varieties obtained intersecting (\ref{genreduced}) with the
hyperplanes $x_i= 0$,
$x_i=1$, $x_i= x_j$, for $i,j=1,\ldots,k-2$, $i\neq j$ and so on.
In  each step we have to
compute the number of solutions of some quadric equations over $\Fq$.
However we are
not able to control how the discriminant of the quadrics behaves in the generic step.

In the case when $\Cl=[l_1\ \ldots\ l_s]$ is a general conjugation class
(after a transformation which reduces the number of variables to $c-s$, being
$c=l_1+\ldots+l_s$ the number of elements moved by any permutation in $\Cl$),
one will have to consider a quadric hypersurface whose matrix will be
\begin{equation}\label{finsper}\left(
\begin{array}{cccc}
 M_{l_1} &         &        &    0    \\
         & M_{l_2} &        &         \\
         &         & \ddots &         \\
    0    &         &        & M_{l_s} \\
\end{array}\right)
\end{equation}
with determinant $l_1\cdots l_s$. This can be used to deduce
an upper bound for $N_{\Cl}(q)$.

If we use the other notation introduced in the first section for a
conjugation class $\Cl=(m_1;m_2;\ldots;m_t)$, then we have that
$$N_{\Cl}(q)=\frac{q(q-1)}{m_2!2^{m_2}\cdots m_t!t^{m_t}}P_\Cl(q)
\ \ \ \ \textrm{and}\ \ \ \
P_\Cl(q)=a_0+a_1q+\cdots+a_{c-3}q^{c-3}$$
where, when $q$ is odd, each $a_i$ is an expression of the form
$$a_i=a_{i1}\eta(\alpha_{i1})+\cdots+a_{ij_i}\eta(\alpha_{ij_i})$$
with $a_{ij},\alpha_{ij}\in\mathbb Z$. Furthermore $a_{c-3}=1$ for $q$
large enough with respect to $c$.

Finally note that $\alpha_{ij}$ is, up to a sign, the discriminant
of a quadratic form which is the intersection of  (\ref{genreduced}) with
a number of hyperplanes among $x_i= 0,1$, $x_i= x_j$,
$i,j=1,\ldots,k-2$. This implies that there are finitely many
possibilities for $\alpha_{ij}$. Hence the expressions
$P_{\Cl}(q)$ can be calculated by computing $N_\Cl(q)$
for sufficiently many values of $q$ and by solving linear
equations. For example using Pari \cite{Pari} we calculated,
if $q=p^n$ and $p>3$:

\begin{small}
$$ \begin{array}{|rcl|}
\hline&&\\
P_{[6]}(q)&=&{q}^{3}-14\,{q}^{2}+[68-6\,\eta(5)-6\,
\eta(50)]q-[154+66\,\eta(-3)+93\,\eta(-1)+12\,
\eta(-2)+54\,\eta(-7)]\\
&& \\
P_{[4\ 2]}(q)& =&
{q}^{3}-[14-\eta(2)]{q}^{2}+[71+12\,\eta(-1)+\eta(-2)+4\,\eta(-3)-8\,\eta
(50)]q\\
& &-[148+100\,\eta(-1)+24\,\eta(-2)+44\,\eta(-3)+40\,\eta(-7)]\\
&& \\
P_{[3\ 3]}(q)& =& q^3-13\,{q}^{2}+[62+9\,\eta(-1)+4\,\eta(-3)]q
 -[150+99\,\eta(-1)+42\,\eta(-3)+72\,\eta(-7)]\\
&&\\
P_{[2\ 2\ 2]}(q)&=&q^3-[14+3\,\eta(-1)]{q}^{2}+[70+36\,\eta(-1)+6\,\eta(-2)]q
-[136+120\,\eta(-1)+48\,\eta(-2)+8\,\eta(-3)]\\
&&\\
\hline \end{array}
$$\end{small}

$$ \begin{array}{|rcl|}\hline&&\\
P_{[6]}(3^n)&=&3^{3n}-[14+2(-1)^n]3^{2n}+[71+39(-1)^n]3^n-[162+147(-1)^n]
\\
&&\\
P_{[4\ 2]}(3^n)&=& {3}^{3n}-[14+3\left (-1\right )^{n}]{3}^{2n}+[72+40\left (-1
\right )^{n}]{3}^{n}-[164+140\left (-1\right )^{n}]
\\
&& \\
P_{[3\ 3]}(3^n)&=&\left (1+\left (-1\right )^{n}\right ){3}^{3\,n}-[14+15\,\left (-1
\right )^{n}]{3}^{2\,n}+[71+81\,\left (-1\right )^{n}]{3}^{n}-
[150+171\,\left (-1\right )^{n}]
\\
&& \\
P_{[2\ 2\ 2]}(3^n)&=&3^{3n}-[14+3(-1)^n]{3}^{2n}+[76+36(-1)^n]3^n
-[168+120(-1)^n]\\
&&\\\hline \end{array}$$
and
$$ \begin{array}{|rcl|}
\hline &&\\
P_{[6]}(2^n)& =& (2^n-3-(-1)^n)(2^{2n}-(11-(-1)^n)2^n+(41+7(-1)^n)) \\
&&\\
P_{[4\ 2]}(2^n)& =&(2^n-3-(-1)^n)(2^{2n}-11\cdot2^n+37+(-1)^n)\\
&& \\
P_{[3\ 3]}(2^n)& =&(2^n-3-(-1)^n)(2^{2n}-(10-(-1)^n)2^n+45-3(-1)^n))\\
&&\\
P_{[2\ 2\ 2]}(2^n)&=& (2^n-2)(2^n-4)(2^n-8).\\
&&\\
\hline
 \end{array}
$$
\bigskip

\noindent As a last consequence of the above discussion we have that

\begin{Proposition} Suppose $\Cl$ is a fixed conjugation class
of permutations. Then
$$N_\Cl(q)=\frac{q^{c-1}}{c_2!2^{c_2}\cdots c_t!t^{c_t}}
\left(1+O\left(\frac{1}{q}\right)\right).$$
Therefore, since by (\ref{conjusize})
$$|\Cl|=\frac{q^{c}}{c_2!2^{c_2}\cdots
c_t!t^{c_t}}\left(1+O\left(\frac{1}{q}\right)\right),$$
the probability that an element of $\sigma\in\Cl$
is such that $\partial f_\sigma<q-2$ is
$$\frac{1}{q}+O\left(\frac{1}{q^2}\right).$$
\end{Proposition}

\section{Permutations of $\mathbb F_{2^n}$ that are product of $2$--cycles.}

A permutation has order $2$ if and only if its cycle decomposition
consists only of cycles
of length $2$. Let $\Tl_{r}=[2\ 2\ \ldots \ 2]$ be the
conjugation class of those permutations
of $\Fq$ which have a cycle decomposition consisting of
$r$ cycles of length $2$.

\begin{theorem} Let $q=2^n$.
Then $N_{[2]}(2^n)=0$ and the following recursive relation holds:
$$N_{\Tl_{r}}(q)=
\frac{q!}{r!2^r(q-2r+1)!}-\frac{(q-2(r-1))(2r-1)}{2r}N_{\Tl_{r-1}}(q).$$
Therefore (in accordance with the formulas already proven):
$$\begin{array}{rcl}
{N}_{[2\ 2]}(2^n)&=&\frac{1}{8}2^n(2^n-1)(2^n-2),\\
 {N}_{[2\ 2\ 2]}(2^n)&=&\frac{1}{48}2^n(2^n-1)(2^n-2)(2^n-4)(2^n-8),\\
{N}_{[2\ 2\ 2\ 2]}(2^n)&=&\frac{1}{384}{2}^{n}({2}^{n}-1)({2}^{n}-2)({2}^{n}-
4)({2}^{n}-6 )({2}^{2n}-15\,{2}^{n}+71).\end{array}$$
\end{theorem}

\noindent\textbf{Proof.} We have observed in the introduction that all transpositions
have permutation polynomial with degree exactly $q-2$ so
$$N_{[2]}(2^n)=0.$$
The polynomial $A_{\Cl}$ of (\ref{polinomiaccio}) is in this case:
$$
\begin{array}{rcl}
A_{T_r}({\underline x}) & = &
        \displaystyle\sum_{\sopra{i=1}{ i \textrm{ odd }}} ^{2r}
        x_i(x_i-x_{i+1})
        + \displaystyle\sum_{i=1}^r x_{2i}(x_{2i}-(x_{2i-1}))\\
 & = &\displaystyle\sum_{i=1}^{2r} x_i^2  + 2 \sum_{i=1}^r x_{2i}x_{2i-1} \\
 & = & \displaystyle\sum_{i=1}^{2r} x_i^2 \\
 & = & \left( \displaystyle\sum_{i=1}^{2r} x_i\right)^2
\end{array}
$$
From this, applying (\ref{gen2}), we have that
$$
  N_{\Tl_{r}}(q)  = \frac{1}{r!2^r}\left|
\left\{{\underline x} \in \Fq^{2r}: \quad
   {\underline x}\textrm{ has coordinates all distinct and }
\sum_{i=1}^{2r}x_{i}=0 \right\}\right|
$$
For simplicity, call  $\mathcal N_r$  the last set above. Let start
selecting arbitrarily $2r-1$ distinct values $(x_1,\ldots,x_{2r-1})$
for the first  $2r-1$ coordinates
of ${\underline x}\in \Fq^{2r}$: this can be done in
$q(q-1)\cdots(q-(2r-2))$ ways. For each such choice,
the value of the last coordinate is uniquely determined by $x_{2r}=\sum_{j< 2r}x_j$
if we want to have $\underline{x}\in\mathcal N_r$.
However a value for $x_{2r}$ is not admissible if it coincides with
one of the previous coordinates. There are $2r-1$ possible indices $j_0$
where this can happen and if  $\sum_{j< 2r}x_j=x_{2r}=x_{j_0}$, then
$$\sum_{\genfrac{}{}{0pt}{}{j=1}{j\neq j_0}}^{2r-1} x_j=2x_{j_0}+
\sum_{\genfrac{}{}{0pt}{}{j=1}{j\neq j_0}}^{2r-1} x_j=\sum_{j=1}^{2r} x_j=0,$$
that is $
%{\underline x'}=
(x_1,\ldots,x_{j_0-1},x_{j_0+1},\ldots,x_{2r-1})
\in \mathcal N_{r-1}$. So for each choice of index $j_0$,
the number of possible values for
$\{x_1,\cdots,x_{2r-1}\} \setminus\{x_{j_0}\}$
above is $\left|\mathcal N_{r-1}\right|$.
Taking into account that for
any choice of an element in $\mathcal N_{r-1}$
%$\{x_1,\cdots,x_{2r-1}\}\setminus \{x_{j_0}\}$,
there are $q-2r+2$
choices for $x_{j_0}$,
we deduce that
$$\left|\mathcal N_r\right|=\frac{q!}{(q-2r+1)!}-
(q-2(r-1))(2r-1)\left|\mathcal N_{r-1}\right|.$$
which, in view of the fact that
$$N_{\Tl_{r}}(q)=\frac{\left|\mathcal N_r\right|}{r!2^r},$$
is equivalent to  the statement.
This concludes the proof.\hfill$_\square$\bigskip


\section{Conclusion.}

In a forthcoming paper we will deal with the problem of counting
permutation polynomials with
minimal possible degree in a fixed conjugation class. Note also that S. Konyagin and the
second author have recently proved that if
$$\mathcal N=\left|\{\sigma\ \textrm{permutation of $\Fq$ such that }
\partial f_\sigma<q-2\}\right|,$$
then
$$\left|\mathcal N -(q-1)!\right|\leq \sqrt{\frac{2e}{\pi}}q^{q/2}$$
which confirms the common belief that almost all permutation
polynomials have degree $q-2$. A similar result has been
proven independently by Pinaki Das.\bigskip

\noindent\textbf{Acknowledgments.}
We would like to thank the referee for some useful comments and corrections.

\begin{thebibliography}{99}

\bibitem{LB} \textsc{Levine J. \& Brawley J. V.},
\textit{Some cryptographic applications of permutation polynomials}, Cryptologia
\textbf{1}, N. 1, (1977)  76--92.

\bibitem{LMu} \textsc{Lidl R. \& M\"uller W. B.}, \textit{A note on polynomials
and functions in cryptography}, Ars Comb. \textbf{17}-A (1984) 223--229.

\bibitem{LM1} \textsc{Lidl R. \& Mullen G. L.},
\textit{When does a polynomial over a finite field permute the elements of the field?},
Amer. Math. Mon. \textbf{95} (1988) 243--246.

\bibitem{LM2} \textsc{Lidl R. \& Mullen G. L.}, \textit{When does a polynomial
over a finite field permute the elements of the field? II}, Amer. Math. Mon.
\textbf{100} (1993) 71--74.

\bibitem{LN} \textsc{Lidl R. \& Niederreiter H.}, \textit{Finite fields}, Encyclopedia of Mathematics and its Applications 20, Addison--Wesley Publishing Company, Reading, MA 1983.

%\bibitem{CP} \textsc{Malvenuto C. \& Pappalardi F.}, 
%\textit{On the enumeration of Permutation
%Polynomials II: $k$-cycles with minimal degree}, preprint (2000).

\bibitem{M} \textsc{Mullen G. L.}, \textit{Permutation polynomials over finite fields},
 Finite fields, coding theory, and advances in communications and computing
(Las Vegas, NV, 1991), 131--151, Lecture Notes in Pure and Appl. Math., 141,
Dekker, New York, 1993.

\bibitem{W} \textsc{Wells C.}, \textit{The degrees of permutation polynomials
over finite
fields}, J. Combinatorial Theory \textbf{7} (1969) 49--55.

\bibitem{Pari} \textsc{Batut C.,  Belabas K., Bernardi D.,
Cohen H. \& Olivier M.}, GP/PARI Calculator Version 2.0.14
1989-1999.

%\bibitem{Maple} Maple V Release 5.1 (1999), Waterloo Maple Inc.
\end{thebibliography}


\begin{footnotesize}
\noindent Claudia Malvenuto\\
Dipartimento di Scienze dell'Informazione\\
Universit\`a degli studi ``La Sapienza''\\
Via Salaria, 113\\
I--00198, Roma -- ITALY.\\
\texttt{claudia@dsi.uniroma1.it}
\bigskip

\noindent Francesco Pappalardi\\
Dipartimento di Matematica\\
Universit\`a degli studi Roma Tre\\
Largo S. L. Murialdo, 1\\
I--00146, Roma -- ITALY.\\
\texttt{pappa@mat.uniroma3.it}\vfill
\end{footnotesize}



\end{document}
