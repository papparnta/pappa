\documentclass{amsart}
\hfuzz=7.65pt
\vfuzz=1.23pt
\newcommand{\F}{\mathbb F_q}
\newtheorem{prop}{Proposition}[section]

\title[Corrigendum]{Corrigendum to\\ ``Enumerating Permutation Polynomials I: Permutations with
Non-Maximal Degree''}
\author{Claudia Malvenuto and Francesco Pappalardi}
\address[Malvenuto]{Dipartimento di Informatica\\
Universit\`a degli studi ``La Sapienza''\\
Via Salaria, 113\\
I--00198, Roma -- ITALY.} \email{claudia@di.uniroma1.it}

\address[Pappalardi]
{Dipartimento di Matematica\\
Universit\`a degli studi Roma Tre\\
Largo S. L. Murialdo, 1\\
I--00146, Roma -- ITALY.}
\email{pappa@mat.uniroma3.it}

\date{\today}

\begin{document}
\begin{abstract}
This note corrects a statement that appeared in \textit{Finite
Fields Appl.} \textbf{8} (2002) no. 4, 531--547. It also provides a
self contained proof.
\end{abstract}
\maketitle \setcounter{section}{5}

Let $\F$ be a finite field with $q$ elements and suppose
$\mathcal{C}$ is a conjugation class of permutations of the elements
of $\F$. We denote by $\mathcal{C}=(c_1;c_2;\ldots;c_t)$ the
conjugation class of permutations that admit a cycle decomposition
with $c_i$ $i$--cycles ($i=1,\ldots,t$). Further we set
$c=2c_2+\cdots+tc_t=q-c_1$ to be the number of elements of $\F$
moved by any permutation in $\mathcal C$.

If $\sigma\in\mathcal C$ then the permutation polynomial associated
to $\sigma$ is defined as
$$f_\sigma(t)=\sum_{x\in\F}\sigma(x)\left(1-(t-x)^{q-1}\right).$$
Therefore for $q>3$ the function
$$N_\mathcal{C}(q)=\#\left\{\sigma\in\mathcal{C}\ |\
\sum_{x\in\F}x(\sigma(x)-x)=0\right\}$$ enumerates the permutations
in $\mathcal C$ whose permutation polynomials have degree strictly
less than $q-2$.

The reviews of \cite{CP} that appeared in Math Reviews (MR1933624
(2003j:11146)) and in Zentralblatt Math (Zbl 1029.11068) point out
that the case of transpositions provides a counterexample to the
statement of Proposition 5.1. The statement is also obviously wrong
in the cases of three--cycles and the case  of the conjugation class
of the product of two transpositions (see \cite[Theorems 1.1 and
3.1]{CP}). For brevity, we will indicate these three classes of
permutations respectively by $[2], [3]$ and $[2\ 2]$. Other
counterexamples to Proposition 5.1 occur if the characteristic of
$\F$ is ``small''. The correct statement should read:

\begin{prop} Suppose $\mathcal{C}=(c_1;c_2;\ldots;c_t)$
is a fixed conjugation class of permutations which does not coincide with
the classes $[2], [3]$ and $[2\ 2]$. Then,
if $\operatorname{char}(\F)>2$ and does not divide $2^{c_2}\cdots t^{c_t}$,
we have
$$N_\mathcal{C}(q)=\frac{q^{c-1}}{c_2!2^{c_2}\cdots c_t!t^{c_t}}
\left(1+O\left(\frac{1}{q}\right)\right).$$
Therefore, since
$|\mathcal{C}|=\frac{q^{c}}{c_2!2^{c_2}\cdots
c_t!t^{c_t}}\left(1+O\left(\frac{1}{q}\right)\right),$ as
$\operatorname{char}(\F)\rightarrow\infty$, the probability that an
element of $\sigma\in\mathcal{C}$ is such that the degree $\partial
f_\sigma<q-2$ is $\frac{1}{q}+O\left(\frac{1}{q^2}\right).$
\end{prop}

\begin{proof} We associate to $\mathcal C$ the quadratic form in $c$
variables
$$Q_\mathcal
C(\underline{X})=\sum_{i=2}^t\sum_{j=1}^{c_i}
\left(X_{ij1}(X_{ij1}-X_{ij2})+
 X_{ij2}(X_{ij2}-X_{ij3})+\cdots+X_{iji}(X_{iji}-X_{ij1})\right).
$$

Let $\hat{N}(Q_\mathcal C)$ be the number of solutions of
$Q_\mathcal C(\underline{X})=0$ over $\F$ with all distinct
coordinates. We have that
\begin{equation}\label{uno}
N_\mathcal{C}(q)=\frac{\hat{N}(Q_\mathcal C)}{c_2!2^{c_2}\cdots
c_t!t^{c_t}}\end{equation} since the denominator above counts the
number of distinct representations of any permutation in $\mathcal
C$ as product of disjoint cycles.

Next we consider the inequalities:
$$ N(Q_\mathcal C) -\sum_{I}N(Q^{I}_\mathcal C)\leq \hat{N}(Q_\mathcal C)\leq N(Q_\mathcal
C)$$ where the sum ranges over $2$--element sets of variables
$I=\{X_{i_1j_1k_1}, X_{i_2j_2k_2}\}$ of the quadratic form
$Q_\mathcal C$, where $Q^{I}_\mathcal C$ denotes the quadratic form
in $c-1$ variables obtained by $Q_\mathcal C$ substituting
$X_{i_1j_1k_1}=X_{i_2j_2k_2}$ and for any quadratic form $Q$, $N(Q)$
denotes the number of solutions over $\F$ of $Q=0$.

Note that if $\mathcal C\neq[2]$, then $Q^{I}_\mathcal C\neq0$
(indeed if $X_{rst}$ is a variable not in $I$, then the coefficient
of $X_{rst}^2$ in $Q^{I}$ is $1$). From this we deduce that
$N(Q^{I}_\mathcal C)\leq 2q^{c-2}$. Hence we obtain that
 \begin{equation}\label{due}
\hat{N}(Q_\mathcal C)=N(Q_\mathcal C)+O(q^{c-2}).
\end{equation}
Furthermore note that $Q_\mathcal C$ is equivalent to the form in
$\tilde{c}=c-(c_2+\cdots+c_t)$ variables $\tilde{Q}_\mathcal
C(\underline{Y})$
$$=\sum_{\substack{2< i\leq t\\ j\leq c_i}}
\left(Y_{ij1}(Y_{ij1}-Y_{ij2})+
\cdots+Y_{ij(i-2)}(Y_{ij(i-2)}-Y_{ij(i-1)})+Y_{ij(i-1)}^2\right)+\sum_{j\leq
c_2}Y_{2j1}^2,$$ where the equivalence is obtained with the linear
transformation
$$\begin{cases}X_{ijk}=Y_{ijk}+Y_{iji} & i=2,\cdots,t;\ j=1,\cdots,c_i;\ k=1,\ldots,i-1;\\
X_{iji}=Y_{iji} &i=2,\cdots,t;\ j=1,\cdots,c_i.\end{cases}$$ If $q$
is odd and $i>2$, the $(i-1)\times(i-1)$ matrix
$$\mathcal M_i=\left(\begin{array}{rrrrrr}
1& -\frac12 & 0 & 0& \cdots & 0\\
-\frac12& 1 & -\frac12 & 0 & \cdots & 0\\
0 & -\frac12 & 1 & -\frac12 &\cdots &\vdots\\
\vdots & \ddots& \ddots & \ddots& \ddots&\vdots\\
0 & 0 &\ddots & -\frac12 & 1 & -\frac12\\
0& 0& \cdots & 0 &-\frac12 & 1
\end{array}\right)$$
associated to the quadratic form $$Y_{ij1}(Y_{ij1}-Y_{ij2})+
Y_{ij2}(Y_{ij2}-Y_{ij3})+\cdots+Y_{ij(i-2)}(Y_{ij(i-2)}-Y_{ij(i-1)})+Y_{ij(i-1)}^2$$
has determinant equal to $i/2^{i-1}$.

To see this it is enough to notice that $\det \mathcal M_3=3/4$,
$\det\mathcal M_4=1/2$ and if, for $i>4$, we expand the determinant
with respect to the first column, we obtain the recursive formula:
$$\det\mathcal M_i=\det\mathcal M_{i-1}-\frac14\det\mathcal M_{i-2}$$
which allows us to deduce the claim by induction.

Hence $\tilde{Q}_\mathcal C$ has discriminant $\Delta_\mathcal
C=2^{c_2}\cdots t^{c_t}/2^{\tilde{c}}$. This implies that if the
characteristic of $\F$ is larger than $2$ and does not divide
$2^{c_2}\cdots t^{c_t}$, then $\tilde{Q}_\mathcal C$ is
non--singular.

Therefore for odd characteristics coprime to $\Delta_\mathcal C$, we can use the
formulas of \cite[Theorems 6.26 and 6.27]{LN} to enumerate
$N(Q_\mathcal C)=q^{c_2+\cdots+c_t}N(\tilde{Q}_\mathcal C)$,
obtaining:
$$N(\tilde{Q}_\mathcal C)=\begin{cases}q^{\tilde{c}-1}+
\eta((-1)^{\tilde{c}/2}\Delta_\mathcal C)
(q-1)q^{(\tilde{c}-2)/2}&\text{if $\tilde{c}$ is even;}\\
q^{\tilde{c}-1} & \text{if $\tilde{c}$ is odd,}\end{cases}$$ where
$\eta$ is the quadratic character of $\F^*$. Observe that the power
of $q$ in the first term above is larger than the one in the second
term except when $\tilde{c}=c_2+2c_3+\cdots+ (t-1)c_t=2$ which is
satisfied only in the cases: $c=3,c_3=1$ or $c=4,c_2=2$.

Hence, if $\mathcal C\not\in\{[2],[3],[2\ 2]\}$ and
$\operatorname{char}(\F)$ is even and does not divide
$\Delta_\mathcal C$, then $N(Q_\mathcal C)=q^{c-1}+O(q^{c-2})$.
Substituting this in (\ref{due}) and then in (\ref{uno}), we
conclude the proof.\end{proof}

\begin{thebibliography}{99}
\bibitem{LN} \textsc{Lidl R. \& Niederreiter, H.}, \textit{Finite Fields},
Encyclo. Math. and Appls. V. 20, Addison--Wesley, Reading, MA 1983.
\bibitem{CP} \textsc{Malvenuto C. \& Pappalardi F.},
\textit{Enumerating Permutation Polynomials I: Permutations with
Non-Maximal Degree}. Finite Fields Appl. \textbf{8} (2002) no. 4,
531--547.
\end{thebibliography}
\end{document}
