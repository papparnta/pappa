%From pappa@mat.uniroma3.it Sun Nov 24 13:20:16 2002
\documentclass[a4paper,twoside]{article}
\hfuzz=13pt
\usepackage{latexsym,amsmath,amssymb,amsbsy,amsfonts,amsthm}
\newtheorem{Proposition}{Proposition}[section]
\newtheorem{Theorem}{Theorem}[section]
\newtheorem{Lemma}{Lemma}[section]
\newtheorem{Corollary}{Corollary}[section]

\title{Enumerating Permutation Polynomials II\\
$k$--cycles with minimal degree}
\author{Claudia Malvenuto and Francesco Pappalardi}

\date{\today}


\begin{document}

\maketitle

\begin{abstract}
We consider the function $m_{[k]}(q)$ that counts the number of
cycle permutations of a finite field $\mathbb F_q$ of fixed length
$k$ such that their permutation polynomial has the smallest
possible degree. We prove the upper--bound $m_{[k]}(q)\leq
(k-1)!(q(q-1))/k$  for $\operatorname{char}(\mathbb F_q)>
e^{(k-3)/e}$ and the lower--bound $m_{[k]}(q)\geq
\varphi(k)(q(q-1))/k$ for $q\equiv1\pmod k$. This is done by
establishing a connection with the $\mathbb F_q$--solutions of a
system of equations $\mathcal A_k$ defined over $\mathbb Z$. As
example, we give complete formulas for $m_{[k]}(q)$ when $k=4,5$
and partial formulas for $k=6$. Finally we analyze the Galois
structure of the algebraic set $\mathcal A_k$
\end{abstract}

\tableofcontents\vfill\pagebreak

\section{Introduction}

Let $q$ be a power of a prime and denote with $\mathbb F_q$ the finite
field with $q$ elements. If $\sigma$ is a permutation of the elements of $\mathbb F_q$,
then one can associate to $\sigma$ the polynomial in $\mathbb F_q[x]$

\begin{equation}\label{primadef}
f_\sigma(x)=\sum_{c\in\mathbb F_q}\sigma(c)\left(1-\left(x-c\right)^{q-1}\right).
\end{equation}

Such a polynomial has the property that

\begin{enumerate}
\item $f_\sigma(b)=\sigma(b)$ for all $b\in\mathbb F_q$;
\item The degree $\partial(f_\sigma)\leq q-2$ (since the sum of all the elements of $\mathbb F_q$ is zero).
\end{enumerate}

$f_\sigma$ is the unique polynomial in $\mathbb F_q$ with these
two properties and it is called the \emph{permutation polynomial of $\sigma$}.

Permutation polynomials have increasingly attracted the attentions of various researchers
in the past couple of decades. We suggest the inspiring survey papers by Rudolf Lidl and
Gary Mullen \cite{M}, \cite{LM1} and \cite{LM2}
for an introduction to the subject.

Various cryptographic applications, including a key
exchange protocol for public key cryptography based on permutation polynomials have been
proposed.
(see J. Levine and Joel V. Brawley \cite{LB} and R. Lidl and W. M\"uller \cite{LMu}).

In their paper of 1988, Rudolf Lidl and Gary Mullen \cite{LM1}
discuss a number of open problems regarding permutation
polynomials. Among these, problem P6 asks to determine the number
$N_d(q)$ of permutation polynomials of degree $d$ where $1\leq
d\leq q-2$ and $d\nmid q-1$. This seems to be a difficult problem
at the moment. Some partial results were given by Wells in
\cite{W}. We will state his results later. See also the paper of
Sergey Konyagin and the second author \cite{KP} and the results by
Pinaki Das \cite{Das}.

For a given permutation $\sigma$ of $\mathbb F_q$, let us denote by
$S_\sigma$ the set of elements of $\mathbb F_q$ that are moved by $\sigma$. Note that
if $\sigma$ and $\sigma'$ are conjugated, then $\#S_\sigma=\# S_{\sigma'}$.

If $\sigma$ is not the identity we have that $\partial(f_\sigma)\geq q-\#S_\sigma$.
To see this it is enough to note that the polynomial $f_\sigma(x)-x$ has as roots all the elements
of $\mathbb F_q$ which are not in $S_\sigma$. Therefore, if not identically zero, $f_\sigma$ has to have degree at
least $q-\#S_\sigma$.

Let $\mathcal C$ be a conjugation class of permutations of a finite field $\mathbb F_q$ and
$c(\mathcal C)$ the number of elements of $\mathbb F_q$ moved by any permutation in $\mathcal C$
(that is: $c(\mathcal C)=\#S_\sigma$ for any $\sigma\in \mathcal C$).
As we just noticed, for any $\sigma\in\mathcal C$,
\begin{equation}\label{degfund}
q-2\geq \partial(f_\sigma)\geq q-c(\mathcal C).
\end{equation}

An immediate consequence is that all transpositions have polynomials
with degree $q-2$ while the degree of a $3$--cycle can be $q-2$ or $q-3$.

In the first paper of this series \cite{CP} we dealt
with the problem of determining $l_{\mathcal C}(\mathbb F_q)$,
defined as the number of permutation polynomials associated to permutations
in the class $\mathcal C$ whose degree is strictly less then $q-2$.
There we obtained a number of formulas and estimates. For classes of permutations
that move up to 6 elements we have computed closed formulas for $l_{\mathcal C}(\mathbb F_q)$.
These results extend those of Wells.

In this paper  we consider
$$M_{\mathcal C}(\mathbb F_q)=\left\{\sigma\in\mathcal C\ |\ \partial(f_\sigma)=q-\#S_\sigma\right\}$$
(i.e. the permutations in
$\mathcal C$ for which the permutation polynomial has the minimum possible degree  $q-c(\mathcal C)$) and set $m_{\mathcal C}(q)=\# M_{\mathcal C}(\mathbb F_q)$.

Let us also denote by $[k]$ the class consisting of all the
$k$--cycle permutations of $\mathbb F_q$.

\begin{Theorem} \label{ciclotomico} Let $\varphi$ be the Euler totient function.
If  $q\equiv1(\bmod k)$ then
$$m_{[k]}(q)\geq \frac{\varphi(k)}{k} q(q-1).$$
\end{Theorem}

Next we will show the upper bound

\begin{Theorem}\label{geometrico} Suppose
$\operatorname{char}(\mathbb F_q)>e^{(k-3)/e}$. Then
$$m_{[k]}(q)\leq \frac{(k-1)!}{k}q(q-1).$$
\end{Theorem}

The hypothesis $\operatorname{char}(\mathbb F_q)>e^{(k-3)/e}$ in Theorem~\ref{geometrico} rules
out the interesting case when $k$ has approximately the same size
as $q$. Our proof breaks down for these values of $k$. However we are convinced
that the upper bound for $m_{[k]}(q)$ holds for any value of $k<q$.

In general, if $\mathcal C$ is any
conjugation class of permutations then an analogous
upper bound as the one in Theorem~\ref{geometrico} can be proved
for $m_{\mathcal C}(q)$.
In some cases the bounds are stronger.
We have decided to restrict ourselves to the case of cycle permutations.

We will prove Theorem~\ref{geometrico} in Section 3, Theorem~\ref{ciclotomico} in section 4.
Section 5 is dedicated to examples. We will consider
$k$--cycles ($k=3,4,5,6$) and give detailed description of $m_{[k]}(q)$
in these special cases. In the last
section we investigate Galois properties of the algebraic sets $\mathcal A_k$ (see (\ref{algsets})).

\section{Reduction to normalized permutations}

A permutation $\sigma$ of $\mathbb F_q$ is said to be
\emph{normalized}\footnote{Note that this definition of
\emph{normalized permutation} is different from the usual one
where a permutation polynomial $f(x)\in\mathbb F_q[x]$ is said
normalized if it is monic, such that $f(0) = 0$ and when the
degree $n$ of $f$ is not divisible by the characteristic $p$ the
coefficient of $x^{n-1}$ is $0$.} if $\sigma(0)=1$. We denote by
$N_{\mathcal C}(\mathbb F_q)$ the set of normalized permutations
of $\mathcal C$ that have (minimal) degree $q-c(\mathcal C)$ and
we set $n_{\mathcal C}(q)=\#N_{\mathcal C}(\mathbb F_q)$.

\begin{Proposition}\label{normal} With the above notations
we have
$$m_{\mathcal C}(q)=\frac{1}{c(\mathcal C)}q(q-1)n_{\mathcal C}(q).$$
Hence if $m_{\mathcal C}(q)\neq0$, then
$$m_{\mathcal C}(q)\geq \frac{1}{c(\mathcal C)}q(q-1).$$
\end{Proposition}

\noindent\textbf{Proof.} Let $\mathbb A^1(\mathbb F_q)$ be the
group of affine transformations of $\mathbb F_q$, that is the
group of applications
$$L_{a,b}:\mathbb F_q \rightarrow \mathbb F_q, x\mapsto ax+b.$$
It is clear that $\#\mathbb A^1(\mathbb F_q)=q(q-1)$.

Consider the map
$$\begin{array}{rcl}
\Pi: \mathbb A^1(\mathbb F_q)\times N_{\mathcal C}(\mathbb F_q) & \rightarrow &  M_{\mathcal C}(\mathbb F_q)\\
\left(L_{a,b},\sigma\right) & \mapsto & (L_{a,b}^{-1}\sigma L_{a,b}).\end{array}$$
Clearly $\Pi$ is well defined since
$$\partial((L_{a,b}^{-1}f_\sigma L_{a,b}(x))=\partial( a^{-1}(f(ax+b)-b))=\partial(f_\sigma).$$
Furthermore $\Pi$ is surjective. This follows from the fact that,
given $\tau\in M_{\mathcal C}(\mathbb F_q)$, chosen $b\in S_\tau$
and set $a=\tau(b)-b$, we have that $L_{a,b}^{-1}\tau L_{a,b}$ is
normalized and therefore
$$\tau=\Pi(L_{a,b}^{-1},L_{a,b}^{-1}\tau L_{a,b}).$$
To complete the proof we need to show that for every $\tau\in M_{\mathcal C}(\mathbb F_q)$,
the fibre $\Pi^{-1}(\tau)$ has exactly $c(\mathcal C)$ elements. Indeed consider the map
$$\begin{array}{rcl}
\Sigma: S_\tau & \rightarrow & \Pi^{-1}(\tau)\\
      b & \mapsto & (L_{a,b}^{-1},L_{a,b}^{-1}\tau L_{a,b})\end{array}$$
where $a=\tau(b)-b$. It is clear that $\Sigma$ is well defined and injective.
Furthermore $\Sigma$ is also surjective since if $(L_{c,d},\sigma)\in \Pi^{-1}(\tau)$,
then $$\tau(-d/c)=L_{c,d}^{-1}\sigma L_{c,d}(-d/c)=L_{c,d}^{-1}\sigma(0)=(1-d)/c.$$
Therefore $-d/c\in S_\tau$, $1/c=\tau(-d/c)-(-d/c)$ and
$$\Sigma(-d/c)=(L_{c,d},\sigma).$$
Finally $\#\Pi^{-1}(\tau)=\# S_\tau=c(\mathcal C)$
and this concludes the proof.\hfill$_\Box$\bigskip

\noindent\textbf{Remark.} The previous proposition allows us to reduce the problem of computing
$m_{\mathcal C}(\mathbb F_q)$ to the easier one of computing $n_{\mathcal C}(\mathbb F_q)$.
Indeed since $c([k])=k$, Theorem~\ref{geometrico} is equivalent to $n_{[k]}(\mathbb F_q)\leq (k-1)!$ and Theorem~\ref{ciclotomico} is equivalent to $n_{[k]}(\mathbb F_q)\geq\varphi(k)$
for $q\equiv 1\pmod k$.

\section{From normalized permutation polynomials to affine algebraic sets. Proof of Theorem~\ref{geometrico}.}

Let us write
$$f_\sigma(x)=A_{q-1}+A_{q-2}x+\cdots+A_{1}x^{q-2}.$$
{From} definition (\ref{primadef}) it follows that for every $i=1,\ldots,q-2$,

$$A_i=A_i(\sigma)=(-1)^{i+1} \binom{q-1}{i}
\sum_{c\in\mathbb F_q}\sigma(c)c^i.$$

It is well known (see for example \cite[Exercise 7.1]{LN}) that
for $0\leq i\leq q-1$, $(-1)^{i+1} \binom{q-1}{i} =-1$ in $\mathbb
F_q$. Furthermore using the identity (see for example \cite[Lemma
6.3]{LN}),
$$\sum_{c\in\mathbb F_q}c^{i+1}=0$$
for $i<q-2$, we deduce that
$$A_i(\sigma)=\sum_{c\in\mathbb F_q}c^i(c-\sigma(c))=\sum_{c\in S_\sigma}c^i(c-\sigma(c)).$$
{From} these observations it follows that
$$m_{\mathcal C}(q)=\#\left\{\sigma\in\mathcal C\ \text{such that} \sum_{c\in S_\sigma}c^i(c-\sigma(c))=0\
\text{ for }\ i=1,\ldots,c(\mathcal C)-2\right\}.$$

Let us now specialize to the case when $\sigma$ is a normalized
$k$--cycle:
$$\sigma=(0,\ 1,\ a_{1},\ a_{2},\ \cdots,\ a_{k-2}).$$
In this case
$$A_i(\sigma)=(1-a_1)+a_1^i(a_1-a_2)+\cdots+a_{k-3}^i(a_{k-3}-a_{k-2})+a_{k-2}^{i+1}.$$

For $i=1,\ldots,k-2$, define the polynomial with integer coefficients:
\begin{equation}\label{prime}
G_{i}(x_1,\ldots,x_{k-2})=1-x_1+\sum_{j=1}^{k-3}x_{j}^i(x_{j}-x_{j+1})+x_{k-2}^{i+1}
\in\mathbb Z[x_1,\ldots,x_{k-2}]\end{equation}

The degree $\partial(f_\sigma)=q-\#S_\sigma$ if and only if
$$A_1(\sigma)=\cdots=A_{k-2}(\sigma)=0$$
Therefore
\begin{equation}\nonumber
n_{[k]}(q)=\#
\left\{\underline{x}\in(\mathbb F_q\setminus\{0,1\})^{k-2}\  \genfrac{}{}{0pt}{}{\text{such that } G_{1}(\underline{x})=\cdots=G_{k-2}(\underline{x})=0}
{\text{and all the components of }\ \underline{x}\ \text{ are distinct}}
\right\}.\end{equation}

We are naturally lead to consider the affine algebraic set $\mathcal A_k$ in
$\mathbb A^{k-2}$ defined by the equations:
\begin{equation}\mathcal A_k:\label{algsets}
\left\{\begin{array}{rcl}
(1-x_1)+x_1(x_1-x_2)+\cdots +x_{k-3}(x_{k-3}-x_{k-2})+x_{k-2}^{2}& = &0\\
(1-x_1)+x_1^2(x_1-x_2)+\cdots +x_{k-3}^2(x_{k-3}-x_{k-2})+x_{k-2}^{3}& = &0\\
& \vdots & \\
(1-x_1)+x_1^{k-2}(x_1-x_2)+\cdots+ x_{k-3}^{k-2}(x_{k-3}-x_{k-2})+x_{k-2}^{k-1}& = &0
\end{array}\right.\end{equation}
Clearly $\mathcal A_k$ is defined over $\mathbb Z$ and therefore over any
field.

We can also write that
\begin{equation}\label{citato}
n_{[k]}(q)=\# \left\{\underline{x}\in\mathcal A_{k}(\mathbb F_q)
\text{ with components not in } \{0,1\}, \text{ and all
distinct}\right\}.
\end{equation}

\begin{Theorem}\label{Lucia} Let $\mathbf K$ be any algebraic closed fields
and $k\in \mathbb N$ be an integer such that either char$(\mathbf K)=0$
or char$(\mathbf K)> e^{(k-3)/e}$.
Then we have that the algebraic variety dimension
$$\dim_{\mathbf K}(\mathcal A_k)=0.$$
\end{Theorem}

\noindent\textbf{Remark.} Note that the hypothesis char$(\mathbf K)> e^{(k-3)/e}$ is not redundant.
In fact it can be seen that, if $p=\operatorname{char}(\mathbf K)$
is fixed, then
$$\lim_{k\rightarrow\infty}\dim_{\mathbf K}(\mathcal A_k)=+\infty.$$
\medskip

\begin{Corollary} Let $\mathbf K$ be any algebraic closed fields
and $k\in \mathbb N$ be an integer such that either char$(\mathbf
K)=0$ or char$(\mathbf K)> e^{(k-3)/e}$. Then
$$\#\mathcal A_k({\mathbf K})= (k-1)!$$
where the points are counted with multiplicity.
\end{Corollary}

\noindent\textbf{Proof of the Corollary} We apply the Theorem of
Bezout (see for example the book of J. Harris \cite{Har}) which
states that if $k-2$ hyper-surfaces in $\mathbb
P^{k-2}(\overline{\mathbf K})$ do intersect in a zero dimensional
sub-variety of  $\mathbb P^{k-2}(\overline{\mathbf K})$, then the
number of points that they have in common is given by the product
of the degrees of the equations. In our case the product of the
degrees is $2\cdot3\cdots (k-1)$ and since none of the points is
``at infinity'' we have the claim.\hfill$_\Box$\medskip

In order to prove Theorem~\ref{Lucia}, we will need the following
three auxiliary lemmas:

\begin{Lemma} \label{luciachi} Let $\mathbf K$ be any field and let
$X_1,\ldots,X_n\in\mathbf K^*$.
The linear system
$$
\left\{\begin{array}{rcl}
X_1U_1+\cdots+X_nU_n & = & 0\\
X_1^2U_1+\cdots+X_n^2U_n & = & 0\\
 &\vdots & \\
X_1^nU_1+\cdots+X_n^nU_n & = & 0\\
U_1+\cdots+U_n &=&X_1
\end{array}\right.$$
has no solutions $(U_1,\ldots,U_n)$ in $\mathbf K^n$.
\end{Lemma}

\noindent\textbf{Proof of Lemma~\ref{luciachi}.}
The proof is done by induction on $n$. If $n=1$, then
the conditions $X_1U_1=0$ and $U_1=X_1$ imply
that $X_1=0$. Therefore no solution exists.
Assume $n\geq2$.

Let $A$ be the matrix of the coefficients of the first $n$
equations. Expanding the Vandermonde determinant we obtain:
$$\det(A)=X_1\cdots X_n\prod_{i>j}(X_i-X_j).$$
If the system of equations admits a solution $(u_1,\ldots,u_n)$,
then not all the $u_i's$ can be equal to $0$ otherwise the last
equation cannot be satisfied. Therefore the homogeneous system
given by the first $n$ equations has to have a non trivial
solution. This implies that $\det(A)=0$ and therefore $X_i= X_j$
for some $i\neq j$. Let us assume without loss of generality that
$X_n=X_{n-1}$. Now $(u_1,\ldots,(u_{n-1}+u_n))$ is a solution of
the system
$$
\left\{\begin{array}{rcl}
X_1U_1+\cdots+X_{n-1}U_{n-1} & = & 0\\
X_1^2U_1+\cdots+X_{n-1}^2U_{n-1} & = & 0\\
 &\vdots & \\
X_1^{n-1}U_1+\cdots+X_{n-1}^{n-1}U_{n-1} & = & 0\\
U_1+\cdots+U_{n-1} &=&X_1
\end{array}\right.$$
which is a contradiction to the inductive hypothesis.\hfill$_\Box$\medskip


\begin{Lemma} \label{topo} Let $A=(a_{ij})$ be a $t\times t$ matrix with integer
entries such that the following properties hold:
\begin{enumerate}
\item For all $i,j=1,\ldots,t$, $i\neq j$, $a_{ii}>0$, $a_{ij}\leq
0$ (i.e. the elements in the diagonal of $A$ are strictly positive
and those outside are negative); \item For every $i=1,\ldots,t$
there exists $j\neq i$ such that $a_{ij}\neq0$ (i.e. every row has
a a least a non zero entry outside the diagonal); \item For every
$j=1,\ldots,t$, $\displaystyle{\sum_{i=1}^t a_{ij}}\geq0$ and
there exists $j$ with $\displaystyle{\sum_{i=1}^t a_{ij}}>0$ (i.e.
the sum of the elements in every column is positive and for at
least one column is strictly positive).
\end{enumerate}
Then
$$0<\det(A)\leq a_{11}\cdots a_{tt}.$$
\end{Lemma}

\noindent\textbf{Proof of Lemma~\ref{topo}.} We proceed by induction on $t$.

If $t=2$, then $A=\left(\begin{array}{cc}
a_{11}&a_{12}\\ a_{21}&a_{22}
\end{array}\right)$ and
$\det(A)=a_{11}a_{22}-a_{21}a_{12}.$ By the third hypothesis we
have that $a_{11}\geq -a_{21}$, $a_{22}\geq -a_{12}$ and one of
the two inequalities is a strict one. Therefore, since by property
\textit{1}, $-a_{21}\geq0$ and $-a_{12}\geq0$, we have
$$a_{11}a_{22}>a_{21}a_{12}.$$
Finally $\det(A)>0$. The inequality $\det(A)\leq a_{11}a_{22}$ follows
from the first hypothesis.

Assume now that $t\geq3$ and also assume without loss of
generality that $\displaystyle{\sum_{i=1}^ta_{i1}}\geq1$. If
$A_1,A_2,\ldots A_t$ are the rows of $A$, then consider that
matrix:
$$\left(\begin{array}{c}
A_1\\ a_{11}A_2- a_{21}A_1\\
\vdots \\ a_{11}A_t-a_{t1}A_1\end{array}\right)=
\left(\begin{array}{cccc}
a_{11} & a_{12} & \cdots & a_{1t}\\
0 & & &\\
\vdots && B & \\
0 & & &
\end{array}\right)
$$
where
$B=(b_{ij})$, $i,j=2,\ldots,t$ and
$$b_{ij}=a_{11}a_{ij}-a_{i1}a_{1j}.$$
It is clear that $a_{11}^{t-1}\det(A)=a_{11}\det(B)$.
We claim that $B$ verifies the hypothesis of the Lemma and
therefore, as $a_{j1}a_{1j}\ge0$ for $j=2\ldots t$, by induction
$$0<\det(B)\leq (a_{11}a_{22}-a_{21}a_{12})\cdots
(a_{11}a_{tt}-a_{t1}a_{1t})\leq a_{11}^{t-1}a_{22}\cdots a_{tt}$$
and this implies the claim.

Let us check that $B$ verifies the hypothesis of the Lemma:
\begin{enumerate}
\item Since
$\displaystyle{\sum_{i=1}^ta_{i1}}\geq1$, for every $i=2,\ldots t$,
$a_{11}>-a_{i1}$. Furthermore $a_{ii}\geq -a_{1i}$, therefore
$$b_{ii}=a_{11}a_{ii}-a_{i1}a_{1i}>0$$
Also $b_{ij}=a_{11}a_{ij}-a_{i1}a_{1j}\leq0$ (if $i\neq j$) since
it is the sum of two negative numbers. \item For every
$i=2,\ldots,t$, let $j\neq i$ be such that $a_{ij}\neq0$. Then
$b_{ij}\leq a_{11}a_{ij}<0$ is also non--zero. \item Consider
$$\sum_{i=2}^t b_{ij}=a_{11}\sum_{i=2}^t a_{ij}-a_{1j}
\sum_{i=2}^t a_{i1}\geq -a_{11}a_{1j}-a_{1j}(1-a_{11})=
-a_{1j}.$$
Therefore $\sum_{i=2}^t b_{ij}\geq0$ for all $j=2,\ldots,t$ and if
$j$ is such that $a_{1j}\neq0$, then $\sum_{i=2}^t b_{ij}>0$.
\end{enumerate}
This concludes the proof.\hfill$_\Box$\medskip

The following lemma is a standard application of Calculus

\begin{Lemma}\label{max} If $T\in\mathbb N$ is given, then
$$\max\left\{x_1\cdots x_s\ |\ x_1,\ldots,x_s\in\mathbb N_{\geq2},\ x_1+\cdots+x_s\leq T\right\}\leq e^{T/e}$$
where $e$ is the Napier constant.
\end{Lemma}

\noindent{\textbf{Proof of Lemma~\ref{max}}} Since the arithmetic mean always
bounds the geometric mean, we have
$$x_1\cdots x_s\leq \left(\frac{x_1+\cdots+x_s}{s}\right)^s\leq
\left(\frac{T}{s}\right)^s.$$
The real variable function on the right hand side above has a maximum for
$s=T/e$. The result follows from the fact that for $T\geq3$,
$$\max\{\left(\frac{T}{[T/e]}\right)^{[T/e]},\left(\frac{T}{[T/e]+1}\right)^{[T/e]+1}\}\leq e^{T/e}.\hfill_\Box$$\medskip

\noindent\textbf{Proof of Theorem~\ref{Lucia}.} The proof will proceed as follows:
We denote by $\mathcal V_k$ the projective variety in $\mathbb P^{k-2}$
corresponding to $\mathcal A_k$.
\begin{scriptsize}
\begin{equation}\mathcal V_k:\label{secondeequazioni}
\!\left\{\!\!\begin{array}{rl}
X_0(X_0-X_1)+X_1(X_1-X_2)+\cdots +X_{k-3}(X_{k-3}-X_{k-2})+X_{k-2}^{2}&=0\\
X_0^2(X_0-X_1)+X_1^2(X_1-X_2)+\cdots +X_{k-3}^2(X_{k-3}-X_{k-2})+X_{k-2}^{3}&=0\\
& \vdots \\
X_0^{k-2}(X_0-X_1)+X_1^{k-2}(X_1-X_2)+\cdots+ X_{k-3}^{k-2}(X_{k-3}-X_{k-2})+X_{k-2}^{k-1}&=0
\end{array}\right.\end{equation}
\end{scriptsize}

To prove that $\mathcal V_k(\overline{\mathbf K})$ is
$0$-dimensional. we will show that it has no points of
intersection with the projective hyper-plane ``at infinity''
$X_0=0$. Indeed note that if $\mathcal V_k(\overline{\mathbf K})$
contains a positive dimensional sub-variety, then it has to have
non empty intersection with any plane. In particular, if we
substitute $X_0=0$ in (\ref{secondeequazioni}) we should obtain
some nontrivial solutions. We will see that this is impossible and
that the only solution is $(X_1,\ldots,X_{k-2})=(0,\ldots,0)$.

Assume that $k>3$, otherwise the statement can be verified directly and also
follows from the work of Wells \cite{W} (see (\ref{wello}) below) and let $n=k-2$.
If $n=2$, then we have the equation
$$\left\{\begin{array}{l}
X_1^2-X_1X_2+X_2^2=0\\
X_1^3-X_1^2X_2+X_2^3=0
\end{array}\right.$$
which is quickly seen to have as solutions only $(X_1,X_2)=(0,0)$ over
any field.

Assume $n\geq3$ and let $(X_1,\ldots,X_{n})\neq(0,\ldots,0)$ be
a non trivial solution. We can assume that $X_1\neq0$ otherwise
we would have a non trivial solution $(X_2,\ldots,X_{n})$
that we rule out by induction. For the same reason we can assume that
$X_n\neq0$ and that $X_i\neq X_{i+1}$ for $i=1,\ldots,n-1$.

Let us rewrite the equations in the following way:

$$\left(\begin{array}{cccc}
X_1 & X_2 & \cdots & X_n\\
X_1^2& X_2^2&\cdots & X_n^2\\
\vdots & &\cdots & \vdots\\
X_1^n & X_2^n & \cdots & X_n^n
\end{array}\right)\cdot\left(
\begin{array}{ccccc}1& -1 & 0 & \cdots & 0\\
                   0& 1 & -1 & \cdots & 0\\
                   \vdots & & \ddots  & \vdots & \vdots\\
                   0& 0 &\cdots & 0 & 1\end{array}
\right)\cdot\left(
\begin{array}{c}X_1\\X_2\\\vdots\\X_n\end{array}
\right)=
\left(\begin{array}{c}0\\0\\\vdots\\0\end{array}\right).$$

Note that the first matrix has determinant
$$X_1\cdots X_n\cdot\prod_{i>j}(X_i-X_j)$$
while the second has determinant $1$.

This immediately implies that the first matrix has to have
determinant equal to $0$ otherwise we would obtain the
contradiction  that $(X_1,\ldots,X_{n})=(0,\ldots,0)$

By setting $U_i=X_i-X_{i+1}$, if $i<n$ and $U_n=X_n$, and applying
Lemma~\ref{luciachi}, we obtain that at least one
of the $X_i=0$.

Now let us relabel the set $\{X_1,X_2,\ldots,X_n\}\subseteq\mathbf K$
as $\{y_1,y_2,\ldots,y_t,0\}$ in such a way that
\begin{enumerate}
\item $y_1,\ldots,y_t$ are all distinct;
\item $y_1,\ldots,y_t$ are all not zero;
\item For every $s\in\{1,\ldots,n\}$ there exists $i\in\{1,\ldots,t\}$ such
that $X_s=y_i$.
\end{enumerate}

Let us also assume that $y_1=X_1$ and note that $t\leq n-1$.
Now consider the first $t$ equations of (\ref{secondeequazioni})
and replace $(X_1,\ldots,X_n)$ with
 $(y_1,\ldots,y_t)$, so that

\begin{equation}\label{newrels}\left\{\begin{array}{lcr}
y_1L_1(y_1,\ldots,y_t)+\cdots+y_tL_t(y_1,\ldots,y_t) &=&0\\
&\vdots&\\
y_1^tL_1(y_1,\ldots,y_t)+\cdots+y_t^tL_t(y_1,\ldots,y_t) &=&0\\
\end{array}
\right.\end{equation}
where for $i=1,\ldots,t$,
$$L_i(y_1,\ldots,y_t)=\sum_{j=1}^t a_{ij}y_j$$
and
\begin{equation}\label{idea}
a_{ij}=\left\{
\begin{array}{lr}
 \#\{s\in\{1,\ldots,n\}\ |\ X_s=y_i\} & \text{if } i=j;\\
\\
-  \#\{s\in\{1,\ldots,n-1\}\ |\ X_s=y_i, X_{s+1}=y_j\} & \text{if } i\neq j.
\end{array}
\right.
\end{equation}
Let $A=(a_{ij})$ be the $t\times t$ matrix with integer entries defined by
(\ref{idea}) and $\hat{A}$ be the matrix obtained by $A$ and reducing
the entries in $\mathbf K$ where we assume that either $\operatorname{char}(\mathbf K)=0$ or $\operatorname{char}(\mathbf K)>e^{(k-3)/e}$.

Note that $a_{ii}\geq2$ otherwise one
row of $A$ would contain only one $1$ and possibly one $-1$ and this
would imply the contradiction that either two $y_i's$ are equal
or one $y_i$ is zero.

The relations (\ref{newrels}) can be written
as
$$\left(\begin{array}{ccc}
y_1 &  \cdots & y_t\\
y_1^2& \cdots & y_t^2\\
\vdots  &\cdots & \vdots\\
y_1^t & \cdots & y_t^t
\end{array}\right)\cdot
\hat{A}\cdot
\left(
\begin{array}{c}y_1\\y_2\\\vdots\\y_t\end{array}
\right)=
\left(\begin{array}{c}0\\0\\\vdots\\0\end{array}\right).$$
Since the first matrix has determinant
$$y_1\cdots y_t\prod_{i>j}(y_i-y_j)\neq0,$$
we deduce that
$$\hat{A}\cdot
\left(
\begin{array}{c}y_1\\y_2\\\vdots\\y_t\end{array}
\right)=
\left(\begin{array}{c}0\\0\\\vdots\\0\end{array}\right).$$
we want to obtain a contradiction by showing that $\det(\hat{A})\neq 0$.
We will do this by applying Lemma~\ref{topo} to $A$ which will give
$$0<\det A\leq a_{11}\cdots a_{tt}$$
and since
$$\sum_{i=1}^ta_{ii}=\#\{s\in\{1,\ldots,n\}\ |\ X_s\neq0\}\leq n-1=k-3,$$
we have by Lemma~\ref{max} that
$$0<\det A \leq e^{(k-3)/e}< \operatorname{char}(\mathbf K).$$
Therefore $\det(\hat{A})\neq0$ which implies the claim.

The only thing left to show is that $A$ satisfies the hypothesis
of Lemma~\ref{topo}: the first hypothesis is immediately verified
by the definition of the matrix $A$ in (\ref{idea}). Similarly the
second hypothesis follows from the fact that if all the elements
outside the diagonal were $0$ this would imply that
$a_{ii}y_i=0\in\mathbf K$ and since
$a_{ii}<n<\operatorname{char}(\mathbf K)$, $a_{ii}\neq0$ would
give a contradiction.

Let us check that the third hypothesis holds. Indeed by (\ref{idea}),
$${\sum_{i=1}^t a_{ij}}=\#\{s\in[2,\ldots,n]\ |\ X_{s}=y_j, X_{s-1}=0\}+\epsilon_j\geq0.$$
where $\epsilon_j=1$ if $j=1$ and $0$ otherwise. It follows that the sum
of the elements in the first column is strictly positive.
This concludes the proof of the Theorem.\hfill$_\Box$\medskip

\noindent\textbf{Proof of Theorem~\ref{geometrico}.}
Apply the Corollary to Theorem~\ref{Lucia} with
$\mathbf K=\overline{\mathbb F}_q$.
For char($\mathbb F_q)>e^{(k-3)/e}$, we have the bound
$$\#\mathcal A_k(\mathbb F_q)\leq \#\mathcal A_k(\overline{\mathbb F}_q)= (k-1)!$$

Finally, from (\ref{citato}) and from Proposition~\ref{normal} we obtain
$$m_{[k]}(q)= \frac{q(q-1)}{k}n_{[k]}(q)
\leq \frac{q(q-1)}{k}\#\mathcal A_k({\mathbb F}_q)
\leq \frac{(k-1)!}{k}q(q-1).$$
and this concludes the proof of Theorem~\ref{geometrico}.\hfill$_\Box$\medskip


\section{Cyclotomic permutation polynomials. Proof of Theorem~\ref{ciclotomico}}

We want to prove Theorem~\ref{ciclotomico} by producing, in the
case $q\equiv1\pmod k$, $\varphi(k)$ distinct normalized
$k$--cycles in $N_{[k]}(\mathbb F_q)$.

Let us start noticing that the condition $q\equiv1\pmod k$ implies
that $\mathbb F_q$ contains all the $k$-th roots of unity and that
they are all distinct. Denote by $\zeta$ a primitive $k$-th root
of unity in ${\mathbb F}_q$. Consider the normalized $k$-cycle:
$$\sigma_\zeta=\left(0,\ 1,\ \ (1+\zeta),\ \ \cdots,
 \ \  (1+\zeta+\cdots +\zeta^{k-2})\right).$$
Clearly as $\zeta$ varies among the $\varphi(k)$ primitive
$k$--roots of unity, we obtain distinct normalized $k$--cycles. We
want to check that $\partial(f_{\sigma_\zeta}) =q-k$ (i.e.
$\sigma_\zeta\in N_{[k]}(\mathbb F_q)$).

Let us compute,  for $i=1,\ldots,k-2$,
$$\begin{array}{rl}
A_i(\sigma_\zeta)=& G_{i}((1+\zeta),\ldots,(1+\zeta+\cdots +\zeta^{k-2})) \\
&\\
=&-\left({\sum_{j=0}^{k-3}}\zeta^{j+1}
(1+\zeta+\cdots+\zeta^{j})^i\right)+\left(1+\zeta+\cdots+\zeta^{k-2}\right)^{i+1}\\
&\\
  =& \frac{-1}{(\zeta-1)^i}\left(\displaystyle{\sum_{j=0}^{k-3}}
\zeta^{j+1}(\zeta^{j+1}-1)^i-\frac{(\zeta^{k-1}-1)^{i+1}}{\zeta-1}\right)\\
&\\
 =& \frac{-1}{(\zeta-1)^i}\left(\displaystyle{\sum_{j=0}^{k-3}} \zeta^{j+1}\sum_{t=0}^i
\binom{i}{t}(-1)^{i-t}\zeta^{(j+1)t}\ -\frac{(\zeta^{k-1}-1)^{i+1}}{\zeta-1}\right).
\end{array}$$
Interchange the two sums of the last equation and observe that, since $t+1\leq i+1\leq k-1$
and $\zeta$ is primitive, we have
$$\sum_{j=0}^{k-3} (\zeta^{t+1})^j=\frac{\zeta^{(k-2)(t+1)}-1}{\zeta^{t+1}-1}.$$
Therefore
$$\begin{array}{rl}
A_i(\sigma_\zeta)=&
\frac{-1}{(\zeta-1)^i}\left(\displaystyle{\sum_{t=0}^i}\binom{i}{t}(-1)^{i-t}\zeta^{t+1}
\frac{\zeta^{(k-2)(t+1)}-1}{\zeta^{t+1}-1}\ \ \ - \frac{(\zeta^{k-1}-1)^{i+1}}{\zeta-1}\right)
\end{array}$$
Now use the fact that $\zeta^{k-1}=\zeta^{-1}$. The above becomes
$$\begin{array}{rl}
A_i(\sigma_\zeta)=&
\frac{-1}{(\zeta-1)^i}\left(\displaystyle{\sum_{t=0}^i}\binom{i}{t}(-1)^{i-t}\zeta^{1+t}\frac{\zeta^{-2(t+1)}-1}{
\zeta^{t+1}-1}\ \ \ - \frac{(\zeta^{-1}-1)^{i+1}}{\zeta-1}\right)\\
& \\
=&\frac{-1}{(\zeta-1)^i}\left(\displaystyle{-\left(\sum_{t=0}^i\binom{i}{t}(-1)^{i-t}\zeta^{-t}\right)}\zeta^{-1}
\ \ \ + \zeta^{-1}(\zeta^{-1}-1)^{i}\right)\\
& \\
=& \frac{1}{\zeta(\zeta-1)^i}\left(\displaystyle{\sum_{t=0}^i}\binom{i}{t}(-1)^{i-t}\zeta^{-t}\ \ \
 -(\zeta^{-1}-1)^{i}\right)\\
&  \\
=& 0.
\end{array}$$

Finally, recalling that $\partial(f_\sigma)\geq q-c(\cal C)$, we have
$\sigma_\zeta\in N_{[k]}(\mathbb F_q)$ for all primitive $\zeta$. Therefore
$n_{[k]}(q)\geq \varphi(k)$ and by Proposition~\ref{normal},
this concludes the proof of Theorem~\ref{ciclotomico}.\hfill$_\Box$\bigskip

\noindent\textbf{Remark.} We will call the permutations
$\sigma_\zeta$ \textit{cyclotomic permutations}. In the case
$k=3$, Theorem~\ref{geometrico} gives that $N_{[3]}(\mathbb
F_q)\leq \frac{2}{3}q(q-1)$ while Theorem~\ref{ciclotomico} gives
that if $q\equiv1\pmod 3$, then $N_{[3]}(\mathbb
F_q)\geq\frac{2}{3}q(q-1)$. Therefore all normalized $3$-cycles
are cyclotomic permutations if $q\equiv1\pmod3$. On the other hand
in 1969, C. Wells \cite{W} proved the formula
\begin{equation}\label{wello}
N_{[3]}(\mathbb F_q)=
\left\{\begin{array}{lr}
\frac{2}{3}q(q-1) & \textrm{if\ } q\equiv1\pmod3\\
& \\
0 & \textrm{if\ } q\equiv2\pmod3\\
& \\
\frac{1}{3}q(q-1) & \textrm{if\ } q\equiv0\pmod3\\
\end{array}\right..
\end{equation}
Our results can be seen as generalizations of the above. Note that
in \cite[page 50]{W}  there is a missprint in the case
$q\equiv0\pmod3$ where the claim that $N_{[3]}(3^n)=3^n(3^n-1)$
should be corrected into $N_{[3]}(3^n)=3^{n-1}(3^n-1)$ as all
possible $3$-cycles permutations are
$$(a,\ a+b,\ a-b)$$
which for all choices of $a, b\in\mathbb F_{3^n}$ give rise to the
above amount of permutations.

\section{Numerical examples: the number of $k$--cycles with minimal degree for $k\leq 6$}

In this Section we consider the specific examples of $4$, $5$ and
$6$--cycles. The case of $3$--cycles has been analyzed with by
Wells \cite{W} (see the Remark in the previous section).

\subsection{Computation of $m_{[4]}(q)$}

We will prove the following:

\begin{Theorem}\label{uno4} Let $m_{[4]}(\mathbb F_q)$ be the number of $4$--cycle permutations
of $\mathbb F_q$ such that their permutation polynomial has minimal degree $q-4$.
Then, if $(q,10)=1$,
$$m_{[4]}(\mathbb F_q)=\frac{1}{4}q(q-1)K_q$$
where
$$K_q=\left\{
\begin{array}{rrll}
6 & \textrm{if } q\equiv &1 &\pmod{20}\\
4 & \textrm{if } q\equiv &11 &\pmod{20}\\
2 & \textrm{if } q\equiv &9,13,17&\pmod{20}\\
0 & \textrm{if } q\equiv & 3,7,19&\pmod{20}\\
\end{array}\right.$$
while
$$m_{[4]}(\mathbb F_{5^n})=\frac{1}{2}{5^n}({5^n-1})$$
and
$$m_{[4]}(\mathbb F_{2^n})=\left\{\begin{array}{rl}
{2^n}({2^n-1}) & \text{if $4|n$}\\
0 & \text{otherwise.}\end{array}\right.
$$
\end{Theorem}

\noindent\textbf{Remark.} {From} (\ref{degfund}) it follows that the degree of a $4$--cycle permutation polynomial
can either be $q-2$, $q-3$
or $q-4$. In \cite{CP} we proved that the number of $4$--cycle permutation polynomials over $\mathbb F_q$ with
degree strictly less then $q-3$ is
$$\frac{1}{4}q(q-1)t_q
\ \ \text{ where }\ \
t_q=\left\{\begin{array}{rl}
q-11 & \text{if } q\equiv 1\pmod{12}\\
q-3  & \text{if } q\equiv 5\pmod{12}\\
q-7 & \text{if } q\equiv 7\pmod{12}\\
q+1 1 & \text{if } q\equiv 11\pmod{12}\\
(q-4)(1+(-1)^n) &  \text{if } q=2^n\\
q-5-2(-1)^n &  \text{if } q=3^n
\end{array}\right.$$

This result together with Theorem~\ref{uno4} provides complete
information of the number of $4$--cycles of each given
degree.\bigskip

\noindent\textbf{Proof of Theorem~\ref{uno4}.}
{From} Proposition~\ref{normal}, we have that
$$m_{[4]}(\mathbb F_q)=\frac{q(q-1)}{4}n_{[4]}(\mathbb F_q)$$
and from (\ref{citato}) it follows that
$$n_{[4]}(\mathbb F_q)
\#\left\{(x,y)\in(\mathbb F_q\setminus\{0,1\})^2\ \left| {x\neq y},
(x,y)\in\mathcal A_4(\mathbb F_q)\right.\right\}$$
where
$$\mathcal A_4:\left\{
\begin{array}{r}
(1-x)+x(x-y)+y^2=0\\
\\
(1-x)+x^2(x-y)+y^3=0\\
\end{array}
\right.$$

The resultant $R$ with respect to the variable $y$ of the two
equations defining $\mathcal A_4$ is
$$\begin{array}{rl}
R= &10\,{x}^{4}-4\,{x}^{5}+{x}^{6}+15\,{x}^{2}-15\,{x}^{3}-8\,x+2=\\
= & \left ({x}^{2}-2\,x+2\right )\left
({x}^{4}-2\,{x}^{3}+4\,{x}^{2}-3\,x +1\right ).\end{array}
$$
Now denote by $h_1(x)$ the first factor and by $h_2(x)$ the second. The resultant
of $h_1$ and $h_2$ is equal to  $5$. Therefore, if $(q,5)=1$, $h_1$ and $h_2$
will never have common roots.

The number of roots of $h_1(x)$ is
\begin{equation}\label{one}
\left\{\begin{array}{rl}
$0$ & \text{if $q\equiv 3\pmod 4$}\\
$2$ & \text{if $q\equiv 1\pmod 4$}\\
$1$ & \text{if $q$ is even}.\end{array}\right.\end{equation}
Furthermore, if $q\equiv1\pmod4$ and $\iota=\sqrt{-1}$ is a
primitive 4---th root of unity in $\mathbb F_q^*$, from the roots
of $h_1$ we can construct the two points of $\mathcal A_4(\mathbb
F_q)$
\begin{equation}\label{ciclo4}
(x_1,y_1)=(1+\iota,1+\iota+\iota^2),\ \ (x_2,y_2)=(1-\iota,1-\iota+\iota^2).
\end{equation}
These points give rise to the two distinct (normalized) cyclotomic
permutations:
$$
(0,\ \ 1,\ \ (1+\iota),\ \ (1+\iota+\iota^2));\ \ \ (0,\ \ 1,\ \
(1-\iota),\ \ (1-\iota+\iota^2)).$$ If $q$ is even, then the root
$x=0$ of $h_1$ gives the point $(0,1)\in\mathbb F_{2^n}$ that
leads to no permutation polynomials.

Let us now deal with $h_2$. We claim that the number of roots of $h_2$ is
\begin{equation}\left\{\label{two}\begin{array}{rl}
$4$ & \text{if $q\equiv 1\pmod 5$}\\
$1$ & \text{if $q\equiv 0\pmod 5$}\\
$0$ & \text{otherwise.}\end{array}\right.\end{equation} Indeed a
calculation shows that if $\zeta$ is a primitive $5$--th root of
unity in $\mathbb F_q$, then
$$h_2(x)=\prod_{i=1}^4\left (x-1-\zeta^i-{\zeta}^{2i}\right )$$
while
$$h_2(x)\equiv (x+2)^4\pmod5.$$
Hence (\ref{two}) follows.

If $q\equiv 1\pmod 5$  and $x_i$ is a root of $h_2$ then a
computation shows that
$y_i=1-2\,x_{{i}}+{x_{{i}}}^{2}-{x_{{i}}}^{3}$ is the only value
for which $(x_i,y_i)\in\mathcal A_4(\mathbb F_q)$.

The conditions $x_i=0$ or $x_i=1$ are never satisfied since
$h_2(0)=1$ and $h_2(1)=1$ and the other conditions
$$x_i=y_i,\ y_i=0,\ y_i=1$$
are also never satisfied. This is easily checked by some
computation. For example the condition $y_i=0$ can be checked by
calculating the resultant between $h_2(x)$ and
$1-2\,x+x^{2}-x^{3}$. This resultant is equal to $1$.

Putting together (\ref{one}) and (\ref{two}), and working out the
various congruence relations modulo $20$, we obtain the claim for
characteristic different from $2$ and $5$.

Let us now deal with the case when $q=5^n$. The two roots of
$h_1(y)$ will provide the two points of $\mathcal A_4(\mathbb
F_{5^n})$ $(3,2)$ and $(4,3)$, while
$$h_2(x)=(x+2)^4$$
has only one root $x_1=3$ which gives $y_1=2$, but the point
$(2,3)\in\mathcal A_4(\mathbb F_{5^n})$ has already been counted.
Therefore $\#\mathcal A_4(\mathbb F_{5^n})=2$.

Finally let us deal with the case when $q=2^n$. The root $x=0$ of $h_1(x)$ has to be excluded and
$h_2(x)$ provides $4$ distinct points if $2^n\equiv 1\pmod 5$ (i.e. $n|4$).

This concludes the proof of the Theorem.\hfill$_\Box$\bigskip

\noindent\textbf{Remark.} We want to summarize the process that we
used to construct all the points in $\mathbb A_4(\mathbb F_q)$
since we will adopt the same approach in the following examples.
\begin{enumerate}
\item We have decomposed
$$\mathcal A_4(\overline{\mathbb F_q}) = \mathcal A_4(\mathbb F_q(\sqrt{-1}))
\cup \mathcal A_4(\mathbb F_q(\zeta_5))$$
where the union is disjoint except in the case $5|q$;
\item We have checked that the coordinates of each point of $\mathcal A_4(\overline{\mathbb F_q})$ were distinct
and different from $0$ or $1$. This has always been the case except when $2|q$.
\item If $(q,10)=1$, then the number $m_{[4]}(q)$ is $q(q-1)/4$ times $n_1+n_2$ where $n_1$ is the number
of points in $\mathcal A_4(\mathbb F_q(\sqrt{-1}))$ and $n_2$ is the number
of points in $\mathcal A_4(\mathbb F_q(\zeta_5))$.
\end{enumerate}

Note that for every prime $p\neq 2,5$, $n_1$ is the number of prime ideals of $\mathbb Q(\sqrt{-1})$ over $p$ and
$n_2$ is the number of prime ideals of $\mathbb Q(\zeta_5)$ over $p$. This property suggests to first look
at $\mathcal A_4(\overline{\mathbb Q})$ and then consider the reduction in the various finite fields. We will follow this approach in the sequel.

\subsection{Computation of $m_{[5]}(q)$.}

We will prove the following:

\begin{Theorem} Let $q$ be a power of a prime $p$ which is not
in the set
$$\{2, 13, 61, 3719, 3100067\}.
$$
Then
$$m_{[5]}(\mathbb F_q)=\frac{q(q-1)}{5}s_q.$$
Where
$$s_q=r_q+t_q+u_q,\ t_q=\left\{
\begin{array}{rl}
4 & \text{if $q\equiv1\pmod 5$}\\
1& \text{if $q\equiv0\pmod 5$}\\
0 & \text{otherwise,}\end{array}
\right.
\ \ u_q=\left\{
\begin{array}{rl}
-1 & \text{if $p=11,41$}\\
0 & \text{otherwise}\end{array}
\right.
$$
and $r_q$ is the number of roots in $\mathbb F_q$ of the
polynomial
$$\begin{array}{rl}
g_2(x)=&
2\,{x}^{20}-29\,{x}^{19}+229\,{x}^{18}-1249\,{x}^{17}+5187\,{x}^{16}-\\
&17222\,{x}^{15}+47040\,{x}^{14}-107505\,{x}^{13}+207622\,{x}^{12}-\\
&340496\,{x}^{11}+474638\,{x}^{10}-560999\,{x}^{9}+559052\,{x}^{8}-\\
&465487\,{x}^{7}+319628\,{x}^{6}-177653\,{x}^{5}+77807\,{x}^{4}-25797\,{x}^{3}+\\
&6074\,{x}^{2}-904\,x+64
\end{array}.$$
\end{Theorem}

\noindent{\textbf{Proof.}} Again we start from the formula:
$$m_{[5]}(\mathbb F_q)=\frac{q(q-1)}{5}\#\{(x,y,z)\in
\mathcal A_5(\mathbb F_q), x,y,z\not\in\{0,1\}, x\neq y\neq z\neq
x\},$$ where
$$\mathcal A_5:\left\{
\begin{array}{r}
H_1=(1-x)+x(x-y)+y(y-z)+z^2=0\\
\\
H_2=(1-x)+x^2(x-y)+y^2(y-z)+z^3=0\\
\\
H_3=(1-x)+x^3(x-y)+y^3(y-z)+z^4=0.\\
\end{array}
\right.$$ Let us first compute $\mathcal A_5(\overline{\mathbb
Q})$.

{From} Theorem~\ref{Lucia} we know that $\#\mathcal A_5(\overline{\mathbb Q})=24$.
Furthermore $4$ points of $\mathcal A_5(\overline{\mathbb Q})$ are the cyclotomic
ones
\begin{equation}
\label{ciclo5}
(1+\zeta^j,1+\zeta^j+\zeta^{2j},1+\zeta^j+\zeta^{2j}+\zeta^{3j}),\ \ \zeta=e^{2\pi i/5}, j=1,2,3,4.
\end{equation}
We solve the system of equations defining $\mathcal A_5$ in the
following way. Consider  $H_2-(z+y)H_1=0$ and note that we can
solve it for $z$ obtaining
\begin{equation}\label{one5}z={\frac {{x}^{3}-2\,{x}^{2}y+x{y}^{2}+xy-x-y+1}{{x}^{2}-xy+{y}^{2}-x+1}
}
\end{equation}
Similarly, consider  $H_3-zH_2-y^2H_1=0$. Also here we can solve it for $z$
obtaining:
\begin{equation}\label{two5}
z=
{\frac {{x}^{4}-{x}^{3}y-{x}^{2}{y}^{2}+x{y}^{3}+x{y}^{2}-{y}^{2}-x+1}
{{x}^{3}-{x}^{2}y+{y}^{3}-x+1}}
\end{equation}
Now, subtracting $H_3-z^2H_1-yH_2=0$, we can solve it for $z^2$
obtaining:
\begin{equation}\label{three5}
z^2=
{\frac {{x}^{4}-2\,{x}^{3}y+{x}^{2}{y}^{2}+x y-x-y+1}{{x}^{2}-x y+{y}^{2}-x+1}}
\end{equation}
Replacing $z^2$ in $H_1$ with the right hand side of
(\ref{three5}) and $z$ with the right hand side of (\ref{one5}) we
obtain (after simplification):
\begin{equation}\label{four5}\begin{array}{r}
 ({{x}^{4}-2\,{x}^{3}y+{x}^{2}{y}^{2}+x y-x-y+1})-y({{x}^{3}-2\,{x}^{2}y+x{y}^{2}+x y-x-y+1})
\\
(1-x+x(x-y)+y^2)^2 =0\end{array}
\end{equation}
Finally consider the equation obtained replacing $z$ in
(\ref{one5}) by the right hand side of (\ref{two5})
\begin{equation}\label{five5}
    \begin{array}{rl}
\left({{x}^{4}-{x}^{3}y-{x}^{2}{y}^{2}+x{y}^{3}+x{y}^{2}-{y}^{2}-x+1}\right)
\left({{x}^{2}-x y+{y}^{2}-x+1}\right)-&\\
\left({{x}^{3}-{x}^{2}y+{y}^{3}-x+1}\right)\left({{x}^{3}-2\,{x}^{2}y+x{y}^{2}+x y-x-y+1}\right)=0
    \end{array}
\end{equation}
In this way we have eliminated the variable $z$. We might have introduced new solutions but we
will see later that this is not the case.

We have used Maple V to compute the resultant $R$ of (\ref{four5}) and (\ref{five5}) with respect
to $y$ and we obtained:
$$R=g_1(x)\cdot g_2(x)$$
where
\begin{equation}\label{g1}
g_1(x)={x}^{4}-3\,{x}^{3}+4\,{x}^{2}-2\,x+1
\end{equation}
and
\begin{equation}\label{g2}
\begin{array}{rl}
g_2(x)=&
2\,{x}^{20}-29\,{x}^{19}+229\,{x}^{18}-1249\,{x}^{17}+5187\,{x}^{16}-\\
&17222\,{x}^{15}+47040\,{x}^{14}-107505\,{x}^{13}+207622\,{x}^{12}-\\
&340496\,{x}^{11}+474638\,{x}^{10}-560999\,{x}^{9}+559052\,{x}^{8}-\\
&465487\,{x}^{7}+319628\,{x}^{6}-177653\,{x}^{5}+77807\,{x}^{4}-25797\,{x}^{3}+\\
&6074\,{x}^{2}-904\,x+64
\end{array}\end{equation}

Now the splitting field of $g_1(x)$ is $\mathbb Q(e^{2\pi i/5})$. Furthermore the roots of
$g_1$ are
$$x_j=(1+\zeta^i),\ \ \ j=1,2,3,4.$$
We can also easily compute  $x$ and $y$ for each of the above.
Hence $\mathcal A_5(\mathbb Q(\zeta_5))$ is exactly the set
described in (\ref{ciclo5}).

Let ${\mathbf M}_5$ be the splitting field of $g_2$. For each root
$\alpha$ of $g_2(x)$, one can compute
$(\alpha,y(\alpha),z(\alpha))\in A_5({\mathbf M}_5)$ where:

\begin{small}
$$\begin{array}{l}\mbox{\normalsize $y(x)=$}
\frac{1}{2^{3}\cdot13\cdot61\cdot3719\cdot3100067}\left(6245340990732510-74275247020348477\,x\right.\\
+425897367479627411\,x^{2}-1556772755104088477\,x^{3}+4068122356423765520\,x^{4}\\
-8092377944341897339\,x^{5}+12739155747072503154\,x^{6}-16281608694400072277\,x^{7}+\\
17191467892889878476\,x^{8}-15176855331347725064\,x^{9}+11289210111615920188\,x^{10}\\
-7103742513094855073\,x^{11}+3782081407301444460\,x^{12}-1696979431552752820\,x^{13}\\
+635807089991226023\,x^{14}-195705738631474759\,x^{15}+48121368022605621\,x^{16}\\
\left.-9009616966592957\,x^{17}+1165803130533438\,x^{18}-82558295396232\,x^{19}\right)
\end{array}
$$
\end{small}


and from (\ref{one5}) and some computation
\begin{small}
$$\begin{array}{l}\mbox{\normalsize $z(x)=$}
\displaystyle{{\frac {{x}^{3}-2\,{x}^{2}y(x)+x{y(x)}^{2}+x y(x)-
x-y(x)+1}{{(x)}^{2}-x y(x)+{y(x)}^{2}-x+1}
}}=\\
\frac{1}{2^{3}\cdot13\cdot61\cdot3719\cdot3100067}\left(-292290150269490\,{x}^{19}+3950333490943181\,{x}^{18}\right.\\
-29484664428617801\,{x}^{17}+152268243151302965\,{x}^{16}-599002775464475543\,{x}^{15}\\
+1880438345917167218\,{x}^{14}-4841135989461751552\,{x}^{13}+10378374551469856881\,{x}^{12}\\
-18679878403151115130\,{x}^{11}+28303942873286020848\,{x}^{10}-36041151267474587782\,{x}^{9}\\
+38336702176933085823\,{x}^{8}-33711958096174593304\,{x}^{7}+24129466512539278343\,{x}^{6}\\
-13742359416000756136\,{x}^{5}+6020424561116746133\,{x}^{4}-1925677501494324283\,{x}^{3}\\
\left(+413273185040891961\,{x}^{2}-51203861193252214\,x+2593061963570136\right)
\end{array}
$$
\end{small}

Finally
$$\mathcal A_5(\overline{\mathbb Q})=\mathcal A_5({\mathbb Q}(e^{2\pi i/5}))\cup
\mathcal A_5({\mathbf M}_5)$$
where the union is disjoint.

We are now ready to investigate $\mathcal A(\mathbb F_q)$.

The roots of $g_1(x)$ in $\mathcal A(\mathbb F_q)$ are $4$ if
$q\equiv1\pmod 5$ and in this case the $4$ points give the
cyclotomic permutation polynomials. If $q=5^n$, then $g_1(x)=
(x+3)^4$ and the root $x_0=2$ leads to the point $(2,3,4)\in
\mathcal A(\mathbb F_{5^n})$ and therefore to the normalized
$5$--cycle $(0,\ 1,\ 2,\ 3,\ 4)$.

Let us deal with the roots of $g_2(x)$. The characteristics
\begin{equation}\label{bad5}
\{2, 13, 61, 3719, 3100067 \}\end{equation}
appearing in the denominators of $y(x)$ and $z(x)$ will have to
be treated separately and we have not done it here.

For all other primes, note that $g_2(0)=2^6$, $g_2(1)=2$ and we have the following resultants:
$$
\begin{array}{rlrlrl}
\operatorname{R}(y(x),g_2(x))=&2^{24} &
\operatorname{R}(y(x)-1,g_2(x))=&2^{24} &
\operatorname{R}(y(x)-x,g_2(x))=&2^{19}\\
\operatorname{R}(z(x),g_2(x))=& 2^{19} &
\operatorname{R}(z(x)-1,g_2(x))=&2^{24} &
\operatorname{R}(z(x)-x,g_2(x))=&2^{24}\\
\end{array}
$$
$$\operatorname{R}(y(x)-z(x),g_2(x)) = 2^{19}
$$

where $R(a,b)$ is the resultant of the univariate polynomials $a$
and $b$. Therefore, for any finite field $\mathbb F_q$ (of
characteristic distinct from those in (\ref{bad5})), if
$g_2(x_0)=0$, then  $(x_0,y(x_0),z(x_0))\in\mathcal A(\mathbb
F_q)$ and $\sigma=(0,\ 1,\ x_0,\ y(x_0),\ z(x_0))$ is a well
defined normalized permutation in $n_{[5]}(\mathbb F_q)$.

The characteristics $\{11, 41, 160591\}$ are those for which
$g_1(x)$ and $g_2(x)$ have roots in common. These can be
determined by computing the resultant $R(g_1,g_2)$.

For $p=11$, the only common root is $x=6$ and the only point in
$\mathcal A_5(\mathbb F_{11^n})$ that has such a value as first
coordinate is $(6,9,2)$; for $p=41$, the only common root is
$x=38$ and the only point in $\mathcal A_5(\mathbb F_{41^n})$ that
has such an $x$ is $(38,13,31)$. Therefore in these two cases the
number of normalized permutation should be one less. Finally for
$p=160591$ the only common root is $x=93$ but there are two points
in $\mathcal A_5(\mathbb F_{160591^n})$ with $x=93$ which are
$(93,8557,144881)$ and $(93,36072,14312)$.

This concludes the proof.\hfill$_\Box$\bigskip


\subsection{Partial computation of $m_{[6]}(q)$.}

Let us consider the affine algebraic set $\mathcal A_{6}$:
$$\mathcal A_6:\left\{
\begin{array}{r}
H_1=(1-x)+x(x-y)+y(y-z)+z(z-t)+t^2=0\\
\\
H_2=(1-x)+x^2(x-y)+y^2(y-z)+z^2(z-t)+t^3=0\\
\\
H_3=(1-x)+x^3(x-y)+y^3(y-z)+z^3(z-t)+t^4=0\\
\\
H_4=(1-x)+x^4(x-y)+y^4(y-z)+z^4(z-t)+t^5=0.\\
\end{array}
\right.$$

We know from Theorem~\ref{Lucia} that $\#\mathcal
A_6(\overline{\mathbb Q})=120$. The problem can be solved along
the same lines as in the last subsection. Here is the Maple V
program that
we used: %\bigskip

\begin{tt}
\begin{tabular}{l}
restart:\\
H[1]:=1-x+x*(x-y)+y*(y-z)+z*(z-t)+t\^{}2:\\
H[2]:=1-x+x\^{}2*(x-y)+y\^{}2*(y-z)+z\^{}2*(z-t)+t\^{}3:\\
H[3]:=1-x+x\^{}3*(x-y)+y\^{}3*(y-z)+z\^{}3*(z-t)+t\^{}4:\\
H[4]:=1-x+x\^{}4*(x-y)+y\^{}4*(y-z)+z\^{}4*(z-t)+t\^{}5:\\
F[1]:=solve(H[2]-(t+z)*H[1],t):\\
F[2]:=solve(H[3]-t*H[2]-z\^{}2*H[1],t):\\
F[3]:=solve(H[4]-t*H[3]-z\^{}3*H[1],t):\\
F[4]:=solve(H[3]-z*H[2]-t\^{}2*H[1],t)[1]\^{}2:\\
G[1]:=numer(F[1]-F[2]):\\
G[2]:=numer(F[3]-F[1]):\\
G[3]:=numer(F[4]-z*F[1]+1-x+x*(x-y)+y*(y-z)+z\^{}2):\\
A[1]:=resultant(G[1],G[2],z):\\
A[2]:=resultant(G[1],G[3],z):\\
A[3]:=resultant(G[2],G[3],z):\\
B[1]:=resultant(A[1],A[2],y):\\
B[2]:=resultant(A[1],A[3],y):\\
B[3]:=resultant(A[2],A[3],y):\\
factor(gcd(B[1],gcd(B[2],B[3])));
\end{tabular}
\end{tt}
%\bigskip

It produces as output:
$$f_1(x)\cdot f_2(x)\cdot f_3(x)\cdot f_4(x)\cdot g_2(x)^2 \cdot g_1(x)^2$$
where $g_1$ and $g_2$ are the same polynomials of the previous
subsection and do not yield any point in $\mathcal
A_6(\overline{\mathbb Q})$,
$$f_1(x)={x}^{2}-3\,x+3,\ f_2(x)={x}^{4}-3\,{x}^{3}+9\,{x}^{2}-9\,x
+3,$$
$$f_3(x)={x}^{6}-4\,{x}^{5}+12\,{x}^{4}-22\,{x}^{3}+25\,{x}^{2}-14\,x+3$$
and $f_4(x)$ is a degree 108 polynomial shown below. Very little
can be done about it (e.g. we cannot factor its discriminant).
However we know that given one of its 108 roots $x$, there exist
rational functions $y(x)$, $z(x)$, $t(x)$ such that
$(x,y(x),z(x),t(x))\in\mathcal A_6(\overline{\mathbb Q})$.

\begin{tiny}
$$
\begin{array}{l}
\mbox{$\displaystyle{f_4(x)=}$}\\
2048\,{x}^{108}-165888\,{x}^{107}+6799872\,{x}^{106}-187752960\,{x}^{105}+3922763776\,{x}^{104}-66068319680\,{x}^{103}\\
+933320077408\,{x}^{102}-11363232453904\,{x}^{101}+121609445410488\,{x}^{100}-1161198732496436\,{x}^{99}\\
+10008850476882864\,{x}^{98}-78606667549447068\,{x}^{97}+566828548445747784\,{x}^{96}-3776776878293093668\,{x}^{95}\\
+23377338985281206132\,{x}^{94}-135038479362980318078\,{x}^{93}+730833294640515925896\,{x}^{92}\\
-3718457594383449440377\,{x}^{91}+17839854280234088048504\,{x}^{90}-80918773915266921688911\,{x}^{89}\\
+347817829980603691940144\,{x}^{88}-1419720414224675767707558\,{x}^{87}+5513288219047478965908265\,{x}^{86}\\
-20403343418466290909559217\,{x}^{85}+72065722093337704619789754\,{x}^{84}-243267380374046351368535386\,{x}^{83}\\
+785782176891688617129372777\,{x}^{82}-2431475137872624992934580357\,{x}^{81}+7214881866132247318290915548\,{x}^{80}\\
-20548659512217571859089105859\,{x}^{79}+56221257258312886794846517663\,{x}^{78}\\
-147882404554712812657831273826\,{x}^{77}+374230043847540583315597499959\,{x}^{76}\\
-911691931385646228439986925230\,{x}^{75}+2139449841280212409103799322605\,{x}^{74}\\
-4838781255382865924142092881113\,{x}^{73}+10552734185292011384044411424566\,{x}^{72}\\
-22201680743797784367677070019329\,{x}^{71}+45079400421222501688611989232857\,{x}^{70}\\
-88370131835128893374420804013985\,{x}^{69}+167308044867058677528870842347726\,{x}^{68}\\
-306018091440642946312309370096773\,{x}^{67}+540901707766162203714093161026402\,{x}^{66}\\
-924145503563203698506557364196092\,{x}^{65}+1526550997692643704549449565023087\,{x}^{64}\\
-2438475861371766718260022687359403\,{x}^{63}+3767372156555906676771592362227252\,{x}^{62}\\
-5630386387795750914878787596953278\,{x}^{61}+8140939139357659835287965640730513\,{x}^{60}\\
-11389249472014526491272002805160961\,{x}^{59}+15418372730959804119154464592501925\,{x}^{58}\\
-20199212963332568595992849480574793\,{x}^{57}+25609653568875523492121650783680523\,{x}^{56}\\
-31423786674815287982856648211485665\,{x}^{55}+37316636201275774332720329351064002\,{x}^{54}\\
-42887487674528678056202555216506600\,{x}^{53}+47701189936940459634356766395673379\,{x}^{52}\\
-51342370723861934089578323730701241\,{x}^{51}+53473605808047948459645336373877904\,{x}^{50}\\
-53886451035411447622870618958843743\,{x}^{49}+52534755181805885450898834956212846\,{x}^{48}\\
-49542943308323803416607202273116258\,{x}^{47}+45187255671028388860208926229245651\,{x}^{46}\\
-39853834776380146454538798283342894\,{x}^{45}+33982440027129229551505906960627180\,{x}^{44}\\
-28007124276968506959166892217933313\,{x}^{43}+22304864978517515995360691909021985\,{x}^{42}\\
-17160233130232486543207338901006740\,{x}^{41}+12749752732446670751318525720373287\,{x}^{40}\\
-9145009691119593703082103400176233\,{x}^{39}+6330015056336126215839775082630388\,{x}^{38}\\
-4226514548260823401239096740258288\,{x}^{37}+2720942861141654856875643560186175\,{x}^{36}\\
-1688109362923362914905627143729438\,{x}^{35}+1008768415625546502227835131362059\,{x}^{34}\\
-580279479672452851412256035043958\,{x}^{33}+321115048795909727104320629068345\,{x}^{32}\\
-170828586279878576733366714859762\,{x}^{31}+87299009196449969872102046466464\,{x}^{30}\\
-42820393799631399542004753630026\,{x}^{29}+20141728505903344673399260414668\,{x}^{28}\\
-9076637830955551006671020183951\,{x}^{27}+3914484242526498208181639312379\,{x}^{26}\\
-1613770892718885947479095327793\,{x}^{25}+635149654477638378211058318534\,{x}^{24}\\
-238326908840088875601414025833\,{x}^{23}+85127356062476807845436220758\,{x}^{22}\\
-28895680606614726658303804088\,{x}^{21}+9303712106309033749916140254\,{x}^{20}\\
-2835587512930135705228988470\,{x}^{19}+816198705952614985217016076\,{x}^{18}-221309370680671620177529840\,{x}^{17}\\
+56364041482436139001235584\,{x}^{16}-13439641318120378785990472\,{x}^{15}+2989147250976033209662704\,{x}^{14}\\
-617501175174317760066496\,{x}^{13}+117904318811134669800960\,{x}^{12}-20688959726700010283264\,{x}^{11}\\
+3313845039406468383232\,{x}^{10}-480619489043461936640\,{x}^{9}+62499665119858375680\,{x}^{8}\\
-7198855775276720128\,{x}^{7}+723131989749039104\,{x}^{6}-62070327274504192\,{x}^{5}+4427182693416960\,{x}^{4}\\
-251951884271616\,{x}^{3}+10728106885120\,{x}^{2}-303868936192\,x+4294967296
\end{array}
$$
\end{tiny}
We have named the above polynomial: ``\textit{The Devil's Hat}''.
For every root $\zeta$ of the polynomial $f_1$, we have the
cyclotomic points $$(\zeta,2\zeta-2,2\zeta-3,\zeta-2)\in\mathcal
A_6(\overline{\mathbb Q}).$$

For every root $\tau$ of the polynomial $f_2$, we have the points
in $\mathcal A_6(\overline{\mathbb Q})$
\begin{small}
$$\left(\tau,
\frac{1}{5}({11}- {19}\,\tau+{7}\,{\tau}^{2}-{3}\,{\tau}^{3}),
\frac{1}{5}({9}-{11}\,\tau+{3}\,{\tau}^{2}-{2}\,{\tau}^{3}),
\frac{1}{5}(-{7}+{13}\,\tau-{4}\,{\tau}^{2}+\,{\tau}^{3})\right).$$
\end{small}
Furthermore, for every root $\gamma$ of $f_3$,   we have the points of
$\mathcal A_6(\mathbb Q)$
$(\gamma,y_\gamma,z_\gamma,t_\gamma)$ where
$$\begin{array}{rl}
y_\gamma=&7-{\frac {61}{3}}\,\gamma+22\,{\gamma}^{2}-{\frac {41}{3}}\,{\gamma}^{
3}+14/3\,{\gamma}^{4}-4/3\,{\gamma}^{5},\\
z_\gamma=&6-{\frac {61}{3}}\,\gamma+22\,{\gamma}^{2}-{\frac {41}{3}}\,{\gamma}^{
3}+14/3\,{\gamma}^{4}-4/3\,{\gamma}^{5},\\
t_\gamma=&7-{\frac {64}{3}}\,\gamma+22\,{\gamma}^{2}-{\frac {41}{3}}\,{\gamma}^{
3}+14/3\,{\gamma}^{4}-4/3\,{\gamma}^{5}.\end{array}.$$

Finally
$$\mathcal A_6(\overline{\mathbb Q})=
\mathcal A_6({\mathbf K}_1)\cup\mathcal A_6({\mathbf
K}_2)\cup\mathcal A_6({\mathbf K}_3)\cup\mathcal A_6({\mathbf
K}_4),$$ where $\mathbf K_i$ is the splitting field of $f_i$. Note
however that the union is not disjoint this time. Indeed $\mathbf
K_1=\mathbb Q(\sqrt{-3})\subset\mathbf K_2=\mathbb Q(\sqrt
{-18+2\,\sqrt {-3}})$.

Furthermore
$$\#\mathcal A_6({\mathbf K}_i)=\left\{
\begin{array}{rl}
2   & \text{if $i=1$}\\
4   & \text{if $i=2$}\\
6   & \text{if $i=3$}\\
108 & \text{if $i=4$}
\end{array}
\right.
$$


Numerically if can be verified that all the coordinates of each
point of $\mathcal A_6(\overline{\mathbb Q})$ are distinct and
never in $\{0,1\}$. This allows us to conclude

\begin{Theorem} For all but finitely many characteristics
$$m_{[6]}(\mathbb F_q)=\frac{q(q-1)}{4}(s_1+s_2+s_3+s_4)$$
where $s_i$ is the number of roots of $f_i$ in $\mathbb F_q$.\hfill$_\Box$
\end{Theorem}

\section{Conclusion}

The complete computations of $\mathcal A_7$ is out of our reach
at the present.

It is natural to ask whether the construction of the present paper
can be extended to more general classes of permutations. The
answer is yes. Indeed if $\mathcal C$ is any partition with parts
larger then $1$, then one can define an algebraic set $\mathcal
A_{\mathcal C}$ analogue to $\mathcal A_k$. The connection with
normalized permutation polynomials with minimal degree can be
established also in this more general setting. However the
extensions of Theorem~\ref{geometrico} and
Theorem~\ref{ciclotomico} are not straightforward. We expect in
some cases stronger estimates to hold. For example it can be shown
that
$$m_{[2\ 3]}(\mathbb F_q)\leq 2q(q-1).$$
Numerical examples indicate interesting arithmetical properties.
For these reasons we have decided to dedicate a future paper to
general classes of permutation.
\bigskip

\noindent\textbf{Acknowledgements.} We would like to thank Lucia
Caporaso for suggesting the idea that led us to prove
Theorem~\ref{Lucia}, Furthermore we thank Igor Shparlinski for
offering shelter at Macquarie University where this manuscript was
finally written and revised.\bigskip

\begin{thebibliography}{99}

\bibitem{Das} \textsc{Das, P.}, \textit{The number of permutation polynomials
of a given degree over a finite field}, Finite Fields and Their
Applications \textbf{8} No. 4 (2002) 478--490.

\bibitem{Har} \textsc{Harris J.}, \textit{Algebraic Geometry, A First Course},
Springer-Verlag Graduate Texts in Math. 133, 1992.

\bibitem{KP} \textsc{Konyagin S. \& Pappalardi F.},
\textit{Enumerating Permutation Polynomials over Finite Fields by
Degree}, Finite Fields and Their Applications \textbf{8} No. 4
(2002) 548--553.

\bibitem{LB} \textsc{Levine J. \& Brawley J. V.},
\textit{Some Cryptographic Applications of Permutation Polynomials}, Cryptologia
\textbf{1}, Number 1, (1977)  76--92.

\bibitem{LMu} \textsc{Lidl. R. \& M\"uller W. B.}, \textit{A note on polynomials
and functions in cryptography}, Ars Combin. \textbf{17}A (1984) 223-229.

\bibitem{LM1} \textsc{Lidl R. \& Mullen G. L.},
\textit{When does a polynomial over a finite field permute the
elements of the field?} Amer. Math. Mon. \textbf{95} (1988)
243--246.

\bibitem{LM2} \textsc{Lidl R. \& Mullen G. L.},
\textit{When does a polynomial over a finite field permute the
elements of the field? II} Amer. Math. Mon. \textbf{100} (1993)
71--74.

\bibitem{LN} \textsc{Lidl R. \& Niederreiter, H.}, \textit{Finite Fields},
Encyclo. Math. and Appls. V. 20, Addison--Wesley, Reading, MA 1983.

\bibitem{CP} \textsc{Malvenuto C. \& Pappalardi F.},
\textit{Enumerating Permutation Polynomials I: Permutations with
Non-Maximal Degree}, Finite Fields and Their Applications
\textbf{8} No. 4 (2002) 531--547.

\bibitem{M} \textsc{Mullen G. L.}, \textit{Permutation Polynomials over finite fields},
 Finite fields, coding theory, and advances in communications and computing
 (Las Vegas, NV, 1991), 131--151, Lecture Notes in Pure and Appl. Math., 141, Dekker, New York, 1993.

\bibitem{W} \textsc{Wells C.}, \textit{The degrees of permutation polynomials over finite
fields}, J. Combinatorial Theory \textbf{7} (1969) 49--55.

\bibitem{Maple} Maple V Release 5.1 (1999), Waterloo Maple Inc.

\end{thebibliography}
\bigskip

\begin{footnotesize}
\noindent Claudia Malvenuto\\
Dipartimento di Informatica\\
Universit\`a degli studi ``La Sapienza''\\
Via Salaria, 113\\
I--00198, Roma -- ITALY.\\
\texttt{claudia@dsi.uniroma1.it}
\bigskip

\noindent Francesco Pappalardi\\
Dipartimento di Matematica\\
Universit\`a degli studi Roma Tre\\
Largo S. L. Murialdo, 1\\
I--00146, Roma -- ITALY.\\
\texttt{pappa@mat.uniroma3.it}
\end{footnotesize}

\vfill

\end{document}
