\documentclass[twoside,final,reqno,noamsfonts]{birkartspecial}
\usepackage{amssymb,amsfonts}
\newtheorem{theorem}{Theorem}[section]
\newtheorem{lemma}[theorem]{Lemma}
\newtheorem{corollary}[theorem]{Corollary}

\begin{document}
\volinfo
\setcounter{page}{33}

\title[Density Estimates Related to {Gau\ss} Periods]
{Density Estimates Related to\\ {Gau\ss} Periods}
\author[J.~von zur Gathen]{Joachim von zur Gathen}
\address{Fachbereich Mathematik-Informatik, Universit\"at Paderborn\br
D--33095 Paderborn,  Germany\br E-mail: {gathen@upb.de}}

\author[F.~Pappalardi]{Francesco Pappalardi}
\address{Dipartimento di Matematica, Universit\`{a} Roma Tre\br Largo S. L.
Murialdo, 1\br I--00146, Roma,  Italy\br E-mail:
{pappa@mat.uniroma3.it}}

\begin{abstract}
Given two integers $q$ and $k$, for any prime $r$ not dividing $q$ with
$r\equiv1\bmod k$, we denote by $\operatorname{ind}_r(q)$ the index of $q \bmod r$.
In~\cite{gaogat95} the question was raised
of calculating the density of the primes $r$ for which
$\operatorname{ind}_r(q)$ and $(r-1)/k$ are coprime; this is the
condition that the Gau{\ss} period in
${\mathbb F}_{{q}^{(r-1)/k}}$ defined by these data be normal over ${\mathbb F}_{q}$.
We assume the Generalized Riemann Hypothesis and calculate a formula
for this  density for
all $q$ and
$k$. We prove unconditionally that our formula  is an upper bound for the density and
then express it as an Euler product. Finally we apply the results to characterize the
existence of a special type of Gau\ss\ periods.
\end{abstract}

\maketitle

\section{Introduction}
Let $q$ and $k$ be integers with $| q | >1$ and $k>0$.
 For any prime $r$ not dividing
$q$, we define the index of $q\bmod r$ as $
\operatorname{ind}_r(q)=
[{\mathbb F}_r^*:\langle q\bmod r \rangle]$,
so that  $\operatorname{ind}_r(q)= (r-1)/\operatorname{ord}_{r}(q)$. If
$r\equiv 1\bmod k$, we also set
$$g_{q,k}(r)=\gcd
\left(\operatorname{ind}_r(q),(r-1)/ k\right).$$
Finally we let $M_{q,k}(x)$  be the
number of primes $r \equiv 1 \bmod k$
up to $x$ for which $g_{q,k}(r)=1$.

The interest in this quantity comes from the
construction of normal Gau\ss\ periods in ${\mathbb F}_{q^n}$ over
${\mathbb F}_q$, where $q \in {\mathbb N}$ is a prime power. If
$n = (r-1)/k$, $g_{q,k} (r) = 1$, $\beta \in {\mathbb F}_{q^{r-1}}$ is
 a primitive $r$--th root of unity,
$K \subseteq {\mathbb F}^{*}_r$ is
the unique subgroup of order $k$, and
$\alpha  = \sum_{i \in K} \beta^i$,
then $(n, k)$ is called in ~\cite{gaogat95} a \emph{Gau\ss\ pair}
(over ${\mathbb F}_a$), and indeed the {\em Gau{\ss} period} $\alpha$ generates
a normal basis for ${\mathbb F}_{q^n}$ over ${\mathbb F}_q$.
It was noted a few years ago that
such a normal basis is useful for fast exponentiation
in finite fields, which in turn has various
cryptographic applications. Theory and applications of this,
including implementations, are discussed in
\cite{gaogat95}, \cite{gaogat98}, \cite{gaogat00},
\cite{gaolen92},
\cite{gatnoe97}, \cite{gatnoe99a}.
A survey of these results is in
\cite{gatshp00a}.
%  see ~\cite{G-G-P}
% for an overview of the literature.
In particular, two elements of ${\mathbb F}_{q^n}$
represented in such a basis can be multiplied
at essentially the same cost as multiplying
two polynomials of degree $nk$ over ${\mathbb F}_q$.

Therefore a natural question is:
given $q$ and $n$ as above,
what is the smallest $k$ such that $(n,k)$
is a Gau\ss\ pair over ${\mathbb F}_q$?

In this paper we turn this question around
and ask: given $q$ and a (small) $k$,
for how many $n$ is $(n,k)$ a Gau\ss\ pair
over ${\mathbb F}_q$?

The paper \cite{feigat99}
gives a generalization of Gau{\ss} periods, where
basically the prime $r$ is replaced by an arbitrary integer;
our considerations only apply to the classical case as treated
by Gau{\ss}, where $r=nk+1$ is prime.

For $k=1$, it is clear that $g_{q,k}(r)=1$ if and only if
$\operatorname{ind}_r(q)=1$, and this happens exactly when $q$ is a primitive root
$\text{modulo } r$. Hence $M_{q,1}(x)$ is the number of primes $r$ up to $x$ for which
$q$  is a primitive root $\text{modulo } r$;
the famous Artin Conjecture
for primitive roots states that the set of these primes has a positive
density
unless $q$ is a square or equals $-1$. In 1965, C. Hooley~\cite{H}
proved that the Generalized Riemann Hypothesis implies the asymptotic formula
$$M_{q,1}(x)=\left(\delta_q+\operatorname{O}\left(\frac{\log\log x +\log q}{\log x}\right)\right)
\frac{x}{\log x} $$
uniformly with respect to $q$, where $\delta_q$  depends only
upon $q$. Unconditionally, the work of Gupta and  Murty~\cite{GM}
and of Heath-Brown~\cite{HB} provides evidence for the Artin Conjecture.

Our question can be considered as a natural generalization
of Hooley's famous result. This generalization is meaningful
also if $q$ is a square.

For $r \in {\mathbb N}$, we let
$\zeta_r \in {\mathbb C} $
be a primitive $r$th root of unity.
We will prove the following results.

\begin{theorem} Let $q$ and $k$ be  integers with $|q|> 1$
and $ k>0$,   and  for $m\in
{\mathbb N}$
set $K_m={\mathbb Q}(\zeta_{km},q^{1/m})$  and
$n_{m}=\left[K_m:{\mathbb Q}\right]$,
and
$$\delta_{q,k}=\sum_{1\le m}\frac{\mu(m)}{n_m}.$$
Then there exists  $c_{q,k} \in {\mathbb R}$
that depends only on $q$ and $k$ such that
$$M_{q,k}(x)\leq\left(\delta_{q,k}+\frac{c_{q,k}}{\log\log x}\right)
\frac{x}{\log x}.$$
If the Generalized Riemann Hypothesis holds for all these fields $K_m$,
then
$$M_{q,k}(x)=\left(\delta_{q,k}+\operatorname{O}\left(\frac{\log\log x}{\log x}\right) %+\log \mid ak\mid
\right)\frac{x}{\log x}.$$
%uniformly with respect to $a$ and $k$.
\end{theorem}

Next we express the densities as Euler products.
The parameter $l$ in the products below ranges over the primes. We
let $$A = \displaystyle{\prod_{l \text{ prime}}} (1 - \frac{1}{l(l-1)})
\approx 0.373956$$
be Artin's constant, and $\mu$ the M\"obius function.

\begin{theorem} With  the notation of Theorem~1.1, we write
$q=b^h$  and $b=b^{2}_{1} b_{2}$ with integers
$b$, $b_{1}$, $b_{2}$, and $h$, where $b$ is not a perfect power and
 $b_2$ is squarefree,  set
$$b_3=\begin{cases} 4b_2/\gcd(4b_2,k) & \text{ if } b_2\equiv 2,3 \bmod 4,\\
b_2/\gcd(b_2,k) & \text{ if } b_2\equiv 1 \bmod 4, \end{cases} $$
write $ b_3=\alpha b_4$ with $\alpha$ a power of two and $b_4$
odd, so that the values of $\alpha$ are given by the following
table:
\begin{center}
\begin{tabular}{|c|c|c|c|c|}
\hline
                        & $2\nmid k$ & $2\| k$ & $4\| k$ & $8\mid k$ \\
\hline
 $b_2\equiv 1\bmod 4$ &    $1$     &   $1$   &   $1$   &    $1$    \\
\hline
 $b_2\equiv 3 \bmod 4$ &    $4$     &   $2$   &   $1$   &    $1$    \\
\hline
 $b_2\equiv 2 \bmod 4$ &    $8$     &   $4$   &   $2$   &    $1$  \\
\hline
\end{tabular}
.
\end{center}
Furthermore, we set
% $$A_{h,k}=\frac{A}{k}\prod_{\genfrac{}{}{0pt}{}{l\mid k}{l\nmid h}}
$$A_{h,k}=\frac{A}{k}\prod_{l \mid k}
\left(1+\frac{l}{l^2-l-1}\right)
\prod_{\genfrac{}{}{0pt}{}{l\mid h}{l\nmid k}}
\left(1-\frac{l-1}{l^2- l-1}\right).$$
Then we have
\begin{equation}\label{density}
\delta_{q,k}=A_{h,k}\cdot\left(1-\frac{\mu(b_4\cdot\gcd(h,2)^2)\cdot |\mu(\alpha)|}
{2 \gcd (2,k)-1}\prod_{\genfrac{}{}{0pt}{}{l\mid b_4}{l\nmid h}}\frac{1}{l^2-l-1}
\prod_{\genfrac{}{}{0pt}{}{l\mid b_4}{l\mid h}}\frac{1}{l-2}\right),
\end{equation}
and $A_{h,k}=0$ if and only if $h$ is even and $k$ is odd.
\end{theorem}

Finally we apply the above results to  the problem of Gau{\ss} pairs.

\begin{corollary}\label{integers}
Let $p$ be a prime, $h$ and $k$ be positive integers,
$q=p^{h}$, and assume that the GRH holds for all fields $K_{m}$  of
Theorem 1.1.
\begin{itemize}
\item [(i)] $\delta_{q,k}= 0$ if and only if at least one of the
following two conditions is satisfied:
  \begin{itemize}
  \item [(a)] $2 \mid h$ and $2\nmid k$,
  \item [(b)] $2\nmid k$, $p \mid k$, and $p\equiv 1\bmod 4$.
  \end{itemize}
\item [(ii)] If $\delta_{q,k}=0$, then there is no Gau{\ss} pair
$(n,k)$  over ${\mathbb F}_{q}$.
\end{itemize}
\end{corollary}

\begin{proof}
(i) We write (\ref{density}) as $\delta_{q,k}=A_{h,k}\cdot B$,
so that
$$
\delta_{q,k}=0\Longleftrightarrow A_{h,k}=0  {\text{ or }}
B=0\Longleftrightarrow (2 \mid h {\text{ and }} 2\nmid k) {\text{ or }}
B=0,
$$
using Theorem 1.2. Furthermore,
$$
B=0\Longleftrightarrow
{\mu(b_4) | \mu(\alpha)|}=
\left({2 \gcd (2,k)-1}\right)\prod_{\genfrac{}{}{0pt}{}{l\mid b_4}{l\nmid h}}
\left({l^2-l-1}\right)
\prod_{\genfrac{}{}{0pt}{}{l\mid b_4}{l\mid h}}\left({l-2}\right).
$$
The left-hand side has absolute value $1$, and the right hand side
is positive, since $b_{4}$ is odd. They are equal if and only if
both are equal to $1$. If that is the case, then $b_{4}=1$, since
otherwise it would have at least two distinct prime factors, by
$\mu(b_{4})=1$, and then one of the factors on the right hand side
would be greater than $1$. Since $|\mu(\alpha)|=1$ if and only if
$\alpha\le 2$, we have
\begin{eqnarray*}
B=0 & \Longleftrightarrow & \alpha\le 2, 2\nmid k, b_{4}=1\\
& \Longleftrightarrow & 2\nmid k,\alpha =1, b_{3}=b_{4}=1, b_{2}\equiv
1\bmod 4\\
& \Longleftrightarrow &  2\nmid k, p \mid k, p\equiv 1 \bmod 4,
\end{eqnarray*}
since $b_{2}=b=p$.

(ii) Since $\delta_{q,k} =0$,  either (a) or (b)
holds. From (a) we find that $\operatorname{ind}_r (q)$ and
$(r-1)/k$ are both even, so that $g_{q,k}(r)$ is even, for all
odd primes $r$, and thus there is no Gau{\ss} pair $(n,k)$
over ${\mathbb F}_{q}$. So now we assume that (b) holds, and let
$r$ be an odd prime with $r\equiv 1\bmod k$. Then $(r-1)/k$ is even.
Since $p$ divides $k$, we also have $r\equiv 1\bmod p$.
We may assume that $h$ is odd, since otherwise (a) holds.
Then the quadratic reciprocity law gives the following for the
Legendre symbol
$$
\biggl(\frac{q}{r}\biggr)=
\left(\frac{p^h}{r}\right)=\left(\frac{p}{r}\right)=\left(\frac{r}{p}\right)=\left(\frac{1}
{p}\right)=1.
$$
Thus $q$ is a square modulo $r$ and  $\operatorname{ind}_r (q)$
is even. Therefore again $g_{q,k}(r)$ is even, and there is no
Gau{\ss} pair, as claimed.
\end{proof}
In particular, for $q$ and $k$ as in Corollary \ref{integers}, the set of
primes $r$ for which $((r-1)/k,k)$ is a Gau{\ss} pair over ${\mathbb F}_{q}$
is either empty or has the positive density $\delta_{q,k}$.

Wassermann proves in
\cite{was93} an existence result starting from a different
set of parameters. His Theorem 3.3.4 states that for any given
integers $h$, $n$ and a prime $p$, there exists a Gau{\ss} pair
$(n,k)$ over ${\mathbb F}_{p^{h}}$ if and only if $\gcd(h,n)=1$
and
\begin{align*}
2p\nmid n \text{ if } p &  \equiv 1\bmod 4,\\
4p\nmid n \text{ if } p &  \equiv2,3\bmod 4.
\end{align*}


\section{Proof of the Theorems}

The following lemma is the Chebotarev Density Theorem. The proof of the two versions that we
state here is due to Lagarias and Odlyzko~\cite{L-O}.

\begin{lemma} \label{cdt}
Suppose that $L$ is a Galois extension of ${\mathbb Q}$ with absolute discriminant $d_L$ and degree
$n_L$ over ${\mathbb Q}$, and define
$$\pi(x,L\colon{\mathbb Q})=\#\{p\leq x \colon  p \text{ is unramified and
splits completely in } L\}.
$$
If the Generalized Riemann Hypothesis holds for the Dedekind zeta function of $L$, then
$$
\pi(x,L\colon{\mathbb Q})= \frac{1}{n_L}\operatorname{li}{(x)}+ \operatorname{O}(x^{1/2}\log (x\cdot d_L^{1/n_L})).
$$
In general (unconditionally)
there exists absolute constants $C_1$ and $B$ such that for
\begin{equation}
\sqrt{\log x}\geq C_1\ n_L^{1/2}\max\{\log\mid d_L\mid,
\mid d_L\mid^{1/n_L}\},\end{equation}
one has
$$\pi(x,L\colon{\mathbb Q})= \frac{1}{n_L}\operatorname{li}(x)+\operatorname{O}(x\exp({-B n_L^{-1/2}\sqrt{\log x}})).
$$\qed
\end{lemma}

\removelastskip\par\medskip
\noindent\emph{Proof of Theorem 1.1.}
The argument is similar to the original one of Hooley, therefore we  only
mention the main steps.

We start by noticing that the condition for a prime $l\not=
 p$ to divide the index
$\operatorname{ind}_p(q)$
is equivalent to $p$ splitting completely in
${\mathbb Q}(\zeta_l,q^{1/l})$, while the condition that
$l$ divides $(p-1)/k$
is equivalent to $p$ splitting completely in the cyclotomic field
${\mathbb Q}(\zeta_{lk})$. Since a prime splits completely in two
extensions if and only if it splits completely in the compositum,
by the inclusion--exclusion principle we gather that
$$M_{q,k}(x)=\sum_{1\le m}\mu(m)\pi(x,{\mathbb Q}(\zeta_{km},q^{1/m})
\colon{\mathbb Q}).$$

We now consider the set $S(y)$ of those squarefree
``$y$-smooth'' integers $m \ge 1$
all of whose prime divisors are less
than a (sufficiently small) parameter $y$. We note that
$S(y)$ has $2^{\pi(y)}$ elements, and if $m\in S(y)$, then $m\leq P(y)$,
where $P(y)$ denotes the product of the primes up to $y$.

Furthermore, we let $N$ and $D$ denote the degree and the discriminant of $K_m$ over $\mathbb Q$.
Then $\sqrt{N}\leq \sqrt{k} m\le \sqrt k P(y)$, $\log D\ll N\log N\ll y P(y)^2$,
and $D^{1/N}\ll N\prod_{l \mid D}l\ll P(y)^3 $,
where the implied constants depend on $a$ and $k$.
By choosing $y$ such that $P(y)=C_2(\log x)^{1/8}$ for some constant $C_2$,
we can use the unconditional part of Lemma~\ref{cdt}.
The inclusion--exclusion principle then yields
the (unconditional) upper bound
% \begin{eqnarray}\nonumber
% M_{a,k}(x)   \leq &\sum_{m \epsilon S (y)}
% \mu(m)\pi(x,{\mathbb Q}(\zeta_{km},a^{1/m});{\mathbb Q})\nonumber\\
%  = &\sum_{m \epsilon S (y)}\nonumber
% \mu(m)\left\{\frac{\operatorname{li}(x)}{n_m}+\operatorname{O}\left(x\exp( -C_3\sqrt{(\log x)/n_m})\right)\right\}\\
%    = & \left(\delta_{a,k}+ \operatorname{O}{\left(\displaystyle{\sum_{m>y}\frac{1}{m\varphi(m)}}\right)}
% \right)\operatorname{li}(x)+\operatorname{O}{\left(2^{\pi(y)}x\exp\left({-C_4\frac{\sqrt{\log x}}{P(y)}}\right)\right)}\nonumber\\
%  = & \left(\delta_{a,k}+ \operatorname{O}{\left(\frac{1}{y}\right)}
% \right)\frac{x}{\log x}+\operatorname{O}{\left(x\exp\left(-C_5(\log x)^{3/8}\right)\right)}\nonumber\\
%  =  &\left(\delta_{a,k}+ \operatorname{O}{\left(\frac{1}{\log\log x}\right)}
% \right)\frac{x}{\log x},
% \nonumber
% \end{eqnarray}
\begin{eqnarray}\nonumber
M_{q,k}(x)   & \leq &\sum_{m \epsilon S (y)}
\mu(m)\pi(x,{\mathbb Q}(\zeta_{km},q^{1/m})\colon{\mathbb Q})\nonumber\\
& = &\sum_{m \epsilon S (y)}\nonumber
\mu(m)\left\{\frac{\operatorname{li}(x)}{n_m}+\operatorname{O}\left(x\exp( -C_3\sqrt{(\log x)/n_m})\right)\right\}\\
&  = & \left(\delta_{q,k}+ \operatorname{O}{\left(\displaystyle{\sum_{m>y}\frac{1}{m\varphi(m)}}\right)}
\right)\operatorname{li}(x)+\operatorname{O}{\left(2^{\pi(y)}x\exp\left({-C_4\frac{\sqrt{\log x}}{P(y)}}\right)\right)}\nonumber\\
&  = & \left(\delta_{q,k}+ \operatorname{O}{\left(\frac{1}{y}\right)}
\right)\frac{x}{\log x}+\operatorname{O}{\left(x\exp\left(-C_5(\log x)^{3/8}\right)\right)}\nonumber\\
& =  &\left(\delta_{q,k}+ \operatorname{O}{\left(\frac{1}{\log\log x}\right)}
\right)\frac{x}{\log x},
\nonumber
\end{eqnarray}

\noindent where we used the fact that $\varphi(m)m\ll n_m$. This proves the second part of
 Theorem~1.1. We note that the
method of A.~I.~Vinogradov~\cite{V} could be used here to establish a sharper error term.

For the second claim we note that
\begin{eqnarray*}
M_{q,k}(x) & \leq &\sum_{m \epsilon S (y)}
\mu(m)\pi(x,{\mathbb Q}(\zeta_{km},q^{1/m})\colon{\mathbb Q})\\
 & \leq & M_{q,k}(x)+\#\left\{p\leq x\colon \exists l\geq y
\quad l \mid g_{q,k}\right\}.
\end{eqnarray*}
Therefore
$$ M_{q,k}(x) =\sum_{m \epsilon S (y)}
\mu(m)\pi(x,{\mathbb Q}(\zeta_{km},q^{1/m}):{\mathbb Q})
+\operatorname{O}\left(\#\left\{p\leq x\colon\exists l\geq y \quad l \mid g_{q,k}
\right\}\right).$$
The main term is estimated using the version of the Chebotarev Density Theorem in Lemma~\ref{cdt}
dependent on the Generalized Riemann Hypothesis which leads to a choice of $y=\frac{1}{6}\log x$.
%The $\log a k$ in the error term comes from the fact that all
%primes dividing $a k$ ramify in ${\mathbb Q}(\zeta_{km},a^{1/m});{\mathbb Q}$
%while
The error term can be handled exactly as in Hooley's case, ignoring the
condition that $l\mid (p-1)/k$.\penalty-20\null\hfill$\square$\par\medbreak

\removelastskip\par\medskip
\noindent For the proof of Theorem 1.2,
we need the following two lemmas. We will have an integer $h$, and
for an integer $m$ we set
$$
\hat{m}=m/ \gcd (h,m).
$$
\begin{lemma}\label{density1} Let $q,k,m\in{\mathbb Z}$ with $m,k>0, |q|>1$,
and $m$ squarefree. We write
$q=b^h$ with $b$ not a perfect power, $b=b_1^2b_2$ with $b_2$
squarefree, and set
$$\varepsilon=\begin{cases}
2 &  \text{if\ } 2\mid\hat{m}, b_2\mid m k,\text{ and }b_2\equiv 1 \bmod4, \\
2 &  \text{if\ } 2\mid\hat{m}, 4b_2\mid m k,\text{ and } b_2\not\equiv
 1 \bmod 4,\\
1 &  \text{otherwise.}
\end{cases}$$
Then $n_m=\varphi(km)\cdot\left[{\mathbb Q}(\zeta_{km},q^{1/m}):{\mathbb Q}
\right]=\varphi(km)\hat{m}/\varepsilon$.
\end{lemma}

\removelastskip\par\medskip
\noindent\emph{Proof.}
First we note that
${\mathbb Q}(\zeta_{km},q^{1/m})={\mathbb Q}(\zeta_{km},b^{1/\hat{m}})$. Since $[{\mathbb Q}(b^{1/\hat{m}}):\mathbb{Q}]=\hat{m}$ and
$[{\mathbb Q}(b^{1/\hat{m}})(\zeta_{km}):\mathbb{Q}(b^{1/\hat{m}})]$
is a divisor of $\varphi(km)$, from the identity
$$[{\mathbb Q}(\zeta_{km},b^{1/\hat{m}}):{\mathbb Q}(\zeta_{km})]\cdot[{\mathbb Q}(\zeta_{km}):{\mathbb Q}]= [{\mathbb Q}(b^{1/\hat{m}},\zeta_{km}):\mathbb{Q}(b^{1/\hat{m}})]\cdot [{\mathbb Q}(b^{1/\hat{m}}):\mathbb{Q}] $$
we deduce that
$$n_m=\varphi(km)\left[{\mathbb Q}(\zeta_{km},b^{1/\hat{m}}):{\mathbb
    Q}(\zeta_{km})\right]
=\varphi(km)\frac{\hat{m}
}{d}$$ for some divisor $d$ of $\hat{m}$. We claim that $d$ is $1$ or $2$.
Indeed, if $l$ is a prime dividing $d$, then we have extensions
$$
{\mathbb Q}(\zeta_{km})
\subseteq
{\mathbb Q}(\zeta_{km},b^{1/l})
\subseteq
{\mathbb Q}(\zeta_{km},b^{1/\hat{m}})
.$$
Since $\hat{m}$ is squarefree, $l$ does not divide $\hat{m}$,
 hence ${\mathbb Q}(\zeta_{km},b^{1/l})={\mathbb Q}(\zeta_{km})$ and
 $b^{1/l}\in{\mathbb Q}(\zeta_{km})$.
Therefore we have an inclusion of Abelian extensions
${\mathbb Q}(b^{1/l})\subseteq{\mathbb Q}(\zeta_{km})$ of ${\mathbb Q}$.
This can only happen when $l$ is $1$ or $2$.

Furthermore ${\mathbb Q}(\sqrt{b})={\mathbb Q}(\sqrt{b_2})$, so that $d=2$ if and
only if $\hat{m}$ is even and $\sqrt{b_2}\in{\mathbb Q}(\zeta_{km})$.

The quadratic subfields of ${\mathbb Q}(\zeta_{km})$ are
% $$\cases
$$
\begin{array}{ll}
% \left\{{\mathbb Q}(\sqrt{{\scriptstyle \left(\frac{-1}{D}\right)}|D|})
\left\{{\mathbb Q}(\sqrt{{\left(\frac{-1}{D}\right)}|D|})
\colon D \mid km, D \text{ odd squarefree}
\right\}
& \text{if }4\nmid km,\\
% \\
% \left\{{\mathbb Q}(\sqrt{\scriptstyle D})
\left\{{\mathbb Q}(\sqrt{D})
\colon D \mid km, D \text{ odd squarefree}
\right\}
 & \text{if }4\,\|\, km, \\
% \\
% \left\{{\mathbb Q}(\sqrt{\scriptstyle D})
\left\{{\mathbb Q}(\sqrt{D})
\colon D \mid km, D \text{ squarefree}
\right\}
 & \text{if }8\mid km.
\end{array}
$$
% \endcases$$

In the first case, $d=2$ if and only if $b_2|km$ and $b_2\equiv1 \bmod 4 $,
and  in the second
case, $d=2$ if and only if $b_2$ is odd and divides $km$,
and in the third case $d=2$ if and only if $b_2|km$.

Finally, $d=\varepsilon$ and hence the claim.\qed\par\medskip

\begin{lemma}\label{prud} Let $A_{h,k}$ be as in the statement of
Theorem 1.2 and $t\in {\mathbb N}$.
 Then
$$A_{h,k}=\sum_{1\le m}\frac{\mu(m)}{\varphi(km)\hat{m}}=\frac{1}{\varphi(k)}
\prod_{l\; \mathrm{prime}}
\left(1-\frac{\gcd (l,h)\varphi( \gcd (l,k))}{l \gcd (l,k)(l-1)}\right),$$
\begin{equation}\sum_{\substack{{1\le m}\\{\gcd(m,t)=1}}}\frac{\mu(m)}{\varphi(km)\hat{m}}=\frac{1}{\varphi(k)}\prod_{l\nmid t}
\left(1-\frac{\varphi(\gcd(l,k))}{(l-1)\hat{l} \gcd (l,k)}\right).\label{uso}
\end{equation}
\end{lemma}

\removelastskip\par\medskip
\noindent\emph{Proof.} We have
$$\sum_{1\le m}\frac{\mu(m)}{\varphi(km)\hat{m}}=\sum_{d\mid k}
\sum_{\genfrac{}{}{0pt}{}{1\le m}{\gcd (m,k)=d}}\frac{\mu(m)}{\varphi(km)\hat{m}}$$
\begin{equation}=\left(\sum_{\genfrac{}{}{0pt}{}{1\le m}{\gcd(m,k)=1}}\frac{\mu(m)}{\varphi(km)\hat{m}}
 \right)\cdot\left(\sum_{d\mid k}\frac{\mu(d)}{d\hat{d}}\right)
=\frac{1}{\varphi(k)}\prod_{l\nmid k}\left(1-\frac{1}{ \hat{l}(l-1)}\right)
\prod_{l\mid k}\left(1-\frac{1}{ \hat{l}l}\right),\nonumber\end{equation}
since if $d\mid k$, then $\varphi(k m d)=d\varphi(km)$, and the claim is easily deduced. The second part is proven similarly.\qed\par\medskip

Let us now prove Theorem 1.2.

If $h$ is even, then $\hat{m}$ is odd for any squarefree $m$,
 and this implies that $n_m=\varphi(km)\hat{m}$. Therefore
 by Lemma~2.3, we have that
$\delta_{a,k}=A_{h,k}.$ We now assume that $h$ is odd (so that
 $\hat{m}$ is even if and only if $m$ is), and
consider $b_{3}$, $b_{4}$, and $\alpha$ as in the theorem.
%$$b_3=\cases 4b_2/\gcd(4b_2,k) & if b_2\equiv 2,3 \bmod 4 \\
%b_2/\gcd(b_2,k) & if b_2\equiv 1 \bmod 4, \endcases
%\ \ \ \ \ \ b_3=2^\alpha b_4\text{ with $b_4$ odd}$$
We note that $\gcd(b_4,k)=1$. Furthermore, for any squarefree $m$,
$\varepsilon$ as defined in Lemma~\ref{density1} equals $2$ if and only if
$\alpha\leq 2$ and $2b_4|m$.

Therefore, if $\alpha\geq4$, then $\delta_{q,k}=A_{h,k}$.
If $\alpha\leq2$, then
$$\delta_{q,k}=\sum_{2b_4\nmid m}\frac{\mu(m)}{\varphi(km)\hat{m}}+
2\sum_{2b_4\mid m}\frac{\mu(m)}{\varphi(km)\hat{m}}
=A_{h,k}+\frac{\mu(2b_4)}{2\hat{b_4}\varphi(b_4)}\sum_{\gcd(m,2b_4)=1}\frac{\mu(m)}{\varphi(2km)\hat{m}}.$$
By applying the multiplicative property (\ref{uso}) to the
last sum above (with $t=2b_4$ and $2k$ instead of $k$), we have
$$\delta_{q,k}=A_{h,k}-\frac{\mu(b_4)}{2\hat{b_4}\varphi(b_4)\varphi(2k)}\prod_{l\nmid
  2b_4}\left(1-\frac{\varphi(\gcd(k,l))}
{(l-1)\hat{l} \gcd (l,k)}\right).$$
In the inner product we write  $\gcd(k,l)$ instead of $\gcd(2k,l)$, since $l$ is odd. Now,
we can factor out $A_{h,k}$ as follows. We multiply and divide the inner product by
 $\prod_{l\mid 2b_4}\left(1-\frac{\varphi((k,l))}{\hat{l} \gcd (l,k)(l-1)}\right)
,$
and  obtain:
$$
\begin{array}{rl}
\delta_{q,k}= & \displaystyle{
A_{h,k}-\frac{\mu(b_4)}{2\hat{b_4}\varphi(b_4)\varphi(2k)}
\prod_{l}\left(1-\frac{\varphi(\gcd(k,l))}{
\hat{l} \gcd (l,k)(l-1)}\right)}
\\
& \qquad \cdot \displaystyle
\prod_{l\mid 2b_4}{\left(1-\frac{\varphi(\gcd(k,l))}{\hat{l} \gcd (l,k)(l-1)}\right)^{-1}}
\\
=& \displaystyle{
A_{h,k}\left(1-\frac{\mu(b_4)}{2\hat{b_4}\varphi(b_4)}\frac{\varphi(k)}{\varphi(2k)}\prod_{l\mid 2b_4}
\left(\frac{\hat{l} \gcd (l,k)(l-1)}{
\hat{l} \gcd (l,k)(l-1)-\varphi( \gcd (k,l))}\right)\right)
}.
\end{array}
$$
It is easy to see that $\gcd(2,k)\varphi(k)=\varphi(2k)$ and $\hat{2}=2$.
 If $l\mid b_4$, then $\gcd(l,k)=1$, since $\gcd(b_4,k)=1$.
Therefore
$$
\begin{array}{rl}
\delta_{q,k}= & \displaystyle{
A_{h,k}\left(1-\frac{\mu(b_4)}{2\hat{b_4}\varphi(b_4)}\frac{\varphi(k)}{\varphi(2k)}
\prod_{l\mid 2}
\left(\frac{\hat{l} \gcd (l,k)(l-1)}{
\hat{l} \gcd (l,k)(l-1)-\varphi( \gcd (k,l))}\right)
\right.
}
\\
& \displaystyle {
\left. \qquad \cdot
\prod_{l\mid b_4}
\left(\frac{\hat{l}(l-1)}{
\hat{l}(l-1)- 1}\right)\right)
}
\\
= & \displaystyle{
A_{h,k}\left(1-\frac{\mu(b_4)}{2\hat{b_4}\varphi(b_4)}\frac{\varphi(k)}{\varphi(2k)}
\frac{\hat{2} \gcd (2,k)}{
\hat{2} \gcd (2,k)- 1}
\right.
}
\\
& \displaystyle {
\left. \qquad \cdot
\prod_{l\mid b_4}
\left({\hat{l}(l-1)}\right)
\prod_{l\mid b_4}\frac{1}
{\hat{l}(l-1)- 1}\right)
}
\\
= &\displaystyle{
A_{h,k}\left(1-\frac{\mu(b_4)}{2 \gcd (2,k)-1}\prod_{l\mid b_4}\frac{1}{\hat{l} (l-1)-1}\right).}
\end{array}$$

Finally we can combine the three cases $h$ even, $h$ odd and $\alpha \ge
4$, and $h$ odd and $\alpha \le 2$, in a single formula as
$$\delta_{a,k}=A_{h,k}\left(1-\frac{\mu(b_4\cdot\gcd(h,2)^2)|\mu(\alpha)|}
{2 \gcd (2,k)-1}\prod_{l\mid b_4}\frac{1}{\hat{l} (l-1)-1}\right).$$
\penalty-20\null\hfill$\square$\par\medbreak






\noindent\textbf{Acknowledgements} The authors would like to thank Hans Roskam for having pointed out
and corrected a mistake in Lemma 2.2 of the original version of the paper.

% \bibliographystyle{plain}
% \bibliography{journals,refs}
\begin{thebibliography}{99}
\itemsep=0.6pt%\tinyskipamount
\bibitem{feigat99}
S. Feisel, J. von~zur Gathen, and M.~A. Shokrollahi,
\emph{Normal bases via general {Gau\ss} periods},
Mathematics of Computation, \textbf{68}(225) (1999), 271--290.
\bibitem{gaogat95}
S. Gao, J. von~zur Gathen, and D. Panario,
\emph{Gauss periods and fast exponentiation in finite fields},
In {\em Proceedings of LATIN~'95, {\rm Valpara{\'{\i}}so, Chile}},
  number 911 in Lecture Notes in Computer Science, 311--322,
  Springer-Verlag, 1995.
\bibitem{gaogat98}
S. Gao, J. von~zur Gathen, and D. Panario,
\emph{Gauss periods: orders and cryptographical applications},
{Mathematics of Computation}, \textbf{67}(221) (1998), 343--352,
Microfiche supplement.
\bibitem{gaogat00}
S. Gao, J. von~zur Gathen, D. Panario, and V. Shoup,
\emph{Algorithms for exponentiation in finite fields},
{Journal of Symbolic Computation} \textbf{29}(6) (2000), 879--889.
\bibitem{gaolen92}
S.~Gao and H.~W. Lenstra, Jr.
\newblock Optimal normal bases.
\newblock {\em Designs, Codes, and Cryptography}, \textbf{2} (1992), 315--323.
\bibitem{gatnoe97}
J. von~zur Gathen and M. N{\"o}cker,
\emph{Exponentiation in finite fields: Theory and practice},
In Teo Mora and Harold Mattson, editors, {\em Applied Algebra,
  Algebraic Algorithms and Error-Correcting Codes: AAECC-12, {\rm Toulouse,
  France}}, number 1255 in Lecture Notes in Computer Science, 88--113,
  Springer-Verlag, 1997.
\bibitem{gatnoe99a}
J. von~zur Gathen and M. N{\"o}cker,
\emph{Computing special powers in finite fields: Extended abstract},
In Sam Dooley, editor, {\em Proceedings of the 1999 International
  Symposium on Symbolic and Algebraic Computation ISSAC~'99, {\rm Vancouver,
  Canada}},  83--90, ACM Press, 1999.
\bibitem{gatshp00a}
Joachim von~zur Gathen and Igor Shparlinski,
\emph{Gauss periods in finite fields},
In {\em Proceedings {Fq5}} (2000), Birkh\"auser Verlag, Basel.
To appear.
\bibitem{GM} {R.~Gupta and M.~R.~Murty},
\emph{A remark on {A}rtin's conjecture}, {Invent. Math.}, \textbf{78} (1),
(1984), {127--130}.
\bibitem{HB} {D.~ R.~Heath-Brown}, \emph{Artin's conjecture for primitive roots}, {Quart. J. Math. Oxford Ser. (2)}, \textbf{37}, (1986), {27--38}.
\bibitem{H} C. Hooley, \emph{On Artin's conjecture,} J. f\"ur Angew. und Reine Math.
\textbf{225}, (1967), 209--220.
\bibitem{L-O} J.~C.~Lagarias and A.~M.~Odlyzko,
\emph{Effective versions of the Chebotarev Density Theorem in algebraic number fields,}
in: \emph{Algebraic Number Theory},
Ed. A. Fr\"ohlich, Academic Press, New York  1977, 409--464.
\bibitem{V} A.~I.~Vinogradov, \emph{Artin $L$--series and his conjectures,}
Proc. Steklov Inst. Math. \textbf{112} (1971), 124--142.
\bibitem{was93}
Alfred Wassermann,
\emph{Zur {Arithmetik} in endlichen {K\"orpern}},
{\em Bayreuther Math. Schriften}, \textbf{44}, 147--251, 1993.
\end{thebibliography}
\end{document}
