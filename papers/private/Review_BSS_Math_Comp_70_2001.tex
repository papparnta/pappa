\documentclass[final]{mcom-l}
\newcommand{\Mod}{\operatorname{mod}}
\newcommand{\Gal}{\operatorname{Gal}}


\begin{document}
\issueinfo{70}{236}{October}{2001}
\pagespan{1755}{1759}
\dateposted{May 22, 2001}
\PII{S 0025-5718(01)01400-4}
\copyrightinfo{2001}{American Mathematical Society}

{\scriptsize
\noindent MATHEMATICS OF COMPUTATION\\
Volume 70, Number 236, Pages 1755-1759\\
S 0025-5718(01)01400-4\\
Article electronically published on May 22, 2001}\bigskip\bigskip

\centerline{\textbf{REVIEWS AND DESCRIPTIONS OF TABLES AND BOOKS}}\bigskip\bigskip

\begin{quote}\noindent \textbf{14[94-02, 94A60, 14H52]} \emph{Elliptic
curves in cryptography} by {Ian Blake, Gadiel Seroussi, and Nigel
Smart}, {Cambridge University Press}, {New York, NY}, {1999},
{xv+204} pp.,  {23} cm. {softcover} \$39.95\end{quote}\bigskip

Elliptic curves have been studied for more than a century from the
perspectives of modular forms, complex analysis, algebraic geometry, and number
theory.  Schoof's discovery \cite[1984]{10}, that there is a polynomial time
algorithm for establishing the size of the elliptic curve group over any finite
field opened the way to various computational applications of these groups.

One by one, most applications which were customary in the multiplicative group
of finite fields were adapted to the elliptic curve group.  In the space of a
few years, elliptic curves emerged in primality proving \cite{5}, integer
factoring \cite{6}, and cryptography \cite{8}.  The first two applications take
advantage of the large variety of available groups of the chosen order of
magnitude, while the interest of the latter is based on the fact that in
general no subexponential algorithm for computing the discrete logarithm in the
elliptic curve group is known or likely to be found.  Such algorithms had been
known for some time in the multiplicative groups.  However, many practical
questions were still asking for improvements and clarity, so the last 15 years
have seen intense research in this domain of applications.

The book at hand is a welcome, in-depth treatment of the various research and
improvements up to 1999.  It offers a comprehensive presentation of both the
deeper theoretical background of algorithmic developments and the
implementational bottlenecks.  Whoever wants to deal with algorithmal aspects
of elliptic curves will find here an excellent  and, in most cases, sufficient
starting point.  The book is therefore not intended either as primary
didactical material (proofs are scarcely given) nor as
a compendium of the various
short or long lived cryprographical mechanisms related to elliptic curves.  For
the first, books such as \cite{7} for the practical and \cite{3} for the
mathematical aspects, are recommendable.  For the latter, the technical IEEE
standard P1363, which has been meanwhile released, is the relevant source for
those mechanisms which are likely to be spread in practice.

The first two chapters offer a succinct introduction to general ideas of public
key cryptography and the underlying arithmetic in finite fields.  The
important third chapter introduces the arithmetic of elliptic curves together
with the various connections to division polynomials, Weil pairing, and modular
functions, which have found explicit use in applications.  The fourth chapter
gives an overview of efficient implementations of elliptic curves arithmetic
for the practitioner, and the fifth treats the discrete logarithm problem on
elliptic curves.  From the sixth to the eighth chapters the authors discuss the
determination of the group order, by Schoof's algorithm and its later
improvements and by an a priori choice of the complex multiplication fields of
the target curve.  The book closes with an overview of primality proving and
integer factoring using elliptic curves in the ninth chapter, and with
generalizations of cryptosystems to abelian varieties of higher genus in the
tenth chapter.

Let $\mathbb{F}_q$ be a finite field with $q$ elements.  An elliptic curve $E$
over $\mathbb{F}_q$ is defined by a \emph{long Weierstrass equation}
\[E:y^2+a_1xy+a_3y=x^3+a_2x^2+a_4x+a_6,\]
where $a_1,a_2,a_3,a_4,a_6\in \mathbb{F}_q$ are such that the equation is
nonsingular.  This is equivalent to saying that the \emph{discriminant}
\[\Delta=-b^2_2b_8-8b^3_4-27b^2_6+90b_2b_4b_6\ne 0,\]
where $b_2=a^2_1+4a_2, b_4=a_1a_3+2a_4, b_6=a^2_3+4a_6,b_8=a^2_1a_6+4a_2a_6-
a_1a_3a_4+a_2a^2_3-a^2_4$.  From the definition of the discriminant, it follows
that it has special behavior in fields of characteristic 2 or 3.  In fact, most
applications in finite fields are treated differently in characteristic $p\le
3$ and $p>3$.  In the book the clear choice to deal only with the cases of
characteristics 2 and prime fields $\mathbb{F}_p$ with $p>3$ was made.  It is
customary to indicate by $E(\mathbb{F}_q)$ the set of points on $\mathbb{F}^2_q$
of the equation defining $E$ together with an extra point $\mathcal{O}$ at
infinity, which one may think of as lying on the top of the $y$-axis.  For two
points, $P_1=(x_1,y_1), P_2(x_2,y_2)\in E(\mathbb{F}_q)$, the sum is $P_1\oplus
P_2=(x_3,y_3)=(\lambda^2+a_1\lambda-a_2-x_1-x_2,-(\lambda+a_1)x_3-\mu-a_3)$,
where
\[(\lambda,\mu)=\begin{cases}
\left(\frac {y_2-y_1}{x_2-x_1}, \frac{y_1x_2-y_2x_1}{x_2-x_1}\right)
&\text{if } x_1\ne x_2,\\
\left(\frac{3x^2_1+2a_2x_1+a_4-a_1y_1} {2y_1+a_2x_1+a_3},
\frac{-x^3_1+a_4x_1+2a_6-a_3y_1}{2y_1+a_2x_1+a_3}\right) &\text{if }x_1=x_2.
\end{cases}\]
This composition rule makes $(E(\mathbb{F}_q),\oplus)$ into a commutative group
with $\mathcal{O}$ as its neutral element and $E(\mathbb{F}_q)\cong
(\mathbb{Z}/d_1\mathbb{Z})\times (\mathbb{Z}/d_2\mathbb{Z})$, where $d_1$
divides both $d_2$ and $q-1$.  The $n$-fold addition of $P$ to itself is $[n]P$
and $E[n]\cong (\mathbb{Z}/(n\cdot \mathbb{Z}))^2$ is the $n$-torsion group of
$E$ over $\overline{\mathbb{F}}_q$.  The $x$-coordinates of the $n$ torsion
points are zeroes of the division polynomials $\Psi_n$.  These notions,
together with the more subtle connections to modular functions and complex
multiplication, which we shall not describe here, are introduced in the third
chapter, yielding a self-sufficient base for the understanding of all
algorithms treated subsequently.  The fourth chapter is an extensive overview
of the main practical aspects which the implementer will encounter, from
arithmetic tricks to point compression---a technical term (in cryptography) for
the idea that a point on the curve carries essentially the information of its
$x$ coordinate plus one bit allowing to distinguish a solution of a quadratic
equation.

The problem of taking the discrete logarithm on elliptic curves is the door to
cryptographic applications of these groups.  The known general techniques are
treated comprehensively in chapter five.  In the following special cases,
described in this chapter, more performant algorithms than the generic ones are
possible: First, the supersingular curves for which the Weil pairing yields an
isomorphism to the roots of unity of the ground field or a small extension
thereof, where subexponential index calculus methods can be applied.  Second,
the recent algorithm of Smart for computing the discrete logarithms on curves
with the number of points equal to the (prime) characteristic of the field over
which they are defined.

 Unsurprisingly, the problem of determining the number of points in the
elliptic curve group, which brought curves into computational algebra, is
covered in three extensive chapters.  The first short one gives an overview of
naive approaches and problems related to subgroups of the elliptic curve group.

The sixth and seventh chapters cover the generic algorithm of Schoof and its
ulterior improvements and adaptations to \emph{tricky} characteristics (i.e.,
$p=2,3$, with special behavior of the discriminant).  The basic algorithm of
Schoof for computing $\#E(\mathbb{F}_q)$ (that we briefly outline being the
first step in the journey of which the book is telling the story) is based on
the fact that from Hasse's Theorem, namely $\#E(\mathbb{F}_q)=q+1-t$ with
$|t|\le 2\sqrt q$, it is enough to determine $t$ modulo $l$ for sufficiently
many small primes $l$.  More precisely, it suffices to take primes $l\le
l_{\max}$ with $\prod_{l\le l_{\max}} l>4\sqrt q$.  One uses the fact that the
Frobenius endomorphism $\phi$ satisfies the equation $\phi^2-t\phi+q=0$.
Considering a nontrivial point $P=(x,y)\in E[l]$, one lets $q_l=q\Mod l$.  Then
$(x^{q^2},y^{q^2})+[q_l](x,y)=[\tau](x^q,y^q)$ must be satisfied exactly for
$\tau=t\Mod l$.  The value of $\tau$ can be found by computing symbolically
$(x^{q^2},y^{q^2})+[q_l](x,y)$ modulo the $l$-division polynomials $\Psi_l$ and
comparing with the possible values of $[\tau](x^q,y^q)$, $(\tau=1,\dotsc,l)$.
The computations are polynomially bounded by $O(\log q)$ and the degree
$\frac{l^2-1}{2}$ of the $l$-division polynomials.  This degree grows in
practice quite fast, which makes the arithmetic modulo division polynomials the
essential bottleneck in the original version of Schoof's algorithm.

However, the division polynomials are in general not irreducible and one can
sensibly reduce the complexity by replacing $\Psi_l$ by some smaller---not
necessarily irreducible---factors.  Since $E[l]\cong (\mathbb{Z}/(l\cdot
\mathbb{Z}))^2$, there is a representation of
$\Gal(\overline{\mathbb{F}}_q/\mathbb{F}_q)$ in $GL_2(\mathbb{F}_l)$, which
yields information on the factorization  patterns of $\Psi_l$.  This fact was
basically exploited in the subsequent contributions due to Atkin, Elkies and
worked out and implemented by F.~Morain, some of his students, and V.~M\"uller.

The applications of elliptic curves to factoring and primality proving are
briefly outlined, for the sake of completeness, in the ninth chapter.  The last
chapter summarizes the main ideas about cryptographic use of hyperelliptic
curves at the time the was book printed.

For the interested reader, it may be important to mention some of the
outstanding results of the last few years, which are ulterior to the conception
of this book and thus not covered by it.

Counting points on curves over fields of small characteristics $p$ have been
sensibly simplified by an algorithm of Satoh \cite{9}, based on $p$-adic logarithms.
The algorithm has been implemented, and curves over the field
$\mathbb{F}_{2^{8009}}$ can be currently treated; without Satoh's approach, the
best methods could calculate the curve orders in extensions of $\mathbb{F}_2$
of degree up to 2000.

In the domain of implementation, a beautiful paper of H.~Cohen, A.~Miyaji, and
T.~Ono \cite{2} studies a variety of curve representations, with the aim of
optimizing the performance of group operations; this can certainly be of great
help for implementors.  Fields of odd characteristic which are  adapted to machine
world length---\emph{medium Galois fields} or simply \emph{extension fields},
according to different terminologies---receive some attention.  A run time
study, \cite{12} by Smart, of implementations of elliptic curve operations over
fields of characteristics of various sizes suggests that these fields may have
interesting practical properties.

Finally, \emph{Weil descents} have been proposed by G.~Frey \cite{4} as a
possible method for solving special instances of the discrete logarithms
problem.  This has already motivated a series of important research with pro
and con arguments, and is likely to become an important research topic.

\bibliographystyle{amsplain}
\begin{thebibliography}{10}

\bibitem{1} Blake, I. F.; Seroussi, G.; Smart, N. P.: \emph{Elliptic curves in
cryptography}. Reprint of the 1999 original. London Mathematical
Society Lecture Note Series, 265. Cambridge University Press,
Cambridge, 2000. CMP {2000:15}

\bibitem{2} Cohen, H.; Miyaji, A.; Ono, T.: \emph{Efficient elliptic curve
exponentiation using mixed coordinates}, Asiacrypt 98, Lecture
Notes in Comput. Sci., 1514, Springer, Berlin, 1998. CMP {2000:06}

\bibitem{3} Cox, D. A.: \emph{Primes of the form $x^2+ny^2$}, Wiley \& Sons,
1989. \MR{90m:11016}

\bibitem{4} Frey, G.: \emph{Applications of arithmetical geometry to
cryptographic constructions}, Preprint.

\bibitem{5} Goldwasser, S.; Killian, J.: \emph{Almost all primes can be quickly
certified}, Proc. 18-th Annual ACM Symp. on Theory of Computing (1986),
316--329.

\bibitem{6} Lenstra, H. W.: \emph{Factoring integers with elliptic curves},
Ann. of Math., \textbf{126} (1987), 649--673. \MR{89g:11125}

\bibitem{7} Menezes, Alfred J.: \emph{Elliptic curve public key cryptosystems},
Kluwer Academic Publishers, 1993. \MR{2000d:94023}

\bibitem{8} Miller, V.: \emph{Use of elliptic curves in cryptography}, Advances
in Cryptology, Proceedings of CRYPTO'85, Lecture Notes in
Comput. Sci. \textbf{218}, Springer, Berlin, 1986, pp. 417--426. \MR{88b:68040}

\bibitem{9} Satoh, T.: \emph{The canonical lift of an ordinary elliptic curve
over a finite field and its point counting}, J. Ramanujan Math.
Soc. \textbf{15} (2000), no. 4, 247--270. CMP {2001:05}

\bibitem{10} Schoof, R.: \emph{Elliptic curves over finite fields and the
computation of square roots mod p}, Math. Comp. \textbf{44},
(1985), 483--494. \MR{86e:11122}

\bibitem{11} Silverman, J. H.: \emph{The arithmetic of elliptic curves}.
Corrected reprint of the 1986 original.  Graduate Texts in Mathematics, 106.
Springer-Verlag, New York, 1999. \MR{95m:11054}

\bibitem{12} Smart, N.: \emph{A comparison of different finite fields for use
in elliptic curve cryptosystems}, University of Bristol, Department of Computer
Science, June 2000 preprint.

\end{thebibliography}

\address{\noindent Preda Mih\u{a}ilescu,\newline $M_EC$ Consulting and Gesamthochschule Paderborn,
\newline Germany}
\email{preda@math.upb.de}
\address{F. Pappalardi\newline Dipartimento di Matematica\newline Universit\`a degli studi Roma Tre\newline Largo
S. L. Murialdo 1\newline I-00146 Roma\newline Italy}
\email{pappa@mat.uniroma3.it}
\end{document}
