\documentclass{elsart}

\usepackage{amsmath}

\journal{Theoretical Computer Science}

\newcommand{\map}{\Psi}
\newcommand{\thue}{\mathbf{t}}
\newcommand{\set}[2]{\left \{\, #1 \,\colon \, #2 \, \right \}}
\newcommand{\congmod }[1]{\!\!\pmod #1}
\DeclareMathOperator{\Part}{Part}
\DeclareMathOperator{\orb}{Orb}
\DeclareMathOperator{\fix}{Fix}
\DeclareMathOperator{\ord}{ord}
%\newtheorem{theorem}{Theorem}
%\newtheorem{lemma}[theorem]{Lemma}
%\newtheorem{corollary}[theorem]{Corollary}
%\newtheorem{prop}[theorem]{prop}
%\newtheorem{conjecture}{Conjecture}



%\theoremstyle{remark}
\newtheorem{remark}{Remark}

\newcommand{\N}{{\bf N}}
\newcommand{\Z}{{\bf Z}}



%\def\T(#1){\ifmmode{\cal T}_{#1}\else${\cal T}_{#1}$\fi}
%\def\alue(#1,#2){\ifmmode\{#1,\ldots,#2\}\else$\{#1,\ldots,#2\}$\fi}

\def\laa<#1>{\ifmmode \langle #1\rangle \else$\langle#1\rangle$\fi}

%\def\laa<#1>{\ifmmode\mathrm{<}#1\mathrm{>}\else$\mathrm{<}#1\mathrm{>}$\fi}
%\def\perm(#1,#2){{\ifmmode{{\begin{pmatrix}#1 \cr #2\end{pmatrix}}}%
% \else${{\begin{pmatrix}#1 \cr #2\end{pmatrix}}}$\fi}}
%\def\eritaso#1#2{\raise2pt\hbox{$#1$}\big/\lower2pt\hbox{$#2$}}
\def\eritaso#1#2{\raise1pt\hbox{$#1$}\big/\!\lower2pt\hbox{$#2$}}

\begin{document}



\begin{frontmatter}


\title{Transposition Invariant Words}

\author[arto]{Arto Lepist\"o}
\ead{alepisto@utu.fi}
\author[francesco]{Francesco Pappalardi}
\ead{pappa@mat.uniroma3.it}
\author[kalle]{Kalle Saari}
\thanks{Supported by the Academy of Finland under grant 203354.}
\ead{kasaar@utu.fi}

\address[arto]{Department of Mathematics \\University of Turku \\20014 Turku, Finland}

\address[francesco]{Dipartimento di Matematica \\ Universita` degli studi Roma TRE \\ Largo S. L. Murialdo, 1 \\ I-00146, Rome - Italy}

\address[kalle]{Department of Mathematics and Turku Centre for Computer Science\\University of Turku \\20014 Turku, Finland}

\begin{abstract}
We define an operation called transposition on words of fixed length called.
This operation arises naturally when the letters of
a word are considered as entries of a matrix. Words that are
invariant with respect to transposition are of special interest.
It turns out that transposition invariant words have
a simple interpretation by means of elementary group theory.
This leads us to investigate some properties of the ring of integers modulo $n$ and primitive roots.
In particular, we show that there are infinitely many prime numbers $p$ with a primitive root dividing $p+1$ and
infinitely many prime numbers $p$ without a primitive root dividing $p+1$.
We also consider the orbit of a word under transposition.

\end{abstract}
\begin{keyword}
transposition invariant words, partition generated by a
 subgroup, primitive roots, favorable prime numbers.
 \MSC 68R15 \sep \MSC 10A10
\end{keyword}
\end{frontmatter}

\section{Introduction}
Let us consider a word $w$ of length $pq$ where $p$ and $q$ are positive integers. Construct a $p\times q$ matrix $A$ by filling the entries with
consecutive letters of $w$ row by row, and then transpose it. The entries of $A^T$ read row by row corresponds to another word, which we will denote by
 $w^T$. The process of transposing a word to obtain a new word is the inspiration of this paper. If $w$ is invariant with respect to the transposition, we express this by saying that $w$ is $p\times q$--invariant. Words that are $p\times q$--invariant
for every appropriate $p$ and $q$ are called \emph{transposition invariant}. These words are the main topic and inspiration of this paper.

In Section~2 we introduce the formal definitions and give some examples. Section~3 contains some introductory results, such as the fact that the power-of-2 length prefixes of the infinite Thue-Morse word are transposition invariant. We also show that any word with at least two different letters can be extended periodically so that the resulting word is nontrivially  transposition invariant. In Section~4 we
switch to a more number theoretic aspect of the topic and give a characterization of transposition invariant words. Section~5 considers two questions:
How many distinct letters can a transposition invariant word have?
How long an orbit a word travels when we iterate the transposition operation with
respect to a  fixed $p\times q$ matrix? The last part of this paper, Section~6, considers primitive roots modulo a prime number. Namely, the characterization of transposition
invariant words gives rise to a classification of prime numbers into two disjoint sets that we call favorable and unfavorable prime numbers. We show that both sets are infinite.


\section{Definitions}

Let  $ w=w_0 w_1 \cdots w_n$ be a word, that is, a string of symbols over some alphabet~$\Sigma$. The length of $w$ is denoted by $|w|$, so that $|w|=n+1$.
 Assume then that
$|w|=n+1=pq$ for some integers
$p,q>0$, and consider the $p \times q$-matrix
$$
A= \left(
\begin{array}{cccc}
w_0 & w_1 & \cdots & w_{q-1} \\
w_q & w_{q+1} & \cdots & w_{2q-1} \\
\vdots & \vdots & \ddots & \vdots \\
w_{(p-1)q} &  w_{(p-1)q+1} & \cdots & w_{pq-1}
\end{array}
\right).
$$
By reading the entries of this matrix row by row starting from the upper
left corner, we obtain the word $w$. When reading the entries column by
column, we get another word
$$
w^T=w_0 w_q  \cdots w_{(p-1)q} \,  w_1 w_{q+1} \cdots w_{(p-1)q+1} \, \cdots
w_{q-1} w_{2q-1} \cdots w_{pq-1}\,.
$$
Equivalently, we obtain $w^T$ by reading the entries in the transpose matrix $A^T$ row by
row.

If $w^T =w $, we say that $w$ is {\em $p\times q$--invariant}. The word $w$ is
{\em transposition invariant} if it is
$p\times q$--invariant for all integers $p,q>0$ such that $pq= |w|$.
If the subword $w_1w_2 \cdots w_{n-1}$ of $w$ is unary or if $|w|$ is a prime number, then $w$ is
trivially transposition invariant. In the former case, we say that $w$ is {\em trivial}.

The Finnish word {\em m\"oh\"omahat}---the people with a
big belly---is $3\times 3$--invariant word. Examples of the same length in English are {\em Malayalam}
and {\em votometer}. Note that, as $9$ has only one proper factorization, $3\cdot 3$, these words are transposition invariant.
A more complicated instance of transposition invariant words in natural language is given by the Latin sentence below.
\begin{center}
\begin{tabular}[b]{lllll}
S & A & T & O & R \\
A & R & E & P & O \\
T & E & N & E & T \\
O & P & E & R & A \\
R & O & T & A & S
\end{tabular}
\end{center}
This translates roughly as ``Seed man
Arepo holds wheels in his work.'' Our last example comes from classic cryptography. Namely, transposition invariant words are the messages that cannot be encrypted  using the rail fence cipher without padding.

Sometimes it is more convenient to write $w=w(0)w(1)\cdots w(n)$ instead of
$w=w_0 w_1 \cdots w_n$; we will use both notations.
We conclude this section with the following lemma, which could have been taken as a formal definition of $p\times q$-invariance. The proof is immediate, so we omit it.

\begin{lem}\label{def}
The word $w$ is $p\times q$-invariant if and only if
\begin{equation}\label{ehto}
w(ip+j)=w(jq+i)
\end{equation}
for all $0\leq i < q$ and $ 0 \leq j < p$.
\end{lem}


\section{Motivating Results}

First we show that any prefix of length $2^k$ of the celebrated Thue-Morse word (see~\cite{AS} or~\cite{Lot}) is transposition invariant. The Thue-Morse word, denoted by $\thue$, is defined as the limit of an infinite iteration of the morphism $\mu: 0 \mapsto 01$, $1\mapsto 10$
on the letter 0. Thus
$$
\thue = \lim_{n\rightarrow \infty}\mu^n(0)= 01101001100101101001\cdots\,.
$$

\begin{prop}
For all integers $k\geq 1$, the prefix of length $2^k$ of the Thue-Morse
word  is transposition invariant.
\end{prop}
\begin{pf}
Let us denote $\thue=t(0) t(1) t(2) \cdots$. It can be proved that $t(i)$
is the number of occurrences of the letter~1 in the binary expansion of $i$  modulo~2  (see~\cite{Lot}). Using this
property it is easy to see that, for all integers $e,i \geq 0$ and $ 0 \leq j < 2^e$, we have
\[
t(i 2^e + j) = t(i)+t(j) \mbox{\,mod\,} 2.
\]
Here---as well as later in this paper---\mbox{``$\bmod\ 2\,$''} denotes the operation of taking the least nonnegative integer modulo~2;
when dealing with congruences, we use \mbox{``$\congmod 2$''} instead.

Let $u$ be the prefix of $\thue$ of length $2^k$. We have to show that $u$ is
$p\times q$-invariant whenever $pq=|u|=2^k$, that is,  $p=2^e$ and $
q=2^f$ with $e+f=k$. To do this, assume $0 \leq i <  q=2^f$ and $0 \leq j <  p=2^e$. Then
$$
u(ip+j)=t(i 2^e+j) = t(i)+t(j) \mbox{\,mod\,} 2 =  t(j 2^f+i)=u(jq+i)\,.
$$
By Lemma~\ref{def}, the word $u$ is $p\times q$-invariant. Since this
is true for all appropriate $p$ and $q$, the word $u$ is  transposition invariant.
\end{pf}


As a simple corollary we get the following result, which could easily be proved directly, as well.

\begin{cor}
There exists infinitely many nontrivial transposition invariant words of composite length.
\end{cor}

Note that the requirement that  the length be a composite number is essential since the statement is trivial for prime number lengths.

Next we will show that any non-unary word can be extended to a transposition invariant word of composite length. Again, we want the length to be composite: Otherwise the problem is trivial.

Suppose $\alpha=a/b$ is a rational number in its lowest terms. Suppose furthermore that $w$ is a word such that $b$ divides $|w|$.
Then $w^\alpha$ denotes the word $w^k w^{\prime}$, where $k=\lfloor \alpha \rfloor$,  $w^k = w w \cdots w$
($k$ times), and $w^\prime$ is a prefix of $w$ such that $|w^k w^{\prime}| = \alpha |w|$. A word $u$ is a \emph{prefix} of $w$ if $w=uv$ for some word $v$.


\begin{prop}
For any non-unary word $w$, there exists a rational number
$\alpha \geq 1$ such that $w^\alpha$ is  a nontrivial
transposition invariant word of composite length.
\end{prop}
\begin{pf}
Assume $|w|=m$. Choose  two positive integers $k$ and $p$ such that $p$ is prime and $km+1=p$. There exists such integers by the Dirichlet's theorem on
primes in arithmetic progressions (see \cite{Apo}). In fact, we can choose $p=O(m^{5.5})$ by a result of Heath-Brown~\cite{HB}.

Now set $\alpha = p^2/m$, so that $|w^\alpha|=\alpha |w|=p^2$. We
show that $w^\alpha$ is $p \times p$-invariant. To do so, assume
$0
\leq i,j < p$. Using the fact that $w^\alpha$ has a period $m$, we
get
$$
w^\alpha(ip+j)=w^\alpha\left(ikm+i+j\right)= w ^\alpha(i+j) =
w^\alpha(jkm+j+i)= w^\alpha(jp+i)\,.
$$
By Lemma~\ref{def},  $w^\alpha$ is $p \times p$-invariant and, moreover,
transposition invariant since $p$ is prime. The word $w^{\alpha}$ is nontrivial because $w$ is not unary and $\alpha\geq 2$.
Finally, the proof is completed by observing that $|w^\alpha|$ is a composite integer.
\end{pf}




\section{A Characterization of Transposition Invariant Words}

In this section we prove a number theoretic criterion for $p\times
q$-invariance, which then allows us to give a
characterization of transposition invariant words.  But let us
first fix some further notation for this and forthcoming sections.

We
denote $w=w_0w_1 \cdots w_n$ and $|w|=n+1=pq$, where $p,q\geq 1$
are integers. We will be working in $\Z_n$, the ring of integers
modulo $n$.
Integers  $0,1,\ldots,n-1$ are considered both  as positions of letters in $w$ and as elements
of $\Z_n$. For a subset $S \subseteq \Z^*_n$, where $ \Z^*_n$
denotes the unit group of $\Z_n$, $\laa<S>$ denotes the multiplicative subgroup of
$\Z_n^*$ generated by $S$. If $S=\{p\}$, we will leave the braces
out and simply write $\laa<p>$. Note that $\laa<p>=\laa<q>$ because
$p=q^{-1}$ in $\Z_n^*$. Note also that the last position of $w$,
namely $n$, is not included in $\Z_n$, and thus will be left out
from our considerations. However, it is not a problem since $w_n$
maps to itself in transposition. We define
\[
k\laa<p> = \set{k a}{ a \in \laa<p>}
\]
for all $k \in \Z_n$. Note that $k\laa<p>$ is  a generalization of the usual definition, see, e.g., the left column of the table on page \pageref{taulukko} for $n=45$.

\begin{prop} \label{pq-tapaus}
The word $w$ is $p\times q$-invariant if and only if
\begin{equation}\label{ehto''}
w(h)=w(k)
\end{equation}
for all $k\in \Z_n$ and $ h \in k\laa<p>$.
\end{prop}
\begin{pf}
We only need to prove the implication
\begin{equation}\label{2006}
k = jq + i \qquad \Longrightarrow \qquad ip + j = kp \mbox{\,mod\,} n
\end{equation}
whenever $0\leq i < q$, $0 \leq j < p$, and $0\leq k < n$. For if \eqref{2006} holds true, then by applying it repeatedly
we easily obtain an equivalence
between \eqref{ehto''} and Lemma~\ref{def}. For future reference, observe that \eqref{2006} tells us that the letter at the position $k$ is mapped to the position $pk \mbox{\,mod\,}  n$ in transposition.

To prove the implication in~\eqref{2006}, we note that $pq \equiv 1 \pmod{n}$, and hence
\[
k = jq + i \quad \Longrightarrow \quad k \equiv jq + i \pmod{n} \quad \Longrightarrow \quad kp \equiv j + ip \pmod{n}.
\]
Since we have $k < n$, it follows that $j + ip <n$, and so the last congruence gives $ip + j = kp \mbox{\,mod\,} n$.
This completes the proof.
\end{pf}

Now we are ready to establish a number theoretic characterization
for transposition invariant words. With the same trouble we can prove a somewhat
more general result that goes as follows.

Let $S_n$ be the set of all positive divisors of $n+1$, that is,
\[
S_n= \set{ d \geq 1 }{ d \,|\, (n+1) }.
\]
Let $S \subseteq S_n$. We say that the
word $w$ is {\em $S$-invariant} if it is
$p \times (n+1)/p\,$-invariant for all $p\in S$.
Then the concepts  $p\times q$-invariant and transposition invariant coincide with
$\{p\}$-invariant  and $S_n$-invariant, respectively.


\begin{thm}\label{main}
Let $S \subseteq S_n$. Then the word $w$ is
$S$-invariant if and
only if, for every $k \in \Z_n$, all letters at positions indicated by the set
$k\laa<S>$ are the same.
\end{thm}
\begin{pf}
Suppose  $w$ is $S$-invariant, that is, $w$ is $p \times (n+1)/p$-invariant
for every $p \in S$. Let  $r,s \in S$. Using the condition
(\ref{ehto''}), we see that $kr^{i} s^{j} \in k r^{i}\laa<s>$ implies
$w(kr^{i}s^{j})= w(kr^{i})$, and moreover,  $k r^{i} \in k\laa<r>$
implies $w(kr^{i})=w(k)$. Thus, for all elements $h \in k \laa<r>\laa<s>= k
\laa<r,s>$, we have $w(h)=w(k)$. Using this argument repeatedly, we see that, for every $h \in
k\laa<S>$, $w(h)=w(k)$.

Conversely, assume that $w(h)=w(k)$ for every $h \in k\laa<S>$. Then, because
$\laa<p> \subseteq \laa<S>$ for all $p \in S$, it certainly holds that
$w(h)=w(k)$ for every $h \in k\laa<p>$. According to Proposition~\ref{pq-tapaus}, the word $w$
is $p\times (n+1)/p$-invariant for every $p \in S$, that is,
$S$-invariant.
\end{pf}


\section{Maximum Number of Letters  and the Orbit of a Word}

It is natural to ask how many distinct letters can a transposition invariant word have. To answer this question, we need the following auxiliary observation which can be proved in a standard manner.

\begin{lem}
Let $S \subseteq \Z_n^*$ and $k,h \in \Z_n$. Then  either
$$ k\laa<S> = h \laa<S> \qquad\textrm{or}\qquad k\laa<S> \cap h\laa<S> = \emptyset
\,.$$
\end{lem}






Hence every subset $S\subseteq \Z_n^*$ induces a
partition of $\Z_n$ by means of the subgroup~$\laa<S>$.
More precisely, there
exist integers $k_1,k_2,\ldots,k_r \in \Z_n$ such that
$$
\Z_n= \bigcup_{1\leq i \leq r} k_i\laa<S>
$$
and the sets $k_i\laa<S>$ and $k_j\laa<S>$
are disjoint if $i \neq j$. We  denote this partition generated
by the set $S$  by $\Part_n(S)$, that is,
$$
\Part_n(S) = \left \{ k_1 \laa<S>, \ldots , k_r \laa<S> \right\}.
$$
It follows from Theorem~\ref{main} that  the maximal number of distinct letters in an $S$-invariant word is
the number  of elements in $\#\Part_n(S)+ 1$ (remember that the position of the last letter of $w$ is not in $\Part_n(S)$).

Assume that  $S\subseteq \Z_n^*$, and let  $d\geq 1$ be a divisor of $n$. In what
follows, we use the notation $\laa<S>_{d}$ for the subgroup
generated by $S$ in the group $\Z_{d}^*$, when the elements of $S$ are viewed as elements of $\Z_d^*$.
The order of the quotient group $\eritaso{\Z_d^*}{\laa<S>_d}  $
is denoted by $\left[ \Z_{d}^* \, : \, \laa<S>_{d} \right]$.




\begin{prop} \label{kaava}
Let $S \subseteq \Z_n^*$. Then
$$
\# \Part_n(S) \,=\, \sum_{d | n} \,\left[ \Z_{d}^* \, : \, \laa<S>_{d} \right]
\,=\, \sum_{d | n} \,  \frac{\varphi(d)}{\# \laa<S>_d} \,,
$$
where $\varphi$ denotes the Euler totient function.
\end{prop}
\begin{pf}
For all  $k_i\laa<S> \in \Part_n(S)$,  write $k_i=a_id_i$, where $d_i=\gcd(k_i,n)$ and
$a_i= k_i/d_i$, so that
$$
\Part_n(S) = \{ a_1 d_1 \laa<S>, \ldots , a_r d_r \laa<S> \} \,.
$$
We need a few auxiliary results to prove the claim. They are numbered accordingly.


If $\gcd(a,n)=1$, then   $adi \equiv adj \congmod n$  if and only if $di \equiv dj \pmod
n$, and thus
$$
1) \qquad \# ad\laa<S>  = \# d\laa<S> \,.$$

Consider next the mapping $d\laa<S> \longrightarrow \laa<S>_{n/d}$
defined by $di \mapsto i$. Firstly, this mapping is well-defined
because $\gcd(i,n)=1$ implies
$\gcd(i,n/d)=1$. Moreover, it is injective, which
is easily seen by using the equivalence
$ di \equiv dj  \congmod n$ if and only if  $i \equiv j \pmod{n/d}\,.$
Consequently,
$$
2) \qquad \# d \laa<S>  \leq \# \laa<S>_{n/d} \,.
$$


Now, let $\psi$ be the mapping
$$
\psi \,:\, \Part_n(S) \longrightarrow \bigcup_{d\, |\, n }
\eritaso{\Z_{n/d}^*}{\laa<S>_{n/d}} \,, \qquad ad\laa<S> \mapsto a\laa<S>_{n/d},
$$
so that $\psi$ associates the elements of $ \Part_n(S)$ with
conjugacy classes of the quotient groups
$\eritaso{\Z_{n/d}^*}{\laa<S>_{n/d}}$, where $d \,|\, n$. We leave
it to the reader to verify that $\psi$ is both well-defined and
injective. Next, consider the number $\alpha_d$ of the sets in
$\Part_n(S)$ of the form $ad\laa<S>$ with $\gcd(a,n)=1$.  Clearly
$$
\alpha_d = \#\{ i \, : \, d = \frac{k_i}{a_i} \},
$$
so that
\begin{equation} \label{vika}
 \qquad \sum_{d | n} \alpha_d = \#\Part_n(S)\,.
\end{equation}



It follows from the definition and injectivity of $\psi$  that $\alpha_d$ is at
most the number of elements in the quotient group $\eritaso{\Z^*_{n/k}}{\laa<S>_{n/d}}$,
that is to say,
$$
3) \qquad \alpha_d \leq [\Z^*_{n/d}\,:\,\laa<S>_{n/d}] \,.
$$

Now we are ready to employ these observations. Recall that $\Part_n(S)$ is a partition
of $\Z_n$, and hence
\begin{eqnarray}
n & = & \sum_{ad\laa<S> \in \Part(S)} \# ad\laa<S>
\,\stackrel{1)}{=}\,\sum_{ad\laa<S>
  \in \Part(S)} \# d\laa<S>  \nonumber \\
& = &  \sum_{d\, |\, n} \alpha_d \, \#d\laa<S> \, \stackrel{2)\, ,\,
  3)}{\leq} \,  \sum_{d \, |
  \, n} \, [\Z_{n/d}^* :\, \laa<S>_{n/d}] \,  \#  \laa<S>_{n/d}  \label{toka} \\
& = &  \sum_{d \,|\, n} \varphi(\frac{n}{d})\, = \, n \nonumber
\end{eqnarray}
(For the last equality, see~\cite[Th. 2.2]{Apo}.) Thus the inequality in~\eqref{toka} is actually an
equality. Consequently, also inequalities in $2)$ and $3)$ are
equalities. Hence $\alpha_d = [ \Z_{n/d}^* \, : \, \laa<S>_{n/d}]$, and
this together with \eqref{vika} proves the claim.
\end{pf}

















Let us introduce the notation
\[ \iota_n(S)=  \sum_{d | n} \,  \frac{\varphi(d)}{\#\laa<S>_d}.
\]
So by Proposition~\ref{kaava}, $\iota_n(S)$ denotes the number of elements in the partition of $\Z_n^*$ induced by $S$.

By combining the previous considerations and Theorem~\ref{main}, we can sum up our conversation
about $S$-invariant words so far as follows:

\begin{thm}\label{teoreema}
Let $S\subseteq S_n$. A word $w$  of length $n+1$ is $S$-invariant if and only if, for every set
$P\in \Part_n(S)$, the letters occupying the positions in $P$ are the same. Hence there exists, up to renaming the letters, a unique alphabetically maximal $S$-invariant word, and it has $\iota_n(S)+1$ distinct letters.
\end{thm}



Next we discuss the behavior of the function
$\iota_n(S_n)$. The following result tells us that its values
fluctuate heavily.

\begin{thm}\label{2006c} We have
\[
\liminf_{n \longrightarrow \infty} \frac{\iota_n(S_n)}{n} = 0
\qquad \textrm{and} \qquad
\limsup_{n\longrightarrow \infty} \frac{\iota_n(S_n)}{n} = 1
.
\]
\end{thm}
\begin{pf}
The value $\liminf_{n \longrightarrow \infty} \frac{\iota_n(S_n)}{n} = 0$ follows immediately from the fact that there exist infinitely many prime numbers $n$ with a primitive root that divides $n+1$.  We will prove this fact in Theorem~\ref{reviewer}.

The latter equality is trivial: Consider the values $n = p - 1 $ with $p$ a~prime number. Now, $\laa<S_{n}>_d = \laa<1>$ for all $d\,|\, n$, so
\[
\iota_{n}(S_{n}) =  \sum_{d | n} \,  \frac{\varphi(d)}{\#\laa<S_{n}>_d} =  \sum_{ d | n} \varphi(d) = n,
\]
and hence  $\limsup_{n\longrightarrow \infty} \iota_n(S_n)/n = 1$.

The proof is now complete.
\end{pf}




The last topic of this section was inspired by a question of J.~Cassaigne in WORDS'05 conference.
So far we have only considered words that are
invariant under transposition. But it is also natural to
consider the orbit that a word $w$ makes when the transposition operation is iterated  with respect to some fixed $p\times q$-matrix.
More precisely, if $f_{p,q} \colon \Sigma^{n+1} \rightarrow \Sigma^{n+1}$ is defined by $f_{p,q}(w)=w^T$, where transposition is carried out in a $p\times q$ matrix,
then the orbit we are interested in is the set
\[
\orb_{p,q}(w) = \set{f_{p,q}^i(w)}{i\geq 0}.
\]
We will characterize the possible sizes of these orbits in Theorem~\ref{2006b}, but first we need the following lemma.
\begin{lem}\label{Lag}
Let $G$ be a subgroup of $\Z_n^*$. Then for any $a \in \Z_n$, the order of the set $aG = \set{ax}{x\in G}$ divides the order of $G$.
\end{lem}
\begin{pf}
Note that, for $a \in \Z_n^*$, the claim is the Lagrange's theorem on orders of subgroups.
Let us denote
\[
\fix_a(G) = \set{x \in G }{ ax = a}.
\]
Clearly, $\fix_a(G)$ is a subgroup of $G$, so we can form the quotient group $\eritaso{G}{\,\fix_a(G)}$. Let
$\Psi \colon \eritaso{G}{\,\fix_a(G)}\rightarrow aG$ be the
mapping defined by $b \cdot \fix_a(G) \mapsto ab$. It is easy to show that $\Psi$ is well-defined and bijective. Hence
\[
|G| = \#\fix_a(G) \cdot \# aG,
\]
and the claim follows.
\end{pf}

Now we are ready for the last result of this section.

\begin{thm} \label{2006b}
Let $w$ be a word of length $n+1$, and suppose $n+1 = pq$. Then $\# \orb_{p,q}(w)$  divides  $\#\laa<p>$. Conversely, if
$d\geq 1$ divides  $\#\laa<p>$, then
there
exists a word $w$ such that $\#\orb_{p,q}(w)=d$.
\end{thm}
\begin{pf}
As was seen in the proof of Proposition~\ref{pq-tapaus}, the letter at the position $k$ moves to the
position $kp \mbox{\,mod\,} n$ in transposition. Hence, the letters at positions $k\laa<p>$ travel on an independent orbit,
and therefore $\# \orb_{p,q}(w)$ equals the least common multiple of the sizes of orbits of letters occupied at positions
$k\laa<p>$ for all $k\in \Z_n$.

Now consider the orbit $k\laa<p>$, and let $r=\# k\laa<p>$.  Let
$a_1, a_2, \ldots, a_r$ denote the letters of $w$ at positions $k$,
$kp$, $kp^2$, $\ldots$, $kp^{r-1}$ modulo $n$, respectively. After
each iteration of $f_{p,q}$ on $w$, the mutual arrangement of these
letters in the positions $k\laa<p>$ changes as follows:
\[
a_1a_2\cdots a_{r-1}a_r \rightarrow a_r a_1 a_2 \cdots a_{r-1} \rightarrow \cdots \rightarrow a_2 a_3 \cdots a_r a_1
\rightarrow a_1a_2\cdots a_{r-1}a_r.
\]
Hence when the orbit of these letters is full, the corresponding words give rise to the word equation $uv=vu$, where $u=a_1 a_2 \cdots a_i$
and $v = a_{i+1}a_{i+2} \cdots a_r$. One of the fundamental theorems of Combinatorics on Words (see~\cite{Lot}) says that  two words commute if and only
they are powers (repetitions) of some common word.
Hence the length of the orbit of the letters at positions $k\laa<p>$ ,$|u|$,  divides $\#k\laa<p>$ (=$|uv|$), and so by Lemma~\ref{Lag},
it divides $\# \laa<p>$. Since $\# \orb(w)$ is the least common multiple  of $ \# k\laa<p>$ with $k\in \Z_n$, it follows that also $\# \orb(w)$ divides $\#\laa<p>$.



Conversely, suppose $d$ divides  $\#\laa<p> = \#\laa<q>$. Define a word $w=w_0w_1 \cdots w_n$ as follows:
\[
w_i = \left\{
\begin{array}{cl}
1  & \quad\textrm{if $i$ is of the form $i=q^{dj} \mbox{\,mod\,} n$ for some $j\geq 0$,} \\
0  & \quad \textrm{otherwise.} \\
\end{array} \right.
\]
Then, by the construction, the orbit of $w$ consists of exactly $d$ distinct words. The proof is now complete.
\end{pf}

Here is an example of the situation of the previous theorem.
Let us consider a word $w=w_0w_1w_2\ldots w_{45}$ of length 46 in
$2\times 23$ matrix. In this case $n=45$, $p=2$, and $q=23$. The
mapping $f_{2,23}$ generates nine separate orbits for letters
$w_i$ in $w$ which are obtained from $\laa<2>$ except the trivial
one that corresponds to the last symbol $w_{45}$: \medbreak
\def\goto{, }

\noindent \bgroup \fontsize{8}{9} \selectfont
$\begin{array}{rll} \label{taulukko}
 & \mbox{orbit} & \mbox{corresponding subword}\\\hline
0\laa<2>: & 0 & w_0\\
1\laa<2>: & 1 \goto 2 \goto 4 \goto 8 \goto 16 \goto 32 \goto 19
\goto 38 \goto 31
\goto 17 \goto 34 \goto 23 & w_1w_2w_4w_8w_{16}w_{32}w_{19}w_{38}w_{31}w_{17}w_{34}w_{23}\\
3\laa<2>: & 3 \goto 6 \goto 12 \goto 24 & w_3w_{6}w_{12}w_{24}\\
5\laa<2>: & 5 \goto 10 \goto 20 \goto 40 \goto 35 \goto 25 & w_5w_{10}w_{20}w_{40}w_{35}w_{25}\\
7\laa<2>: & 7 \goto 14 \goto 28 \goto 11 \goto 22 \goto 44 \goto 43
\goto 41 \goto 37
\goto 29 \goto 13 \goto 26 & w_7w_{14}w_{28}w_{11}w_{22}w_{44}w_{43}w_{41}w_{37}w_{29}w_{13}w_{26}\\
9\laa<2>: & 9 \goto 18 \goto 36 \goto 27 & w_9w_{18}w_{36}w_{27}\\
15\laa<2>: & 15 \goto 30 & w_{15}w_{30}\\
21\laa<2>: & 21 \goto 42 \goto 39 \goto 33 & w_{21}w_{42}w_{39}w_{33}\\
\mbox{last symb.}:& 45 & w_{45}
\end{array}
$ \egroup \medbreak

As described in the proof of Theorem~\ref{2006b}, the length of the orbit of the
letters in $k\laa<p>$, say $r_k$, divides $\#k\laa<p>$. More
precisely, $$z_k=z_{k_0}\ldots z_{k_{\#k\laa<p>-1}}=u_k^{r_k},$$
where $z_{k_i}= w_{kp^i-1 \mbox{\,mod\,} n}$ and $u_k$ is the shortest word
such that $z_k\in u_k^+$. Thus, by simple combinatorics, we obtain
$$\#Orb(w)=\mathop{\mbox{lcm }}_{k\in\Z_{n}}(r_k).$$

If we let
$$w = aabbcacabacacbbaabbccbbbbccbcacabbccacbbabbacc,$$
we have
$$
\begin{array}{rll}
i\ \ & x_i & \mbox{length of period}\\
\hline
0\laa<2> & z_0 = a & 1\\
1\laa<2> & z_1 = abcbabcbabcb & 4\\
3\laa<2> & z_3 = bccb & 4\\
5\laa<2> & z_5 = accacc & 3\\
7\laa<2> & z_7 = abcabcabcabc & 3\\
9\laa<2> & z_9 = abab & 2\\
15\laa<2> & z_{15} = ac & 2\\
21\laa<2> & z_{22} = bbbb & 1
\end{array}
$$
Moreover, by above observation, $\#Orb(w)=12$












\section{Favorable Prime Numbers}


In this section we study the lengths, or integers, such that the only transposition invariant words of that length are of the form $ab^*c$.   For example, $4,6,12,$ and $14$ are such lengths, as is readily verified.

We begin with an example: Let $w$ be a word of length $n+1$ over four letter alphabet $A=\{a,b,c,d\}$ defined
by
$$
w_i = \left\{
\begin{array}{cl}
a  & \quad\textrm{if $i=0$,} \\
b  & \quad \textrm{if $\gcd(i,n)=1$,} \\
c   &  \quad \textrm{if $\gcd(i,n)>1$ and  $i<n$,} \\
d & \quad \textrm{if $i=n$}
\end{array} \right.
$$
for all $ 0\leq i  \leq n$. Then $w$ is transposition invariant.
For if $0 \leq i,j \leq n$,  where $\gcd(i,n)=1$ and
$\gcd(j,n)>1$, then $i\laa<S> \cap j\laa<S> = \emptyset$. Moreover,
if $n$ is composite, then both letters $b$ and $c$ occur in $w$,
and thus $w$ is nontrivial. We conclude that if a positive integer
$n$ is composite, then there always exist nontrivial
$S$-invariant words of length $n+1$ for every $S\subseteq S_n$.
This leads us to the next result.


\begin{thm}\label{favorable}
Let $S\subseteq S_n$. There exist only trivial $S$-invariant words of
length $n+1$ if and only if $n$ is prime and $\laa<S> = \Z_n^*$.
\end{thm}
\begin{pf}
Assume there exist only trivial $S$-invariant words of length $n+1$. This is equivalent to the
condition that the partition of $\Z_n$ generated by $\laa<S>$ has only two
elements, $\{0\}$ and $\{1,2,\ldots,n-1\}$, which then have to be $0\laa<S>$
and $1\laa<S>$, respectively. This is equivalent to $\laa<S>=\Z_n \setminus
\{0\}$, which in turn, happens exactly when $n$ is prime and $\laa<S> = \Z_n\setminus \{0\} = \Z_n^*$.
\end{pf}


Motivated by Theorem~\ref{favorable}, we say that a prime number
$n$  is {\em favorable} (for the existence of nontrivial transposition invariant
words) if there exists a nontrivial transposition invariant word of length $n+1$, that
is, if $\laa<S_n> \neq \Z_n^*$. Next we will prove that there exist
infinitely many favorable primes. This is done with the help of quadratic residues
(see~\cite{Apo,IR}). To do that, we
need the following lemma.

\begin{lem} \label{residue}
If a positive integer $n$ is prime and $n \equiv 7 \pmod{8}$, then
every integer dividing $n+1$ is a quadratic residue modulo~$n$.
\end{lem}
\begin{pf}
Since the product of two quadratic residues modulo~$n$ is a
quadratic residue, it is enough to show that each prime divisor of
$n+1$ is a quadratic residue modulo~$n$. So assume that $p$ is
prime and $p\,|\,n+1$. It is well-known that $2$ is a quadratic residue modulo prime $n$ exactly when
$n \equiv \pm 1 \pmod 8$. Hence the case $p=2$ is clear. Suppose then that $p>2$.
Now, by using the basic principles of residue calculation,
we get
\begin{eqnarray}
\left(\frac{p}{n}\right) &=&  (-1)^{\frac{p-1}{2} \cdot \frac{n-1}{2}}
\left(\frac{n}{p}\right) \quad \textrm{\big(law of quadratic reciprocity\big)}
\nonumber \\
&=& (-1)^\frac{p-1}{2} \left(\frac{n}{p}\right) \quad \textrm{\big( $n \equiv 7
 \congmod 8$  implies $\frac{n-1}{2}$ odd \big)}
\nonumber \\
&=& (-1)^\frac{p-1}{2} \left(\frac{-1}{ p}\right) \quad \textrm{\big( $n \equiv
-1 \congmod p$ \big)} \nonumber \\
&=&  (-1)^\frac{p-1}{2}\cdot
(-1)^\frac{p-1}{2} \quad \left(\left(\tfrac{-1}{ p}\right) = (-1)^{\frac{p-1}{2}} \right)  \nonumber \\
&=& 1 \,. \nonumber
\end{eqnarray}
Hence $p$ is a quadratic residue modulo~$n$, and the claim follows.
\end{pf}

\begin{thm}
There exist infinitely many favorable primes.
\end{thm}
\begin{pf}
 Let $n$ be a prime number with $n\equiv 7  \pmod{8}$. Dirichlet's
theorem on arithmetic progressions (see \cite{Apo}) says
that there exist infinitely many such primes $n$. By
Lemma~\ref{residue}, every integer in the set $S_n$ is a quadratic
residue modulo~$n$. Thus every integer in the set $\laa<S_n>$ also
is a quadratic residue. But exactly half of all the integers in
$\Z_n^*$ are quadratic residues modulo~$n$; the other half is the
set of quadratic non-residues modulo~$n$. Therefore, $\laa<S_n>
\neq \Z_n^*$, and nontrivial transposition invariant words of length $n+1$ exist, so that
$n$ is favorable.
\end{pf}


The other direction, the infinitude of unfavorable primes, is more
difficult, and requires some more advanced techniques. We prove that
there exists infinitely many primes $p$ with a primitive root
dividing $p+1$, which is  stronger statement than just
unfavorability of $p$. The proof is a little application of
Heath-Brown's \cite{Heath-Brown} ideas. To present the proof, we
need some technical definition and a lemma.

Suppose $\alpha, \delta >0$ and $\alpha + \delta < \tfrac{1}{2}$.
The notation $ n = P_2(\alpha, \delta)$ means that either $n$ is prime, or $n = p_1 p_2$ with $p_1, p_2$ primes and $n^{\alpha} \leq p_1 \leq n^{\tfrac{1}{2} - \delta}$.

In what follows, $p$ always denotes a prime number. The notation $f(x) \ll g(x)$ means the same as $f(x) = O(g(x))$.
Finally, $\mbox{exp}_p(a)$ denotes  $\# \laa<a>_p$.
The following lemma is proved in \cite{Heath-Brown}.

\begin{lem}\label{hb}
Let $k=1,2,$ or $3$, and put $K = 2^k$. Suppose $u$ and $v$ are coprime integers such that $K \mid u - 1$, $16 \mid v$, and
$\left( \tfrac{u-1}{K} , v\right) = 1$. Then there exist $\alpha \in (\tfrac{1}{4}, \tfrac{1}{2})$ and $\delta \in (0, \tfrac{1}{2} - \alpha)$  such that the set
\[
S(x) = \set{ p \leq x }{ p \equiv u \pmod{v}, \tfrac{p-1}{K} = P_2(\alpha, \delta)},
\]
satisfies
\[
\# S(x)   \gg \frac{x}{ \log^2 x}.
\]
\end{lem}

%For the purposes of the proof below, we denote
%\[
%S = \lim_{x \longrightarrow \infty} S(x).
%\]

\begin{thm} \label{reviewer}
There exist infinitely many primes $p$ with a  primitive root  that divides
$p+1$.
\end{thm}


\begin{pf}
By  setting $u = 2 \cdot 3 \cdot 7 \cdot 11 - 1 = 461$, $v = 16 \cdot 3 \cdot 7 \cdot 11=3696$, and $k=2$, we can
apply Lemma~\ref{hb}, since  $u,v$, and $k$ clearly satisfy its conditions.

We will show that there are infinitely many primes $p$ such that $p \equiv u \pmod{v}$ and either $3,7,$ or $11$ is a primitive root $\bmod\ p$. This will then attest the claim because each of $3,7,11$ is a divisor of $p+1$.

First we show that if $p \in S(x)$, then $4$ divides each of $\mbox{exp}_p(3), \mbox{exp}_p(7),$ and $\mbox{exp}_p(11)$.
Since $4 \mid p - 1$,  the law quadratic reciprocity gives
\[
\left( \frac{3}{p} \right) = \left( \frac{7}{p} \right) = \left( \frac{11}{p} \right) = -1.
\]
Now, let $a$ denote some primitive root  modulo $p$, and suppose $a^k \equiv 3 \pmod{p}$ for some $k$. A well-known result
(see \cite[Lemma~1 on p. 206]{Apo})
gives
\[
\mbox{exp}_p(3) = \frac{p-1}{\mbox{gcd}(k,p-1)}.
\]
Since $3$ is a quadratic nonresidue modulo $p$, $k$ must be odd. Thus, since $p-1$ is divisible by $4$, so is $\mbox{exp}_p(3)$. The same reasoning applies for $7$ and $11$.

Next we analyze different possibilities that can occur with $p\in S(x)$. We have either $p=4q + 1$ with $q$ a prime, or $p=4p_1 p_2 + 1$ with $p_1,p_2$ primes. If $p=4q + 1$  and $p > 11^4$, then  $\ord_p(3) \mid 4q = p - 1$ implies $\ord_p(3) = p - 1$. Hence if $\lim_{x\rightarrow \infty} S(x)$ contains infinitely many primes of the form $4q+1$, we are done.

Therefore we may assume that $S(x)$ has  $\gg \tfrac{x}{\log^2 x}$ primes of them form $p=p_1 p_2 +1$. Let us denote by $S_2(x)$ this subset of primes in $S(x)$.

If $p \in S_2(x)$ and $p > 11^4$, then
\[
\mbox{exp}_p(3), \mbox{exp}_p(7), \mbox{exp}_p(11) \in \{4p_1 ,4p_2, p-1\}.
\]
In what follows, we will show that for infinitely many $p\in S_2(x)$, one of $\mbox{exp}_p(3), \mbox{exp}_p(7), \mbox{exp}_p(11$ ) must equal $p-1$.

Suppose that $p \in S_2(x)$ and $\mbox{exp}_p(3) = 4p_1$. We make the following auxiliary estimation. Let
\[
T(X) = \set{p \leq x}{ \mbox{exp}_p(3) \leq 4x^{1/2-\delta} }.
\]
Then
\begin{align*}
\# T(x) &
\leq
\sum_{e \leq 4x^{1/2-\delta}} \# \set{p \leq x}{ p \mid 3^e - 1}
\ll
\sum_{e \leq 4x^{1/2-\delta}} \log(3^e - 1)
\\
&
\ll
\sum_{e \leq 4x^{1/2-\delta}} e
\ll
x^{1-2\delta}.
\end{align*}
Since $1-2\delta < 1$, we have $x^{1-2\delta} = o\left(\tfrac{x}{\log^2 x}\right)$.
Now, observe  that $p$ is in $T(x)$ because if $p \in S_2(x)$, then  $4 p_1 \leq 4x^{\tfrac{1}{2} - \delta}$.
Hence the number of primes $p$ in $S_2(x)$ with $\mbox{exp}_p(3) = 4p_1$ is $o\left(\tfrac{x}{\log^2 x}\right)$.
The same estimate holds for $7$ and $11$.

Therefore there has to be  $\gg \tfrac{x}{\log^2 x}$ primes $p$ in $S_2(x)$ such that
\[
\mbox{exp}_p(3), \mbox{exp}_p(7), \mbox{exp}_p(11) \in \{4p_2, p-1\}.
\]
If there are infinitely many $p \in S_2(x)$ such  that $\mbox{exp}_p(l) = p - 1$ for some $l=3,7,11$, our proof is done. If not, there must be $\gg \tfrac{x}{\log^2 x}$ primes $p\in S_2(x)$ such that $\mbox{exp}_p(3) = \mbox{exp}_p(7) = \mbox{exp}_p(11) = 4p_2$. We will derive a contradiction by showing that the number of these primes is $o\left(\tfrac{x}{\log^2 x}\right)$.

The integers $n = 3^e \cdot 7^f \cdot 11^g$ for $0 \leq e,f,g \leq 2x^{(1-\alpha)/3}$ all satisfy $n^{4p_2} \equiv 1 \pmod{p}$.
Hence by Lagrange's theorem, $n$ can have at most $4p_2$ values. But the number of triples $(e,f,g)$ is at least $8x^{1-\alpha} \geq 8p^{1-\alpha} \geq 8p_2$, so that there are at least two distinct triples that produce the same $n$ $\bmod\ p$. It follows that $p$ divides the numerator $N$ of $3^e \cdot 7^f \cdot 11^g - 1$ for some triple $(e,f,g)$ with $ |e|,|f|,|g| \leq 2x^{(1-\alpha)/3}$ and $(e,f,g) \neq (0,0,0)$. Clearly, $N\neq 0$. It follows that the number of primes $p$ such that  $p \in S_2(x)$ and $\mbox{exp}_p(3) = \mbox{exp}_p(7) = \mbox{exp}_p(11) = 4p_2$ is at most the number of different triples $(e,f,g)$ with $ |e|,|f|,|g| \leq 3x^{(1-\alpha)/3}$ times the number of prime divisors of $N$. Since,
\[
\log |N| \ll \max\left( |e|, |f|, |g| \right) \ll x^{(1-\alpha)/3},
\]
$N$ has $\ll x^{(1-\alpha)/3}$ prime factors. Thus the number of such primes $p$ in $S_2(x)$ is $ \ll x^{(1-\alpha)/3} x^{1-\alpha}  = o\left(\tfrac{x}{\log^2 x}\right)$ (observe that $4(1-\alpha)/3 < 1$)
This contradiction shows that there are infinitely many primes in $\lim_{x\rightarrow\infty}S(x)$ such that either $3,7,$ or $11$ is primitive root $\bmod\ p$. The proof is now complete.


\end{pf}


If we denote by $F(x)$ the number of primes $p$ up to $x$ for which
there exists a primitive root modulo $p$ dividing $p+1$, then
clearly
\[
F(x)\geq \#\{p\leq x\textrm{ such that }2\textrm{ is a
primitive root modulo }p\}.
\]
Let $N_2(x)$ be the quantity on the
right hand side above. A famous result by Hooley \cite{Hooley}
states that the Generalized Riemann Hypothesis implies
 \[
 N_2(x)\sim A \frac x{\log x},
 \]
where $A=\prod_{l\textrm{
prime}}\left(1-(l^2-l)^{-1}\right)=0.3739558136192\cdots$ is the
Artin's constant. This observation implies immediately that (on GRH)
there exists a positive density of primes $p$ with a primitive root
dividing $p+1$.


On the other hand numerical evidence suggests that
\[
F(x)\sim B  \frac x{\log x},
\]
where $ B\geq 0.65$. We believe that the problem
of computing the exact value of~B can be tackled via the Generalized
Riemann Hypothesis.





\begin{thebibliography}{9}

\bibitem{AS}
J.-P. Allouche and J. Shallit. \emph{Automatic Sequences}, Cambridge, 2003.

\bibitem{Apo}
T. M. Apostol. \emph{Introduction to Analytic Number Theory}, New York:
Springer-Verlag, 1976.

%\bibitem{Goldstein}
%L. Goldstein. Density questions in algebraic number theory, \emph{Amer. Math. Monthly}
%{\bf 78}, 342--351, 1971.

\bibitem{Heath-Brown}
D. R. Heath-Brown. Artin's conjecture for primitive roots, \emph{Quart. J. Math. Oxford} (2) {\bf 37}, 27-38, 1986.

\bibitem{HB}
D. R. Heath-Brown. Zero-free regions for Dirichlet $L$-functions and the least prime in an arithmetic progression,
\emph{Proc. London Math. Soc.} (3)  {\bf 64}, 265--338,  1992.

\bibitem{Hooley}
C. Hooley. On Artin's Conjecture, \emph{J. Reine Angew. Math.} {\bf 225},
209--220, 1967.

\bibitem{IR}
K. Ireland and M. Rosen. \emph{A Classical Introduction to Modern Number Theory (2nd ed.)},
Graduate texts in mathematics {\bf 84}, New York: Springer-Verlag, 1990.

\bibitem{Lot}
M. Lothaire. \emph{Combinatorics on Words}, Vol. 17 of \emph{Encyclopedia of
 Mathematics and Its Applications.} Addison-Wesley, 1983.

\end{thebibliography}

\end{document}
