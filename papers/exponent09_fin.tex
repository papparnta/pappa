\documentclass{amsart}
\DeclareMathSymbol{\nmid}{\mathrel}{AMSb}{"2D}
\def\Z{{\mathbb Z}}
\def\F{{\mathbb F}}
\newtheorem*{theorem}{Theorem}
\newtheorem{lemma}{Lemma}

\title[On the exponent of elliptic curves]{On the exponents of the group of points of an Elliptic curve over a finite field}
\author{Francesco Pappalardi}
\address{Dipartimento di Matematica\\ Universit\`a Roma Tre\\ Largo San Leonardo Murialdo 1\\ I--00146\\ Roma\\ ITALY}
\email{pappa@mat.uniroma3.it}
\date{December 23, 2009}
\keywords{keywords: elliptic curves, finite fields}
\subjclass[2000]{Primary: 11G20; Secondary: 11G5}
\begin{document}
\begin{abstract} We present a lower bound for the exponent of the group
of rational points of an elliptic curve over a finite field. Earlier results 
considered finite fields $\F_{q^m}$ where either $q$ is fixed or $m=1$ and
$q$ is prime. Here we let both $q$ and $m$ vary and our estimate is explicit
and does not depend on the elliptic curve.\end{abstract}
\maketitle
\section{introduction}
Let $\F_q$ be a finite fields with $q=p^m$ elements and let $E$ be an elliptic 
curve defined over $\F_q$. It is well known (see for example the book of Washington \cite{washington})
that the group of rational point of $E$ over $\F_q$ satisfies
$$E(\F_q)\cong \Z_n\times\Z_{nk}$$
where $n,k\in\mathbb N$ are such that $n\mid q-1$. The exponent of $E(\F_q)$ is
$$\exp(E(\F_q))=nk.$$
In 1989 Schoof \cite{schoof} proved that if $E$ is an elliptic curve over $\mathbb Q$ without complex multiplication, then
for every prime $p>2$ of good reduction for $E$, one has the estimate 
$$\exp(E(\F_p))> C_E \sqrt{p}\frac{\log p}{(\log\log p)^2}$$
where $C_E>0$ is a constant depending only on $E$.

In 2005 Luca and Shparlinski \cite{lucashparlinski} consider the case when $q$ is fixed and they prove that if $E/\F_q$
is ordinary, the there exists an effectively computable constant $\vartheta(q)$ depending only on $q$
such that
\begin{equation}\label{eq:florianigor}
\exp(E(\F_{q^m}))> q^{m/2+\vartheta(q)m/\log m}
\end{equation}
holds for all positive integers $m$.

Other lower bounds that hold for families of primes (resp. for families of powers of fixed primes) with density one were
proven by Duke in \cite{duke} (resp. by Luca and Shparlinski in \cite{lucashparlinski}).

Here we let both $p$ and $m$ vary and we prove the following
\begin{theorem} Let $E$ be any elliptic curve over $\F_{p^m}$ where $m\ge3$ then either $m=2r$ is even and 
$$E(\F_{p^{2r}})\cong \Z_{p^r\pm1}\times\Z_{p^r\pm1}$$
or
$$\exp(E(\F_{p^m}))\ge 2^{-46}p^{m/2}\frac{m^{1/3}}{(\log m)^{7/3}(\log\log m)^{1/3}}.$$
\end{theorem}

In the above we assume that $\log$ denotes the logarithm in base $2$. 

Note that the result also applies to supersingular elliptic curves and that it improves on that in (\ref{eq:florianigor})
only for values of $m$ which are small with respect to $p$.  

\section{lemmata}
The proof is based on estimates for the distance between perfect powers due to Bugeaud. More
precisely, we will apply the following result from \cite{bugeaud}:
\begin{lemma}\label{lem:bug}
Let $f\in\Z[X]$ be a monic polynomial of degree $d\ge 2$ without multiple roots. 
Let $H$ be the maximum of absolute values of its coefficients and $D$ its discriminant. 
Let $a$, $x$, $y$, $m$ be rational integers satisfying 
$a\not=0$, $|y|\ge 2$, $m\ge 2$, $f(x)=ay\sp{m}$. 
Denote by $\log\sb{2}$ the logarithm in base $2$ and 
write $\log\sb{*}x$ for $\max\{\log x,1\}$. 
The inequality $m\le\max\{U,V\}$ holds with $U=d\log\sb{2}(2H+3)$ and 
$$V=2\sp{15(d+6)}d\sp{7d}|D|\sp{3/2} (\log|D|)\sp{3d}(\log\sb{*}|a|)\sp{2}\log\sb{*}\log\sb{*}|a|.$$
\end{lemma}

We need the following elementary lemma:
\begin{lemma}
\label{lem:sydney} If $q$ is a prime power and $E$ is an elliptic
curve defined over $\F_q$ such that $E(\F_q)\cong\Z_n\times\Z_{nk}$,
then $q=n^2k+n\ell+1$ for some integer $\ell$ that satisfies
$|\ell|\le 2\sqrt{k}$.
\end{lemma}
\begin{proof}
By the Hasse bound, we can write $n^2k=q+1-a_q$ for some integer
$a_q$ that satisfies $a_q^2\le 4q$. Using the Weil pairing
one also sees that $q\equiv 1\pmod{n}$. Hence $a_q=2+n\ell$ for some
integer $\ell$ and $q=n^2k+n\ell+1$. Finally
$$
n^2\ell^2+4n\ell+4=a_q^2\le 4q=4n^2k+4n\ell+4
$$
and the result follows.
\end{proof}

We will also need the classical characterizations of the group structures due 
to Waterhouse (see \cite[Theorem~4.3, page~98]{washington}) which describes 
possible cardinalities $\#E(\F_q)$ of the set of $\F_q$-rational 
points of elliptic curves over $\F_q$. 

\begin{lemma} 
\label{lem:Water}                          
 Let $q=p^m$ be a power of a prime $p$ and let $N = q + 1-a$. There is an elliptic 
curve $E$ defined over $\F_q$ such that $\#E(\F_q) = N$ if and only if $|a|\le 2\sqrt q$ 
and $a$ satisfies one of the following:
\begin{itemize} 
\item[(i)] $\gcd(a, p) = 1$;
\item[(ii)] $m$ even and $a=\pm 2\sqrt q$;
\item[(iii)] $m$ is even, $p\not\equiv 1 \pmod 3$, and $a = \pm\sqrt q$;
\item[(iv)] $m$ is odd, $p = 2$ or $3$, and $a = \pm p^{(n+1)/2}$;
\item[(v)] $m$ is even, $p\not\equiv 1 \pmod 4$, and $a = 0$;
\item[(vi)] $m$ is odd and $a = 0$.
\end{itemize}
\end{lemma}

For each admissible cardinality, R\"uck (see Washington \cite[Theorem~4.4, page~98]{washington})
describes possible group structures. 

\begin{lemma}
\label{lem:Ruck}
Let $N$ be an integer that occurs as the  order  of an elliptic curve 
over a finite field $\F_q$ where  $q=p^m$
is a power of a prime $p$.  Write 
$N = p^e n_1 n_2$ with $p\nmid n_1 n_2$ and $n_1\mid n_2$.
(possibly $n_1 = 1$). There is an elliptic curve $E$ over $\F_q$ such 
that
$$E(\F_q)\cong \Z_{p^e}\times\Z_{n_1}\times\Z_{n_2}$$
if and only if
\begin{enumerate}
\item $n_1 = n_2$ in the case~(ii) of Lemma~\ref{lem:Water};
\item $n_1 |q - 1$ in all other cases of Lemma~\ref{lem:Water}.
\end{enumerate}
\end{lemma}

Finally we need the following numerical statement:

\begin{lemma}\label{lem:numerical} Assume that $\alpha$ and $\beta$ are real numbers with $\alpha>4$ and $\beta\ge4$.
If 
$$\alpha\le \beta^{3/2}\cdot(\log\beta)^7\cdot\log\log\beta,$$
then
$$
\beta\ge\frac{\alpha^{2/3}}{(\log\alpha)^{14/3}(\log\log\alpha)^{2/3}}.$$
\end{lemma}

\begin{proof} If $\alpha\ge\beta\ge4$, then
$$\beta\ge\left(\frac\alpha{(\log\beta)^7\log\log\beta}\right)^{2/3}
\ge\left(\frac\alpha{(\log\alpha)^7\log\log\alpha}\right)^{2/3}.$$
If $\alpha\le\beta$, 
$$\beta\ge \alpha\ge\frac{\alpha^{2/3}}{(\log\alpha)^{14/3}(\log\log\alpha)^{2/3}}.$$
\end{proof}

\section{Proof of the Theorem}
Assume that 
$E(\F_{p^m})\cong\Z_n\times\Z_{nk}.$ Then, by Lemma~\ref{lem:sydney}, we have that
$$p^m=kn^2+\ell n+1\qquad\text{for some }\ell\text{ with }|\ell|\le2\sqrt{k}.$$

If $\ell=\pm2\sqrt{k}$, then $k$ must be a perfect square and we write $k=M^2$ so that
$\ell=\pm2M$. Therefore in the
above identity we have
$$p^m=(Mn\pm1)^2$$
which implies that $m=2r$ is even and that $Mn=p^r\mp1$. Furthermore in this case
$$p^m+1-\#E(\F_{p^m})=\ell n+2=\pm2Mn+2=\pm2p^{m/2}.$$
This happens precisely in the case (ii) of Lemma~\ref{lem:Water}. Note also that
in this case $p\nmid\#E(\F_{p^m})$. Hence, by Case (1) in Lemma~\ref{lem:Ruck}, we
have that $n=nk$ so that $k=1$.

We conclude that if $l=\pm2\sqrt{k}$, then $k=1$, $n=p^{r}\mp1$ and finally
$$E(\F_{p^{2r}})\cong\Z_{p^r\pm1}\times\Z_{p^r\pm1}.$$

From now on we can assume that $|\ell|<2\sqrt{k}.$ 
We apply Lemma~\ref{lem:bug} with the following data:
$$f(X)=X^2+\ell X+k,\quad d=2,\quad |D|=4k-\ell^2,\quad H=k,$$
$$ x=kn,\quad y=p\quad\text{ and }\quad a=k.$$ 
Note that since $|\ell|<2\sqrt{k}$, we have that $D\neq0$ so that $f$ has
two distinct roots.

From the identity $kp^m=(kn)^2+\ell(kn)+k$ and from Lemma~\ref{lem:bug}, 
it follows that
$$m\le\min\{2\log_{2}2k+3,2^{134}(4k)^{3/2}(\log4k)^6\log_*k\log_*\log_*k\}.$$
Since we can assume that $k\ge2$, it follows that
$$m\le2^{134}(4k)^{3/2}(\log4k)^{7}\log\log4k.$$

If $4\le m\le 2^{136}$, then 
$m^{1/3}/(2^{46}(\log m)^{7/3}(\log\log m)^{1/3}) <1/4$ and the statement of the Theorem is vacuous 
since $\exp(E(\F_q))\ge\sqrt{q}-1$ for every $q$.
 
If $m>2^{136}$, we apply Lemma~\ref{lem:numerical} with $\alpha=m/2^{134}>4$ and $\beta=4k$ and we obtain:
$$k\ge\frac14\cdot\frac{(m/2^{134})^{2/3}}{(\log_* \frac m{2^{134}})^{\frac{14}3}(\log_*\log_* \frac m{2^{134}})^{\frac23}}
\ge
\frac1{2^{\frac{274}3}}\cdot\frac{m^{2/3}}{(\log m)^{14/3}(\log\log m)^{2/3}}.
$$
and so
$$\exp(E(\F_{p^m})=nk\ge (\sqrt{p^m}-1)\sqrt{k}\ge p^{m/2}\frac{m^{1/3}}{2^{46}(\log m)^{7/3}(\log\log m)^{1/3}}.$$
This concludes the proof of the Theorem.\hfill$\square$\bigskip

The constant $2^{-46}$ can be slightly improved with a more careful analysis but this
is not too important.

\section{Conclusion}

In the opposite direction a recent result of Matom\"aki in \cite{kaisa} states that, for any $\epsilon>0$, 
there exists infinitely many primes $p$ of the form $p=an^2+1$ with $a<p^{1/2+\epsilon}$.
By Lemma~\ref{lem:Ruck} and Lemma~\ref{lem:sydney}, it follows that for infinitely many primes $p$ there exists an
elliptic curve over $\F_{p}$ such that
$$E(\F_{p})\cong \Z_{n}\times\Z_{na}.$$
This implies that there exists a infinite sequence of distinct $p$ with an ordinary elliptic curve $E/\F_p$ such that
$$\exp(E(\F_p))=p^{3/4+\epsilon}(1+o(1)).$$

One can also consider the polynomial
identity
$$X^3=(X+2)(X-1)^2+3(X-1)+1.$$
By Lemma~\ref{lem:Ruck} and Lemma~\ref{lem:sydney}, it follows that for every prime $p$ there exists an ordinary
elliptic curve over $\F_{p^3}$ such that
$$E(\F_{p^3})\cong \Z_{p-1}\times\Z_{(p+2)(p-1)}.$$

We immediately conclude that there exists a infinite sequence of distinct
$q$ with an elliptic curve $E/\F_q$ such that
\begin{equation}\label{eq:twothirds}
\exp(E(\F_q))=q^{2/3}(1+o(1)).\end{equation}

This should be compared on one side with Schoof result in \cite{schoof} that (Assuming 
GRH), if $E$ is an elliptic curve over $\mathbb Q$, there exists a constant $c_E$
such that $\exp(E(\F_p))<c_Ep^{7/8}\log p$ for infinitely many primes $p$ and on another side
with Luca, McKee and Shparlinski's results in \cite{lms} that there exists an absolute
constant $\rho>0$ such that if $E/\F_q$ is a fixed
elliptic curve, the inequality
$$\exp(E(\F_{q^m}))< q^m\exp\left(-m^{\rho/\log\log m}\right)$$
holds for infinitely many positive integers $m$.  

We wonder if for every $\epsilon>0$ one can construct an infinite family of
prime powers $q$ each with an elliptic curve $E/\F_q$ such that
$$E(\F_q)\not\cong\Z_{\sqrt{q}\pm1}\times\Z_{\sqrt{q}\pm1}$$
and
$$\exp(E(\F_q))\ll_\epsilon q^{1/2+\epsilon}$$
or if the $2/3$ in (\ref{eq:twothirds}) can be improved.

More generally, the polynomial identity
$$X^n=(X-1)^2\cdot\left(X^{n-2}+2X^{n-3}+\cdot+(n-2)X+n-1\right)+n(X-1)+1$$
leads, for any fixed $n\ge2$ to an infinite sequence  of distinct
primes $p$ with an elliptic curve $E/\F_{p^n}$ such that
$$
\exp(E(\F_{p^n}))=p^{n-1}(1+o(1)).
$$
Furthermore, if $n\ge3$ each $E/F_{p^n}$ is ordinary.
\bigskip


\noindent\textbf{Acknowledgement:} The author would like to thank Bill Banks and Igor Shparlinski
for some useful conversations. 

\begin{thebibliography}{99}
\bibitem{duke}
\textsc{Duke, William}
\textit{Almost all reductions modulo $p$ of an elliptic curve have a large exponent.}  C. R. Math. Acad. Sci. Paris.  
\textbf{337}  (2003),  no. 11, 689--692.

\bibitem{bugeaud}
\textsc{Bugeaud,Yann}
\textit{Sur la distance entre deux puissances pures}
C. R. Acad. Sci. Paris S\'er. I Math. \textbf{322} (1996), no. 12, 1119--1121

\bibitem{lms}
\textsc{Luca, Florian; McKee, James; Shparlinski, Igor E.} 
\textit{Small exponent point groups on elliptic curves.}
J. Th\'eor. Nombres Bordeaux  \textbf{18}  (2006),  no. 2, 471--476.

\bibitem{lucashparlinski}
\textsc{Luca, Florian; Shparlinski, Igor E.} 
\textit{On the exponent of the group of points on elliptic curves in extension fields.} Int. Math. Res. Not. 2005, no. 23, 1391--1409.

\bibitem{kaisa}
\textsc{Matom\"aki, Kaisa}
\textit{A note on primes of the form $p=aq^2+1$.}
Acta Arith. \textbf{137} (2009), 133--137

\bibitem{schoof} 
\textsc{Schoof, Ren\'e}
\textit{The exponents of the groups of points on the reductions of an elliptic curve.}
Arithmetic algebraic geometry (Texel, 1989), 325--335, Progr. Math., \textbf{89}, Birkh\"auser Boston,
Boston, MA, 1991.

\bibitem{washington}
\textsc{Washington, Lawrence C.} 
Elliptic curves. Number theory and cryptography. Second edition. 
\textit{Discrete Mathematics and its Applications (Boca Raton)}. 
Chapman \& Hall/CRC, Boca Raton, FL, 2008.
\end{thebibliography}
\end{document}

