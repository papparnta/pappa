\nopagenumbers \font\title=cmti12
% \def\ve{\vfill\eject}
% \def\vv{\vfill}
% \def\vs{\vskip-2cm}
% \def\vss{\vskip10cm}
% \def\vst{\vskip13.3cm}

\def\ve{\bigskip\bigskip}
\def\vv{\bigskip}
\def\vs{}
\def\vss{}
\def\vst{\bigskip}

\hsize=19.5cm
\vsize=27.58cm
\hoffset=-1.6cm
\voffset=0.5cm
\parskip=-.1cm
\ \vs \hskip -6mm AL310 AA21/22\ (Teoria delle Equazioni)\hfill ESAME
DI MET\`{A} SEMESTRE \hfill Roma, 11 Aprile 2022 \hrule
\bigskip\noindent
{\title COGNOME}\  \dotfill\ {\title NOME}\ \dotfill {\title
MATRICOLA}\ \dotfill\
\smallskip  \noindent
Risolvere il massimo numero di esercizi accompagnando le risposte
con spiegazioni chiare ed essenziali. \it Inserire le risposte
negli spazi predisposti. NON SI ACCETTANO RISPOSTE SCRITTE SU
ALTRI FOGLI.
\rm 1 Esercizio = 4 punti. Tempo previsto: 2 ore. Nessuna domanda
durante la prima ora e durante gli ultimi 20 minuti.
\smallskip
\hrule\smallskip
\centerline{\hskip 6pt\vbox{\tabskip=0pt \offinterlineskip
\def \trl{\noalign{\hrule}}
\halign to277pt{\strut#& \vrule#\tabskip=0.7em plus 1em& \hfil#&
\vrule#& \hfill#\hfil& \vrule#& \hfil#& \vrule#& \hfill#\hfil&
\vrule#& \hfil#& \vrule#& \hfill#\hfil& \vrule#& \hfil#& \vrule#&
\hfill#\hfil& \vrule#& \hfil#& \vrule#& \hfill#\hfil& \vrule#&
\hfil#& \vrule#& \hfill#\hfil& \vrule#& \hfil#& \vrule#& \hfil#&
\vrule#\tabskip=0pt\cr\trl && FIRMA && 1 && 2 && 3 && 4 &&
5 && 6 && 7 && 8  &&  TOT. &\cr\trl && &&   &&
&&     &&   &&     &&   &&   &&    && &\cr &&
\dotfill &&       &&   &&   &&     &&   && && && &&
&\cr\trl }}}
\medskip

\item{1.} Rispondere alle seguenti domande fornendo una giustificazione di una riga:
%\bigskip\bigskip
\itemitem{a.} E' vero che se $E/F$ \`e un estensione e  se $\alpha\in E$  \`e trascendente su $F$ allora  $\forall m\in{\bf Z}, m\neq0$, $\alpha^m$ \`e trascendente su $F$?\medskip
% \bigskip\bigskip\bigskip

% \ \dotfill\ \bigskip\bigskip\vfil

\itemitem{b.} E' vero che se $E_1$ e $E_2$ sono sottocampi di ${\bf C}$, entrambi di dimensione finita su ${\bf Q}$ , allora il campo composto $E_1E_2$ \`e di dimensione finita su ${\bf Q}$
?\medskip
% \bigskip\bigskip\bigskip

% \ \dotfill\ \bigskip\bigskip\vfil

\itemitem{c.} Determinare il grado del campo ${\bf Q}(2^{1/2},2^{1/3},2^{1/4},\cdots,2^{1/15})$ su ${\bf Q}$.\medskip

% \bigskip\bigskip\bigskip
 
% \ \dotfill\ \bigskip\bigskip\vfil

\itemitem{d.} E' vero che ${\bf Q}(\pi)$ \`e isomorfo a ${\bf Q}(e)$ (dove $e$ \`e il numero di Nepero)?

 \medskip
 
 %\bigskip\bigskip\bigskip

% \ \dotfill\ \bigskip\bigskip


% \vfil\eject

\item{2.} Sia $F[\alpha]/F$ un estensione algebrica semplice. Dimostrare che se $a,b,c,d\in F$ sono tali che $ad-bc\neq0$, allora $F[\alpha]=F[(a\alpha+b)/(c\alpha+d)].$\vv


\item{3.} Dopo aver dimostrato che $2\cos(2\pi/15)$ \`e un numero algebrico, se ne calcoli il polinomio minimo su ${\bf Q}$.


\ve\ \vs

%Dopo aver verificato che \`e algebrico, calcolare
%il polinomio minimo di $\cos \pi/9$ su ${\bf Q}$.

\item{4.} Dopo aver descritto tutti gli elementi di Aut(${\bf Q}(7^{1/4},i)/{\bf Q})$, si determini l'ordine di ciascuno di essi.\vv

\item{5.} Determinare il campo di spezzamento su ${\bf Q}$ di $f(X)=(X^4-3)(X^3-3)((X-3)^2-3)\in{\bf Q}[X]$ e calcolarne il grado su ${\bf Q}$.
\ve\ \vs

%--\item{6.} Descrivere la nozione di campo perfetto dimostrando che i campi finiti
%sono perfetti.

\item{6.} Dopo aver definito la nozione di campo perfetto, si forniscano esempi di campi perfetti e di campi non perfetti.\vv\vv

\item{7.} Dopo aver mostrato che $x^3-2x-2\in{\bf Q}[x]$ \`e irriducibile,
si consideri il campo ${\bf Q}[\theta], \theta^3=2\theta+2$. 

\itemitem{a.} Determininare $a,b,c\in{\bf Q}$ tali che $\theta^{-5}=a+b\theta+c\theta^2;$ 

\itemitem{b.} Calcolare il polinomio minimo su ${\bf Q}$ di $\theta^2$. \vv\vv

\item{8.} 
\itemitem{a.} verificare che ${\bf Q}(\sqrt{3})\subset{\bf Q}(\sqrt{15},\sqrt5)$; \itemitem{b.} descrivere i ${\bf Q}(\sqrt{3})$--omomorfismi del campo 
${\bf Q}(\sqrt{15},\sqrt5)$ in ${\bf C}$;
\itemitem{c.} calcolare il polinomio minimo di $\sqrt{3}+\sqrt{5}$ su ${\bf Q}[\sqrt{15}]$.

\vv

\ \vst
 \bye
