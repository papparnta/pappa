\nopagenumbers \font\title=cmti12
\def\ve{\vfill\eject}
\def\vv{\vfill}
\def\vs{\vskip-2cm}
\def\vss{\vskip10cm}
\def\vst{\vskip13.3cm}

%\def\ve{\bigskip\bigskip}
%\def\vv{\bigskip\bigskip}
%\def\vs{}
%\def\vss{}
%\def\vst{\bigskip\bigskip}

\hsize=19.5cm
\vsize=27.58cm
\hoffset=-1.6cm
\voffset=0.5cm
\parskip=-.1cm
\ \vs \hskip -6mm TE1 AA06/07\ (Teoria delle Equazioni)\hfill APPELLO B \hfill Roma, 20 Luglio 2007. \hrule
\bigskip\noindent
{\title COGNOME}\  \dotfill\ {\title NOME}\ \dotfill {\title
MATRICOLA}\ \dotfill\
\smallskip  \noindent
Risolvere il massimo numero di esercizi accompagnando le risposte
con spiegazioni chiare ed essenziali. \it Inserire le risposte
negli spazi predisposti. NON SI ACCETTANO RISPOSTE SCRITTE SU
ALTRI FOGLI. Scrivere il proprio nome anche nell'ultima pagina.
\rm 1 Esercizio = 4 punti. Tempo previsto: 2 ore. Nessuna domanda
durante la prima ora e durante gli ultimi 20 minuti.
\smallskip
\hrule\smallskip
\centerline{\hskip 6pt\vbox{\tabskip=0pt \offinterlineskip
\def \trl{\noalign{\hrule}}
\halign to300pt{\strut#& \vrule#\tabskip=0.7em plus 1em& \hfil#&
\vrule#& \hfill#\hfil& \vrule#& \hfil#& \vrule#& \hfill#\hfil&
\vrule#& \hfil#& \vrule#& \hfill#\hfil& \vrule#& \hfil#& \vrule#&
\hfill#\hfil& \vrule#& \hfil#& \vrule#& \hfill#\hfil& \vrule#&
\hfil#& \vrule#& \hfill#\hfil& \vrule#& \hfil#& \vrule#& \hfil#&
\vrule#\tabskip=0pt\cr\trl && FIRMA && 1 && 2 && 3 && 4 &&
5 && 6 && 7 && 8 && 9 &&  TOT. &\cr\trl && &&   &&
&&     &&   &&   &&   &&   &&   &&    && &\cr &&
\dotfill &&     &&   &&   &&   &&     &&   && && && &&
&\cr\trl }}}
\medskip

\item{1.} Calcolare il numero di elementi del gruppo di Galois su ${\bf Q}$ del polinomio
$x^6-2$.

\vv\item{2.} Calcolare il polinomio minimo su ${\bf Q}$, di
$\big(2-\cos(\pi/4)\big)^{1/2}$.%
\ve\ \vs

\item{3.} Dimostrare che un estensione di campi finita \`{e} necessariamente  algebrica.

\vv\item{4.} Costruire un'estensione $F$ di Galois di ${\bf Q}$ tale
che Gal$(F/{\bf Q})\simeq C_7\times C_{14}$ spiegando la teoria usata. \vv

\item{5.} Dimostrare che per ogni intero positivo $n$, esiste un polinomio a coefficienti interi
che ammette $S_n$ come gruppo di Galois.
\ve\ \vs

\item{6.} Dimostrare che per ogni $q=p^n$ esiste un unico campo finito con $q$ elementi.

\vv \item{7.} Si enunci nella completa generalit\`a il Teorema di
corrispondenza di Galois.

\ve\ \vs

\item{8.} Si dimosri che $\cos(\pi/24)$ \`{e} costruibile fornendo esplicitamente una costruzione.

\vv

\item{9.} Dopo aver calcolato il numero di polinomi irriducibili di grado $3$ su
${\bf F}_3$, illustrare  un esempio di campo finito ${\bf F}_{27}$ con $27$
elementi determinando tutti i generatori del gruppo moltiplicativo
${\bf F}_{27}^*$.%%%%

\ \vst
 \bye
