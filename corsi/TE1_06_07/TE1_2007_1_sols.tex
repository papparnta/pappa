\nopagenumbers \font\title=cmti12
\centerline{TE1 AA06/07\ (Teoria delle Equazioni)}
\hrule
\centerline{\bf SOLUZIONE DELL'ESAME
DI MET\`{A} SEMESTRE DEL  5/4/07}\hrule
\medskip

\item{1.} Dimostrare che un estensione finita \`{e} necessariamente algebrica. Produrre
un esempio di un estensione algebrica non finita.
\smallskip
\item{\bf RISPOSTA:}\it
Vedi note Milne. Prima parte della Proposizione 1.30 a pagina 11 (Erano richieste solo le prime due righe della dimostrazione e non l'intero enunciato). Per quanto riguarda l'esempio, basta considerare $\overline{\bf Q}/{\bf Q}$.
\rm\bigskip

\item{2.} Descrivere gli elementi del gruppo di Galois del polinomio $(x^5-2)\in{\bf Q}[x]$ determinando anche \it alcuni \rm sottocampi
 del campo di spezzamento.
\smallskip
\item{\bf RISPOSTA:}\it Il campo di spezzamento del polinomio \`{e} ${\bf Q}(\zeta_5,2^{1/5})$. Pertanto il gruppo di Galois
ha $[{\bf Q}(\zeta_5,2^{1/5}):{\bf Q}]=20$ elementi. Inoltre
$${\rm Gal}({\bf Q}(\zeta_5,2^{1/5})/{\bf Q})=\left\{\left.{2^{1/5}\mapsto\zeta^i2^{1/5}\atop \zeta\mapsto\zeta^j}\right| i=0,\ldots,4, j=1,\ldots,4\right\}.$$
Per quanto riguarda i sottocampi, alcuni dei sottocampi sono:
$${\bf Q}(\zeta_5,2^{1/5}),
{\bf Q}(2^{1/5}), {\bf Q}(\zeta_52^{1/5}), {\bf Q}(\zeta_5^22^{1/5}),{\bf Q}(\zeta_5^32^{1/5}), {\bf Q}(\zeta_5^42^{1/5}),
{\bf Q}(\sqrt{5}2^{1/5}),
 {\bf Q}(\zeta_5), {\bf Q}(\sqrt{5})\ {\rm e }\ {\bf Q}$$
 ce ne sono altri?
\rm\bigskip

\item{3.} Dopo aver verificato che \`e algebrico, calcolare
il polinomio minimo di $\cos \pi/9$ su ${\bf Q}$.
\smallskip
\item{\bf RISPOSTA:}\it $\cos \pi/9={1\over2}(\zeta^2_{18}+\zeta^{-2}_{18})$ \`{e} algebrico in
quanto combinazione lineare a coefficienti razionali di radici dell'unit\`{a}. Inoltre $f_{\cos \pi/9}=X^3-{3\over2}X-{1\over8}$\rm.\bigskip

\item{4.} Si consideri $E={\bf F}_3[\alpha]$ dove $\alpha$ \`{e}
una radice del polinomio $X^2+1$. Determinare il polinomio minimo
su ${\bf F}_3$ di $1/(\alpha+2)$.
\smallskip
\item{\bf RISPOSTA:}\it Notare che $1/(\alpha+2)=\alpha+1$ e che $(\alpha+1)^2=2\alpha$. Pertanto il
polinomio minimo \`{e} $X^2+X+2\in{\bf F}_3[X].$\rm\bigskip

\item{5.} Descrivere il reticolo dei sottocampi di ${\bf Q}(\zeta_{11})$.
\smallskip
\item{\bf RISPOSTA:}\it Il gruppo di Galois Gal$({\bf Q}(\zeta_{11})/{\bf Q})$ \`{e} ciclico e ha 10 elementi.
I suoi sottogruppi sono tutti ciclici e hanno rispettivamente $10$, $5$, $2$ e $1$ elemento. I sottocampi
corrispondenti secondo la corrispondenza di Galois sono: ${\bf Q}$, ${\bf Q}(\cos2\pi/11))$, ${\bf Q}(\sqrt{-11})$
e ${\bf Q}(\zeta_{11})$. Si noti inoltre che $X^5+X^4-4X^3-3X^2+3X+1$ \`{e} il polinomio minimo di $2\cos2\pi/11$.\rm\bigskip

\item{6.} Descrivere la nozione di campo perfetto dimostrando che i campi finiti
sono perfetti.
\smallskip
\item{\bf RISPOSTA:}\it Vedi note Milne. Definizione 2.13 a pagina 23 e Esempio 2.16 a pagina 24.\rm\bigskip

\item{7.} Si enunci nella completa generalit\`a il Teorema di
corrispondenza di Galois.
\smallskip
\item{\bf RISPOSTA:}\it Vedi note Milne. Teorema 3.16 a pagina 29.\rm\bigskip

\item{8.} Produrre un esempio di un polinomio di grado $3$ il cui gruppo di Galois ha tre elementi giustificando la risposta.
\smallskip
\item{\bf RISPOSTA:}\it Il polinomio dell'esercizio 3 \`{e} un esempio. Altrimenti $X^3+X^2-2X-1$ che il polinomio minimo di $2\cos2\pi/7$ va anche bene.\rm\bigskip

\item{9.} Calcolare il polinomio minimo di $\zeta_{16}\in{\bf Q}(\zeta_{16})$ su ${\bf Q}(\sqrt{-1})$.
\smallskip
\item{\bf RISPOSTA:}\it $X^2-\sqrt{-1}\in{\bf Q}(\sqrt{-1})[X]$.\rm
\bye
