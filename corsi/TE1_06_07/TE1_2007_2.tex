\nopagenumbers \font\title=cmti12
\def\ve{\vfill\eject}
\def\vv{\vfill}
\def\vs{\vskip-2cm}
\def\vss{\vskip10cm}
\def\vst{\vskip13.3cm}

%\def\ve{\bigskip\bigskip}
%\def\vv{\bigskip\bigskip}
%\def\vs{}
%\def\vss{}
%\def\vst{\bigskip\bigskip}

\hsize=19.5cm
\vsize=27.58cm
\hoffset=-1.6cm
\voffset=0.5cm
\parskip=-.1cm
\ \vs \hskip -6mm TE1 AA06/07\ (Teoria delle Equazioni)\hfill ESAME
DI FINE SEMESTRE \hfill Roma, 6 Giugno  2007. \hrule
\bigskip\noindent
{\title COGNOME}\  \dotfill\ {\title NOME}\ \dotfill {\title
MATRICOLA}\ \dotfill\
\smallskip  \noindent
Risolvere il massimo numero di esercizi accompagnando le risposte
con spiegazioni chiare ed essenziali. \it Inserire le risposte
negli spazi predisposti. NON SI ACCETTANO RISPOSTE SCRITTE SU
ALTRI FOGLI. Scrivere il proprio nome anche nell'ultima pagina.
\rm 1 Esercizio = 4 punti. Tempo previsto: 2 ore. Nessuna domanda
durante la prima ora e durante gli ultimi 20 minuti.
\smallskip
\hrule\smallskip
\centerline{\hskip 6pt\vbox{\tabskip=0pt \offinterlineskip
\def \trl{\noalign{\hrule}}
\halign to300pt{\strut#& \vrule#\tabskip=0.7em plus 1em& \hfil#&
\vrule#& \hfill#\hfil& \vrule#& \hfil#& \vrule#& \hfill#\hfil&
\vrule#& \hfil#& \vrule#& \hfill#\hfil& \vrule#& \hfil#& \vrule#&
\hfill#\hfil& \vrule#& \hfil#& \vrule#& \hfill#\hfil& \vrule#&
\hfil#& \vrule#& \hfill#\hfil& \vrule#& \hfil#& \vrule#& \hfil#&
\vrule#\tabskip=0pt\cr\trl && FIRMA && 1 && 2 && 3 && 4 &&
5 && 6 && 7 && 8 && 9 &&  TOT. &\cr\trl && &&   &&
&&     &&   &&   &&   &&   &&   &&    && &\cr &&
\dotfill &&     &&   &&   &&   &&     &&   && && && &&
&\cr\trl }}}
\medskip

\item{1.} Si calcoli il gruppo di Galois del polinomio $x^4 + 10x^2 - 4x + 2\in{\bf Q}[x]$.

\vv\item{2.} Determinare tutti i sottocampi del campo ${\bf Q}(\zeta_{17})$.

\ve\ \vs

\item{3.} Descrivere la chiusura algebrica di ${\bf F}_{7}$ giustificando la risposta.

\vv\item{4.} Dopo aver dimostrato che $\cos(\pi/8)$ \`{e} costruibile, se ne determini esplicitamente
una costruzione. \vv

\item{5.} Determinare almeno due valori distinti di $M$ tali che ${\bf Q}(\zeta_{M})$ contiene un
sottocampo con gruppo di Galois su ${\bf Q}$ isomorfo a $C_6\times C_{12}$.
\ve\ \vs

\item{6.} Dimostrare giustificando la risposta che se $p$ \`{e} primo allora $(x^{p^5}-x)/(x^p-x)\in{\bf F}_p[x]$ \`{e} il prodotto di tutti
i polinomi irriducibili su ${\bf F}_p$ di grado $5$.

\vv \item{7.} Si enunci nella completa generalit\`a il Teorema di
corrispondenza di Galois.

\ve\ \vs


\item{8.} Dimostrare che se $f$ \`{e} un polinomio a coefficienti razionali senza fattori multipli di grado $n$,
allora $G_f\subset A_n$ se e solo se il discriminante di $f$ \`{e} un quadrato perfetto.

\vv

\item{9.} Calcolare il numero di elementi del campo di spezzamento del polinomio $(x^{2^8}-x)(x^8+x^4+1)(x^{12}+x^4+1)(x^5+x)\in{\bf F}_2[x]$.

\ \vst
 \bye
