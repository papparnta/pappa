\nopagenumbers \font\title=cmti12
\def\ve{\vfill\eject}
\def\vv{\vfill}
\def\vs{\vskip-2cm}
\def\vss{\vskip10cm}
\def\vst{\vskip13.3cm}

%\def\ve{\bigskip\bigskip}
%\def\vv{\bigskip\bigskip}
%\def\vs{}
%\def\vss{}
%\def\vst{\bigskip\bigskip}

\hsize=19.5cm
\vsize=27.58cm
\hoffset=-1.6cm
\voffset=0.5cm
\parskip=-.1cm
\ \vs \hskip -6mm TE1 AA06/07\ (Teoria delle Equazioni)\hfill APPELLO A \hfill Roma, 12 Giugno  2007. \hrule
\bigskip\noindent
{\title COGNOME}\  \dotfill\ {\title NOME}\ \dotfill {\title
MATRICOLA}\ \dotfill\
\smallskip  \noindent
Risolvere il massimo numero di esercizi accompagnando le risposte
con spiegazioni chiare ed essenziali. \it Inserire le risposte
negli spazi predisposti. NON SI ACCETTANO RISPOSTE SCRITTE SU
ALTRI FOGLI. Scrivere il proprio nome anche nell'ultima pagina.
\rm 1 Esercizio = 4 punti. Tempo previsto: 2 ore. Nessuna domanda
durante la prima ora e durante gli ultimi 20 minuti.
\smallskip
\hrule\smallskip
\centerline{\hskip 6pt\vbox{\tabskip=0pt \offinterlineskip
\def \trl{\noalign{\hrule}}
\halign to300pt{\strut#& \vrule#\tabskip=0.7em plus 1em& \hfil#&
\vrule#& \hfill#\hfil& \vrule#& \hfil#& \vrule#& \hfill#\hfil&
\vrule#& \hfil#& \vrule#& \hfill#\hfil& \vrule#& \hfil#& \vrule#&
\hfill#\hfil& \vrule#& \hfil#& \vrule#& \hfill#\hfil& \vrule#&
\hfil#& \vrule#& \hfill#\hfil& \vrule#& \hfil#& \vrule#& \hfil#&
\vrule#\tabskip=0pt\cr\trl && FIRMA && 1 && 2 && 3 && 4 &&
5 && 6 && 7 && 8 && 9 &&  TOT. &\cr\trl && &&   &&
&&     &&   &&   &&   &&   &&   &&    && &\cr &&
\dotfill &&     &&   &&   &&   &&     &&   && && && &&
&\cr\trl }}}
\medskip

\item{1.} Si descriva il campo di spezzamento e gli elementi del gruppo di Galois (specificando
il numero di elementi) del polinomio $(x^3-5)(x^2+1)(x^2-3)\in{\bf Q}[x]$.

\vv\item{2.} Si fornisca un esempio di campo ciclotomico che ammette almeno $5$ sottocampi quadratici.

\ve\ \vs

\item{3.} Sia $\Omega/F$ un estensione di campi. Dimostrare che l'insieme degli elementi di $\Omega$ che sono algebrici su $F$ \`{e}
un campo.

\vv\item{4.} Si fornisca una costruzione (in senso algebrico) del decagono regolare. \vv

\item{5.} Dopo aver descritto la nozione di campo perfetto e averne elencato alcune propriet\`{a}, si dia un esempio di campo imperfetto.
\ve\ \vs

\item{6.} Dimostrare che il gruppo di Galois di un campo finito \`{e} sempre ciclico.

\vv \item{7.} Si enunci nella completa generalit\`a il Teorema di
corrispondenza di Galois.

\ve\ \vs


\item{8.} Dimostrare la formula per calcolare il discriminante di $X^3+aX+b$.

\vv

\item{9.} Calcolare il numero di polinomi quadratici irriducibili su ${\bf F}_5$ e dopo averne scelti due distinti, si scriva un
isomorfismo tra i rispettivi campi a gambo.

\ \vst
 \bye
