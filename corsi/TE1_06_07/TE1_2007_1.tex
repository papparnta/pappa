\nopagenumbers \font\title=cmti12
%\def\ve{\vfill\eject}
%\def\vv{\vfill}
%\def\vs{\vskip-2cm}
%\def\vss{\vskip10cm}
%\def\vst{\vskip13.3cm}

\def\ve{\bigskip\bigskip}
\def\vv{\bigskip\bigskip}
\def\vs{}
\def\vss{}
\def\vst{\bigskip\bigskip}

\hsize=19.5cm
\vsize=27.58cm
\hoffset=-1.6cm
\voffset=0.5cm
\parskip=-.1cm
\ \vs \hskip -6mm TE1 AA06/07\ (Teoria delle Equazioni)\hfill ESAME
DI MET\`{A} SEMESTRE \hfill Roma, 5 Aprile 2007. \hrule
\bigskip\noindent
{\title COGNOME}\  \dotfill\ {\title NOME}\ \dotfill {\title
MATRICOLA}\ \dotfill\
\smallskip  \noindent
Risolvere il massimo numero di esercizi accompagnando le risposte
con spiegazioni chiare ed essenziali. \it Inserire le risposte
negli spazi predisposti. NON SI ACCETTANO RISPOSTE SCRITTE SU
ALTRI FOGLI. Scrivere il proprio nome anche nell'ultima pagina.
\rm 1 Esercizio = 4 punti. Tempo previsto: 2 ore. Nessuna domanda
durante la prima ora e durante gli ultimi 20 minuti.
\smallskip
\hrule\smallskip
\centerline{\hskip 6pt\vbox{\tabskip=0pt \offinterlineskip
\def \trl{\noalign{\hrule}}
\halign to300pt{\strut#& \vrule#\tabskip=0.7em plus 1em& \hfil#&
\vrule#& \hfill#\hfil& \vrule#& \hfil#& \vrule#& \hfill#\hfil&
\vrule#& \hfil#& \vrule#& \hfill#\hfil& \vrule#& \hfil#& \vrule#&
\hfill#\hfil& \vrule#& \hfil#& \vrule#& \hfill#\hfil& \vrule#&
\hfil#& \vrule#& \hfill#\hfil& \vrule#& \hfil#& \vrule#& \hfil#&
\vrule#\tabskip=0pt\cr\trl && FIRMA && 1 && 2 && 3 && 4 &&
5 && 6 && 7 && 8 && 9 &&  TOT. &\cr\trl && &&   &&
&&     &&   &&   &&   &&   &&   &&    && &\cr &&
\dotfill &&     &&   &&   &&   &&     &&   && && && &&
&\cr\trl }}}
\medskip

\item{1.} Dimostrare che un estensione finita \`{e} necessariamente algebrica. Produrre
un esempio di un estensione algebrica non finita.

\vv\item{2.} Descrivere gli elementi del gruppo di Galois del
polinomio $(x^5-2)\in{\bf Q}[x]$ determinando anche \it alcuni \rm sottocampi
 del campo di spezzamento.

\ve\ \vs

\item{3.} Dopo aver verificato che \`e algebrico, calcolare
il polinomio minimo di $\cos \pi/9$ su ${\bf Q}$.

\vv\item{4.} Si consideri $E={\bf F}_3[\alpha]$ dove $\alpha$ \`{e}
una radice del polinomio $X^2+1$. Determinare il polinomio minimo
su ${\bf F}_3$ di $1/(\alpha+2)$. \vv

\item{5.} Descrivere il reticolo dei sottocampi di ${\bf Q}(\zeta_{11})$.
\ve\ \vs

\item{6.} Descrivere la nozione di campo perfetto dimostrando che i campi finiti
sono perfetti.

\vv \item{7.} Si enunci nella completa generalit\`a il Teorema di
corrispondenza di Galois.

\ve\ \vs


\item{8.} Produrre un esempio di un polinomio di grado $3$ il cui gruppo di Galois ha tre elementi giustificando la risposta.

\vv

\item{9.} Calcolare il polinomio minimo di $\zeta_{16}\in{\bf Q}(\zeta_{16})$ su ${\bf Q}(\sqrt{-1})$.

\ \vst
 \bye
