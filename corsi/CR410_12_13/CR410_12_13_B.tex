\nopagenumbers \font\title=cmti12
\def\ve{\vfill\eject}
\def\vv{\vfill}
\def\vs{\vskip-2cm}
\def\vss{\vskip10cm}
\def\vst{\vskip13.3cm}

% \def\ve{\bigskip\bigskip}
% \def\vv{\bigskip\bigskip}
% \def\vs{}
% \def\vss{}
% \def\vst{\bigskip\bigskip}

\hsize=19.5cm
\vsize=27.58cm
\hoffset=-1.6cm
\voffset=0.5cm
\parskip=-.1cm
\ \vs \hskip -6mm CR410 AA12/13\ (Crittografia 1)\hfill APPELLO A \hfill Roma, 7 GIUGNO 2013. \hrule
\bigskip\noindent
{\title COGNOME}\  \dotfill\ {\title NOME}\ \dotfill {\title
MATRICOLA}\ \dotfill\
\smallskip  \noindent
Risolvere il massimo numero di esercizi accompagnando le risposte
con spiegazioni chiare ed essenziali. \it Inserire le risposte
negli spazi predisposti. NON SI ACCETTANO RISPOSTE SCRITTE SU
ALTRI FOGLI. Scrivere il proprio nome anche nell'ultima pagina.
\rm 1 Esercizio = 3 punti. Tempo previsto: 2 ore. Nessuna domanda
durante la prima ora e durante gli ultimi 20 minuti.
\smallskip
\hrule\smallskip
\centerline{\hskip 6pt\vbox{\tabskip=0pt \offinterlineskip
\def \trl{\noalign{\hrule}}
\halign to560pt{\strut#& \vrule#\tabskip=0.7em plus 2em& \hfil#&
\vrule#& \hfill#\hfil& \vrule#& \hfil#& \vrule#& \hfill#\hfil&
\vrule#& \hfil#& \vrule#& \hfill#\hfil& \vrule#& \hfil#& \vrule#&
\hfill#\hfil& \vrule#& \hfil#& \vrule#& \hfill#\hfil& \vrule#&
\hfil#& \vrule#& \hfill#\hfil& \vrule#& \hfil#& \vrule#& \hfil#&
\vrule#\tabskip=0pt\cr\trl && FIRMA && 1 && 2 && 3 && 4 &&
5 && 6 && 7 && 8 && 9 &&  10 &\cr\trl && &&   &&
&&     &&   &&   &&   &&   &&   &&    && &\cr &&
\dotfill &&     &&   &&   &&   &&     &&   && && && &&
&\cr\trl }}}
\medskip


%\item{1.}
%Se $n\in{\bf N}$, sia $\varphi(n)$ la funzione di Eulero. Supponiamo che sia nota
%la fattorizzazione (unica) di $n=p_1^{\alpha_1}\cdots p_s^{\alpha_s}$. Stimare il
%numero di operazioni bit necessarie per calcolare $\varphi(n)$.
%\vv
%\item{2.} Stimare in termini di $k$ il numero di operazioni bit necessarie per calcolare $\left[\sqrt{2^{k^k}\bmod 3^k}
%\right]$.
%\vv

\item{1.} Si descrivano le complessit\`a delle operazioni elementari tra interi.\vv

\item{2.} Descrivere l'algoritmo dei quadrati successivi in un qualsiasi monoide moltiplicativo discutendone
la complessit\`a. \ve\vs

\item{3.} Dimostrare che il gruppo moltiplicativo di un campo finito \`e ciclico. 
\vv


\item{4.} Dopo aver descritto la nozione di base forte, si dimostri che tutte le basi modulo un primo sono forti e si fornisca
un esempio di un numero composto e di una sua base forte (non banale cio\`e diversa da $-1$).  
\vv


\item{5.} Si descriva e si dimostri il Teorema Cinese dei resti discutendo in particolare l'analisi della complessit\`a
per determinare le soluzioni di un sistema di congruenze.
\ve\vs

\item{6.} Si descriva il reticolo dei sottocampi di ${\bf F}_{2^6}$ e per ciascun sottocampo proprio, si elenchino i polinomi
irriducibili e quelli primitivi.

\vv

\item{7.} Determinare i polinomi minimi e gli ordini degli elementi di ${\bf F}_{9}$.
\vv

\item{8.} Fornite un esempio di curva ellittica definita su un campo con $25$ elementi per cui $E({\bf F}_{25}$ non \`e
ciclico.
\ve \vs

\item{9.} Sia $E : y^2 = x^3 − 5x + 8$ e siano $P = (6, 3),Q = (9, 10) \in E({\bf F}_{101})$. Calcolare $2P$ e $P+Q$.
\vv

\item{10.} Spiegare il funzionamento di tutti i protocolli crittografici incontrati nel corso.
\ \vst

 \bye
