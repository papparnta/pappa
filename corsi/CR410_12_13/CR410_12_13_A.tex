\nopagenumbers \font\title=cmti12
\def\ve{\vfill\eject}
\def\vv{\vfill}
\def\vs{\vskip-2cm}
\def\vss{\vskip10cm}
\def\vst{\vskip13.3cm}

% \def\ve{\bigskip\bigskip}
% \def\vv{\bigskip\bigskip}
% \def\vs{}
% \def\vss{}
% \def\vst{\bigskip\bigskip}

\hsize=19.5cm
\vsize=27.58cm
\hoffset=-1.6cm
\voffset=0.5cm
\parskip=-.1cm
\ \vs \hskip -6mm CR410 AA12/13\ (Crittografia 1)\hfill APPELLO A \hfill Roma, 7 GIUGNO 2013. \hrule
\bigskip\noindent
{\title COGNOME}\  \dotfill\ {\title NOME}\ \dotfill {\title
MATRICOLA}\ \dotfill\
\smallskip  \noindent
Risolvere il massimo numero di esercizi accompagnando le risposte
con spiegazioni chiare ed essenziali. \it Inserire le risposte
negli spazi predisposti. NON SI ACCETTANO RISPOSTE SCRITTE SU
ALTRI FOGLI. Scrivere il proprio nome anche nell'ultima pagina.
\rm 1 Esercizio = 3 punti. Tempo previsto: 2 ore. Nessuna domanda
durante la prima ora e durante gli ultimi 20 minuti.
\smallskip
\hrule\smallskip
\centerline{\hskip 6pt\vbox{\tabskip=0pt \offinterlineskip
\def \trl{\noalign{\hrule}}
\halign to560pt{\strut#& \vrule#\tabskip=0.7em plus 2em& \hfil#&
\vrule#& \hfill#\hfil& \vrule#& \hfil#& \vrule#& \hfill#\hfil&
\vrule#& \hfil#& \vrule#& \hfill#\hfil& \vrule#& \hfil#& \vrule#&
\hfill#\hfil& \vrule#& \hfil#& \vrule#& \hfill#\hfil& \vrule#&
\hfil#& \vrule#& \hfill#\hfil& \vrule#& \hfil#& \vrule#& \hfil#&
\vrule#\tabskip=0pt\cr\trl && FIRMA && 1 && 2 && 3 && 4 &&
5 && 6 && 7 && 8 && 9 &&  10 &\cr\trl && &&   &&
&&     &&   &&   &&   &&   &&   &&    && &\cr &&
\dotfill &&     &&   &&   &&   &&     &&   && && && &&
&\cr\trl }}}
\medskip


%\item{1.}
%Se $n\in{\bf N}$, sia $\varphi(n)$ la funzione di Eulero. Supponiamo che sia nota
%la fattorizzazione (unica) di $n=p_1^{\alpha_1}\cdots p_s^{\alpha_s}$. Stimare il
%numero di operazioni bit necessarie per calcolare $\varphi(n)$.
%\vv
%\item{2.} Stimare in termini di $k$ il numero di operazioni bit necessarie per calcolare $\left[\sqrt{2^{k^k}\bmod 3^k}
%\right]$.
%\vv

\item{1.} Si descriva un algoritmo per calcolare in tempo polinomiale la parte intera di $m^{1/5}$ per ogni intero positivo $m$.\vv

\item{2.} Descrivere l'algoritmo di moltiplicazione di Karatsuba. \ve\vs

\item{3.} Dimostrare che se $p$ \`e primo, allora $x^4\equiv1\bmod p$ ammette $\gcd(p-1,4)$ soluzioni.
Determinare un velore di $m$ tale che $X^4\equiv 1\bmod m$ ammette esattamente $32$ soluzioni. 
\vv


\item{4.} Calcolare il simbolo di Legendre $\left({97543\over 21345}\right)$ utilizzando le propritet\`a dei simboli di Jacobi.  
\vv


\item{5.} Si illustri l'algoritmo di Euclide esteso con particolare riguardo alle relazioni ricorsive per il calcolo dell'identit\`a di
Bezout. Lo si abblichi per calcolare l'identit\`a di Bezout tra $54$ e $98$.
\ve\vs

\item{6.} Si determini la probabilit\`a che un polinomio irriducibile su ${\bf F}_5$ di grado $6$ risulti primitivo.

\vv

\item{7.} Determinare i polinomi minimi e gli ordini degli elementi di ${\bf F}_{16}$.
\vv

\item{8.} Considerare una curva ellittica $E$ definita su un campo con $2^{10}$ elementi. Supponiamo che $P\in E({\bf F}_{2^{10}})$
abbia ordine $7$ e che $Q\in E({\bf F}_{2^{10}})$ abbia ordine $19$. Se sappiamo che $E({\bf F}_{2^{10}})$ non \`e ciclico, cosa possiamo
dire della sua struttura?
\ve \vs

\item{9.} Sia $E: y^2=x^3+x$,  Dimostrare che se $p\equiv1\bmod4$ allora il gruppo $E({\bf F}_p)$ non \`e ciclico.
Determinare tale gruppo nel caso in cui $p=3$.
\vv

\item{10.} Spiegare il funzionamento di tutti i protocolli crittografici incontrati nel corso.
\ \vst

 \bye
