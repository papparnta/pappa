\nopagenumbers \font\title=cmti12
\def\ve{\vfill\eject}
\def\vv{\vfill}
\def\vs{\vskip-2cm}
\def\vss{\vskip10cm}
\def\vst{\vskip13.3cm}

%\def\ve{\bigskip\bigskip}
%\def\vv{\bigskip\bigskip}
%\def\vs{}
%\def\vss{}
%\def\vst{\bigskip\bigskip}

\hsize=19.5cm
\vsize=27.58cm
\hoffset=-1.6cm
\voffset=0.5cm
\parskip=-.1cm
\ \vs \hskip -6mm CR410 AA12/13 (Crittografia a chiave pubblica)\hfill ESAME DI FINE SEMESTRE \hfill Roma, 28 Maggio, 2013. \hrule
\bigskip\noindent
{\title Cognome}\  \dotfill\ {\title Nome}\ \dotfill {\title
Matricola}\ \dotfill\
\smallskip  \noindent
Risolvere il massimo numero di esercizi fornendo spiegazioni chiare e sintetiche. \it Inserire le risposte negli spazi
predisposti. NON SI ACCETTANO RISPOSTE SCRITTE SU ALTRI FOGLI.
\rm 1 Eesrcizio = 4.5 punti. Tempo previsto: 2 ore. Nessuna domanda durante le prima ora e durante gli ultimi 20 minuti.
\smallskip
\hrule\smallskip
\centerline{\hskip 6pt\vbox{\tabskip=0pt \offinterlineskip
\def \trl{\noalign{\hrule}}
\halign to225pt{\strut#& \vrule#\tabskip=0.7em plus 1em& \hfil#&
\vrule#& \hfill#\hfil& \vrule#& \hfil#& \vrule#& \hfill#\hfil&
\vrule#& \hfil#& \vrule#& \hfill#\hfil& \vrule#& \hfil#& \vrule#&
\hfill#\hfil& \vrule#& \hfil#& \vrule#& \hfill#\hfil& \vrule#&
\hfil#& \vrule#& \hfill#\hfil& \vrule#& \hfil#& \vrule#& \hfil#&
\vrule#\tabskip=0pt\cr\trl && 1 && 2 && 3 && 4 &&
5 && 6 && 7 && 8  && TOT. &\cr\trl  &&   &&
&&     &&   &&     &&  &&    &&  && &\cr &&       &&   &&      &&   && && && &&
 && &\cr\trl }}}
\medskip

\item{1.} Rispondere alle seguenti domande che forniscono una giustificazione di 1 riga:\bigskip\bigskip\bigskip


\itemitem{a.} Fornire un esempio di un'equazione di Weierstrass singolare.\medskip\bigskip\bigskip

\ \dotfill\ \bigskip\bigskip\bigskip\vfil

\itemitem{b.} E' vero che in alcuni gruppi ciclici il logaritmo discreto \`e particolarmente
facile da calcolare?\medskip\bigskip\bigskip

\ \dotfill\ \bigskip\bigskip\bigskip\vfil

\itemitem{c.} Fornire due esempi di campi finiti ${\bf F}_q$ in cui tutti gli elementi di ${\bf F}_q^*\setminus\{1\}$
sono generatori.\medskip\bigskip\bigskip
 
\ \dotfill\ \bigskip\bigskip\bigskip\vfil

\itemitem{d.} Fornire un esempio di un polinomio primitivo in un campo con $9$ elementi.\medskip\bigskip\bigskip

\ \dotfill\ \bigskip\bigskip\bigskip


\vfil\eject


\item{2.} Enunciare e dimostrare il Teorema di struttura dei sottocampi di ${\bf F}_{p^n}$. Lo si utilizzi per costruire
un esempio di campo finito con esattamente $5$ sottocampi. \vv


\item{3.} Supponiamo che $n,m$ siano interi, che $m\equiv 5\bmod 4n$, che $n\equiv7\bmod10$. Calcolare il simbolo di
Jacobi $\left({n\atop m}\right)$.\ve\ \vs



\item{4.} Spiegare il funzionamento di alcuni sistemi crittografici che basano la propria sicurezza sul problema del 
logaritmo discreto.\vv


\item{5.} Spiegare la rilevanza del metodo Baby-Steps-Giant-Steps nella teoria delle curve ellittiche su campi finiti.
\ve\ \vs

\item{6.} Sia $E: y^2=x^3-x$. Determinare la struttura del gruppo $E({\bf F}_5)$  e calcolare $\#E({\bf F}_{125})$. E' possibile
determinare anche la struttura di $E({\bf F}_{125})$?\vskip8cm

\item{7.} Dimostrare che se $E$ \`e una curva ellittica definita su un campo finito ${\bf F}_q$
con caratteristica dispari da un'equazione $y^2=x^3+a_2x^2+a_4x+a_6$, allora i punti di ordine $2$ hanno la forma $(\alpha,0)$
dove $\alpha^3+a_2\alpha^2+a_4\alpha+a_6=0$. Si forniscano esempi di curve ellittiche con $0$, $1$ e $3$ punti di ordine $2$ e
si spieghi perc\`e non \`e possibilie che ve ne siano $2$.\vskip8cm


%Prima calcoliamo $\#E({\bf F}_4)$. Se ${\bf F}_4=\{0,1,\xi,1+\xi\}$ allora per $x\in{\bf F}_4$, il valore di $x^3+\xi$ \`e 
%$\xi, 1+\xi, 1+\xi, 1+\xi$ rispettivamente. Al variare di $y\in{\bf F}_4$, il valore di $y^2+y$ \`e 
%$0, 0, 1, 1$ rispettivamente. Pertanto l'unico punto in $E({\bf F}_4)$ \`e $\infty$ e $\#E({\bf F}_4)=4+1-4=1$ . 
%
%Per calcolare $\#E({\bf F}_{4^3})$ usiamo le formule ricorsive $s_0=2, s_1=4, s_{n+1}=4s_{n}-4s_{n-1}$. Otteniamo che
%$s_2= 16-8=8$ e $s_3= 8\cdot 4 -4 *4 =16$. Pertanto  $\#E({\bf F}_{4^3})=4^3+1-16=48.$

\item{8.} Scrivere e dimostrare le formula per l'inverso $-P$ e per il punto $2P$ del punto $P(x,y)\in E({\bf F}_q)$ dove
$E$ \`e una curva ellittica definita da una equazione di Weierstrass generale.

\ \vst
 \bye
