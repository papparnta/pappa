\nopagenumbers \font\title=cmti12
%\def\ve{\vfill\eject}
%\def\vv{\vfill}
%\def\vs{\vskip-2cm}
%\def\vss{\vskip10cm}
%\def\vst{\vskip13.3cm}

\def\ve{\bigskip\bigskip}
\def\vv{\bigskip\bigskip}
\def\vs{}
\def\vss{}
\def\vst{\bigskip\bigskip}

\hsize=19.5cm
\vsize=27.58cm
\hoffset=-1.6cm
\voffset=0.5cm
\parskip=-.1cm
\ \vs \hskip -6mm CR410 AA11/12\ (Crittografia 1)\hfill ESAME DI MET\`{A} SEMESTRE \hfill Roma, 3 Aprile 2012. \hrule
\bigskip\noindent
{\title COGNOME}\  \dotfill\ {\title NOME}\ \dotfill {\title
MATRICOLA}\ \dotfill\
\smallskip  \noindent
Risolvere il massimo numero di esercizi accompagnando le risposte
con spiegazioni chiare ed essenziali. \it Inserire le risposte
negli spazi predisposti. NON SI ACCETTANO RISPOSTE SCRITTE SU
ALTRI FOGLI. Scrivere il proprio nome anche nell'ultima pagina.
\rm 1 Esercizio = 3 punti. Tempo previsto: 2 ore. Nessuna domanda
durante la prima ora e durante gli ultimi 20 minuti.
\smallskip
\hrule\smallskip
\centerline{\hskip 6pt\vbox{\tabskip=0pt \offinterlineskip
\def \trl{\noalign{\hrule}}
\halign to300pt{\strut#& \vrule#\tabskip=0.7em plus 1em& \hfil#&
\vrule#& \hfill#\hfil& \vrule#& \hfil#& \vrule#& \hfill#\hfil&
\vrule#& \hfil#& \vrule#& \hfill#\hfil& \vrule#& \hfil#& \vrule#&
\hfill#\hfil& \vrule#& \hfil#& \vrule#& \hfill#\hfil& \vrule#&
\hfil#& \vrule#& \hfill#\hfil& \vrule#& \hfil#& \vrule#& \hfil#&
\vrule#\tabskip=0pt\cr\trl && FIRMA && 1 && 2 && 3 && 4 &&
5 && 6 && 7 && 8 && 9 &&  TOT. &\cr\trl && &&   &&
&&     &&   &&   &&   &&   &&   &&    && &\cr &&
\dotfill &&     &&   &&   &&   &&     &&   && && && &&
&\cr\trl }}}
\medskip


%\item{1.}
%Se $n\in{\bf N}$, sia $\varphi(n)$ la funzione di Eulero. Supponiamo che sia nota
%la fattorizzazione (unica) di $n=p_1^{\alpha_1}\cdots p_s^{\alpha_s}$. Stimare il
%numero di operazioni bit necessarie per calcolare $\varphi(n)$.
%\vv
%\item{2.} Stimare in termini di $k$ il numero di operazioni bit necessarie per calcolare $\left[\sqrt{2^{k^k}\bmod 3^k}
%\right]$.
%\vv

\item{1.} Dato il numero binario
$n=(101010110)_2$, calcolare $[{\sqrt{n}]$ usando l'algoritmo
delle approssimazioni successive (Non passare a base 10 e  non
usare la calcolatrice!) \vv

\item{2.} Determinare una stima per il numero di operazioni bit necessarie per calcolare $[\sqrt{a}]^{b^a}\bmod b}}$ dove
$b\leq a^a$. \ve\vs

\item{3.} Trovare le soluzioni $X\in{\bf Z}$ della congruenza $X^3\equiv 1\bmod 91$?
\vv


\item{4.} Mostrare che se $f(X)=aX^2+bX+c\in{\bf Z}/k{\bf Z}[X]$, le moltiplicazioni nell'anello quoziente ${\bf Z}/k{\bf Z}[x]/(f(X))$ 
si possono calcolare in $O(\log^2 k)$ operazioni bit. Vale la stessa conclusione se $\deg f>2$?
\vv


\item{5.} Si illustri il funzionamento dell'algoritmo di Stein (algoritmo binario) per calcolare il massimo comune divisore
di $72$ e $90$.
\ve\vs

\item{6.} Supponiamo $a, m\in{\bf Z}$, e $(a,m)=1$. Dimostrare che l'inverso moltiplicativo $a^*(\bmod m)$ \`e una potenza 
di $a$. Spiegare perch\`e se $m$ ha al pi\`u due fattori primi allora conoscere tale potenza \`e computazionalmente equivalente
a fattorizzare $m$.

\vv

\item{7.} Dopo aver enunciato il criterio di Korselt per i numeri di Carmichael lo si applichi per mostrare che
$2821=7\times 13\times 31$ \`e un numero di Carmichael.
\vv

\item{8.} Quale è la probabilità che un numero minore di $100$ coprimo con $14$ risulti primo?
\ve \vs

\item{9.} Calcolare la successione di Miller Rabin di $3$ modulo $49$.
\vv

\item{10.} Spiegare nei dettagli il funzionamento del crittosistema RSA.
\ \vst

 \bye
