\documentclass[a4paper,11pt]{article}
\usepackage{times}
\usepackage[latin1]{inputenc}
\usepackage[T1]{fontenc}
\usepackage[italian]{babel}
\usepackage{amssymb}
\usepackage{amsmath}
\usepackage{amsthm}
%\nopagenumbers
\def\Q{{\mathbb Q}}
\def\Z{{\mathbb Z}}
\def\R{{\mathbb R}}
\def\F{{\mathbb F}}
\def\C{{\mathbb C}}
\def\QQ{{\rm Q}}

\title{Teoria di Galois 1 - Tutorato V}
\author{\Large{Esercizi di ricapitolazione}}
\date{Venerd� 19 Maggio 2005}

\begin{document}
\maketitle

\theoremstyle{definition}

\newtheorem{es}{Esercizio}





\begin{es}
In ciascuno dei seguenti casi si calcoli il numero di elementi nel campo di spezzamento del polinomio
\item{a.} $(x^4+x+1)(x^4+x^3+x+1)(x^3+x+1)(x^2+x+1)   \in\F_2[x]$;
\item{b.} $ (x^3+x+1) (x^3+x^2+1)  (x^9+x^6+1) (x^{27}+x^9+1)    \in \F_3[x];$
\item{c.}  $(x^3+x^2+1)(x^3+x+1)  (x^{15}+x^{10}+1)  (x^{25}+x^3+1)       \in\F_5[x]$;
\end{es}
\medskip


\begin{es}
Calcolare quanti sono i polinomi irriducibili si grado 8 su  $\F_2[x]$ e quanti sono
quelli monici ed irriducibili di grado 6 su $\F_7[x]$.
\end{es}
\medskip


\begin{es}
Quali sono le radici di  $ x^{16}+x^{12}+1 $  in $ \, \F_2[\alpha] \, $ dove $\, \alpha^4=\alpha+1$.
\hfill({\it provare con $\alpha^3+1$})
\end{es}
\medskip


\begin{es}
Determinare il Gruppo di Galois del polinomio $ x^4+8x^2+2 $.
\end{es}
\medskip


\begin{es}
Sia $F$ un campo con 16 elementi. Determinare quante radici hanno in $F$ ciascuno dei seguenti polinomi:
$ x^3-1 ; \, \,   x^4-1 ; \, \,     x^{15}-1 ; \, \, x^{17}-1 . $
\end{es}
\medskip


\begin{es}
Trovare un elemento primitivo per il campo $ \, \Q(\sqrt{7},\sqrt{3}) $.
\end{es}
\medskip


\begin{es}
Sia $E$ il campo di spezzamento di   $(x^2 - 2) (x^2 - 5) (x^2 - 7) $ su $\Q$;
trovare un elemento $\alpha \in E$ tale che $E=\Q[\alpha]$.
\end{es}
\medskip




\begin{es}
Trovare il gruppo di Galois del polinomio $x^6-5$ sia su $\Q$ che su $\R$.
\end{es}
\medskip



\begin{es}
Sia $G$ il gruppo di Galois del polinomio    $(x^4 - 2) (x^3 - 5) $  su  $\Q$:
\item{a.} indicare un insieme di generatori per $G$;
\item{b.} individuare la struttura  di $G$ come gruppo astratto.
\end{es}
\medskip


\begin{es}
Quanti campi sono strettamente contenuti tra  $\Q[\zeta_{12}] $  e  $\Q[\zeta_{12}^3] $ ?
\end{es}
\medskip


\begin{es}
Determinare il gruppo di Galois del campo di spezzamento $K$ del
polinomio $x^4-3$ su $\Q$ e il numero dei sottocampi quadratici contenuti in esso.
\end{es}
\medskip


\begin{es}
Descrivere il gruppo di Galois del polinomio $x^6-7$ su $\Q$.
\end{es}
\medskip


\clearpage
\nocite{*}
\end{document}






