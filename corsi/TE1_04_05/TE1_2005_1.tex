\nopagenumbers \font\title=cmti12
%\def\ve{\vfill\eject}
%\def\vv{\vfill}
%\def\vs{\vskip-2cm}
%\def\vss{\vskip10cm}
%\def\vst{\vskip13.3cm}

\def\ve{\bigskip\bigskip}
\def\vv{\bigskip\bigskip}
\def\vs{}
\def\vss{}
\def\vst{\bigskip\bigskip}

\hsize=19.5cm
\vsize=27.58cm
\hoffset=-1.6cm
\voffset=0.5cm
\parskip=-.1cm
\ \vs \hskip -6mm TE1 AA04/05\ (Teoria delle Equazioni)\hfill ESAME
DI MET\`{A} SEMESTRE \hfill Roma, 12 Aprile 2005. \hrule
\bigskip\noindent
{\title COGNOME}\  \dotfill\ {\title NOME}\ \dotfill {\title
MATRICOLA}\ \dotfill\
\smallskip  \noindent
Risolvere il massimo numero di esercizi accompagnando le risposte
con spiegazioni chiare ed essenziali. \it Inserire le risposte
negli spazi predisposti. NON SI ACCETTANO RISPOSTE SCRITTE SU
ALTRI FOGLI. Scrivere il proprio nome anche nell'ultima pagina.
\rm 1 Esercizio = 3 punti. Tempo previsto: 2 ore. Nessuna domanda
durante la prima ora e durante gli ultimi 20 minuti.
\smallskip
\hrule
\medskip

\item{1.} Sia $F$ un campo e sia $f\in F[x]$ irriducibile. Mostrare
che $f$ ha radici multiple se e solo se $F$ ha caratteristica finita
$p\neq0$ e esiste $g\in F[x]$ tale che $f(x)=g(x^p)$.

\vv\item{2.} Descrivere gli elementi del gruppo di Galois del
polinomio $(x^2+1)(x^4-3)\in{\bf Q}[x]$ determinando anche tutti i
sottocampi del campo di spezzamento.

\ve\ \vs \item{3.} Dopo aver verificato che \`e algebrico, calcolare
il polinomio minimo di $\cos \pi/18$ su ${\bf Q}$.

\vv\item{4.} Si consideri $E={\bf F}_2[\alpha]$ dove $\alpha$ \`{e}
una radice del polinomio $X^3+X+1$. Determinare il polinomio minimo
su ${\bf F}_2$ di $\alpha+1$. \ve\ \vs

\item{5.} Dimostrare che ${\bf Q}(\zeta_m)$ possiede almeno un sottocampo
quadratico e fornire un esempio in cui i sottocampi quadratici sono
pi\`u di uno.

\vv \item{6.} Mostrare che se $F$ \`e un campo e $g\in F[X]$, allora
il grado del campo di spezzamento $E_g$ di $g$ su $F$ soddisfa
$$[E_g:F]\leq \partial(g)!.$$ \ve\ \vs

\item{7.} Mostrare che se $p_n$ denota l'$n$--esimo numero primo, allora
$$[{\bf Q}(\sqrt{p_1},\cdots,\sqrt{p_n}):{\bf Q}]=2^n.$$ Quanti sono
i sottocampi quadratici?

\vv \item{8.} Si enunci nella completa generalit\`a il Teorema di
corrispondenza di Galois.

\ve\ \vs

\item{9.} Mostrare che i polinomi irriducibili di grado 3 a coefficienti
razionali che ammettono un'unica radice reale hanno  $S_3$ come
gruppo di Galois.

\vv\item{10.} Dopo aver definito la nozione di campo perfetto,
dimostrare che i campi finiti sono perfetti.

\ve\ \vs

\item{11.} Sia $\zeta_{16}$ una radice primitiva 16--esima
dell'unit\`a. Descrivere gli ${\bf Q}(\sqrt{-1})$--omomorfismi di
${\bf Q}(\zeta_{16})$ in ${\bf C}$. \vss

\item{12.} Sia $E$ un'estensione finita di ${\bf Q}$ e siano
 $E_1$ e $E_2$ due sottocampi di $E$. Dimostrare che se $E_1$ e $E_2$ sono
 estensioni di Galois di ${\bf Q}$ allora anche il composto $E_1E_2$ \`{e}
 un'estensione di Galois di ${\bf Q}$.\vv

\ \vst

\centerline{\hskip 6pt\vbox{\tabskip=0pt \offinterlineskip
\def \trl{\noalign{\hrule}}
\halign to500pt{\strut#& \vrule#\tabskip=0.7em plus 1em& \hfil#&
\vrule#& \hfill#\hfil& \vrule#& \hfil#& \vrule#& \hfill#\hfil&
\vrule#& \hfil#& \vrule#& \hfill#\hfil& \vrule#& \hfil#& \vrule#&
\hfill#\hfil& \vrule#& \hfil#& \vrule#& \hfill#\hfil& \vrule#&
\hfil#& \vrule#& \hfill#\hfil& \vrule#& \hfil#& \vrule#& \hfil#&
\vrule#\tabskip=0pt\cr\trl && NOME E COGNOME && 1 && 2 && 3 && 4 &&
5 && 6 && 7 && 8 && 9 && 10 && 11 && 12 &&  TOT. &\cr\trl && &&   &&
&&     &&   &&   &&   &&   &&   &&    &&   &&   &&  && &\cr &&
\dotfill &&     &&   &&   &&   &&   &&   &&    &&  &&   && && && &&
&\cr\trl }}} \bye
