\documentclass[a4paper,11pt]{article}
\usepackage{times}
\usepackage[latin1]{inputenc}
\usepackage[T1]{fontenc}
\usepackage[italian]{babel}
\usepackage{amssymb}
\usepackage{amsmath}
\usepackage{amsthm}
%\nopagenumbers
\def\Q{{\mathbb Q}}
\def\Z{{\mathbb Z}}
\def\N{{\mathbb N}}
\def\F{{\mathbb F}}
\def\C{{\mathbb C}}
\def\QQ{{\rm Q}}

\title{Teoria di Galois 1 - Tutorato IV}
\author{\Large{Gruppi di Galois, campi ciclotomici, campi finiti}}
\date{Venerd� 29 Aprile 2005}

\begin{document}
\maketitle

\theoremstyle{definition}

\newtheorem{es}{Esercizio}



\begin{es}
Si calcoli il gruppo di Galois (cio� il numero di elementi e la struttura) di ciascuno dei seguenti polinomi:
\item{a.} $x^4 + 2x^3 + 15x^2 + 14x + 73$;    \hfill{b.} $x^4 + 8x^3 + 26x^2 + 24x + 28$;
\item{c.} $x^4 - 354x^2 + 29929$;      \hfill{d.} $x^4 - 11x^3 + 41x^2 - 61x + 30$;
\item{e.} $x^4 + 8x^3 + 14x^2 - 8x - 23;$  \hfill{f.} $x^4 - 13x^3 + 64x^2 - 142x + 121$;
\item{g.} $x^4 + x^3 + 2x^2 + 4x + 2$  \hfill{h.} $x^4+25x^2+5$;
\item{i.} $x^4+3x^3+3$  \hfill{l.} $ x^4 + x^3 + 4x^2 + 3x + 3$;
\item{m.} $x^4 + 60x^3 + 99x^2 + 60x + 1$ \hfill{n.} $ x^4 - 356x^2 + 29584$;
\item{o.} $x^4+8x+12$;
\end{es}
\medskip


\begin{es}
Si elenchino i sottogruppi transitivi di $S_4$ descrivendone gli elementi come permutazioni.
\end{es}
\medskip


\begin{es}
Descrivere gli elementi del gruppo di Galois del polinomio $x^5-2$ mostrando
che ha $20$ elementi.
\end{es}
\medskip



\begin{es}
Dimostrare che il gruppo di Galois de polinomio (che si pu\`{o} assumere irriducibile) $x^5+x^4+x^3+2x^2+3x+4$ non
ha $20$ elementi n\'{e} $10$ mostrando che contiene un $3$ ciclo.\hfill ({\it Pensare al numero primo $2$})
\end{es}
\medskip



\begin{es}
In ciascuno dei seguenti casi si calcoli il campo di spezzamento e il numero
di campi intermedi tra il campo base e il campo di spezzamento.
\item{a.} $(x^4+x^2+x+1)(x^3+x+1)  \in\F_2[x]$;
\item{b.} $(x^3+x+1)(x^6+x+1)\in \F_3[x];$
\item{c.}  $(x^4+x^2+1)(x^3+x+1)(x^3+1)\in\F_5[x]$;
\item{d.} $(x^4+x^2+1)(x^3+x+1)(x^3+1)\in\F_7[7]$.
\end{es}
\medskip



\begin{es}
Mostrare che se $f$ � un polinomio irriducibile di grado tre a coefficienti
in un campo $F$, $G_f$ � di tipo $A_3$ se e solo se $F_f$ non contiene sottocampi quadratici.
\end{es}
\medskip



\begin{es}
Calcolare una fomula per il discriminante di $X^n+aX+b$.
\end{es}
\medskip


\begin{es}
Si calcoli il gruppo di Galois di $y^5 - 3y^2 + 1$.
\end{es}
\medskip




\begin{es}
Mostrare che $\Phi_{p^r}(x)=\Phi_{p}(x^{p^{r-1}})$ e dedurne una formula
per il discriminante di $\Phi_{p^r}(x)$.
\end{es}
\medskip



\begin{es}
Sia $\Phi_p(x)=1+x+\cdots+x^{p-1}$ il polinomio ciclotomico. Mostrare che
$${\rm disc}\ \Phi_p(x)=(-1)^{(p-1)/2}p^{p-2}.$$
\end{es}
\medskip





\begin{es}
Mostrare che se $n$ \`{e} dispari, allora $\Psi_{2n}(x)=\Psi_n(-x)$ e che
$$\Psi_n(x)=\prod_{d\mid n}(x^d-1)^{\mu(n/d)}$$
dove $\mu$ \`{e} la funzione di M\"{o}bius.
\end{es}
\medskip


\begin{es}
L'obbiettivo di questo esercizio \`{e} di scoprire per passi successivi del seguente:\hfill\break
{\bf Teorema.} {\it Dato un gruppo {\bf abeliano} $G$, esiste sempre $f\in\Q[x]$ tale che $G\cong G_f$.}\smallskip
\item{i.} Il famoso Teorema di Dirichlet per primi in progressione aritmentica afferma (tra l'altro) che per ogni intero $m$,
esiste sempre un numero primo congruente a $1$ modulo $m$. Dedurne che esiste un polinomio
a coefficienti razionali il cui gruppo di Galois \`{e} isomorfo al gruppo ciclico $\Z/m\Z$;

\  \hfill {\it Suggerimento:} cercare tra i sottocampi di un opportuno
campo ciclotomico.

\item{ii.} Dimostrare $f$ e $g$ sono polinomi $\in\Q[x]$ con campi di spezzamento {\it linearmente
disgiunti} (i.e. $\Q_f\cap\Q_f=\Q$) allora $G_{fg}\cong G_f\times G_f.$

\ \hfill {\it Suggerimento:} Utilizzare la propriet�  che
${\rm Gal}(E_1E_2/F)\cong\{(\sigma_1,\sigma_2)\in{\rm Gal}(E_1/F)\times
{\rm Gal}(E_2/F)\ |\ \sigma_1|_{E_1\cap E_2}=\sigma_2|_{E_1\cap E_2}\}.$

\item{iii.} Dedurre il teorema dal Teorema di classificazione dei gruppi abeliani finiti
che dice che ogni gruppo abeliano \`{e} il prodotto di gruppi ciclici con ordini coprimi.

\end{es}
\medskip




\clearpage
\nocite{*}
\end{document}











