\documentclass[a4paper,11pt]{article}
\usepackage{times}
\usepackage[latin1]{inputenc}
\usepackage[T1]{fontenc}
\usepackage[italian]{babel}
\usepackage{amssymb}
\usepackage{amsmath}
\usepackage{amsthm}
%\nopagenumbers
\def\Q{{\mathbb Q}}
\def\Z{{\mathbb Z}}
\def\N{{\mathbb N}}
\def\F{{\mathbb F}}
\def\C{{\mathbb C}}
\def\QQ{{\rm Q}}

\title{Teoria di Galois 1 - Tutorato II}
\author{\Large{Estensioni, campi di spezzamento, omomorfismi di campi}}
\date{Venerd� 11 Marzo 2005}

\begin{document}
\maketitle

\theoremstyle{definition}

\newtheorem{es}{Esercizio}




\begin{es}
In ciascuno dei seguenti casi, determinare la dimensione
del campo di spezzamento del polinomio sul campo assegnato $F$:
\item{a.} $f(x)=x^3$                                                    \hfill $F=\Q$; 
\item{b.} $f(x)=(x^2-3)(x^2-27)(x^2-12) $                 \hfill $F=\Q( 3^{\frac{1}{3} })$;
\item{c.} $f(x)=x^8-4$                                                  \hfill $F=\Q$;
\item{d.}  $f(x)=x^h-3$                                                 \hfill $F=\Q(e^{ \frac{ 2\pi i}{h} })$;
\item{f.}  $f(x)=x^3+30x+1$                                           \hfill $F=\Q$;
\item{g.}  $f(x)=x^{15}+3x^5+1$                                    \hfill $F=\F_5$;
\item{h.}  $f(x)=x^4-x^3-4x^2+1$                                \hfill $F=\Q$;
\item{i.} $f(x)=x^{10}+x+1$                                             \hfill $F=\F_2$.
\end{es}
\medskip


\begin{es}
In ciascuno dei seguenti numeri algebrici, si calcoli il polinomio minimo

\item{a.} $e^{ \frac{2\pi i }{33}}  $; \ \ \ \ \ \ \ \ {b.} $\cos \frac{2\pi}{9} $;   \ \ \ \ \  {c.} $\cos\frac{ 2\pi}{11} $;
\medskip\item{d.} $\cos \frac{2\pi}{13} $;   \ \ \ \ \ {e.} $\cos \frac{\pi}{5}  $;   \ \ \ \ \ {f.} $\sin \frac{\pi}{7} $.
\end{es}
\medskip



\begin{es}
Descrivere gli $F$--omomorfismi di $E$ in $ \C $ in ciascuno dei seguenti casi:
\item{a.} $E=\Q(e^{\frac{\pi i}{8}} )$                                   \hfill $F=\Q(e^{\frac{\pi i}{2} })$;
\item{b.} $E=\Q(\sqrt{2},\sqrt{3},\sqrt{5})$                    \hfill $F=\Q(\sqrt{6})$;
\item{c.} $E=\Q(\zeta_7)$                                               \hfill $F=\Q(\cos \frac{2\pi}{7} )$;
\item{d.} $E=\Q(\sqrt{\sqrt{3}+1})$                                   \hfill $F=\Q(\sqrt{3})$;
\item{e.} Sostituire ora $\C$ con $\F_{7}(\beta)$, dove $\beta^4+\beta+1=0$
\medskip
\newline $E=\F_{7}(\alpha)$ \,\, con \,\, $ \alpha^4+5\alpha^2+3\alpha+1=0 $                  \hfill $F=\F_7(\sqrt{-2})$.
\end{es}
\medskip



\begin{es}
Si mostri che $\Q(\sqrt{-7}) \subseteq \Q(\zeta_{7})$.
\newline {\it Suggerimento: Considerare il numero
$\zeta_{7}+\zeta_{7}^2-\zeta_{7}^3+\zeta_{7}^4-\zeta_{7}^{5}+\zeta_{7}^6.$}
\end{es}
\medskip


\begin{es}
Mostrare che se $n$ divide $m$, allora $\Q(\zeta_n)\subset
\Q(\zeta_m).$
\end{es}
\medskip


\begin{es}
Dimostrare che se $q\in\Q$, allora $\cos(q\pi)$ \`{e} un numero algebrico.
Calcolare anche la dimensione
$$\left[\Q(\cos(q\pi)):\Q\right].$$
Si pu\`{o} dire la stessa cosa di $\sin(q\pi)$? \newline
{\it Suggerimento: Utilizzare (senza mostrarlo) il  fatto
che $\left[\Q(\zeta_m):\Q\right]=\varphi(m).$}
\end{es}
\medskip


\begin{es}
Mostrare che se $f\in F[x]$ \`{e} un polinomio irriducibile e char$F=p$,
allora il campo di spezzamento di $f$ ha grado $\partial f$.
\end{es}
\medskip


\clearpage
\nocite{*}
\end{document}