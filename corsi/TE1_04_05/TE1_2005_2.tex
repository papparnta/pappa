\nopagenumbers
\font\title=cmti12
\vfuzz=280pt
%\def\ve{\vfill\eject}
%\def\vv{\vfill}
%\def\vs{\vskip-2cm}
%\def\vss{\vskip10cm}
%\def\vst{\vskip13.3cm}

\def\ve{\bigskip\bigskip}
\def\vv{\bigskip\bigskip}
\def\vs{}
\def\vss{}
\def\vst{\bigskip\bigskip}

\hsize=19.5cm
\vsize=27.58cm
\hoffset=-1.6cm
\voffset=0.5cm
\parskip=-.1cm
\ \vs \hskip -6mm TE1 AA04/05\ (Teoria delle Equazioni)\hfill ESAME
DI FINE SEMESTRE \hfill Roma, 25 Maggio 2005. \hrule
\bigskip\noindent
{\title COGNOME}\  \dotfill\  {\title NOME}\ \dotfill {\title
MATRICOLA}\ \dotfill\
\smallskip  \noindent
Risolvere il massimo numero di esercizi accompagnando le risposte
con spiegazioni chiare ed essenziali. \it Inserire le risposte
negli spazi predisposti. NON SI ACCETTANO RISPOSTE SCRITTE SU
ALTRI FOGLI. Scrivere il proprio nome anche nell'ultima pagina.
\rm 1 Esercizio = 3 punti. Tempo previsto: 2 ore. Nessuna domanda
durante la prima ora e durante gli ultimi 20 minuti.
\smallskip
\hrule
\medskip

\item{1.} Dimostrare che il gruppo delle matrici $3\times 3$ invertibili a coefficienti
nel campo con due elementi \`{e} un sottogruppo transitivo di $S_7$.
\hfill{\it Suggerimento: pensare al piano proiettivo.}

\vv \item{2.} Descrivere gli elementi del gruppo di Galois del
polinomio $(x^2-2)(x^2-3)(x^2-5)(x^2-30)$.

\ve\ \vs \item{3.} Enunciare il Teorema di Dedekind per gruppi di
Galois di polinomi a coefficienti interi e lo si applichi per
mostrare che il gruppo di Galois si ${\bf Q}$ del polinomio
$(x^2+x+1)(x^3+x+1)+2$ \`{e}  $S_5$. \vv

\item{4.} Calcolare quanti sono i polinomi irriducibili (monici) di grado $l^2$ ($l$ primo) su
${\bf F}_q$ ($q$ primo).\ve\ \vs

\item{5.} Calcolare il gruppo di Galois del polinomio $x^4-2x+1$.

\vv \item{6.} Fornire una condizione necessaria e sufficiente
affinch\'e il gruppo di Galois di un polinomio irriducibile di grado
$n$ a coefficienti razionali sia contenuto in $A_n$. \ve\ \vs

\item{7.} Spiegare come si fa a costruire un polinomio il cui
gruppo di Galois � isomorfo a $(C_5)^k$.

\vv \item{8.} Si enunci nella completa
generalit\`{a} il Teorema di corrispondenza di Galois.

\ve\ \vs

\item{9.} Sia $p>2$ un primo. Dimostrare che il $p$-esimo campo
ciclotomico ${\bf Q}[\zeta_p]$
contiene ${\bf Q}[\sqrt{(-1)^{(p-1)/2}p}]$ e che questo \`{e}
l'unico sottocampo quadratico.\vv

\item{10.} Quali sono i generatori del gruppo moltiplicativo di
${\bf F}_2[\alpha]$, con $\alpha^4=\alpha+1$?

\ve\ \vs

\item{11.} Sia $f\in{\bf F}_p[x]$ con $\partial f=4$. Mostrare che $f$ \`{e} irriducibile se
e solo se ${\rm MCD}(f,x^p-x)={\rm MCD}(f,x^{p^2}-x)=1$. Usare
questo criterio per mostrare che $x^4+x^3-x^2-x-1$ \`{e}
irriducibile su ${\bf F}_3$.\vss

\item{12.} Enunciare e dimostrare il Teorema di construibilit\`{a} degli $n$-agoni regolari fornendo anche
qualche esempio.\vv

\ \vst

\centerline{\hskip 6pt\vbox{\tabskip=0pt \offinterlineskip
\def \trl{\noalign{\hrule}}
\halign to500pt{\strut#& \vrule#\tabskip=0.7em plus 1em&
\hfil#& \vrule#& \hfill#\hfil& \vrule#&
\hfil#& \vrule#& \hfill#\hfil& \vrule#&
\hfil#& \vrule#& \hfill#\hfil& \vrule#&
\hfil#& \vrule#& \hfill#\hfil& \vrule#&
\hfil#& \vrule#& \hfill#\hfil& \vrule#&
\hfil#& \vrule#& \hfill#\hfil& \vrule#&
\hfil#& \vrule#& \hfil#& \vrule#\tabskip=0pt\cr\trl
&& NOME E COGNOME && 1 && 2 && 3 && 4 && 5 && 6 && 7 && 8 && 9 && 10 && 11 && 12 &&  TOT. &\cr\trl
&& &&   &&   &&     &&   &&   &&   &&   &&   &&    &&   &&   &&  && &\cr
&& \dotfill &&     &&   &&   &&   &&   &&   &&    &&  &&   && && && && &\cr\trl
}}}
 \bye
