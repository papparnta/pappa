\nopagenumbers
\font\title=cmti12
\vfuzz=280pt
%\def\ve{\vfill\eject}
%\def\vv{\vfill}
%\def\vs{\vskip-2cm}
%\def\vss{\vskip10cm}
%\def\vst{\vskip13.3cm}

\def\ve{\bigskip\bigskip}
\def\vv{\bigskip\bigskip}
\def\vs{}
\def\vss{}
\def\vst{\bigskip\bigskip}

\hsize=19.5cm
\vsize=27.58cm
\hoffset=-1.6cm
\voffset=0.5cm
\parskip=-.1cm
\ \vs \hskip -6mm TE1 AA04/05\ (Teoria delle Equazioni)\hfill ESAME
SCRITTO \hfill APPELLO C - Roma, 14 Settembre 2005. \hrule
\bigskip\noindent
{\title COGNOME}\  \dotfill\  {\title NOME}\ \dotfill {\title
MATRICOLA}\ \dotfill\
\smallskip  \noindent
Risolvere il massimo numero di esercizi accompagnando le risposte
con spiegazioni chiare ed essenziali. \it Inserire le risposte
negli spazi predisposti. NON SI ACCETTANO RISPOSTE SCRITTE SU
ALTRI FOGLI. Scrivere il proprio nome anche nell'ultima pagina.
\rm 1 Esercizio = 3 punti. Tempo previsto: 2 ore. Nessuna domanda
durante la prima ora e durante gli ultimi 20 minuti.
\smallskip
\hrule
\medskip

\item{1.} Calcolare il polinomio minimo su ${\bf Q}$, di
$\big(2-\cos(\pi/4)\big)^{1/3}$.%

\vv \item{2.} Mostrare che due campi di spezzamento dello stesso
polinomio sono isomorfi.

\ve\ \vs \item{3.} Determinare tutti i sottocampi quadratici di
${\bf Q}(\zeta_{77})$. \vv

\item{4.} Mostrare che per ogni primo $p$ e $n\in{\bf N}$ esiste un unico campo finito
a meno di isomorfismi.\ve\ \vs %

\item{5.} Calcolare il numero di elementi del gruppo di Galois su ${\bf Q}$ del polinomio
$x^6-2$.

\vv \item{6.} Spiegare come si fa a calcolare il gruppo di Galois di
un polinomio di grado 4.
\ve\ \vs  %%%%%%%%%%

\item{7.} Costruire un'estensione $F$ di Galois di ${\bf Q}$ tale
che
Gal$(F/{\bf Q})\simeq C_9\times C_9$ spiegando la teoria usata. %

\vv \item{8.} Si enunci nella completa generalit\`{a} il Teorema
di corrispondenza di Galois. %%%%
\ve\ \vs

\item{9.} Si calcoli il numero di elementi nel campo di spezzamento
del polinomio
$x^9+x^5+x$ su ${\bf F}_2$.\vv %%%%%%%%

\item{10.} Dare un esempio di campo finito ${\bf F}_{16}$ con $16$
elementi determinando tutti i generatori del gruppo moltiplicativo
${\bf F}_{16}^*$.%%%%
\ve\ \vs

\item{11.} Mostrare che i polinomi a coefficienti in un campo finito hanno
gruppo di Galois ciclico.\vss

\item{12.} Esibire (se esiste) una costruzione del numero
$\big((2+\sqrt{2})^{1/4}+1\big)^{1/8}$.\vv %%%%%%%%%%%
\ \vst

%\centerline{\hskip 6pt\vbox{\tabskip=0pt \offinterlineskip
%\def \trl{\noalign{\hrule}}
%\halign to500pt{\strut#& \vrule#\tabskip=0.7em plus 1em&
%\hfil#& \vrule#& \hfill#\hfil& \vrule#&
%\hfil#& \vrule#& \hfill#\hfil& \vrule#&
%\hfil#& \vrule#& \hfill#\hfil& \vrule#&
%\hfil#& \vrule#& \hfill#\hfil& \vrule#&
%\hfil#& \vrule#& \hfill#\hfil& \vrule#&
%\hfil#& \vrule#& \hfill#\hfil& \vrule#&
%\hfil#& \vrule#& \hfil#& \vrule#\tabskip=0pt\cr\trl
%&& NOME E COGNOME && 1 && 2 && 3 && 4 && 5 && 6 && 7 && 8 && 9 && 10 && 11 && 12 &&  TOT. &\cr\trl
%&& &&   &&   &&     &&   &&   &&   &&   &&   &&    &&   &&   &&  && &\cr
%&& \dotfill &&     &&   &&   &&   &&   &&   &&    &&  &&   && && && && &\cr\trl
%}}}
 \bye
