\nopagenumbers
\font\title=cmti12
\vfuzz=280pt
\def\ve{\vfill\eject}
\def\vv{\vfill}
\def\vs{\vskip-2cm}
\def\vss{\vskip10cm}
\def\vst{\vskip13.3cm}

%\def\ve{\bigskip\bigskip}
%\def\vv{\bigskip\bigskip}
%\def\vs{}
%\def\vss{}
%\def\vst{\bigskip\bigskip}

\hsize=19.5cm
\vsize=27.58cm
\hoffset=-1.6cm
\voffset=0.5cm
\parskip=-.1cm
\ \vs \hskip -6mm TE1 AA04/05\ (Teoria delle Equazioni)\hfill ESAME
SCRITTO \hfill APPELLO A - Roma, 8 Giugno 2005. \hrule
\bigskip\noindent
{\title COGNOME}\  \dotfill\  {\title NOME}\ \dotfill {\title
MATRICOLA}\ \dotfill\
\smallskip  \noindent
Risolvere il massimo numero di esercizi accompagnando le risposte
con spiegazioni chiare ed essenziali. \it Inserire le risposte
negli spazi predisposti. NON SI ACCETTANO RISPOSTE SCRITTE SU
ALTRI FOGLI. Scrivere il proprio nome anche nell'ultima pagina.
\rm 1 Esercizio = 3 punti. Tempo previsto: 2 ore. Nessuna domanda
durante la prima ora e durante gli ultimi 20 minuti.
\smallskip
\hrule
\medskip

\item{1.} Calcolare il polinomio minimo su ${\bf Q}$, di
$\zeta_{7}+\zeta^4_{7}+\zeta^2_{7}$.%

\vv \item{2.} Mostrare che il campo di spezzamento di un polinomio
di grado $m$ a coefficienti razionali ha dimensione su ${\bf Q}$ minore o uguale
a $m!$. %

\ve\ \vs \item{3.} Determinare tutti i sottocampi $K$ di ${\bf
Q}(\zeta_{36})$ tali che $[{\bf
Q}(\zeta_{36}):K]=2$. %

\vv

\item{4.} Calcolare quanti sono i polinomi irriducibili (monici)
di grado $6$ su ${\bf F}_{13}$.\ve\ \vs %

\item{5.} Calcolare il gruppo di Galois si ${\bf Q}$ del polinomio
$x^4+3x^3+3x+18$. %%%%%%%%%

\vv \item{6.}Mostrare the se
$f(x)=\displaystyle{\prod_{i=1}^m\big(x-\alpha_i\big)}\in F[x],$
allora il discriminante $D(f)$ soddisfa:
$D(f)=\displaystyle{(-1)^{{m(m-1)\over2}}\prod_{i=1}^mf'(\alpha_i)}$. \ve\ \vs  %%%%%%%%%%

\item{7.} Costruire un estensione $F$ di Galois di ${\bf Q}$ tale che
Gal$(F/{\bf Q})\simeq C_2\times C_2 \times C_4$. %

\vv \item{8.} Si enunci nella completa generalit\`{a} il Teorema
di corrispondenza di Galois. %%%%
\ve\ \vs

\item{9.} Si calcoli il numero di elementi nel campo di spezzamento
del polinomio
$(x^2+x+1)(x^3+x^2+1)(x^4+x+1)(x^5-x)$ su ${\bf F}_2$.\vv %%%%%%%%

\item{10.} Dare un esempio di campo finito ${\bf F}_{25}$ con $25$
elementi determinando tutti i generatori del gruppo moltiplicativo
${\bf F}_{25}^*$.%%%%
\ve\ \vs

\item{11.} Dimostrare che i campi finiti sono perfetti.\vss

\item{12.} Definire la nozione di numero costruibile e mostrare che l'insieme
dei numeri costruibili \`{e} un campo.\vv %%%%%%%%%%%
\ \vst

\centerline{\hskip 6pt\vbox{\tabskip=0pt \offinterlineskip
\def \trl{\noalign{\hrule}}
\halign to500pt{\strut#& \vrule#\tabskip=0.7em plus 1em&
\hfil#& \vrule#& \hfill#\hfil& \vrule#&
\hfil#& \vrule#& \hfill#\hfil& \vrule#&
\hfil#& \vrule#& \hfill#\hfil& \vrule#&
\hfil#& \vrule#& \hfill#\hfil& \vrule#&
\hfil#& \vrule#& \hfill#\hfil& \vrule#&
\hfil#& \vrule#& \hfill#\hfil& \vrule#&
\hfil#& \vrule#& \hfil#& \vrule#\tabskip=0pt\cr\trl
&& NOME E COGNOME && 1 && 2 && 3 && 4 && 5 && 6 && 7 && 8 && 9 && 10 && 11 && 12 &&  TOT. &\cr\trl
&& &&   &&   &&     &&   &&   &&   &&   &&   &&    &&   &&   &&  && &\cr
&& \dotfill &&     &&   &&   &&   &&   &&   &&    &&  &&   && && && && &\cr\trl
}}}
 \bye
