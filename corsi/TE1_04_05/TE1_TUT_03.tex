\documentclass[a4paper,11pt]{article}
\usepackage{times}
\usepackage[latin1]{inputenc}
\usepackage[T1]{fontenc}
\usepackage[italian]{babel}
\usepackage{amssymb}
\usepackage{amsmath}
\usepackage{amsthm}
%\nopagenumbers
\def\Q{{\mathbb Q}}
\def\Z{{\mathbb Z}}
\def\N{{\mathbb N}}
\def\F{{\mathbb F}}
\def\C{{\mathbb C}}
\def\QQ{{\rm Q}}

\title{Teoria di Galois 1 - Tutorato III}
\author{\Large{Estensioni, reticoli di sottocampi, gruppi di Galois}}
\date{Venerd� 1 Aprile 2005}

\begin{document}
\maketitle

\theoremstyle{definition}

\newtheorem{es}{Esercizio}


\begin{es}
Descrivere gli elementi del gruppo di Galois, determinando anche tutti i sottocampi
del campo di spezzamento, del polinomio $(x^2-2)(x^2+3)$
\end{es}
\medskip



\begin{es}
Descrivere gli elementi del gruppo di Galois del campo di spezzamento di $x^n-1$.
\end{es}
\medskip


\begin{es}
In cisacuno dei seguenti casi si dica se si tratta di estensioni separabili,
normali o di Galois (nel qual caso descrivere il gruppo di Galois):
\item{{\it i.}} $\F_7(T)/\F_7(T^7)$;\hfill {\it ii.} $\Q(3^{1/5})/\Q$;
\item{{\it iii.}} $\F_{11}(T)/\F_{11}$;\hfill {\it iv.} $\Q(3^{1/5},\zeta_{30})/\Q(\zeta_{30})$;
\item{{\it v.}} $\Q(\sqrt{-1},5^{1/4})/\Q$;\hfill {\it vi.} $\Q(\pi,\sqrt{\pi})/\Q(\pi)$.
\end{es}
\medskip



\begin{es}
Si descrivano tutti i campi intermedi tra $E$ e $\Q$ in ciascuno dei seguenti casi:
\item{a.} $E=\Q(\zeta_{n}) \,\,$  con  $ \,\,n= 16,  24, 13 $
\item{b.} $E=\Q_f$ il campo di spezzamento di $x^4-2$
\item{c.} $E=\Q_f$ il campo di spezzamento di $(x^2-2)(x^2-3)(x^2-5)$.

\  \hfill{\it Suggerimento: Usare la corrispondenza di Galois.}
\end{es}
\medskip



\begin{es}
Per ciascuno dei punti dell'esercizio precedente si descrivano gli
elementi del gruppo di Galois Gal$(E/F)$.
\end{es}
\medskip






\begin{es}
Sia $E=\Q(\zeta_{13})$. Dimostrare che se $\eta=\zeta_{13}+\zeta_{13}^3+\zeta_{13}^9$,
allora il polinomio minimo $f_\eta$ di $\eta$ su $\Q$ ha grado $4$.
Dopo averne evidenziato le radici, mostrare (calcolando) che
$$f_\eta(x)=x^4+x^3+2x^2-4x+3.$$
Qual'\`{e} la dimensione del campo di spezzamento di $f_\eta$ su $\Q$?

{\it Suggerimento: Usare il gruppo {\rm Gal}$(\Q(\zeta_{13})/\Q)$ e la corrispondenza
di Galois.}
\bigskip
\end{es}
\medskip



\begin{es}
Dimostrare $\Q(\zeta_p)$ (dove $p>2$ \`{e} primo) ha sempre esattamente un sottocampo
quadratico. Dedurre che ogni campo ciclotomico ammette sempre un sottocampo che \`{e}
un estensione quadratica di $\Q$.

\  \hfill {\it Suggerimento: Usare il gruppo {\rm Gal}$(\Q(\zeta_{p})/\Q)$ e la corrispondenza
di Galois.}
\end{es}
\medskip



\begin{es}
Mostrare che, ${\rm Gal}(\Q(\zeta_{n^2})/\Q(\zeta_{n}))\cong \Z/n\Z$ esibendo un isomorfismo esplicito.

\hfill {\it Sugg: considerare $\sigma_j:\zeta_{n^2}\mapsto \zeta_{n^2}^{nj+1}.$}
\end{es}
\medskip




\begin{es}[per che soffre di insonnia]
Mostrare la seguente identit\`{a}:
$$\sum_{j=1}^p\left({\frac{j}{p}}\right)\zeta_{p}^j=\pm\sqrt{(-1)^{(p-1)/2}p}$$
(N.B. $\left({ \frac{j}{p}  }\right)$ \`{e} il classico simbolo di Legendre).
Dedurre che ogni campo quadratico \`{e} sempre contenuto in un campo ciclotomico.
\end{es}
\medskip



\clearpage
\nocite{*}
\end{document}


