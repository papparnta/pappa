\nopagenumbers
\font\title=cmti12
\vfuzz=280pt
\def\ve{\vfill\eject}
\def\vv{\vfill}
\def\vs{\vskip-2cm}
\def\vss{\vskip10cm}
\def\vst{\vskip13.3cm}

%\def\ve{\bigskip\bigskip}
%\def\vv{\bigskip\bigskip}
%\def\vs{}
%\def\vss{}
%\def\vst{\bigskip\bigskip}

\hsize=19.5cm
\vsize=27.58cm
\hoffset=-1.6cm
\voffset=0.5cm
\parskip=-.1cm
\ \vs \hskip -6mm TE1 AA04/05\ (Teoria delle Equazioni)\hfill ESAME
SCRITTO \hfill APPELLO B - Roma, 22 Luglio 2005. \hrule
\bigskip\noindent
{\title COGNOME}\  \dotfill\  {\title NOME}\ \dotfill {\title
MATRICOLA}\ \dotfill\
\smallskip  \noindent
Risolvere il massimo numero di esercizi accompagnando le risposte
con spiegazioni chiare ed essenziali. \it Inserire le risposte
negli spazi predisposti. NON SI ACCETTANO RISPOSTE SCRITTE SU
ALTRI FOGLI. Scrivere il proprio nome anche nell'ultima pagina.
\rm 1 Esercizio = 3 punti. Tempo previsto: 2 ore. Nessuna domanda
durante la prima ora e durante gli ultimi 20 minuti.
\smallskip
\hrule
\medskip

\item{1.} Calcolare il polinomio minimo su ${\bf Q}$, di
$\cos(2\pi/5)+\cos^2(2\pi/5)$.%

\vv \item{2.} Dopo aver definito la nozione di polinomio minimo, si
dimostri che \`{e} sempre irriducibile.

\ve\ \vs \item{3.} Determinare tutti i sottocampi di ${\bf
Q}(\zeta_{16})$. \vv

\item{4.} Dimostrare che il polinomio $X^{p^n}-X\in{\bf F}_p[X]$ \`{e} separabile
ed \`{e} il prodotto di tutti i polinomi irriducibili (monici) in ${\bf F}_p[X]$
il cui grado divide $n$.\ve\ \vs %

\item{5.} Calcolare il gruppo di Galois su ${\bf Q}$ del polinomio
$x^4+4x^2+18$.

\vv \item{6.} Definire la nozione di discriminante di un polinomio
in ${\bf Q}[x]$ e mostrare che che il gruppo di Galois di un
polinomio \`{e} contenuto nel gruppo alterno $A_n$ se e solo se
il discriminante \`{e} un quadrato perfetto.\ve\ \vs  %%%%%%%%%%

\item{7.} Costruire un estensione $F$ di Galois di ${\bf Q}$ tale che
Gal$(F/{\bf Q})\simeq C_3\times C_3 \times C_4$. %

\vv \item{8.} Si enunci nella completa generalit\`{a} il Teorema
di corrispondenza di Galois. %%%%
\ve\ \vs

%\item{9.} Si calcoli il numero di elementi nel campo di spezzamento
%del polinomio
%$(x^2+x+1)(x^3+x^2+1)(x^4+x+1)(x^5-x)$ su ${\bf F}_2$.\vv %%%%%%%%

\item{10.} Dare un esempio di campo finito ${\bf F}_{27}$ con $27$
elementi determinando tutti i generatori del gruppo moltiplicativo
${\bf F}_{27}^*$.%%%%
\ve\ \vs

\item{11.} Dopo aver definito la nozione di campo perfetto, si dia un esempio di
campi imperfetto.\vss

\item{12.} Enunciare e dimostrare il Teorema di costruibilit\`{a} dei poligoni regolari.\vv %%%%%%%%%%%
\ \vst

\centerline{\hskip 6pt\vbox{\tabskip=0pt \offinterlineskip
\def \trl{\noalign{\hrule}}
\halign to500pt{\strut#& \vrule#\tabskip=0.7em plus 1em&
\hfil#& \vrule#& \hfill#\hfil& \vrule#&
\hfil#& \vrule#& \hfill#\hfil& \vrule#&
\hfil#& \vrule#& \hfill#\hfil& \vrule#&
\hfil#& \vrule#& \hfill#\hfil& \vrule#&
\hfil#& \vrule#& \hfill#\hfil& \vrule#&
\hfil#& \vrule#& \hfill#\hfil& \vrule#&
\hfil#& \vrule#& \hfil#& \vrule#\tabskip=0pt\cr\trl
&& NOME E COGNOME && 1 && 2 && 3 && 4 && 5 && 6 && 7 && 8 && 9 && 10 && 11 && 12 &&  TOT. &\cr\trl
&& &&   &&   &&     &&   &&   &&   &&   &&   &&    &&   &&   &&  && &\cr
&& \dotfill &&     &&   &&   &&   &&   &&   &&    &&  &&   && && && && &\cr\trl
}}}
 \bye
