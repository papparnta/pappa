\magnification 1200
\nopagenumbers
\def\F{{\bf F}}
\def\N{{\bf N}}
\vskip 2cm
\centerline{\bf Esame finale di CR3 - Mercoled\`\i\ 20 Maggio 2009}
\centerline{\it NB. Si consegna entro luned\`\i\ 25 Maggio alle 9:00AM}\bigskip\bigskip
 
\item{a.} Sia $E$ una curva ellittica definita su un campo con $p$ elementi e tale che
$E(\F_p)=C_{27}\times C_{81}$. Determinare tutti i possibili valori di $p$ e per ciascun valore
determinare una curva ellitticha con tale propriet\`a. Giustificare tutti i passi.\bigskip
%2161
%2269

\item{b.} Sia $E: y^2=x^3+x+1$. Simulare l'algoritmo di Schoof per calcolare $E(\F_{31})$. \bigskip

\item{c.} Calcolare $\#E(\F_{3^{100}})$ dove $E: y^2=x^3+2x+1$. Giustificare la risposta.\bigskip

\item{d.} Calcolare l'ordine del punto $(0,0)\in E(\F_{16})$ dove $E: y^2+y=x^3+x$ utilizzando
l'algoritmo Baby Step Giant Step. Dedurne l'ordine di $E(\F_{16})$.\bigskip

\item{f.} Sia $E$ una curva definita su un campo finito $\F_q$ e sia $E'$ il suo twist. 
Dimostrare che $E(\F_q)\times E'(\F_q) \cong E(\F_{q^2})$.\bigskip

\item{g.} Sia $y^2+a_1xy+a_3y=x^3+a_2x^2+a_4x+a_6$ un equazione di Weierstrass su un campo e sia 
$\alpha$ la trasformazione affine definita da $(x,y)\mapsto(u^2x+r,u^3y+su^2x+t)$. Dimostrare che $\alpha$
trasforma l'equazione di Weierstrass in un'altra equazione di Weierstrass. Inoltre se due equazioni di
Weierstrass si ottengono l'una dall'altra attraverso una trasformazione affine, allora questa deve avere la
forma di $\alpha$.\bigskip

\item{h.} Sia $\F_q$ un campo finito di caratteristica dispari e siano $a, b\in\F_q$ 
$a \neq \pm2b$ e $b\neq 0$ si consideri la curva ellittica di equazione $y^2 = x^3 + ax^2 + b^2 x$.\medskip 
\itemitem{(1)} Dimostrare che i punti $(b,b\sqrt{a+2b})$ e $(-b,-b\sqrt{a-2b})$ hanno ordine $4$.\smallskip
\itemitem{(2)} Dimostrare che almeno uno tra $a+2b, a-2b$ e $a^2-4b^2$ \`e un quadrato un $\F_q$.\smallskip 
\itemitem{(3)} Dimostrare che se $a^2-4b^2$ \`e un quadrato in $\F_q$, allora $E[2]\subseteq E(\F_q)$.\smallskip
\itemitem{(4)} Mostrare che $\#E(\F_q)$ \`e un multiplo di $4$.\smallskip
\itemitem{(5)} Sia $E'$ la curva ellitticha definita da $y^2 = x^3-2ax^2+(a^2-4b^2)x$.\smallskip
Mostrare che $E'[2]\subseteq E'(\F_q)$ e dedurre che anche $\#E'(\F_q)$ \`e un multiplo di $4$.\bigskip

\item{i.} Produrre vari esempi di interi $m,n\in\N$ tali che non esiste alcun campo finito $\F_q$ per
il quale esiste una curva ellittica $E/\F_q$ tale che $E(\F_q)\cong C_m\times C_{mn}$.\bigskip

\item{j.} Si consideri l'equazione proiettiva della curva ellittica $E$: $F (x, y, z) = y^2 z - x^3 - Axz^2 - Bz^3 = 0$.
Dimostrare che un punto $P$ su $E$ appartiene a $E[3]$ se e solo se
$$\det\pmatrix{F_{xx}& F_{xy} & F_{xz}\cr F_{yx} & F_{yy} & F_{yz}\cr F_{zx}& F_{zy}& F_{zz}}=0$$
nel punto $P$, where $F_{ab}$ denota la derivata parziale seconda rispetto a $a$ e $b$. Il determinante si chiama
Hessiano. I punti della curva che annullano l'Hessiano si chiamano flessi.
\bye
