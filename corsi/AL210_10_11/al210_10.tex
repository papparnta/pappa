\documentclass[italian,a4paper,11pt]
{article}
\usepackage{babel,amsmath,amssymb,amsbsy,amsfonts,latexsym,exscale,
amsthm,epsf,colordvi,enumerate}

\usepackage[latin1]{inputenc}
\usepackage[all]{xy}
\usepackage{textcomp}
\usepackage{graphicx} 


\newcommand{\Q}{\mathbb{Q}}
\newcommand{\Z}{\mathbb Z}
\newcommand{\R}{\mathbb{R}}
\newcommand{\PP}{\mathbb{P}}
\newcommand{\A}{\mathbb{A}}
\newcommand{\I}{\mathcal{I}}

\newcommand{\F}{\mathbb{F}}
\newcommand{\N}{\mathbb{N}}
\newcommand{\C}{\mathbb{C}}
\newcommand{\T}{\mathcal{T}}
\newcommand{\Zeri}{\mathcal{Z}}
\newcommand{\U}{\mathcal{U}}
\newcommand{\p}{\mathfrak{p}}
\newcommand{\ga}{\mathfrak{a}}
\newcommand{\gb}{\mathfrak{b}}

\newcommand{\q}{\mathfrak{q}}
\newcommand{\m}{\mathfrak{m}}
\newcommand{\X}{\mathbf{X}}

\newcommand{\D}{\mbox{\rm{\textbf{Dom}}}}
\newcommand{\Ze}{\mbox{\rm{\textbf{Rie}}}}

\newcommand{\esse}{\mbox{\rm{\textbf{Spec}}}}
\newcommand{\Ci}{\mathbf{C}}
\newcommand{\Ex}{\textbf{Esercizio}}


\newcommand{\Sse}{\Longleftrightarrow}
\newcommand{\sse}{\Leftrightarrow}
\newcommand{\implica}{\Rightarrow}

\newcommand{\frecdl}{\longrightarrow}
\newcommand{\frecd}{\rightarrow}
\newcommand{\st}{\scriptstyle}
\newcommand{\svol}{\textbf{Svolgimento:}}
\newcommand{\cvd}{\begin{flushright} \qed \end{flushright}}
\newcommand{\acc}{\`}
\begin{document}
\begin{center}

\textbf{Universit\`a degli Studi Roma Tre}\\

\textbf{Corso di Laurea in Matematica, a.a. 2010/2011}\\

\textbf{AL210 - Algebra 2: Gruppi, Anelli e Campi}\\

\textbf{Prof. F. Pappalardi}\\

\textbf{Tutorato 10 - 13 Dicembre 2010}\\

\textbf{Tutore: Matteo Acclavio}\\

www.matematica3.com\\
\end{center}

\vspace{0.4 cm}
\noindent
\begin{Ex}\textbf{ 1.}\\
Sia $K$ un campo e consideriamo l'anello $A = K[X;Y]/(X^2;Y^2)$.
\begin{itemize}
\item Dette $x$ e $y$ le classi di $A$ determinate da $X$ e $Y$ , provare che ogni
elemento di $A$ si pu\acc o esprimere in un unico modo nella forma:\\ $axy + bx + cy + d$ con $a, b, c, d \in K$
\item Calcolare il prodotto tra due elementi di $A$ generici.
\item Determinare gli zero divisori di $A$.
\item Determinare gli invertibili di $A$.
\end{itemize}
\end{Ex}

\vspace{0.4 cm}
\noindent
\begin{Ex}\textbf{ 2.}\\
Sia $D$ un dominio euclideo e $v: D^* \rightarrow \N $ la sua valutazione, mostrare che:
\begin{itemize}
\item $v(1)=v(u) \forall u \in \U(D)$
\item $v(1)\leq v(a) \  \forall a\in A $
\item $x|y \Leftrightarrow v(x)\leq v(y)$
\end{itemize}
\end{Ex}

\vspace{0.4 cm}
\noindent
\begin{Ex}\textbf{ 2.}\\
Sia $\Z[i]:=\{ a+bi \ | a,b\in \Z, \ i^2=-1\}$ e sia $|\cdot|: \Z[i]^* \rightarrow \N $ tale che $|(a+ib)|=a^2+b^2$ il modulo su $\Z[i]$. Dimostrare che:
\begin{itemize}
\item $\Z[i]$ \acc e isomorfo a $\frac {Z[x]}{(x^2+1)}$
\item $\Z[i]$ \acc e un dominio euclideo (verificare se il modulo rispetta le propriet\acc a della valutazione)
\end{itemize}
\end{Ex}


\vspace{0.4 cm}
\noindent
\begin{Ex}\textbf{ 3.}\\
Dimostrare che $\Z[\sqrt 6]$ non \acc e un UFD
\end{Ex}



\end{document}
