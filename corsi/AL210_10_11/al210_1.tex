\documentclass[italian,a4paper,11pt]
{article}
\usepackage{babel,amsmath,amssymb,amsbsy,amsfonts,latexsym,exscale,
amsthm,epsf,colordvi,enumerate}

\usepackage[latin1]{inputenc}
\usepackage[all]{xy}
\usepackage{textcomp}
\usepackage{graphicx} 


\newcommand{\Q}{\mathbb{Q}}
\newcommand{\Z}{\mathbb Z}
\newcommand{\R}{\mathbb{R}}
\newcommand{\PP}{\mathbb{P}}
\newcommand{\A}{\mathbb{A}}
\newcommand{\I}{\mathcal{I}}

\newcommand{\F}{\mathbb{F}}
\newcommand{\N}{\mathbb{N}}
\newcommand{\C}{\mathbb{C}}
\newcommand{\T}{\mathcal{T}}
\newcommand{\Zeri}{\mathcal{Z}}
\newcommand{\U}{\mathcal{U}}
\newcommand{\p}{\mathfrak{p}}
\newcommand{\ga}{\mathfrak{a}}
\newcommand{\gb}{\mathfrak{b}}

\newcommand{\q}{\mathfrak{q}}
\newcommand{\m}{\mathfrak{m}}
\newcommand{\X}{\mathbf{X}}

\newcommand{\D}{\mbox{\rm{\textbf{Dom}}}}
\newcommand{\Ze}{\mbox{\rm{\textbf{Rie}}}}

\newcommand{\esse}{\mbox{\rm{\textbf{Spec}}}}
\newcommand{\Ci}{\mathbf{C}}
\newcommand{\Ex}{\textbf{Esercizio}}


\newcommand{\Sse}{\Longleftrightarrow}
\newcommand{\sse}{\Leftrightarrow}
\newcommand{\implica}{\Rightarrow}

\newcommand{\frecdl}{\longrightarrow}
\newcommand{\frecd}{\rightarrow}
\newcommand{\st}{\scriptstyle}
\newcommand{\svol}{\textbf{Svolgimento:}}
\newcommand{\cvd}{\begin{flushright} \qed \end{flushright}}
\newcommand{\acc}{\`}
\begin{document}
\begin{center}

\textbf{Universit\`a degli Studi Roma Tre}\\

\textbf{Corso di Laurea in Matematica, a.a. 2010/2011}\\

\textbf{AL210 - Algebra 2: Gruppi, Anelli e Campi}\\

\textbf{Prof. F. Pappalardi}\\

\textbf{Tutorato 1 - 27 Settembre 2010}\\

\textbf{Tutore: Matteo Acclavio}\\

www.matematica3.com\\
\end{center}



\vspace{0.4cm}

\noindent
\begin{Ex}\textbf{ 1.}\\
Stabilire se i seguenti sottoinsiemi di $\R$ sono gruppi rispetto alla somma e al prodotto usuali (specificando eventuali propriet\'a mancanti)
\begin{itemize}
\item $A=\{n^3 \mid n\in \N\}$
\item $B=\{\frac{n^2}{m^2}\mid n,m\in \N ,m\neq 0,\ \ MCD(n,m)=1\}$
\item $C=\{\frac{n^3}{m^3}\mid n,m\in \N,m\neq 0 ,\  \ MCD(n,m)=1\}$
\end{itemize}
\end{Ex}
\vspace{0.4cm}

\noindent
\begin{Ex}\textbf{ 2.}\\
Dato $(\Z_{15}, +)$ calcolare $o(x)$  $\forall x \in \Z_{12}$
\end{Ex}

\vspace{0.4cm}
\noindent
\begin{Ex}\textbf{ 3.}\\
Determinare se $(\Z, \heartsuit)$ \acc e un gruppo, dove:
\begin{itemize}
\item $x\heartsuit y= x-y$
\item $x\heartsuit y= xy-y$
\item $x\heartsuit y= xy-yx$
\end{itemize}
\end{Ex}

\vspace{0.4cm}
\noindent
\begin{Ex}\textbf{ 4.}\\
Si considerino in $\Z_{18}$, $<\bar 3>$ e $<\bar 2>$. Dimostrare che $<\bar 3>\cap <\bar 2>=<\bar 6>$ e dimostrare che $<\bar2>$ \acc e il pi\acc u piccolo sottogruppo contenete sia $\bar 4$ e $\bar 6$
\end{Ex}
\vspace{0.4cm}

\noindent
\begin{Ex}\textbf{ 5.}\\
Sia $S=\R \setminus \{-1\}$ e $\star : S \longrightarrow S$ t.c. $x\star y= xy+x+y$:
\begin{itemize}
\item Provare che $\star$ definisce un'operazione binaria su $S$
\item Dimostrare che $(S,\star)$'e un gruppo. (individuando elemento neurto e inverso del generico elemento $x$)
\item Mostrare perch\acc  e se definisco $\star$ in modo analogo su tutto $\R$, $(\R,\star)$ non \acc e un gruppo
\end{itemize}
\end{Ex}

\vspace{0.4cm}
\noindent
\begin{Ex}\textbf{ 7.}\\
Determinare l'ordine di tutti gli elementi di $S_5$.
\end{Ex}

\noindent
\begin{Ex}\textbf{ 6.}\\
Date le seguenti permutazioni $\sigma$ e $\tau \in S_{10}$, calcolare i prodotti dove necessario, decomporre in cicli disgiunti e calcolare la parit\acc a di $\sigma$, $\tau$, $\sigma\tau$, $\tau\sigma$, $\sigma^2$, $\sigma^2\tau$, $\tau^2$, $\tau^2\sigma$.
$$\sigma=\left(\begin{array}{cccccccccc} 1 & 2 & 3 & 4 & 5 & 6 & 7 & 8 & 9 & 10 \\ 2 & 4 & 5 & 7 & 9 & 10 & 8 & 6 & 3 & 1 \end{array} \right )$$
$$\tau=\left(\begin{array}{cccccccccc} 1 & 2 & 3 & 4 & 5 & 6 & 7 & 8 & 9 & 10 \\ 2 & 5 & 7 & 9 & 3 & 1 & 4 & 6 & 10 & 8 \end{array} \right )$$
\end{Ex}



\end{document}
