\nopagenumbers \font\title=cmti12
\def\ve{\vfill\eject}
\def\vv{\vfill}
\def\vs{\vskip-2cm}
\def\vss{\vskip10cm}
\def\vst{\vskip13.3cm}

%\def\ve{\bigskip\bigskip}
%\def\vv{\bigskip\bigskip}
%\def\vs{}
%\def\vss{}
%\def\vst{\bigskip\bigskip}

\hsize=19.5cm
\vsize=27.58cm
\hoffset=-1.6cm
\voffset=0.5cm
\parskip=-.1cm
\ \vs \hskip -6mm AL210 AA10/11\ (Algebra: gruppi, anelli e campi)\hfill APPELLO B \hfill Roma, 23 Febbraio 2011. \hrule
\bigskip\noindent
{\title COGNOME}\  \dotfill\ {\title NOME}\ \dotfill {\title
MATRICOLA}\ \dotfill\
\smallskip  \noindent
Risolvere il massimo numero di esercizi accompagnando le risposte
con spiegazioni chiare ed essenziali. \it Inserire le risposte
negli spazi predisposti. NON SI ACCETTANO RISPOSTE SCRITTE SU
ALTRI FOGLI.
\rm 1 Esercizio = 4 punti. Tempo previsto: 2 ore. Nessuna domanda
durante la prima ora e durante gli ultimi 20 minuti.
\smallskip
\hrule\smallskip
\centerline{\hskip 6pt\vbox{\tabskip=0pt \offinterlineskip
\def \trl{\noalign{\hrule}}
\halign to277pt{\strut#& \vrule#\tabskip=0.7em plus 1em& \hfil#&
\vrule#& \hfill#\hfil& \vrule#& \hfil#& \vrule#& \hfill#\hfil&
\vrule#& \hfil#& \vrule#& \hfill#\hfil& \vrule#& \hfil#& \vrule#&
\hfill#\hfil& \vrule#& \hfil#& \vrule#& \hfill#\hfil& \vrule#&
\hfil#& \vrule#& \hfill#\hfil& \vrule#& \hfil#& \vrule#& \hfil#&
\vrule#\tabskip=0pt\cr\trl && FIRMA && 1 && 2 && 3 && 4 &&
5 && 6 && 7 && 8  &&  TOT. &\cr\trl && &&   &&
&&     &&   &&     &&   &&   &&    && &\cr &&
\dotfill &&       &&   &&   &&     &&   && && && &&
&\cr\trl }}}
\medskip

\item{1.} Rispondere alle sequenti domande fornendo una giustificazione di una riga:\bigskip\bigskip\bigskip


\itemitem{a.} \`E vero che tutti i gruppi con 101 elementi sono ciclici?\medskip\bigskip\bigskip

\ \dotfill\ \bigskip\bigskip\bigskip\vfil

\itemitem{b.} \`E vero che $S_4$ contiene due sottogruppi con $4$ elementi tra di loro non isomorfi?\medskip\bigskip\bigskip

\ \dotfill\ \bigskip\bigskip\bigskip\vfil

\itemitem{c.} \`E vero esistono anelli non commutativi in cui tutti gli elementi non nulli sono invertibili?\medskip\bigskip\bigskip
 
\ \dotfill\ \bigskip\bigskip\bigskip\vfil

\itemitem{d.} \`E vero che esistono anelli a fattorizzazione unica in cui non tutti gli ideali sono principali?\medskip\bigskip\bigskip

\ \dotfill\ \bigskip\bigskip\bigskip


\vfil\eject

\item{2.} Dopo aver definito la nozione di sottogruppo, dimostrare che ${\bf Z}_p\times {\bf Z}_p$ ($p$ primo)
ammette esattamente $p+3$ sottogruppi.\vv

\item{3.} Siano $G$ un gruppo e $H$ un suo sottogruppo. Dimostrare che $N_G(H) := \{
g\in G : gH = Hg\}$ \`e un sottogruppo di $G$ che contiene $H$ come
sottogruppo normale.\ve\vs

\item{4.} Immergere il gruppo di Klein ${\bf Z}_2\times{\bf Z}_2$ nel gruppo $S_8$
delle permutazioni su $8$ elementi.\vv

\item{5.} Sia $S$ un insieme non vuoto, $A$ un anello commutativo, $(A^S, + ,
\cdot)$ l'anello delle applicazioni di $S$ in $A$ con le operazioni definite
puntualmente. Si dimostri che se $I$ \`e un ideale di $A$,  allora $\{ f \in
A^S : f(S)\subseteq I \}$ \`e un ideale di $A^S$.\ve\vs


\item{6.} Dopo aver fornito la definizione di dominio di dominio Euclideo (ED), dimostrare che se $A$ \`e un
campo, allora l'anello dei polinomi $K[X]$ \`e un PID. 
\vv

\item{7.} Sia $A=\left\{\pmatrix{a &b\cr b&a}, a,b\in{\bf Z}_4\right\}$. Dopo aver verificato che $A$ \`e
un sottoanello di $M_2({\bf Z}_4)$, contarne il numero di elementi, determinare il gruppo $U(A)$ e verificare
se $U(A)\cong {\bf Z}_2\times{\bf Z}_2$.\vv

\item{8.} Sia $A$ un anello commutativo (unitario). Si dimostri che $J_0:=\{ z \in A
: z^n = 0,$  per qualche $n\in {\bf N}^+\}$ \`e un ideale di $A$. Dimostrare
poi che ogni ideale primo di $A$ contiene $J_0$.
\ve \vs
 \bye

\item{9.} Considerare $f(x)=X^3+2X^2+X+2\in{\bf Z}/3{\bf Z}[X]$. Dimostrare che ${\bf Z}/3{\bf Z}[X]/f(X)$
non \`e un campo esibendo un elemento che non \`e invertibile. Quanti elementi ha ${\bf Z}/3{\bf Z}[X]/f(X)$?
\ \vst
