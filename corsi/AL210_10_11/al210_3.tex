\documentclass[italian,a4paper,11pt]
{article}
\usepackage{babel,amsmath,amssymb,amsbsy,amsfonts,latexsym,exscale,
amsthm,epsf,colordvi,enumerate}

\usepackage[latin1]{inputenc}
\usepackage[all]{xy}
\usepackage{textcomp}
\usepackage{graphicx} 


\newcommand{\Q}{\mathbb{Q}}
\newcommand{\Z}{\mathbb Z}
\newcommand{\R}{\mathbb{R}}
\newcommand{\PP}{\mathbb{P}}
\newcommand{\A}{\mathbb{A}}
\newcommand{\I}{\mathcal{I}}

\newcommand{\F}{\mathbb{F}}
\newcommand{\N}{\mathbb{N}}
\newcommand{\C}{\mathbb{C}}
\newcommand{\T}{\mathcal{T}}
\newcommand{\Zeri}{\mathcal{Z}}
\newcommand{\U}{\mathcal{U}}
\newcommand{\p}{\mathfrak{p}}
\newcommand{\ga}{\mathfrak{a}}
\newcommand{\gb}{\mathfrak{b}}

\newcommand{\q}{\mathfrak{q}}
\newcommand{\m}{\mathfrak{m}}
\newcommand{\X}{\mathbf{X}}

\newcommand{\D}{\mbox{\rm{\textbf{Dom}}}}
\newcommand{\Ze}{\mbox{\rm{\textbf{Rie}}}}

\newcommand{\esse}{\mbox{\rm{\textbf{Spec}}}}
\newcommand{\Ci}{\mathbf{C}}
\newcommand{\Ex}{\textbf{Esercizio}}


\newcommand{\Sse}{\Longleftrightarrow}
\newcommand{\sse}{\Leftrightarrow}
\newcommand{\implica}{\Rightarrow}

\newcommand{\frecdl}{\longrightarrow}
\newcommand{\frecd}{\rightarrow}
\newcommand{\st}{\scriptstyle}
\newcommand{\svol}{\textbf{Svolgimento:}}
\newcommand{\cvd}{\begin{flushright} \qed \end{flushright}}
\newcommand{\acc}{\`}
\begin{document}
\begin{center}

\textbf{Universit\`a degli Studi Roma Tre}\\

\textbf{Corso di Laurea in Matematica, a.a. 2010/2011}\\

\textbf{AL210 - Algebra 2: Gruppi, Anelli e Campi}\\

\textbf{Prof. F. Pappalardi}\\

\textbf{Tutorato 3 - 11 Ottobre 2010}\\

\textbf{Tutore: Matteo Acclavio}\\

www.matematica3.com\\
\end{center}

\vspace{0.2cm}
\noindent
\begin{Ex}\textbf{ 1.}\\
Sia $(G, \cdot)$ un gruppo abeliano e $a,b \in G$ dimostrare se sono vere le seguenti affermazioni (in caso contrario fornire un controesempio):
\begin{itemize}
\item $o(a)=m, o(b)=n \Rightarrow o(ab)=MCD(n,m)$.
\item $o(a)=\infty,\; o(b)$ finito $\Rightarrow o(ab)=\infty$.
\item $o(a)=\infty ,\; o(b)=\infty \Rightarrow o(ab)=\infty $
\end{itemize}
\end{Ex}

\vspace{0.2cm}
\noindent
\begin{Ex}\textbf{ 2.}\\
Sia $(G,\cdot )$ gruppo e $\sim_\gamma$ la relazione cos\acc i definita:
$g \sim _\gamma h \Leftrightarrow g=x^{-1}hx\;\exists x \in G$.\\
Dimostare che $\sim_\gamma$  \acc e una relazione di equivalenza
\end{Ex}

\vspace{0.2cm}
\noindent
\begin{Ex}\textbf{ 3.}\\
Sia $G$ un gruppo e sia $x\in G$. Mostrare che:
\begin{description}
	\item[i)] L'insieme $C(x):=\{g\in G\ \mid \ gx=xg\}$ \acc e un sottogruppo di $G$. Questo sottogruppo si chiama  il $centralizzante$ di $x$.
		\item[ii)] $C(x)$ pu\acc o non essere normale in $G$;
		\item[iii)] $Z(G)=\displaystyle \bigcap_{x\in G} C(x)$;
		\item[iv)] $gxg^{-1}=hxh^{-1}$ se e soltanto se $gC(x)=hC(x)$. Dedurne che il numero dei coniugati distinti di $x$ \acc e $\left|G\right| /\left| C(x)\right|$.
		\end{description}
\end{Ex}


\vspace{0.2cm}
\noindent
\begin{Ex}\textbf{ 4.}\\
Dimostrare  se $N$ \acc e normale in $H$ e se $H$ \acc e normale in $GL_2(\R)$ dove:\\ 
$N=\left\{ \left(\begin{array}{cc}1 &b\\0 & 1\end{array}\right) | b\in \R\right\}$\\
$H=\left\{ \left(\begin{array}{cc}a&b\\0 & d\end{array}\right) | a,b,c\in \R ab\neq 0 \right\}.$
\end{Ex}


\vspace{0.2cm}
\noindent
\begin{Ex}\textbf{ 5.}\\
Dimostrare  che $H$ \acc e normale in $V_4$ e $V_4$ \acc e normale in $A_4$ ma $H$ non \acc e normale in $A_4$dove:\\ 
$H=<(12)(34)>$\\
$V_4=<(12)(34),(14)(23)>$
\end{Ex}


		



\end{document}
