\nopagenumbers \font\title=cmti12
\def\ve{\vfill\eject}
\def\vv{\vfill}
\def\vs{\vskip-2cm}
\def\vss{\vskip10cm}
\def\vst{\vskip13.3cm}

%\def\ve{\bigskip\bigskip}
%\def\vv{\bigskip\bigskip}
%\def\vs{}
%\def\vss{}
%\def\vst{\bigskip\bigskip}

\hsize=19.5cm
\vsize=27.58cm
\hoffset=-1.6cm
\voffset=0.5cm
\parskip=-.1cm
\ \vs \hskip -6mm AL210 AA10/11\ (Algebra: gruppi, anelli e campi)\hfill APPELLO X \hfill Roma, 16 Settembre 2011. \hrule
\bigskip\noindent
{\title COGNOME}\  \dotfill\ {\title NOME}\ \dotfill {\title
MATRICOLA}\ \dotfill\
\smallskip  \noindent
Risolvere il massimo numero di esercizi accompagnando le risposte
con spiegazioni chiare ed essenziali. \it Inserire le risposte
negli spazi predisposti. NON SI ACCETTANO RISPOSTE SCRITTE SU
ALTRI FOGLI.
\rm 1 Esercizio = 4 punti. Tempo previsto: 2 ore. Nessuna domanda
durante la prima ora e durante gli ultimi 20 minuti.
\smallskip
\hrule\smallskip
\centerline{\hskip 6pt\vbox{\tabskip=0pt \offinterlineskip
\def \trl{\noalign{\hrule}}
\halign to277pt{\strut#& \vrule#\tabskip=0.7em plus 1em& \hfil#&
\vrule#& \hfill#\hfil& \vrule#& \hfil#& \vrule#& \hfill#\hfil&
\vrule#& \hfil#& \vrule#& \hfill#\hfil& \vrule#& \hfil#& \vrule#&
\hfill#\hfil& \vrule#& \hfil#& \vrule#& \hfill#\hfil& \vrule#&
\hfil#& \vrule#& \hfill#\hfil& \vrule#& \hfil#& \vrule#& \hfil#&
\vrule#\tabskip=0pt\cr\trl && FIRMA && 1 && 2 && 3 && 4 &&
5 && 6 && 7 && 8  &&  TOT. &\cr\trl && &&   &&
&&     &&   &&     &&   &&   &&    && &\cr &&
\dotfill &&       &&   &&   &&     &&   && && && &&
&\cr\trl }}}
\medskip

\item{1.} Rispondere alle sequenti domande fornendo una giustificazione di una riga:\bigskip\bigskip\bigskip


\itemitem{a.} \`E vero che $({\bf Z},+)$ \`e (a meno di isomorfismi) l'unico
gruppo ciclico infinito?\medskip\bigskip\bigskip

\ \dotfill\ \bigskip\bigskip\bigskip\vfil

\itemitem{b.} \`E vero che se $H_1,H_2\le S_n$ e $\#H_1=\#H_2$ allore $H_1\cong H_2$?\medskip\bigskip\bigskip

\ \dotfill\ \bigskip\bigskip\bigskip\vfil

\itemitem{c.} \`E vero esistono corpi infiniti non commutativi?\medskip\bigskip\bigskip
 
\ \dotfill\ \bigskip\bigskip\bigskip\vfil

\itemitem{d.} \`E vero che se $K$ \`e un campo, allora $U(K[X])=K^*$?\medskip\bigskip\bigskip

\ \dotfill\ \bigskip\bigskip\bigskip


\vfil\eject

\item{2.} Dimostrare che se $G$ \`e un gruppo abeliano con $35$ elementi, allora \`e necessariamente
ciclico.\vv

\item{3.} Si considerino i seguenti insiemi:
$G_1 =(\{ 3^a 5^b, a, b\in{\bf Z}\},\cdot)$, sottogruppo di $({\bf Q},\cdot)$
e
$G_2 =(\{a + ib, a, b \in{\bf Z}\}, +)$ sottogruppo di $({\bf C}, +)$. 
Dopo aver dimostrato che sono gruppi ed averne individuato gli elementi neutri, si dimostri 
che $G_1$ e $G_2$ sono isomorfi.\ve\vs

\item{4.} Dopo aver enunciato il
Teorema di Lagrange per gruppi finiti, considerare 
$D_4 = \langle(1, 2, 3, 4), (1, 3)\rangle \le S_4$. Determinare tutte le classi laterali destre di $D_4$ in $S_4$.\vv

\item{5.} Si consideri l'insieme
$$A = \left\{{m\over 1 + 2n}, m, n \in\in{\bf Z}\right\}.$$
Dimostrare che $A$, con le usuali operazioni di somma e moltiplicazione tra numeri, \`e un anello commutativo.
Determinare $U(A)$ e dimostrare che $A\setminus U(A)$ \`e un ideale di $A$\ve\vs

\item{6.} Dopo aver richiamato la defnizione di ideale in un anello commutativo unitario, dimostrare che l'intersezione di una qualsiasi
famiglia di ideali \`e un ideale mentre non \`e detto che l'unione di ideali sia un ideale.\vv

\item{7.} Dopo aver fornito la definizione di dominio euclideo (ED), si dimostri che
${\bf Z}[i]$ \`e un ED e calcolare la fattorizzazione unica di $10$ in ${\bf Z}[i]$ 
\vv

\item{8.} Considerare $f(x)=X^2+1\in{\bf Z}/5{\bf Z}[X]$. Dimostrare che ${\bf Z}/{5\bf Z}[X]/f(X)$
non \`e un campo esibendo un elemento che non \`e invertibile. Quanti elementi ha ${\bf Z}/{5\bf Z}[X]/f(X)$?
\ve \vs
 \bye


\ \vst
