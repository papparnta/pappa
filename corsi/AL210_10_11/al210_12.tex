\documentclass[italian,a4paper,11pt]
{article}
\usepackage{babel,amsmath,amssymb,amsbsy,amsfonts,latexsym,exscale,
amsthm,epsf,colordvi,enumerate}

\usepackage[latin1]{inputenc}
\usepackage[all]{xy}
\usepackage{textcomp}
\usepackage{graphicx} 


\newcommand{\Q}{\mathbb{Q}}
\newcommand{\Z}{\mathbb Z}
\newcommand{\R}{\mathbb{R}}
\newcommand{\PP}{\mathbb{P}}
\newcommand{\A}{\mathbb{A}}
\newcommand{\I}{\mathcal{I}}

\newcommand{\F}{\mathbb{F}}
\newcommand{\N}{\mathbb{N}}
\newcommand{\C}{\mathbb{C}}
\newcommand{\T}{\mathcal{T}}
\newcommand{\Zeri}{\mathcal{Z}}
\newcommand{\U}{\mathcal{U}}
\newcommand{\p}{\mathfrak{p}}
\newcommand{\ga}{\mathfrak{a}}
\newcommand{\gb}{\mathfrak{b}}

\newcommand{\q}{\mathfrak{q}}
\newcommand{\m}{\mathfrak{m}}
\newcommand{\X}{\mathbf{X}}

\newcommand{\D}{\mbox{\rm{\textbf{Dom}}}}
\newcommand{\Ze}{\mbox{\rm{\textbf{Rie}}}}

\newcommand{\esse}{\mbox{\rm{\textbf{Spec}}}}
\newcommand{\Ci}{\mathbf{C}}
\newcommand{\Ex}{\textbf{Esercizio}}


\newcommand{\Sse}{\Longleftrightarrow}
\newcommand{\sse}{\Leftrightarrow}
\newcommand{\implica}{\Rightarrow}

\newcommand{\frecdl}{\longrightarrow}
\newcommand{\frecd}{\rightarrow}
\newcommand{\st}{\scriptstyle}
\newcommand{\svol}{\textbf{Svolgimento:}}
\newcommand{\cvd}{\begin{flushright} \qed \end{flushright}}
\newcommand{\acc}{\`}
\begin{document}
\begin{center}

\textbf{Universit\`a degli Studi Roma Tre}\\

\textbf{Corso di Laurea in Matematica, a.a. 2010/2011}\\

\textbf{AL210 - Algebra 2: Gruppi, Anelli e Campi}\\

\textbf{Prof. F. Pappalardi}\\

\textbf{Tutorato 12  - Gennaio 2011}\\

\textbf{Tutore: Matteo Acclavio}\\

www.matematica3.com\\
\end{center}

\vspace{0.4 cm}
\noindent
\begin{Ex}\textbf{ 1.}\\
Sia $R$ un anello commutativo ed unitario. Siano $I,J$ due suoi ideali.\\
Sia $I + J := \{x + y$ con $x \in I, y \in J\}$. Sia $\phi : R\longrightarrow R/I \times R/J$, l'applicazione definita come $\phi(r):= (r + I, r + J)$ per ogni $r\in R$.
\begin{itemize}
\item Si dimostri che $I + J$ \acc e un ideale di $R$.
\item Si dimostri che $\phi$ \acc e un omomorfismo unitario di anelli.
\item Si dimostri che $\phi$ \acc e suriettivo se e solo se $I + J = R$.
\item Si dimostri che il nucleo di $\phi$ \acc e $I\cap J$.
\item Nel caso $R = \Z$, $I = 5\Z, J = 12\Z$, si dimostri che\\ $\Z/60\Z = \Z/5\Z \times \Z/12\Z$.
\end{itemize}
\end{Ex}

\vspace{0.4cm}
\noindent
\begin{Ex}\textbf{ 2.}
\begin{description}
	\item[(a)]Quali elementi del campo $\Z_7$ sono quadrati perfetti? (Cio\`e quali \\ elementi $q\in \Z_7$ sono tali che l'equazione $X^2=q$ ha soluzione in $\Z_7$?).
	\item[(b)] Dedurre da \textbf{(a)} per quali valori $a\in \Z_7$ si ha che $\Z_7[X]/(X^2+a)$ \`e un campo;
	\item[(c)] Determinare l'inverso del generico elemento di $\Z_7[X]/(X^2+4)$;
	\item[(d)] Determinare i divisori dello zero in $\Z_7[X]/(X^2+3)$.
\end{description}
\end{Ex}

\vspace{0.4cm}
\noindent
\begin{Ex}\textbf{ 3.}\\
Sia $A$ l'insieme dei numeri complessi del tipo $a+ib\sqrt{5}$, con $a,b \in \Z$.
\begin{itemize}
	\item Provare che $A$ \`e un sottoanello di $\C$.
	\item Stabilire se $A$ \`e un sottocampo di $\C$.
	\item Provare che $I=\{a+ib\sqrt{5}\mid a,b\in 2\Z\}$ \`e un ideale di $A$, ma non \`e un ideale primo di $A$.
	\item Provare che $J=\{a+ib\sqrt{5}\mid a\in 5\Z\}$ \`e un ideale massimale di $A$.
\end{itemize}
\end{Ex}

\vspace{0.4cm}
\noindent
\begin{Ex}\textbf{ 4.}\\
Sia $A$ anello commutativo unitario che ammette (per ogni suo elemento non nullo) una fattorizzazione in irriducibili, le seguenti sono equivalenti
\begin{description}
	\item[(a)] la fattorizzazione \acc e essenzialemente unica (a meno dell'ordine dei fattori e di invertibili)
	\item[(b)] $\forall a,b \in A \ \exists MCD(a,b)$
	\item[(c)] $p\in A, p\ irriducibile \Rightarrow p  \ primo$
\end{description}
\end{Ex}

\end{document}

