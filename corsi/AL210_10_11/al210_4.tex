\documentclass[italian,a4paper,11pt]
{article}
\usepackage{babel,amsmath,amssymb,amsbsy,amsfonts,latexsym,exscale,
amsthm,epsf,colordvi,enumerate}

\usepackage[latin1]{inputenc}
\usepackage[all]{xy}
\usepackage{textcomp}
\usepackage{graphicx} 


\newcommand{\Q}{\mathbb{Q}}
\newcommand{\Z}{\mathbb Z}
\newcommand{\R}{\mathbb{R}}
\newcommand{\PP}{\mathbb{P}}
\newcommand{\A}{\mathbb{A}}
\newcommand{\I}{\mathcal{I}}

\newcommand{\F}{\mathbb{F}}
\newcommand{\N}{\mathbb{N}}
\newcommand{\C}{\mathbb{C}}
\newcommand{\T}{\mathcal{T}}
\newcommand{\Zeri}{\mathcal{Z}}
\newcommand{\U}{\mathcal{U}}
\newcommand{\p}{\mathfrak{p}}
\newcommand{\ga}{\mathfrak{a}}
\newcommand{\gb}{\mathfrak{b}}

\newcommand{\q}{\mathfrak{q}}
\newcommand{\m}{\mathfrak{m}}
\newcommand{\X}{\mathbf{X}}

\newcommand{\D}{\mbox{\rm{\textbf{Dom}}}}
\newcommand{\Ze}{\mbox{\rm{\textbf{Rie}}}}

\newcommand{\esse}{\mbox{\rm{\textbf{Spec}}}}
\newcommand{\Ci}{\mathbf{C}}
\newcommand{\Ex}{\textbf{Esercizio}}


\newcommand{\Sse}{\Longleftrightarrow}
\newcommand{\sse}{\Leftrightarrow}
\newcommand{\implica}{\Rightarrow}

\newcommand{\frecdl}{\longrightarrow}
\newcommand{\frecd}{\rightarrow}
\newcommand{\st}{\scriptstyle}
\newcommand{\svol}{\textbf{Svolgimento:}}
\newcommand{\cvd}{\begin{flushright} \qed \end{flushright}}
\newcommand{\acc}{\`}
\begin{document}
\begin{center}

\textbf{Universit\`a degli Studi Roma Tre}\\

\textbf{Corso di Laurea in Matematica, a.a. 2010/2011}\\

\textbf{AL210 - Algebra 2: Gruppi, Anelli e Campi}\\

\textbf{Prof. F. Pappalardi}\\

\textbf{Tutorato 4 - 18 Ottobre 2010}\\

\textbf{Tutore: Matteo Acclavio}\\

www.matematica3.com\\
\end{center}

\vspace{0.4 cm}
\noindent
\begin{Ex}\textbf{ 2.}\\
Sia $G := \left\{\left(\begin{matrix} a & 0 \\ c & b \end{matrix}\right) \mid a, b, c \in \Z_3 , a \neq 0, b \neq 0 \right\}$ \\
Provare che:
\begin{itemize}
\item $G$ con l'usuale moltiplicazione fra matrici \acc e un gruppo e dire se $G$ \acc e abeliano.
\item $H := \left\{M \in G \mid det(M) =1\right\}$ \acc e un sottogruppo di $G$.
\item $H$ \acc e un sottogruppo normale.
\item $H$ \acc e ciclico e trovare un suo generatore.
\item Ogni elemento di $G$ che non sta in $H$ ha ordine 2.
\item Determinare il centro di $G$.
\end{itemize}
\end{Ex}

\vspace{0.4cm}
\noindent
\begin{Ex}\textbf{ 3.}\\
Sia $f_n : (\Z, +) \frecdl (\Z, +)$ definita da $f_n(x) = nx$. Verificare che $f_n$ \'e
un omomorfismo, trovare il nucleo e l'immagine di $f_n$.
\end{Ex}


\vspace{0.4 cm}
\noindent
\begin{Ex}\textbf{ 4.}\\
Mostra che l'applicazione $Re: (\C, +) \longrightarrow (\R, +)$ definita da $Re(a+ib)=a$ \acc e un omomorfismo di gruppi. Determinarne il nucleo $N$ e l'immagine $H$. Applicando il teorema di omomorfismo, definire l'isomorfismo canonico.
\end{Ex}


\vspace{0.4cm}
\noindent
\begin{Ex}\textbf{ 5.}\\
Sia $G$ un gruppo e sia $H$ un suo sottogruppo. Definiamo $N(H)=\{x\in G \mid xHx^{-1}=H\}$. Dimostrare che $N(H)$ \`e un sottogruppo di $G$ e che $H$ \`e normale in $N(H)$. $N(H)$ si dice $normalizzante$ di $H$ in $G$ ed \`e il pi\`u grande sottogruppo di $G$ in cui $H$ \`e normale (verificare che se $H$ normale in $G$ allora $N(H)=G$).
\end{Ex}


\vspace{0.4cm}
\noindent
\begin{Ex}\textbf{ 6.}\\
Sia $\varphi: G\longrightarrow G'$ omomorfismo. Dimostrare che:
\begin{itemize}
\item $\forall g \in G, \; o(g)\mid |G|$
\item $\forall g' \in G', \; o(g')\mid |G'|$
\item $\forall g \in G ,\; o(\varphi(g))\mid o(g)$
\item se $G=<g>$ allora $\varphi(G)=<\varphi(g)>$
\end{itemize}
Determinare gli elementi di $Hom(\Z_n, \Z_m)=\{$omomorfismi da $\Z_n$ in $\Z_m\}$.
Dimostrare che il gruppo $Aut(Z_n)$ \`e isomorfo a $(\U(\Z_n), \cdot)$
\end{Ex}



\end{document}
