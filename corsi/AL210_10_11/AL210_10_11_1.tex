w.\nopagenumbers \font\title=cmti12
\def\ve{\vfill\eject}
\def\vv{\vfill}
\def\vs{\vskip-2cm}
\def\vss{\vskip10cm}
\def\vst{\vskip13.3cm}

%\def\ve{\bigskip\bigskip}
%\def\vv{\bigskip\bigskip}
%\def\vs{}
%\def\vss{}
%\def\vst{\bigskip\bigskip}

\hsize=19.5cm
\vsize=27.58cm
\hoffset=-1.6cm
\voffset=0.5cm
\parskip=-.1cm
\ \vs \hskip -6mm AL210 AA10/11\ (Algebra: gruppi, anelli e campi)\hfill ESAME DI MET\`{A} SEMESTRE \hfill Roma, 2 Novembre 2010. \hrule
\bigskip\noindent
{\title COGNOME}\  \dotfill\ {\title NOME}\ \dotfill {\title
MATRICOLA}\ \dotfill\
\smallskip  \noindent
Risolvere il massimo numero di esercizi accompagnando le risposte
con spiegazioni chiare ed essenziali. \it Inserire le risposte
negli spazi predisposti. NON SI ACCETTANO RISPOSTE SCRITTE SU
ALTRI FOGLI. Scrivere il proprio nome anche nell'ultima pagina.
\rm 1 Esercizio = 4 punti. Tempo previsto: 2 ore. Nessuna domanda
durante la prima ora e durante gli ultimi 20 minuti.
\smallskip
\hrule\smallskip
\centerline{\hskip 6pt\vbox{\tabskip=0pt \offinterlineskip
\def \trl{\noalign{\hrule}}
\halign to277pt{\strut#& \vrule#\tabskip=0.7em plus 1em& \hfil#&
\vrule#& \hfill#\hfil& \vrule#& \hfil#& \vrule#& \hfill#\hfil&
\vrule#& \hfil#& \vrule#& \hfill#\hfil& \vrule#& \hfil#& \vrule#&
\hfill#\hfil& \vrule#& \hfil#& \vrule#& \hfill#\hfil& \vrule#&
\hfil#& \vrule#& \hfill#\hfil& \vrule#& \hfil#& \vrule#& \hfil#&
\vrule#\tabskip=0pt\cr\trl && FIRMA && 1 && 2 && 3 && 4 &&
5 && 6 && 7 && 8 &&   TOT. &\cr\trl && &&   &&
&&     &&   &&   &&   &&   &&    && &\cr &&
\dotfill &&     &&   &&   &&     &&   && && && &&
&\cr\trl }}}
\medskip

\item{1.} Determinare il numero di elementi di ordine $2$ in $S_5$ e in $S_6$.\vv

\item{2.} Sia $D_4=\langle (1,2,3,4), (1,3)\rangle\le S_4$. Determinare tutte le classi 
laterali destre di $D_4$ in $S_4$.\ve\vs

\item{3.} Dimostrare che ogni gruppo abeliano con $77$ elementi \`e necessariamente ciclico.\vv

\item{4.} Determinare tutti gli omomorfismi tra i gruppi ${\bf Z}_{12}$ e ${\bf Z}_{30}$.\ve\vs

\item{5.} Dimostrare che l'insieme degli automorfismi interni Inn($G$) di un gruppo $G$ \`e un sottogruppo
del gruppo degli automorfismi. Dimostrare che tale gruppo \`e banale se e solo se il gruppo $G$ 
\`e abeliano.\vv

\item{6.} Dimostrare che il centro del prodotto diretto di due gruppi \`e il prodotto dei centri.
\ve \vs

\item{7.} Dimostrare che se $G$ \`e un gruppo ciclico finito e $d\mid |G|$, allora $G$ ammette esattamente
un sottogruppo con $d$ elementi.
\vv

\item{8.} Dopo aver calcolato il numero di elementi di $G={\rm GL}_3({\bf F}_5)$, calcolare il numero di elementi del suo centro e
dimostrare che ammette un sottogruppo normale di indice $4$.
\ \ve \vs
\bye
