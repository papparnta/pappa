\nopagenumbers \font\title=cmti12
\def\ve{\vfill\eject}
\def\vv{\vfill}
\def\vs{\vskip-2cm}
\def\vss{\vskip10cm}
\def\vst{\vskip13.3cm}

%\def\ve{\bigskip\bigskip}
%\def\vv{\bigskip\bigskip}
%\def\vs{}
%\def\vss{}
%\def\vst{\bigskip\bigskip}

\hsize=19.5cm
\vsize=27.58cm
\hoffset=-1.6cm
\voffset=0.5cm
\parskip=-.1cm
\ \vs \hskip -6mm AL210 AA10/11\ (Algebra: gruppi, anelli e campi)\hfill APPELLO A \hfill Roma, 11 Gennaio 2011. \hrule
\bigskip\noindent
{\title COGNOME}\  \dotfill\ {\title NOME}\ \dotfill {\title
MATRICOLA}\ \dotfill\
\smallskip  \noindent
Risolvere il massimo numero di esercizi accompagnando le risposte
con spiegazioni chiare ed essenziali. \it Inserire le risposte
negli spazi predisposti. NON SI ACCETTANO RISPOSTE SCRITTE SU
ALTRI FOGLI.
\rm 1 Esercizio = 4 punti. Tempo previsto: 2 ore. Nessuna domanda
durante la prima ora e durante gli ultimi 20 minuti.
\smallskip
\hrule\smallskip
\centerline{\hskip 6pt\vbox{\tabskip=0pt \offinterlineskip
\def \trl{\noalign{\hrule}}
\halign to277pt{\strut#& \vrule#\tabskip=0.7em plus 1em& \hfil#&
\vrule#& \hfill#\hfil& \vrule#& \hfil#& \vrule#& \hfill#\hfil&
\vrule#& \hfil#& \vrule#& \hfill#\hfil& \vrule#& \hfil#& \vrule#&
\hfill#\hfil& \vrule#& \hfil#& \vrule#& \hfill#\hfil& \vrule#&
\hfil#& \vrule#& \hfill#\hfil& \vrule#& \hfil#& \vrule#& \hfil#&
\vrule#\tabskip=0pt\cr\trl && FIRMA && 1 && 2 && 3 && 4 &&
5 && 6 && 7 && 8  &&  TOT. &\cr\trl && &&   &&
&&     &&   &&     &&   &&   &&    && &\cr &&
\dotfill &&       &&   &&   &&     &&   && && && &&
&\cr\trl }}}
\medskip

\item{1.} Rispondere alle sequenti domande fornendo una giustificazione di una riga:\bigskip\bigskip\bigskip


\itemitem{a.} \`E vero che tutti i gruppi ciclici infiniti sono isomorfi?\medskip\bigskip\bigskip

\ \dotfill\ \bigskip\bigskip\bigskip\vfil

\itemitem{b.} \`E vero che ogni gruppo finito \`e isomorfo a un sottogruppo di GL$_n({\bf Z}_2)$ per 
un opportuno $n\in{\bf N}$?\medskip\bigskip\bigskip

\ \dotfill\ \bigskip\bigskip\bigskip\vfil

\itemitem{c.} \`E vero che due anelli finiti con lo stesso numero di elmementi sono isomorfi?\medskip\bigskip\bigskip
 
\ \dotfill\ \bigskip\bigskip\bigskip\vfil

\itemitem{d.} \`E vero che qualsiasi dominio euclideo esiste sempre il massimo comun divisore di due 
qualsiasi elementi non nulli?\medskip\bigskip\bigskip

\ \dotfill\ \bigskip\bigskip\bigskip


\vfil\eject

\item{2.} Descrivere tutti i sottogruppi di $S_3\times {\bf Z}_2$.\vv

\item{3.} Dimostrare che il gruppo quoziente ${\bf R}/{\bf Z}$ \`e isomorfo a gruppo dei 
numeri complessi di norma $1$.\ve\vs

\item{4.} Sia $\varphi: G\rightarrow H$ un omomorfismo di gruppi finiti. Dimostrare che per ogni $x\in G$, l'ordine
o$(x)$ \`e un divisore di $o(\varphi(x))$ e che se $o(\varphi(x))=o(x)$ per ogni $x\in G$, allora $\varphi$ \`e iniettivo.\vv

\item{5.} Sia $G=$GL$_3({\bf Z}_5)$. Dopo aver determinato il centro di $G$ e l'ordine di $G$, determinare un sottogruppo
non abeliano di $G$ con $125$ elementi.\ve\vs

\item{6.} Dopo aver fornito la definizione di dominio a ideali principali (PID), dimostrare che se $A$ \`e un
PID, allora per ogni $a,b\in A^*$ esiste un MCD$(a,b)$ . 
\vv

\item{7.} Sia $A=\left\{\pmatrix{a &5b\cr 4b&a}, a,b\in{\bf Z}_8\right\}$. Dopo aver verificato che $A$ \`e
un sottoanello di $M_2({\bf Z}_8)$, contarne il numero di elementi e dire se $A$ \`e un dominio di integrit\`a.\vv


\item{8.} Determinare tutti i divisori dello zero dell'anello  $({\bf Z}/5{\bf Z})[X]/(x^2-1)$.
\ve \vs

 \bye

\item{9.} Considerare $f(x)=X^3+2X^2+X+2\in{\bf Z}/3{\bf Z}[X]$. Dimostrare che ${\bf Z}/3{\bf Z}[X]/f(X)$
non \`e un campo esibendo un elemento che non \`e invertibile. Quanti elementi ha ${\bf Z}/3{\bf Z}[X]/f(X)$?
\ \vst
