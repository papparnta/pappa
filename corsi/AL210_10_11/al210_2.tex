\documentclass[italian,a4paper,11pt]
{article}
\usepackage{babel,amsmath,amssymb,amsbsy,amsfonts,latexsym,exscale,
amsthm,epsf,colordvi,enumerate}

\usepackage[latin1]{inputenc}
\usepackage[all]{xy}
\usepackage{textcomp}
\usepackage{graphicx} 


\newcommand{\Q}{\mathbb{Q}}
\newcommand{\Z}{\mathbb Z}
\newcommand{\R}{\mathbb{R}}
\newcommand{\PP}{\mathbb{P}}
\newcommand{\A}{\mathbb{A}}
\newcommand{\I}{\mathcal{I}}

\newcommand{\F}{\mathbb{F}}
\newcommand{\N}{\mathbb{N}}
\newcommand{\C}{\mathbb{C}}
\newcommand{\T}{\mathcal{T}}
\newcommand{\Zeri}{\mathcal{Z}}
\newcommand{\U}{\mathcal{U}}
\newcommand{\p}{\mathfrak{p}}
\newcommand{\ga}{\mathfrak{a}}
\newcommand{\gb}{\mathfrak{b}}

\newcommand{\q}{\mathfrak{q}}
\newcommand{\m}{\mathfrak{m}}
\newcommand{\X}{\mathbf{X}}

\newcommand{\D}{\mbox{\rm{\textbf{Dom}}}}
\newcommand{\Ze}{\mbox{\rm{\textbf{Rie}}}}

\newcommand{\esse}{\mbox{\rm{\textbf{Spec}}}}
\newcommand{\Ci}{\mathbf{C}}
\newcommand{\Ex}{\textbf{Esercizio}}


\newcommand{\Sse}{\Longleftrightarrow}
\newcommand{\sse}{\Leftrightarrow}
\newcommand{\implica}{\Rightarrow}

\newcommand{\frecdl}{\longrightarrow}
\newcommand{\frecd}{\rightarrow}
\newcommand{\st}{\scriptstyle}
\newcommand{\svol}{\textbf{Svolgimento:}}
\newcommand{\cvd}{\begin{flushright} \qed \end{flushright}}
\newcommand{\acc}{\`}
\begin{document}
\begin{center}

\textbf{Universit\`a degli Studi Roma Tre}\\

\textbf{Corso di Laurea in Matematica, a.a. 2010/2011}\\

\textbf{AL210 - Algebra 2: Gruppi, Anelli e Campi}\\

\textbf{Prof. F. Pappalardi}\\

\textbf{Tutorato 2 - 4 Ottobre 2010}\\

\textbf{Tutore: Matteo Acclavio}\\

www.matematica3.com\\
\end{center}



\noindent
\begin{Ex}\textbf{ 1.}\\
Dimostrare che se $G$ \acc e un gruppo privo di sottogruppi non banali allora \acc e finito ed ha ordine $p$ con $p$ primo.
\end{Ex}

\vspace{0.4cm}
\noindent
\begin{Ex}\textbf{ 2.}\\
Dimostrare che $\Z=\langle p, q \rangle$ con $p,q$ primi.
\end{Ex}

\vspace{0.4cm}
\noindent
\begin{Ex}\textbf{ 3.}\\
Dimostrare che i numeri primi generano $\Q^{\ast}$.
\end{Ex}

\vspace{0.4cm}
\noindent
\begin{Ex}\textbf{ 4.}\\
Sia $G$ un grppo finito di ordine $m$, dimostrare le seguenti propriet\acc a:
\begin{itemize}
	\item $H\leq G$ \acc e l'unico sottogruppo di ordine $n$ $\implica$ $H\triangleleft G$;
	\item $(G:H)=2 $ $\implica$ $H\triangleleft G$.
\end{itemize}
\end{Ex}

\vspace{0.4cm}
\noindent
\begin{Ex}\textbf{ 5.}\\
Sia $(G,\cdot)$ un gruppo. Dimostrare che se $\forall \ g \in G$ si ha che $g\cdot g=1$ allora $G$ \acc e abeliano.
\end{Ex}

\vspace{0.4cm}
\noindent
\begin{Ex}\textbf{ 6.}\\
Verificare che l'insieme dei polinomi a coefficienti in $K$ campo e l'insieme dei polinomi a coefficenti in $\Z$ , con l'operazione di somma, sono e un gruppo. \\
Stabilire se sono abeliani e dimostrare che non sono ciclici. \\
Stabilire, inoltre, se $\langle2,X \rangle =\langle3,X \rangle = \langle 3X \rangle$ e descrivere esplicitamente gli elementi dei due sottogruppi.
\end{Ex}



\vspace{0.4cm}
\noindent
\begin{Ex}\textbf{ 7.}\\
Determinare tutti i sottogruppi, e le loro relative classi destre e sinistre, dei seguenti gruppi verificando il teoreama di Lagrange. Verificare, inoltre, che le classi laterali formano una partizione del gruppo.
Siano:
\begin{itemize}
	\item $(\Z_{18}, +)$;
	\item $(U(\Z_{15}), \cdot)$;
	\item $(\mathcal{F}_{[0,1]}^*, \circ)$ dove $\mathcal{F}_{[0,1]}^*$ \acc e l'insieme delle funzioni $f: [0,1] \longrightarrow \R$ con $f(x)\neq 0 \; \forall x\in [0,1]$.
\end{itemize}
\end{Ex}




\vspace{0.4cm}
\noindent
\begin{Ex}\textbf{ 8.}\\
Sia $G=GL_3(K)$ ove $K$ \acc e un campo con 5 elementi. Calcolare l'ordine di $G$ e dimostrare che:
\begin{itemize}
\item Il gruppo delle matrici diagonali \acc e un sottogruppo non normale di $G$ e se ne determini l'ordine;

\item Il gruppo delle matrici scalari \acc e un sottogruppo
normale di $G$ e se ne determini l'ordine;

\item Il gruppo delle matrici triangolari (superiori o inferiori) \acc e un sottogruppo non normale di $G$ e se ne determini l'ordine;

\item Il gruppo delle matrici triangolari con tutti 1 sulla
diagonale \acc e un sottogruppo non normale di $G$ e se ne determini l'ordine;

\item Il gruppo delle matrici con determinante 1 \acc e un sottogruppo
normale di $G$ e se ne determini l'ordine.
\end{itemize}
In ciascuno dei casi sopra, si stabiliscano eventuali inclusioni dei gruppi
presi in considerazione e si dica se essi sono normali negli eventuali gruppi
contenenti.
\\
\textbf{Per chi soffre di insonnia:} Ripetere quanto fatto sopra per un generico $GL_n(\mathbb{F}_q)$.
\end{Ex}


\end{document}
