\documentclass[italian,a4paper,11pt]
{article}
\usepackage{babel,amsmath,amssymb,amsbsy,amsfonts,latexsym,exscale,
amsthm,epsf,colordvi,enumerate}

\usepackage[latin1]{inputenc}
\usepackage[all]{xy}
\usepackage{textcomp}
\usepackage{graphicx} 


\newcommand{\Q}{\mathbb{Q}}
\newcommand{\Z}{\mathbb Z}
\newcommand{\R}{\mathbb{R}}
\newcommand{\PP}{\mathbb{P}}
\newcommand{\A}{\mathbb{A}}
\newcommand{\I}{\mathcal{I}}

\newcommand{\F}{\mathbb{F}}
\newcommand{\N}{\mathbb{N}}
\newcommand{\C}{\mathbb{C}}
\newcommand{\T}{\mathcal{T}}
\newcommand{\Zeri}{\mathcal{Z}}
\newcommand{\U}{\mathcal{U}}
\newcommand{\p}{\mathfrak{p}}
\newcommand{\ga}{\mathfrak{a}}
\newcommand{\gb}{\mathfrak{b}}

\newcommand{\q}{\mathfrak{q}}
\newcommand{\m}{\mathfrak{m}}
\newcommand{\X}{\mathbf{X}}

\newcommand{\D}{\mbox{\rm{\textbf{Dom}}}}
\newcommand{\Ze}{\mbox{\rm{\textbf{Rie}}}}

\newcommand{\esse}{\mbox{\rm{\textbf{Spec}}}}
\newcommand{\Ci}{\mathbf{C}}
\newcommand{\Ex}{\textbf{Esercizio}}


\newcommand{\Sse}{\Longleftrightarrow}
\newcommand{\sse}{\Leftrightarrow}
\newcommand{\implica}{\Rightarrow}

\newcommand{\frecdl}{\longrightarrow}
\newcommand{\frecd}{\rightarrow}
\newcommand{\st}{\scriptstyle}
\newcommand{\svol}{\textbf{Svolgimento:}}
\newcommand{\cvd}{\begin{flushright} \qed \end{flushright}}
\newcommand{\acc}{\`}
\begin{document}
\begin{center}

\textbf{Universit\`a degli Studi Roma Tre}\\

\textbf{Corso di Laurea in Matematica, a.a. 2010/2011}\\

\textbf{AL210 - Algebra 2: Gruppi, Anelli e Campi}\\

\textbf{Prof. F. Pappalardi}\\

\textbf{Tutorato 5 - 25	 Ottobre 2010}\\

\textbf{Tutore: Matteo Acclavio}\\

www.matematica3.com\\
\end{center}


\vspace{1cm}
\noindent
\begin{Ex}\textbf{ 1.}\\
Dire quale dei seguenti gruppi \`e esprimibile come prodotto diretto o\\
semidiretto di due sottogruppi:\\
\textsl{i)} $(\Z, +)$;\ \ \ \ \ \ \ \ \ \ \ \ \ \ \ \ \textsl{ii)}	 $(\Z_8, +)$;\\
\textsl{iii)}	 $(D_4, \circ)$;\ \ \ \ \ \ \ \ \ \ \ \ \  \textsl{iv)}	 $(\Z_6, +)$;\\
\textsl{v)} $(\C, +)$\ \ \ \ \ \ \ \ \ \ \ \ \ \ \ \  \textsl{vi)} $(\C^{\ast}, \cdot)$.

\end{Ex}


\vspace{0.4cm}
\noindent
\begin{Ex}\textbf{ 2.}\\
Sia $G$ un gruppo di ordine $o(G)=2p^2$, con $p\neq 2$ primo.
Provare che:
\begin{description}
	\item[a)] Se $H$ \acc e un sottogruppo normale proprio di $G$ con $o(H)\neq p$, allora $G/H$ \acc e abeliano;
	\item[b)] Se $o(H)=p$ e $G/H$ \acc e abeliano, allora $G/H$ \acc e ciclico;
	\item[c)] Se $G'$ \acc e un gruppo con $o(G')=2p$, allora per ogni omomorfismo suriettivo $\varphi:G\frecdl G'$, $Ker \varphi$ risulta ciclico;
	\item[d)] Nel caso particolare $G=(\Z_{50},+)$ e $G'=(\Z_{10},+)$, trovare gli omomorfismi suriettivi da $G$ su $G'$ e determinarne il nucleo.
\end{description}
\end{Ex}

\vspace{0.4cm}
\noindent
\begin{Ex}\textbf{ 3.}\\
Sia $G=G_1 \times G_2 \times \cdots \times G_k$, dimostrare che $ord((g_1,\dots,g_k))=m.c.m(ord(g_i))$. Derivare che $G$ \`e ciclico $\sse$ $G_i$ \`e ciclico $\forall i$ e M.C.D($|G_i|,|G_j|$)=1 $\forall i\neq j$.\\
Dimostrare inoltre che $\forall \; H=H_1\times \dots \times H_k $ t.c. $ H_i \trianglelefteq G_i \forall k \Rightarrow H\trianglelefteq G$
\end{Ex}

\vspace{0.4cm}
\noindent
\begin{Ex}\textbf{ 4.}\\
Sia $G$ un gruppo abeliano di ordine dispari, dimostrare che la corrispondenza che manda ogni elemento nel suo quadrato \acc e un automorfismo
\end{Ex}


\vspace{0.4cm}
\noindent
\begin{Ex}\textbf{ 5.}\\
Sia $G=\Z$, $H=6Z$, $N=4Z$ esibire un isomorfismo tra $\frac{\frac{G}{H\cap N}}{\frac{H}{H\cap N}}$ e $Z_6$. 
\end{Ex}


\vspace{0.4cm}
\noindent
\begin{Ex}\textbf{ 5.}\\
Dimostrare che $A_4$ non ha sottogruppi di ordine 6.
\end{Ex}

\vspace{0.4 cm}
\noindent
\begin{Ex}\textbf{ 6.}\\
Siano $(G, +)$ e $(G', +)$ due gruppi abeliani. Sia $Hom(G,G')$ l'insieme degli
omomorfismi da $G$ in $G'$. Si consideri l'applicazione
$$ + : Hom(G,G') \times Hom(G,G') \longrightarrow Hom(G,G') $$
tale che $(\varphi + \psi )(x) := \varphi(x) + \psi(x)$.
\begin{description}
\item[a)] Dimostrare che $+$ \`e effettivamente un'operazione binaria.
\item[b)] Dimostrare che $(Hom(G,G'), +)$ \`e un gruppo abeliano.
\end{description}
Si consideri ora l'applicazione $f : (Hom(Z_n,Z_m), +) \rightarrow (Z_m, +)$ definita
come $f(\varphi) := \varphi([1]_n)$.
\begin{description}
\item[c)] Dimostrare che $f$ \`e un omomorfismo iniettivo di gruppi.
\item[d)] Trovare l'immagine di $f$ e dire a quale gruppo \`e isomorfo $Hom(Z_n,Z_m)$.
\end{description}
Sia ora $Aut(\Z_n)$ l'insieme degli automorfismi di $\Z_n$. Mostrare che:
\begin{description}
\item[e)] $(Aut(\Z_n), +) \subseteq (Hom(\Z_n,\Z_n), +)$ non \`e un sottogruppo
\item[f)] $(Aut(\Z_n), \circ)$ \`e un gruppo
\item[g)] $(Aut(\Z_n), \circ)$ \`e isomorfo a $(U(\Z_n), \cdot)$.
\item[h)] Trovare tutti gli automorfismi di $Z_{18}$
\end{description}
Si consideri infine il gruppo degli endomorfismi di $\Z$. Sia $\nu_a : \Z \rightarrow \Z$ la
moltiplicazione per $a$, i.e. $\nu_a(x) = ax$.
\begin{description}
\item[i)] Dimostrare che per ogni $a \in \Z$, $\nu_a \in Hom(\Z,\Z)$
\item[j)] Applicando il teorema di omomorfismo dire a cosa \`e isomorfo $Hom(\Z,\Z)$
\end{description}
\end{Ex}


\end{document}
