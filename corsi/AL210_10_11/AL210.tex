\input programma.sty
\def\abbrcorso{AL210}  
\def\titolocorso{Algebra 210: Gruppi, Anelli e Campi}
\def\sottotitolo{} 
\def\docente{Prof. Francesco Pappalardi}  
\def\crediti{9} 
\def\semestre{I}
\def\esoneri{1} 
\def\scrittofinale{1}
\def\oralefinale{1}
\def\altreprove{0} 
\Intestazione
\titoloparagr{Teoria dei Gruppi}
Definizione di gruppo e prime propriet\`a. Propriet\`a delle permutazioni. La nozione di sottogruppo. Criterio per la verifica che un sottoinsieme \`e un sottogruppo. I sottogruppi di $S_3$. Intersezione di sottogruppi.
Sottogruppo generato da un sottoinsieme di un gruppo. Gruppi finitamente generati. Gruppi ciclici. Gruppi ciclici e ordini degli elementi. 
Unione di sottogruppi e catene ascendenti di sottogruppi.

Classi laterali (destre e sinistre). Esempi nel gruppo $S_3$ e in ${\bf Z}$. 
Cardinalit\`a delle classi laterali. L'indice di un sottogruppo. Il Teorema di Lagrange e sue conseguenze.
Gruppi ciclici di ordine $p$. Gruppi senza sottogruppi. Esponente di un gruppo. Gruppo diedrale, 
definizione come gruppo delle simmetrie del quadrato e come sottogruppo di $S_4$. Tabella moltiplicativa.
Prodotto diretto di gruppi. Il gruppo dei quaternioni e i suoi sottogruppi.

Sottogruppi normali. Caratterizzazioni, propriet\`a ed esempi. Gruppo generale lineare su un campo GL$_n(K)$ e alcuni suoi sottogruppi.
Il centro di un gruppo e sue propriet\`a. La defizione di gruppo quoziente. Omomorfismi di gruppi. Isomorfismi, propriet\`a degli omomorfismi.
Nuclei di omomorfismi e loro relazione con l'iniettivit\`a dell'omomorfismo e con i sottogruppi normali.

Il teorema di corrispondenza e suoi corollari; il primo teorema di omomorfismo di gruppi. 
Il gruppo ortogonale \`e sottogruppo del gruppo generale lineare; descrizione esplicita del sottogruppo generato da un insieme di elementi; un gruppo in cui ogni elemento ha ordine 2 \`e abeliano. Il secondo e il terzo teorema di omomorfismo.
Il centro del gruppo diedrale $D_n$, il centro del gruppo delle permutazioni $S_n$, il centro del gruppo generale lineare GL$_n(K)$; elementi coniugati hanno lo stesso ordine; il gruppo generale lineare di ordine 2 sul campo con due elementi \`e isomorfo a $S_3$.

Il logaritmo come omomorfismo, applicazioni varie del primo Teorema di Omomorfismo, Il gruppo dei numeri complessi di norma 1 \`e isomorfo al gruppo quoziente ${\bf R}/{\bf Z}$. Il gruppo degli automorfismi di un gruppo, Il sottogruppo degli automorfismi interni.
Propriet\`a dei gruppi ciclici e abeliani. Automorfismi dei gruppi ciclici. Il teorema di classificazione dei gruppi abeliani finiti (senza dimostrazione). Il primo teorema di Sylow (senza dimostrazione). 

\titoloparagr{Teoria degli Anelli}
Definizione di anello e prime propriet\`a. Il gruppo delle unit\`a, divisori dello zero e leggi di cancellazione. Il corpo dei Quaternioni.
La nozione di sottoanello, propriet\`a dei sottoanelli. Sottoanello fondamentale e caratteristica. 
Ideali destri, sinistri e bilateri. Propriet\`a degli ideali.

Anelli senza ideali e campi/corpi. Definizione di Anello quoziente. Ideali Primi e Ideali massimali. Caratterizzazioni mediante i quozienti. 
Omomorfismi di anelli, Nucleo, priezione canonica, Teoreama di corrispondenza di sottoanelli e ideali. 
Il Teorema di Omomorfismo per anelli, secondo e terzo Teorema di omomorfismo per anelli (senza dimostrazione), 
ideali primi e massimali e omomorfismi. 

Il campo dei quozienti di un dominio di integrit\`a.
Costruzione dell'anello dei polinomi $A[X]$, teorema di divisione con resto. 
Similarit\`a tra ${\bf Z}$ e $K[X]$, con $K$ campo: entrambi sono PID, entrambi possiedono MCD e identit\`a di B\'ezout.
Prodotti di anelli, idempotenti centrali e ortogonali. Caratterizzazione dei prodotti diretti. Estensioni semplici e anelli di polinomi. grado. Divisioni in $A[x]$. Campi delle funzioni razionali su un dominio di integrit\`a. 

Elementi primi e irriducibili di un anello. Ogni elemento primo \`e irriducibile. elementi associati. Domini a fattorizzazione unica (UFD). Propriet\`a dei domini a fattorizzazione unica. Catene ascendenti di ideali principali. Esistenza di MCD in un UFD. Esistenza di MCD e identit\`a in un dominio a ideali principali (PID). Dimostrazione che i domini a ideali principali sono a fattorizzazione unica. Lemma di Gauss. Ideali e elementi primi. caratterizzazione degli UFD in termini di elementi irriducibili e catenene ascendenti di ideali principali stazionarie. 
L'anello dei polinomi a coefficienti in un campo \`e un dominio Euclideo. ${\bf Z}[i]$ \`e un dominio euclideo. I primi della forma $p=4k+1$ sono la somma di due quadrati.

%Il Teorema di Ruffini e i suoi corollari. Un polinomio di grado $n$ su un dominio ha al pi\`u $n$ radici distinte. Principio di identit\`a per polinomi. 
%Fattorizzazione di $X^p-X$ in ${\bf F}_p[X]$. 
Il contenuto di un polinomio. La $p$-proiezione di $A[X]$. 
Il lemma di Gauss sul prodotto di polinomi primitivi. Ideali del prodotto diretto di anelli, il teorema cinese dei resti in anelli commutativi unitari.  Esempi di domini non a fattorizzazione unica; esempi di domini senza il MCD. 
Fattorizzazione negli anelli di polinomi. $A[X]$ \`e fattoriale se $A$ \`e fattoriale. Criteri di irriducipilit\`a di Eisenstein. Irriducibilit\`a degli anelli di polinomi.

%\titoloparagr{Teoria della Cardinalit\`a}
%Enunciati dei Teoremi di Cantor-Bernstein e di Hertogs. Cardinalit\`a del numerabile, cardinalit\`a del continuo, Metodo diagonale di Cantor, propriet\`a fondamentali.

%\titoloparagr{Teoria dei Campi}
%Caratteristica di un campo, Teorema dei gradi, Costruzione dei campi mediante i quozienti negli anelli di polinomi, radici di polinomi, esistenza del campo di spezzamento (solo enunciato).

\testi 
\bib
\autore{D. Dikranjan - M.S. Lucido} \titolo{Aritmetica e algebra}
\editore{Liguori} \annopub{2007}
\endbib

\altritesti
\bib
\autore{G.M. Piacentini Cattaneo} \titolo{Algebra, un approccio
algoritmico} \editore{Decibel -- Zanichelli} \annopub{1996}
\endbib
\bib
\autore{I. N. Herstein} \titolo{Algebra} \editore{Editori Riuniti}
\annopub{2003}
\endbib
\bib
\autore{M. Artin} \titolo{Algebra} \editore{Bollati Boringhieri}
\annopub{1997}
\endbib

\esami
\bye

