\nopagenumbers \font\title=cmti12
\def\ve{\vfill\eject}
\def\vv{\vfill}
\def\vs{\vskip-2cm}
\def\vss{\vskip10cm}
\def\vst{\vskip13.3cm}

%\def\ve{\bigskip\bigskip}
%\def\vv{\bigskip\bigskip}
%\def\vs{}
%\def\vss{}
%\def\vst{\bigskip\bigskip}

\hsize=19.5cm
\vsize=27.58cm
\hoffset=-1.6cm
\voffset=0.5cm
\parskip=-.1cm
\ \vs \hskip -6mm AL210 AA10/11\ (Algebra: gruppi, anelli e campi)\hfill APPELLO C \hfill Roma, 13 Giugno 2011. \hrule
\bigskip\noindent
{\title COGNOME}\  \dotfill\ {\title NOME}\ \dotfill {\title
MATRICOLA}\ \dotfill\
\smallskip  \noindent
Risolvere il massimo numero di esercizi accompagnando le risposte
con spiegazioni chiare ed essenziali. \it Inserire le risposte
negli spazi predisposti. NON SI ACCETTANO RISPOSTE SCRITTE SU
ALTRI FOGLI.
\rm 1 Esercizio = 4 punti. Tempo previsto: 2 ore. Nessuna domanda
durante la prima ora e durante gli ultimi 20 minuti.
\smallskip
\hrule\smallskip
\centerline{\hskip 6pt\vbox{\tabskip=0pt \offinterlineskip
\def \trl{\noalign{\hrule}}
\halign to277pt{\strut#& \vrule#\tabskip=0.7em plus 1em& \hfil#&
\vrule#& \hfill#\hfil& \vrule#& \hfil#& \vrule#& \hfill#\hfil&
\vrule#& \hfil#& \vrule#& \hfill#\hfil& \vrule#& \hfil#& \vrule#&
\hfill#\hfil& \vrule#& \hfil#& \vrule#& \hfill#\hfil& \vrule#&
\hfil#& \vrule#& \hfill#\hfil& \vrule#& \hfil#& \vrule#& \hfil#&
\vrule#\tabskip=0pt\cr\trl && FIRMA && 1 && 2 && 3 && 4 &&
5 && 6 && 7 && 8  &&  TOT. &\cr\trl && &&   &&
&&     &&   &&     &&   &&   &&    && &\cr &&
\dotfill &&       &&   &&   &&     &&   && && && &&
&\cr\trl }}}
\medskip

\item{1.} Rispondere alle sequenti domande fornendo una giustificazione di una riga:\bigskip\bigskip\bigskip


\itemitem{a.} \`E vero che esistono almeno due gruppi non isomorfi con $49$ elementi?\medskip\bigskip\bigskip

\ \dotfill\ \bigskip\bigskip\bigskip\vfil

\itemitem{b.} \`E vero che se $n\ge k+1$, $S_n$ contiene sempre un sottogruppo isomorfo a $S_{n-k}$?\medskip\bigskip\bigskip

\ \dotfill\ \bigskip\bigskip\bigskip\vfil

\itemitem{c.} \`E vero esistono domini di integrit\`a finiti che non sono campi?\medskip\bigskip\bigskip
 
\ \dotfill\ \bigskip\bigskip\bigskip\vfil

\itemitem{d.} \`E vero che se $A$ \`e un anello, allora $U(A[X])=U(A)$?\medskip\bigskip\bigskip

\ \dotfill\ \bigskip\bigskip\bigskip


\vfil\eject

\item{2.} Dimostrare che se $G$ \`e un gruppo ciclico e $n\mid |G|$, allora $G$ ammette un unico 
sottogruppo con $n$ elementi.\vv

\item{3.} Sull'insieme ${\bf Z}\times{\bf Z}$ consideriamo la legge di composione $(*)$ definita nel modo seguente:(1)
$$(a; b) * (c; d) = (a + c; (-1)^c b + d)$$
Dimostrare che $({\bf Z}\times{\bf Z};*)$ \`e un gruppo (non commutativo) determinandone il centro.\ve\vs

\item{4.} Dimostrare che i gruppi $({\bf Q}^*,\times)$ e $({\bf Z}[X],+)$ sono isomorfi.\vv

\item{5.} Sia dato l’anello ${\bf Z}_{15}, +, \times,[1]_{15}$. Sia $B$ l’insieme dei multipli di
$3$, ossia
$$B =\{ [0]_{15}, [3]_{15}, [6]_{15}, [9]_{15}, [12]_{15}\}.$$ 
Si provi che $B$ è un campo rispetto alle
operazioni $+$ e $\times$ di ${\bf Z}_{15}$, ma non un suo sottoanello. 
Chi è il suo elemento unit\`a?.\ve\vs

\item{6.} Dopo aver fornito la definizione di caratteristica di un anello commutativo con unit\`a, si determini la
caratteristica dei seguenti anelli: ${\bf Z}_{12}\times{\bf Z}_{28}$, ${\bf Z}\times{\bf Z}_{8}$, ${\bf F}_{81}$ (${\bf F}_{81}$
indica un campo con $81$ elementi).\vv

\item{7.} Dopo aver fornito la definizione di anello a ideali principali (PID), si dimostri che
${\bf Z}[i]$ \`e un PID.\vv

\item{8.} Sia dato l’insieme delle matrici ad elementi reali, quadrate d’ordine $n > 1$
e triangolari superiori:
$$T = \{A \in M_n({\bf R}): A = [a_{ij}], a_{ij} = 0\  \forall i > j\}.$$
\itemitem{a)} Si dimostri che costituisce un anello rispetto alle consuete operazioni di
addizione e di moltiplicazione righe per colonne fra matrici.
\itemitem{b)} Si tratta di un dominio d'integrità\`a?.
\ve \vs
 \bye

\item{9.} Considerare $f(x)=X^3+2X^2+X+2\in{\bf Z}/3{\bf Z}[X]$. Dimostrare che ${\bf Z}/3{\bf Z}[X]/f(X)$
non \`e un campo esibendo un elemento che non \`e invertibile. Quanti elementi ha ${\bf Z}/3{\bf Z}[X]/f(X)$?
\ \vst
