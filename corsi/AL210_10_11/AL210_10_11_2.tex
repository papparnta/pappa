\nopagenumbers \font\title=cmti12
\def\ve{\vfill\eject}
\def\vv{\vfill}
\def\vs{\vskip-2cm}
\def\vss{\vskip10cm}
\def\vst{\vskip13.3cm}

%\def\ve{\bigskip\bigskip}
%\def\vv{\bigskip\bigskip}
%\def\vs{}
%\def\vss{}
%\def\vst{\bigskip\bigskip}

\hsize=19.5cm
\vsize=27.58cm
\hoffset=-1.6cm
\voffset=0.5cm
\parskip=-.1cm
\ \vs \hskip -6mm AL2 AA10/11\ (Algebra: gruppi, anelli e campi)\hfill ESAME DI FINE SEMESTRE \hfill Roma, 7 Gennaio 2011. \hrule
\bigskip\noindent
{\title COGNOME}\  \dotfill\ {\title NOME}\ \dotfill {\title
MATRICOLA}\ \dotfill\
\smallskip  \noindent
Risolvere il massimo numero di esercizi accompagnando le risposte
con spiegazioni chiare ed essenziali. \it Inserire le risposte
negli spazi predisposti. NON SI ACCETTANO RISPOSTE SCRITTE SU
ALTRI FOGLI.
\rm 1 Esercizio = 4 punti. Tempo previsto: 2 ore. Nessuna domanda
durante la prima ora e durante gli ultimi 20 minuti.
\smallskip
\hrule\smallskip
\centerline{\hskip 6pt\vbox{\tabskip=0pt \offinterlineskip
\def \trl{\noalign{\hrule}}
\halign to277pt{\strut#& \vrule#\tabskip=0.7em plus 1em& \hfil#&
\vrule#& \hfill#\hfil& \vrule#& \hfil#& \vrule#& \hfill#\hfil&
\vrule#& \hfil#& \vrule#& \hfill#\hfil& \vrule#& \hfil#& \vrule#&
\hfill#\hfil& \vrule#& \hfil#& \vrule#& \hfill#\hfil& \vrule#&
\hfil#& \vrule#& \hfill#\hfil& \vrule#& \hfil#& \vrule#& \hfil#&
\vrule#\tabskip=0pt\cr\trl && FIRMA && 1 && 2 && 3 && 4 &&
5 && 6 && 7 && 8 &&   TOT. &\cr\trl && &&   &&
&&     &&   &&   &&   &&   &&    && &\cr &&
\dotfill &&     &&   &&   &&     &&   && && && &&
&\cr\trl }}}
\medskip

\item{(1)} Dopo aver richiamato la definizione di ideale in un anello commutativo unitario, dimostrare che l'intersezione di una qualsiasi famiglia di ideali \`e
un ideale mentre non \`e detto che l'unione di ideali sia un ideale.\vv

\item{(2)} Sia ${\bf Z}_{(p)}=\{{m\over n}\in{\bf Q}\ {\rm t.c.\ } p\ {\rm non\ divide\ }n\}$. 
Verificare che ${\bf Z}_{(p)}$ \`e un sottoanello  di ${\bf Q}$ e determinarne
tutti gli elementi invertibili. Inoltre dimostrare che l'insieme di tutti gli elementi
non invertibili forma un ideale.\ve\vs

\item{(3)} Dimostrare che in ${\bf Z}[i]$ gli elementi $1+2i$, $3$ e $1+i$
sono irriducibili. Dedurne la fattorizzazione (unica) di $30\in{\bf Z}[i]$. \vv

\item{(4)} Considerare l'applicazione 
$\Psi: M_2({\bf Z})\rightarrow M_2({\bf Z}_8), \pmatrix{a & b \cr c & d}\mapsto \pmatrix{a\bmod 8 & b\bmod 8 \cr c\bmod 8 & d\bmod 8}$.
Dopo aver verificato che si tratta di un omomorfismo, se ne calcoli il nucleo e l'immagine.\ve\vs

\item{5.} Sia $A=\{{n\over 9^\alpha}|\ n\in{\bf Z}, \alpha\in{\bf N}\}$. Dopo aver dimostrato che $A$ \`e un
anello, verificare se il suo campo dei quozienti \`e ${\bf Q}$.\vv

\item{(6)} Determinare tutti i divisori dello zero nell'anello ${\bf Z}\times{\bf Z}_6$.
\ve\vs

\item{(7)} Dopo aver ricordato la definizione di anello euclideo, dimostrare che se $K$ \`e un campo, allora $K[X]$ \`e euclideo.
\vv

\item{(8)} Dimostrare che se $k\in{\bf Z}$, il polinomio $X^4+(2k+1)X+1\in{\bf Q}[X]$ \`e irriducibile.\hfill{\it Suggerimento: Ridurre modulo $2$.}
%\vv\vv

%\item{9.} Si consideri il campo con $7$ elementi ${\bf F}_7$ e il  polinomio $f%(X)=X^2+1\in{\bf F}_7[X]$. Dimostrare che l'anello
%quoziente ${\bf F}_7[x]/(f(X))$ \`e un campo e se ne calcoli il numero di elementi.
%\ \vst
\ve\vs

 \bye
