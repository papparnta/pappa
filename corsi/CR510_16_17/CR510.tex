\input programma.sty
\def\abbrcorso{CR510}
\def\titolocorso{Crittosistemi ellittici }
\def\sottotitolo{http://www.mat.uniroma3.it/users/pappa/CORSI/CR510$_-$16$_-$17/CR510.htm}
\def\docente{Prof. Francesco Pappalardi}
\def\crediti{6}
\def\semestre{II}
\def\esoneri{0}
\def\scrittofinale{0}
\def\oralefinale{0}
\def\altreprove{1}
\Intestazione
\titoloparagr{Teoria delle Curve Ellittiche}

L'equazione di Weierstrass, La struttura di gruppo sui punti
razionali, formule per la somma e la duplicazione. Generalit\`a
sulle intersezioni fra rette e curve in $\P({\bf K})^2$. Risultati
preparatori alla dimostrazione dell'associativit\`a dei punti sulle
curve ellittiche. Dimostrazione dell'associa\-tivit\`a della somma
per i punti di una curva ellittica. Altre equazioni per curve
ellittiche, Equazione di Legendre, Equazioni cubiche, Equazioni
quartiche, intersezioni di due superfici cubiche. L'invariante $j$,
curve ellittiche in caratteristica 2, Endomorfismi, curve singolari,
curve ellittiche modulo $n$.

\titoloparagr{Punti di Torsione}

Punti di torsione, Polinomi di divisione. L'accoppiamento di Weil.

\titoloparagr{Curve ellittiche su campi finiti}

L'endomorfismo di Frobenius. Il problema di determinare l'ordine del
gruppo. Curve su sottocampi, Simboli di Legendre, Ordini dei punti,
L'algoritmo "Baby Step, Giant Step" di Shanks. Famiglie particolari
di curve ellittiche. L'algoritmo di Schoof.

\titoloparagr{Crittosistemi sulle Curve Ellittiche}

Il problema del Logaritmo Discreto. Algoritmi per il calcolo del
logaritmo discreto: Baby-Step  Giant-Step e Polig-Hellman. Attacco
MOV. Attacco sulle curve anomale. Scambio di Chiavi di
Diffie-Hellman. Crittosistemi di Massey Omura e El Gamal. Schema
di Firma di El Gamal. Crittosistemi  sulle curve ellittiche basati
sul problema della fattorizzazione. Un crittosistema basato
sull'accoppiamento di Weil. Fattorizzzione di numeri interi
utilizzando le curve ellittiche. Utilizzo di Pari.

\testi
\bib
\autore{Lawrence C. Washington} \titolo{Elliptic Curves: Number
Theory and Crptography} \editore{Chapman \& Hall (CRC)}
\annopub{2003}
\endbib

\bib
\autore{Alfred J. Menezes} \titolo{Elliptic Curve Public Key
Cryptosystems, The Kluwer International Series in Engineering and
Computer Science, Vol. 234} \editore{Kluwer} \annopub{1993}
\endbib

\altritesti
\bib
\autore{Darrel Hankerson, Alfred J. Menezes e Scott Vanstone}
\titolo{Guide to Elliptic Curve Cryptography}
\editore{Springer Professional Computing}
\annopub{2004}
\endbib
\bib
\autore{Michael Rosing}
\titolo{Implementing Elliptic Curve Cryptography}
\editore{Manning Greenwich}
\annopub{1998}
\endbib
\bib
\autore{Ian Blake, Gadiel Seroussi e Nigel Smart}
\titolo{Elliptic Curves in Cryptography}
\editore{Cambridge University Press}
\annopub{1999}
\endbib
\bib
\autore{Andreas Enge}
\titolo{Elliptic Curves and Their Applications to Cryptography. An Introduction}
\editore{Springer Verlag}
\annopub{1999}
\endbib
\esami

Si richiede che gli studenti espongano dei seminari
concordati con i docenti e che svolgano una serie di esercizi a
casa. 
\bye
