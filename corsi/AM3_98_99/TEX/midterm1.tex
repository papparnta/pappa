\documentclass[12pt,a4paper]{report}\pagenumbering{roman}
\pagestyle{empty}
\begin{document}
\begin{center}
{\bf COMPITO DI MET\`A SEMESTRE}\\
{\bf Analisi due (Primo modulo) - Corso di Laurea in FISICA}\\
{\bf Sabato 21 Novembre, 1998}
\end{center}
{\Large\bf LEGGERE ATTENTAMENTE:}
\begin{itemize}
\item Il presente esame consiste di 10 esercizi. Ogni esercizio 
vale 10 punti su 100.
\item Il compito non sar\`a sufficiente se non si risolve almeno 
un esercizio del gruppo 1. 2. 3., almeno 
uno del gruppo 4. 5. 6. e almeno uno del gruppo  7. 8. 9. 10.
\item Non sono ammessi appunti, calcolatrici, 
libri, tavole di integrali e telefoni cellulari.
\item Il tempo concesso per svolgere il compito \`e di 3 ore.
\item Per la brutta copia \`e consentito utilizzare 
esclusivamente fogli consegnati dal docente.
\item Tutti gli effetti personali, compresi borse e cappotti, devono 
essere lasciati accanto agli attaccapanni (ad eccezione della penna!).
\item Non \`e consentito consegnare altri fogli oltre agli 11 (undici)
del presente fascicolo.
\item Scrivere a penna e tenere il libretto (o un altro documento) sul banco per
il riconoscimento.
\item {\bf Non \`e consentito parlare o comunicare in nessun modo, pena
il ritiro immediato del compito.} 
\end{itemize}
\begin{center}
\begin{tabular}{||r|r||}
\hline ESERCIZIO & PUNTEGGIO\\  \hline
\hline 1.& \\
\hline 2.& \\
\hline 3.& \\
\hline 4.& \\
\hline 5.& \\
\hline 6.& \\
\hline 7.& \\
\hline 8.& \\
\hline 9.& \\
\hline 10.& \\
\hline TOTALE & /100\\
\hline VOTO& /30\\
\hline
\end{tabular}
\end{center}
\pagebreak

\begin{enumerate}
\item Si trovi la soluzione generale della seguente equazione:
$$y'''-5y''+7y'-3y=0$$
\hspace*{-3.5cm}\begin{tabular}{c}\hline\\\hspace*{16cm}\end{tabular}\\
\hspace*{-3.5cm}{\bf SVOLGIMENTO:}\pagebreak

\item Si risolva il seguente problema di Cauchy:
$$\left\{\begin{array}{l} y'={y\over x}+{x\over y} \\ y(1)=1
\end{array}\right.$$
\hspace*{-3.5cm}\begin{tabular}{c}\hline\\\hspace*{16cm}\end{tabular}\\
\hspace*{-3.5cm}{\bf SVOLGIMENTO:}\pagebreak

\item Si determini la soluzione generale del seguente sistema di 
equazioni differenziali e si classifichi il flusso associato allo spazio 
delle soluzioni:
$$\left\{\begin{array}{l}y_1'=y_1+3y_2\\ y_2'=-y_1-y_2\end{array}\right.$$
\hspace*{-3.5cm}\begin{tabular}{c}\hline\\\hspace*{16cm}\end{tabular}\\
\hspace*{-3.5cm}{\bf SVOLGIMENTO:}\pagebreak

\item Sia ${\bf Dom}(f)$ il dominio della funzione $f(x,y)=\sqrt{e^{-xy}
(y+2+x^2)}$. Dopo aver tracciato la figura di ${\bf Dom}(f)$, se ne determini
l'interno, la chiusura, la frontiera e il derivato.\\
\hspace*{-3.5cm}\begin{tabular}{c}\hline\\\hspace*{16cm}\end{tabular}\\
\hspace*{-3.5cm}{\bf SVOLGIMENTO:}\pagebreak

\item Dopo averne tracciato la figura, si dimostri che il seguente sottoinsieme
di ${\bf R}^2$ non \`e compatto costruendo un ricoprimento di aperti che non
ammette un sottoricoprimento finito.
$$S=\left\{(x,y)\in{\bf R}^2\ \mbox{t.c.}\ y\in[0,2]\ \mbox{e}\ y>x^2\right\}$$
\hspace*{-3.5cm}\begin{tabular}{c}\hline\\\hspace*{16cm}\end{tabular}\\
\hspace*{-3.5cm}{\bf SVOLGIMENTO:}\pagebreak 

\item Si discuta la continuit\`a della seguente funzione $f:{\bf R}^2\longrightarrow
{\bf R}$:
$$f(x,y)=\left\{\begin{array}{ll} {xy\arctan{x}\over y^2+(\arctan{x})^2} & 
\mbox{se}\ \ (x,y)\neq (0,0)\\ \\0&\mbox{se}\ \ (x,y)=(0,0)\end{array}\right.$$
\hspace*{-3.5cm}\begin{tabular}{c}\hline\\\hspace*{16cm}\end{tabular}\\
\hspace*{-3.5cm}{\bf SVOLGIMENTO:}\pagebreak

\item Si calcoli il differenziale e il piano tangente nel punto $(1,1,e\ln2)$
della superficie di equazione $z=e^x\ln(y+1)$.\\
\hspace*{-3.5cm}\begin{tabular}{c}\hline\\\hspace*{16cm}\end{tabular}\\
\hspace*{-3.5cm}{\bf SVOLGIMENTO:}\pagebreak

\item Si calcoli il polinomio di Taylor di grado tre intorno al punto $(0,0)$
della funzione $f(x,y)=\ln(x+y+1).$ \\
\hspace*{-3.5cm}\begin{tabular}{c}\hline\\\hspace*{16cm}\end{tabular}\\
\hspace*{-3.5cm}{\bf SVOLGIMENTO:}\pagebreak

\item Sia $f(x,y)=y^3-3y+x^3-12x$. Determinare i punti critici di 
$f$ e classificarli con il metodo della matrice Hessiania.\\
\hspace*{-3.5cm}\begin{tabular}{c}\hline\\\hspace*{16cm}\end{tabular}\\
\hspace*{-3.5cm}{\bf SVOLGIMENTO:}\pagebreak

\item Si risolva la seguente equazione differenziale:
$$(\cos(x+y)+\cos x)dx+(\cos y+\cos(x+y))dy=0$$
\hspace*{-3.5cm}\begin{tabular}{c}\hline\\\hspace*{16cm}\end{tabular}\\
\hspace*{-3.5cm}{\bf SVOLGIMENTO:}

\end{enumerate}
\end{document}
