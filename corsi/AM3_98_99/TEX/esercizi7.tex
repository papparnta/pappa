\documentstyle{article}
\begin{document}
\centerline{\Large ESERCIZI SULLE CURVE DIFFERENZIABILI.}
\bigskip\bigskip

\begin{enumerate}
\item Si determini:
\begin{enumerate}
\item il versore tangente
\item La retta tangente e la retta normale

di ciascuna delle curve associate alle seguenti rappresentazioni parametriche
([TF] capitolo 12.3).

\begin{center}
\begin{tabular}{|r|l|r|l|}
\hline
i. & $(x(t),y(t))=(3\cos t,3\sin t)$;&
ii. & $(x(t),y(t))=(e^t,t^2)$;\\
\hline
iii. & $(x(t),y(t))=(\cos^3t,\sin^3t)$;&
iv. & $(x(t),y(t))=(t,t^2)$;\\
\hline
v. & $(x(t),y(t))=(\cos 2t,2\cos t)$;&
vi. &  $(x(t),y(t))=(t^3/3,t^2/2)$;\\
\hline
vii. & $(x(t),y(t))=(e^t\cos t,e^t\sin t)$;&
viii.& $(x(t),y(t))=(\cosh t,t)$;\\
\hline
\end{tabular}
\end{center}
Per le seguenti curve si verifichi anche 
\item se le curve sono piane;
\item si calcoli la lunghezze della curva tra $t=0$ e $t=\pi$;
\item Il piano normale, quello osculatore e quello binormale;
\item la curvature e l'equazione del cerchio osculatore.
\end{enumerate}
\begin{center}
\begin{tabular}{|r|l|}
\hline
ix. & $(x(t),y(t),z(t))=(6\sin2t,6\cos2t,5t)$;\\
\hline
x. & $(x(t),y(t),z(t))=(e^t\cos t,e^t\sin t,e^t)$;\\
\hline
xi. & $(x(t),y(t),z(t))=(3\cosh 2t,3\sinh 2t,6t)$; \\
\hline
xii. & $(x(t),y(t),z(t))=(3t\cos t,3t\sin t,4t)$.\\
\hline
\end{tabular}
\end{center}

\item Si determini la curvatura delle curve associate a ciascuna delle
seguenti rappresentazioni parametriche:
\begin{enumerate}
\item $y=a\cosh(x/a)$;
\item $y=\ln\cos x$;
\item $y=e^{2x}$
\item $\left\{\begin{array}{l}
x(t)=a\cos^3t \\
y(t)=a\sin^3t
\end{array}\right.$
\item $\left\{\begin{array}{l}
x(\theta)=a(\cos\theta+\theta\sin\theta) \\
y(\theta)=a(\sin\theta-\theta\cos\theta)
\end{array}\right.$
\end{enumerate}

\end{enumerate}
\end{document}


