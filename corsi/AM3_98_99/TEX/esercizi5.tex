\documentstyle{article}
\pagestyle{empty}
\begin{document}
\centerline{\Large ESERCIZI SULLE DERIVATE PARZIALI (II).}
\bigskip\bigskip

\begin{enumerate}
\item Calcolare il polinomio di Taylor di grado 3 intorno a $(0,0)$ delle
seguenti funzioni:
\begin{enumerate}
\item $f(x,y)=\sin(xy);$
\item $f(x,y)=\sqrt{x+y+2};$
\item $f(x,y)=e^{x^2+y^3};$
\item $f(x,y)=\arctan(x^2+y^3);$
\end{enumerate}
\item Calcolare il polinomio di Taylor intorno a $(0,0,0)$ di grado 10
della funzione $f(x,y,z)=\ln(xyz+1)$.
\item Determinare i punti critici delle seguenti funzioni e classificarli
usando il metodo della matrice Hessiana:
\begin{enumerate}
\item $f(x,y)=-\ln{((x-3)^4-(y-2)^2+1)};$
\item $f(x,y)=||(x,y)||;$
\item $f(x,y)=\sin(xy);$
\item $f(x,y)=e^{x^2+y^3};$
\item $f(x,y)=\arctan((x-2)^2+(y+2)^3);$
\end{enumerate}
\end{enumerate}
\end{document}