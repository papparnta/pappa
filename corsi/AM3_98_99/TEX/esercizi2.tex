\documentclass[11pt]{article}\pagestyle{empty}
\begin{document}
\centerline{\Large Esercizi di Topologia in ${\bf R}^n$ (I)} 
\section{Insiemi Aperti e Chiusi}
Negli esercizi che seguono ``Si dimostri'' \`e da intendersi nel seguente modo: Fornire un
argomento molto \underline{succinto}, utilizzando anche figure, per convincere il lettore della 
verit\`a dell'affermazione.
\begin{enumerate}
\item Si dimostri che il rettangolo $(a,b)\times(c,d)\subset{\bf R}^2$ \`e
aperto in ${\bf R}^2$;
\item Si dimostri che l'unione di una qualsiasi famiglia insiemi aperti in ${\bf R}^n$
\`e un insieme aperto;
\item Si dimostri che l'intersezione di due insiemi aperti di ${\bf R}^n$ \`e un
insieme aperto (Quindi lo stesso vale per un numero finito di insiemi aperti);
\item Si dimostri che il complementare di un iperpiano $a_1x_1+b_2x_2+\cdots+a_nx_n=b$
in ${\bf R}^n$ \`e un insieme aperto;
\item Si disegni il grafico dell'insieme $\{x=(x_1,x_2)\in{\bf R}^2\ \mbox{t.c.}\ |x_1| +|x_2|<1\}$ e
si dimostri che \`e un insieme aperto;
\item Si dimostri che il disco chiuso ${\bar D}_r(x)=\{ z\in{\bf R}^n\ \mbox{t.c.} \ d(x,z)\leq r\}$
\`e un insieme chiuso;
\item Si dimostri che il grafico di una funzione continua $y=f(x)$ \`e un insieme chiuso di ${\bf R}^2$;
\item Si dimostri che l'intersezione di una qualsiasi famiglia di insiemi chiusi di ${\bf R}^n$
\`e un insieme chiuso di ${\bf R}^n$;
\item Si dimostri che l'unione di due chiusi di ${\bf R}^n$ \`e un chiuso di ${\bf R}^n$;
\item Si dimostri che il rettangolo chiuso $[a,b]\times[c,d]\in{\bf R}^2$ \`e un chiuso;
\item Si dimostri che il rettangolo semichiuso $(a,b]\times[c,d)\in{\bf R}^2$ non \`e
 aperto n\'e chiuso;
\item Si dimostri che il cilindro $\{ (x_1,x_2,x_3)\in{\bf R}^3\ \mbox{t.c.}\ x_1^2+x_2^2<1\}$
\`e un insieme aperto di ${\bf R}^3$;
\item si dimostri che il disco ``sottile'' 
$$\{ (x_1,x_2,x_3)\in{\bf R}^3\ \mbox{t.c.}\ x_1^2+x_2^2<1, x_3=0\}$$
non \`e aperto n\'e chiuso in  ${\bf R}^3$;
\end{enumerate}
\pagebreak

\section{Chiusura, Interno, Frontiera e Punti di Accumulazione} 
Per ciascuno dei seguenti insiemi, si tracci la figura dell' insieme e 
si determini la chiusura, l'interno, la frontiera e 
l'insieme dei punti di accumulazione nello spazio specificato:
\begin{enumerate}
\item in ${\bf R}^2$, $\{(x,y)\ \mbox{t.c.}\ x^2+y^2<4, x\geq 1\};$
\item in ${\bf R}^2$, $\cup_{n=0}^\infty \{(x,y)\ \mbox{t.c.}\ (x-n)^2+y^2=1/n\};$
\item in ${\bf R}^2$, 
$$D_1((0,0))\cup\{(x,y)\ \mbox{t.c.}\ x\geq0, x^2+y^2=1\}\cup\left\{(0,1),(-1,0),(2,0),(-2,0)\right\};$$
\item in ${\bf R}^3$,
$$\left\{(x_1,x_2,x_3)\ \mbox{t.c.}\ x_1^2+x_2^2<1, x_3\in[-2,2]\right\}\cup\left\{(0,0,2),(1,0,2),(0,-1,2),(2,1,0)\right\}$$
\item in ${\bf R}^3$,
$$\left\{(x_1,x_2,x_3)\ \mbox{t.c.}\ 1<x_1^2+x_2^2+x_3^2\leq 2\right\}
\cup\left\{(x_1,x_2,x_3)\ \mbox{t.c.}\ x_1=0,x_2=0\right\}$$
\item in ${\bf R}^3$,
$$\left\{(x_1,x_2,x_3)\ \mbox{t.c.}\ x_1+x_2+x_3<0\right\}\cup\{(1,1,-10),(3,3,-6),(1,1,1)\}\cup {\bf Z}^3.$$
\end{enumerate}
\section{Sottoinsiemi Densi e Discreti}
{a.} Si dimostri che i seguenti insiemi sono densi;
$$\mbox{1.}\ {\bf R}\times{\bf Q}\ \mbox{in}\ {\bf R}^2;\ \ \ \mbox{2.}\ {\bf R}^n
\setminus\{P_1,\ldots,P_s\}\ \mbox{in}\ {\bf R}^n;\ \ \ 
\mbox{3.}\ {\bf R}^2\setminus\{(x,y)\ \mbox{t.c.}\ x+y=0\ \}\ \mbox{in}\ {\bf R}^2;$$
{b.} Si dimostri che i seguenti insiemi sono discreti;
\begin{enumerate}
\item Ogni sottoinsieme finito di ${\bf R}^n$;
\item ${\bf Z}^{a}\times{\bf N}^{n-a}$ in ${\bf R}^n$;
\item ${\bf Z}\times{\bf N}\times S$ (dove $S$ \`e finito) in ${\bf R}^3$;
\end{enumerate}
{c.} Si dimostri che il complementare di un insieme discreto \`e denso.

\noindent {d.} Si dimostri che non \`e vero che il complementare di un insieme denso \`e 
discreto;

\noindent {e.} Si dimostri che ${\bf N}\times{\bf Q}$ non \`e denso n\'e discreto in ${\bf R}^2$.
\end{document}
