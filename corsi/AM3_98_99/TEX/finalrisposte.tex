\documentclass[12pt,a4paper]{report}\pagenumbering{roman}
%\input{psfig.sty}
\begin{document}
\begin{center}
{\bf COMPITO FINALE}\\
{\bf Analisi due (Primo modulo) - Corso di Laurea in FISICA}\\
{\bf Sabato 23 Dicembre, 1998}
\end{center}
\begin{enumerate}
\item Si calcoli il polinomio di Taylor intorno a $(0,0)$ di grado 20
della seguente funzione:
$$f(x,y)=\ln(1+x^4y^3)+\arctan(x^6y^4).$$
\bigskip 

{\bf SOLUZIONE:} {\sl Sia $g(t)=\ln(1+t)$ e $h(t)=\arctan t$. Calcolando
i polinomi di Taylor di grado due intorno a $t=0$ in una variabile, 
otteniamo:
$$g(t)=t-{t^2\over2}+O(t^3)\ \ \mbox{e}\ \ h(t)=t+O(t^3).$$
Adesso notiamo che
$$f(x,y)=g(x^4y^3)+h(x^6y^4)=x^4y^3-{x^8y^6\over2}+x^6y^4+O(x^{12}y^9)+
O(x^{18}y^{12}),$$
Osservando che poich\'e $x\leq||(x,y)||$ e $y\leq||(x,y)||$, si ha che
$$O(x^{12}y^9)+
O(x^{18}y^{12})=O(||(x,y)||^{21})+O(||(x,y)||^{30})=O(||(x,y)||^{21})$$
e quindi $x^4y^3-{x^8y^6\over2}+x^6y^4$ \`e il polinomio di Taylor
di grado 20.}\bigskip

\item Si scriva il polinomio di Taylor di grado due intorno al punto $0$
della funzione $y=f(x)$ definita implicitamente da
$$\left\{\begin{array}{l}
x^3y+y^3-\cos x=0\\
f(0)=1
\end{array}\right.$$

{\bf SOLUZIONE:} {\sl Sia $F(x,y)=x^3y+y^3-\cos x$. Osservando che
poich\'e $F(0,1)=0$ e $F_y(0,1)=\left(x^3+3y^2\right)_{(0,1)}=3\neq0$,
la funzione implicita $y=f(x)$ \`e ben definita e dal Teorema della
Funzione Implicita (TFI), otteniamo 
$$f'(0)=-{F_x(0,1)\over F_y(0,1)}=
\left(-{3x^2y+\sin x\over x^3+3y^2}\right)_{(0,1)}=0.$$
Per calcolare $f''(0)$, osserviamo che
$$f''(x)={d\over dx}\left(-{F_x(x,f(x))\over F_y(x,f(x))}\right)=$$
$$\left(-{(F_{xx}+F_{yx}f'(x))F_y-
F_x(F_{xy}+F_{yy}f'(x))\over (F_y)^2}\right).$$
e quindi, poich\'e 
$$f'(0)=F_{x}(0,1)=0, F_y(0,1)=3\ \ \mbox{e}\ \ F_{xx}(0,1)=\left(6xy+\cos x\right)_{(0,1)}=1$$
otteniamo $f''(0)=-{1\over3}$. Infine il polinomio di Taylor $P_2(x)$ 
di grado $2$ intorno a $0$ \`e:
$$P_2(x)=f(0)+f'(0)x+{1\over2}f''(0)x^2=1-{1\over6}x^2.$$}

\item  Sia
$$\underline{f}(x,y,z)=\left(\begin{array}{l}
xz\\
y^2+x+1\\
xyz+1\end{array}
\right).$$ 
Dopo aver verificato che $\underline{f}$ \`e invertibile in $(1,0,1)$,
si scriva la matrice Jacobiana nel punto $(1,2,1)$ della
funzione inversa.\\
({\it Suggerimento:} $(1,2,1)=f(1,0,1)$.)
\bigskip 

{\bf SOLUZIONE:} {\sl La matrice Jacobiana di $\underline{f}$ in $(1,0,1)$
\`e:
$$J(\underline{f})_{(1,0,1)}=
\left(\begin{array}{ccc} z& 0 & x \\ 1&2y  & 0 \\ yz& xz & xy
\end{array}\right)_{(1,0,1)}=
\left(\begin{array}{ccc}1 & 0 & 1 \\1 & 0 &0  \\ 0& 1 & 0\end{array}
\right).
$$
Dal teorema della funzione inversa otteniamo che 
$$J(\underline{f}^{-1})_{f(1,0,1)}=
\left(J(\underline{f})_{(1,0,1)}\right)^{-1}
=\left(\begin{array}{ccc}1 & 0 & 1 \\1 & 0 &0  \\ 0& 1 & 0\end{array}
\right)^{-1}.$$
Facendo i calcoli (dell'inversa di una matrice $3\times3$) troviamo
$$J(\underline{f}^{-1})_{(1,2,1)}=
\left(\begin{array}{ccc} 0& 1 & 0 \\0 & 0 & 1 \\ 1&-1  &0 \end{array}
\right).
$$}

\item Si calcoli la lunghezza della curva associata alla seguente rappresentazione 
parametrica:
$$\underline{x}(t)=\left(2t,\ln t,t^2\right), t\in[1,10].$$
%% RISPOSTA: $99+\ln 10$
\bigskip 

{\bf SOLUZIONE:} {\sl Utilizzando la formula
$$L=\int_{t_1}^{t_2}||\underline{x}'(t)||dt$$
si trova che $\underline{x}'(t)=(2,{1\over t},2t)$ e quindi
$$L=\int_{1}^{10}\sqrt{4+{1\over t^2}+4t^2}dt=
\int_{1}^{10}\sqrt{(2t+{1\over t})^2}dt=
\int_{1}^{10}(2t+{1\over t})dt=$$
$$\left[t^2+\ln t\right]^{10}_1=99+\ln 10.$$}


\item Si calcoli l'equazione del piano tangente e quella della retta normale alla
superficie:
$$x^4+3y^3-4z^6=0$$
nel punto $P=(1,1,1)$. Si dica inoltre rispetto a quale delle tre variabili
si pu\`o applicare il teorema della funzione implicita nel punto $P$ e si calcoli
il gradiente delle funzioni cos\`\i\ definite.
\bigskip 

{\bf SOLUZIONE:} {\sl Sia $F(x,y,z)=x^4+3y^3-4z^6$. Osserviamo che il
gradiente 
$$\nabla(F)(1,1,1)=(4x^3,9y^2,-24z^5)_{(1,1,1)}=(4,9,-24)$$
\`e normale alla superficie $F(x,y,z)=0$ nel punto $(1,1,1)$.
Quindi l'equazione del piano tangente $\pi$ in $(1,1,1)$
\`e data da:
$$(x-1,y-1,z-1)\cdot\nabla(F)(1,1,1)=0$$
e l'equazione parametrica
della retta ${\bf r}$ normale \`e data da 
$(x,y,z)=(1,1,1)+t\nabla(F)(1,1,1).$ Facendo i calcoli arriviamo a
$$\pi: 4x+9y-24z+11=0\ \ \ \ \mbox{e}\ \ \ \ {\bf r}:\left\{ 
\begin{array}{l}x=1+4t \\ y=1+9t \\ z=1-24t
\end{array}\right.$$
Poich\'e tutte e tre le derivate parziali sono non nulle nel punto $(1,1,1)$,
possiamo applicare il TFI rispetto a tutte le variabili e si ha
$$\nabla(f(x,y))(1,1)=-\left({F_x(1,1,1)\over F_z(1,1,1)},
{F_y(1,1,1)\over F_z(1,1,1)}\right)=\left({1\over 6},{3\over8}\right).$$
$$\nabla(h(x,z))(1,1)=-\left({F_x(1,1,1)\over F_y(1,1,1)},
{F_z(1,1,1)\over F_y(1,1,1)}\right)=\left(-{4\over 9},{8\over3}\right).$$
$$\nabla(g(y,z))(1,1)=-\left({F_y(1,1,1)\over F_x(1,1,1)},
{F_z(1,1,1)\over F_x(1,1,1)}\right)=\left(-{9\over 4},6\right).$$
dove $f$, $h$ e $g$ sono definite da
$F(x,y,f(x,y))=F(x,h(x,z),z)=F(g(y,z),y,z)=0$.}\bigskip

\item Si calcoli il seguente integrale:
$$\int\!\!\int_{D}x^2+y^2$$
dove $D$ \`e il dominio limitato dalle parabole $y=x^2$ e $x=y^2$.

\bigskip {\bf SOLUZIONE:} {\sl Osserviamo che $D$ \`e un $x$-dominio
infatti:
$$D=\left\{(x,y)\ |\ 0\leq x\leq1, x^2\leq y\leq\sqrt{x}\right\}$$
applicando la relativa formula per gli integrali doppi si ha
$$\int\!\!\int_{D}x^2+y^2=
\int_0^1dx\ \int_{x^2}^{\sqrt{x}}dy\ (x^2+y^2)=
\int_0^1\left[x^2y+{y^3\over3}\right]_{x^2}^{\sqrt{x}}dx=$$
$$= \int_0^1\left(x^{5/2}+{x^{3/2}\over3}-x^4-{x^6\over3}
\right)dx=\left[{2\over7}x^{7/2}+{2\over15}x^{5/2}
-{x^5\over5}-{x^7\over21}\right]^1_0={6\over35}.$$}

\item Si calcoli il seguente integrale 
$$\int\!\!\int\!\!\int_{\Omega} {1\over \sqrt{x^2+y^2+(z-2)^2}}$$
dove $\Omega$ \`e la sfera $x^2+y^2+z^2\leq1$.
%% RISPOSTA: $2\pi/3$.

\bigskip {\bf SOLUZIONE:} {\sl Utilizzando la trasformazione in coordinate
sferiche
$$\left\{\begin{array}{l} x=\rho\cos\theta\sin\omega\\
                          y=\rho\sin\theta\sin\omega\\
                          z=\rho\cos\omega
         \end{array} 
  \right.\ \ |\det(J(x,y,z))|=\rho^2\sin\omega,$$
osservando che si sta integrando su una sfera unitaria e che
$$x^2+y^2+(z-2)^2=\rho^2-4\rho\cos\omega+4,$$
otteniamo 
$$\int\!\!\int\!\!\int_{\Omega} {1\over \sqrt{x^2+y^2+(z-2)^2}}=
\int_0^{2\pi}d\theta\ \int_0^1d\rho\ \int_0^\pi d\omega
{\rho^2\sin\omega\over \sqrt{\rho^2-4\rho\cos\omega+4}}.$$
Mentre la variabile $\theta$ \`e gi\`a separata, \`e opportuno
integrare prima rispetto a $\omega$ e poi rispetto a $\rho$:
$$\int\!\!\int\!\!\int_{\Omega} {1\over \sqrt{x^2+y^2+(z-2)^2}}=
2\pi\int_0^1\rho^2d\rho\ \int_0^\pi 
{-d\cos\omega\over \sqrt{\rho^2-4\rho\cos\omega+4}}=$$
$$2\pi\int_0^1\rho^2d\rho
\left[{1\over 2\rho}\sqrt{\rho^2-4\rho\cos\omega+4}\right]_0^{\pi}=$$
$$\pi\int_0^1\rho\left[\sqrt{\rho^2+4\rho+4}-\sqrt{\rho^2-4\rho+4}
\right]d\rho=$$
$$=\pi\int_0^1\rho\left[\sqrt{(\rho+2)^2}-\sqrt{(\rho-2)^2}\right]d\rho
=4\pi\int_0^1\rho d\rho=2\pi.$$}

\item Si calcoli l'area della superficie del solido ottenuto ruotando
intorno all'asse $y$ la curva associata alla rappresentazione 
$$(x(t),y(t))=(t+1,{t^2\over2}+t)\ \ \ \mbox{con}\ \ \ t\in[0,4].$$
\bigskip 

{\bf SOLUZIONE:} {\sl L'area della superficie ottenuta dalla rotazione
intorno all asse $y$
di una rappresentazione parametrica $(x(t),y(t))$ \`e data da:
$$A=2\pi\int_{t_1}^{t_2}x(t)\cdot||(x'(t),y'(t))||dt.$$
Nel nostro caso $||(x'(t),y'(t))||=\sqrt{1+(t+1)^2}$ e quindi
$$A=2\pi\int_0^4(t+1)\sqrt{1+(t+1)^2}dt=2\pi\int_1^{25}\sqrt{1+u}{1\over2}du$$
dopo aver posto $u=(t+1)^2$, $du=2(t+1)dt$. Risolvendo l'integrale
$$A=\pi\left[{2\over3}(1+u)^{3/2}\right]_1^{25}={2\pi\over3}
(26^{3/2}-2^{3/2}).$$}


\item Si verifichi se il seguente campo \`e conservativo e si calcoli
il lavoro compiuto dal campo lungo la traiettoria ${\cal C}$
$$\underline{f}(x,y,z)=\left({1\over z},{-3\over z},{3y-x+z^3\over z^2}\right)$$
$${\cal C}=\left\{\begin{array}{ll}
x(t)=t\\ y(t)=t^2 \\ z(t)=t-1
\end{array}\right.\hspace{3cm}t\in[2,4]$$
{\it (Suggerimento: Provare a calcolare un potenziale)}
%% Risposta: $U={x-3y\over z}+z^2/2+C$
\bigskip 

{\bf SOLUZIONE:} {\sl Il campo $\underline{f}=(f_1,f_2,f_3)$ 
soddisfa le equazioni di compatibilit\`a
$${\partial f_1\over\partial y}=0={\partial f_2\over\partial x},\ \ \ \
{\partial f_1\over\partial z}=-{1\over z^2}={\partial f_3\over\partial x},
\ \ \ \ {\partial f_2\over\partial z}={3\over z^2}={\partial f_3\over\partial y}
$$
quindi $\underline{f}$ \`e un campo conservativo in una regione 
semplicemente connessa di ${\bf R}^3$ non contenente il piano $z=0$. 
Ora calcoliamo il potenziale di $\underline{f}$:
$$U=-\int f_1 dx+g(y,z)=-{x\over z}+g(y,z).$$
Adesso, da
$$f_2=-{\partial U\over\partial y}={\partial g\over\partial y}=-{3\over z}$$
si ottiene
$$g(y,z)=-{3y\over z}+h(z)\ \ \ \ \mbox{e}\ \ \ \ U(x,y,z)=-{x\over z}
+{3y\over z}+h(z).$$
Inoltre, da
$$f_3=-{\partial U\over\partial z}={3y-x\over z^2}+
{\partial h\over\partial z}={3y-x+z^3\over z^2} $$
si ottiene $h(z)={z^2\over2}$. Quindi 
$$U=-{x-3y\over z}-{z^2\over2}+C$$
e il campo \`e conservativo su tutto il suo dominio. Alla luce
di questa deduzione, per calcolare il lavoro \`e sufficiente calcolare
$$L=U(x(2),y(2),z(2))-U(x(4),y(4),z(4))=U(2,4,1)-U(4,16,3)=$$
$$=-{(-10)}-{1\over 2}+{-44\over3}+{9\over2}=-{2\over3}.$$}

\item Si utilizzi il Teorema di Green per calcolare l'area racchiusa all'interno
della curva piana associata alla rappresentazione parametrica
$$\underline{x}(t)=(a\cos^3 t ,a\sin^3 t),\ \ \  t\in[0,2\pi]$$
mediante un integrale curvilineo.
%% RISPOSTA: $6\pi a^2$
\bigskip 

{\bf SOLUZIONE:} {\sl In un Corollario del Teorema di Green si afferma che
l'area racchiusa all'interno di una curva piana chiusa ${\cal C}$ \`e data 
dall'integrale curvilineo di seconda specie:
$$A={1\over2}\oint_{\cal C}xdy-ydx.$$
Nel nostro caso abbiamo
$$A={1\over2}\int_{0}^{2\pi}(a\cos^3 t)d(a\sin^3 t)-(a\sin^3 t)d(a\cos^3 t)=$$
$$={a^2\over2}\int_{0}^{2\pi}3(\cos^4t\sin^2t+ \cos^2t\sin^4t)dt=
={3a^2\over2}\int_{0}^{2\pi}(\cos t\sin t)^2dt=$$
$$={3a^2\over2}\int_{0}^{2\pi}{1\over4}\sin^2tdt=
{3a^2\over2}\int_{0}^{2\pi}{1-\cos 4t\over8}dt={3a^2\pi\over8}.$$
}

\end{enumerate}
\end{document}
