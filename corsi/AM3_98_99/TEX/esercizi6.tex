\documentstyle{article}
\begin{document}
\centerline{\Large ESERCIZI SULLE FUNZIONI IMPLICITE, INVERSE E FORMULA DI TAYLOR.}
\bigskip\bigskip

\begin{enumerate}
\item Per ciascuna delle seguenti funzioni, si determini il polinomio di Taylor
 di grado $k$ intorno al punto $\underline{x}_0$ utilizzando il polinomio
di Taylor di una funzione di una variabile reale opportuna. Si verifichi
anche che il resto $R_k((\underline{x}-\underline{x}_0))=
O(||\underline{x}-\underline{x}_0||^{k+1})$.
\begin{center}
\begin{tabular}{|r||l|c|c|}
\hline
 & $f(\underline{x})$ & $\underline{x}_0$ & $k$\\
\hline
\hline
{\it i} &$\sqrt{1+x_1^3x_2} $ & $(0,0)$ & $11$\\
\hline
{\it ii} &$\arctan(x_1x_2x_3^2)$ & $(0,0,0)$ & $13$\\
\hline
{\it iii} &$\sqrt{1+x_1x_2}\ln{(1+\sqrt{xy})}$ & $(0,0)$ & $8$\\
\hline
{\it iv} &$(x_1^3+x_2^2)\sin{(x_1^2x_2)}$ & $(0,0)$ & $10$ \\
\hline
{\it v} & $e^{x^2}\cos{y^4}$ & $(0,0)$ & $10$\\
\hline
\end{tabular}
\end{center}
\item Per ciascuna delle seguenti funzioni, si determini
\begin{enumerate}
\item la matrice Jacobiana nel generico punto $\underline{x}$;
\item Il luogo dei punti in cui non \`e possibile applicare
il teorema della funzione inversa;
\item Il differenziale e lo spazio affine tangente a $f^{-1}$ 
nel punto $\underline{x}_0$.
\item verificare anche che $J(f)_{\underline{x}_0}^{-1}=
J(f^{-1})_{f(\underline{x}_0)}$.
\end{enumerate}
\begin{center}
\begin{tabular}{|r||l|c|}
\hline
 & $f(\underline{x})$ & $\underline{x}_0$ \\
\hline
\hline
{\it i} &$\underline{f}(x,y)=\left(\begin{array}{l}\sqrt{x^2+y} \\
\cos xy +x +3y \end{array}\right)$ & $(0,0)$ \\
\hline
{\it ii} &$\underline{f}(x,y)=\left(\begin{array}{l} 2x+y^2e^{xy} \\
\ln(1+xy) \end{array}\right) $ & $(0,1)$ \\
\hline
{\it iii} &$\underline{f}(x,y,z)=\left(\begin{array}{l} e^{x+y+z} \\
2y^2+x+\cos(z\pi)\\
\ln(1+xz) \end{array}\right) $ & $(1,1,0)$ \\
\hline
{\it iv}& $\underline{f}(x,y,z,t)=\left(\begin{array}{l} t^2 \\
x^3+y\\
y^2+tz\\
z+\ln x \end{array}\right) $ & $(1,1,1,1)$\\
\hline
\end{tabular}
\end{center}

\item Per ciascuna delle seguenti funzioni di una variabile reale 
$y=g(x)$, definite in modo implicito da $f(x,y)=0$, dopo aver verficato 
che \`e possibile applicare il Teorema della funzione implicita, si calcoli 
\begin{enumerate}
\item la derivata nel punto $x_0$;
\item l'equazione della retta tangente nel punto $(x_0,g(x_0))$;
\item il polinomio di Taylor di grado due intorno al punto $x_0$.
\end{enumerate}
\begin{center}
\begin{tabular}{|r||l|c|}
\hline
 & $f(x,y)$ & $(x_0,g(x_0))$ \\
\hline
\hline
{\it i} &$x^4+y^4-17$ & $(2,1)$ \\
\hline
{\it ii} &$\arctan(x^3+y-7)+xy^3+2$ & $(2,-1)$ \\
\hline
{\it iii} &$\ln(1+\cosh(yx^2+1))+y^3-\ln(2x)+1$ & $(1,-1)$ \\
\hline
{\it iv} &$\sqrt{x^2+y^3}-\sqrt{5}e^{xy-2} $ & $(2,1)$ \\
\hline
\end{tabular}
\end{center}
\item Per ciascuna delle seguenti funzioni di due variabili reali $z=g(x,y)$
definite in modo implicito da $f(z,y,z)=0$, si calcoli (se esiste) nel punto $(x_0,y_0)$:
\begin{enumerate}
\item Il gradiente $\nabla g(x_0,y_0)$;
\item Il differenziale $dg_{(x_0,y_0)}$;
\item La derivata direzionale ${\partial g\over \partial (v_1,v_2)}$ nel 
generico punto $(x,y)$;
\item L'equazione del piano tangente alla superficie $z=f(x,y)$ nel punto 
$(x_0,y_0,g(x_0,y_0))$.
\end{enumerate}
\begin{center}
\begin{tabular}{|r||l|c|c|}
\hline
 & $f(x,y,z)$ & $(x_0,y_0,g(x_0,y_0))$ & $(v_1,v_2)$\\
\hline
\hline
{\it i} &$x^4+y^4+z^4-16$ & $(0,0,2)$ & $(1,1)$\\
\hline
{\it ii} &$x^2\cos z+e^{yz}$ & $(1,0,1)$ & $(1,2)$\\
\hline
{\it iii} &$\ln(1+x^2z^3\arctan y)$ & $(2,0,1)$ & $(-1,1)$\\
\hline
{\it iv} &$\cosh(x^3-z^2+y)$ & $(1,0,-1)$ & $(-1,-1)$ \\
\hline
\end{tabular}
\end{center}
\item Per ciascuna delle seguenti funzioni da ${\bf R}^3$ a ${\bf R}^2$,
si verifichi se si pu\`o
applicare il Teorema della funzione implicita nel punto $({x}_0,
\underline{y}_0)$
e si calcoli la matrice Jacobiana e l'equazione dello spazio
affine tangente nel punto ${x}_0$ della funzione $\underline{g}(x)$
definita implicitamente da $\underline{y}_0=\underline{g}({x}_0)$
e $F({x},\underline{g}({x}))=\underline{0}$.

\begin{enumerate}
\item $f(x,y_1,y_2)=(xy_1^2+y_2^3,y_2^2+x)\hspace{2cm}(x_0,\underline{y}_0)=(1,2,2)$
\item $f(x,y_1,y_2)=(\ln(x+y_1),y_1\ln(x+y_2))\hspace{1cm}(x_0,\underline{y}_0)=(1,1,1)$
\item $f(x,y_1,y_2)=(\cos{xy_1}+y_2,\sin(xy_2)+y_1)\hspace{1cm}(x_0,\underline{y}_0)=(1,0,\pi)$
\item $f(x,y_1,y_3)=(\sqrt{x+y_1},y_1y_2+x^2)\hspace{1cm}(x_0,\underline{y}_0)=(1,1,1)$
\end{enumerate}
\end{enumerate}
\end{document}
