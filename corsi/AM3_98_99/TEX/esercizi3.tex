\documentstyle[12pt]{article}
\begin{document}
\centerline{\Large Esercizi di Topologia in ${\bf R}^n$ (II)}
\bigskip\bigskip
\noindent{\bf A.} Dopo averne tracciato il grafico, si dimostri che
 ciascuno dei seguenti insiemi non \`e compatto costruendo in ciascun caso
un  ricoprimento di dischi che non ammette un sottoricoprimento 
finito:
\begin{enumerate}
\item in ${\bf R}^2$:\ \ $\left\{(x,y)\ \mbox{t.c.}\ |x|+|y|<2\right\};$
\item in ${\bf R}^3$:\ \ $\left\{(x_1,x_2,x_3)\ \mbox{t.c.}\ x_1^2+x_2^2=1,\ 
x_3\in(-2,2)\right\};$
\item in ${\bf R}^2$:\ \ $\left\{(x,y)\ \mbox{t.c.}\ |x-y|<1, x\in[-2,2]
\right\};$
\item in ${\bf R}^3$:\ \ $\left\{(x,y,z)\ \mbox{t.c.}\ 1<x^2+y^2+z^2\leq 4\right\};$
\item in  ${\bf R}^3$:\ \ $\left\{(x,y,z)\ \mbox{t.c.}\ x^2-y^2-z^2<0, x\in[0,4]\right\}.$
\end{enumerate}
\bigskip
\noindent{\bf B.} Per ciascuna delle seguenti funzioni si tracci il grafico del 
dominio ${\bf Dom}(f)$ e si determinino:\\
l'interno ${\bf Dom}(f)^o$,
la frontiera $\partial{\bf Dom}(f)$, la chiusura $\overline{{\bf Dom}(f)}$
e il derivato $D({\bf Dom}(f))$:\medskip
\begin{enumerate}
\item $f(x,y)=\sqrt{{x\over y}(x^2+(y-3)^2)};$
\item $f(x,y)=\log(x^2+(y-5)^2)$;
\item $f(x,y)=\tan\left({\pi\over2}xy\right)$;
\item $f(x,y)=\sqrt{\log(x^2+3y^2)};$
\item $f(x,y)=\tan\left({\pi\over 2}(x+y)\right)+\log(9-x^2-y^2)$
\item $f(x,y,z)={1\over \cos (x^2+y^2+z^2)}.$
\end{enumerate}\bigskip
\noindent{\bf C.} Si dimostri il seguente ``criterio di continuit\`a in coordinate polari'':\\
{\it Sia $f:A\longrightarrow {\bf R}$ una funzione dove $(0,0)\in A^o\subset{\bf R}^n$. Allora
$f$ \`e continua in $(0,0)$ se e solo se per ogni $\epsilon>0$ esiste $\delta_\epsilon$ tale che
se $0<r<\delta_\epsilon$ allora $|f(r\cos\theta,r\sin\theta)-f(0,0)|<\epsilon$.}\\
Si applichi questo criterio per studiare la continuit\`a della seguente funzione al variare di
$\alpha, \beta\in{\bf R}$.
$$f(x,y)=\left\{ \begin{array}{ll}
{x^\alpha y^\beta\over (x^2+y^2)\arctan(y^2+1)} &\ \mbox{if}\ (x,y)\neq(0,0)\\
\\
0 & \ \mbox{if}\ (x,y)=(0,0)\\
\end{array}\right.$$
\bigskip

\noindent{\bf D.} Si discuta la continuit\`a delle seguenti 
funzioni in $(x,y)=(0,0)$:
\begin{enumerate}
\item 
$$f(x,y)=\left\{ \begin{array}{ll}
{x^3y^4\cos(1/x)\over (x^2+y^2)\arctan(y^2+x^6+1)} &\ \mbox{if}\ (x,y)\neq(0,0)\\
\\
0 & \ \mbox{if}\ (x,y)=(0,0)\\
\end{array}\right.$$
\item
$$f(x,y)=\left\{ \begin{array}{ll}
{x^3y^4\over (x^{6}+y^{6})} &\ \mbox{if}\ (x,y)\neq(0,0)\\
\\
0 & \ \mbox{if}\ (x,y)=(0,0)\\
\end{array}\right.$$
\item$$f(x,y)=\left\{ \begin{array}{ll}
{x^2y^4\over x^2+y^6} &\ \mbox{if}\ (x,y)\neq(0,0)\\
\\
0 & \ \mbox{if}\ (x,y)=(0,0)\\
\end{array}\right.$$
\item$$f(x,y)=\left\{ \begin{array}{ll}
{x^3+y^3\over x+y} &\ \mbox{if}\ x+y\neq0\\
\\
0 & \ \mbox{if}\ x+y=0\\
\end{array}\right.$$
\item 
$$f(x,y)=\left\{ \begin{array}{ll}
{xy\sin(x+y)\over \pi-2\arctan(y/x)} &\ \mbox{if}\ x\neq0\\
\\
0 & \ \mbox{if}\ x=0\\
\end{array}\right.$$
\end{enumerate}
\end{document}
