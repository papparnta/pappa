\documentclass[12pt,a4paper]{report}\pagenumbering{roman}
\pagestyle{empty}
\begin{document}
\begin{center}
{\bf COMPITO FINALE}\\
{\bf Analisi due (Primo modulo) - Corso di Laurea in FISICA}\\
{\bf Sabato 23 Dicembre, 1998}
\end{center}
{\Large\bf LEGGERE ATTENTAMENTE:}
\begin{itemize}
\item Il presente esame consiste di 10 esercizi. Ogni esercizio 
vale 10 punti su 100.
%\item Il compito non sar\`a sufficiente se non si risolve almeno 
%un esercizio del gruppo 1. 2. 3., almeno 
%uno del gruppo 4. 5. 6. e almeno uno del gruppo  7. 8. 9. 10.
\item Non sono ammessi appunti, calcolatrici, 
libri, tavole di integrali e telefoni cellulari.
\item Il tempo concesso per svolgere il compito \`e di 3 ore.
\item Per la brutta copia \`e consentito utilizzare 
esclusivamente fogli consegnati dal docente.
\item Tutti gli effetti personali, compresi borse e cappotti, devono 
essere lasciati accanto agli attaccapanni (ad eccezione della penna!).
\item Non \`e consentito consegnare altri fogli oltre agli 11 (undici)
del presente fascicolo.
\item Scrivere a penna e tenere il libretto (o un altro documento) sul banco per
il riconoscimento.
\item {\bf Non \`e consentito parlare o comunicare in nessun modo, pena
il ritiro immediato del compito.} 
\end{itemize}
\begin{center}
\begin{tabular}{||r|r||}
\hline ESERCIZIO & PUNTI\\  \hline
\hline 1.& \\
\hline 2.& \\
\hline 3.& \\
\hline 4.& \\
\hline 5.& \\
\hline 6.& \\
\hline 7.& \\
\hline 8.& \\
\hline 9.& \\
\hline 10.& \\
\hline TOTALE & /100\\
\hline VOTO& /30\\
\hline
\end{tabular}
\end{center}
\pagebreak

\begin{enumerate}
\item Si Calcoli il polinomio di Taylor intorno a $(0,0)$ di grado 20
della seguente funzione:
$$f(x,y)=\ln(1+x^4y^3)+\arctan(x^6y^4).$$
\hspace*{-3.5cm}\begin{tabular}{c}\hline\\\hspace*{16cm}\end{tabular}\\
\hspace*{-3.5cm}{\bf SVOLGIMENTO:}\pagebreak

\item Si scriva il polinomio di Taylor di grado due intorno al punto $0$
della funzione $y=f(x)$ definita implicitamente da
$$\left\{\begin{array}{l}
x^3y+y^3-\cos x=0\\
f(0)=1
\end{array}\right.$$
\hspace*{-3.5cm}\begin{tabular}{c}\hline\\\hspace*{16cm}\end{tabular}\\
\hspace*{-3.5cm}{\bf SVOLGIMENTO:}\pagebreak

\item  Sia
$$\underline{f}(x,y,z)=\left(\begin{array}{l}
xz\\
y^2+x+1\\
xyz+1\end{array}
\right).$$ 
Dopo aver verificato che $\underline{f}$ \`e invertibile in $(1,0,1)$,
si scriva la matrice Jacobiana nel punto $(1,2,1)$ della
funzione inversa.\\
({\it Suggerimento:} $(1,2,1)=f(1,0,1)$.)

\hspace*{-3.5cm}\begin{tabular}{c}\hline\\\hspace*{16cm}\end{tabular}\\
\hspace*{-3.5cm}{\bf SVOLGIMENTO:}\pagebreak

\item Si calcoli la lunghezza della curva associata alla seguente rappresentazione 
parametrica:
$$\underline{x}(t)=\left(2t,\ln t,t^2\right), t\in[1,10].$$
%% RISPOSTA: $99+\ln 10$
\hspace*{-3.5cm}\begin{tabular}{c}\hline\\\hspace*{16cm}\end{tabular}\\
\hspace*{-3.5cm}{\bf SVOLGIMENTO:}\pagebreak

\item Si calcoli l'equazione del piano tangente e quella della retta normale alla
superficie:
$$x^4+3y^3-4z^6=0$$
nel punto $P=(1,1,1)$. Si dica inoltre rispetto a quale delle tre variabili
si pu\`o applicare il teorema della funzione implicita nel punto $P$ e si calcoli
il gradiente delle funzioni cos\`\i\ definite.

\hspace*{-3.5cm}\begin{tabular}{c}\hline\\\hspace*{16cm}\end{tabular}\\
\hspace*{-3.5cm}{\bf SVOLGIMENTO:}\pagebreak 

\item Si calcoli il seguente integrale:
$$\int\!\!\int_{D}x^2+y^2$$
dove $D$ \`e il dominio limitato dalle parabole $y=x^2$ e $x=y^2$.
%% RISPOSTA: $$33/140$$ 

\hspace*{-3.5cm}\begin{tabular}{c}\hline\\\hspace*{16cm}\end{tabular}\\
\hspace*{-3.5cm}{\bf SVOLGIMENTO:}\pagebreak

\item Si calcoli il seguente integrale 
$$\int\!\!\int\!\!\int_{\Omega} {1\over \sqrt{x^2+y^2+(z-2)^2}}$$
dove $\Omega$ \`e la sfera $x^2+y^2+z^2\leq1$.
%% RISPOSTA: $2\pi/3$.

\hspace*{-3.5cm}\begin{tabular}{c}\hline\\\hspace*{16cm}\end{tabular}\\
\hspace*{-3.5cm}{\bf SVOLGIMENTO:}\pagebreak

\item Si calcoli l'area della superficie del solido ottenuto ruotando
intorno all'asse $x$ la curva associata alla rappresentazione 
$$(x(t),y(t))=(t+1,t^2/2+t)\ \ \ \mbox{con}\ \ \ t\in[0,4].$$

\hspace*{-3.5cm}\begin{tabular}{c}\hline\\\hspace*{16cm}\end{tabular}\\
\hspace*{-3.5cm}{\bf SVOLGIMENTO:}\pagebreak

\item Si verifichi se il seguente campo \`e conservativo e si calcoli
il lavoro compiuto dal campo lungo la traiettoria ${\cal C}$
$$\underline{f}(x,y,z)=\left({1\over z},{-3\over z},{3y-x+z^3\over z^2}\right)$$
$${\cal C}=\left\{\begin{array}{ll}
x(t)=t\\ y(t)=t^2 \\ z(t)=t-1
\end{array}\right.\hspace{3cm}t\in[2,4]$$
{\it (Suggerimento: Provare a calcolare un potenziale)}

%% Risposta: $U={x-3y\over z}+z^2/2+C$
\hspace*{-3.5cm}\begin{tabular}{c}\hline\\\hspace*{16cm}\end{tabular}\\
\hspace*{-3.5cm}{\bf SVOLGIMENTO:}\pagebreak

\item Si utilizzi il Teorema di Green per calcolare l'area racchiusa all'interno
della curva piana associata alla rappresentazione parametrica
$$\underline{x}(t)=(a\cos^3 t ,a\sin^3 t),\ \ \  t\in[0,2\pi]$$
mediante un integrale curvilineo.
%% RISPOSTA: $6\pi a^2$

\hspace*{-3.5cm}\begin{tabular}{c}\hline\\\hspace*{16cm}\end{tabular}\\
\hspace*{-3.5cm}{\bf SVOLGIMENTO:}

\end{enumerate}
\end{document}
