\documentstyle{article}
\begin{document}
\centerline{\Large ESERCIZI SULLE DERIVATE PARZIALI (I).}
\bigskip\bigskip

\begin{enumerate}
\item Si calcolino tutte le derivate parziali delle seguenti funzioni:
$$x^{(y^z)}, \hspace{1cm}(x^y)^z.$$
\item Per ciascuna delle seguenti funzioni, si calcoli (se esiste):
\begin{enumerate}
\item Il gradiente $\nabla f(x_0,y_0)$;
\item Il differenziale $df_{(x_1,y_1)}$;
\item La derivata direzionale ${\partial f\over \partial (v_1,v_2)}$ nel 
generico punto $(x,y)$;
\item L'equazione del piano tangente alla superficie $z=f(x,y)$ nel punto 
$(x_2,y_2,f(x_2,y_2))$.
\end{enumerate}
\begin{center}
\begin{tabular}{|r|l|c|c|c|c|}
\hline
& $f(x,y)$ & $(x_0,y_0)$ & $(x_1,y_1)$ & $(x_2,y_2)$ & $(v_1,v_2)$\\ 
\hline
\hline
{\it i} &$\sin(xy)$ & $(1,0)$ & $({\pi\over2},1)$ & $(7,{\pi\over 7})$ & $(3,2)$ \\
\hline
{\it ii} &$\log(x^2+y^2)$ & $(1,1)$ & $(-1,0)$ & $(2,1)$ & $(2,-1)$\\
\hline
{\it iii} &$e^{x+2y}$ & $(0,0)$ & $(1,0)$& $(2,-1)$& $(1,-3)$\\
\hline
{\it iv} &$\tan(x^3y)$ & $(1,{\pi\over 4})$ & $(1,2)$ & $(1,1)$ & $(1,-3)$\\
\hline
{\it v} &${x+y\over x^4+y^4+1}$ & $(1,-1)$& $(0,1)$& $(-2,1)$& $(5,2)$\\
\hline
{\it vi} &$\arctan{x\over y}$& $(-1,1)$& $(\sqrt{3},1)$& $(0,1)$& $(3,4)$\\
\hline
{\it vii} &$\log(\cos(x+y))$ & $({\pi\over 8},{\pi\over 8})$& $(0,{\pi\over4})$& $({\pi\over3}-1,1)$& $(2,0)$\\
\hline
{\it viii} &${\cos x\over y}$ & $({\pi\over3},1)$& $(0,1)$& $({\pi\over3},2)$& $(1,-1)$\\
\hline
{\it ix} &$e^{{x+y\over x+1}}$ & $(1,2)$& $(-1,2)$& $(1,0)$& $(0,3)$\\
\hline
{\it x} &$\sqrt{y^2+\cos^2(x)}$& $({\pi\over3},1)$& $({\pi\over2},0)$& $(0,2)$& $(-1,-4)$\\
\hline
\end{tabular}
\end{center}
\item Per ciascuna delle seguenti, si calcoli la derivata 
${d h\over d t}(t_0)$ dove $h(t)=f(l(t))$ usando la regola di derivazione
delle funzioni composte:
\begin{center}
\begin{tabular}{|r|l|c|c|}
\hline
& $f(x,y)$ & $l(t)=(l_1(t),l_2(t))$ & $t_0$\\
\hline
\hline
{\it i} & $x^3+y^3+1$ & $(t\cos t,t\sin t)$ & $3$\\
\hline
{\it ii} &$e^{x+y}$ & $(t^3,\log(t^2+1))$ & $2$\\
\hline
{\it iii} &$3x^2+3y^2$ & $(\cos 2t,\sin 2t)$ & $100$\\
\hline
\end{tabular}
\end{center}
\item Si determinino le derivate parziali di $f(x,y)=g(h(x,y))$ (usando la regola di 
derivazione delle funzioni composte) dove
\begin{enumerate}
\item $h(x,y)=(3x^2+\cos(xy), y+2x^3),\ \ \ g(s,t)=e^s\cos t$;
\item $h(x,y)=({x\over y+1}, \sin y, \cos y),\ \ \ g(r,s,t)=r^2+s^2+t^2$;
\item $h(x,y)=(xy,y\cos x,x\log y,y),\ \ \ g(r,s,t,u)=\log sr+\cos tu+s+t$;
\end{enumerate}
\item Si derminino tutti i punti di ${\bf R}^2$ per cui almeno una delle derivate 
direzionali della seguente funzione \`e non zero:
$$f(x,y)=\left\{\begin{array}{ll} 0 & \mbox{se}\ x\neq y \\ x & x=y \end{array} \right.$$
Dimostrare che in $(0,0)$, $f$ \`e continua, ha tutte le derivate direzionali ma non \`e 
differenziabile.
\item Si determini il luogo dei punti di ${\bf R}^2$ per cui l'angolo tra il vettore 
$(3,-1)$ e il gradiente della funzione $f(x,y)=x^6y+3$ \`e pari a $\pi/6$ radianti.
\item Si determinino i punti stazionari delle seguenti funzioni:
\begin{enumerate}
\item $f(x,y)=x^2+(y-1)^3$;
\item $f(x,y)=xy^3+y$;
\item $f(x,y,z)=(x-1)^3+(y-2)^2+\cos  z$.
\end{enumerate}
\item Si risolvano le seguenti equazioni differenziali usando il metodo (se
possibile) dei differenziali esatti
\begin{enumerate}
\item $\left(3x^2e^{x^3+y^2}+y^2\right)dx+\left(2ye^{x^3+y^2}+2xy\right)dy=0;$ 
            %$e^{x^3+y^2}+xy^2=C$ 
\item $\left(1+2xe^{x^2+y}\right)dx+\left(e^{x^2+y}-\sin y\right)dy=0$
      %$(x+2xe^{x^2+y}+\cos y=C$
\item $\left(y+{3x^2\over 2\sqrt{x^3+y^3+1}}\right)dx+
\left(x+{3y^2\over2\sqrt{x^3+y^3+1}}\right)dy=0$
% $xy+\sqrt{x^3+y^3+1}=C$
\item $\left(3x^2+{2x\over x^2+y^2}\right)dx+\left({2y\over x^2+y^2}+
{1\over1+y^2}\right)dy=0$ %$x^3+\log(x^2+y^2)+\arctan(y)=C$
\end{enumerate}
\item Si calcoli la matrice Hessiana in un generico punto delle seguenti
funzioni:
\begin{enumerate}
\item $f(x,y)=x^3y+\cos(y^2+x)$;
\item $f(x,y,z)=||(x,y,z)||=\sqrt{x^2+y^2+z^2}$;
\item $f(x,y,z)=\log(x^2+y^2+1)$;
\item $f(x,y)=arctan(x^2+2y+1)$;
\item $f(x,y)=e^{x+y\over x-y}$. 
\end{enumerate}
\end{enumerate}
\end{document}
