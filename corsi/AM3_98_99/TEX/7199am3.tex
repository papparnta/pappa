\documentclass[12pt,a4paper]{report}\pagenumbering{roman}
\pagestyle{empty}
\begin{document}
\begin{center}
\textbf{PRIMO COMPITO}\\
\textbf{Analisi due (Primo modulo) - Corso di Laurea in FISICA}\\
\textbf{Gioved\`\i\ 7 Gennaio, 1999}
\end{center}
{\Large\textbf{LEGGERE ATTENTAMENTE:}}
\begin{itemize}
\item Il presente esame consiste di 10 esercizi. Ogni esercizio 
vale 10 punti su 100.
\item Il compito non sar\`a sufficiente se non si risolve almeno 
un esercizio del gruppo 1. 2. 3., almeno 
uno del gruppo 4. 5. 6. e almeno uno del gruppo  7. 8. 9. 10.
\item Non sono ammessi appunti, calcolatrici, 
libri, tavole di integrali e telefoni cellulari.
\item Il tempo concesso per svolgere il compito \`e di 3 ore.
\item Per la brutta copia \`e consentito utilizzare 
esclusivamente fogli consegnati dal docente.
\item Tutti gli effetti personali, compresi borse e cappotti, devono 
essere lasciati accanto agli attaccapanni (ad eccezione della penna!).
\item Non \`e consentito consegnare altri fogli oltre agli 11 (undici)
del presente fascicolo.
\item Scrivere a penna e tenere il libretto (o un altro documento) sul banco per
il riconoscimento.
\item \textbf{Non \`e consentito parlare o comunicare in nessun modo, pena
il ritiro immediato del compito.} 
\end{itemize}
\begin{center}
\begin{tabular}{||r|r||}
\hline ESERCIZIO & PUNTI\\  \hline
\hline 1.& \\
\hline 2.& \\
\hline 3.& \\
\hline 4.& \\
\hline 5.& \\
\hline 6.& \\
\hline 7.& \\
\hline 8.& \\
\hline 9.& \\
\hline 10.& \\
\hline TOTALE & /100\\
\hline
\end{tabular}
\end{center}
\pagebreak

\begin{enumerate}
% TRE EQUAZIONI DIFFERENZIALI
\item Si trovi la soluzione generale della seguente equazione:
$$y''-y'-2y=2e^{-x}.$$
%y_p(x)=-{2/3}xe^{-x}.
\hspace*{-3.5cm}\begin{tabular}{c}\hline\\\hspace*{15.7cm}\end{tabular}\\
\hspace*{-3.5cm}{\textbf SVOLGIMENTO:}\pagebreak

\item Si risolva il seguente problema di Cauchy:
$$\left\{\begin{array}{l} y'y''=2 \\ y(0)=1\\ y'(0)=2
\end{array}\right.$$
% y={4\over 3}(x+1)^{3/2}-1/3
\hspace*{-3.5cm}\begin{tabular}{c}\hline\\\hspace*{15.7cm}\end{tabular}\\
\hspace*{-3.5cm}{\textbf SVOLGIMENTO:}\pagebreak

\item Si calcoli $e^A$ dove 
$$A=\left(\begin{array}{cc}
-11 & -3 \\
 36 & 10
\end{array}\right).$$

\hspace*{-3.5cm}\begin{tabular}{c}\hline\\\hspace*{15.7cm}\end{tabular}\\
\hspace*{-3.5cm}{\textbf SVOLGIMENTO:}\pagebreak
% TRE ESERCIZI DI TOPOLOGIA
\item Sia 
$$A=\left\{(x,y)\in{\mathbf R}^2\ \left|\ x\in[-1,1)\ y\in(-1,1]\right.\right\}\bigcap
\left\{(x,y)\in{\mathbf R}^2\ \left|\ x^2+y^2\geq1\right.\right\}.$$ 
Dopo aver tracciato la figura di $A$, se ne determini
l'interno, la chiusura, la frontiera e il derivato.

\hspace*{-3.5cm}\begin{tabular}{c}\hline\\\hspace*{15.7cm}\end{tabular}\\
\hspace*{-3.5cm}{\textbf SVOLGIMENTO:}\pagebreak

\item Si dimostri che il seguente sottoinsieme di ${\mathbf R}^2$ \`e denso in ${\mathbf R}^2$.
$$S=\left\{(x,y)\in{\mathbf R}^2\ \mbox{t.c.}\ y+x\not\in{\mathbf Z}\right\}.$$
Si dimostri che il complementare di $S$ non \`e discreto.

\hspace*{-3.5cm}\begin{tabular}{c}\hline\\\hspace*{15.7cm}\end{tabular}\\
\hspace*{-3.5cm}{\textbf SVOLGIMENTO:}\pagebreak 

\item Si discuta la continuit\`a della seguente funzione $f:{\mathbf R}^2\longrightarrow
{\mathbf R}$ su tutti i punti di ${\mathbf R}^2$:
$$f(x,y)=\left\{\begin{array}{ll} {x|y|\over y^2+|x|} & 
\mbox{se}\ \ (x,y)\neq (0,0)\\ \\0&\mbox{se}\ \ (x,y)=(0,0)\end{array}\right.$$
\hspace*{-3.5cm}\begin{tabular}{c}\hline\\\hspace*{15.7cm}\end{tabular}\\
\hspace*{-3.5cm}{\textbf SVOLGIMENTO:}\pagebreak
% QUATTRO ESERCIZI SULLA DIFFERENZIABILITA`
\item Si calcoli il differenziale nel punto $(1,1)$ della funzione 
$$f(x,y)=\sin{\pi(x^2+y^2)\over8}.$$
Si trovi un punto $(x_0,y_0)$ tale che $df_{(x_0,y_0)}=-{\pi\over\sqrt{2}}dx$.

\hspace*{-3.5cm}\begin{tabular}{c}\hline\\\hspace*{15.7cm}\end{tabular}\\
\hspace*{-3.5cm}{\textbf SVOLGIMENTO:}\pagebreak

\item Si calcoli l'equazione della quadrica tangente nel punto $(0,{\pi\over2})$
della funzione $f(x,y)=x\cos(y+x).$  

\hspace*{-3.5cm}\begin{tabular}{c}\hline\\\hspace*{15.7cm}\end{tabular}\\
\hspace*{-3.5cm}{\textbf SVOLGIMENTO:}\pagebreak

\item Sia $f(x,y)=\arctan(y)\ \arctan(x-1)$. Determinare i punti critici di 
$f$ e classificarli con il metodo della matrice Hessiania.

\hspace*{-3.5cm}\begin{tabular}{c}\hline\\\hspace*{15.7cm}\end{tabular}\\
\hspace*{-3.5cm}{\textbf SVOLGIMENTO:}\pagebreak

\item Siano
$$f(x,y)=\ln(x^2+3+\cos(y)),\ \ \ \ g(t)=\sqrt{t},\ \ \ \ h(t)=\arccos t$$
e $F(t)=f(g(t),h(t))$. Utilizzare la regola di derivazione delle funzioni
composte per calcolare 
$${d\over dt}F(t).$$
\hspace*{-3.5cm}\begin{tabular}{c}\hline\\\hspace*{15.7cm}\end{tabular}\\
\hspace*{-3.5cm}{\textbf SVOLGIMENTO:}

\end{enumerate}
\end{document}
