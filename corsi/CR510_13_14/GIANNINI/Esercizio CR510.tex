\documentclass[a4paper]{article}
\usepackage{amsmath}
\usepackage{amsthm}
\usepackage{amssymb}
\usepackage[italian]{babel}
\usepackage[T1]{fontenc}
\usepackage[latin1]{inputenc}
\usepackage[T1]{fontenc}
\usepackage{latexsym}
\usepackage{eucal}
\newcommand{\C}{\mathbb{C}}
\newcommand{\R}{\mathbb{R}}
\newcommand{\bm}{ \begin{pmatrix} }
\newcommand{\edm}{ \end{pmatrix} }  
\newcommand{\sist}{\begin{cases}}
\newcommand{\esist}{\end{cases}}
\newcommand{\deter}{\left| \begin{array}}
 \newcommand{\edet}{\end{array}\right|}
\newcommand{\ds}{\displaystyle}

\begin{document}
  \begin{center} 
    \textbf{Universit\'a degli Studi Roma Tre - Corso di Laurea in Matematica\\
     \begin{Huge}Esercizio supplementare CR510\end{Huge}\\
Dario Giannini}\\
  \end{center}
  
  \begin{tabular}{p{.6\textwidth}p{.4\textwidth}}
    \begin{tabular}{l}
      \textsc{24 Marzo 2014}
    \end{tabular}
  \end{tabular}
\\\\
\begin{enumerate}
\item Dato un primo dispari $p$ e un elemento a in $\mathbb{F}_p$. Dimostrare che:
\begin{enumerate}
\item Per ogni $x$ in $\mathbb{F}_p$ esistono $u$ e $v$ in $\mathbb{F}_p$ tali che $x=u^2-av^2$.
\item Sia $\alpha$ in $\mathbb{F}_{p^2}$ radice quadrata di $a$, allora $\phi:\;\mathbb{F}_p[\alpha]^*\longrightarrow \mathbb{F}_p^*$, $u+\alpha v\longmapsto u^2-(\alpha v)^2$ \'e un omomorfismo di gruppi moltiplicativi.\\
\end{enumerate}
\textbf{SOLUZIONE:}\\
\begin{enumerate}
\item Si considerino i due insiemi:\\ 
$A:=\{u\in\mathbb{F}_p : \;\exists k\in\;\mathbb{F}_p$ per cui $k^2\equiv u(mod p)\}$\\
$B:=\{x+av^2: v\in\mathbb{F}_p\}$\\
$A$ di fatto \'e l'insieme dei redidui quadratici modulo $p$ quindi avr\'a cardinalit\'a uguale a 
$\ds{\frac{p+1}{2}}$. Stessa cosa si pu\'o dire del secondo insieme in quanto $a$ e $x$ sono fissati e i suoi elementi sono ottenuti al variare di $v$ nell'insieme dei residui quadratici modulo $p$.\\
Riassumendo il tutto si ha che:
$$|A|+|B|=p+1>p=|\mathbb{F}_p|$$
Quindi per il principio delle gabbie e dei piccioni si deve avere che $A\cap B \neq \emptyset$, ossia devono esistere $u$ e $v$ in $\mathbb{F}_p$ tali che $x=u^2-av^2$.\\

 
\item Si vuole dimostrare che $\phi$ \'e un omomorfismo di gruppi moltiplicativi:
$\phi(u_1+\alpha v_1)*\phi (u_2+\alpha v_2)=\phi[(u_1+\alpha v_1)*(u_2+\alpha v_2)]$.\\
$\phi(u_1+\alpha v_1)*\phi (u_2+\alpha v_2)=(u_1^2-av_1^2)*(u_2^2-av_2^2)=$\\
$u_1^2u_2^2-av_2^2u_1^2-av_1^2u_2^2+a^2v_1^2v_2^2$.\\
$\phi[(u_1+\alpha v_1)*(u_2+\alpha v_2)]=\phi[(u_1u_2+av_1v_2)+\alpha(v_1u_2+u_2v_1)]=$\\
$u_1^2u_2^2-av_2^2u_1^2-av_1^2u_2^2+a^2v_1^2v_2^2$.\\

\end{enumerate}
\end{enumerate}
\end{document}