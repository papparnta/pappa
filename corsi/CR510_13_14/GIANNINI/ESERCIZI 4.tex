\documentclass[a4paper]{article}
\usepackage{amsmath}
\usepackage{amsthm}
\usepackage{amssymb}
\usepackage[italian]{babel}
\usepackage[T1]{fontenc}
\usepackage[latin1]{inputenc}
\usepackage[T1]{fontenc}
\usepackage{latexsym}
\usepackage{eucal}
\newcommand{\C}{\mathbb{C}}
\newcommand{\R}{\mathbb{R}}
\newcommand{\bm}{ \begin{pmatrix} }
\newcommand{\edm}{ \end{pmatrix} }  
\newcommand{\sist}{\begin{cases}}
\newcommand{\esist}{\end{cases}}
\newcommand{\deter}{\left| \begin{array}}
 \newcommand{\edet}{\end{array}\right|}
\newcommand{\ds}{\displaystyle}

\begin{document}
  \begin{center} 
    \textbf{Universit\'a degli Studi Roma Tre - Corso di Laurea in Matematica\\
     \begin{Huge}Esercizi CR510\end{Huge}\\
Dario Giannini}\\
  \end{center}
  
  \begin{tabular}{p{.6\textwidth}p{.4\textwidth}}
    \begin{tabular}{l}
      \textsc{3 Aprile 2014}
    \end{tabular}
  \end{tabular}
\\\\
\begin{enumerate}
\item Sia $char(K)=3$ e sia $E:\; y^2=x^3+a_2x^2+a_4x+a_6$ curva ellittica con coefficienti in $K$.

$$j(E):=\ds{\frac{a_2^6}{a_2^2a_4^2-a_2^3a_6-a_4^3}}$$
\begin{enumerate}
\item Supponiamo per assurdo che $a_2=a_4=0$, allora la curva ellittica sar\'a del tipo 
$E:\;y^2=x^3+a_6$. In particolare si avr\'a che $x^3+a_6=x^3+\left(\sqrt[3]{a_6}\right)^3=(x+\sqrt[3]{a_6})^3$ in quanto in caratteristica $3$ vale la formula sbagliata. Avrei dunque una radice tripla in $x=\sqrt[3]{a_6}$ e
ci\'o andrebbe contro la non singolarit\'a della curva ellittica $E$.
\item Opero il seguente cambio di variabili:
$$
\sist
x_1=x-\ds{\frac{a_4}{a_2}}\\
y_1=y
\esist
\;\rightarrow
\sist
x=x_1+\ds{\frac{a_4}{a_2}}\\
y=y_1
\esist
$$
$\ds{y_1^2=\left(x_1+\frac{a_4}{a_2}\right)^3+a_2\left(x_1+\frac{a_4}{a_2}\right)^2+a_4\left(x+\frac{a_4}{a_2}\right)+a_6}$;\\
$y_1^2=x_1^3+\frac{a_4^3}{a_2^3}+a_2x_1^2+\frac{a_4^2}{a_2}+2a_4 x+a_4 x+ \frac{a_4^2}{a_2}+a_6$\\
Basta porre $a'_2=a_2$ e $a'_6=\frac{a_4^3}{a_2^3}+2\frac{a_4^2}{a_2}+a_6$ per ottenere:\\
$y_1^2=x_1^3+a'_2x^2+a'_6$.





\item Se $j(E)=j(E')\Rightarrow \ds{\frac{a_2^3}{a_6}=\frac{(a'_2)^3}{a'_6}\Rightarrow}$\\
$a_2^3=\ds{\frac{a_6}{a'_6}(a'_2)^3}$. Se si pone $\frac{a_6}{a'_6}=\mu^6\;\Rightarrow\; a_2=\mu^2a'_2$,
con $\mu\in\overline{K}^*$.\\
Dalla relazione precedente si ha anche che $a_6=\left(\frac{a_2}{a'_2}\right)^3a'_6=\mu^6a'_6$.

\item

\item Se $a_2=0\;\Rightarrow\;j(E)=0$\\
Se $a_4=0$ basta sostituire nella formula del $j-$invariante per avere $j(E)=-\ds{\frac{a_2^3}{a_6}}$.
Per il punto $(b)$ se $a_2\neq 0$ posso fare un cambiamento di variabile che mi riporti al caso $a_4=0$.
Sia $k\in \mathbb{K}$ considero la famiglia di curve ellittiche 
$E_k:\; y^2=x^3+kx^2-k^2\;\Rightarrow\; j(E)=k$.

\item Se $j(E)=0\;\Rightarrow\;a_2=0$ si ha che il cambiamento di variabile esiste per il punto \textbf{(b)}.\\
Se $j(E)\neq 0\;\Rightarrow\;a_2\neq 0$ e posso sempre supporre $a_4=0$, il cambiamento esiste per il punto 
\textbf{(c)}.  
 

\end{enumerate}




\item Sia $\alpha(x,y):=\ds{\left(\frac{p(x)}{q(x)},y\frac{s(x)}{t(x)}\right)}$ t.c. 
$gcd(p(x),q(x))=gcd(s(x),t(x))=1$;\\
$\alpha$ \'e un endomorfismo di $E:\;y^2=x^3+A x+B$.\\
\begin{enumerate}
\item Per ipotesi si ha che $\alpha(x,y)\in E$, quindi:
$$y^2\ds{\frac{s^2(x)}{t^2(x)}=\left(\frac{p(x)}{q(x)}\right)^3+A\left(\frac{p(x)}{q(x)}\right)+B}$$
Ora si fa il minimo comune multiplo e si sfrutta il fatto che anche $(x,y)\in E$:
$$(x^3+Ax+B)\ds{\frac{s^2(x)}{t^2(x)}=\frac{p^3(x)+A p(x) q^2(x)+ B q^3(x)}{q^3(x)}}$$
Basta porre $u(x)=p^3(x)+A p(x) q^2(x)+ B q^3(x)$ per ottenere l'identit\'a cercata.\\
Mi rimane da mostrare che $gcd(u(x),q(x))=1$.\\
Supponiamo per assurdo che esista $r\in K$, dove $K$ \'e il campo in cui \'e definito $E$, t.c. 
$u(r)=q(r)=0$.\\ 
Tuttavia si ha che $u(r)=p^3(r)=0\Leftrightarrow p(r)=0$ ma ci\'o contraddice l'ipotesi di coprimalit\'a dei polinomi $p(x)$ e $q(x)$ (avrebbero $r$ come radice in comune).
\item Sia $t(x_0)=0$; dal punto precedente si ricava che:
$$\ds{\frac{t^2(x_0)}{s^2(x_0)(x_0^3+Ax_0+B)}=\frac{q^3(x_0)}{u(x_0)}}$$
Affinch\'e $q(x_0)$ sia uguale a $0$ mi basta che si annulli sempre il primo membro dell'equazione. 
Ci\'o \'e sempre vero in quanto $gcd(s(x),t(x))=1$ per ipotesi e le radici di $t^2(x)$ sono tutte con molteplicit\'a almeno $2$, mentre quelle di $x^3+Ax+B$ sono semplici poich\'e $E$ curva ellittica. In particolare ci\'o vuol dire che lo zero del numeratore non potr\'a mai essere semplificato con un eventuale zero del denominatore.\\
\end{enumerate}


\item Sia $E:\;y^2=x^3+ax$, con $a\neq 0$ e $L:\;y=mx$.\\
Vado a vedere qual \'e l'ordine di intersezione tra retta e curva:
$$\sist
y^2=x^3+ax\\
y=mx
\esist\;\;\Rightarrow\;\;x^2(x+a-m^2)=0$$
Se $m\neq \pm\sqrt{a}$ si ha che $ord_{L,(0,0)}(E)=2$ in quanto $x=0$ \'e radice doppia del polinomio.\\
Se $m=\pm\sqrt{a}$ si ha che $ord_{L,(0,0)}(E)=3$ in quanto $x=0$ \'e radice tripla del polinomio.



\item 
\begin{enumerate}
\item Sia $C:\;u^2+v^2=1$ e sia $P=(-1,0)$.\\
Considero la retta nel piano $(u,v)$ passante per $P$ e avente coefficiente angolare $m$, descritta dalle cordinate parametriche
$$
\sist
u=-1+t \\
v=mt
\esist
$$
Per cercare una parametrizzazione della curva $C$ dipendente da un solo parametro considero il secondo punto di intersezione tra $C$ stessa e la retta dipendente da $m$.\\
$(t-1)^2+m^2t^2=1\;\;\;\;t^2-2t+m^2t^2=0\;\;\;\;t(t(1+m^2)-2)=0\;\Leftrightarrow$\\
$t=0\rightarrow P$,\\
$t=\ds{\frac{2}{1+m^2}}$ che descrive il secondo punto di intersezione al variare di $m$. Andando a sostituire nell'equazione parametrica della retta si ottiene la parametrizzazione di $C$ cercata:
$$u=\ds{\frac{1-m^2}{1+m^2}\;\;v=\frac{2m}{1+m^2}}$$

\item Mi basta passare a coordinate proiettive per poter scrivere un punto generico della curva al variare di $m$ come $Q=\left[\ds{\frac{1-m^2}{1+m^2}:\frac{2m}{1+m^2}:1}\right]=[1-m^2:2m:1+m^2]$.
$Q$ nella sua forma omogenea \'e equivalente a $[n^2-m^2:2mn:m^2+n^2]$.\\
Se $x$ \'e pari per simmetria dell'equazione posso considerare il punto $[2mn:n^2-m^2:m^2+n^2]$. Si \'e dunque dimostrato che se $\exists x,y,z\in\mathbb{Z}$ t.c. $x^2+y^2=z^2$ e $x$ \'e pari al variare di $m,n$ in $\mathbb{Z}$ posso descrivere la soluzione dell'equazione come:
$$x=2mn\;\;y=n^2-m^2\;\;z=m^2+n^2$$
Ora, ipotizzando che $gcd(x,y,z)=1$, si vuole dimostrare che $gcd(m,n)=1$ e che $m\not\equiv n\;(mod\;2)$.\\
Supponiamo per assurdo che $gcd(m,n)=d\neq 1$ allora $m=dm'$, $n=dn'$ e si ha che:
$$x=2d^2m'n'\;\;y=d^2(n'^2-m'^2)\;\;z=d^2(n'^2+m'^2)$$
da cui si ricava subito che $gcd(x,y,z)=d^2$ contro l'ipotesi iniziale.\\
In maniera analoga si dimostra che $m\not\equiv n\;(mod\;2)$. Supponiamo per assurdo che non sia vero, allora $2|n-m$. Tuttavia poich\'e $y=n^2-m^2=(n-m)(n+m)\;\Rightarrow\;2|y$ e quindi $2$ divide anche $z$ in quanto $z^2=x^2+y^2$. Tutto ci\'o va contro l'ipotesi di coprimalit\'a di $x,y,z$.\\
\end{enumerate}


\item Siano $p(x),q(x)$ due polinomi t.c. $gcd(p(x),q(x))=1$, si vuole dimostrare che:
$$\ds{\frac{d}{dx}\left(\frac{p(x)}{q(x)}\right)=0\;\Leftrightarrow\;p(x)\equiv q(x)\equiv 0}$$
$\ds{\frac{p'(x)q(x)-p(x)q'(x)}{q^2(x)}=0\;\Leftrightarrow\;p'(x)q(x)=q'(x)p(x)}$\\
Tuttavia ci\'o implica che $q(x)|q'(x)p(x)$. Basta applicare il lemma di Euclide sfruttando l'ipotesi $gcd(p(x),q(x))=1$ per ricavarne che $q(x)|q'(x)$. Tale configurazione risulta possibile se e solo se $q'(x)\equiv 0$, in quanto $q'(x)$ \'e un polinomio di un grado inferiore a $q(x)$. A questo punto se $q'(x)\equiv 0$ si deve avere che $p'(x)q(x) \equiv 0 \;\Leftrightarrow\; p'(x)\equiv 0$.\\ 


\item Sia $E:\; x^3+Ax+B$ una curva ellittica definita su un campo $K$ e sia $d\in K^{*}$.\\
Sia $E^d:=\; x^3+Ad^2x+Bd^3$.\\
\begin{enumerate}
\item Si vuole verificare che $j(E)=j(E^{d})$:
$$\ds{j(E^d)=1728\frac{4d^6A^3}{4d^6A^3+27B^2d^6}=1728\frac{4A^3}{4A^3+27B^2}=j(E)}$$
\item Basta fare il seguente cambiamento di variabile con coefficienti in $E[\sqrt{d}]$:
$$\sist
x\longmapsto dx\\
y\longmapsto d\sqrt{d}y
\esist$$
Si ottiene infatti la seguente curva $d^3y^2=d^3x^3+Ad^3x+Bd^3$; a questo punto baster\'a dividere entrambi i 
membri per $d^3$ (lo posso fare in quanto $K$ campo e $d\in K^{*}$, quindi ammette inverso) per ottenere $E^d$.
\item Invece la curva $E$ pu\'o essere trasformata in $dy^2=x^3+Ax+B$ con il seguente cambiamento di variabile con coefficienti in $K$:
$$\sist
x\longmapsto dx\\
y\longmapsto d^2y
\esist$$
\end{enumerate}


\item Siano $\alpha,\beta\in \mathbb{Z}$ t.c. $gcd(\alpha, \beta) = 1$, con $\alpha\equiv -1\; (mod\; 4)$
e $\beta\equiv 0\;(mod\;32)$. Sia $E$ la curva ellittica data da $y^2=x(x-\alpha)(x-\beta)$.\\
\begin{enumerate}
\item Si deve dimostrare che $\alpha\equiv 0\;(mod\;p)\;\Rightarrow\;\beta\not\equiv 0\;(mod\;p)$.\\
Se $gcd(\alpha,\beta)=1\;\Rightarrow\; \exists a,b\in\mathbb{Z}:\; a\alpha+b\beta=1$.\\
Se $\alpha\equiv 0\;(mod\;p)\;\Rightarrow\;a\alpha+b\beta\equiv b\beta\equiv 1\;{mod\;p}\;\Rightarrow
\; \beta\not\equiv 0\;(mod\;p)$.
\item Applico il cambiamento di variabile indicato:
$$
\sist
x=4x_1\\
y=8y_1+4x_1
\esist$$
$64y_1^2+64x_1y_1+16x_1^2=[4x_1(4x_1-\alpha)(4x_1-\beta)]$\\
$64y_1^2+64x_1y_1+16x_1^2=64x_1^3-16x_1^2\beta-16x_1^2\alpha+4x_1\alpha\beta$\\
Dividendo tutto per $64$ si ha che:\\
$y_1^2+x_1y_1=x_1^3-\ds{\frac{1+\alpha+\beta}{4}x_1^2+\frac{\alpha\beta}{16}x_1}$
ossia l'equazione di $E_1$ che si voleva ottenere.
\item Poich\'e $\beta\equiv 0 \;(mod\;32)$ si ha che $32|\beta$, ossia $\beta=32t$, con $t\in\mathbb{Z}$.
Inoltre $\alpha\equiv -1\;(mod\;4)$, quindi $\alpha=-1+4k$, con $k\in\mathbb{Z}$.\\
L'equazione di $E_1$ diventa $y_1^2+x_1y_1=x_1^3-\frac{4k-32t}{4}x_1^2+2(-1+4k)t x_1$.\\
Se si riduce questa equazione modulo $2$ si ha:
$$y_1^2+x_1y_1=x_1^3-ex_1^2$$
dove $k\equiv e\; (mod\;2)$.
\item Sia $r:\;y_1=\gamma x_1$ con $\gamma= costante$. Si vuole calcolare l'ordine di intersezione della retta con la curva ellittica ridotta modulo $2$ del punto precedente nell'origine.\\
$$
\sist
y_1=\gamma x_1\\
y_1^2+x_1y_1=x_1^3+ex_1^2
\esist
\;\rightarrow\;
\sist
y_1=\gamma x_1\\
\gamma^2x_1^2+\gamma x_1^2=x_1^3+ex_1^2
\esist
$$
$x_1^3+(e-\gamma^2-\gamma)x^2=0\;\Rightarrow\;x^2(x+e-\gamma^2-\gamma)=0$.\\
L'ordine d'intersezione in $(0,0)$ risulta quindi essere $3$ se $\gamma^2+\gamma=e$ e $2$ se $\gamma^2+\gamma\neq e$.

\item Se $e=0$ ho due radici distinti in $\mathbb{F}_2$ (sia $1$, sia $0$ risolvono il polinomio). 
Se $e=1$ per il teorema fondamentale dell'algebra $\exists$ due radici in $\overline{\mathbb{F}_2}$, quindi mi basta dimostrare che sono distinte facendo vedere che il massimo comune divisore fra il polinomio e la sua derivata \'e sempre $1$. In questo caso $\ds{\frac{d}{dx}(x^2+x+1)=2x+1=1}$, da cui si ricava banalmente che  
$gcd(x^2+x+1,1)=1$ e quindi che le due radici sono distinte fra loro.

 
\end{enumerate}



\end{enumerate}
\end{document}