\nopagenumbers \font\title=cmti12
\def\ve{\vfill\eject}
\def\vv{\vfill}
\def\vs{\vskip-2cm}
\def\vss{\vskip10cm}
\def\vst{\vskip13.3cm}

\def\ve{\bigskip\bigskip}
\def\vv{\bigskip\bigskip}
\def\vs{}
\def\vss{}
\def\vst{\bigskip\bigskip}

\hsize=19.5cm
\vsize=27.58cm
\hoffset=-1.6cm
\voffset=0.5cm
\parskip=-.1cm
\ \vs \hskip -6mm AL310 AA19/20\ (Teoria delle Equazioni)\hfill APPELLO B (Scritto) \hfill Roma, 13 Febbraio 2020. \hrule
\bigskip\noindent
{\title COGNOME}\  \dotfill\ {\title NOME}\ \dotfill {\title
MATRICOLA}\ \dotfill\
\smallskip  \noindent
Risolvere il massimo numero di esercizi accompagnando le risposte
con spiegazioni chiare ed essenziali. \it Inserire le risposte
negli spazi predisposti. NON SI ACCETTANO RISPOSTE SCRITTE SU
ALTRI FOGLI. Scrivere il proprio nome anche nell'ultima pagina.
\rm 1 Esercizio = 5 punti. Tempo previsto: 2 ore. Nessuna domanda
durante la prima ora e durante gli ultimi 20 minuti.
\smallskip
\hrule\smallskip
\centerline{\hskip 6pt\vbox{\tabskip=0pt \offinterlineskip
\def \trl{\noalign{\hrule}}
\halign to277pt{\strut#& \vrule#\tabskip=0.7em plus 1em& \hfil#&
\vrule#& \hfill#\hfil& \vrule#& \hfil#& \vrule#& \hfill#\hfil&
\vrule#& \hfil#& \vrule#& \hfill#\hfil& \vrule#& \hfil#& \vrule#&
\hfill#\hfil& \vrule#& \hfil#& \vrule#& \hfill#\hfil& \vrule#&
\hfil#& \vrule#& \hfill#\hfil& \vrule#& \hfil#& \vrule#& \hfil#&
\vrule#\tabskip=0pt\cr\trl && FIRMA && 1 && 2 && 3 && 4 &&
5 && 6 && 7 && 8 &&   TOT. &\cr\trl && &&   &&
&&     &&   &&   &&   &&   &&    && &\cr &&
\dotfill &&     &&   &&   &&     &&   && && && &&
&\cr\trl }}}
\medskip

\item{1.} Rispondere alle sequenti domande fornendo una giustificazione di una riga (giustificazioni
incomplete o poco chiare comportano punteggio nullo):\bigskip\bigskip\bigskip


\itemitem{a.} Quali possono essere tutti i possibili gruppi di Galois
dei polinomi di grado 3 e 4 su ${\bf Q}$ e su ${\bf F}_2$?\medskip\bigskip\bigskip

\ \dotfill\ \bigskip\bigskip\bigskip\vfil

\itemitem{b.} Scrivere una ${\bf Q}[\sqrt{-3}]$--base del campo di spezzamento del polinomio $X^3-2\in{\bf Q}[\sqrt{-3}][X]$.\medskip\bigskip\bigskip

\ \dotfill\ \bigskip\bigskip\bigskip\vfil

\itemitem{c.} \`E vero che due polinomi irriducibili in ${\bf F}_p[X]$ aventi
lo stesso grado potrebbero avere campi di spezzamento non isomorfi?\medskip\bigskip\bigskip
 
\ \dotfill\ \bigskip\bigskip\bigskip\vfil

\itemitem{d.} Elencare tutti i polinomi irriducibili (monici) di grado minore uguale a $2$ su ${\bf F}_{3}$.\medskip\bigskip\bigskip

\ \dotfill\ \bigskip\bigskip\bigskip

\itemitem{e.} Si scriva un espressione con radicali per $\cos 2\pi/24$ utilizzando la formula di duplicazione $\cos2\alpha=2\cos^2\alpha-1$.\medskip\bigskip\bigskip

\ \dotfill\ \bigskip\bigskip\bigskip

\vfil\eject

%Dimostrare che un estensione finita \`{e} necessariamente algebrica. Produrre
%un esempio di un estensione algebrica non finita.

\item{2.} Dato un gruppo finito $G$, dimostrare che esiste una estensione di campi $E/F$ opportuna tale che
Gal$(E(F)\cong G$.\hfill\break 
{\it Suggerimento:} Usare il Teorema di Cayley, il fatto che l'enunciato \`e vero
per $G=S_n$ e il Teorema di Corrispondenza.

\vv


\item{3.} Dimostrare che se $p>5$ \`e primo tale che $(p-1)/2$ \`e il prodotto di $k$ primi dispari distinti allora 
${\bf Q}[\zeta_{p}]$ ammette esattamente $2^{k+1}$ sottocampi.
\ve\ \vs

%Dopo aver verificato che \`e algebrico, calcolare
%il polinomio minimo di $\cos \pi/9$ su ${\bf Q}$.

\item{4.} Descrivere il gruppo di Galois del polinomio $(X^3-2)(X^2+X+1)\in{\bf Q}[X]$ come sottogruppo di $S_5$. \vv

\item{5.} Dimostrare che se $p\geq3$ \`e primo, allora il discriminante di $X^p-2$ \`e $(-1)^{(p-1)/2}2^{p-1}p^p$.
\hfill\break {\it Suggerimento:} Usare la formula per il discriminante che ha a che fare con la derivata prima.
\ve\ \vs

%--\item{6.} Descrivere la nozione di campo perfetto dimostrando che i campi finiti
%sono perfetti.

\item{6.} Si enunci nella completa generalit\`a il Teorema di
corrispondenza di Galois dando qualche cenno sulla dimostrazione.\vskip 6cm\bigskip\bigskip\bigskip\vv\vv


\item{7.} Quanti sono i fattori irriducibili del polinomio $(X^{80}-1)\in{\bf F}_3[X]$ e in ${\bf Q}[X]$?
\vskip 6cm\bigskip\bigskip\bigskip\vv\vv

\item{8.}  Dopo aver fornito la definizione di numero costruibile, dimostrare che tutti gli elementi del campo di spezzamento del polinomio $x^4-4\in{\bf Q}[x]$
sono costruibili.

\vv

\ \vst
\bye
