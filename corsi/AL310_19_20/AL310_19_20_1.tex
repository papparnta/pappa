\nopagenumbers \font\title=cmti12
% \def\ve{\vfill\eject}
% \def\vv{\vfill}
% \def\vs{\vskip-2cm}
% \def\vss{\vskip10cm}
% \def\vst{\vskip13.3cm}

\def\ve{\bigskip}
\def\vv{\bigskip}
\def\vs{}
\def\vss{}
\def\vst{\bigskip}

\hsize=19.5cm
\vsize=27.58cm
\hoffset=-1.6cm
\voffset=0.5cm
\parskip=-.1cm
\ \vs \hskip -6mm AL310 AA19/20\ (Teoria delle Equazioni)\hfill ESAME
DI MET\`{A} SEMESTRE \hfill Roma, 13 Novembre 2019 \hrule
\bigskip\noindent
{\title COGNOME}\  \dotfill\ {\title NOME}\ \dotfill {\title
MATRICOLA}\ \dotfill\
\smallskip  \noindent
Risolvere il massimo numero di esercizi accompagnando le risposte
con spiegazioni chiare ed essenziali. \it Inserire le risposte
negli spazi predisposti. NON SI ACCETTANO RISPOSTE SCRITTE SU
ALTRI FOGLI.
\rm 1 Esercizio = 4 punti. Tempo previsto: 2 ore. Nessuna domanda
durante la prima ora e durante gli ultimi 20 minuti.
\smallskip
\hrule\smallskip
\centerline{\hskip 6pt\vbox{\tabskip=0pt \offinterlineskip
\def \trl{\noalign{\hrule}}
\halign to277pt{\strut#& \vrule#\tabskip=0.7em plus 1em& \hfil#&
\vrule#& \hfill#\hfil& \vrule#& \hfil#& \vrule#& \hfill#\hfil&
\vrule#& \hfil#& \vrule#& \hfill#\hfil& \vrule#& \hfil#& \vrule#&
\hfill#\hfil& \vrule#& \hfil#& \vrule#& \hfill#\hfil& \vrule#&
\hfil#& \vrule#& \hfill#\hfil& \vrule#& \hfil#& \vrule#& \hfil#&
\vrule#\tabskip=0pt\cr\trl && FIRMA && 1 && 2 && 3 && 4 &&
5 && 6 && 7 && 8  &&  TOT. &\cr\trl && &&   &&
&&     &&   &&     &&   &&   &&    && &\cr &&
\dotfill &&       &&   &&   &&     &&   && && && &&
&\cr\trl }}}
\medskip

\item{1.} Rispondere alle seguenti domande fornendo una giustificazione di una riga:\bigskip\bigskip\bigskip


\itemitem{a.} E’ vero che se $E$ \`e il campo di spezzamento di un polinomio $f\in F[x]$ di grado $n$, allora l’ordine del gruppo degli automorfismi Aut$(E/F)$ \`e minore di $n!$?\medskip\bigskip\bigskip
% 
% \ \dotfill\ \bigskip\bigskip\bigskip\vfil

\itemitem{b.} E' vero che un estensione $E$ di $F$ pu\`o contenere
sia elementi algebrici che trascendenti su $F$?\medskip\bigskip\bigskip
% 
% \ \dotfill\ \bigskip\bigskip\bigskip\vfil

\itemitem{c.} Sia $q$ un numero razionale. E' vero che $\tan q\pi$, se definito, \`e algebrico?
\medskip\bigskip\bigskip
%  
% \ \dotfill\ \bigskip\bigskip\bigskip\vfil

\itemitem{d.} Fornire un esempio di estensione infinita algebrica.\medskip\bigskip\bigskip
% 
% \ \dotfill\ \bigskip\bigskip\bigskip
% 
% 
% \vfil\eject

%Dimostrare che un estensione finita \`{e} necessariamente algebrica. Produrre
%un esempio di un estensione algebrica non finita.

\item{2.} Calcolare il polinomio minimo di $1/\gamma$ e di $1/(\gamma+2)$ nel campo ${\bf Q}[\gamma],\gamma^4=4\gamma+1$.\vv


\item{3.} Sia $\Phi_n(X)\in{\bf Q}[X]$ il polinomio minimo di $e^{2\pi i/n}$, si dimostrino le seguenti propriet\`a:\smallskip
\itemitem{a.} Se $p$ \`e primo, $\Phi_p(X)=(X^p-1)/(X-1)$\smallskip
\itemitem{b.} Se $\alpha\ge1$, $\Phi_{p^\alpha}(X)=\Phi_p(X^{p^{\alpha-1}})$\smallskip
\itemitem{c.} Se $n$ \`e dispari, $\Phi_{2n}(X)=\Phi_n(-X)$


\ve\ \vs

%Dopo aver verificato che \`e algebrico, calcolare
%il polinomio minimo di $\cos \pi/9$ su ${\bf Q}$.

\item{4.} Sia $E$ us estensione di grado $3$ di ${\bf Q}$. Dimostrare che:\smallskip
\itemitem{a.} Esiste un polinomio $f(X)\in{\bf Q}[X]$ irriducibile di grado tre tale che $E\cong{\bf Q}[\beta], f(\beta)=0$\smallskip
\itemitem{b.} Dimostrare che ogni ogni elemento di ${\bf Q}[\beta]$ si pu\`o scrivere nella forma $(a+b\beta)/(c+d\beta)$ dove $a,b,c,d\in{\bf Q}$.

%\item{4.*} Dopo aver descritto tutti gli elementi di Aut(${\bf Q}(5^{1/3},\sqrt{-3})/{\bf Q})$, si determini l'ordine di ciascuno di essi.
\vv

\item{5.} Sia $K$ il campo di spezzamento del polinomio $x^4-2\in{\bf Q}[x]$.\smallskip
\itemitem{a.} Determinare $\alpha\in{\bf C}$ tale che $K={\bf Q}(\alpha)$. \smallskip
\itemitem{b.} Sia $G={\rm Aut}(K/{\bf Q})$. Determinare l'ordine di $G$ dimostrando che $G$ non \`e abeliano e descrivendone gli elementi.
\ve\ \vs

\item{6.} Dopo aver definito la nozione di campo perfetto, si forniscano esempi di campi perfetti e di campi non perfetti.\vv\vv


\item{7.} Descrivere la nozione di punti costruibili del piano. 
\vv\vv


\item{8.} Sia $K={\bf Q}[\zeta],\zeta^2+\zeta+ 1 = 0$.\smallskip
\itemitem{a.} Calcolare il polinomio minimo di $i+\sqrt5 +\sqrt3$ su $K$. \smallskip
\itemitem{b.} Determinare i ${\bf Q}(\sqrt5)$--omomorfismi di $K$.

\vv

\ \vst
 \bye
