\nopagenumbers \font\title=cmti12
\def\ve{\vfill\eject}
\def\vv{\vfill}
\def\vs{\vskip-2cm}
\def\vss{\vskip10cm}
\def\vst{\vskip13.3cm}
% 
% \def\ve{\bigskip\bigskip}
% \def\vv{\bigskip\bigskip}
% \def\vs{}
% \def\vss{}
% \def\vst{\bigskip\bigskip}

\hsize=19.5cm
\vsize=27.58cm
\hoffset=-1.6cm
\voffset=0.5cm
\parskip=-.1cm
\ \vs \hskip -6mm AL310 AA19/20\ (Teoria delle Equazioni)\hfill APPELLO C (Scritto) \hfill Roma, 16 Maggio 2020. \hrule
\bigskip\noindent
{\title COGNOME}\  \dotfill\ {\title NOME}\ \dotfill {\title
MATRICOLA}\ \dotfill\
\smallskip  \noindent
Risolvere il massimo numero di esercizi accompagnando le risposte
con spiegazioni chiare ed essenziali. \it Inserire le risposte
negli spazi predisposti. NON SI ACCETTANO RISPOSTE SCRITTE SU
ALTRI FOGLI. Scrivere il proprio nome anche nell'ultima pagina.
\rm 1 Esercizio = 5 punti. Tempo previsto: 2 ore. Nessuna domanda
durante la prima ora e durante gli ultimi 20 minuti.
\smallskip
\hrule\smallskip
\centerline{\hskip 6pt\vbox{\tabskip=0pt \offinterlineskip
\def \trl{\noalign{\hrule}}
\halign to277pt{\strut#& \vrule#\tabskip=0.7em plus 1em& \hfil#&
\vrule#& \hfill#\hfil& \vrule#& \hfil#& \vrule#& \hfill#\hfil&
\vrule#& \hfil#& \vrule#& \hfill#\hfil& \vrule#& \hfil#& \vrule#&
\hfill#\hfil& \vrule#& \hfil#& \vrule#& \hfill#\hfil& \vrule#&
\hfil#& \vrule#& \hfill#\hfil& \vrule#& \hfil#& \vrule#& \hfil#&
\vrule#\tabskip=0pt\cr\trl && FIRMA && 1 && 2 && 3 && 4 &&
5 && 6 && 7 && 8 &&   TOT. &\cr\trl && &&   &&
&&     &&   &&   &&   &&   &&    && &\cr &&
\dotfill &&     &&   &&   &&     &&   && && && &&
&\cr\trl }}}
\medskip

\item{1.} Rispondere alle sequenti domande fornendo una giustificazione di una riga (giustificazioni
incomplete o poco chiare comportano punteggio nullo):\bigskip\bigskip\bigskip



\itemitem{a.} \`E vero che esistono gruppi che non sono gruppi di Galois di estensioni di campi finiti?\medskip\bigskip\bigskip

\ \dotfill\ \bigskip\bigskip\bigskip\vfil

\itemitem{b.} Scrivere una ${\bf Q}$--base del campo di spezzamento del polinomio $(X^3-5)(X^3-7)\in{\bf Q}[X]$.\medskip\bigskip\bigskip

\ \dotfill\ \bigskip\bigskip\bigskip\vfil

\itemitem{c.} \`E vero che ogni estensione di un campo di caratteristica $0$ ammette un elemento primitivo?\medskip\bigskip\bigskip
 
\ \dotfill\ \bigskip\bigskip\bigskip\vfil

\itemitem{d.} \`E vero che se l'$n$--agono regolare \`e costruibile allora anche
l'$4n$--agono lo \`e?\medskip\bigskip\bigskip

\ \dotfill\ \bigskip\bigskip\bigskip

\itemitem{e.} Fornire un esempio di estensione algebrica e infinita e dire se ogni estensione finita \`e algebrica.\medskip\bigskip\bigskip

\ \dotfill\ \bigskip\bigskip\bigskip


\vfil\eject

%Dimostrare che un estensione finita \`{e} necessariamente algebrica. Produrre
%un esempio di un estensione algebrica non finita.

\item{2.} Dopo aver enunciato la definizione di campo di spezzamento, dimostrare 
che ogni polinomio a coefficienti in qualsiasi campo ammette un campo di spezzamento.

\vv


\item{3.} Fornire un esempio di polinomio in ${\bf Q}[X]$ il cui gruppo di Galois
\`e isomorfo a $({\bf Z}/2{\bf Z})^4$.
\ve\ \vs


\item{4.} Calcolare il gruppo di Galois del polinomio $X^3+5X+8\in{\bf F}_3[X]$.\vv

\item{5.} Fornire la definizione di sottogruppo transitivo di $S_n$ e spiegare l'utilit\`a di
tale nozione in Teoria di Galois.
\ve\ \vs

%--\item{6.} Descrivere la nozione di campo perfetto dimostrando che i campi finiti
%sono perfetti.

\item{6.} Si enunci nella completa generalit\`a il Teorema di
corrispondenza di Galois dando qualche cenno sulla dimostrazione.\vskip 6cm\bigskip\bigskip\bigskip\vv\vv


\item{7.} Quanti sono i fattori irriducibili del polinomio $(X^{124}-1)\in{\bf F}_5[X]$ e in ${\bf Q}[X]$?
\vskip 6cm\bigskip\bigskip\bigskip\vv\vv

\item{8.}  Dopo aver verificato che \`e algebrico, calcolare
il polinomio minimo di $\cos \pi/9$ su ${\bf Q}$.


\vv

\ \vst
\bye
