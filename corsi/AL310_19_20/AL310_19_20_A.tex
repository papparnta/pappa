\nopagenumbers \font\title=cmti12
\def\ve{\vfill\eject}
\def\vv{\vfill}
\def\vs{\vskip-2cm}
\def\vss{\vskip10cm}
\def\vst{\vskip13.3cm}

%\def\ve{\bigskip\bigskip}
%\def\vv{\bigskip\bigskip}
%\def\vs{}
%\def\vss{}
%\def\vst{\bigskip\bigskip}

\hsize=19.5cm
\vsize=27.58cm
\hoffset=-1.6cm
\voffset=0.5cm
\parskip=-.1cm
\ \vs \hskip -6mm AL310 AA19/20\ (Teoria delle Equazioni)\hfill APPELLO A (Scritto) \hfill Roma, 23 Gennaio 2020. \hrule
\bigskip\noindent
{\title COGNOME}\  \dotfill\ {\title NOME}\ \dotfill {\title
MATRICOLA}\ \dotfill\
\smallskip  \noindent
Risolvere il massimo numero di esercizi accompagnando le risposte
con spiegazioni chiare ed essenziali. \it Inserire le risposte
negli spazi predisposti. NON SI ACCETTANO RISPOSTE SCRITTE SU
ALTRI FOGLI. Scrivere il proprio nome anche nell'ultima pagina.
\rm 1 Esercizio = 4 punti. Tempo previsto: 2 ore. Nessuna domanda
durante la prima ora e durante gli ultimi 20 minuti.
\smallskip
\hrule\smallskip
\centerline{\hskip 6pt\vbox{\tabskip=0pt \offinterlineskip
\def \trl{\noalign{\hrule}}
\halign to277pt{\strut#& \vrule#\tabskip=0.7em plus 1em& \hfil#&
\vrule#& \hfill#\hfil& \vrule#& \hfil#& \vrule#& \hfill#\hfil&
\vrule#& \hfil#& \vrule#& \hfill#\hfil& \vrule#& \hfil#& \vrule#&
\hfill#\hfil& \vrule#& \hfil#& \vrule#& \hfill#\hfil& \vrule#&
\hfil#& \vrule#& \hfill#\hfil& \vrule#& \hfil#& \vrule#& \hfil#&
\vrule#\tabskip=0pt\cr\trl && FIRMA && 1 && 2 && 3 && 4 &&
5 && 6 && 7 && 8 &&   9 &\cr\trl && &&   &&
&&     &&   &&   &&   &&   &&    && &\cr &&
\dotfill &&     &&   &&   &&     &&   && && && &&
&\cr\trl }}}
\medskip

\item{1.} Rispondere alle sequenti domande fornendo una giustificazione di una riga (giustificazioni
incomplete o poco chiare comportano punteggio nullo):\bigskip\bigskip\bigskip


\itemitem{a.} E' vero che le estensione finite di campi finiti sono sempre estensioni abeliane?\medskip\bigskip\bigskip

\ \dotfill\ \bigskip\bigskip\bigskip\vfil

\itemitem{b.} Scrivere una ${\bf Q}$--base del campo di spezzamento del polinomio $(X^2-5)(X^2+25)\in{\bf Q}[X]$.\medskip\bigskip\bigskip

\ \dotfill\ \bigskip\bigskip\bigskip\vfil

\itemitem{c.} \`E vero che tutti i campi con 16 elementi sono 
isomorfi a ${\bf F}_2[\alpha], \alpha^4=\alpha+1$?\medskip\bigskip\bigskip
 
\ \dotfill\ \bigskip\bigskip\bigskip\vfil

\itemitem{d.} Fornire un esempio, se esiste, di estensione trascendente di ${\bf F}_{103}$.\medskip\bigskip\bigskip

\ \dotfill\ \bigskip\bigskip\bigskip

\vfil\eject

%Dimostrare che un estensione finita \`{e} necessariamente algebrica. Produrre
%un esempio di un estensione algebrica non finita.

\item{2.} Dopo aver dato la definizione di risolvente cubica di un polinomio di grado $4$, verificare che un polinomio e la sua risolvente cubica hanno lo stesso discriminante.  
\vv


\item{3.} Sia $p$ un primo e si consideri il polinomio $f(x)=x^4 + px + p$.
\itemitem{a.} Si scriva la risolvente cubica $g$ e il discriminante di $f$
\itemitem{b.} Determinare i valori di $p$ per cui $g$ \`e
irriducibile-
\itemitem{c.} Dimostrare che se $g$ \`e irriducibile, allora
il gruppo di Galois di $f$ \`e $S_4$.
\vv


%Dopo aver verificato che \`e algebrico, calcolare
%il polinomio minimo di $\cos \pi/9$ su ${\bf Q}$.

\item{4.} Costruire un campo con 16 elementi e determinare l'ordine moltiplicativo di ciascuno dei suoi elementi non nulli.  
\ve\ \vs

\item{5.} Sia $E$ il campo di spezzamento su ${\bf Q}(\zeta_5)$ di $(x^5-3)(x^5-7)\in{\bf Q}[x]$. 
\itemitem{a.} Quale \`e il grado di $E$ su ${\bf Q}(\zeta_5)$?
\itemitem{b.} Dimostrare i campi intermedi $L$ ($E\supset L\supset{\bf Q}(\zeta_5)$) sono esattamente $6$.
\itemitem{c.} Elencate i campi del punto $b.$.
\vv

%--\item{6.} Descrivere la nozione di campo perfetto dimostrando che i campi finiti
%sono perfetti.

\item{6.} Si enunci e si dimostri nella completa generalit\`a il Teorema di
corrispondenza di Galois.\vv


\item{7.} Dopo aver definito il discriminante $D_f$ di un polinomio irriducibile $f\in{\bf F}[X]$ di grado $n$, (${\bf F}$ campo di caratteristica zero), 
si dimostri che $D_f\in{\bf F}$ \`e un quadrato perfetto se e solo se il gruppo di Galois $G_f$ \`e isomorfo a un sottogruppo dell
gruppo alterno $A_n$.\ve\ \vs

\item{8.} Dopo aver richiamato la definizione di numero algebrico costruibile, determinare i valori di $n\in{\bf Z}$ tali che
$\cos n^{\circ}$ \`e costruibile ($n^{\circ}$ significa $n$ gradi).
\vv

\item{9.} 
\itemitem{a.} Quanti sono i fattori irriducibili di $x^{80}-1\in{\bf Q}[x]$ e quali sono i loro gradi?
\itemitem{b.} Quanti sono i fattori irriducibili di $x^{80}-1\in{\bf F}_3[x]$ e quali sono i loro gradi?
\ \vst

\vfill\vfill

\hrule\medskip

\noindent{\bf la risolvente cubica di $x^4+ax^2+bx+c$ \`e $x^3+2ax^2+(a^2-4c)x-b^2$}
 \bye
