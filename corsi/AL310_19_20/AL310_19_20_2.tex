\nopagenumbers \font\title=cmti12
\def\ve{\vfill\eject}
\def\vv{\vfill}
\def\vs{\vskip-2cm}
\def\vss{\vskip10cm}
\def\vst{\vskip13.3cm}

%\def\ve{\bigskip\bigskip}
%\def\vv{\bigskip\bigskip}
%\def\vs{}
%\def\vss{}
%\def\vst{\bigskip\bigskip}

\hsize=19.5cm
\vsize=27.58cm
\hoffset=-1.6cm
\voffset=0.5cm
\parskip=-.1cm
\ \vs \hskip -6mm AL310 AA19/20\ (Teoria delle Equazioni)\hfill ESAME
DI FINE SEMESTRE \hfill Roma, 14 Gennaio  2020. \hrule
\bigskip\noindent
{\title COGNOME}\  \dotfill\ {\title NOME}\ \dotfill {\title
MATRICOLA}\ \dotfill\
\smallskip  \noindent
Risolvere il massimo numero di esercizi accompagnando le risposte
con spiegazioni chiare ed essenziali. \it Inserire le risposte
negli spazi predisposti. NON SI ACCETTANO RISPOSTE SCRITTE SU
ALTRI FOGLI. Scrivere il proprio nome anche nell'ultima pagina.
\rm 1 Esercizio = 4 punti. Tempo previsto: 2 ore. Nessuna domanda
durante la prima ora e durante gli ultimi 20 minuti.
\smallskip
\hrule\smallskip
\centerline{\hskip 6pt\vbox{\tabskip=0pt \offinterlineskip
\def \trl{\noalign{\hrule}}
\halign to300pt{\strut#& \vrule#\tabskip=0.7em plus 1em& \hfil#&
\vrule#& \hfill#\hfil& \vrule#& \hfil#& \vrule#& \hfill#\hfil&
\vrule#& \hfil#& \vrule#& \hfill#\hfil& \vrule#& \hfil#& \vrule#&
\hfill#\hfil& \vrule#& \hfil#& \vrule#& \hfill#\hfil& \vrule#&
\hfil#& \vrule#& \hfill#\hfil& \vrule#& \hfil#& \vrule#& \hfil#&
\vrule#\tabskip=0pt\cr\trl && FIRMA && 1 && 2 && 3 && 4 &&
5 && 6 && 7 && 8 && 9 &&  TOT. &\cr\trl && &&   &&
&&     &&   &&   &&   &&   &&   &&    && &\cr &&
\dotfill &&     &&   &&   &&   &&     &&   && && && &&
&\cr\trl }}}
\medskip

\item{1.} Sia $f(x)=x^5-2$.
\itemitem{-a-} Determinarne il gruppo di Galois;
\itemitem{-b-} Etichettarne le radici usando i numeri da uno a 5 e descriverne esplicitamente il gruppo di
Galois come sottogruppo di $S_{5}$.

\vv\item{2.} Considerare ${\bf Q}(\zeta_{60})$.
\itemitem{-a-} Descriverne il gruppo di Galois e scriverlo come prodotto di gruppi ciclici
\itemitem{-b-} Elencarne i sottocampi

\ve\ \vs

\item{3.} Calcolare le radici di $X^3+X+1$ nel campo $({\bf F}_2[\alpha], \alpha^3=1+\alpha^2$)

\vv\item{4.} Sia $\alpha=\cos(\pi/20)$ 
\itemitem{-a-}  Dimostrare che $\alpha$ \`{e} costruibile;
\itemitem{-b-} Determinare esplicitamente una costruzione di ${\bf Q}(\alpha)$;
\itemitem{-c-} Scrivere una formula esplicita usando radicali per $\alpha$.\vv

\item{5.} Determinare un polinomio a coefficienti razionali con gruppo di Galois isomorfo a  $S_3\times S_{3}$.
\ve\ \vs

\item{6.} Fornire un esempio di un polinomio irriducibile di grado sei il cui gruppo di Galois \`e isomorfo a $S_3$.

\vv \item{7.} Si enunci nella completa generalit\`a il Teorema di
corrispondenza di Galois e se ne dimostrino le parti salienti.

\ve\ \vs


\item{8.} Spiegare il metodo per calcolare il gruppo di Galois di un polinomio irriducibile di grado $4$.\vv

\item{9.} Un polinomio irriducibile $f\in{\bf F}_p[x]$ di grado $m$ si dice {\it primitivo} se $\gamma\in{\bf F}_p[\gamma], f(\gamma)=0$ ha ordine (moltiplicativo) $p^m-1$.
\itemitem{-a-} Determinare tutti i polinomi primitivi in ${\bf F}_2[x]$ di grado minore o uguale a $4$
\itemitem{-b-} Dimostrare che il numero di polinomi primitivi ${\bf F}_p[x]$ di grado $m$ \`e pari a $\varphi(p^m-1)/m$.
\ \vst

\bye
