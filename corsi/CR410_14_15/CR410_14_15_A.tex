\nopagenumbers \font\title=cmti12
\def\ve{\vfill\eject}
\def\vv{\vfill}
\def\vs{\vskip-2cm}
\def\vss{\vskip10cm}
\def\vst{\vskip13.3cm}

% \def\ve{\bigskip\bigskip}
% \def\vv{\bigskip\bigskip}
% \def\vs{}
% \def\vss{}
% \def\vst{\bigskip\bigskip}

\hsize=19.5cm
\vsize=27.58cm
\hoffset=-1.6cm
\voffset=0.5cm
\parskip=-.1cm
\ \vs \hskip -6mm CR410 AA14/15\ (Crittografia 1)\hfill APPELLO A \hfill Roma, 3 GIUGNO 2015. \hrule
\bigskip\noindent
{\title COGNOME}\  \dotfill\ {\title NOME}\ \dotfill {\title
MATRICOLA}\ \dotfill\
\smallskip  \noindent
Risolvere il massimo numero di esercizi accompagnando le risposte
con spiegazioni chiare ed essenziali. \it Inserire le risposte
negli spazi predisposti. NON SI ACCETTANO RISPOSTE SCRITTE SU
ALTRI FOGLI. Scrivere il proprio nome anche nell'ultima pagina.
\rm 1 Esercizio = 4 punti. Tempo previsto: 2 ore. Nessuna domanda
durante la prima ora e durante gli ultimi 20 minuti.
\smallskip
\hrule\smallskip
\centerline{\hskip 6pt\vbox{\tabskip=0pt \offinterlineskip
\def \trl{\noalign{\hrule}}
\halign to320pt{\strut#& \vrule#\tabskip=0.7em plus 1.5em& \hfil#&
\vrule#& \hfill#\hfil& \vrule#& \hfil#& \vrule#& \hfill#\hfil&
\vrule#& \hfil#& \vrule#& \hfill#\hfil& \vrule#& \hfil#& \vrule#&
\hfill#\hfil& \vrule#& \hfil#& \vrule#& \hfill#\hfil& \vrule#&
\hfil#& \vrule#& \hfil#& \vrule#& \hfil#&
\vrule#\tabskip=0pt\cr\trl && FIRMA && 1 && 2 && 3 && 4 &&
5 && 6 && 7 && 8 && TOT&\cr\trl && &&   &&
&&     &&   &&   &&   &&   &&    && &\cr &&
\dotfill &&      &&   &&   &&     &&   && && && &&
&\cr\trl }}}
\medskip


%\item{1.}
%Se $n\in{\bf N}$, sia $\varphi(n)$ la funzione di Eulero. Supponiamo che sia nota
%la fattorizzazione (unica) di $n=p_1^{\alpha_1}\cdots p_s^{\alpha_s}$. Stimare il
%numero di operazioni bit necessarie per calcolare $\varphi(n)$.
%\vv
%\item{2.} Stimare in termini di $k$ il numero di operazioni bit necessarie per calcolare $\left[\sqrt{2^{k^k}\bmod 3^k}
%\right]$.
%\vv

\item{1.} Determinare il numero di elementi del campo di spezzamento di \hfill\break
$(T^{2^{45}}+30T^{2^{40}}+27T^{2^{27}})(T^{2^5}+21T^{2^4}+31)(T^{24}+20T^{15}+9T^{16}+7)(T^{3}+2T+1)\in{\bf F}_2[T]$ \vv

\item{2.} Dopo aver spiegato il funzionamento del crittosistema RSA, dimostrare che se Carlo 
          conosce il modulo RSA $n$ e il valore $\varphi(n)$, allora pu\`o agevolmente trovare 
          i fattori primi di $n$.\ve\vs

\item{3.} Descrivere in dettagli l'Algoritmo di Pholig Hallmann per il calcolo dei logaritmi discreti.\vv

\item{4.} Siano $n$ e $m$ interi tali che $n\equiv 6\bmod 20m$ e $m\equiv 7\bmod 24$. 
          Calcolare il simbolo di Jacobi $\left({m\over n}\right)$ giustificando ogno passaggio.  \ve\vs

\item{5.} Enunciare e dimostrare il criterio di caratterizzazione (criterio di Korselt) per i numeri di Carmichael.\vv

\item{6.} Si determini la probabilit\`a che un polinomio irriducibile 
          su ${\bf F}_3$ di grado minore usuale a $5$ risulti primitivo.\ve\vs

\item{7.} Dimostrare che una curva ellittica definita su un campo di caratteristica due ha al pi\`u un punto di ordine due.\vv

\item{8.} Determinare l'ordine della curve ellittica $y^2=x^3+2x+1$ su ${\bf F}_{25}$. Cosa \`e possibile affermare sulla
          struttura di $E({\bf F}_{25})$?%\ve

%\item{9.}* Siano $E_1: y^2=x^3+x+1$ e $E_2: y^2=x^3+x+4$ due curve definite su ${\bf F}_5$. Dopo aver verificato se sono ellittiche determinarne ka struttura della
 %         gruppo dei punti razionali su ${\bf F}_5$ e si ${\bf F}_{5^2}$.
 
%\ \vst
\bye
