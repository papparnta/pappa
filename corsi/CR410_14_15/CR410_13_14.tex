\documentclass[a4paper,12pt]{article}
% The preamble section.
\usepackage[cm]{fullpage}
\title{CR410 (Crittografia a chiave pubblica)}
\author{AA13/14 -- Compiti d'esame}
\date{\ }
 
\begin{document}

\maketitle

\begin{itemize}
\item  Risolvere il massimo numero di esercizi fornendo spiegazioni chiare e sintetiche.
\item NON SI ACCETTANO RISPOSTE SCRITTE SU ALTRI FOGLI.
\item 1 Esercizio = 4 punti. 
\item Tempo previsto: 2 ore. 
\item Nessuna domanda durante le prima ora e durante gli ultimi 20 minuti.
\end{itemize}\bigskip

\hrule\smallskip

\noindent ESAME DI MET\`A SEMESTRE \hfill 4 Aprile 2014 

\hrule\smallskip

\begin{enumerate}
\item Rispondere alle seguenti domande che forniscono una giustificazione di 1 riga:
\begin{enumerate}
\item Esistono campi finiti con 48 elementi?
\item E' vero che non esistono identit\`a di Bezout con coefficienti a segno discorde?
\item Fornire un esempio di campi finiti diversi con 16 elementi.
\item Scrivere tutti i polinomi primitivi in ${\bf F}_2[x]$ di grado minore uguale a $4$.
\end{enumerate}
\item Enunciare e dimostrare il Teorema di struttura dei sottocampi di ${\bf F}_{p^n}$. Lo si utilizzi per costruire
un esempio di campo finito con esattamente $6$ sottocampi. 
\item Determinare tutti le radici primitive di ${\bf F}_{5}[\tau], \tau^2=2$.
\item Spiegare il funzionamento di alcuni sistemi crittografici che basano la propria sicurezza sul problema del 
logaritmo discreto.
\item Spiegare in dettaglio if funzionamento dell'Algoritmo Pohlig--Hellman.
\item Si applichi l'algoritmo delle approssimazioni successive per calcolare la parte intera del numero binario $\sqrt{101011101}$
\item Si determini il grado del campo di spezzamento su ${\bf F}_3$ del sequente polinomio $(x^{3^{11}}+6x-x^9+30)(x^6+1)(x^9+15x-1)$
\item Calcolare il massimo comun divisore  $\gcd(273,130)$ utilizzando sia l'algoritmo binario che quello esteso di Euclide. Utilizzare l'algoritmo di Euclide anche per 
calcolare un identit\`a di Bezout.
\end{enumerate}

\hrule\smallskip

\noindent ESAME DI FINE SEMESTRE \hfill 26 Maggio 2014 

\hrule\smallskip

\begin{enumerate}
\item Rispondere alle seguenti domande con una giustificazione di 1 riga:
\begin{enumerate}
\item E' possibile calcolare i simboli di Jacobi senza fattorizzare?
\item I simboli di Jacobi hanno applicazioni in crittografia?
\item E' possibile implementare RSA con un esponente di cifratura pari? perch\`e?
\item Dare un esempio di curva ellittica $E/{\bf F}_p$ in cui $\#E({\bf F}_p)$ \`e dispari.
\end{enumerate}
\item Spiegare il funzionamento del crittosistema RSA. 
\item Definire la nozione di pseudo primo di Miller Rabin e dimostrare che $91$ \`e pseudo primo di Miller Rabin in base $10$ e in base $22$.
\item Calcolare il simbolo di Jacobi $\left({m\over n}\right)$ sapendo che $n\equiv 7\bmod 4m$ e che $m\equiv 3\bmod 28$.
\item Dopo aver definito la nozione di numero di Carmichael, si enunci e dimostri il criterio di Korselt.
\item Sia $E: y^2=x^3+Ax+B$ una curva ellittica su un campo ${\bf F}_p$ di caratteristica maggiore di $3$. 
Dimostrare che se $P=(\alpha,\beta)\in E({\bf F}_p)$ \`e un punto di ordine tre, allora $\alpha$ \`e una radice del polinomio:
$$\Psi_3(X)=3X^4+6AX^2+12BX-A^2$$
\item Sapendo che una curva ellittica $E$ su ${\bf F}_{101}$ ha un punto di ordine $50$, cosa possiamo dire su $\#E({\bf F}_{101})$?
\item Si determini $\#E({\bf F}_{5^{20}})$ quando $E: y^2=x^3+2x-3$.
\item Si dimostri la legge di reciprocit\`a quadratica.
\end{enumerate}

\hrule\smallskip

\noindent APPELLO A \hfill 3 GIUGNO 2014

\hrule\smallskip

\begin{enumerate}
\item Si descriva un algoritmo per calcolare in tempo polinomiale la parte intera di $m^{1/2}$ per ogni intero positivo $m$.
\item Supponiamo che $e=5$ sia la chiave di cifratura di un crittosistema RSA con modulo $n=53\cdot43$. Si calcoli la chiave $d$ di 
decifratura.
\item Dimostrare che in ${\bf F}_p$ l'equazione $x^m\equiv1\bmod p$ 
          ammette $\gcd(p-1,m)$ soluzioni. Quante ne ammette in ${\bf Z}/(101\cdot 103){\bf Z}$?
\item Definire il simbolo di Jacobi ed illustrare un algoritmo polinomiale per calcolarlo.
\item Spiegare il funzionamento dei protocolli crittografici incontrati nel corso.
\item Si determini la probabilit\`a che un polinomio irriducibile 
          su ${\bf F}_2$ di grado $8$ risulti primitivo.
\item Determinare tutti i generatori di ${\bf F}_{5}[\tau], \tau^2=2$ e di ciascuno determinare il polinomio minimo.
\item Determinare la struttura del gruppo dei punti razionali di una curva ellittica definita su ${\bf F}_{101}$ sapendo che
ha un punto $P$ di ordine $41$.
\item Siano $E_1: y^2=x^3+x+1$ e $E_2: y^2=x^3+x+4$ due curve definite su ${\bf F}_5$. Dopo aver verificato se sono ellittiche determinarne ka struttura della
          gruppo dei punti razionali su ${\bf F}_5$ e si ${\bf F}_{5^2}$.
\end{enumerate}

\hrule\smallskip

\noindent APPELLO B \hfill  30 GIUGNO 2014 

\hrule\smallskip

\begin{enumerate}
\item Si descriva un algoritmo per calcolare in tempo polinomiale la parte intera di $m^{1/2}$ per ogni intero positivo $m$.
\item Supponiamo che $e=5$ sia la chiave di cifratura di un crittosistema RSA con modulo $n=53\cdot43$. Si calcoli la chiave $d$ di 
decifratura.
\item Dimostrare che in ${\bf F}_p$ l'equazione $x^m\equiv1\bmod p$ ammette $\gcd(p-1,m)$ soluzioni. Quante ne ammette in ${\bf Z}/(101\cdot 103){\bf Z}$?
\item Definire il simbolo di Jacobi ed illustrare un algoritmo polinomiale per calcolarlo.
\item Spiegare il funzionamento dei protocolli crittografici incontrati nel corso.
\item Si determini la probabilit\`a che un polinomio irriducibile su ${\bf F}_2$ di grado $8$ risulti primitivo.
\item Determinare tutti i generatori di ${\bf F}_{5}[\tau], \tau^2=2$ e di ciascuno determinare il polinomio minimo.
\item Determinare la struttura del gruppo dei punti razionali di una curva ellittica definita su ${\bf F}_{101}$ sapendo che
ha un punto $P$ di ordine $41$.
\item Siano $E_1: y^2=x^3+x+1$ e $E_2: y^2=x^3+x+4$ due curve definite su ${\bf F}_5$. Dopo aver verificato se sono ellittiche determinarne ka struttura della
 gruppo dei punti razionali su ${\bf F}_5$ e si ${\bf F}_{5^2}$.
\end{enumerate}

\hrule\smallskip

\noindent APPELLO C \hfill Roma, 30 GENNAIO 2015 

\hrule\smallskip

\begin{enumerate}
\item[] Si descrivano:
\item L'algoritmo di Euclide (per l'identit\`a di Bezout) e suo il tempo di esecuzione. ;
\item Gli algoritmi per la moltiplicazione degli interi a la loro complessit\`a;
\item L'algoritmo Baby Steps Giant Steps per il calcolo dell'ordine di una curva ellittica su un campo  finito;
\item L'algorimo di Pholig--Hellman per il calcolo dei logaritmi discreti; 
\item La varie definizioni di pseudo primi e le loro principali propriet\`a.
\item Determinare ordine e struttura di $E({\bf F}_7)$ dove $E: y^2=x^3-1$.
\item Dopo aver descritto quali sono i fattori irriducibili in ${\bf F}_11[x]$ di $x^{11^6}-x$ ($p$ primo), 
si determini il numero di tali fattori che sono primitivi.
\item Dopo aver fornito la definizione di numero di Carmichael, si enuncino e dimostrino le principali propriet\`a
dei numeri di Carmichael fornendone esempi.
\item Dimostrare che su ${\bf F}_q, q$ dispari, c'\`e sempre una curva ellittica con gruppo dei punti razionali non ciclico.
\item Si descrivano i principali algoritmi di cifratura e decifratura.
\end{enumerate}

\hrule\smallskip

\noindent APPELLO X \hfill Roma, 3 SETTEMBRE 2014 

\hrule\smallskip

\begin{enumerate}
\item[] Si descrivano:
\item L'algoritmo dei quadrati successivi;
\item L'algoritmo per calcolare i simboli di Jacobi (senza fattorizzare);
\item L'algoritmo Baby Steps Giant Steps per il calcolo dei logaritmi discreti;
\item L'algorimo di Pholig--Hellman per il calcolo dei logaritmi discreti; 
\item Dopo aver descritto la nozione di algoritmo probabilistico di tipo Montecarlo, l'algoritmo di Miller--Rabin.  
\item Determinare ordine e struttura di $E({\bf F}_7)$ dove $E: y^2=x^3+3$.
\item Dopo aver descritto quali sono i fattori irriducibili in ${\bf F}_p[x]$ di $x^{p^4}-x$ ($p$ primo), nel caso in cui $p=2$, li
si elenchino tutti specificando quali tra questi sono primitivi.
\item Siano $n$ e $m$ interi tali che $m\equiv3\bmod4$, $m\equiv2\bmod n$ e $n\equiv1\bmod8$. Si calcoli il simbolo di Jacobi
$\left({(5m+n)^7\over m}\right)$.
\item Dimostrare che se ${\bf F}_q$ \`e un campo finito di caratteristica dispari,
allora esiste sempre una curva ellittica su ${\bf F}_q$ con gruppo dei punti razionali non ciclico.
\item Si descrivano i principali algoritmi di cifratura e decifratura.
\end{enumerate}
\end{document}