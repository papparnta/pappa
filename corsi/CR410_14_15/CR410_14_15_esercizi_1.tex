\documentclass[a4paper,10pt]{article}
\usepackage[utf8]{inputenc}

%opening
\title{CR410 - Esercizi (primo foglio)}
\author{AA 2014/2015}
\date{9 Marzo 2015}

\begin{document}

\maketitle

\begin{enumerate}
 \item In ciascuno dei seguenti casi calcolare l'inverso aritmetico 
 di $a$ in due mode: con l'algoritmo esteso di Euclide e con il Piccolo 
 Teorema di Fermat:
 \begin{enumerate}
  \item $p=31 \qquad a=7$;
  \item $p=101 \qquad a=90$;
  \item $p=103 \qquad a=56$.
 \end{enumerate}
\item Effettuare una simulazione di ciascuno dei tre crittosistemi seguenti con primi di tre cifre decimali.
\begin{enumerate}
\item Scambio Chiavi Diffie Hellman
\item Crittosistema Massey Omura
\item Crittosistema ElGamal
\end{enumerate}
\item Trovare tutte le radici primitive in $\textbf F_{11}^*$, $\textbf F_{13}^*$, $\textbf F_{19}^*$ e $\textbf F_{23}^*$.
\item Dimostrare che se $p=2q+1$ è primo con $q$ primo allora $\textbf F_{p}^*$ ammette $q-1$ radici primitive. E' vero
anche il contrario (cioè che se $\textbf F_{p}^*$ ammette esattamente $(p-3)/2$ radici primitive, allora $p=2q+1$ con $q$ primo)?
\item Dimostrare che se $p=2q+1$ è primo con $q$ primo e se $g\in\textbf F_{p}^*$ è tale che $g\not\equiv\pm1\bmod p$ e
$g^q\not\equiv1\bmod p$, allora $g$ è una radice primitiva modulo $p$.
\item Sia $G$ un gruppo ciclico e sia $|G|=q_1^{\alpha_1}\cdots p_s^{\alpha_s}$ la fattorizzazione unica. Dimostrare che $g\in G$
è una radice primitiva (i.e. un generatore) se e solo se $g^{|G|/q_j}\ne1$ per ogni $j=1,\ldots,s$.
\end{enumerate}

\end{document}
