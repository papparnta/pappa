\nopagenumbers \font\title=cmti12
\def\ve{\vfill\eject}
\def\vv{\vfill}
\def\vs{\vskip-2cm}
\def\vss{\vskip10cm}
\def\vst{\vskip13.3cm}

% \def\ve{\bigskip\bigskip}
% \def\vv{\bigskip\bigskip}
% \def\vs{}
% \def\vss{}
% \def\vst{\bigskip\bigskip}

\hsize=19.5cm
\vsize=27.58cm
\hoffset=-1.6cm
\voffset=0.5cm
\parskip=-.1cm
\ \vs \hskip -6mm CR410 AA13/14\ (Crittografia 1)\hfill APPELLO C \hfill Roma, 13 GENNAIO 2016 \hrule
\bigskip\noindent
{\title COGNOME}\  \dotfill\ {\title NOME}\ \dotfill {\title
MATRICOLA}\ \dotfill\
\smallskip  \noindent
Risolvere il massimo numero di esercizi accompagnando le risposte
con spiegazioni chiare ed essenziali. \it Inserire le risposte
negli spazi predisposti. NON SI ACCETTANO RISPOSTE SCRITTE SU
ALTRI FOGLI. Scrivere il proprio nome anche nell'ultima pagina.
\rm 1 Esercizio = 3 punti. Tempo previsto: 2 ore. Nessuna domanda
durante la prima ora e durante gli ultimi 20 minuti.
\smallskip
\hrule\smallskip
\centerline{\hskip 6pt\vbox{\tabskip=0pt \offinterlineskip
\def \trl{\noalign{\hrule}}
\halign to560pt{\strut#& \vrule#\tabskip=0.7em plus 2em& \hfil#&
\vrule#& \hfill#\hfil& \vrule#& \hfil#& \vrule#& \hfill#\hfil&
\vrule#& \hfil#& \vrule#& \hfill#\hfil& \vrule#& \hfil#& \vrule#&
\hfill#\hfil& \vrule#& \hfil#& \vrule#& \hfill#\hfil& \vrule#&
\hfil#& \vrule#& \hfill#\hfil& \vrule#& \hfil#& \vrule#& \hfil#&
\vrule#\tabskip=0pt\cr\trl && FIRMA && 1 && 2 && 3 && 4 &&
5 && 6 && 7 && 8 && 9 &&  10 &\cr\trl && &&   &&
&&     &&   &&   &&   &&   &&   &&    && &\cr &&
\dotfill &&     &&   &&   &&   &&     &&   && && && &&
&\cr\trl }}}
\medskip

\item{1.} Rispondere alle seguenti domande che forniscono una giustificazione di 1 riga:\bigskip\bigskip\bigskip

\itemitem{a.} Determinate due fattori propri di $1000036000099$?\medskip\bigskip\bigskip

\ \dotfill\ \bigskip\bigskip\bigskip\vfil

\itemitem{b.} \`E vero che $X^{101}-X+3$ non ha radici monulo $101$?\medskip\bigskip\bigskip

\ \dotfill\ \bigskip\bigskip\bigskip\vfil

\itemitem{c.} Quanti sono i polinomi primitivi di grado $8$ su ${\bf F}_2$?\medskip\bigskip\bigskip
 
\ \dotfill\ \bigskip\bigskip\bigskip\vfil

\itemitem{d.} Perch\`e, se $M=p\cdot q$ \`e un modulo RSA, l'esponente
di cifratura $e$ deve essere scelto in modo tale che $\gcd(e,\varphi(M))=1$?\medskip\bigskip\bigskip

\ \dotfill\ \bigskip\bigskip\bigskip\vfil\eject

\item{2.} Descrivere due algoritmi per il calcolo del massimo comun divisore di interi, determinarne la complessit\`a e sfruttarli
per calcolare con entrambi MCD$(39,91)$.\vv

\item{3.} Determinare una stima per il numero di operazioni bit necessarie a moltiplicare due matrici $n^a\times n^b$ 
i cui coefficienti sono minori di $2^{n^c}$.\vv

\item{4.} Calcolare il seguente simbolo di Jacobi senza fattorizzare
$\left({21342\over83831}\right)$.\ve\ \vs

\item{5.} Si illustri il funzionamento del metodo di fattorizzazione $\rho$ di  Pollard.\vv

\item{6.} Dopo aver descritto il crittosistema ElGamal su ${\bf F}_p$, se ne illustri il funzionamento con un esempio con $p = 17$.\vv

\item{7.} Realizzare il campo ${\bf F}_{49}$ e determinare l'ordine di tutti i suoi elementi.\ve\ \vs

\item{8.} Supponiamo che l'ordine del gruppo dei punti razionali di una curva ellittica $E$, definita su ${\bf F}_{101}$
sia $115$, quanti elementi ha il gruppo $E({\bf F}_{101^3})$.\vv

\item{9.} Dopo aver dimostrato che \`e una curva ellittica su ${\bf F}_7$, 
calcolare la struttura del gruppo dei punti razionali di $y^2 = x^3 + 2x + 3$.\ \vst\bye