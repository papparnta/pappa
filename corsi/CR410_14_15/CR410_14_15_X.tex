\nopagenumbers \font\title=cmti12
\def\ve{\vfill\eject}
\def\vv{\vfill}
\def\vs{\vskip-2cm}
\def\vss{\vskip10cm}
\def\vst{\vskip13.3cm}

% \def\ve{\bigskip\bigskip}
% \def\vv{\bigskip\bigskip}
% \def\vs{}
% \def\vss{}
% \def\vst{\bigskip\bigskip}

\hsize=19.5cm
\vsize=27.58cm
\hoffset=-1.6cm
\voffset=0.5cm
\parskip=-.1cm
\ \vs \hskip -6mm CR410 AA14/15\ (Crittografia 1)\hfill APPELLO X \hfill Roma, 24 Settembe 2015. \hrule
\bigskip\noindent
{\title COGNOME}\  \dotfill\ {\title NOME}\ \dotfill {\title
MATRICOLA}\ \dotfill\
\smallskip  \noindent
Risolvere il massimo numero di esercizi accompagnando le risposte
con spiegazioni chiare ed essenziali. \it Inserire le risposte
negli spazi predisposti. NON SI ACCETTANO RISPOSTE SCRITTE SU
ALTRI FOGLI. Scrivere il proprio nome anche nell'ultima pagina.
\rm 1 Esercizio = 4 punti. Tempo previsto: 2 ore. Nessuna domanda
durante la prima ora e durante gli ultimi 20 minuti.
\smallskip
\hrule\smallskip
\centerline{\hskip 6pt\vbox{\tabskip=0pt \offinterlineskip
\def \trl{\noalign{\hrule}}
\halign to320pt{\strut#& \vrule#\tabskip=0.7em plus 1.5em& \hfil#&
\vrule#& \hfill#\hfil& \vrule#& \hfil#& \vrule#& \hfill#\hfil&
\vrule#& \hfil#& \vrule#& \hfill#\hfil& \vrule#& \hfil#& \vrule#&
\hfill#\hfil& \vrule#& \hfil#& \vrule#& \hfill#\hfil& \vrule#&
\hfil#& \vrule#& \hfil#& \vrule#& \hfil#&
\vrule#\tabskip=0pt\cr\trl && FIRMA && 1 && 2 && 3 && 4 &&
5 && 6 && 7 && 8 && TOT&\cr\trl && &&   &&
&&     &&   &&   &&   &&   &&    && &\cr &&
\dotfill &&      &&   &&   &&     &&   && && && &&
&\cr\trl }}}
\medskip


%\item{1.}
%Se $n\in{\bf N}$, sia $\varphi(n)$ la funzione di Eulero. Supponiamo che sia nota
%la fattorizzazione (unica) di $n=p_1^{\alpha_1}\cdots p_s^{\alpha_s}$. Stimare il
%numero di operazioni bit necessarie per calcolare $\varphi(n)$.
%\vv
%\item{2.} Stimare in termini di $k$ il numero di operazioni bit necessarie per calcolare $\left[\sqrt{2^{k^k}\bmod 3^k}
%\right]$.
%\vv

\item{1.} Dimostrare che se $p$ \`e primo, allora $x^5\equiv1\bmod p$ ammette $\gcd(p-1,5)$ soluzioni.
Determinare un valore di $m$ tale che $X^5\equiv 1\bmod m$ ammette esattamente $25$ soluzioni modulo $m$.\vv

\item{2.} Dopo aver spiegato il funzionamento dei sistemi crittografici che usano i logaritmi discreti, si illustri il funzionamento 
di ElGamal utilizzando come gruppo ${\bf F}_{16}^*$.\ve\vs

\item{3.} Descrivere l'algoritmo dei quadrati successivi in un qualsiasi monoide moltiplicativo discutendone la complessit\`a.\vv

\item{4.} Calcolare il simbolo di Legendre $\left({97543\over 21345}\right)$ utilizzando le propritet\`a dei simboli di Jacobi e giustificando
ogni passaggio.\ve\vs

\item{5.} Dopo aver definito la nozione di pseudo primo forte, si illustri come utilizzarla per scrivere un test di primalit\`a 
probabilistico.\vv

\item{6.} Si determini la probabilit\`a che un polinomio irriducibile su ${\bf F}_7$ di grado $6$ risulti primitivo.\ve\vs

\item{7.} Fornite un esempio di curva ellittica definita su un campo con $27$ elementi per cui $E({\bf F}_{27})$ \`e
ciclico.\hfill \break \ \hfill {\it sugg: cercare una curva ellettica su ${\bf F}_{3}$ con un opportuno numero di elementi}.\vv

\item{8.} Sia $E: y^2=x^3+x$,  Dimostrare che se $p\equiv1\bmod4$ allora il gruppo $E({\bf F}_p)$ non \`e ciclico.
Determinare tale gruppo nel caso in cui $p=5$.%\ve

%\item{9.}* Siano $E_1: y^2=x^3+x+1$ e $E_2: y^2=x^3+x+4$ due curve definite su ${\bf F}_5$. Dopo aver verificato se sono ellittiche determinarne ka struttura della
 %         gruppo dei punti razionali su ${\bf F}_5$ e si ${\bf F}_{5^2}$.
 
%\ \vst
\bye
