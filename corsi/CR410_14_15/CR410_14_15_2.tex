\nopagenumbers \font\title=cmti12
%\def\ve{\vfill\eject}
%\def\vv{\vfill}
%\def\vs{\vskip-2cm}
%\def\vss{\vskip10cm}
%\def\vst{\vskip13.3cm}

\def\ve{\bigskip\bigskip}
\def\vv{\bigskip\bigskip}
\def\vs{}
\def\vss{}
\def\vst{\bigskip\bigskip}

\hsize=19.5cm
\vsize=27.58cm
\hoffset=-1.6cm
\voffset=0.5cm
\parskip=-.1cm
\ \vs \hskip -6mm CR410 AA14/15 (Crittografia a chiave pubblica)\hfill ESAME DI FINE SEMESTRE \hfill Roma, 27 Maggio, 2015. \hrule
\bigskip\noindent
{\title Cognome}\  \dotfill\ {\title Nome}\ \dotfill {\title
Matricola}\ \dotfill\
\smallskip  \noindent
Risolvere il massimo numero di esercizi fornendo spiegazioni chiare e sintetiche. \it Inserire le risposte negli spazi
predisposti. NON SI ACCETTANO RISPOSTE SCRITTE SU ALTRI FOGLI.
\rm 1 Esercizio = 4 punti. Tempo previsto: 2 ore. Nessuna domanda durante le prima ora e durante gli ultimi 20 minuti.
\smallskip
\hrule\smallskip
\centerline{\hskip 6pt\vbox{\tabskip=0pt \offinterlineskip
\def \trl{\noalign{\hrule}}
\halign to248pt{\strut#& \vrule#\tabskip=0.7em plus 1em& \hfil#&
\vrule#& \hfill#\hfil& \vrule#& \hfil#& \vrule#& \hfill#\hfil&
\vrule#& \hfil#& \vrule#& \hfill#\hfil& \vrule#& \hfil#& \vrule#&
\hfill#\hfil& \vrule#& \hfil#& \vrule#& \hfill#\hfil& \vrule#&
\hfil#& \vrule#& \hfill#\hfil& \vrule#& \hfil#& \vrule#& \hfil#&
\vrule#\tabskip=0pt\cr\trl && 1 && 2 && 3 && 4 &&
5 && 6 && 7 && 8  && 9 && TOT. &\cr\trl  &&   &&
&&  &&   &&   &&     &&  &&    &&  && &\cr &&       &&   &&      &&   && && && &&
 && && &\cr\trl }}}
\medskip

\item{1.} Rispondere alle seguenti domande con una giustificazione di 1 riga:\bigskip
\itemitem{a.} E' vero che se $n$ \`e dispari, i fattori irriducibili di $x^{3^{n}}-x$ hanno tutti grado dispari?\bigskip

\ \dotfill\ \vfil\bigskip\bigskip

\itemitem{b.} E' vero che non \`e possibile utilizzare le curve ellittiche su campi finiti per 
lo scambio chiavi Diffie -- Hellman?\bigskip

\ \dotfill\ \vfil\bigskip\bigskip

\itemitem{c.} Quale \`e la probabilit\`a che un polinomio irriducibile di grado 5 su ${\bf F}_5$ risulti primitivo?\bigskip
 
\ \dotfill\ \vfil\bigskip\bigskip

\itemitem{d.} E' vero che non esistono curve ellittiche su ${\bf F}_{{11}^2}$ tali che 
$E({\bf F}_{11^2})\cong {\bf Z}/7{\bf Z}\oplus{\bf Z}/42{\bf Z}$? \bigskip

\ \dotfill\ \bigskip\bigskip\bigskip

\item{2.} Dopo aver spiegato brevemente il funzionamento del crittosistema RSA, si decifri il testo cifrato ${\cal C}=25$ 
sapendo che la chiave pubblica $(7,143)$. 
\vfill\eject\ \vskip-2cm

\item{3.} Spiegare il funzionamento del test di primalit\`a di Miller -- Rabin.\vfill

\item{4.} Spiegare il funzionamento del crittosistema Goldwasser -- Micali.\vfill\eject\ \vskip-2cm


\item{5.} Dopo aver definito la nozione di numero di Carmichael ed averne richiamato le principali propriet\`a, dimostrare che
$6601$ \`e Carmichael.
\vfill

\item{6.} Sia $E: y^2+\alpha y=x^3$ una curva ellittica sul campo ${\bf F}_8={\bf F}_2[\alpha], \alpha^3=\alpha+1$. Determinare
$\#E({\bf F}_2[\alpha])$ e $\#E({\bf F}_{2^6})$. E' possibile dire nulla sulla struttura di tali gruppi?
\vfill\eject\ \vskip-2cm

\item{7.} Determinare la struttura del gruppo dei punti razionali di una curva ellittica $E$ su ${\bf F}_{53^2}$ sapendo che contiene un 
punto di ordine $392$ e
che ammette un unico punto di ordine $2$.\vfill

\item{8.} Si consideri la curva ellittica $E: y^2=x^3+x+1$ e si calcoli $\#E({\bf F}_{3^{100}})$ .\vfill

\item{9.} Dimostrare che una curva definita su un campo con caratteristica $3$ ammette al pi\`u due punti di ordine tre.\hfill\break 
%{\scriptsize
$\ {}\qquad\qquad$\hfill{\bf suggerimento:} \it Studiare l'equazione 
$2P=-P$ per l'equazione di Weierstrass $y^2=x^3+ax^2+bx+c$.
%}

\vfill \eject\bye
