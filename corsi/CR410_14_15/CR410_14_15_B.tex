\nopagenumbers \font\title=cmti12
\def\ve{\vfill\eject}
\def\vv{\vfill}
\def\vs{\vskip-2cm}
\def\vss{\vskip10cm}
\def\vst{\vskip13.3cm}

% \def\ve{\bigskip\bigskip}
% \def\vv{\bigskip\bigskip}
% \def\vs{}
% \def\vss{}
% \def\vst{\bigskip\bigskip}

\hsize=19.5cm
\vsize=27.58cm
\hoffset=-1.6cm
\voffset=0.5cm
\parskip=-.1cm
\ \vs \hskip -6mm CR410 AA14/15\ (Crittografia 1)\hfill APPELLO B \hfill Roma, 14 LUGLIO 2015. \hrule
\bigskip\noindent
{\title COGNOME}\  \dotfill\ {\title NOME}\ \dotfill {\title
MATRICOLA}\ \dotfill\
\smallskip  \noindent
Risolvere il massimo numero di esercizi accompagnando le risposte
con spiegazioni chiare ed essenziali. \it Inserire le risposte
negli spazi predisposti. NON SI ACCETTANO RISPOSTE SCRITTE SU
ALTRI FOGLI. Scrivere il proprio nome anche nell'ultima pagina.
\rm 1 Esercizio = 4 punti. Tempo previsto: 2 ore. Nessuna domanda
durante la prima ora e durante gli ultimi 20 minuti.
\smallskip
\hrule\smallskip
\centerline{\hskip 6pt\vbox{\tabskip=0pt \offinterlineskip
\def \trl{\noalign{\hrule}}
\halign to320pt{\strut#& \vrule#\tabskip=0.7em plus 1.5em& \hfil#&
\vrule#& \hfill#\hfil& \vrule#& \hfil#& \vrule#& \hfill#\hfil&
\vrule#& \hfil#& \vrule#& \hfill#\hfil& \vrule#& \hfil#& \vrule#&
\hfill#\hfil& \vrule#& \hfil#& \vrule#& \hfill#\hfil& \vrule#&
\hfil#& \vrule#& \hfil#& \vrule#& \hfil#&
\vrule#\tabskip=0pt\cr\trl && FIRMA && 1 && 2 && 3 && 4 &&
5 && 6 && 7 && 8 && TOT&\cr\trl && &&   &&
&&     &&   &&   &&   &&   &&    && &\cr &&
\dotfill &&      &&   &&   &&     &&   && && && &&
&\cr\trl }}}
\medskip


%\item{1.}
%Se $n\in{\bf N}$, sia $\varphi(n)$ la funzione di Eulero. Supponiamo che sia nota
%la fattorizzazione (unica) di $n=p_1^{\alpha_1}\cdots p_s^{\alpha_s}$. Stimare il
%numero di operazioni bit necessarie per calcolare $\varphi(n)$.
%\vv
%\item{2.} Stimare in termini di $k$ il numero di operazioni bit necessarie per calcolare $\left[\sqrt{2^{k^k}\bmod 3^k}
%\right]$.
%\vv

\item{1.} Dimostrare che ogni elemento di ${\bf F}_{p^\alpha}$ ammette esattamente una radice $p$--esima. Calcolare la radice quadrata di
$\alpha\in{\bf F}_2[\alpha], \alpha^{5}=\alpha^2+1$ assumendo che $X^5+X^2+1$ \`e irriducibile su ${\bf F}_2$.\vv

\item{2.} Dopo aver spiegato il funzionamento dei sistemi crittografici che usano i logaritmi discreti, si illustri il funzionamento 
dello scambio chiavi Diffie Hellmann utilizzando come gruppo ${\bf F}_{16}^*$.\ve\vs

\item{3.} Descrivere in dettagli l'Algoritmo Baby Steps Giant Steps per il calcolo dei logaritmi discreti.\vv

\item{4.} Siano $n$ e $m$ interi tali che $5\not| m$, $5n\equiv 3\bmod 8m$ e $m\equiv 13\bmod 60$. 
          Calcolare il simbolo di Jacobi $\left({m\over n}\right)$ giustificando ogni passaggio.  \ve\vs

\item{5.} Dato un intero dispari $m\in{\bf N}$, Dimostrare che l'insieme ${\cal B}(m)=
\{a\in({\bf Z}/m{\bf Z}^*): a^{(m-1)/2}\equiv \left({a\over m}\right)\bmod m\}$ \`e un sottogruppo di ${\bf Z}/m{\bf Z}^*$.
Determinare la cardinalit\`a di tale sottogruppo nel caso in cui $m=p$ \`e primo e nel caso in cui $m=21$.\vv

\item{6.} Determinare i polinomi minimi e gli ordini degli elementi di ${\bf F}_{9}$. Scrivere la fattorizzazione in irriducibili di 
$X^{9}-X\in{\bf F}_3[X]$ e specificare quali dei fattori risultano primitivi.\ve\vs

\item{7.} Fornite un esempio di curva ellittica definita su un campo con $25$ elementi per cui $E({\bf F}_{25})$ \`e
ciclico.\hfill \break \ \hfill {\it sugg: cercare una curva ellettica su ${\bf F}_{5}$ con un opportuno numero di elementi}.\vv

\item{8.}  Sia $E : y^2 = x^3 - 5x + 8$ e siano $P = (10, 7),Q = (3, 11) \in E({\bf F}_{101})$. Calcolare $2P$ e $P+Q$. Sapendo che 
il punto $R=(1,2)\in E({\bf F}_{101})$ ha ordine $31$, cosa possiamo dire della struttura di $E({\bf F}_{101})$?%\ve

%\item{9.}* Siano $E_1: y^2=x^3+x+1$ e $E_2: y^2=x^3+x+4$ due curve definite su ${\bf F}_5$. Dopo aver verificato se sono ellittiche determinarne ka struttura della
 %         gruppo dei punti razionali su ${\bf F}_5$ e si ${\bf F}_{5^2}$.
 
%\ \vst
\bye
