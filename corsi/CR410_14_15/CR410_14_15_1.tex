\nopagenumbers \font\title=cmti12
\def\ve{\vfill\eject}
\def\vv{\vfill}
\def\vs{\vskip-2cm}
\def\vss{\vskip10cm}
\def\vst{\vskip13.3cm}

% \def\ve{\bigskip\bigskip}
% \def\vv{\bigskip\bigskip}
% \def\vs{}
% \def\vss{}
% \def\vst{\bigskip\bigskip}

\hsize=19cm
\vsize=27.58cm
\hoffset=-1.6cm
\voffset=0.5cm
\parskip=-.1cm
\ \vs \hskip -6mm CR410 AA14/15 (Crittografia a chiave pubblica)\hfill ESAME DI MET\`A SEMESTRE \hfill Roma, 31 Marzo, 2015. \hrule
\bigskip\noindent
{\title Cognome}\  \dotfill\ {\title Nome}\ \dotfill {\title
Matricola}\ \dotfill\
\smallskip  \noindent
Risolvere il massimo numero di esercizi fornendo spiegazioni chiare e sintetiche. \it Inserire le risposte negli spazi
predisposti. NON SI ACCETTANO RISPOSTE SCRITTE SU ALTRI FOGLI.
\rm 1 Esercizio = 4 punti. Tempo previsto: 2 ore. Nessuna domanda durante le prima ora e durante gli ultimi 20 minuti.
\smallskip
\hrule\smallskip
\centerline{\hskip 6pt\vbox{\tabskip=0pt \offinterlineskip
\def \trl{\noalign{\hrule}}
\halign to327pt{
\strut#& \vrule#\tabskip=1.1em plus 2.6em& \hfil#
       & \vrule#& \hfill#\hfil
       & \vrule#& \hfil#
       & \vrule#& \hfill#\hfil
       & \vrule#& \hfil#
       & \vrule#& \hfill#\hfil
       & \vrule#& \hfil#
       & \vrule#& \hfill#\hfil
       & \vrule#& \hfil#
       & \vrule#& \hfill#\hfil
       & \vrule#&\hfil#
       & \vrule#& \hfil#
       &\vrule#\tabskip=0pt\cr
\trl && 1 && 2 && 3 && 4 && 5 && 6 && 7 && 8 && TOTALE&\cr
\trl && &&   &&     &&   &&   &&  &&   &&    && &\cr 
&&   &&   &&   &&     &&   && && && && &\cr
\trl }}}
\medskip


\item{1.} Rispondere alle seguenti domande che forniscono una giustificazione di 1 riga:\bigskip
\itemitem{a.} Lo scambio chiavi Diffie Hellmann \`e definito solo ne gruppo ciclico ${\bf F}_{p^n}^*$?\medskip\bigskip\bigskip

\ \dotfill\ \vfil

\itemitem{b.} E' vero che esistono campi finiti non isomorfi in cui i rispettivi gruppi moltiplicativi
hanno lo stesso numero di generatori?\medskip\bigskip\bigskip

\ \dotfill\ \vfil

\itemitem{c.} Se $f,g\in{\bf F}_p[x]$ hanno lo stesso grado, \`e vero che le il campo di spezzamento di $f$
contiene le radici di $g$?\medskip\bigskip\bigskip
 
\ \dotfill\ \vfil

\itemitem{d.} Scrivere tutti i polinomi irriducibili in ${\bf F}_2[x]$ di grado minore uguale a $4$.\medskip\bigskip\bigskip

\ \dotfill\ \bigskip\bigskip\bigskip

\item{2.} Dopo aver scritto le formule ricorsive per il calcolo dell'identit\`a di Bezout tra due interi, si calcoli
quella per $(1345,9875)$. In seguito si calcoli il massimo comun divisore  $(1345,9875)$ utilizzando l'algoritmo binario.\ve\vs

\item{3.} Dopo aver dimostrato che $3$ \`e una radice primitiva modulo $31$, calcolare il logaritmo discreto $\log_32\in{\bf Z}/30{\bf Z}$ 
utilizzando l'Algoritmo Baby Steps Giant Steps.\vv

\item{4.} Spiegare il funzionamento di alcuni sistemi crittografici che basano la propria sicurezza sul problema del 
logaritmo discreto.\ve\vs

\item{5.} Determinare tutti gli interi $X$ nell'intervallo $[-200,10]$ tali che $\cases{X\equiv 2\bmod 4\cr X\equiv 4\bmod 5\cr 3X\equiv 4\bmod 7.}$ \vv

\item{6.} Fornire un esempio esplicito di campo finito con $32$ elementi e tra i suoi elementi si determini una radice primitiva.\ve \vs

\item{7.} Determinare il grado su ${\bf F}_{13}$ del campo di spezzamento del polinomio
$$(T^{13^8}-27T^{13^5}+26T^{13^{4}})(T^2+13T+27)(T^3+14)(T^{13^8}+25T^{13})\in{\bf F}_{13}[T].$$\vv

\item{8.} Dopo aver spiegato brevemente l'algoritmo dei quadrati successivi, calcolare $\alpha^{1047}\in{\bf F}_7[\alpha], \alpha^3=\alpha-2$.
 
\ \vst\bye
