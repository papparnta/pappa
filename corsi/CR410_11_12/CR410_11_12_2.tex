\nopagenumbers \font\title=cmti12
\def\ve{\vfill\eject}
\def\vv{\vfill}
\def\vs{\vskip-2cm}
\def\vss{\vskip10cm}
\def\vst{\vskip13.3cm}

%\def\ve{\bigskip\bigskip}
%\def\vv{\bigskip\bigskip}
%\def\vs{}
%\def\vss{}
%\def\vst{\bigskip\bigskip}

\hsize=19.5cm
\vsize=27.58cm
\hoffset=-1.6cm
\voffset=0.5cm
\parskip=-.1cm
\ \vs \hskip -6mm CR410 AA11/2 (Crittografia a chiave pubblica)\hfill ESAME DI FINE SEMESTRE \hfill Roma, 28 Maggio, 2012. \hrule
\bigskip\noindent
{\title Cognome}\  \dotfill\ {\title Nome}\ \dotfill {\title
Matricola}\ \dotfill\
\smallskip  \noindent
Risolvere il massimo numero di esercizi fornendo spiegazioni chiare e sintetiche. \ it Inserire le risposte negli spazi
predisposti. NON SI ACCETTANO RISPOSTE SCRITTE SU ALTRI FOGLI.
\rm 1 Eesrcizio = 4 punti. Tempo previsto: 2 ore. Nessuna domanda durante le prima ora e durante gli ultimi 20 minuti.
\smallskip
\hrule\smallskip
\centerline{\hskip 6pt\vbox{\tabskip=0pt \offinterlineskip
\def \trl{\noalign{\hrule}}
\halign to247.5pt{\strut#& \vrule#\tabskip=0.7em plus 1em& \hfil#&
\vrule#& \hfill#\hfil& \vrule#& \hfil#& \vrule#& \hfill#\hfil&
\vrule#& \hfil#& \vrule#& \hfill#\hfil& \vrule#& \hfil#& \vrule#&
\hfill#\hfil& \vrule#& \hfil#& \vrule#& \hfill#\hfil& \vrule#&
\hfil#& \vrule#& \hfill#\hfil& \vrule#& \hfil#& \vrule#& \hfil#&
\vrule#\tabskip=0pt\cr\trl && 1 && 2 && 3 && 4 &&
5 && 6 && 7 && 8  && 9 && TOT. &\cr\trl  &&   &&
&&     &&   &&     &&   &&   &&    &&  && &\cr &&       &&   &&   &&     &&   && && && &&
 && &\cr\trl }}}
\medskip

\item{1.} Rispondere alle seguenti domande che forniscono una giustificazione di 1 riga:\bigskip\bigskip\bigskip


\itemitem{a.} E' vero che tutte le curve ellittiche sono non singolari?\medskip\bigskip\bigskip

\ \dotfill\ \bigskip\bigskip\bigskip\vfil

\itemitem{b.} Fornire un esempio di una curva ellittica su un campo finito con gruppo dei
punti razionali non ciclico.\medskip\bigskip\bigskip

\ \dotfill\ \bigskip\bigskip\bigskip\vfil

\itemitem{c.} Determinare le radici primitive (i.e. generatori) in ${\bf F}_2[\alpha]$ dove $\alpha^4=1+\alpha$.\medskip\bigskip\bigskip
 
\ \dotfill\ \bigskip\bigskip\bigskip\vfil

\itemitem{d.} E' vero che in ${\bf F}_q[X]$ esistono polinomi irriducibili di ogni grado?\medskip\bigskip\bigskip

\ \dotfill\ \bigskip\bigskip\bigskip


\vfil\eject


\item{2.} Dopo aver definito la nozione di polinomio primitivo, calcolare la probabilit\`a che un polinomio irriducibile di grado $8$ su ${\bf F}_7$ sia 
primitivo.\vv


\item{3.} Dimostrare che un polinomio monico, riducibile e senza fattori quadratici di grado $5$ in ${\bf F}_q[X]$ \`e un
fattore di $X^{q^{12}}-X$. \vv



\item{4.} Spiegare il funzionamento del Crittosistema ElGamal fornendo un esempio esplicito su un campo con $13$ elementi.
\ve\ \vs

\item{5.} Dopo averne spiegato il funzionamento, implementare uno scambio chiavi Diffie--Hellmann in un campo finito
con $32$ elementi.
\vv

\item{6.} Spiegare la rilevanza del metodo Baby-Steps-Giant-Steps nella teoria delle curve ellittiche su campi finiti.
\vv


\item{7.} Sia $E: y^2=x^3-x$. Determinare la struttura del gruppo $E({\bf F}_7)$ .
\ve\ \vs

\item{8.} Supponiamo ${\bf F}_4={\bf F}_2[\xi], \xi^2=1+\xi$. 
Determinare il numero di punti su un campo con $2^{100}$ elementi della curva ellittica su ${\bf F}_4$ 
$$E: y^2+y=x^3+\xi$$
\vv

%Prima calcoliamo $\#E({\bf F}_4)$. Se ${\bf F}_4=\{0,1,\xi,1+\xi\}$ allora per $x\in{\bf F}_4$, il valore di $x^3+\xi$ \`e 
%$\xi, 1+\xi, 1+\xi, 1+\xi$ rispettivamente. Al variare di $y\in{\bf F}_4$, il valore di $y^2+y$ \`e 
%$0, 0, 1, 1$ rispettivamente. Pertanto l'unico punto in $E({\bf F}_4)$ \`e $\infty$ e $\#E({\bf F}_4)=4+1-4=1$ . 
%
%Per calcolare $\#E({\bf F}_{4^3})$ usiamo le formule ricorsive $s_0=2, s_1=4, s_{n+1}=4s_{n}-4s_{n-1}$. Otteniamo che
%$s_2= 16-8=8$ e $s_3= 8\cdot 4 -4 *4 =16$. Pertanto  $\#E({\bf F}_{4^3})=4^3+1-16=48.$

\item{9.} Scrivere e dimostrare le formule per la duplicazione di un punto (finito) su un curva ellittica in un campo
finito con caratteristica maggiore di $3$.

\ \vst
 \bye
