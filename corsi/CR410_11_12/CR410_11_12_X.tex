\nopagenumbers \font\title=cmti12
\def\ve{\vfill\eject}
\def\vv{\vfill}
\def\vs{\vskip-2cm}
\def\vss{\vskip10cm}
\def\vst{\vskip13.3cm}

%\def\ve{\bigskip\bigskip}
%\def\vv{\bigskip\bigskip}
%\def\vs{}
%\def\vss{}
%\def\vst{\bigskip\bigskip}

\hsize=19.5cm
\vsize=27.58cm
\hoffset=-1.6cm
\voffset=0.5cm
\parskip=-.1cm
\ \vs \hskip -6mm CR410 AA11/2 (Crittografia a chiave pubblica)\hfill APPELLO X \hfill Roma, 14 Settembre 2012. \hrule
\bigskip\noindent
{\title Cognome}\  \dotfill\ {\title Nome}\ \dotfill {\title
Matricola}\ \dotfill\
\smallskip  \noindent
Risolvere il massimo numero di esercizi fornendo spiegazioni chiare e sintetiche. \ it Inserire le risposte negli spazi
predisposti. NON SI ACCETTANO RISPOSTE SCRITTE SU ALTRI FOGLI.
\rm 1 Eesrcizio = 4 punti. Tempo previsto: 2 ore. Nessuna domanda durante le prima ora e durante gli ultimi 20 minuti.
\smallskip
\hrule\smallskip
\centerline{\hskip 6pt\vbox{\tabskip=0pt \offinterlineskip
\def \trl{\noalign{\hrule}}
\halign to247.5pt{\strut#& \vrule#\tabskip=0.7em plus 1em& \hfil#&
\vrule#& \hfill#\hfil& \vrule#& \hfil#& \vrule#& \hfill#\hfil&
\vrule#& \hfil#& \vrule#& \hfill#\hfil& \vrule#& \hfil#& \vrule#&
\hfill#\hfil& \vrule#& \hfil#& \vrule#& \hfill#\hfil& \vrule#&
\hfil#& \vrule#& \hfill#\hfil& \vrule#& \hfil#& \vrule#& \hfil#&
\vrule#\tabskip=0pt\cr\trl && 1 && 2 && 3 && 4 &&
5 && 6 && 7 && 8  && 9 && TOT. &\cr\trl  &&   &&
&&     &&   &&     &&   &&   &&    &&  && &\cr &&       &&   &&   &&     &&   && && && &&
 && &\cr\trl }}}
\medskip

\item{1.} Rispondere alle seguenti domande che forniscono una giustificazione di 1 riga:\bigskip\bigskip\bigskip

\itemitem{a.} E' vero che se $E$ \`e una curva ellittica definita su ${\bf F}_{3^n}$, allora
non ha mai un equazione della forma $y^2=x^3+ax+b$?\medskip\bigskip\bigskip

\ \dotfill\ \bigskip\bigskip\bigskip\vfil

\itemitem{b.} E' vero che se tutti i fattori primi di $n-1$ sono pi\`u piccoli di $\log n$, allora \`e
possibile determinare un fattore non banale di $n$ in modo rapido? come?\medskip\bigskip\bigskip

\ \dotfill\ \bigskip\bigskip\bigskip\vfil

\itemitem{c.} E' vero che se $p>3$, il polinomio $X^2+1\in{\bf F}_p$ non \`e mai primitivo ma qualche volta \`e
irriducibile?\medskip\bigskip\bigskip
 
\ \dotfill\ \bigskip\bigskip\bigskip\vfil

\itemitem{d.} E' vero che esistono modi per moltplicare interi con complessit\`a 
inferiore a quella quadratica?\medskip\bigskip\bigskip

\ \dotfill\ \bigskip\bigskip\bigskip\vfil\eject

\item{2} Se $n\in{\bf N}$, sia $\sigma(n)$ la somma dei divisori di $n$. Supponiamo che sia nota
la fattorizzazione (unica) di $n=p_1^{\alpha_1}\cdots p_s^{\alpha_s}$. Calcolare il
numero di operazioni bit necessarie per calcolare $\sigma(n)$. (\it Suggerimento: Usare il
fatto che $\sigma$ \`{e} una funzione moltiplicativa e calcolare una formula per $\sigma(p^\alpha)$ \rm).\vv

\item{3.} Siano $m,n$ interi tali che $m\equiv3\bmod4$,
che $m\equiv2\bmod n$ e che $n\equiv1\bmod8$. Si calcoli il
seguente simbolo di Jacobi: $\left({(5m+n)^3 \over m}\right)$. \vv

\item{4.}  Illustrare l'algoritmo dei quadrati successivi in un gruppo analizzandone la complessit\`{a}. Considerare
la curva ellittica $E: y^2=x^3-x.$ Illustrare l'algoritmo appena descritto calcolando $[5](1,0)$ dove $(1,0)\in E({\bf F}_{13})$. \ve\ \vs

\item{5.}  Si dia la definizione di pseudo primo forte in base $2$ e si mostri che
se $n=2^\alpha+1$ \`{e} pseudo primo forte in base $2$, allora
$2^{2^\beta}\equiv -1\bmod n$ per qualche $\beta<\alpha$.\vv

\item{6.} Fissare una radice primitiva di ${\bf F}_{3^3}$ ed
 utilizzarla per simulare un scambio chiavi alla Diffie--Hellmann.\vv

\item{7.} Dopo aver definito la nozione di polinomio primitivo su un campo finito, si calcoli la probabilit\`{a} che un polinomio
irriducibile $f$ di grado $8$ su ${\bf F}_{7}$ risulti primitivo?.\ve\ \vs

\item{8.}  Fattorizzare
$f(x)=(x^{12}+3x^{4}+1)(x^2+x+2)(x^{10}+x^2+1)$ su ${\bf
  F}_2$ e determinare il numero di elementi del campo di spezzamento di $f$.\vv

\item{9.}  Dopo aver verificato che si tratta di una curva ellittica, determinare (giustificando la risposta)
l'ordine e la struttura del gruppo dei punti razionali della curva ellittica su
${\bf F}_7$
$$y^2=x^3-x+5.$$.\ \vst\bye
