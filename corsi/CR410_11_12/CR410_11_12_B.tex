\nopagenumbers \font\title=cmti12
\def\ve{\vfill\eject}
\def\vv{\vfill}
\def\vs{\vskip-2cm}
\def\vss{\vskip10cm}
\def\vst{\vskip13.3cm}

%\def\ve{\bigskip\bigskip}
%\def\vv{\bigskip\bigskip}
%\def\vs{}
%\def\vss{}
%\def\vst{\bigskip\bigskip}

\hsize=19.5cm
\vsize=27.58cm
\hoffset=-1.6cm
\voffset=0.5cm
\parskip=-.1cm
\ \vs \hskip -6mm CR410 AA11/2 (Crittografia a chiave pubblica)\hfill APPELLO B \hfill Roma, 26 Giugno, 2012. \hrule
\bigskip\noindent
{\title Cognome}\  \dotfill\ {\title Nome}\ \dotfill {\title
Matricola}\ \dotfill\
\smallskip  \noindent
Risolvere il massimo numero di esercizi fornendo spiegazioni chiare e sintetiche. \ it Inserire le risposte negli spazi
predisposti. NON SI ACCETTANO RISPOSTE SCRITTE SU ALTRI FOGLI.
\rm 1 Eesrcizio = 4 punti. Tempo previsto: 2 ore. Nessuna domanda durante le prima ora e durante gli ultimi 20 minuti.
\smallskip
\hrule\smallskip
\centerline{\hskip 6pt\vbox{\tabskip=0pt \offinterlineskip
\def \trl{\noalign{\hrule}}
\halign to247.5pt{\strut#& \vrule#\tabskip=0.7em plus 1em& \hfil#&
\vrule#& \hfill#\hfil& \vrule#& \hfil#& \vrule#& \hfill#\hfil&
\vrule#& \hfil#& \vrule#& \hfill#\hfil& \vrule#& \hfil#& \vrule#&
\hfill#\hfil& \vrule#& \hfil#& \vrule#& \hfill#\hfil& \vrule#&
\hfil#& \vrule#& \hfill#\hfil& \vrule#& \hfil#& \vrule#& \hfil#&
\vrule#\tabskip=0pt\cr\trl && 1 && 2 && 3 && 4 &&
5 && 6 && 7 && 8  && 9 && TOT. &\cr\trl  &&   &&
&&     &&   &&     &&   &&   &&    &&  && &\cr &&       &&   &&   &&     &&   && && && &&
 && &\cr\trl }}}
\medskip

\item{1.} Rispondere alle seguenti domande che forniscono una giustificazione di 1 riga:\bigskip\bigskip\bigskip

\itemitem{a.} E' vero che se $E$ \`e una curva ellittica definita su ${\bf F}_{3}$, allora
si pu\`o agevolmente calcolare $E({\bf F}_{3^{100}})$?\medskip\bigskip\bigskip

\ \dotfill\ \bigskip\bigskip\bigskip\vfil

\itemitem{b.} E' vero che se $p-1$ ha soltanto fattori piccoli allora i logaritmi discreti in ${\bf F}_p$ si
calcolano efficientemente?\medskip\bigskip\bigskip

\ \dotfill\ \bigskip\bigskip\bigskip\vfil

\itemitem{c.} Quanti sono i polinomi primitivi di grado minore di $5$ in ${\bf F}_2[X]$?\medskip\bigskip\bigskip
 
\ \dotfill\ \bigskip\bigskip\bigskip\vfil

\itemitem{d.} E' vero che in ${\bf F}_7[X]$ due polinomi di grado $n$ si moltiplicano in $O(n^3)$ operazioni bit?\medskip\bigskip\bigskip

\ \dotfill\ \bigskip\bigskip\bigskip\vfil\eject

\item{2} Dopo aver descritto e dimostrato l'algoritmo per determinare i coefficienti di Bezout di due interi, lo si applichi per
calcolarli nel caso in cui i due interi sono $130$ e $78$.\vv

\item{3.} Dopo aver definito il simbolo di Legendre dimostrare che il numero di elementi in ${\bf F}_p^*$ che
hanno simbolo di Legendre pari a 1 \`e $(p-1)/2$. \vv

\item{4.} Spiegare in tutti i dettagli il funzionamento del crittosistema RSA e in particolare spiegare le accortezze
necessarie per scegliere le chiavi\ve\ \vs

\item{5.} Dato un intero dispari e composto $m$ si dimostri che l'insieme delle basi euleriane in $U({\bf Z}/m{\bf Z})$ \`e
un sottogruppo mentre quello delle basi forti non lo \`e.\vv

\item{6.} Spiegare il funzionamento del crittosistema Massey--Omura e lo si illustri mediante un esempio esplicito.\vv

\item{7.} Enunciare e dimostrare il teorema di classificazione dei sottocampi di ${\bf F}_{p^m}$.\ve\ \vs

\item{8.} Supponiamo ${\bf F}_8={\bf F}_2[\xi], \xi^3=1+\xi$. 
Determinare il numero di punti di $E({\bf F}_8)$ dove 
$$E: y^2+\xi y=x^3+\xi$$\vv

\item{9.} Supponiamo che $E$ sia un curva ellittica definita su ${\bf F}_{25}$, che $P\in E({\bf F}_{25})$ sia un punto di
ordine $7$ e che $E$ abbia almeno due punti di ordine $2$. Calcolare $\#E({\bf F}_{25})$.\ \vst\bye
