\input programma.sty 
\def\abbrcorso{CR410}
\def\titolocorso{Crittografia I}
\def\sottotitolo{Crittografia a chiave pubblica}
\def\docente{Prof. Francesco Pappalardi} \def\crediti{7}
\def\semestre{II} 
\def\esoneri{1} 
\def\scrittofinale{1}
\def\oralefinale{1} 
\def\altreprove{0}

\Intestazione

\titoloparagr{Argomenti di Teoria dei numeri elementare.}
Il concetto di operazione bit tipo somma o sottrazione. Stima del
numero di operazioni bit (tempo macchina) per eseguire le
operazioni fondamentali. Algoritmi che convergono in tempo
esponenziale o polinomiale. Divisibilit\`a. Algoritmo di Euclide
(identit\'a di Bezout) e suo tempo di esecuzione. Congruenze.
Teorema cinese dei resti. L'algoritmo dei quadrati successivi.

\titoloparagr{RSA.}% L'algoritmo di Adleman, Shamir e Rivest.
Formulazione dell'algoritmo e sua analisi  del suo tempo di esecuzione.
Esempi concreti non realistici. Distribuzione di Numeri primi. Il
Teorema di Chebicev. Costruzione di numeri primi (grandi): Simboli
di Legendre e simboli di Jacobi. Legge di reciprocit\`{a}
quadratica generale (senza dimostrazione) -- algoritmo polinomiale
per il calcolo del simbolo di Jacobi. Numeri di Carmichael.
Pseudo--primi, pseudo--primi di Eulero e pseudo--primi forti.
Algoritmi Montecarlo e Las--Vegas. Il test di Solovay--Strassen e
quello di Miller--Rabin. Teorema di Pocklington e certificazione
di primalit\`{a}. Fattorizzazione alla Fermat. Metodo $\rho$ di Pollard
di Fattorizzazione.

\titoloparagr{Campi finiti.} Fatti fondamentali di teoria dei
campi. Teorema dell'elemento primitivo in un campo finito.
Esistenza e unicit\`a dei campi finiti (campi di spezzamento).
Esempi. Polinomi irriducibili e primitivi. Enumerazione dei
polinomi irriducibili e primitivi. Aritmetica in tempo polinomiale
sui campi finiti. Test deterministici di irriducibilit\`{a} in
campi finiti.

\titoloparagr{Logaritmi discreti.} Il problema del logaritmo
discreto in un gruppo ciclico astratto. Metodo di Diffie Hellman
per lo scambio delle chiavi. Metodo di Massey Omura per la
trasmissione dei messaggi. Il crittositema di ElGamal. Firme
digitali. Esempi. Algoritmi per il calcolo dei
logaritmi discreti nei campi finiti: L' algoritmo di Shanks
Baby Steps Giant Steps, l'algoritmo Pohlig - Hellman.

\titoloparagr{Curve Ellittiche.}
Crittosistemi Ellittici: Generalit\`{a} sulle curve ellittiche senza
dimostrazioni,
definizione di addizione sui punti razionali di una curva
ellittica. Teorema di Struttura del
gruppo dei punti razionali di una curva ellittica su un campo campo
finito (solo enunciato), Teorema di Hasse (solo enunciato). Sottogruppi
di torsione e loro struttura. Classificazione 
delle curve ellittiche in caratteristica 2. Baby Step Giant Step per determinare
l'ordine del Gruppo. Curve ellittiche definite su sottocampi, polinomio caratteristico.


%\titoloparagr{Sistema Pari GP.} Programmazione nel Sistema Pari
%per calcolare con numeri a precisione arbitraria, Implementazione degli algoritmi
%analizzati durante il corso.\bigskip

\testi

\bib
\autore{Neal Koblitz}
\titolo{A Course in Number Theory and Cryptography}
\editore{Springer}
\annopub{1994}
\altro{Graduate Texts in Mathematics, No 114}
\endbib

\bib
\autore{Douglas R. Stinson}
\titolo{Cryptography: Theory and Practice}
\editore{CRC Pr}
\annopub{1995}
\endbib

\bib
\autore{Rudolf Lidl, Harald Niederreiter}
\titolo{Finite Fields}
\editore{Cambridge University Press}
\annopub{1997}
\endbib

\bib
\autore{C. Batut, K. Belabas, D. Bernardi, H. Cohen, M. Olivier}
\titolo{Pari--GP (2.014)}\par \hskip 6mm
\editore{{http://pari.home.ml.org}}
\annopub{1998}
\endbib

\bib
\autore{Richard Crandall, Carl Pomerance}
\titolo{Prime numbers, a computational Perspective}
\editore{Sprin\-ger}
\annopub{2001}
\endbib

\bib
\autore{F. Pappalardi} \titolo{NOTE DI CRITTOGRAFIA A CHIAVE
PUBBLICA } \editore{Fascicolo 1. Prerequisiti di Matematica}
\annopub{2003}
\endbib


%\altritesti
%
%\bib
%\autore{Jan C. A. van der Lubbe}
%\titolo{Basic Mathods of Cryptography}
%\editore{Cambridge University Press}
%\annopub{1988}
%\endbib
%
%\bib
%\autore{Neal Koblitz}
%\titolo{Algebraic Aspects of Cryptography}
%\editore{Springer}
%\annopub{1998}
%\altro{Algorithms and Computation in Mathematics, Vol 3}
%\endbib

%
%\bib
%\autore{Bruce Schneier}
%\titolo{Applied Cryptography}
%\editore{John Wiley \& Sons, Inc.}
%\annopub{1996}
%\altro{seconda edizione}
%\endbib

\esami
\bye
