\nopagenumbers \font\title=cmti12
\def\ve{\vfill\eject}
\def\vv{\vfill}
\def\vs{\vskip-2cm}
\def\vss{\vskip10cm}
\def\vst{\vskip13.3cm}

%\def\ve{\bigskip\bigskip}
%\def\vv{\bigskip\bigskip}
%\def\vs{}
%\def\vss{}
%\def\vst{\bigskip\bigskip}

\hsize=19.5cm
\vsize=27.58cm
\hoffset=-1.6cm
\voffset=0.5cm
\parskip=-.1cm
\ \vs \hskip -6mm CR410 AA11/2 (Crittografia a chiave pubblica)\hfill APPELLO A \hfill Roma, 5 Giugno, 2012. \hrule
\bigskip\noindent
{\title Cognome}\  \dotfill\ {\title Nome}\ \dotfill {\title
Matricola}\ \dotfill\
\smallskip  \noindent
Risolvere il massimo numero di esercizi fornendo spiegazioni chiare e sintetiche. \ it Inserire le risposte negli spazi
predisposti. NON SI ACCETTANO RISPOSTE SCRITTE SU ALTRI FOGLI.
\rm 1 Eesrcizio = 4 punti. Tempo previsto: 2 ore. Nessuna domanda durante le prima ora e durante gli ultimi 20 minuti.
\smallskip
\hrule\smallskip
\centerline{\hskip 6pt\vbox{\tabskip=0pt \offinterlineskip
\def \trl{\noalign{\hrule}}
\halign to247.5pt{\strut#& \vrule#\tabskip=0.7em plus 1em& \hfil#&
\vrule#& \hfill#\hfil& \vrule#& \hfil#& \vrule#& \hfill#\hfil&
\vrule#& \hfil#& \vrule#& \hfill#\hfil& \vrule#& \hfil#& \vrule#&
\hfill#\hfil& \vrule#& \hfil#& \vrule#& \hfill#\hfil& \vrule#&
\hfil#& \vrule#& \hfill#\hfil& \vrule#& \hfil#& \vrule#& \hfil#&
\vrule#\tabskip=0pt\cr\trl && 1 && 2 && 3 && 4 &&
5 && 6 && 7 && 8  && 9 && TOT. &\cr\trl  &&   &&
&&     &&   &&     &&   &&   &&    &&  && &\cr &&       &&   &&   &&     &&   && && && &&
 && &\cr\trl }}}
\medskip

\item{1.} Rispondere alle seguenti domande che forniscono una giustificazione di 1 riga:\bigskip\bigskip\bigskip

\itemitem{a.} E' vero che l'algoritmo Pohlig--Hellman si applica a qualsiasi gruppo finito ciclico?\medskip\bigskip\bigskip

\ \dotfill\ \bigskip\bigskip\bigskip\vfil

\itemitem{b.} Quale \`e la probabilit\`a che dati $(x_1,\ldots,x_{100})\in({\bf Z}/500{\bf Z}^{100})$ 
ci siano $i\neq j$ tali che $x_i= x_j$?\medskip\bigskip\bigskip

\ \dotfill\ \bigskip\bigskip\bigskip\vfil

\itemitem{c.} Che differenza c'\`e tra polinomi irriducibili e polinomi primitivi?\medskip\bigskip\bigskip
 
\ \dotfill\ \bigskip\bigskip\bigskip\vfil

\itemitem{d.} E' vero che in ${\bf F}_p[X]$ due polinomi di grado $30$ si moltiplicano in $O(\log^2p)$ operazioni bit?\medskip\bigskip\bigskip

\ \dotfill\ \bigskip\bigskip\bigskip\vfil\eject

\item{2.} Descrivere due algoritmi per il calcolo del massimo comun divisore di interi, determinarne la complessit\`a e sfruttarli
per calcolare con entrambi MCD$(75,42)$.\vv

\item{3.} Dopo aver definito i simboli di Jacobi e di Legendre dimostrare che se $p$ e $q$ sono numeri primi tali che $q\equiv5\bmod 4p$ e 
$p\equiv2\bmod5$, allora il simbolo di Legendre $\left({p\over q}\right)=-1$. \vv

\item{4.} Mostrare che se $n$ \`e un modulo RSA di cui si conosce il valore di $\varphi(n)$, allora \`e possibile
determinare efficientemente i fattori primi di $n$. Come si pu\`o utilizzare questa informazione
per decifrare messaggi cifrati con RSA?\ve\ \vs

\item{5.} Descritto l'algoritmo di Miller Rabin per verificare la primalit\`a di un intero, stimarne la probabilit\`a
d'errore quando \`e applicato con 10 iterazioni su interi con 1000 cifre decimali.\vv

\item{6.} Descrivere brevemente tutti gli algoritmi crittografici che basano la propria sicurezza sul problema del logaritmo
discreto.\vv

\item{7.} Determinare tutti i sottocampi di ${\bf F}_{7^{50}}$ che contengono un sottocampo con $49$ elementi.\ve\ \vs

\item{8.} Supponiamo ${\bf F}_4={\bf F}_2[\xi], \xi^2=1+\xi$. 
Determinare il numero di punti su un campo con $2^{12}$ elementi della curva ellittica su ${\bf F}_4$ 
$$E: y^2+\xi y=x^3+\xi$$\vv

\item{9.} Descrivere il gruppo $E({\bf F}_5)$ dove $E$ \`e la curva ellittica definita da $y^2=x^3-x$.\ \vst\bye
