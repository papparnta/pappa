\nopagenumbers
\font\title=cmti12
\vfuzz=280pt
\input psfig.sty
\def\ve{\vfill\eject}
\def\vv{\vfill}
\def\vs{\vskip-3cm}
\def\vss{\vskip10cm}
\def\vst{\vskip14cm}

%\def\ve{\bigskip\bigskip}
%\def\vv{\bigskip\bigskip}
%\def\vs{}
%\def\vss{}
%\def\vst{\bigskip\bigskip}

\hsize=19.5cm
\vsize=27.58cm
\hoffset=-1.6cm
\voffset=0.9cm
\parskip=-.1cm

\ \vs \hskip -6mm AA06/07\ (Analisi Uno per Fisica)\hfill Prova
Scritta \hfill Roma, 29 Gennaio 2007. \hrule
\bigskip\noindent
{\title COGNOME}\  \dotfill\  {\title NOME}\ \dotfill {\title
MATRICOLA}\ \dotfill\
\smallskip  \noindent
Risolvere il massimo numero di esercizi accompagnando le risposte
con spiegazioni chiare ed essenziali. \it Inserire le risposte negli
spazi predisposti. NON SI ACCETTANO RISPOSTE SCRITTE SU ALTRI FOGLI.
Scrivere il proprio nome anche nell'ultima pagina. \rm 1 Esercizio =
4 punti. Tempo previsto: 2 ore. Nessuna domanda durante la prima ora
e durante gli ultimi 20 minuti.
\smallskip
\hrule
\medskip

\item{1.} Successione definita per ricorrenza

\vv \item{2.} Numeri complessi

%\bigskip\bigskip \bf SOLUZIONE: \it \rm


\ve\ \vs \item{3.}  Si disegni il grafico della seguente funzione:
$$
f(x)=.
$$

\vv

\centerline{\psfig{figure=assi_22_1_07.eps,width=16cm}}\ \ \vskip 2cm
\eject

\ \vskip -3cm\item{4.} Serie


\vv \item{5.} Topologia

\vv

\item{6.} Calcolare il seguente integrale: $\displaystyle{\int_0^1 {x^3\over (x-2)^3}}.$

%\bigskip\bigskip \bf SOLUZIONE: \it
%$$-{4\over (x-2)^2} - {12\over x-2} + x +6\log(x-2)$$\rm
\ve\ \vs  %%%%%%%%%%

\item{7.} Teorico sulla continuit\`{a} .%%%%
\vskip 8cm

%\bigskip\bigskip \bf SOLUZIONE: \it Vedi libro di testo
%
%\rm

\item{8.} Polinomio di Taylor:
$\displaystyle{\lim_{x\rightarrow0}{e^x\cos x -
\left(1+x\right)\over x^3}}.$\vskip 8cm

\item{9.} Massimo e minimo limite.

\vskip 9cm

\centerline{\hskip 6pt\vbox{\tabskip=0pt \offinterlineskip
\def \trl{\noalign{\hrule}}
\halign to364pt{\strut#& \vrule#\tabskip=0.7em plus 1em&
\hfil#& \vrule#& \hfill#\hfil& \vrule#&
\hfil#& \vrule#& \hfill#\hfil& \vrule#&
\hfil#& \vrule#& \hfill#\hfil& \vrule#&
\hfil#& \vrule#& \hfill#\hfil& \vrule#&
\hfil#& \vrule#& \hfill#\hfil& \vrule#&
\hfil#& \vrule#& \hfill#\hfil& \vrule#&
\hfil#& \vrule#& \hfil#& \vrule#\tabskip=0pt\cr\trl
&& NOME E COGNOME && 1 && 2 && 3 && 4 && 5 && 6 && 7 && 8 && 9 &&  TOT. &\cr\trl
&& &&   &&   &&     &&   &&   &&   &&   &&   &&    &&    &\cr
&& \dotfill &&     &&   &&   &&   &&   &&   &&    &&  &&    && &\cr\trl
}}}
 \bye
