\nopagenumbers
\font\title=cmti12
\vfuzz=280pt
\input psfig.sty

\hsize=19cm
%\vsize=27.58cm
\hoffset=-1.6cm
%\voffset=-0.9cm
%\parskip=-.1cm

AA06/07\ (Analisi Uno per Fisica)\hfill Secondo
esonero (Soluzioni) \hfill Roma, 22 Gennaio 2007. \hrule
\bigskip

\item{1.} Definire la nozione di derivabilit\`{a} per una funzione di una variabile reale e si dimostri che il prodotto di funzioni
continue in un punto \`{e} continua nel punto.
\medskip \bf SOLUZIONE: \it Vedi libro di testo\rm\bigskip

\item{2.} Si studi la continuit\`{a} e la derivabilit\`{a} della
seguente funzione definita a tratti $f(x)=\displaystyle{\cases{e^x \cos x & se $x\leq0$\cr
1+x & se $x>0$.}}$
\medskip \bf SOLUZIONE: \it Sia continua che derivabile in $\bf R$.\rm\bigskip

\item{3.}  Si disegni il grafico della seguente funzione:
$f(x)=\sqrt{x + 1}(x^2 - x - 4).$
\medskip \bf SOLUZIONE: \it\break
\centerline{\psfig{figure=grafico_22_1_07.eps,width=10cm}}\rm

\item{4.} Dopo averne determinato il dominio, si calcoli la
derivata della seguente funzione: $f(x)=(\cos x)^{\log x}.$

\medskip \bf SOLUZIONE: \it
Dominio: $\displaystyle{(0,\pi/2)\cup \bigcup_{k\in{\bf N}}\left(-\pi/2+2k\pi,\pi/2+2k\pi\right)}.$
Derivata
$(\cos x)^{\log x}\left({\log(\cos x)\over x} - \log(x)\tan(x)\right)$\rm

\item{5.} Calcolare l'integrale su $[0,50]$ della funzione
$f(x)={x\over2}\chi_{[0,2)}(x)-5x^2\chi_{(-1,1]}(x).$
Dire inoltre se $f(x)$ \`{e} una funzione a scalini.\bigskip

\item{6.} Calcolare il seguente integrale: $\displaystyle{\int_0^1 {x^3\over (x-2)^3}}.$
\medskip \bf SOLUZIONE: \it
$\displaystyle{-{4\over (x-2)^2} - {12\over x-2} + x +6\log(x-2)}$\rm\bigskip

\item{7.} Dimostrare la formula per la derivata delle funzioni arcotangente e arcoseno.
\medskip \bf SOLUZIONE: \it Vedi libro di testo\rm\bigskip

\item{8.} Calcolare il seguente limite utilizzando il polinomio di Taylor:
$\displaystyle{\lim_{x\rightarrow0}{e^x\cos x -
\left(1+x\right)\over x^3}}.$
\medskip \bf SOLUZIONE: \it $-1/3$\rm \bigskip

\item{9.} Utilizzare il criterio di Newton per il calcolo delle radici di una funzione derivabile per definire una successione di numeri
razionali che converge a $-\sqrt{5}$.
\medskip \bf SOLUZIONE: \it
$\cases{x_0=-1,\cr x_{n+1}={1\over2}\left(x_n +{5\over x_n}\right)}$
\bye
