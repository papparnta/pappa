\nopagenumbers \font\title=cmti12
\def\ve{\vfill\eject}
\def\vv{\vfill}
\def\vs{\vskip-2cm}
\def\vss{\vskip10cm}
\def\vst{\vskip13.3cm}

%\def\ve{\bigskip\bigskip}
%\def\vv{\bigskip\bigskip}
%\def\vs{}
%\def\vss{}
%\def\vst{\bigskip\bigskip}

\hsize=19.5cm
\vsize=27.58cm
\hoffset=-1.6cm
\voffset=0.5cm
\parskip=-.1cm
\ \vs \hskip -6mm CR1 AA07/08\ (Crittografia a Chiave Pubblica)\hfill APPELLO A \hfill Roma, 9 Giugno 2008. \hrule
\bigskip\noindent
{\title COGNOME}\  \dotfill\ {\title NOME}\ \dotfill {\title
MATRICOLA}\ \dotfill\
\smallskip  \noindent
Risolvere il massimo numero di esercizi accompagnando le risposte
con spiegazioni chiare ed essenziali. \it Inserire le risposte
negli spazi predisposti. NON SI ACCETTANO RISPOSTE SCRITTE SU
ALTRI FOGLI. Scrivere il proprio nome anche nell'ultima pagina.
\rm 1 Esercizio = 4 punti. Tempo previsto: 2 ore. Nessuna domanda
durante la prima ora e durante gli ultimi 20 minuti.
\smallskip
\hrule\smallskip
\centerline{\hskip 6pt\vbox{\tabskip=0pt \offinterlineskip
\def \trl{\noalign{\hrule}}
\halign to300pt{\strut#& \vrule#\tabskip=0.7em plus 1em& \hfil#&
\vrule#& \hfill#\hfil& \vrule#& \hfil#& \vrule#& \hfill#\hfil&
\vrule#& \hfil#& \vrule#& \hfill#\hfil& \vrule#& \hfil#& \vrule#&
\hfill#\hfil& \vrule#& \hfil#& \vrule#& \hfill#\hfil& \vrule#&
\hfil#& \vrule#& \hfill#\hfil& \vrule#& \hfil#& \vrule#& \hfil#&
\vrule#\tabskip=0pt\cr\trl && FIRMA && 1 && 2 && 3 && 4 &&
5 && 6 && 7 && 8 && 9 &&  TOT. &\cr\trl && &&   &&
&&     &&   &&   &&   &&   &&   &&    && &\cr &&
\dotfill &&     &&   &&   &&   &&     &&   && && && &&
&\cr\trl }}}
\medskip


\item{1.} Dimostrare che \`e possibile calcolare dei coefficienti di Bezout per due interi positivi in tempo
polinomiale e calcolarli nel caso di $65$ e $23$.

\vv

\item{2.} Descrivere l'algoritmo di divisione di Karatsuba analizzandone la complessit\`{a}.

\ve\vs

\item{3.} Calcolare il seguente simbolo di Jacobi senza fattorizzare: $\left({4531\over2317}\right)$.
\vv


\item{4.} Dimostrare che i numeri di Carmichael non hanno fattori quadratici.

 \vv

\item{5.} Spiegare la nozione di algoritmo probabilistico di tipo Montecarlo e illustrare l'algoritmo di Miller Rabin per
la primalit\`a analizzandone
la complessit\`a.
\ve\vs

\item{6.} Calcolare il reticolo dei sottocampi di ${\bf F}_{3^{12}}$ spiegando i risultati teorici utilizzati.
\vv

\item{7.} Dopo averne spiegato il funzionamento, dare un esempio di implementazione del crittosistema Massey--Omura su un campo con $31$ elementi.
\ve \vs

\item{8.} Descrivere in generale la nozione di Firma Digitale e spiegare in particolare il funzionamento dell'algoritmo DSS.
\vv

\item{9.} Dopo aver verificato che si tratta di una curva ellittica, determinare (giustificando la risposta) l'ordine e la struttura del gruppo
dei punti razionali della curva ellittica su ${\bf F}_5$
$$y^2 = x^3 + 3.$$
\ \vst

 \bye
