\nopagenumbers \font\title=cmti12
\def\ve{\vfill\eject}
\def\vv{\vfill}
\def\vs{\vskip-2cm}
\def\vss{\vskip10cm}
\def\vst{\vskip13.3cm}

%\def\ve{\bigskip\bigskip}
%\def\vv{\bigskip\bigskip}
%\def\vs{}
%\def\vss{}
%\def\vst{\bigskip\bigskip}

\hsize=19.5cm
\vsize=27.58cm
\hoffset=-1.6cm
\voffset=0.5cm
\parskip=-.1cm
\ \vs \hskip -6mm CR1 AA07/08\ (Crittografia a Chiave Pubblica)\hfill APPELLO C \hfill Roma, 13 Febbraio 2009. \hrule
\bigskip\noindent
{\title COGNOME}\  \dotfill\ {\title NOME}\ \dotfill {\title
MATRICOLA}\ \dotfill\
\smallskip  \noindent
Risolvere il massimo numero di esercizi accompagnando le risposte
con spiegazioni chiare ed essenziali. \it Inserire le risposte
negli spazi predisposti. NON SI ACCETTANO RISPOSTE SCRITTE SU
ALTRI FOGLI. Scrivere il proprio nome anche nell'ultima pagina.
\rm 1 Esercizio = 4 punti. Tempo previsto: 2 ore. Nessuna domanda
durante la prima ora e durante gli ultimi 20 minuti.
\smallskip
\hrule\smallskip
\centerline{\hskip 6pt\vbox{\tabskip=0pt \offinterlineskip
\def \trl{\noalign{\hrule}}
\halign to300pt{\strut#& \vrule#\tabskip=0.7em plus 1em& \hfil#&
\vrule#& \hfill#\hfil& \vrule#& \hfil#& \vrule#& \hfill#\hfil&
\vrule#& \hfil#& \vrule#& \hfill#\hfil& \vrule#& \hfil#& \vrule#&
\hfill#\hfil& \vrule#& \hfil#& \vrule#& \hfill#\hfil& \vrule#&
\hfil#& \vrule#& \hfill#\hfil& \vrule#& \hfil#& \vrule#& \hfil#&
\vrule#\tabskip=0pt\cr\trl && FIRMA && 1 && 2 && 3 && 4 &&
5 && 6 && 7 && 8 && 9 &&  TOT. &\cr\trl && &&   &&
&&     &&   &&   &&   &&   &&   &&    && &\cr &&
\dotfill &&     &&   &&   &&   &&     &&   && && && &&
&\cr\trl }}}
\medskip


\item{-1-} Determinare una stima per il numero di operazioni bit necessarie
a moltiplicare due interi minori di $m^2$.

\vv

\item{-2-} Calcolare il seguente simbolo di Jacobi $\left({3258\atop9839}\right)$.

\ve\vs

\item{-3-} 
Dato il numero binario n = $(10001011101)_2$ , calcolare $[\sqrt{n}]$ 
usando l'algoritmo delle approssimazioni successive (Non passare
a base $10$ e non usare la calcolatrice!)
\vv

\item{-4-} Carlo scopre il valore di $\varphi(n)$ dove $n$ \`e il modulo RSA 
che Alice e Bernardo stanno usando per comunicare. Come pu\`o usare
questa informazione per decifrare i messaggi?
\vv

\item{-5-} Spiegare il funzionamento del test di primalit\`{a} di Solovay--Strassen introducendo le
nozioni necessarie.

\ve\vs

\item{-6-}  Dopo aver calcolato il numero di polinomi irriducibili di grado 6 su ${\bf F}_{2}$, si dimostri
che il polinomio $X^6+X+1$ \`e irriducibile e si verifichi se \`e primitivo.
\vv

\item{-7-}  Si fornisca un esempio del funzionamento del crittosistema ElGamal su un campo finito con
$49$ elementi \hfill {\it suggerimento: Usare il polinomio $x^2+1$}

\ve \vs

\item{-8-} 
Enunciare l'algoritmo $\rho$ di Pollard spiegandone il funzionamento.
%%%%%%%%%%%
\vv

\item{-9-} Dopo aver verificato che si tratta di una curva ellittica, determinare (giustificando la risposta) l'ordine e la struttura del gruppo
dei punti razionali della curva ellittica su ${\bf F}_7$
$$y^2 = x^3 - x + 1.$$
\ \vst

 \bye
