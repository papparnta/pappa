\nopagenumbers \font\title=cmti12
\def\ve{\vfill\eject}
\def\vv{\vfill}
\def\vs{\vskip-2cm}
\def\vss{\vskip10cm}
\def\vst{\vskip13.3cm}

%\def\ve{\bigskip\bigskip}
%\def\vv{\bigskip\bigskip}
%\def\vs{}
%\def\vss{}
%\def\vst{\bigskip\bigskip}

\hsize=19.5cm
\vsize=27.58cm
\hoffset=-1.6cm
\voffset=0.5cm
\parskip=-.1cm
\ \vs \hskip -6mm CR1 AA07/08\ (Crittografia a Chiave Pubblica)\hfill APPELLO B \hfill Roma, 9 Luglio 2008. \hrule
\bigskip\noindent
{\title COGNOME}\  \dotfill\ {\title NOME}\ \dotfill {\title
MATRICOLA}\ \dotfill\
\smallskip  \noindent
Risolvere il massimo numero di esercizi accompagnando le risposte
con spiegazioni chiare ed essenziali. \it Inserire le risposte
negli spazi predisposti. NON SI ACCETTANO RISPOSTE SCRITTE SU
ALTRI FOGLI. Scrivere il proprio nome anche nell'ultima pagina.
\rm 1 Esercizio = 4 punti. Tempo previsto: 2 ore. Nessuna domanda
durante la prima ora e durante gli ultimi 20 minuti.
\smallskip
\hrule\smallskip
\centerline{\hskip 6pt\vbox{\tabskip=0pt \offinterlineskip
\def \trl{\noalign{\hrule}}
\halign to300pt{\strut#& \vrule#\tabskip=0.7em plus 1em& \hfil#&
\vrule#& \hfill#\hfil& \vrule#& \hfil#& \vrule#& \hfill#\hfil&
\vrule#& \hfil#& \vrule#& \hfill#\hfil& \vrule#& \hfil#& \vrule#&
\hfill#\hfil& \vrule#& \hfil#& \vrule#& \hfill#\hfil& \vrule#&
\hfil#& \vrule#& \hfill#\hfil& \vrule#& \hfil#& \vrule#& \hfil#&
\vrule#\tabskip=0pt\cr\trl && FIRMA && 1 && 2 && 3 && 4 &&
5 && 6 && 7 && 8 && 9 &&  TOT. &\cr\trl && &&   &&
&&     &&   &&   &&   &&   &&   &&    && &\cr &&
\dotfill &&     &&   &&   &&   &&     &&   && && && &&
&\cr\trl }}}
\medskip


\item{-1-} 
Dato il numero binario $n = (100110100101)_2$, calcolare $[\sqrt{n}]$ usando l�algoritmo delle approssimazioni successive (Non passare
a base 10 e non usare la calcolatrice!)

\vv

\item{-2-} Descrivere l'algoritmo dei quadrati successivi in un monoide moltiplicativo e descriverne la complessit\`{a}.

\ve\vs

\item{-3-} Definire il simbolo di Jacobi, elencarne le propriet\`{a} e descrivere in algoritmo per calcolarlo in tempo
polinomiale.
\vv


\item{-4-} Descrivere il gruppo delle basi euleriane modulo un intero dispari $m$. Dopo aver verificato che \`{e}
un gruppo, dimostrare che se $m$ \`{e} composto, allora il gruppo \`{e} un sottogruppo proprio del gruppo degli 
invertibili modulo $m$.
 \vv

\item{-5-} Descrivere il funzionamento di un sistema crittografico in cui RSA viene usato contemporaneamente per cifrare e per 
firmare (in modo digitale) il testo.
\ve\vs

\item{-6-} Determinare il numero di elementi del campo di spezzamento su ${\bf F_2}$ del polinomio
$(x^8+x^4+1)(x^{128}+x)(x^5+x+1).$
\vv

\item{-7-} Dopo averne spiegato il funzionamento, dare un esempio di implementazione del sistema di scambio delle chiavi 
 alla Diffie-Hellman su un campo con $16$ elementi.
\ve \vs

\item{-8-} In ${\bf F}_{31}$ calcolare i seguenti logaritmi discreti: $\log_3(11)$ e $\log_{17}(11)$.
\vv

\item{-9-} Dopo aver verificato che si tratta di una curva ellittica, determinare (giustificando la risposta) l'ordine e la struttura del gruppo
dei punti razionali della curva ellittica su ${\bf F}_7$
$$y^2 = x^3 + x + 3.$$
\ \vst

 \bye
