\nopagenumbers \font\title=cmti12
\def\ve{\vfill\eject}
\def\vv{\vfill}
\def\vs{\vskip-2cm}
\def\vss{\vskip10cm}
\def\vst{\vskip13.3cm}

%\def\ve{\bigskip\bigskip}
%\def\vv{\bigskip\bigskip}
%\def\vs{}
%\def\vss{}
%\def\vst{\bigskip\bigskip}

\hsize=19.5cm
\vsize=27.58cm
\hoffset=-1.6cm
\voffset=0.5cm
\parskip=-.1cm
\ \vs \hskip -6mm CR1 AA07/08\ (Crittografia a Chiave Pubblica)\hfill APPELLO X \hfill Roma, 16 Settembre 2008. \hrule
\bigskip\noindent
{\title COGNOME}\  \dotfill\ {\title NOME}\ \dotfill {\title
MATRICOLA}\ \dotfill\
\smallskip  \noindent
Risolvere il massimo numero di esercizi accompagnando le risposte
con spiegazioni chiare ed essenziali. \it Inserire le risposte
negli spazi predisposti. NON SI ACCETTANO RISPOSTE SCRITTE SU
ALTRI FOGLI. Scrivere il proprio nome anche nell'ultima pagina.
\rm 1 Esercizio = 4 punti. Tempo previsto: 2 ore. Nessuna domanda
durante la prima ora e durante gli ultimi 20 minuti.
\smallskip
\hrule\smallskip
\centerline{\hskip 6pt\vbox{\tabskip=0pt \offinterlineskip
\def \trl{\noalign{\hrule}}
\halign to300pt{\strut#& \vrule#\tabskip=0.7em plus 1em& \hfil#&
\vrule#& \hfill#\hfil& \vrule#& \hfil#& \vrule#& \hfill#\hfil&
\vrule#& \hfil#& \vrule#& \hfill#\hfil& \vrule#& \hfil#& \vrule#&
\hfill#\hfil& \vrule#& \hfil#& \vrule#& \hfill#\hfil& \vrule#&
\hfil#& \vrule#& \hfill#\hfil& \vrule#& \hfil#& \vrule#& \hfil#&
\vrule#\tabskip=0pt\cr\trl && FIRMA && 1 && 2 && 3 && 4 &&
5 && 6 && 7 && 8 && 9 &&  TOT. &\cr\trl && &&   &&
&&     &&   &&   &&   &&   &&   &&    && &\cr &&
\dotfill &&     &&   &&   &&   &&     &&   && && && &&
&\cr\trl }}}
\medskip


\item{-1-} Determinare una stima per il numero di operazioni bit necessarie
a moltiplicare due matrici $n\times n$ i cui coefficienti sono minori di $e^n$.

\vv

\item{-2-} Descrivere un algoritmo per calcolare i massimo comun divisore di due
interi e descriverne la complessit\`{a}.

\ve\vs

\item{-3-} Calcolare il numero di soluzioni della seguente equazione
$$x^5+x^2+x+1\bmod 2\cdot 3\cdot 5.$$
\vv


\item{-4-}  Mostrare che se $n$ \`{e} un modulo RSA, ed \`{e} noto il valore di $\varphi(n)$ allora \`{e} possibile fattorizzare
$n$ in tempo polinomiale.
 \vv

\item{-5-} Spiegare il funzionamento del test di primalit\`{a} di Miller Rabin introducendo le
nozioni necessarie.

\ve\vs

\item{-6-}  Calcolare la probabilit\`{a} che un polinomio irriducibile di grado 11 su ${\bf F}_{7}$
risulti primitivo. Dare un esempio di polinomio irriducibile e non primitivo.

\vv

\item{-7-}  Simulare uno scambio delle chiavi alla Diffie--Hellmann in un campo finito con
$49$ elementi \hfill {\it suggerimento: Usare il polinomio $x^2+1$}

\ve \vs

\item{-8-} 
Enunciare l'algoritmo Pohlig--Hellmann per calcolare i
logaritmi discreti in un gruppo ciclico finito dimostrandone la validit\`{a}.
%%%%%%%%%%%
\vv

\item{-9-} Dopo aver verificato che si tratta di una curva ellittica, determinare (giustificando la risposta) l'ordine e la struttura del gruppo
dei punti razionali della curva ellittica su ${\bf F}_7$
$$y^2 = x^3 + x + 1.$$
\ \vst

 \bye
