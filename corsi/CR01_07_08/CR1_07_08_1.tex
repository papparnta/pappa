\nopagenumbers \font\title=cmti12
\def\ve{\vfill\eject}
\def\vv{\vfill}
\def\vs{\vskip-2cm}
\def\vss{\vskip10cm}
\def\vst{\vskip13.3cm}

%\def\ve{\bigskip\bigskip}
%\def\vv{\bigskip\bigskip}
%\def\vs{}
%\def\vss{}
%\def\vst{\bigskip\bigskip}

\hsize=19.5cm
\vsize=27.58cm
\hoffset=-1.6cm
\voffset=0.5cm
\parskip=-.1cm
\ \vs \hskip -6mm TE1 AA07/08\ (Teoria delle Equazioni)\hfill ESAME DI MET\`{A} SEMESTRE \hfill Roma, 9 Aprile 2008. \hrule
\bigskip\noindent
{\title COGNOME}\  \dotfill\ {\title NOME}\ \dotfill {\title
MATRICOLA}\ \dotfill\
\smallskip  \noindent
Risolvere il massimo numero di esercizi accompagnando le risposte
con spiegazioni chiare ed essenziali. \it Inserire le risposte
negli spazi predisposti. NON SI ACCETTANO RISPOSTE SCRITTE SU
ALTRI FOGLI. Scrivere il proprio nome anche nell'ultima pagina.
\rm 1 Esercizio = 4 punti. Tempo previsto: 2 ore. Nessuna domanda
durante la prima ora e durante gli ultimi 20 minuti.
\smallskip
\hrule\smallskip
\centerline{\hskip 6pt\vbox{\tabskip=0pt \offinterlineskip
\def \trl{\noalign{\hrule}}
\halign to300pt{\strut#& \vrule#\tabskip=0.7em plus 1em& \hfil#&
\vrule#& \hfill#\hfil& \vrule#& \hfil#& \vrule#& \hfill#\hfil&
\vrule#& \hfil#& \vrule#& \hfill#\hfil& \vrule#& \hfil#& \vrule#&
\hfill#\hfil& \vrule#& \hfil#& \vrule#& \hfill#\hfil& \vrule#&
\hfil#& \vrule#& \hfill#\hfil& \vrule#& \hfil#& \vrule#& \hfil#&
\vrule#\tabskip=0pt\cr\trl && FIRMA && 1 && 2 && 3 && 4 &&
5 && 6 && 7 && 8 && 9 &&  TOT. &\cr\trl && &&   &&
&&     &&   &&   &&   &&   &&   &&    && &\cr &&
\dotfill &&     &&   &&   &&   &&     &&   && && && &&
&\cr\trl }}}
\medskip


%\item{1.}
%Se $n\in{\bf N}$, sia $\varphi(n)$ la funzione di Eulero. Supponiamo che sia nota
%la fattorizzazione (unica) di $n=p_1^{\alpha_1}\cdots p_s^{\alpha_s}$. Stimare il
%numero di operazioni bit necessarie per calcolare $\varphi(n)$.
%\vv
%\item{2.} Stimare in termini di $k$ il numero di operazioni bit necessarie per calcolare $\left[\sqrt{2^{k^k}\bmod 3^k}
%\right]$.
%\vv

\item{1.} Dato il numero binario
$n=(101101110101)_2$, calcolare $[\sqrt{n}]$ usando l'algoritmo
delle approssimazioni successive (Non passare a base 10 e  non
usare la calcolatrice!) \vv

\item{2.} Illustrare il metodo di moltiplicazione di due numeri binari $n$ e $m$ mostrando che per eseguirla
sono necessarie tante somme quanti sono gli $1$ nell'espansione di $m$. \ve\vs

\item{3.} Trovare un valore di $n$ intero per cui la congruenza $X^3\equiv 1\bmod n$ ha esattamente $9$ soluzioni modulo $n$?
\vv


\item{4.} Illustrare l'algoritmo dei quadrati successivi in un gruppo analizzandone la complessit\`{a}. $b=2^5+2^3+2^2+1$,
quante moltiplicazioni in ${\bf Z}/m{\bf Z}$ sono necessarie per calcolare $2^b\bmod m$?

 \vv


\item{5.} Spiegare il funzionamento dell'algoritmo binario per il calcolo del massimo comun divisore e se ne analizzi la
complessit\`{a}.
\ve\vs

\item{6.} Calcolare la probabilit\`{a} d'errore di un iterazione del test di primalit\`{a} Solovay Strassen
\vv

\item{7.} Dimostrare che 6601 \`{e} un numero di Carmichael.
\ve \vs

\item{8.} Calcololare il seguente simbolo di Jacobi senza fattorizzare: $\left({727\over325}\right)$.
\vv

\item{9.}
 Spiegare nei dettagli il funzionamento del crittosistema RSA e si dia un esempio di una sua implementazione.
\ \vst

 \bye
