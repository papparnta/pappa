\nopagenumbers \font\title=cmti12
\def\ve{\vfill\eject}
\def\vv{\vfill}
\def\vs{\vskip-2cm}
\def\vss{\vskip10cm}
\def\vst{\vskip13.3cm}

%\def\ve{\bigskip\bigskip}
%\def\vv{\bigskip\bigskip}
%\def\vs{}
%\def\vss{}
%\def\vst{\bigskip\bigskip}

\hsize=19.5cm
\vsize=27.58cm
\hoffset=-1.6cm
\voffset=0.5cm
\parskip=-.1cm
\ \vs \hskip -6mm CR1 AA07/08\ (Crittografia a Chiave Pubblica)\hfill ESAME DI FINE SEMESTRE \hfill Roma, 5 Giugno 2008. \hrule
\bigskip\noindent
{\title COGNOME}\  \dotfill\ {\title NOME}\ \dotfill {\title
MATRICOLA}\ \dotfill\
\smallskip  \noindent
Risolvere il massimo numero di esercizi accompagnando le risposte
con spiegazioni chiare ed essenziali. \it Inserire le risposte
negli spazi predisposti. NON SI ACCETTANO RISPOSTE SCRITTE SU
ALTRI FOGLI. Scrivere il proprio nome anche nell'ultima pagina.
\rm 1 Esercizio = 4 punti. Tempo previsto: 2 ore. Nessuna domanda
durante la prima ora e durante gli ultimi 20 minuti.
\smallskip
\hrule\smallskip
\centerline{\hskip 6pt\vbox{\tabskip=0pt \offinterlineskip
\def \trl{\noalign{\hrule}}
\halign to300pt{\strut#& \vrule#\tabskip=0.7em plus 1em& \hfil#&
\vrule#& \hfill#\hfil& \vrule#& \hfil#& \vrule#& \hfill#\hfil&
\vrule#& \hfil#& \vrule#& \hfill#\hfil& \vrule#& \hfil#& \vrule#&
\hfill#\hfil& \vrule#& \hfil#& \vrule#& \hfill#\hfil& \vrule#&
\hfil#& \vrule#& \hfill#\hfil& \vrule#& \hfil#& \vrule#& \hfil#&
\vrule#\tabskip=0pt\cr\trl && FIRMA && 1 && 2 && 3 && 4 &&
5 && 6 && 7 && 8 && 9 &&  TOT. &\cr\trl && &&   &&
&&     &&   &&   &&   &&   &&   &&    && &\cr &&
\dotfill &&     &&   &&   &&   &&     &&   && && && &&
&\cr\trl }}}
\medskip


\item{1.} Illustrare un algoritmo per determinare se un polinomio a coeffcienti su un campo finito \`e irriducibile.

\vv

\item{2.} Calcolare la probabilit\`a che un polinomio monico di grado $6$ su ${\bf F}_2$ sia irriducibile e
la probabilit\`a un polinomio irriducibile grado $6$ su ${\bf F}_2$ sia primitivo. Si dia un esempio di un polinomio primitivo
di grado $6$ su ${\bf F}_2$.
\ve\vs

\item{3.} Dare un esempio di implementazione dello scambio delle chivi di Diffie Hellmann su un campo con $49$ elementi.
\vv


\item{4.} Utilizzare il metodo Baby Step Giant Step per calcolare il logaritmo discreto $\log_35$ dove $5\in{\bf Z}/29{\bf Z}$.
%%% 10
 \vv


\item{5.} Illustrare il funzionamento dell'algoritmo Pohlig Hellmann per il calcolo dei logaritmi discreti analizzandone
la complessit\`a.

\ve\vs

\item{6.} Spiegare il funzionamento del Crittosistema El Gamal fornendo un esempio esplicito su un campo con $8$ elementi.
\vv

\item{7.} Dimostrare che se ${\bf F}_{q}$ \`e un campo finito che ammette esclusivamente un sottocampo proprio, allora
$q=p^l$ dove $p$ e $l$ sono numeri primi. 

\ve \vs

\item{8.} Determinare le radici primitive di un campo con $2^7$ elementi.

\vv

\item{9.} Dopo aver verificato che si tratta di una curva ellittica, determinare (giustificando la risposta) l'ordine e la struttura del gruppo
dei punti razionali della curva ellittica su ${\bf F}_5$
$$y^2 = x^3-x + 3.$$
\ \vst

 \bye
