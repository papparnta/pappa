\nopagenumbers \font\title=cmti12
\def\ve{\vfill\eject}
\def\vv{\vfill}
\def\vs{\vskip-2cm}
\def\vss{\vskip10cm}
\def\vst{\vskip13.3cm}

%\def\ve{\bigskip\bigskip}
%\def\vv{\bigskip\bigskip}
%\def\vs{}
%\def\vss{}
%\def\vst{\bigskip\bigskip}

\hsize=19.5cm
\vsize=27.58cm
\hoffset=-1.6cm
\voffset=0.5cm
\parskip=-.1cm
\ \vs \hskip -6mm AL310 AA17/18\ (Teoria delle Equazioni)\hfill APPELLO X (Scritto) \hfill Roma, 20 Settembre 2018. \hrule
\bigskip\noindent
{\title COGNOME}\  \dotfill\ {\title NOME}\ \dotfill {\title
MATRICOLA}\ \dotfill\
\smallskip  \noindent
Risolvere il massimo numero di esercizi accompagnando le risposte
con spiegazioni chiare ed essenziali. \it Inserire le risposte
negli spazi predisposti. NON SI ACCETTANO RISPOSTE SCRITTE SU
ALTRI FOGLI. Scrivere il proprio nome anche nell'ultima pagina.
\rm 1 Esercizio = 4 punti. Tempo previsto: 2 ore. Nessuna domanda
durante la prima ora e durante gli ultimi 20 minuti.
\smallskip
\hrule\smallskip
\centerline{\hskip 6pt\vbox{\tabskip=0pt \offinterlineskip
\def \trl{\noalign{\hrule}}
\halign to277pt{\strut#& \vrule#\tabskip=0.7em plus 1em& \hfil#&
\vrule#& \hfill#\hfil& \vrule#& \hfil#& \vrule#& \hfill#\hfil&
\vrule#& \hfil#& \vrule#& \hfill#\hfil& \vrule#& \hfil#& \vrule#&
\hfill#\hfil& \vrule#& \hfil#& \vrule#& \hfill#\hfil& \vrule#&
\hfil#& \vrule#& \hfill#\hfil& \vrule#& \hfil#& \vrule#& \hfil#&
\vrule#\tabskip=0pt\cr\trl && FIRMA && 1 && 2 && 3 && 4 &&
5 && 6 && 7 && 8 &&   9 &\cr\trl && &&   &&
&&     &&   &&   &&   &&   &&    && &\cr &&
\dotfill &&     &&   &&   &&     &&   && && && &&
&\cr\trl }}}
\medskip

\item{1.} Rispondere alle sequenti domande fornendo una giustificazione di una riga (giustificazioni
incomplete o poco chiare comportano punteggio nullo):\bigskip\bigskip\bigskip


\itemitem{a.} Quale \`e polinomio minimo di
$\zeta_{16}\in{\bf Q}[\zeta_{16}]$ su ${\bf Q}[\sqrt{-1}]$?.\medskip\bigskip\bigskip

\ \dotfill\ \bigskip\bigskip\bigskip\vfil

\itemitem{b.} Scrivere una ${\bf F}_5$--base del campo di spezzamento del polinomio $X^{25}-X\in{\bf F}_5[X]$.\medskip\bigskip\bigskip

\ \dotfill\ \bigskip\bigskip\bigskip\vfil

\itemitem{c.} E' vero che se $E/F$ \`e un estensione di Galois e il gruppo di Galois ha 101 elementi, allora necessariamente l'estensione \`e abeliana?\medskip\bigskip\bigskip
 
\ \dotfill\ \bigskip\bigskip\bigskip\vfil

\itemitem{d.} Produrre un esempio di un polinomio di grado 3 il cui gruppo di Galois ha tre elementi giustificando la risposta.\medskip\bigskip\bigskip

\ \dotfill\ \bigskip\bigskip\bigskip

\vfil\eject

%Dimostrare che un estensione finita \`{e} necessariamente algebrica. Produrre
%un esempio di un estensione algebrica non finita.

\item{2.} Sia $\Omega/F$ un estensione di campi. Dimostrare che l’insieme degli elementi di $\Omega$ che sono algebrici su $F$ \`e un campo.  
\vv


\item{3.} Determinare il gruppo di Galois su ${\bf Q}$ di $x^4 - 7x^2 + 3x + 1$ e 
di $x^4+5x+5$.\vv


%Dopo aver verificato che \`e algebrico, calcolare
%il polinomio minimo di $\cos \pi/9$ su ${\bf Q}$.

\item{4.} Calcolare il numero di polinomi quadratici irriducibili su
${\bf F}_5$ e dopo averne scelti due distinti, si scriva un isomorfismo tra i
rispettivi campi a gambo. 
\ve\ \vs

\item{5.} Descrivere gli elementi del gruppo di Galois del polinomio $(x^5-2)\in{\bf Q}[x]$ determinando almeno 10 sottocampi del campo di spezzamento (in effetti ce ne sono 14).
\vv

%--\item{6.} Descrivere la nozione di campo perfetto dimostrando che i campi finiti
%sono perfetti.

\item{6.} Si enunci e si dimostri nella completa generalit\`a il Lemma di Artin spiegando che ruolo gioca nella corrispondenza di Galois.\vv


\item{7.} Dopo aver definito il discriminante $D_f$ di un polinomio irriducibile $f\in{\bf Q}[X]$, si determini il valore del discriminante di $(x^{101}-1)/(x-1)$.\ve\ \vs

\item{8.} Determinare esplicitamente un polinomio irriducibile in ${\bf Q}[X]$ 
il cui gruppo di Galois \`e isomorfo a 
$C_{2}\oplus C_{2}\oplus C_2$. 
\vv

\item{9.} 
\itemitem{a.} Dimostrare che il gruppo di Galois di un campo finito \`e sempre ciclico.
\itemitem{b.} Determinare la struttura del gruppo di Galois del polinomio $(x^3-x+1)(x^5-x-1)\in{\bf F}_2[x]$.
\ \vst
 \bye
