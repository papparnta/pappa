\nopagenumbers \font\title=cmti12
\def\ve{\vfill\eject}
\def\vv{\vfill}
\def\vs{\vskip-2cm}
\def\vss{\vskip10cm}
\def\vst{\vskip13.3cm}

%\def\ve{\bigskip\bigskip}
%\def\vv{\bigskip\bigskip}
%\def\vs{}
%\def\vss{}
%\def\vst{\bigskip\bigskip}

\hsize=19.5cm
\vsize=27.58cm
\hoffset=-1.6cm
\voffset=0.5cm
\parskip=-.1cm
\ \vs \hskip -6mm AL310 AA17/18\ (Teoria delle Equazioni)\hfill APPELLO B (Scritto) \hfill Roma, 23 Febbraio 2018. \hrule
\bigskip\noindent
{\title COGNOME}\  \dotfill\ {\title NOME}\ \dotfill {\title
MATRICOLA}\ \dotfill\
\smallskip  \noindent
Risolvere il massimo numero di esercizi accompagnando le risposte
con spiegazioni chiare ed essenziali. \it Inserire le risposte
negli spazi predisposti. NON SI ACCETTANO RISPOSTE SCRITTE SU
ALTRI FOGLI. Scrivere il proprio nome anche nell'ultima pagina.
\rm 1 Esercizio = 4 punti. Tempo previsto: 2 ore. Nessuna domanda
durante la prima ora e durante gli ultimi 20 minuti.
\smallskip
\hrule\smallskip
\centerline{\hskip 6pt\vbox{\tabskip=0pt \offinterlineskip
\def \trl{\noalign{\hrule}}
\halign to277pt{\strut#& \vrule#\tabskip=0.7em plus 1em& \hfil#&
\vrule#& \hfill#\hfil& \vrule#& \hfil#& \vrule#& \hfill#\hfil&
\vrule#& \hfil#& \vrule#& \hfill#\hfil& \vrule#& \hfil#& \vrule#&
\hfill#\hfil& \vrule#& \hfil#& \vrule#& \hfill#\hfil& \vrule#&
\hfil#& \vrule#& \hfill#\hfil& \vrule#& \hfil#& \vrule#& \hfil#&
\vrule#\tabskip=0pt\cr\trl && FIRMA && 1 && 2 && 3 && 4 &&
5 && 6 && 7 && 8 &&   9 &\cr\trl && &&   &&
&&     &&   &&   &&   &&   &&    && &\cr &&
\dotfill &&     &&   &&   &&     &&   && && && &&
&\cr\trl }}}
\medskip

\item{1.} Rispondere alle sequenti domande fornendo una giustificazione di una riga (giustificazioni
incomplete o poco chiare comportano punteggio nullo):\bigskip\bigskip\bigskip


\itemitem{a.} Quanti elementi ha il gruppo di Galois di $X^9-2\in{\bf Q}[X]?$\medskip\bigskip\bigskip

\ \dotfill\ \bigskip\bigskip\bigskip\vfil

\itemitem{b.} Scrivere una ${\bf Q}$--base del campo di spezzamento del polinomio $(X^5-2)(X^5-3)\in{\bf Q}[X]$.\medskip\bigskip\bigskip

\ \dotfill\ \bigskip\bigskip\bigskip\vfil

\itemitem{c.} Quanti elementi ha il campo di spezzamento di $(X^{16}+2X+2)(X^8+X^4+3)(X^{32}+X^2)\in{\bf F}_2[X]$?\medskip\bigskip\bigskip
 
\ \dotfill\ \bigskip\bigskip\bigskip\vfil

\itemitem{d.} \`E possibile costruire un esempio di estensione di un campo finito con gruppo di Galois isomorfo
a $D_4$?\medskip\bigskip\bigskip

\ \dotfill\ \bigskip\bigskip\bigskip

\vfil\eject

\item{2.} Siano $A$ un dominio e $F$ un campo con $F\subset A$. Mostrare che se $\dim_F(A)<\infty$, allora $A$ \`e un campo.

\vv


\item{3.} Determinare i valori di $m\in{\bf Z}$ tali che $\alpha=\cos m^\circ$ (il  coseno di $m$ gradi) \`e algebrico. Per quali che questi $\alpha$ risulta
costruibile?
\vv

\item{4.} Determinare i gruppi di Galois su ${\bf Q}$  e su ${\bf F}_5$ del seguente polinomio $x^5 + 5x^4 + 10x^3 + 10x^2 + 5x - 4$.\ve\ \vs

\item{5.} Si calcoli il polinomio minimo di $\sin\pi/5$ dopo aver mostrato che si tratta di un numero algebrico.
\vv


\item{6.} Si enunci e si dimostri il Teorema di Corrispondenza di Galois.\vv


\item{7.}  Dare un esempio di campo finito ${\bf F}_{27}$ con $27$
elementi determinando tutti i generatori del gruppo moltiplicativo
${\bf F}_{27}^*$.\ve\ \vs

\item{8.} Determinare $\alpha\in\overline{\bf Q}$ tale che il gruppo di Galois del polinomio minimo di $f_\alpha\in{\bf Q}[X]$
\`e isomorfo a $\left({\bf Z}/2{\bf Z}\right)^5$.\vv

\item{9.} Descrivere in dettagli is reticolo dei sottocampi del campo di spezzamento di $X^{13}-1\in{\bf Q}[X]$.
\ \vst
 \bye
