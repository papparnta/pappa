\nopagenumbers \font\title=cmti12
\def\ve{\vfill\eject}
\def\vv{\vfill}
\def\vs{\vskip-2cm}
\def\vss{\vskip10cm}
\def\vst{\vskip13.3cm}

%\def\ve{\bigskip\bigskip}
%\def\vv{\bigskip\bigskip}
%\def\vs{}
%\def\vss{}
%\def\vst{\bigskip\bigskip}

\hsize=19.5cm
\vsize=27.58cm
\hoffset=-1.6cm
\voffset=0.5cm
\parskip=-.1cm
\ \vs \hskip -6mm AL310 AA17/18\ (Teoria delle Equazioni)\hfill APPELLO C (Scritto) \hfill Roma, 12 Giugno. \hrule
\bigskip\noindent
{\title COGNOME}\  \dotfill\ {\title NOME}\ \dotfill {\title
MATRICOLA}\ \dotfill\
\smallskip  \noindent
Risolvere il massimo numero di esercizi accompagnando le risposte
con spiegazioni chiare ed essenziali. \it Inserire le risposte
negli spazi predisposti. NON SI ACCETTANO RISPOSTE SCRITTE SU
ALTRI FOGLI. Scrivere il proprio nome anche nell'ultima pagina.
\rm 1 Esercizio = 4 punti. Tempo previsto: 2 ore. Nessuna domanda
durante la prima ora e durante gli ultimi 20 minuti.
\smallskip
\hrule\smallskip
\centerline{\hskip 6pt\vbox{\tabskip=0pt \offinterlineskip
\def \trl{\noalign{\hrule}}
\halign to277pt{\strut#& \vrule#\tabskip=0.7em plus 1em& \hfil#&
\vrule#& \hfill#\hfil& \vrule#& \hfil#& \vrule#& \hfill#\hfil&
\vrule#& \hfil#& \vrule#& \hfill#\hfil& \vrule#& \hfil#& \vrule#&
\hfill#\hfil& \vrule#& \hfil#& \vrule#& \hfill#\hfil& \vrule#&
\hfil#& \vrule#& \hfill#\hfil& \vrule#& \hfil#& \vrule#& \hfil#&
\vrule#\tabskip=0pt\cr\trl && FIRMA && 1 && 2 && 3 && 4 &&
5 && 6 && 7 && 8 &&   9 &\cr\trl && &&   &&
&&     &&   &&   &&   &&   &&    && &\cr &&
\dotfill &&     &&   &&   &&     &&   && && && &&
&\cr\trl }}}
\medskip

\item{1.} Rispondere alle sequenti domande fornendo una giustificazione di una riga (giustificazioni
incomplete o poco chiare comportano punteggio nullo):\bigskip\bigskip\bigskip


\itemitem{a.} E' sempre vero che se $F$ \`e un campo e $\alpha$ \`e algebrico su $F$, allora $[F(\alpha):F]=\deg f_\alpha =\#{\rm Gal}(f_\alpha)$ (dove
${\rm Gal}(f)$ indica il gruppo di Galois del polinomio $f\in F[X]$?\medskip\bigskip\bigskip

\ \dotfill\ \bigskip\bigskip\bigskip\vfil

\itemitem{b.} Scrivere una ${\bf Q}$--base del campo ${\bf Q}[5^{1/4},5^{1/3}]$.\medskip\bigskip\bigskip

\ \dotfill\ \bigskip\bigskip\bigskip\vfil

\itemitem{c.} Quanti elementi ha il campo di spezzamento di $(X^{16}+6X+2)(X^8+X^4+5)(X^{32}+X^4)\in{\bf F}_2[X]$?\medskip\bigskip\bigskip
 
\ \dotfill\ \bigskip\bigskip\bigskip\vfil

\itemitem{d.} \`E possibile costruire un esempio di estensione di un campo finito con gruppo di Galois abeliano e isomorfo ${\bf Z}/4{\bf Z}\times{\bf Z}/4{\bf Z}$?\medskip\bigskip\bigskip

\ \dotfill\ \bigskip\bigskip\bigskip

\vfil\eject

\item{2.} Dimostrare che se $F$ \`e un campo e $G$ \`e un gruppo finito tale che $G\subseteq (F^*,\cdot)$, allora $G$ \`e ciclico.

\vv


\item{3.} Dimostrare che un polinomio a coefficienti razionali \`e irriducibile se e solo 
se il suo gruppo di Galois agisce transitivamente sulle sue radici.
\vv

\item{4.} Determinare i gruppi di Galois su ${\bf Q}$  e su ${\bf F}_7$ del seguente polinomio $x^5-3^5$.\ve\ \vs

\item{5.} Si calcoli il polinomio minimo di $\cos2\pi/15$ dopo aver mostrato che si tratta di un numero algebrico.
\vv


\item{6.} Si enunci e si dimostri il Teorema di Corrispondenza di Galois.\vv


\item{7.}  Dare un esempio, se esiste, di campo finito ${\bf F}_{81}$ con $81$
elementi. Inoltre mostrare che se $\alpha\in{\bf F}_{81}^*$ \`e tale che $\alpha^{16}\neq1$ e 
$\alpha^{40}=80$, allora ${\bf F}_{80}^*=\langle\alpha\rangle$.\ve\ \vs

\item{8.} Dimostrare che se $p$ \`e un numero primo, $\zeta_p=e^{2\pi/p}$ e se $H<{\rm Gal}({\bf Q}(\zeta_p)/{\bf Q})$, allora il campo degli invarianti ${\bf Q}(\zeta_p)^H$ coincide con ${\bf Q}(\eta_H)$ dove $\eta_H=\sum_{\sigma\in H}\sigma(\zeta_p)$.\vv

\item{9.} Descrivere in dettagli is reticolo dei sottocampi del campo di spezzamento di $X^{15}-1\in{\bf Q}[X]$ indicando per ciascun sottocampo il polinomio minimo di un generatore.
\ \vst

 \bye
