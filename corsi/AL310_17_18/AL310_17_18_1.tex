\nopagenumbers \font\title=cmti12
\def\ve{\vfill\eject}
\def\vv{\vfill}
\def\vs{\vskip-2cm}
\def\vss{\vskip10cm}
\def\vst{\vskip13.3cm}

% \def\ve{\bigskip\bigskip}
% \def\vv{\bigskip\bigskip}
% \def\vs{}
% \def\vss{}
% \def\vst{\bigskip\bigskip}

\hsize=19.5cm
\vsize=27.58cm
\hoffset=-1.6cm
\voffset=0.5cm
\parskip=-.1cm
\ \vs \hskip -6mm AL310 AA17/18\ (Teoria delle Equazioni)\hfill ESAME
DI MET\`{A} SEMESTRE \hfill Roma, 10 Novembre 2017 \hrule
\bigskip\noindent
{\title COGNOME}\  \dotfill\ {\title NOME}\ \dotfill {\title
MATRICOLA}\ \dotfill\
\smallskip  \noindent
Risolvere il massimo numero di esercizi accompagnando le risposte
con spiegazioni chiare ed essenziali. \it Inserire le risposte
negli spazi predisposti. NON SI ACCETTANO RISPOSTE SCRITTE SU
ALTRI FOGLI.
\rm 1 Esercizio = 4 punti. Tempo previsto: 2 ore. Nessuna domanda
durante la prima ora e durante gli ultimi 20 minuti.
\smallskip
\hrule\smallskip
\centerline{\hskip 6pt\vbox{\tabskip=0pt \offinterlineskip
\def \trl{\noalign{\hrule}}
\halign to277pt{\strut#& \vrule#\tabskip=0.7em plus 1em& \hfil#&
\vrule#& \hfill#\hfil& \vrule#& \hfil#& \vrule#& \hfill#\hfil&
\vrule#& \hfil#& \vrule#& \hfill#\hfil& \vrule#& \hfil#& \vrule#&
\hfill#\hfil& \vrule#& \hfil#& \vrule#& \hfill#\hfil& \vrule#&
\hfil#& \vrule#& \hfill#\hfil& \vrule#& \hfil#& \vrule#& \hfil#&
\vrule#\tabskip=0pt\cr\trl && FIRMA && 1 && 2 && 3 && 4 &&
5 && 6 && 7 && 8  &&  TOT. &\cr\trl && &&   &&
&&     &&   &&     &&   &&   &&    && &\cr &&
\dotfill &&       &&   &&   &&     &&   && && && &&
&\cr\trl }}}
\medskip

\item{1.} Rispondere alle seguenti domande fornendo una giustificazione di una riga:\bigskip\bigskip\bigskip


\itemitem{a.} Quale \`e il grado del campo di spezzamento del polinomio $(T^2+1)(T^2+2)(T^4-2)\in{\bf Q}[T]$?\medskip\bigskip\bigskip

\ \dotfill\ \bigskip\bigskip\bigskip\vfil

\itemitem{b.} E' sempre vero che, se $\alpha\in{\bf C}$ \`e algebrico, allora $\#\rm{Aut}({\bf Q}[\alpha]/{\bf Q})=[{\bf Q}[\alpha]:{\bf Q}]$?\medskip\bigskip\bigskip

\ \dotfill\ \bigskip\bigskip\bigskip\vfil

\itemitem{c.} E' vero che $\sin2\pi/n\in{\bf Q}(\zeta_n)$?\medskip\bigskip\bigskip
 
\ \dotfill\ \bigskip\bigskip\bigskip\vfil

\itemitem{d.} E' vero che ogni estensione algebrica \`e finita?\medskip\bigskip\bigskip

\ \dotfill\ \bigskip\bigskip\bigskip


\vfil\eject

%Dimostrare che un estensione finita \`{e} necessariamente algebrica. Produrre
%un esempio di un estensione algebrica non finita.

\item{2.} Sia $f\in{\bf Q}[X]$, irriducibile e di grado $8$. Considerare il campo ${\bf Q}[\alpha],f(\alpha)=0$ e dimostrare che
${\bf Q}[\alpha]={\bf Q}[\alpha^3]$. Fornire un esempio esplicito di $f$ come sopra ma tale che valgono le inclusioni proprie
${\bf Q}[\alpha]\supset{\bf Q}[\alpha^2]\supset{\bf Q}$.\vv

\item{3.} Sia $d$ un intero positivo dispari fissato. Dopo aver dimostrato che $f_d=X^4-2X^2-2d\in{\bf Q}[X]$ \`e irriducibile, si denoti con $F_d={\bf Q}[\alpha],\alpha^4=2\alpha^2+2d$.
\itemitem{a.} Dimostrare che $F_d$ ha un sottocampo isomorfo a ${\bf Q}[\sqrt{1+2d}],$
\itemitem{b.} Calcolare il grado del campo di spezzamento di $f_d$ su ${\bf Q}$.
\ve\ \vs

%Dopo aver verificato che \`e algebrico, calcolare
%il polinomio minimo di $\cos \pi/9$ su ${\bf Q}$.

\item{4.} Dopo aver descritto tutti gli elementi di Aut(${\bf Q}(2^{1/4},\sqrt{-1})/{\bf Q}$), si determini l'ordine di ciascuno di essi.\vv

\item{5.} Determinare il campo di spezzamento su ${\bf Q}$ di $f(X)=x^{15} - x^8 - x^7 + 1\in{\bf Q}[X]$ e se ne determini il grado su ${\bf Q}$.
\ve\ \vs

%--\item{6.} Descrivere la nozione di campo perfetto dimostrando che i campi finiti
%sono perfetti.

\item{6.} Dopo aver verificato che ${\bf Q}(\sqrt{2})\subset{\bf Q}(\zeta_8)$, descrivere gli ${\bf Q}(\sqrt{2})$--omomorfismi del campo 
${\bf Q}(\zeta_8)$ in ${\bf C}$.\vv\vv


\item{7.}  Dopo aver verificato che \`e algebrico, calcolare il polinomio minimo di $\sin \pi/12$ su ${\bf Q}$. 
\vv\vv


\item{8.} Enunciare e dimostrare il Lemma di Artin.

\vv

\ \vst
 \bye
