\nopagenumbers \font\title=cmti12
\def\ve{\vfill\eject}
\def\vv{\vfill}
\def\vs{\vskip-2cm}
\def\vss{\vskip10cm}
\def\vst{\vskip13.3cm}

%\def\ve{\bigskip\bigskip}
%\def\vv{\bigskip\bigskip}
%\def\vs{}
%\def\vss{}
%\def\vst{\bigskip\bigskip}

\hsize=19.5cm
\vsize=27.58cm
\hoffset=-1.6cm
\voffset=0.5cm
\parskip=-.1cm
\ \vs \hskip -6mm AL310 AA17/18\ (Teoria delle Equazioni)\hfill APPELLO A (Scritto) \hfill Roma, 31 Gennaio 2018. \hrule
\bigskip\noindent
{\title COGNOME}\  \dotfill\ {\title NOME}\ \dotfill {\title
MATRICOLA}\ \dotfill\
\smallskip  \noindent
Risolvere il massimo numero di esercizi accompagnando le risposte
con spiegazioni chiare ed essenziali. \it Inserire le risposte
negli spazi predisposti. NON SI ACCETTANO RISPOSTE SCRITTE SU
ALTRI FOGLI. Scrivere il proprio nome anche nell'ultima pagina.
\rm 1 Esercizio = 4 punti. Tempo previsto: 2 ore. Nessuna domanda
durante la prima ora e durante gli ultimi 20 minuti.
\smallskip
\hrule\smallskip
\centerline{\hskip 6pt\vbox{\tabskip=0pt \offinterlineskip
\def \trl{\noalign{\hrule}}
\halign to277pt{\strut#& \vrule#\tabskip=0.7em plus 1em& \hfil#&
\vrule#& \hfill#\hfil& \vrule#& \hfil#& \vrule#& \hfill#\hfil&
\vrule#& \hfil#& \vrule#& \hfill#\hfil& \vrule#& \hfil#& \vrule#&
\hfill#\hfil& \vrule#& \hfil#& \vrule#& \hfill#\hfil& \vrule#&
\hfil#& \vrule#& \hfill#\hfil& \vrule#& \hfil#& \vrule#& \hfil#&
\vrule#\tabskip=0pt\cr\trl && FIRMA && 1 && 2 && 3 && 4 &&
5 && 6 && 7 && 8 &&   9 &\cr\trl && &&   &&
&&     &&   &&   &&   &&   &&    && &\cr &&
\dotfill &&     &&   &&   &&     &&   && && && &&
&\cr\trl }}}
\medskip

\item{1.} Rispondere alle sequenti domande fornendo una giustificazione di una riga (giustificazioni
incomplete o poco chiare comportano punteggio nullo):\bigskip\bigskip\bigskip


\itemitem{a.} E' vero che le estensione finite di campi finiti sono sempre estensioni di Galois?\medskip\bigskip\bigskip

\ \dotfill\ \bigskip\bigskip\bigskip\vfil

\itemitem{b.} Scrivere una ${\bf Q}$--base del campo di spezzamento del polinomio $(X^2-2)(X^2-3)\in{\bf Q}[X]$.\medskip\bigskip\bigskip

\ \dotfill\ \bigskip\bigskip\bigskip\vfil

\itemitem{c.} \`E vero che se $f\in{\bf F}_5[X]$ \`e un polinomio riducibile di grado tre allora il suo campo di spezzamento 
\`e sempre contenuto in ${\bf F}_5[\alpha], \alpha^2=\alpha-2$?\medskip\bigskip\bigskip
 
\ \dotfill\ \bigskip\bigskip\bigskip\vfil

\itemitem{d.} Fornire un esempio, se esiste, di estensione trascendente di ${\bf F}_{101}$.\medskip\bigskip\bigskip

\ \dotfill\ \bigskip\bigskip\bigskip

\vfil\eject

%Dimostrare che un estensione finita \`{e} necessariamente algebrica. Produrre
%un esempio di un estensione algebrica non finita.

\item{2.} Dimostrare che se $E/F$ \`e un estensione finita, allora \`e algebrica e finitamente generata. Fornire un esempio di estensione algebrica e 
non finitamente generata  e un emsempio di estensione finitamente generata ma non algebrica.  
\vv


\item{3.} Fornire un esempio di polinomio $f\in{\bf Q}[X]$ con gruppo di Galois $G_f\cong A_4$ e con $G_f\cong C_4$ (pensare a $x^4 - 7x^2 + 3x + 1$ e 
a $x^4+5x+5$).\vv


%Dopo aver verificato che \`e algebrico, calcolare
%il polinomio minimo di $\cos \pi/9$ su ${\bf Q}$.

\item{4.} Calcolare le radici di $X^3+X^2+1$ nel campo ${\bf F}_2[\gamma], \gamma^3=\gamma+1$ e determinare $a,b,c\in{\bf F}_2$, se esistono,
tali che $1/\gamma^4=a+b\gamma+c\gamma^2$.  
\ve\ \vs

\item{5.} Sia $E$ il campo di spezzamento di $(x^5-3)(x^5-7)\in{\bf Q}[x]$. Quale \`e il grado di $E$ su ${\bf Q}$?
Elencare almeno 10 sottocampi di $E$ che non sono contenuti nei campi di spezzamento di $x^5-3$ e di $x^5-7$. Ce ne sono altri con la stessa propriet\ 'a? 
(E' lecito assumere
che $7$ non \`e un quinta potenza nel campo di spezzamento di $x^5-3$)
\vv

%--\item{6.} Descrivere la nozione di campo perfetto dimostrando che i campi finiti
%sono perfetti.

\item{6.} Si enunci e si dimostri nella completa generalit\`a il Teorema di
corrispondenza di Galois.\vv


\item{7.} Dopo aver definito il discriminante $D_f$ di un polinomio irriducibile $f\in{\bf F}[X]$ di grado $n$, (${\bf F}$ campo di caratteristica zero), 
si dimostri che $D_f\in{\bf F}$ \`e un quadrato perfetto se e solo se il gruppo di Galois $G_f$ \`e isomorfo a un sottogruppo dell
gruppo alterno $A_n$.\ve\ \vs

\item{8.} Deteminare un numero algebrico $\gamma$ tale che ${\bf Q}[\gamma]/{\bf Q}$ \`e di Galois e con gruppo di Galois isomorfo a 
$C_{35}\oplus C_{140}$. Quale \`e il grado del polinomio minimo di $\gamma$?
\vv

\item{9.} 
\itemitem{a.} Quanti sono i fattori irriducibili di $x^{255}-1\in{\bf Q}[x]$ e quali sono i loro gradi?
\itemitem{b.} Quanti sono i fattori irriducibili di $x^{255}-1\in{\bf F}_2[x]$ e quali sono i loro gradi?
\ \vst
 \bye
