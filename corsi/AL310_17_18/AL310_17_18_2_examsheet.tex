\nopagenumbers \font\title=cmti12
% \def\ve{\vfill\eject}
% \def\vv{\vfill}
% \def\vs{\vskip-2cm}
% \def\vss{\vskip10cm}
% \def\vst{\vskip13.3cm}

\def\ve{\bigskip\bigskip}
\def\vv{\bigskip\bigskip}
\def\vs{}
\def\vss{}
\def\vst{\bigskip\bigskip}

\hsize=19.5cm
\vsize=27.58cm
\hoffset=-1.6cm
\voffset=0.5cm
\parskip=-.1cm
\ \vs \hskip -6mm AL310 AA17/18\ (Teoria delle Equazioni)\hfill ESAME
DI FINE SEMESTRE \hfill Roma, 18 Dicembre 2017. \hrule
\bigskip\noindent
{\title COGNOME}\  \dotfill\ {\title NOME}\ \dotfill {\title
MATRICOLA}\ \dotfill\
\smallskip  \noindent
Risolvere il massimo numero di esercizi accompagnando le risposte
con spiegazioni chiare ed essenziali. \it Inserire le risposte
negli spazi predisposti. NON SI ACCETTANO RISPOSTE SCRITTE SU
ALTRI FOGLI. Scrivere il proprio nome anche nell'ultima pagina.
\rm 1 Esercizio = 4 punti. Tempo previsto: 2 ore. Nessuna domanda
durante la prima ora e durante gli ultimi 20 minuti.
\smallskip
\hrule\smallskip
\centerline{\hskip 6pt\vbox{\tabskip=0pt \offinterlineskip
\def \trl{\noalign{\hrule}}
\halign to277pt{\strut#& \vrule#\tabskip=0.7em plus 1em& \hfil#&
\vrule#& \hfill#\hfil& \vrule#& \hfil#& \vrule#& \hfill#\hfil&
\vrule#& \hfil#& \vrule#& \hfill#\hfil& \vrule#& \hfil#& \vrule#&
\hfill#\hfil& \vrule#& \hfil#& \vrule#& \hfill#\hfil& \vrule#&
\hfil#& \vrule#& \hfill#\hfil& \vrule#& \hfil#& \vrule#& \hfil#&
\vrule#\tabskip=0pt\cr\trl && FIRMA && 1 && 2 && 3 && 4 &&
5 && 6 && 7 && 8 &&  9 &\cr\trl && &&   &&
&&     &&   &&   &&   &&   &&    && &\cr &&
\dotfill    &&   &&   &&   &&     &&   && && && &&
&\cr\trl }}}
\medskip

\item{1.} Rispondere alle seguenti domande fornendo una giustificazione di una riga:\bigskip\bigskip\bigskip


\itemitem{a.} E' vero che se $f(X),g(X)\in{\bf F}_p[X]$ hanno lo stesso grado, allora  
hanno campi di spezzamento su ${\bf F}_p$ isomorfi? \medskip\bigskip\bigskip

\ \dotfill\ \bigskip\bigskip\bigskip\vfil

\itemitem{b.} Dire quali dei seguenti numeri sono costruibili: $\sqrt{\cos(2\pi/17)}, \cos(2\pi/18), (3+3^{1/2})^{1/4}$, $120\sqrt{\pi}$?\medskip\bigskip\bigskip

\ \dotfill\ \bigskip\bigskip\bigskip\vfil

\itemitem{c.} Calcolare il numero di divisori di $x^{124}-1\in{\bf F}_5[x]$.\medskip\bigskip\bigskip
 
\ \dotfill\ \bigskip\bigskip\bigskip\vfil

\itemitem{d.} \`E vero che se $f\in{\bf Q}[X]$ ha grado $2^n$ 
allora le sue radici sono costruibili?\medskip\bigskip\bigskip

\ \dotfill\ \bigskip\bigskip\bigskip


\vfil\eject

%Dimostrare che un estensione finita \`{e} necessariamente algebrica. Produrre
%un esempio di un estensione algebrica non finita.

\item{2.} Fornire un esempio di un polinomio irriducibile di grado otto il cui gruppo di Galois \`e isomorfo al gruppo delle
simmetrie del quadrato $D_4$.\vv


\item{3.} Sia $f_a(x)=x^3-3x+a$. Determinare quattro valori di $a_1, a_2, a_3, a_4\in{\bf Z}$ in modo tale che i tre gruppi di Galois $G_{f_{a_1}}, 
G_{f_{a_2}},$ 
$G_{f_{a_3}}$ e $G_{f_{a_4}}$ siano a due a due non isomorfi. ({\it suggerimento: considerare $a=0,1,2,3$})
\ve\ \vs

%Dopo aver verificato che \`e algebrico, calcolare
%il polinomio minimo di $\cos \pi/9$ su ${\bf Q}$.

\item{4.} Si determini esplicitamente un isomorfismo tra ${\bf F}_2[\alpha], \alpha^4=\alpha+1$ e ${\bf F}_2[\beta], 
\beta^4=\beta^3+\beta^2+\beta+1$. 
\vv

\item{5.} Descrivere il reticolo dei sottocampi del campo ciclotomico ${\bf Q}[\zeta_{20}]$ menzionando i ciascun caso i generatori.
\vv


%--\item{6.} Descrivere la nozione di campo perfetto dimostrando che i campi finiti
%sono perfetti.

\item{6.} Si enunci nella completa generalit\`a il Teorema di
corrispondenza di Galois.\ve\ \vs


\item{7.} Calcolare il gruppo di Galois del polinomio $x^4 - 8x^3 + 24x^2 - 32x + 14
\in {\bf Q}[x]$.\vskip7cm\vv\vv

\item{8.} Calcolare il gruppo di Galois del polinomio $x^4 - 8x^3 + 24x^2 - 32x + 14
\in {\bf F}_3[x]$. ({\it suggerimento: considerare $x^2+1\in {\bf F}_3[x]$})\vskip7cm\vv\vv

\item{9.} Determinare un numero algebrico il cui polinomio minimo sui razionali ha un gruppo di 
Galois isomorfo a $C_{2}\times C_{50}$.

\vv



\ \vst
 \bye
