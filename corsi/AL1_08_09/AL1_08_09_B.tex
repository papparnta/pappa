\nopagenumbers \font\title=cmti12
\def\ve{\vfill\eject}
\def\vv{\vfill}
\def\vs{\vskip-2cm}
\def\vss{\vskip10cm}
\def\vst{\vskip13.3cm}

%\def\ve{\bigskip\bigskip}
%\def\vv{\bigskip\bigskip}
%\def\vs{}
%\def\vss{}
%\def\vst{\bigskip\bigskip}

\hfuzz=3cm
\hsize=19cm
\hsize=19.5cm
\vsize=27.58cm
\hoffset=-1.6cm
\voffset=0.5cm
\parskip=-.1cm
\ \vs \hskip -6mm AL1 AA08/09\ (Algebra: fondamenti)\hfill Appello B \hfill Roma, 6 Febbraio 2009. \hrule
\bigskip\noindent {\title COGNOME}\  \dotfill\ {\title NOME}\ \dotfill {\title
MATRICOLA}\ \dotfill\
\smallskip  \noindent
Risolvere il massimo numero di esercizi accompagnando le risposte
con spiegazioni chiare ed essenziali. \it Inserire le risposte
negli spazi predisposti. NON SI ACCETTANO RISPOSTE SCRITTE SU
ALTRI FOGLI. Scrivere il proprio nome anche nell'ultima pagina.
\rm 1 Esercizio = 4 punti. Tempo previsto: 2 ore. Nessuna domanda
durante la prima ora e durante gli ultimi 20 minuti.
\smallskip
\hrule\smallskip
\centerline{\hskip 6pt\vbox{\tabskip=0pt \offinterlineskip
\def \trl{\noalign{\hrule}}
\halign to300pt{\strut#& \vrule#\tabskip=0.7em plus 1em& \hfil#&
\vrule#& \hfill#\hfil& \vrule#& \hfil#& \vrule#& \hfill#\hfil&
\vrule#& \hfil#& \vrule#& \hfill#\hfil& \vrule#& \hfil#& \vrule#&
\hfill#\hfil& \vrule#& \hfil#& \vrule#& \hfill#\hfil& \vrule#&
\hfil#& \vrule#& \hfill#\hfil& \vrule#& \hfil#& \vrule#& \hfil#&
\vrule#\tabskip=0pt\cr\trl && FIRMA && 1 && 2 && 3 && 4 &&
5 && 6 && 7 && 8 && 9 &&  TOT. &\cr\trl && &&   &&
&&     &&   &&   &&   &&   &&   &&    && &\cr &&
\dotfill &&     &&   &&   &&   &&     &&   && && && &&
&\cr\trl }}}
\medskip

\item{1.} Dopo aver definito la nozione di iniettivit\`a e suriettivit\`a, si fornisca un esempio esplicito di
applicazione iniettiva e non suriettiva dall'insieme dei numeri reali ${\bf R}$ in se.
%APPLICAZIONI DI INSIEMI
\ve\vs   %foglio1
\item{2.} Dopo aver definito la nozione di relazione di equivalenza, di dimostri che
se una relazione \`e sia di equivalenza che di ordine, allora \`e necessariamente
la banale (cio\`e \`e relazione che dichiara due elementi equivalenti se e solo se
sono uguali)%RELAZIONI DI EQUIVALENZA
.\vv
\item{3.} Dopo aver enunciato gli assiomi di Peano e la
definizione di finitezza per un insieme, dimostrare esplicitamente
che l'unione di due insiemi finiti \`e finita.
%ASSIOMI DI PEANO, INSIEMI INFINITI, RELAZIONI D'ORDINE
\ve\vs   %foglio2
\item{4.} Usare il principio di induzione per dimostrare che $\displaystyle{\sum_{k=2}^n{1\over k^2-1}={3\over4}-{2n+1\over2n(n+1)}}$ per ogni $n\ge2$
%PRINCIPIO DI INDUZIONE
.\ve\vs   %foglio3
\item{5.} Si determinino gli interi nell'intervallo $[20,80]$ che soddisfano la congruenza $25X\equiv 10\bmod 30$.
%Cinque pirati e una scimmia sono naufragati su un'isola. 
%I pirati hanno raccolto un mucchio di noci di cocco e progettano di dividersele in parti uguali la mattina successiva. 
%Non fidandosi degli altri, un pirata si sveglia durante la notte e divide le noci di cocco in cinque parti uguali. Avanzandogliene una, 
%la d\`a scimmia. Poi Il pirata nasconde la sua parte di noci. Successivamente durante la notte, ciascuno degli altri pirati fa esattamente la stessa cosa.
%Divide il mucchio di noci che trova in cinque parti uguali, d\`a alla scimmia quella che avanza e nasconde la sua parte. 
%Di mattina, i pirati si riuniscono e si dividono il mucchio di noci restante in cinque parti uguali e, avanzandogliene una, la lasciano alla scimmia. 
%Quale \`e il  pi\`u piccolo numero totale di noci di cocco i pirati potrebbero aver  raccolto il giorno prima?
% risposta=15621
%CONGRUENZE LINEARI/TCDR
\vv
\item{6.} Dimostrare che l'insieme
${\bf Q}(i)=\{(a+ib)\in{\bf C}|\ a,b\in{\bf Q}, a^2+b^2\neq0\}$
\`e un gruppo rispetto al prodotto.
%GRUPPI E/O CAMPI
\ve\vs   %foglio4
\item{7.} %NUMERI COMPLESSI
Calcolare la parte reale, la immaginaria e la norma di $(5+5\sqrt{3}i)^5+{1\over2+7i}$ .\ve \vs  %foglio5
\item{8.} Dopo aver enunciato il Teorema di Eulero, lo si utilizzi per calcolare
le ultime due cifre decimali di $7^{2009}$.
%FUNZIONE DI EULERO/ P T FERMAT
\vv
\item{9.} %PERMUTAZIONI
Siano $\sigma=(1,2,3,4,5,6,7,8,9,10)$ e $\tau=(2,3,7,4,1)$.
Determinare la parit\`a e il supporto di  $\sigma^5\tau$.\ \ve %foglio 6
\bye
