\documentclass[italian,a4paper,11pt]
{article}
\usepackage{babel,amsmath,amssymb,amsbsy,amsfonts,latexsym,exscale,
amsthm,epsf,colordvi,enumerate}

\usepackage[latin1]{inputenc}
\usepackage[all]{xy}
\usepackage{textcomp}
\usepackage{graphicx}


\newcommand{\Q}{\mathbb{Q}}
\newcommand{\Z}{\mathbb Z}
\newcommand{\R}{\mathbb{R}}

\newcommand{\F}{\mathbb{F}}
\newcommand{\N}{\mathbb{N}}
\newcommand{\C}{\mathbb{C}}
\newcommand{\T}{\mathcal{T}}

\newcommand{\U}{\mathcal{U}}
\newcommand{\p}{\mathfrak{p}}
\newcommand{\ga}{\mathfrak{a}}
\newcommand{\gb}{\mathfrak{b}}

\newcommand{\q}{\mathfrak{q}}
\newcommand{\m}{\mathfrak{m}}
\newcommand{\X}{\mathbf{X}}

\newcommand{\D}{\mbox{\rm{\textbf{Dom}}}}
\newcommand{\Ze}{\mbox{\rm{\textbf{Rie}}}}

\newcommand{\esse}{\mbox{\rm{\textbf{Spec}}}}
\newcommand{\Ci}{\mathbf{C}}
\newcommand{\Ex}{\textbf{Esercizio}}


\newcommand{\Sse}{\Longleftrightarrow}
\newcommand{\sse}{\Leftrightarrow}
\newcommand{\implica}{\Rightarrow}

\newcommand{\frecdl}{\longrightarrow}
\newcommand{\frecd}{\rightarrow}
\newcommand{\st}{\scriptstyle}

\begin{document}
\begin{center}


\textbf{Universit\`a degli Studi Roma Tre}\\

\textbf{Corso di Laurea in Matematica, a.a. 2008/2009}\\

\textbf{AL1 - Algebra 1: Fondamenti}\\

\textbf{Prof. F. Pappalardi}\\

\textbf{Tutorato 10 - 18 Dicembre 2008}\\

\textbf{Elisa Di Gloria, Luca Dell'Anna}\\

www.matematica3.com\\
\end{center}



\vspace{0.5cm}




\noindent
\begin{Ex}\textbf{ 1.}\\
Si dimostri che $\varphi(m)=m-1 \Sse m$ \`e un numero primo.
\end{Ex}

\vspace{0.4cm}
\noindent
\begin{Ex}\textbf{ 2.}\\
Provare che se $p$ \`e un numero primo diverso da 2,3 e 5, allora $p$ divide il numero $u_p=111\dots 1$ \quad $p-1$ volte (i.e. il numero ha $p-1$ cifre uguali a 1). 
\end{Ex}

\vspace{0.4cm}
\noindent
\begin{Ex}\textbf{ 3.}\\
Siano $p$ e $q$ due numeri primi distinti e sia $a\in \Z$ tale che
$$a^q\equiv a \textrm{ (mod $p$)}$$ $$a^p\equiv a \textrm{ (mod $q$)}$$
Dimostrare che $a^{pq}\equiv a$ (mod $pq$).
\end{Ex}


\vspace{0.4cm}
\noindent
\begin{Ex}\textbf{ 4.}\\
Sia $a\in \Z$, dimostrare che $10\mid a^5-a$.
\end{Ex}

\vspace{0.4cm}
\noindent
\begin{Ex}\textbf{ 5.}\\
Dire se e quali dei seguenti sono anelli o campi:
\begin{itemize}
	\item $(\Z,+,\cdot)$
	\item $(\Z_n,+,\cdot)$
	\item $(m\Z,+,\cdot)$  $m\neq 1$
	\item $(\N,+,\cdot)$
	\item $(\R \times \Z,+,\cdot)$
	\item $(\Z_4\times \Z_9,+,\cdot)$.
\end{itemize}
Per ciascuno dei casi precedenti, esplicitare lo zero e l'unit\`a delle operazioni.
\end{Ex}


\vspace{3cm}
\noindent
\begin{Ex}\textbf{ 6.}\\ 
Sia $S$ un insieme fissato e $\mathcal{P}(S)$ l'insieme delle parti di $S$. Definiamo su $\mathcal{P}(S)$ le seguenti operazioni: per ogni $A,B \in \mathcal{P}(S)$
$$A+B=A\triangle B$$
$$A\cdot B=A\cap B$$
Dove $A\triangle B=A\cup B \setminus A\cap B$.\\
Stabilire se \`e un anello.\\
Pu\`o essere mai un campo?
\end{Ex}

\vspace{0.4cm}
\noindent
\begin{Ex}\textbf{ 7.}\\ 
Su $\Z$ si definiscano le seguenti operazioni:
$$\begin{array}{cc}
x\oplus y=&x+y-1\\
x\otimes y=&xy-x-y+2\\
x\star y=&x+y-xy
\end{array}$$
\begin{itemize}
	\item Provare che $(\Z,\oplus,\otimes)$ \`e un anello.
	\item Provare che $(\Z,\oplus,\star)$ \`e un dominio.
\end{itemize}
Si scrivano esplicitamente gli elementi neutri delle tre operazioni.
\end{Ex}
\end{document}
