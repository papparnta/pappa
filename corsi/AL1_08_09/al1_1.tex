%&Latex

\documentclass{article}
\usepackage{amssymb}
\usepackage[latin1]{inputenc}
%\usepackage{amsfonts}
\begin{document}
\newtheorem{definition}{Definition}
\newtheorem{example}{Example}
\newtheorem{theorem}{Teorema}
\newtheorem{proposition}{Proposition}
\newtheorem{lemma}{Lemma}
\newtheorem{corollary}{Corollary}
\newtheorem{remark}{Remark}
\textheight = 47\baselineskip
\newcounter{pippo}
\def\stp{\stepcounter{pippo}}
\renewcommand{\theequation}{\arabic{section}.\arabic{equation}}
\renewcommand{\thetheorem}{\arabic{section}.\arabic{pippo}}
\renewcommand{\theexample}{\arabic{section}.\arabic{pippo}}
\renewcommand{\theremark}{\arabic{section}.\arabic{pippo}}
\renewcommand{\thedefinition}{\arabic{section}.\arabic{pippo}}
\renewcommand{\theproposition}{\arabic{section}.\arabic{pippo}}
\renewcommand{\thelemma}{\arabic{section}.\arabic{pippo}}
\renewcommand{\thecorollary}{\arabic{section}.\arabic{pippo}}
\newcommand{\cvd}{$\hfill \sqcap \hskip-6.5pt \sqcup$} %(quadratino
%bianco)
\newcommand{\erre}{\mathop{{\rm I}\mskip -4.0mu{\rm R}}\nolimits}
\newcommand{\eps}{\varepsilon}
\newcommand{\na}{\nabla}
\newcommand{\ue}{u_\eps}
\newcommand{\Om}{\Omega}
\newcommand{\D}{\Delta}
\newcommand{\dd}{{\displaystyle}}
\newcommand{\lfr}{\longrightarrow}
\newcommand{\fr}{\rightarrow}
\newcommand{\vs}{\vskip0cm\noindent}
\newcommand{\1}{\lambda_{1}}
\newcommand{\2}{\lambda_{2}}
\newcommand{\INT}{\int_{\Omega}}
\newcommand{\un}{u_{n}}
\newcommand{\en}{e_{n}}
\newcommand{\acca}{H^{1}_{0}}
\newcommand{\be}{\begin{equation}}
\newcommand{\ee}{\end{equation}}
\newcommand{\beqa}{\begin{eqnarray}}
\newcommand{\eeqa}{\end{eqnarray}}
\newcommand{\vn}{v_{n}}
%%%%%%%%%%%%%%%%%%%%%%%%%%
\centerline{Corso di Algebra I del Prof. Pappalardi}
\vskip0,2cm
\centerline{Tutorato I del $02-10-2008$}
\vskip0,2cm
\centerline{Tutori: Luca Dell'Anna, Elisa Di Gloria}
\vskip0,2cm
\centerline{http://www.matematica3.com}
\vskip0,5cm

\noindent {\bf{Esercizio $1$}}\\
Siano $A$,$B$ e $C$ tre insiemi. Mostrare che:

\begin{enumerate}

\item $A \setminus B = A \setminus(A \cap B)=(A \cup B) \setminus B$

\item $(A \setminus B)\cap (B \setminus A)= \emptyset $ 

\item $A \cup B=(A \cap B) \cup (A \setminus B) \cup (B \setminus A)$

\item $(A \setminus B) \setminus C=A \setminus (B \cup C)$

\item $A \setminus (B \setminus C)=(A \setminus B) \cup (A \cap C)\supset (A \setminus B) \setminus C$

\item $A\cup (B \setminus C)=(A \cup B)\setminus (C \setminus A)=((A\cup B)\setminus C)\cup A $

\item $A\cap (B \setminus C)=(A \cap B) \setminus (A\cap C)$

\end{enumerate}
\vspace{0,2cm}
\noindent {\bf{Esercizio $2$}}
\\
$\forall$ r $\in \mathbb{R}_+$ e $\forall$ n $\in \mathbb{N}\setminus \{0\}=\mathbb{N}_+$ Siano:\\
$T_r$ :=\{$x\in \mathbb{R}|-\frac{1}{r} \leq x \leq \frac{1}{r}$\} \\$R_n :=\{x\in \mathbb{R}|-\frac{1}{n} \leq x \leq \frac{1}{n}\}$\\
Determinare:
\begin{itemize}
	\item $\bigcup_{r \in \mathbb{R}_+}T_r$;
	\item $\bigcap_{r \in \mathbb{R}_+}T_r$;
	\item $\bigcup_{n \in \mathbb{N}_+}R_n$;
	\item $\bigcap_{n \in \mathbb{N}_+}R_n$.

\end{itemize}
\vspace{0,2cm}
\noindent {\bf{Esercizio $3$}}
\\
Per ogni intero positivo $n$, si consideri il seguente sottoinsieme dei numeri reali:
\\
\begin{center}
$A_n :=\{x \, \, numero\, \, reale\,\, : \, \, x\leq 1/n \}$
\end{center}

Determinare:
\begin{enumerate}
\item$\cap (A_n : n\geq 1)  $
\item$\cup (A_n : n\geq 1)$
\end{enumerate}

\vspace{0,2cm}
\noindent {\bf{Esercizio $4$}}
\\
Determinare $A\cap B$ e $A\cup B$ nei seguenti casi:
\begin{itemize}
	\item $A=\{x\in \mathbb{Z} |x^2-4x-5\leq 0 \}$, $B=\{x\in \mathbb{Z} |x^2-12x+20\leq0\}$;
	\item $A=\{x\in \mathbb{Z} |x^2-5x+6\geq 0\}$, $B=\{x\in \mathbb{Z} |x^2-4\leq 0\}$;
	\item $A=\{x\in \mathbb{Z} |x^2-8x+7\geq 0\}$, $B=\{x\in \mathbb{Z} |x-3\leq 0\}$;
	\item $A=\{x\in \mathbb{Z} |x=5n,\ n\in \mathbb{Z}\}$, $B=\{x\in \mathbb{Z} |x=12n,\ n\in \mathbb{Z}\}$;
	
\end{itemize}

\vspace{0,2cm}
\noindent {\bf{Esercizio $5$}}
\\
Determinare $A\cap B$, $A\cup B$, $B\setminus A$ e $A\setminus B$ nei seguenti casi:

\begin{enumerate} 

\item $A=\{x\in \mathbb{Z} |x^2 -3x+2\leq 0 \}$; $B=\{x\in \mathbb{Z} |x^2 -12x+20>0\}$
\item $A=\{x\in \mathbb{Z} |\frac {5x-3}{2-x}\geq 0\}$; $B=\{x\in \mathbb{Z} |\frac {-7x+2}{3x-1}\leq 0\}$

\end{enumerate}

\vspace{0,2cm}
\noindent {\bf{Esercizio $6$}}
\\
Supponiamo che gli studenti del secondo anno del Corso di Laurea in
Matematica siano 200 e che essi abbiano superato almeno un esame.
Supponiamo inoltre che 120 studenti abbiano superato Algebra 1 e
130 Analisi 1.
Quanti studenti hanno superato tutti e due gli esami?

\vspace{0,2cm}
\noindent {\bf{Esercizio $7$}}
\\
Sia $A=\{x\in \mathbb{N} |3\leq x\leq 28 \}$. Siano $B=\{x\in A|x=2n$ con$ \: n\in \mathbb{N} \}$
$C=\{x\in A|x=5m \:$ con $  \: m\in \mathbb{N} \}$\\
Determinare $B\cap C$, $B\cup C$, $B\setminus C$.

\vspace{0,2cm}
\noindent {\bf{Esercizio $8$}}
\\
Sia $A :=\{x\in \mathbb{Z}| x^2-1\leq 0\}$. Determinare:
$\mathcal{P} (A)$, $\mathcal{P}(\mathcal{P}(A))$, $\mathcal{P}(A) \setminus A$.

\vspace{0,2cm}
\noindent {\bf{Esercizio $9$}}
\\
Date le seguenti funzioni, stabilire quali di esse sono iniettive e/o suriettive e quando possibile calcolarne esplicitamente l'inversa, definire il codominio, il dominio, l'immagine.

Sia $f: \mathbb{R} \rightarrow \mathbb{R}$ definita come:
\begin{itemize}
	\item $f(x)=x^2$;
	\item $f(x)=x+3$;
	\item $f(x)=2x+6$;
	\item $f(x)=x^3-2$;
	\item $f(x)=6x^4+1$. 
\end{itemize}

Siano ora:
\begin{itemize}
	\item $f: \mathbb{Z} \rightarrow \mathbb{R} $\qquad $f(n)=3n$;
	\item $f: \mathbb{R} \rightarrow [-1,1]$ \qquad $f(x)= \sin(\pi x) +2$;
	\item $f: \mathbb{Q} \rightarrow \mathbb{Q}$ \qquad $f(n)=5n-2$.

\end{itemize}

\end{document}
