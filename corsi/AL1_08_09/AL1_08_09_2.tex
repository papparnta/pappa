\nopagenumbers \font\title=cmti12
\def\ve{\vfill\eject}
\def\vv{\vfill}
\def\vs{\vskip-2cm}

%\def\ve{\bigskip\bigskip}
%\def\vv{\bigskip\bigskip}
%\def\vs{}

\hfuzz=3cm
\hsize=19cm
\vsize=27.58cm
\hoffset=-1.6cm
\voffset=0.5cm
\parskip=-.1cm
\ \vs \hskip -6mm AL1 AA08/09\ (Algebra: fondamenti)\hfill ESAME DI FINE SEMESTRE \hfill Roma, 14 Gennaio 2009. \hrule
\bigskip\noindent {\title COGNOME}\  \dotfill\ {\title NOME}\ \dotfill {\title
MATRICOLA}\ \dotfill\
\smallskip  \noindent
Risolvere il massimo numero di esercizi accompagnando le risposte
con spiegazioni chiare ed essenziali. \it Inserire le risposte
negli spazi predisposti. NON SI ACCETTANO RISPOSTE SCRITTE SU
ALTRI FOGLI. Scrivere il proprio nome anche nell'ultima pagina.
\rm 1 Esercizio = 4 punti. Tempo previsto: 2 ore. Nessuna domanda
durante la prima ora e durante gli ultimi 20 minuti.
\smallskip
\hrule\smallskip
\centerline{\hskip 6pt\vbox{\tabskip=0pt \offinterlineskip
\def \trl{\noalign{\hrule}}
\halign to300pt{\strut#& \vrule#\tabskip=0.7em plus 1em& \hfil#&
\vrule#& \hfill#\hfil& \vrule#& \hfil#& \vrule#& \hfill#\hfil&
\vrule#& \hfil#& \vrule#& \hfill#\hfil& \vrule#& \hfil#& \vrule#&
\hfill#\hfil& \vrule#& \hfil#& \vrule#& \hfill#\hfil& \vrule#&
\hfil#& \vrule#& \hfill#\hfil& \vrule#& \hfil#& \vrule#& \hfil#&
\vrule#\tabskip=0pt\cr\trl && FIRMA && 1 && 2 && 3 && 4 &&
5 && 6 && 7 && 8 && 9 &&  TOT. &\cr\trl && &&   &&
&&     &&   &&   &&   &&   &&   &&    && &\cr &&
\dotfill &&     &&   &&   &&   &&     &&   && && && &&
&\cr\trl }}}
\medskip

\item{1.} Determinare tutte le soluzioni in ${\bf R}$ e in ${\bf C}$ dell'equazione
$z^6=3$.\ve\vs    %foglio1
\item{2.} Determinare tutte le soluzioni del sistema di congruenze:
$\left\{\matrix{X\equiv 3\bmod 5\cr X\equiv 2\bmod 7}\right.$ nell'intervallo $[10,100]$.\vv
\item{3.} Sia $F_n$ l'$n$--esimo numero di Fibonacci (cio\`e $F_0=1$, $F_1=1$ e $F_n=F_{n-1}+F_{n-2}$).
Dimostrare per induzione (forte) che $F_n>(5/4)^n$ per ogni $n\in{\bf N}$, $n\geq2$. 
%Dimostrare che dati due interi $a,b$ tali che $b\neq0$, esistono interi $q$ e $r$ tali che
%$\left\{\matrix{a=bq+r\cr 0\leq r<b.}\right.$

\ve\vs   %foglio2
\item{4.} Calcolare il massimo comun divisore $(105,39)$ e la relativa identit\`a di Bezout.\ve\vs     %foglio3
\item{5.} Enunciare e dimostrare il piccolo Teorema di Fermat.\vv
\item{6.} Dopo aver definito la nozione di campo, dimostrare esplicitamente che l'anello ${\bf Z}/6{\bf Z}$
non \`e un campo.\ve\vs    %foglio4
\item{7.} Consideriamo le seguenti permutazioni in $S_7$:
$$\sigma=\left({1\ 2\ 3\ 4\ 5\ 6\ 7\atop 2\ 5\ 1\ 4\ 7\ 6\ 3}\right)\qquad{\rm e}\qquad 
\tau=\left({1\ 2\ 3\ 4\ 5\ 6\ 7\atop 3\ 2\ 4\ 1\ 7\ 6\ 5}\right).$$ 
\itemitem{a.} Esprimere $\sigma$ e $\tau$ come il prodotto di cicli disgiunti.\medskip
\itemitem{b.} Calcolare la parit\`a di $\sigma$ e di $\tau$.\medskip
\itemitem{c.} Calcolare $\sigma^2\cdot\tau$, $\tau^5$, $\sigma^{-1}$.\ve \vs      %foglio5
\item{8.} Dopo aver dato la definizione di gruppo, si dia un esempio di gruppo abeliano 
infinito e una di gruppo non abeliano finito.\vv
\item{9.} Sia $\varphi$ la funzione di Eulero. 
Dopo averla definita, calcolare il calore $\varphi(60000)$ spiegando i dettaglio
le propriet\`a utilizzate per svolgere il calcolo.  %foglio 6 
\bye
