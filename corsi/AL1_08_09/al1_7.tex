\documentclass[italian,a4paper,11pt]
{article}
\usepackage{babel,amsmath,amssymb,amsbsy,amsfonts,latexsym,exscale,
amsthm,epsf,colordvi,enumerate}

\usepackage[latin1]{inputenc}
\usepackage[all]{xy}
\usepackage{textcomp}
\usepackage{graphicx}


\newcommand{\Q}{\mathbb{Q}}
\newcommand{\Z}{\mathbb Z}
\newcommand{\R}{\mathbb{R}}

\newcommand{\F}{\mathbb{F}}
\newcommand{\N}{\mathbb{N}}
\newcommand{\C}{\mathbb{C}}
\newcommand{\T}{\mathcal{T}}

\newcommand{\U}{\mathcal{U}}
\newcommand{\p}{\mathfrak{p}}
\newcommand{\ga}{\mathfrak{a}}
\newcommand{\gb}{\mathfrak{b}}

\newcommand{\q}{\mathfrak{q}}
\newcommand{\m}{\mathfrak{m}}
\newcommand{\X}{\mathbf{X}}

\newcommand{\D}{\mbox{\rm{\textbf{Dom}}}}
\newcommand{\Ze}{\mbox{\rm{\textbf{Rie}}}}

\newcommand{\esse}{\mbox{\rm{\textbf{Spec}}}}
\newcommand{\Ci}{\mathbf{C}}
\newcommand{\Ex}{\textbf{Esercizio}}


\newcommand{\Sse}{\Longleftrightarrow}
\newcommand{\sse}{\Leftrightarrow}
\newcommand{\implica}{\Rightarrow}

\newcommand{\frecdl}{\longrightarrow}
\newcommand{\frecd}{\rightarrow}
\newcommand{\st}{\scriptstyle}

\begin{document}
\begin{center}


\textbf{Universit\`a degli Studi Roma Tre}\\

\textbf{Corso di Laurea in Matematica, a.a. 2008/2009}\\

\textbf{AL1 - Algebra 1: Fondamenti}\\

\textbf{Prof. F. Pappalardi}\\

\textbf{Tutorato 7 - 27 Novembre 2008}\\

\textbf{Elisa Di Gloria, Luca Dell'Anna}\\

www.matematica3.com\\
\end{center}



\vspace{0.5cm}




\noindent
\begin{Ex}\textbf{ 1.}\\
Utilizzando l'algoritmo euclideo delle divisioni successive, calcolare il MCD delle seguenti coppie di numeri:
\begin{itemize}
	\item (22,14);
	\item (54,77);
	\item (117,99);
	\item (2342, 1764);
	\item (5577,7755);
	\item (2008,2080);
	\item (2178,1221).
\end{itemize}
\end{Ex}

\vspace{0.4cm}
\noindent
\begin{Ex}\textbf{ 2.}\\
Se MCD$(a,b)=d$ e $\lambda$ e $\mu$ formano una coppia di interi soddisfacente l'identit\`a di B\'ezout $d=\lambda a + \mu b$, provare che MCD$(\lambda,\mu)=1$.
\end{Ex}

\vspace{0.4cm}
\noindent
\begin{Ex}\textbf{ 3.}\\
Dimostrare che se $a$ e $b$ sono due interi primi tra loro e tali che $ab$ \`e un quadrato, allora $a$ e $b$ sono quadrati.
\end{Ex}


\vspace{0.4cm}
\noindent
\begin{Ex}\textbf{ 4.}\\
Calcolare MCD e la coppia $(\lambda,\mu)$ dell'identit\`a di B\'ezout tra le seguenti coppie di numeri:
\begin{itemize}
	\item (72,120);
	\item (300,497);
	\item (5865,4416);
	\item (1215,510).
\end{itemize}
\end{Ex}

\vspace{0.4cm}
\noindent
\begin{Ex}\textbf{ 5.}\\
Dimostrare che MCD$(a,b)=a$ $\Sse$ mcm$(a,b)=b$ $\Sse$ $a\mid b$.
\end{Ex}

\vspace{0.4cm}
\noindent
\begin{Ex}\textbf{ 6.}\\
Utilizzando l'algoritmo euclideo delle divisioni successive, stabilire che:
\begin{description}
	\item [(a)] ogni numero intero dispari \`e della forma $4k + 1$, $4k + 3$, con $k\in \Z$;
	\item [(b)] il quadrato di ogni numero intero \`e della forma $3k$, $3k + 1$, con $k\in \Z$;
	\item [(c)] il cubo di ogni numero intero \`e della forma $9k$ oppure $9k + 1$ oppure $9k + 8$, con $k\in \Z$.
\end{description}
\end{Ex}

\vspace{0.4cm}
\noindent
\begin{Ex}\textbf{ 7.}\\
Utilizzando l'algoritmo euclideo delle divisioni successive:
\begin {itemize}
	\item Calcolare MCD$(2424,772)$ e un'identit\`a di B\'ezout;
	\item Provare che MCD$(a,b)\mid (a-b)$;
	\item Usando il punto precedente, trovare MCD$(1962,1965)$ e MCD$(1961,1965)$.
\end{itemize}
\end{Ex}

\end{document}
