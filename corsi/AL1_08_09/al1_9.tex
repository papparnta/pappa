\documentclass[italian,a4paper,11pt]
{article}
\usepackage{babel,amsmath,amssymb,amsbsy,amsfonts,latexsym,exscale,
amsthm,epsf,colordvi,enumerate}

\usepackage[latin1]{inputenc}
\usepackage[all]{xy}
\usepackage{textcomp}
\usepackage{graphicx}


\newcommand{\Q}{\mathbb{Q}}
\newcommand{\Z}{\mathbb Z}
\newcommand{\R}{\mathbb{R}}

\newcommand{\F}{\mathbb{F}}
\newcommand{\N}{\mathbb{N}}
\newcommand{\C}{\mathbb{C}}
\newcommand{\T}{\mathcal{T}}

\newcommand{\U}{\mathcal{U}}
\newcommand{\p}{\mathfrak{p}}
\newcommand{\ga}{\mathfrak{a}}
\newcommand{\gb}{\mathfrak{b}}

\newcommand{\q}{\mathfrak{q}}
\newcommand{\m}{\mathfrak{m}}
\newcommand{\X}{\mathbf{X}}

\newcommand{\D}{\mbox{\rm{\textbf{Dom}}}}
\newcommand{\Ze}{\mbox{\rm{\textbf{Rie}}}}

\newcommand{\esse}{\mbox{\rm{\textbf{Spec}}}}
\newcommand{\Ci}{\mathbf{C}}
\newcommand{\Ex}{\textbf{Esercizio}}


\newcommand{\Sse}{\Longleftrightarrow}
\newcommand{\sse}{\Leftrightarrow}
\newcommand{\implica}{\Rightarrow}

\newcommand{\frecdl}{\longrightarrow}
\newcommand{\frecd}{\rightarrow}
\newcommand{\st}{\scriptstyle}

\begin{document}
\begin{center}


\textbf{Universit\`a degli Studi Roma Tre}\\

\textbf{Corso di Laurea in Matematica, a.a. 2008/2009}\\

\textbf{AL1 - Algebra 1: Fondamenti}\\

\textbf{Prof. F. Pappalardi}\\

\textbf{Tutorato 9 - 11 Dicembre 2008}\\

\textbf{Elisa Di Gloria, Luca Dell'Anna}\\

www.matematica3.com\\
\end{center}



\vspace{0.5cm}




\noindent
\begin{Ex}\textbf{ 1.}\\
Trovare le eventuali soluzioni dei seguenti sistemi di congruenze:
\begin{itemize}

\item $\begin{cases} x\equiv 4 \textrm{  (mod 5)} \\ x\equiv 5 \textrm{  (mod 6)} \\ x\equiv 6 \textrm{  (mod 7)} \end{cases} $

\item $  \begin{cases} x\equiv 2 \textrm{  (mod 5)} \\ x\equiv 5 \textrm{  (mod 6)} \\ x\equiv 5 \textrm{  (mod 12)} \end{cases} $

\item $  \begin{cases} x\equiv 1 \textrm{  (mod 3)} \\ x\equiv 2 \textrm{  (mod 4)} \\ x\equiv 3 \textrm{  (mod 6)}\end{cases} $

\item $  \begin{cases} x\equiv -1 \textrm{  (mod 9)} \\ x\equiv 1 \textrm{  (mod 5)} \\ x\equiv -2 \textrm{  (mod 4)} \\ x\equiv 2 \textrm{  (mod 7)}\end{cases} $

\item $  \begin{cases} x\equiv 11 \textrm{  (mod 19)} \\ x\equiv 7 \textrm{  (mod 8)} \\ x\equiv 10 \textrm{  (mod 6)}\end{cases} $

\end{itemize}
\end{Ex}

\vspace{0.4cm}
\noindent
\begin{Ex}\textbf{ 2.}\\
Sia $p$ un numero primo, $p>3$ tale che $p + 2$ \`e primo. Provare che 12 divide $p + (p + 2)$.
\end{Ex}

\vspace{0.4cm}
\noindent
\begin{Ex}\textbf{ 3.}\\
Trovare tutte le soluzioni in $\C$ dell'equazione $x^{10} = 1$.
\end{Ex}

\vspace{1cm}
\noindent
\begin{Ex}\textbf{ 4.}\\
Trovare in ciascuno dei seguenti casi una scelta per gli $a_i$ tali che i seguenti sistemi non sono risolubili e una, non banale, per i quali lo sono.
\begin{itemize}
\item $  \begin{cases} x\equiv a_1 \textrm{  (mod 5)} \\ x\equiv a_2 \textrm{  (mod 6)} \\ x\equiv a_3 \textrm{  (mod 12)} \end{cases} $
\item $  \begin{cases} x\equiv a_1 \textrm{  (mod 15)} \\ x\equiv a_2 \textrm{  (mod 11)} \\ x\equiv a_3 \textrm{  (mod 4)} \\ x\equiv a_4 \textrm{  (mod 6)}\end{cases} $

\item $  \begin{cases} x\equiv a_1 \textrm{  (mod 19)} \\ x\equiv a_2 \textrm{  (mod 8)} \\ x\equiv a_3 \textrm{  (mod 6)}\end{cases} $

\end{itemize}
\end{Ex}


\vspace{0.4cm}
\noindent
\begin{Ex}\textbf{ 5.}\\
Usando il crivello di Eratostene, stabilire se i seguenti numeri sono primi:\\
167, 253, 137, 151, 1001

\end{Ex}

\vspace{0.4cm}
\noindent
\begin{Ex}\textbf{ 6.}\\ 
Determinare, per $2\leq m \leq 10$, la tavola additiva di $\frac{\Z}{m\Z}$.\\
Determinare inoltre, $\forall m$ come sopra, il gruppo degli invertibili $\mathcal{U}(\Z_m)$ e scriverne la tavola moltiplicativa.
\end{Ex}


\vspace{0.4cm}
\noindent
\begin{Ex}\textbf{ 7.}\\ 
Scrivere una dimostrazione del fatto che $\varphi(p^{\alpha})=p^{\alpha}-p^{\alpha -1}$ se $p$ \`e primo.
\end{Ex}

\vspace{0.4cm}
\noindent
\begin{Ex}\textbf{ 8.}\\
Un fruttivendolo deve sistemare poco meno di un migliaio di arance sui suoi banconi, ma disponendole a gruppi di 3 gli avanzano 2 frutti, a gruppi di 4 gliene avanzano 3, a gruppi di 5 ne rimangono 4, a gruppi di 6 ne rimangono 5. Finalmente riesce a sistemarle a gruppi di 7. Quante sono le arance?
\end{Ex}

\vspace{0.4cm}
\noindent
\begin{Ex}\textbf{ 9.}\\
Il nonno Mario ha tre nipotine, Adele, Bice e Clara, rispettivamente di 17,16 e 4 anni. Il Nonno dice ad Adele:"Per ottenere la mia et\`a occorre un multiplo della tua pi\`u quella di Bice". Poi, rivolto a Bice, afferma:"Per ottenere la mia et\` occorre un multiplo della tua et\`a pi\`u quella di Clara". Infine dice a Clara:"Sai che la mia et\`a \`e proprio un multiplo della tua?"\\
Qual \`e l'et\`a di Nonno Mario?
\end{Ex}

\end{document}
