\nopagenumbers \font\title=cmti12
\def\ve{\vfill\eject}
\def\vv{\vfill}
\def\vs{\vskip-2cm}
\def\vss{\vskip10cm}
%\def\vst{\vskip13.3cm}

%\def\ve{\bigskip\bigskip}
%\def\vv{\bigskip\bigskip}
%\def\vs{}
%\def\vss{}
%\def\vst{\bigskip\bigskip}

\vfuzz=271pt
\hfuzz=3cm
\hsize=19cm
\vsize=27.58cm
\hoffset=-1.6cm
\voffset=0.5cm
\parskip=-.1cm
\ \vs \hskip -6mm AL1 AA08/09\ (Algebra: fondamenti)\hfill Appello A \hfill Roma, 19 Gennaio 2009. \hrule
\bigskip\noindent {\title COGNOME}\  \dotfill\ {\title NOME}\ \dotfill {\title
MATRICOLA}\ \dotfill\
\smallskip  \noindent
Risolvere il massimo numero di esercizi accompagnando le risposte
con spiegazioni chiare ed essenziali. \it Inserire le risposte
negli spazi predisposti. NON SI ACCETTANO RISPOSTE SCRITTE SU
ALTRI FOGLI.
\rm 1 Esercizio = 4 punti. Tempo previsto: 2 ore. Nessuna domanda
durante la prima ora e durante gli ultimi 20 minuti.
\smallskip
\hrule\smallskip
\centerline{\hskip 6pt\vbox{\tabskip=0pt \offinterlineskip
\def \trl{\noalign{\hrule}}
\halign to300pt{\strut#& \vrule#\tabskip=0.7em plus 1em& \hfil#&
\vrule#& \hfill#\hfil& \vrule#& \hfil#& \vrule#& \hfill#\hfil&
\vrule#& \hfil#& \vrule#& \hfill#\hfil& \vrule#& \hfil#& \vrule#&
\hfill#\hfil& \vrule#& \hfil#& \vrule#& \hfill#\hfil& \vrule#&
\hfil#& \vrule#& \hfill#\hfil& \vrule#& \hfil#& \vrule#& \hfil#&
\vrule#\tabskip=0pt\cr\trl && FIRMA && 1 && 2 && 3 && 4 &&
5 && 6 && 7 && 8 && 9 &&  TOT. &\cr\trl && &&   &&
&&     &&   &&   &&   &&   &&   &&    && &\cr &&
\dotfill &&     &&   &&   &&   &&     &&   && && && &&
&\cr\trl }}}
\medskip

\item{1.} %APPLICAZIONI
Dopo aver descritto la nozione di iniettivit\`a e 
suriettivit\`a per applicazioni di insiemi, dire se (e perch\`e) la seguente funzione \`e iniettiva, suriettiva,
se ne descriva l'immagine e si descriva la preimmagine degli elementi del codominio:\hfill\break 
$f:\{1,2,3\}\times\{1,2,3,4\}\rightarrow\{1,2,3,\ldots,12\}, (x,y)\mapsto xy$\ve\vs    %foglio1

\item{2.} Dopo aver descritto la nozione di relazione di equivalenza, si dimostri che la 
seguente relazione su ${\bf Z}$: $x\rho y \Longleftrightarrow xy\in{\bf N}$
\`e di equivalenza e se ne descrivano le classi di equivalenza  %RELAZIONI DI EQUIVALENZA
\vv
\item{3.} %PRINCIPIO DI INDUZIONE
Si dimostri per induzione che per ogni $n\in{\bf N}_>$ i numeri
$2^{2^n}+3^{2^n}+5^{2^n}$ e $6^{2^n}+10^{2^n}+15^{2^n}$ sono divisibili per $19$.\ve\vs   %foglio2

\item{4.} Dopo aver enunciato i cinque assiomi di Peano, si dimostri che \`e possibile costruire
un sistema finito che soddisfa solo $4$ dei $5$ assiomi. 
%ASSIOMI DI PEANO, INSIEMI INFINITI, RELAZIONI D'ORDINE
\ve\vs     %foglio3

\item{5.} Si descrivano tutte le soluzioni della seguente equazione congruenziale:
$6X\equiv 9\bmod15$.%Congruenze Lineari/TCDR
\vv
\item{6.} Dimostrare che l'insieme ${\bf N}[X]$ dei polinomi a coefficienti in ${\bf N}$ \`e un
monoide additivo rispetto alla somma ma non \`e un gruppo. Quali degli altri assiomi di anello soddisfa?
%Gruppi e/o Campi.
\ve\vs    %foglio4
\item{7.} %Numeri Complessi.
Determinare la parte reale e la parte immaginaria del seguente numero complesso:
${3+i\over2-i}+(1+i)^5.$
\ve \vs      %foglio5
\item{8.} Determinare gli elementi invertibili rispetto al prodotto dell'anello ${\bf Z}/16{\bf Z}$ e 
si determini l'inverso di ciascuno di essi. %Funzione di Eulero/ P T Fermat.
\vv
\item{9.} Determinare il numero di permutazioni in $S_6$ che si pu\`o scrivere come il prodotto di 
due $3$--cicli disgiunti e dimostrare che l'insieme di tali permutazioni non \`e chiuso rispetto alla composizione.
%Permutazioni.
%\ \vst %foglio 6 
\bye
