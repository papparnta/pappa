\documentclass[italian,a4paper,11pt]
{article}
\usepackage{babel,amsmath,amssymb,amsbsy,amsfonts,latexsym,exscale,
amsthm,epsf,colordvi,enumerate}

\usepackage[latin1]{inputenc}
\usepackage[all]{xy}
\usepackage{textcomp}
\usepackage{graphicx}


\newcommand{\Q}{\mathbb{Q}}
\newcommand{\Z}{\mathbb Z}
\newcommand{\R}{\mathbb{R}}

\newcommand{\F}{\mathbb{F}}
\newcommand{\N}{\mathbb{N}}
\newcommand{\C}{\mathbb{C}}
\newcommand{\T}{\mathcal{T}}

\newcommand{\U}{\mathcal{U}}
\newcommand{\p}{\mathfrak{p}}
\newcommand{\ga}{\mathfrak{a}}
\newcommand{\gb}{\mathfrak{b}}

\newcommand{\q}{\mathfrak{q}}
\newcommand{\m}{\mathfrak{m}}
\newcommand{\X}{\mathbf{X}}

\newcommand{\D}{\mbox{\rm{\textbf{Dom}}}}
\newcommand{\Ze}{\mbox{\rm{\textbf{Rie}}}}

\newcommand{\esse}{\mbox{\rm{\textbf{Spec}}}}
\newcommand{\Ci}{\mathbf{C}}
\newcommand{\Ex}{\textbf{Esercizio}}


\newcommand{\Sse}{\Longleftrightarrow}
\newcommand{\sse}{\Leftrightarrow}
\newcommand{\implica}{\Rightarrow}

\newcommand{\frecdl}{\longrightarrow}
\newcommand{\frecd}{\rightarrow}
\newcommand{\st}{\scriptstyle}

\begin{document}
\begin{center}


\textbf{Universit\`a degli Studi Roma Tre}\\

\textbf{Corso di Laurea in Matematica, a.a. 2008/2009}\\

\textbf{AL1 - Algebra 1: Fondamenti}\\

\textbf{Prof. F. Pappalardi}\\

\textbf{Tutorato 8 - 4 Dicembre 2008}\\

\textbf{Elisa Di Gloria, Luca Dell'Anna}\\

www.matematica3.com\\
\end{center}



\vspace{0.5cm}




\noindent
\begin{Ex}\textbf{ 1.}\\
Dimostrare che se $a$ e $b$ sono due interi primi tra loro e tali che $ab$ \`e un quadrato, allora $a$ e $b$ sono quadrati.
\end{Ex}

\vspace{0.4cm}
\noindent
\begin{Ex}\textbf{ 2.}\\
Dimostrare che se $a,b,c \in \Z$, $a$ \`e coprimo con $bc$ se e solo se $a$ \`e coprimo con $b$ e con $c$.
\end{Ex}

\vspace{0.4cm}
\noindent
\begin{Ex}\textbf{ 3.}\\
Scrivere i seguenti numeri come classi di resto modulo $2,5,7,10,13,18$:\\
21, 18, 12, 231, 650, 112, 100.
\end{Ex}


\vspace{0.4cm}
\noindent
\begin{Ex}\textbf{ 4.}\\
Risolvere, se possibile, le seguenti congruenze:
\begin{itemize}
	\item $5x\equiv 3$ (mod 7)
	\item $6x\equiv 15$ (mod 9)
	\item $64x\equiv 24$ (mod 20)
	\item $21x\equiv 7$ (mod 8)
	\item $37x\equiv 25$ (mod 117)
	\item $18x\equiv 5$ (mod 51)
	\item $144x\equiv 48$ (mod 120)
\end{itemize}
\end{Ex}

\vspace{0.4cm}
\noindent
\begin{Ex}\textbf{ 5.}\\
Dire quali dei seguenti sono gruppi, semigruppi o monoidi:
\begin{itemize}
	\item $G=\{k\in \Z \  |\  k=5h\  \exists h\in \Z\}$ con l'operazione di somma  
	\item $G=\{m\in \N \ | \ m>7\}$ con l'operazione di somma.
	\item $(\Z,+)$
	\item $G=\{z\in \C \ | \ |z|=1\}$ con l'operazione di moltiplicazione tra numeri complessi. 
\end{itemize}
\end{Ex}

\vspace{0.4cm}
\noindent
\begin{Ex}\textbf{ 6.}\\ 
Determinare, usando la fattorizzazione in primi, il mcm e MCD tra le seguenti coppie di numeri:
\begin{itemize}
	\item (48,14)
	\item (2292,1120)
	\item (132,84)
	\item (154,308)
	\item (59,528)
	\item (168,273)
\end{itemize}
\end{Ex}


\end{document}
