\documentclass[italian,a4paper,11pt]
{article}
\usepackage{babel,amsmath,amssymb,amsbsy,amsfonts,latexsym,exscale,
amsthm,epsf,colordvi,enumerate}

\usepackage[latin1]{inputenc}
\usepackage[all]{xy}
\usepackage{textcomp}
\usepackage{graphicx}


\newcommand{\Q}{\mathbb{Q}}
\newcommand{\Z}{\mathbb Z}
\newcommand{\R}{\mathbb{R}}

\newcommand{\F}{\mathbb{F}}
\newcommand{\N}{\mathbb{N}}
\newcommand{\C}{\mathbb{C}}
\newcommand{\T}{\mathcal{T}}

\newcommand{\U}{\mathcal{U}}
\newcommand{\p}{\mathfrak{p}}
\newcommand{\ga}{\mathfrak{a}}
\newcommand{\gb}{\mathfrak{b}}

\newcommand{\q}{\mathfrak{q}}
\newcommand{\m}{\mathfrak{m}}
\newcommand{\X}{\mathbf{X}}

\newcommand{\D}{\mbox{\rm{\textbf{Dom}}}}
\newcommand{\Ze}{\mbox{\rm{\textbf{Rie}}}}

\newcommand{\esse}{\mbox{\rm{\textbf{Spec}}}}
\newcommand{\Ci}{\mathbf{C}}
\newcommand{\Ex}{\textbf{Esercizio}}


\newcommand{\Sse}{\Longleftrightarrow}
\newcommand{\sse}{\Leftrightarrow}
\newcommand{\implica}{\Rightarrow}

\newcommand{\frecdl}{\longrightarrow}
\newcommand{\frecd}{\rightarrow}
\newcommand{\st}{\scriptstyle}

\begin{document}
\begin{center}


\textbf{Universit\`a degli Studi Roma Tre}\\

\textbf{Corso di Laurea in Matematica, a.a. 2008/2009}\\

\textbf{AL1 - Algebra 1: Fondamenti}\\

\textbf{Prof. F. Pappalardi}\\

\textbf{Tutorato 6 - 20 Novembre 2008}\\

\textbf{Elisa Di Gloria, Luca Dell'Anna}\\

www.matematica3.com\\
\end{center}



\vspace{0.5cm}




\noindent
\begin{Ex}\textbf{ 1.}\\
Siano $z,w\in \C$,  $n\in \N$, mostrare che
\begin{itemize}
	\item $|\overline{z}|=|z|$;
	\item $\frac{1}{i}=-i$;
	\item $|z^n|=|z|^n$;
	\item $|\overline{z}^n|=|z|^n$;
	\item $\overline{z+w}=\overline{z}+\overline{w}$;
	\item $\overline{\overline{z}}=z$;
	\item $\overline{z\cdot w}=\overline{z}\cdot \overline{w}$;
	\item $z=\overline{z}\ \ \ \iff \ z \in \R$;
	\item $\arg{\overline{z}}=- \arg{z}$;
	\item $\arg{z^n}=n\arg{z}$;
	\item $|z\cdot w|=|z|\cdot |w|$;
	\item $z\cdot \overline{z} \in \R_{\geq 0}$.
\end{itemize}
\end{Ex}

\vspace{0.4cm}
\noindent
\begin{Ex}\textbf{ 2.}\\
Calcolare norma, modulo e argomento dei seguenti numeri complessi
\begin{itemize}
	\item $i$;
	\item $1-i$;
	\item $\frac{i}{2}$;
	\item $\frac{\sqrt{2}}{2}+i\frac{\sqrt{2}}{2}$;
	\item $1+i+i^2+i^3+i^4+i^5$;
	\item $(1+i)^2$;
	\item $(1+i)(1-i)$;
	\item $e^{i\theta}$ con $\theta \in \R$;
	\item $2-3i$.
\end{itemize}
\end{Ex}

\vspace{0.4cm}
\noindent
\begin{Ex}\textbf{ 3.}\\
Svolgere i seguenti calcoli, calcolare poi inverso e scrittura in forma trigonometrica del risultato
\begin{itemize}
	\item $(2+i)(4+2i)+4i$;
	\item $(1+i)^2$;
	\item $-i(\frac{1}{2} + \frac{3i}{2})$;
	\item $(\frac{\sqrt{3}}{2}+\frac{i}{2})^{3600}$.
\end{itemize}
\end{Ex}


\vspace{0.4cm}
\noindent
\begin{Ex}\textbf{ 4.}\\
Esprimere in forma trigonometrica i seguenti numeri complessi:
$5$, $-1+3i$, $-6$, $-3+i\sqrt{3}$, $\frac{1}{2i}$, $\frac{-1}{1+2i}$.
\end{Ex}

\vspace{0.4cm}
\noindent
\begin{Ex}\textbf{ 5.}\\
Calcolare:
\begin{itemize}
	\item $(1-2i)(2+3i)^{-1}$;
	\item $(i)^{49}$, $(-i)^{58}$;
	\item $\frac{(21-3i)+(5-6i)}{(1+i)-(8i+3)}$;
	\item $(\frac{\sqrt{3i}}{2}+\frac{2i}{6})$.
\end{itemize}
\end{Ex}

\vspace{0.4cm}
\noindent
\begin{Ex}\textbf{ 6.}\\
Dimostrare che la somma tra numeri complessi definita come:\\$(a+ib)+(c+id):=(a+c)+i(b+d)$\\ \`e ben definita, commutativa e associativa. Verificare inoltre la propriet\`a distributiva rispetto al prodotto.\\
Di quali propriet\`a gode il prodotto? Ricordiamo la definizione di prodotto tra numeri complessi,\\    $\displaystyle(a+ib)(c+id):=(ac-bd)+i(ad+bc)$.
\end{Ex}

\end{document}
