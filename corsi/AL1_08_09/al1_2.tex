\documentclass[italian,a4paper,11pt]
{article}
\usepackage{babel,amsmath,amssymb,amsbsy,amsfonts,latexsym,exscale,
amsthm,epsf,colordvi,enumerate}

\usepackage[latin1]{inputenc}
\usepackage[all]{xy}
\usepackage{textcomp}
\usepackage{graphicx}


\newcommand{\Q}{\mathbb{Q}}
\newcommand{\Z}{\mathbb Z}
\newcommand{\R}{\mathbb{R}}

\newcommand{\F}{\mathbb{F}}
\newcommand{\N}{\mathbb{N}}
\newcommand{\C}{\mathbb{C}}
\newcommand{\T}{\mathcal{T}}

\newcommand{\U}{\mathcal{U}}
\newcommand{\p}{\mathfrak{p}}
\newcommand{\ga}{\mathfrak{a}}
\newcommand{\gb}{\mathfrak{b}}

\newcommand{\q}{\mathfrak{q}}
\newcommand{\m}{\mathfrak{m}}
\newcommand{\X}{\mathbf{X}}

\newcommand{\D}{\mbox{\rm{\textbf{Dom}}}}
\newcommand{\Ze}{\mbox{\rm{\textbf{Rie}}}}

\newcommand{\esse}{\mbox{\rm{\textbf{Spec}}}}
\newcommand{\Ci}{\mathbf{C}}
\newcommand{\Ex}{\textbf{Esercizio}}


\newcommand{\Sse}{\Longleftrightarrow}
\newcommand{\sse}{\Leftrightarrow}
\newcommand{\implica}{\Rightarrow}

\newcommand{\frecdl}{\longrightarrow}
\newcommand{\frecd}{\rightarrow}
\newcommand{\st}{\scriptstyle}

\begin{document}

\begin{center}

\textbf{Universit\`a degli Studi Roma Tre}\\

\textbf{Corso di Laurea in Matematica, a.a. 2008/2009}\\

\textbf{AL1 - Algebra 1: Fondamenti}\\

\textbf{Prof. F. Pappalardi}\\

\textbf{Tutorato 2 - 16 Ottobre 2008}\\

\textbf{Elisa Di Gloria, Luca Dell'Anna}\\

www.matematica3.com\\
\end{center}



\vspace{0.5cm}




\noindent
\begin{Ex}\textbf{ 1.}\\
Sia $f: \N \frecdl \N$ definita dalla legge $f(x)=\frac{x^2+x}{a}$ con $a\in \N$. Per quali valori di $a$ essa \'e un'applicazione? E in tali casi \'e iniettiva?
\end{Ex}

\vspace{0.4cm}
\noindent
\begin{Ex}\textbf{ 2.}\\
Stabilire se le seguenti applicazioni sono iniettive e/o suriettive e calcolarne, quando possibile, l'inversa.
\begin{itemize}
	\item $f:\Q$ $\longrightarrow$ $\Q$\\
			$x \longmapsto \frac{x-3}{2}$
	\item $f:\N$ $\longrightarrow$ $\Z$\\ definita da 
	$f(x)=\left\{ \begin{matrix} 
	\frac{x}{2}-1, & x\text{ pari} \\ 
	-\frac{x+1}{2}, & x\text{ dispari}
	\end{matrix}\right.$ 
	\item $f:\R_{>0}$ $\longrightarrow$ $\R_{>0}$\\
			$x \longmapsto \frac{1}{\sqrt{x^2+x}}$
        \item $f:\R$ $\longrightarrow$ $\Z$\\
			$x \longmapsto [x]$	ove con  $[x]$ si indica la parte intera di $x$
\end{itemize}
Trovare in ognuno dei casi precedenti le controimmagini $f^{-1}(15)$.
\end{Ex}

\vspace{0.4cm}
\noindent
\begin{Ex}\textbf{ 3.}\\
Se $\varphi : A$ $\longrightarrow$ $B$ \'e un'applicazione ed $A_1,A_2$ sono sottoinsiemi di $A$, posto $\varphi(A_1)=B_1\supseteq B$ e $\varphi(A_2)=B_2\supseteq B$, provare che:
\begin{description}
	\item[i)] $\varphi(A_1 \cup A_2)=\varphi(A_1) \cup \varphi(A_2)$
	\item[ii)] $\varphi(A_1 \cap A_2)\subseteq\varphi(A_1) \cap \varphi(A_2)$.
   	\item[iii)] $\varphi^{-1}(B_1 \cup B_2)=\varphi^{-1}(B_1) \cup \varphi^{-1}(B_2)$
	\item[iv)] $\varphi^{-1}(B_1 \cap B_2)=\varphi^{-1}(B_1) \cap \varphi^{-1}(B_2)$.
\end{description}
Portare un esempio che illustri come, in generale, in \textbf{ii)} non valga l'uguaglianza.
\end{Ex}

\vspace{0.4cm}
\noindent
\begin{Ex}\textbf{ 4.}\\
Dire quali delle seguenti leggi sono applicazioni ed, in caso affermativo, se esse sono iniettive e/o suriettive:
\begin{itemize}
	\item $f:\N$ $\longrightarrow$ $\N \cup \{0\}$\\
			$x \longmapsto \frac{x^3-2x}{6}$
	\item $f:\Z$ $\longrightarrow$ $\Q$\\
			$x \longmapsto \frac{x^2-2}{x+2}$	
	\item $f:\R^*$ $\longrightarrow$ $\R$\\
			$x \longmapsto \frac{2}{x}$	
	\item $f:\R$ $\longrightarrow$ $\R_{\geq 0}$\\
			$x \longmapsto (x-1)^4$
\end{itemize}
Per ognuna delle precedenti applicazioni descrivere l'immagine e determinare la controimmagine $f^{-1}(1)$.
\end{Ex}



\vspace{0.4cm}
\noindent
\begin{Ex}\textbf{ 5.}\\
Provare per induzione che:
\begin{itemize}
	\item $\displaystyle \sum_{k=1}^n (-1)^k k^2 =\frac{(-1)^n n(n+1)}{2}$.
	\item $\displaystyle \sum_{k=1}^n \frac{1}{4k^2-1} =\frac{n}{2n+1}$.
	\item $n!>n^2$		$\forall n\geq 4$;	
	\item $n!>n^3$		$\forall n\geq 6$;
\end{itemize}
\end{Ex}

\vspace{0.4cm}
\noindent
\begin{Ex}\textbf{ 6.}\\
$Momento$ $relax$:\\
Trovare l'errore nel seguente procedimento di induzione usato per provare l'affermazione: $\mathit{Tutti\ gli\ uomini\ hanno\ lo\ stesso\ nome}$.\\
\'E chiaro che basta provare che in ogni insieme di $n$ uomini, con $n\in \N$, essi hanno lo stesso nome.\\
La proposizione \'e vera per insiemi con un solo uomo; supponiamo allora che essa sia vera per insiemi con $n-1$ uomini e proviamola per un insieme con $n$ uomini. Sia allora $I_n=\{U_1,U_2,\dots,U_n\}$ un insieme costituito da $n$ uomini, allora $J=\{U_1,U_2,\dots,U_{n-1}\}$, essendo formato da $n-1$ uomini, per l'ipotesi induttiva, sar\'a tale che $U_1,U_2,\dots,U_{n-1}$ hanno lo stesso nome, diciamo \textbf{Franco}; d'altra parte, anche $K=\{U_1,U_2,\dots,U_{n-2}, U_n\}$ \'e formato da $n-1$ elementi, e per lo stesso motivo anche $U_1,U_2,\dots,U_{n-2}, U_n$ hanno lo stesso nome, ma essendo \textbf{Franco} il nome di $U_1$ ne segue anche che $U_n$ si chiama \textbf{Franco}. Allora tutti gli uomini in $I_n$ hanno lo stesso nome. Per il principio di induzione,allora, tutti gli uomini si chiamano \textbf{Franco}!!!!
\end{Ex}

\end{document}
