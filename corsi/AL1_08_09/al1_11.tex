\documentclass[italian,a4paper,11pt]
{article}
\usepackage{babel,amsmath,amssymb,amsbsy,amsfonts,latexsym,exscale,
amsthm,epsf,colordvi,enumerate}

\usepackage[latin1]{inputenc}
\usepackage[all]{xy}
\usepackage{textcomp}
\usepackage{graphicx}


\newcommand{\Q}{\mathbb{Q}}
\newcommand{\Z}{\mathbb Z}
\newcommand{\R}{\mathbb{R}}

\newcommand{\F}{\mathbb{F}}
\newcommand{\N}{\mathbb{N}}
\newcommand{\C}{\mathbb{C}}
\newcommand{\T}{\mathcal{T}}

\newcommand{\U}{\mathcal{U}}
\newcommand{\p}{\mathfrak{p}}
\newcommand{\ga}{\mathfrak{a}}
\newcommand{\gb}{\mathfrak{b}}

\newcommand{\q}{\mathfrak{q}}
\newcommand{\m}{\mathfrak{m}}
\newcommand{\X}{\mathbf{X}}

\newcommand{\D}{\mbox{\rm{\textbf{Dom}}}}
\newcommand{\Ze}{\mbox{\rm{\textbf{Rie}}}}

\newcommand{\esse}{\mbox{\rm{\textbf{Spec}}}}
\newcommand{\Ci}{\mathbf{C}}
\newcommand{\Ex}{\textbf{Esercizio}}


\newcommand{\Sse}{\Longleftrightarrow}
\newcommand{\sse}{\Leftrightarrow}
\newcommand{\implica}{\Rightarrow}

\newcommand{\frecdl}{\longrightarrow}
\newcommand{\frecd}{\rightarrow}
\newcommand{\st}{\scriptstyle}

\begin{document}
\begin{center}


\textbf{Universit\`a degli Studi Roma Tre}\\

\textbf{Corso di Laurea in Matematica, a.a. 2008/2009}\\

\textbf{AL1 - Algebra 1: Fondamenti}\\

\textbf{Prof. F. Pappalardi}\\

\textbf{Tutorato 11 - 19 Dicembre 2008}\\

\textbf{Elisa Di Gloria, Luca Dell'Anna}\\

www.matematica3.com\\
\end{center}



\vspace{0.5cm}




\noindent
\begin{Ex}\textbf{ 1.}\\
Considerare le seguenti permutazioni:
\begin{itemize}
	\item $\left(\begin{matrix} 1 & 2 & 3 & 4 & 5 & 6 \\ 3 & 5 & 2 & 6 & 4 & 1 \end{matrix} \right )$
	\item $\left(\begin{matrix} 1 & 2 & 3 & 4 & 5 & 6 \\ 2 & 3 & 5 & 4 & 1 & 6 \end{matrix} \right )$
	\item $\left(\begin{matrix} 1 & 2 & 3 & 4 & 5 & 6 & 7 & 8 & 9 \\ 2 & 3 & 4 & 6 & 5 & 7 & 9 & 1 & 8 \end{matrix} \right )$
	\item $\left(\begin{matrix} 1 & 2 & 3 & 4 & 5 & 6 & 7 \\ 2 & 3 & 4 & 1 & 7 & 5 & 6 \end{matrix} \right )$
\end{itemize}
Descriverne il supporto e l'orbita degli elementi 1,4,5.
\end{Ex}

\vspace{0.4cm}
\noindent
\begin{Ex}\textbf{ 2.}\\
Scrivere le seguenti permutazioni come prodotto in cicli disgiunti:
\begin{itemize}
	\item $\left(\begin{matrix} 1 & 2 & 3 & 4 & 5 & 6 & 7 \\ 5 & 1 & 7 & 6 & 2 & 3 & 4 \end{matrix} \right )$
	\item $\left(\begin{matrix} 1 & 2 & 3 & 4 & 5 & 6 & 7 & 8 \\ 5 & 4 & 1 & 6 & 3 & 4 & 8 & 7 \end{matrix} \right )$
	\item $\left(\begin{matrix} 1 & 2 & 3 & 4 & 5 & 6 & 7 & 8 & 9 \\ 2 & 4 & 7 & 5 & 1 & 9 & 8 & 3 & 6 \end{matrix} \right )$
\end{itemize}
Determinarne infine il segno.
\end{Ex}

\vspace{0.4cm}
\noindent
\begin{Ex}\textbf{ 3.}\\
Scrivere le seguenti permutazioni come prodotto in cicli disgiunti e come prodotto di trasposizioni:
\begin{itemize}
\item $\left(\begin{matrix} 1 & 2 & 3 & 4 & 5 & 6 & 7 \\ 2 & 3 & 7 & 4 & 6 & 5 & 1 \end{matrix} \right )$
\item $\left(\begin{matrix} 1 & 2 & 3 & 4 & 5 & 6 & 7 \\ 1 & 7 & 5 & 6 & 3 & 4 & 2 \end{matrix} \right )$
\item $\left(\begin{array}{cccccccccccc} 1 & 2 & 3 & 4 & 5 & 6 & 7 & 8 & 9 & 10 & 11 & 12 \\ 5 & 6 & 7 & 10 & 12 & 9 & 4 & 3 & 11 & 8 & {2} & {1} \end{array} \right )$
\end{itemize}
Di ognuna delle precedenti permutazioni, $\sigma$, calcolare: $\sigma^2$, $\sigma^3$, $\sigma^5$.\\
Che relazione c'\`e tra il segno di una permutazione e il minimo comune multiplo delle lunghezze dei suoi cicli disgiunti?
\end{Ex}


\vspace{0.4cm}
\noindent
\begin{Ex}\textbf{ 4.}\\
Date le seguenti permutazioni $\sigma$ e $\tau$, calcolare i prodotti dove necessario e decomporre in cicli disgiunti $\sigma$, $\tau$, $\sigma\tau$, $\tau\sigma$, $\sigma^2$, $\sigma^2\tau$, $\tau^2$, $\tau^2\sigma$.
$$\sigma=\left(\begin{array}{cccccccccc} 1 & 2 & 3 & 4 & 5 & 6 & 7 & 8 & 9 & 10 \\ 2 & 4 & 5 & 7 & 9 & 8 & 10 & 6 & 3 & 1 \end{array} \right )$$
$$\tau=\left(\begin{array}{cccccccccc} 1 & 2 & 3 & 4 & 5 & 6 & 7 & 8 & 9 & 10 \\ 1 & 3 & 2 & 4 & 5 & 6 & 7 & 8 & 9 & 10 \end{array} \right )$$
\end{Ex}

\vspace{0.4cm}
\noindent
\begin{Ex}\textbf{ 5.}\\
Determinare la struttura dei cicli di $S_5$.
\end{Ex}

\end{document}
