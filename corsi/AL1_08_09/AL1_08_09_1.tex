\nopagenumbers \font\title=cmti12
\def\ve{\vfill\eject}
\def\vv{\vfill}
\def\vs{\vskip-2cm}
\def\vss{\vskip10cm}
\def\vst{\vskip13.3cm}

%\def\ve{\bigskip\bigskip}
%\def\vv{\bigskip\bigskip}
%\def\vs{}
%\def\vss{}
%\def\vst{\bigskip\bigskip}

\hsize=19.5cm
\vsize=27.58cm
\hoffset=-1.6cm
\voffset=0.5cm
\parskip=-.1cm
\ \vs \hskip -6mm AL1 AA08/09\ (Algebra: fondamenti)\hfill ESAME DI MET\`{A} SEMESTRE \hfill Roma, Novembre 2008. \hrule
\bigskip\noindent
{\title COGNOME}\  \dotfill\ {\title NOME}\ \dotfill {\title
MATRICOLA}\ \dotfill\
\smallskip  \noindent
Risolvere il massimo numero di esercizi accompagnando le risposte
con spiegazioni chiare ed essenziali. \it Inserire le risposte
negli spazi predisposti. NON SI ACCETTANO RISPOSTE SCRITTE SU
ALTRI FOGLI. Scrivere il proprio nome anche nell'ultima pagina.
\rm 1 Esercizio = 4 punti. Tempo previsto: 2 ore. Nessuna domanda
durante la prima ora e durante gli ultimi 20 minuti.
\smallskip
\hrule\smallskip
\centerline{\hskip 6pt\vbox{\tabskip=0pt \offinterlineskip
\def \trl{\noalign{\hrule}}
\halign to300pt{\strut#& \vrule#\tabskip=0.7em plus 1em& \hfil#&
\vrule#& \hfill#\hfil& \vrule#& \hfil#& \vrule#& \hfill#\hfil&
\vrule#& \hfil#& \vrule#& \hfill#\hfil& \vrule#& \hfil#& \vrule#&
\hfill#\hfil& \vrule#& \hfil#& \vrule#& \hfill#\hfil& \vrule#&
\hfil#& \vrule#& \hfill#\hfil& \vrule#& \hfil#& \vrule#& \hfil#&
\vrule#\tabskip=0pt\cr\trl && FIRMA && 1 && 2 && 3 && 4 &&
5 && 6 && 7 && 8 && 9 &&  TOT. &\cr\trl && &&   &&
&&     &&   &&   &&   &&   &&   &&    && &\cr &&
\dotfill &&     &&   &&   &&   &&     &&   && && && &&
&\cr\trl }}}
\medskip


%\item{1.}
%Se $n\in{\bf N}$, sia $\varphi(n)$ la funzione di Eulero. Supponiamo che sia nota
%la fattorizzazione (unica) di $n=p_1^{\alpha_1}\cdots p_s^{\alpha_s}$. Stimare il
%numero di operazioni bit necessarie per calcolare $\varphi(n)$.
%\vv
%\item{2.} Stimare in termini di $k$ il numero di operazioni bit necessarie per calcolare $\left[\sqrt{2^{k^k}\bmod 3^k}
%\right]$.
%\vv

\item{1.} Per ogni $n\in{\bf N}$, sia $A_n=[{1\over n},n]\subset{\bf R}$. Determinare
$$\bigcup_{n\in{\bf N}}A_n\qquad{\rm e}\qquad\bigcap_{n\in{\bf N}}{\bf R}\setminus A_n$$ \vv

\item{2.} Dopo aver definito la nozione di partizione di un insieme, si consideri 
$X=\{\alpha,\beta,\gamma\}$ e si descrivano tutte le possibili partizioni di $X$.\ve\vs

\item{3.} Sia $X=\{a,b,c,d\}$, $Y=\{1,2,3\}$.

\itemitem{i.} Fornire un esempio di applicazione iniettiva e non suriettiva $f:Y\rightarrow X$;

\itemitem{ii.} Fornire un esempio di applicazione suriettiva e non iniettiva $f:X\rightarrow Y$;

\itemitem{iii.} Fornire un esempio di applicazione n\'e suriettiva n\'e iniettiva $f:X\rightarrow X$.
\vv

\item{4.} Sia $X$ un insieme e $f:X\rightarrow X$ una funzione tale che $f\circ f\circ f=f$. Dimostrare che $f$ \`{e} iniettiva
se e solo se \`{e} suriettiva. \vv

\item{5.} Dimostrare usando il metodo dell'induzione,
 che per ogni $n\in{\bf N}$ il numero $10^{n+1}+3\cdot 10^{n}+5$ \`{e} divisibile per $9$.
\ve\vs

\item{6.} Dopo aver enunciato gli assiomi di Peano, dimostrare che se
$A=\{n\in{\bf N} | \exists k\in{\bf N}, n=3k\}$ e se $s:A\rightarrow A, n\mapsto n+3$, allora $(A,0,s)$  \`e un sistema di Peano.
\vv

\item{7.} Dopo aver enunciato le tre nozioni di infinit\`a, dimostrare che un insieme $X$ \`e infinito nel senso di Cantor se e solo se \`e in corrispondenza
biunivoca con un suo sottoinsieme proprio.
\ve \vs

\item{8.} Sia $X=\{a,b,c\}$ per ciascuna delle seguenti relazioni stabilire se sono riflessive, simmetriche, antisimmetriche, transitive:
$$R_1=\{(a,a),(b,b),(c,c),(a,b),(a,c)\},\quad R_2=\{(a,b),(b,c),(a,c)\},\quad R_3=\{(a,b),(b,a),(c,c)\}.$$
\vv\vv

\item{9.} Considerare la seguente relazione in ${\bf N_>}$:
$$a\preceq b \Longleftrightarrow {a\over b}\in{\bf N}.$$
Dimostrare che $\preceq$ \`{e} un ordine parziale. Dire se si tratta di un ordine totale e se esistono massimo e minimo. 
Costruire una catena rispetto a tale ordine formata da $7$ elementi.
\ \vst

 \bye
