\nopagenumbers \font\title=cmti12
\def\ve{\vfill\eject}
\def\vv{\vfill}
\def\vs{\vskip-2cm}
\def\vss{\vskip10cm}
\def\vst{\vskip13.3cm}

%\def\ve{\bigskip\bigskip}
%\def\vv{\bigskip\bigskip}
%\def\vs{}
%\def\vss{}
%\def\vst{\bigskip\bigskip}
\hfuzz=3cm
\hsize=19.5cm
\vsize=27.58cm
\hoffset=-1.6cm
\voffset=0.5cm
\parskip=-.1cm
\ \vs \hskip -6mm AL1 AA08/09\ (Algebra: fondamenti)\hfill Appello C \hfill Roma, 15 Settembre 2009. \hrule
\bigskip\noindent {\title COGNOME}\  \dotfill\ {\title NOME}\ \dotfill {\title
MATRICOLA}\ \dotfill\
\smallskip  \noindent
Risolvere il massimo numero di esercizi accompagnando le risposte
con spiegazioni chiare ed essenziali. \it Inserire le risposte
negli spazi predisposti. NON SI ACCETTANO RISPOSTE SCRITTE SU
ALTRI FOGLI. Scrivere il proprio nome anche nell'ultima pagina.
\rm 1 Esercizio = 4 punti. Tempo previsto: 2 ore. Nessuna domanda
durante la prima ora e durante gli ultimi 20 minuti.
\smallskip
\hrule\smallskip
\centerline{\hskip 6pt\vbox{\tabskip=0pt \offinterlineskip
\def \trl{\noalign{\hrule}}
\halign to300pt{\strut#& \vrule#\tabskip=0.7em plus 1em& \hfil#&
\vrule#& \hfill#\hfil& \vrule#& \hfil#& \vrule#& \hfill#\hfil&
\vrule#& \hfil#& \vrule#& \hfill#\hfil& \vrule#& \hfil#& \vrule#&
\hfill#\hfil& \vrule#& \hfil#& \vrule#& \hfill#\hfil& \vrule#&
\hfil#& \vrule#& \hfill#\hfil& \vrule#& \hfil#& \vrule#& \hfil#&
\vrule#\tabskip=0pt\cr\trl && FIRMA && 1 && 2 && 3 && 4 &&
5 && 6 && 7 && 8 && 9 &&  TOT. &\cr\trl && &&   &&
&&     &&   &&   &&   &&   &&   &&    && &\cr &&
\dotfill &&     &&   &&   &&   &&     &&   && && && &&
&\cr\trl }}}
\medskip

\item{1.} 
Dopo aver definito con precisione la nozione di iniettivit\`a e suriettivit\`a, 
si forniscano due esempi espliciti di:\medskip
\itemitem{i.} un'applicazione iniettiva e non
suriettiva dall'insieme dei numeri interi ${\bf Z}$ in se;\smallskip
\itemitem{ii.} un'applicazione suriettiva ma non iniettiva dall'insieme dei numeri interi ${\bf Z}$ in se.
.\ve\vs   %foglio1
\item{2.}Si consideri la relazione $\sim$ su ${\bf R}\times{\bf R}$ definita da
$$(a,b)\sim (c,d)\ \ \ \Longleftrightarrow\ \ \ a^2+b^2=c^2+d^2.$$
Dimostrare che $\sim$ \`e una relazione di equivalenza e se ne determinino le classi di equivalenza.
.\vv
\item{3.} 
Sia $\cal P$ l’insieme costituito da tutti i sottoinsiemi non vuoti di ${\bf N}$ (cio\`e
$\cal P =\cal P({\bf N}) \setminus\emptyset$). Si consideri la seguente relazione definita su $\cal P$:
$$A\leq B\ \ \ \Longleftrightarrow\ \ \ A = B {\ \ \rm oppure \ \ } a\leq b\ \forall a\in A, b\in B.$$
\itemitem{(a)} Provare che si tratta di una relazione d'ordine;\smallskip
\itemitem{(b)} decidere se si tratta di una relazione d'ordine totale.\ve\vs   %foglio2
\item{4.} Dimostrare, usando il principio di induzione, che per ogni $n\in{\bf N}$, 
$$\sum_{k=1}^{n}{1\over 4k^2-1}={n\over 2n+1}$$
.\ve\vs   %foglio3
\item{5.} Trovare (se esistono) tutti gli interi compresi tra $-100$ e $100$ che
divisi per $3$ danno come resto $2$, divisi per $5$ danno come resto $3$ e
che moltiplicati per $5$ sono congrui a $2$ modulo $8$.\vv
\item{6.} Dopo aver definito la nozione di anello e campo, si dia un esempio di un anello che non \`e
un campo.\ve\vs   %foglio4
\item{7.} Calcolare la parte reale e quella immaginaria del numero complesso:
${2+3i\over 5+4i}+(1+i)^{30}$.\ve \vs  %foglio5
\item{8.} Dopo aver enunciato e dimostrato il Teorema di Eulero, lo si utilizzi
per calcolare le ultime due cifre decimali di $1999^{1999}$.\vskip5cm \vv

\item{9.} Consideriamo le seguenti permutazioni in $S_7$:
$$\sigma=\left({1\ 2\ 3\ 4\ 5\ 6\ 7\atop 2\ 1\ 5\ 4\ 7\ 6\ 3}\right)\qquad{\rm e}\qquad 
\tau=\left({1\ 2\ 3\ 4\ 5\ 6\ 7\atop 3\ 1\ 4\ 2\ 7\ 5\ 6}\right).$$ 
\itemitem{a.} Esprimere $\sigma$ e $\tau$ come il prodotto di cicli disgiunti.\medskip
\itemitem{b.} Calcolare la parit\`a di $\sigma$ e di $\tau$.\medskip
\itemitem{c.} Calcolare $\sigma^2\cdot\tau$, $\tau^5$, $\sigma^{-1}$.
.  \ \vst %foglio 6  
\bye
