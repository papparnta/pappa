\documentclass[italian,a4paper,11pt]
{article}
\usepackage{babel,amsmath,amssymb,amsbsy,amsfonts,latexsym,exscale,
amsthm,epsf,colordvi,enumerate}

\usepackage[latin1]{inputenc}
\usepackage[all]{xy}
\usepackage{textcomp}
\usepackage{graphicx}


\newcommand{\Q}{\mathbb{Q}}
\newcommand{\Z}{\mathbb Z}
\newcommand{\R}{\mathbb{R}}

\newcommand{\F}{\mathbb{F}}
\newcommand{\N}{\mathbb{N}}
\newcommand{\C}{\mathbb{C}}
\newcommand{\T}{\mathcal{T}}

\newcommand{\U}{\mathcal{U}}
\newcommand{\p}{\mathfrak{p}}
\newcommand{\ga}{\mathfrak{a}}
\newcommand{\gb}{\mathfrak{b}}

\newcommand{\q}{\mathfrak{q}}
\newcommand{\m}{\mathfrak{m}}
\newcommand{\X}{\mathbf{X}}

\newcommand{\D}{\mbox{\rm{\textbf{Dom}}}}
\newcommand{\Ze}{\mbox{\rm{\textbf{Rie}}}}

\newcommand{\esse}{\mbox{\rm{\textbf{Spec}}}}
\newcommand{\Ci}{\mathbf{C}}
\newcommand{\Ex}{\textbf{Esercizio}}


\newcommand{\Sse}{\Longleftrightarrow}
\newcommand{\sse}{\Leftrightarrow}
\newcommand{\implica}{\Rightarrow}

\newcommand{\frecdl}{\longrightarrow}
\newcommand{\frecd}{\rightarrow}
\newcommand{\st}{\scriptstyle}

\begin{document}
\begin{center}


\textbf{Universit\`a degli Studi Roma Tre}\\

\textbf{Corso di Laurea in Matematica, a.a. 2008/2009}\\

\textbf{AL1 - Algebra 1: Fondamenti}\\

\textbf{Prof. F. Pappalardi}\\

\textbf{Tutorato 4 - 31 Ottobre 2008}\\

\textbf{Elisa Di Gloria, Luca Dell'Anna}\\

www.matematica3.com\\
\end{center}



\vspace{1cm}




\noindent
\begin{Ex}\textbf{ 1.}\\
Dire di quali propriet\`a godono le seguenti relazioni ed individuare quali di esse sono relazioni d'ordine, d'equivalenza o d'ordine totale.\\
In $\Z$, $x$,$y \in \Z$
\begin{itemize}
	\item $x\rho y :\sse \  x+y$ \`e pari
	\item $x\rho y :\sse \ x=y$ oppure $x+y$ \`e multiplo di 3
	\item $x\rho y :\sse \ x < 3y$
	\item $x\rho y :\sse \ x^2 < y^2$
	\item $x\rho y :\sse \ x$ e $y$ non hanno divisori in comune
	\item $x\rho y :\sse \ xy>0$
	\item $x\rho y :\sse \ x=y+3$
	\item $x\rho y :\sse \ x=\pm y$
	\item $x\rho y :\sse \ x^2 \le y^2$
\end{itemize}
Sia $X$ l'insieme degli esseri umani, $x$,$y \in X$ 
\begin{itemize}
	\item $x \rho y :\sse \ x$ e $y$ sono nati nello stesso anno
  \item $x \rho y :\sse \ x$ e $y$ sono figli dello stesso padre
	\item $x \rho y :\sse \ x$ e $y$ hanno un genitore in comune
	\item $x \rho y :\sse \ x$ e $y$ 
	\item Nell'insieme delle rette nel piano, $x \rho y$ se e solo se $x$ e $y$ non sono parallele
\end{itemize}
Determinare infine il relativo insieme quoziente qualora la relazione sia di equivalenza.
\end{Ex}

\vspace{0.4cm}
\noindent
\begin{Ex}\textbf{ 2.}\\
Nell'insieme degli studenti di una classe esibire una relazione che sia:\\
i) Riflessiva, simmetrica ma non transitiva.\\
ii) Simmetrica, transitiva ma non riflessiva.\\
iii) Riflessiva, transitiva ma non simmetrica.\\
iv) Di equivalenza, che non sia stata gi\`a proposta nel primo esercizio, e descrivere l'insieme quoziente relativo.
\end{Ex}

\vspace{0.4cm}
\noindent
\begin{Ex}\textbf{ 3.}\\
Sia $X = \{1, 2, 3, 4\}$.
\begin{itemize}
	\item Esibire una relazione su $X$, che sia riflessiva, simmetrica ma non transitiva.
	\item Esibire una relazione su $X$, che sia simmetrica, transitiva ma non riflessiva.
	\item Esibire una relazione su $X$, che sia riflessiva, transitiva ma non simmetrica.
\end{itemize}
\end{Ex}


\vspace{0.4cm}
\noindent
\begin{Ex}\textbf{ 4.}\\
Sia $X = \{1, 2, 3, 4, 5, 6, 7, 8, 9, 10\}$.
\begin{itemize}
	\item Consideriamo su $X$ la relazione: $x \rho y$ se $x+y$ \`e un numero pari. Dimostrare che $\rho$ \`e relazione di equivalenza e determinare le classi di equivalenza corrispondenti.
	\item Consideriamo su $X$ la relazione: $x \rho y$ se $x+y$ \`e un numero dispari. Determinare se $\rho$ \`e una relazione di equivalenza e calcolarne eventualmente le classi di equivalenza corrispondenti.
\end{itemize}
\end{Ex}

\end{document}
