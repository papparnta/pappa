\nopagenumbers \font\title=cmti12
\def\ve{\vfill\eject}
\def\vv{\vfill}
\def\vs{\vskip-2cm}
\def\vss{\vskip10cm}
\def\vst{\vskip13.3cm}

%\def\ve{\bigskip\bigskip}
%\def\vv{\bigskip\bigskip}
%\def\vs{}
%\def\vss{}
%\def\vst{\bigskip\bigskip}

\hsize=19.5cm
\vsize=27.58cm
\hoffset=-1.6cm
\voffset=0.5cm
\parskip=-.1cm
\ \vs \hskip -6mm AL310 AA16/17\ (Teoria delle Equazioni)\hfill APPELLO A (Scritto) \hfill Roma, 18 Gennaio 2017. \hrule
\bigskip\noindent
{\title COGNOME}\  \dotfill\ {\title NOME}\ \dotfill {\title
MATRICOLA}\ \dotfill\
\smallskip  \noindent
Risolvere il massimo numero di esercizi accompagnando le risposte
con spiegazioni chiare ed essenziali. \it Inserire le risposte
negli spazi predisposti. NON SI ACCETTANO RISPOSTE SCRITTE SU
ALTRI FOGLI. Scrivere il proprio nome anche nell'ultima pagina.
\rm 1 Esercizio = 5 punti. Tempo previsto: 2 ore. Nessuna domanda
durante la prima ora e durante gli ultimi 20 minuti.
\smallskip
\hrule\smallskip
\centerline{\hskip 6pt\vbox{\tabskip=0pt \offinterlineskip
\def \trl{\noalign{\hrule}}
\halign to277pt{\strut#& \vrule#\tabskip=0.7em plus 1em& \hfil#&
\vrule#& \hfill#\hfil& \vrule#& \hfil#& \vrule#& \hfill#\hfil&
\vrule#& \hfil#& \vrule#& \hfill#\hfil& \vrule#& \hfil#& \vrule#&
\hfill#\hfil& \vrule#& \hfil#& \vrule#& \hfill#\hfil& \vrule#&
\hfil#& \vrule#& \hfill#\hfil& \vrule#& \hfil#& \vrule#& \hfil#&
\vrule#\tabskip=0pt\cr\trl && FIRMA && 1 && 2 && 3 && 4 &&
5 && 6 && 7 && 8 &&   TOT. &\cr\trl && &&   &&
&&     &&   &&   &&   &&   &&    && &\cr &&
\dotfill &&     &&   &&   &&     &&   && && && &&
&\cr\trl }}}
\medskip

\item{1.} Rispondere alle sequenti domande fornendo una giustificazione di una riga (giustificazioni
incomplete o poco chiare comportano punteggio nullo):\bigskip\bigskip\bigskip


\itemitem{a.} Quali sono i valori di $b\in{\bf C}$ tali che $[{\bf Q}[\sqrt{bi}]:{\bf Q}]=2$?\medskip\bigskip\bigskip

\ \dotfill\ \bigskip\bigskip\bigskip\vfil

\itemitem{b.} Scrivere una ${\bf Q}$--base del campo di spezzamento del polinomio $X^6-1\in{\bf Q}[X]$.\medskip\bigskip\bigskip

\ \dotfill\ \bigskip\bigskip\bigskip\vfil

\itemitem{c.} \`E vero che se $K$ \`e il campo di spezzamento di $X^6+X^2+1\in{\bf F}_2[X]$, allora
$[K:{\bf F}_2]=3$?\medskip\bigskip\bigskip
 
\ \dotfill\ \bigskip\bigskip\bigskip\vfil

\itemitem{d.} \`E vero che le estensioni finite di campi finiti sono sempre cicliche?\medskip\bigskip\bigskip

\ \dotfill\ \bigskip\bigskip\bigskip

\vfil\eject

%Dimostrare che un estensione finita \`{e} necessariamente algebrica. Produrre
%un esempio di un estensione algebrica non finita.

\item{2.} Dimostrare che se $R$ \`e un dominio che contiene un campo $F$ tale che $\dim_FR$ \`e
finita, allora $R$ \`e un campo. Dimostrare che l'ipotesi che $\dim_FR<\infty$ \`e necessaria.

\vv


\item{3.} Enunciare e dimostrare le formule di Cardano per le radici di un polinomio di terzo grado e applicarle per il
polinomio $X^3+X+1\in{\bf Q}[X]$.
\ve\ \vs

%Dopo aver verificato che \`e algebrico, calcolare
%il polinomio minimo di $\cos \pi/9$ su ${\bf Q}$.

\item{4.} Descrivere tutti gli elementi del gruppo di Galois del polinomio $x^6-9\in{\bf Q}[x]$. \vv

\item{5.} Spiegare perch\`e non \`e possibile quadrare il cerchio.
\ve\ \vs

%--\item{6.} Descrivere la nozione di campo perfetto dimostrando che i campi finiti
%sono perfetti.

\item{6.} Si enunci nella completa generalit\`a il Teorema di
corrispondenza di Galois.\vskip 6cm\bigskip\bigskip\bigskip\vv\vv


\item{7.} Dato un campo finito ${\bf F}_{p^n}$. Dimostrare che se $\gamma\in{\bf F}_{p^n}^*$
\`e un generatore (del gruppo moltiplicativo), allora tutte le radici del polinomio minimo $f_\gamma(X)\in{\bf F}_p[X]$
sono generatori.\vskip 6cm\bigskip\bigskip\bigskip\vv\vv

\item{8.} Considerare l'estensione algebrica semplice ${\bf Q}[\alpha], \alpha^4=\alpha-1$ (assumendo l'irriducibilit\`a di $X^4-X+1$). 
Determinare un espressione per il polinomio minimo su ${\bf Q}$ di $\alpha/(a\alpha+b)$ per ogni $a,b\in{\bf Q}$.


\vv

\ \vst
 \bye
