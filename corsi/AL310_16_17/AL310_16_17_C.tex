\nopagenumbers \font\title=cmti12
\def\ve{\vfill\eject}
\def\vv{\vfill}
\def\vs{\vskip-2cm}
\def\vss{\vskip10cm}
\def\vst{\vskip13.3cm}

%\def\ve{\bigskip\bigskip}
%\def\vv{\bigskip\bigskip}
%\def\vs{}
%\def\vss{}
%\def\vst{\bigskip\bigskip}

\hsize=19.5cm
\vsize=27.58cm
\hoffset=-1.6cm
\voffset=0.5cm
\parskip=-.1cm
\ \vs \hskip -6mm AL310 AA16/17\ (Teoria delle Equazioni)\hfill APPELLO C (Scritto) \hfill Roma, 26 Giugno 2017. \hrule
\bigskip\noindent
{\title COGNOME}\  \dotfill\ {\title NOME}\ \dotfill {\title
MATRICOLA}\ \dotfill\
\smallskip  \noindent
Risolvere il massimo numero di esercizi accompagnando le risposte
con spiegazioni chiare ed essenziali. \it Inserire le risposte
negli spazi predisposti. NON SI ACCETTANO RISPOSTE SCRITTE SU
ALTRI FOGLI. Scrivere il proprio nome anche nell'ultima pagina.
\rm 1 Esercizio = 5 punti. Tempo previsto: 2 ore. Nessuna domanda
durante la prima ora e durante gli ultimi 20 minuti.
\smallskip
\hrule\smallskip
\centerline{\hskip 6pt\vbox{\tabskip=0pt \offinterlineskip
\def \trl{\noalign{\hrule}}
\halign to277pt{\strut#& \vrule#\tabskip=0.7em plus 1em& \hfil#&
\vrule#& \hfill#\hfil& \vrule#& \hfil#& \vrule#& \hfill#\hfil&
\vrule#& \hfil#& \vrule#& \hfill#\hfil& \vrule#& \hfil#& \vrule#&
\hfill#\hfil& \vrule#& \hfil#& \vrule#& \hfill#\hfil& \vrule#&
\hfil#& \vrule#& \hfill#\hfil& \vrule#& \hfil#& \vrule#& \hfil#&
\vrule#\tabskip=0pt\cr\trl && FIRMA && 1 && 2 && 3 && 4 &&
5 && 6 && 7 && 8 &&   TOT. &\cr\trl && &&   &&
&&     &&   &&   &&   &&   &&    && &\cr &&
\dotfill &&     &&   &&   &&     &&   && && && &&
&\cr\trl }}}
\medskip

\item{1.} Rispondere alle sequenti domande fornendo una giustificazione di una riga (giustificazioni
incomplete o poco chiare comportano punteggio nullo):\bigskip\bigskip\bigskip


\itemitem{a.} Quanti elementi ha il gruppo di Galois di $x^{121}-1?$\medskip\bigskip\bigskip

\ \dotfill\ \bigskip\bigskip\bigskip\vfil

\itemitem{b.} Esprimere $1/(\alpha+2)$ come espressione polinomiale in $\alpha$ nel campo ${\bf Q}[\alpha], \alpha^3=7$.\medskip\bigskip\bigskip

\ \dotfill\ \bigskip\bigskip\bigskip\vfil

\itemitem{c.} Quanti elementi ha il campo di spezzamento di $(X^6+X^2+2X+65)(x^{64}+x^2)\in{\bf F}_2[X]$?\medskip\bigskip\bigskip
 
\ \dotfill\ \bigskip\bigskip\bigskip\vfil

\itemitem{d.} \`E possibile costruire un esempio di estensione di un campo finito con gruppo di Galois isomorfo
a ${\bf Z}/56{\bf Z}$?\medskip\bigskip\bigskip

\ \dotfill\ \bigskip\bigskip\bigskip

\vfil\eject

%Dimostrare che un estensione finita \`{e} necessariamente algebrica. Produrre
%un esempio di un estensione algebrica non finita.

\item{2.} Dopo aver dato la definizione di sottogruppo transitivo di
$S_n$, si elenchino i sottogruppi transitivi di $S_4$ descrivendone gli
elementi come permutazioni.

\vv


\item{3.} Mostrare che se $f$ \`{e} un polinomio irriducibile di tre a coefficienti
in un campo $F$, $G_f \cong S_3$ se e solo se $F_f$ non contiene sottocampi quadratici.
\ve\ \vs

%Dopo aver verificato che \`e algebrico, calcolare
%il polinomio minimo di $\cos \pi/9$ su ${\bf Q}$.

\item{4.} Scrivere tutte le radici di $x^{16}+x^{12}+ 1\in {\bf F_2}[\alpha]$, con $\alpha^4=\alpha+1$? Provare con $\alpha^3+1$.\vv

\item{5.} Spiegare come si fa a costruire un polinomio il cui gruppo di Galois \`e isomorfo a ${\bf Z}/5{\bf Z} \oplus {\bf Z}/10{\bf Z} \oplus {\bf Z}/10{\bf Z}$. 
\ve\ \vs

%--\item{6.} Descrivere la nozione di campo perfetto dimostrando che i campi finiti
%sono perfetti.

\item{6.} Sia ${\bf F}$ un campo. Definire il discriminante di un polinomio ${\bf F}[X]$ e dimostrare che \`e un elemento di
${\bf F}$.\vskip 6cm\bigskip\bigskip\bigskip\vv\vv


\item{7.*} Quanti elementi ha il campo di spezzamento di
$(x^2+x+1)(x^3+x^2+1)$ su ${\bf F}_{2^2}$.\vskip 3cm\bigskip\bigskip\bigskip\vv\vv

\item{8.} Si calcoli il gruppo di Galois di $y^5 - 3y^2 + 1$.\vv

\ \vst
 \bye
