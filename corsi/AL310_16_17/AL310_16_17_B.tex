\nopagenumbers \font\title=cmti12
\def\ve{\vfill\eject}
\def\vv{\vfill}
\def\vs{\vskip-2cm}
\def\vss{\vskip10cm}
\def\vst{\vskip13.3cm}

%\def\ve{\bigskip\bigskip}
%\def\vv{\bigskip\bigskip}
%\def\vs{}
%\def\vss{}
%\def\vst{\bigskip\bigskip}

\hsize=19.5cm
\vsize=27.58cm
\hoffset=-1.6cm
\voffset=0.5cm
\parskip=-.1cm
\ \vs \hskip -6mm AL310 AA16/17\ (Teoria delle Equazioni)\hfill APPELLO B (Scritto) \hfill Roma, 6 Febbraio 2017. \hrule
\bigskip\noindent
{\title COGNOME}\  \dotfill\ {\title NOME}\ \dotfill {\title
MATRICOLA}\ \dotfill\
\smallskip  \noindent
Risolvere il massimo numero di esercizi accompagnando le risposte
con spiegazioni chiare ed essenziali. \it Inserire le risposte
negli spazi predisposti. NON SI ACCETTANO RISPOSTE SCRITTE SU
ALTRI FOGLI. Scrivere il proprio nome anche nell'ultima pagina.
\rm 1 Esercizio = 5 punti. Tempo previsto: 2 ore. Nessuna domanda
durante la prima ora e durante gli ultimi 20 minuti.
\smallskip
\hrule\smallskip
\centerline{\hskip 6pt\vbox{\tabskip=0pt \offinterlineskip
\def \trl{\noalign{\hrule}}
\halign to277pt{\strut#& \vrule#\tabskip=0.7em plus 1em& \hfil#&
\vrule#& \hfill#\hfil& \vrule#& \hfil#& \vrule#& \hfill#\hfil&
\vrule#& \hfil#& \vrule#& \hfill#\hfil& \vrule#& \hfil#& \vrule#&
\hfill#\hfil& \vrule#& \hfil#& \vrule#& \hfill#\hfil& \vrule#&
\hfil#& \vrule#& \hfill#\hfil& \vrule#& \hfil#& \vrule#& \hfil#&
\vrule#\tabskip=0pt\cr\trl && FIRMA && 1 && 2 && 3 && 4 &&
5 && 6 && 7 && 8 &&   TOT. &\cr\trl && &&   &&
&&     &&   &&   &&   &&   &&    && &\cr &&
\dotfill &&     &&   &&   &&     &&   && && && &&
&\cr\trl }}}
\medskip

\item{1.} Rispondere alle sequenti domande fornendo una giustificazione di una riga (giustificazioni
incomplete o poco chiare comportano punteggio nullo):\bigskip\bigskip\bigskip


\itemitem{a.} Quanti elementi ha il gruppo di Galois di $(x^8-3)?$\medskip\bigskip\bigskip

\ \dotfill\ \bigskip\bigskip\bigskip\vfil

\itemitem{b.} Scrivere una ${\bf Q}$--base del campo di spezzamento del polinomio $(x^2-2)(x^2-3)(x^3-5)(x^2-30)\in{\bf Q}[X]$.\medskip\bigskip\bigskip

\ \dotfill\ \bigskip\bigskip\bigskip\vfil

\itemitem{c.} Quanti elementi ha il campo di spezzamento di $(X^6+X^2+3)(x^{32}+x^2)\in{\bf F}_2[X]$?\medskip\bigskip\bigskip
 
\ \dotfill\ \bigskip\bigskip\bigskip\vfil

\itemitem{d.} \`E possibile costruire un esempio di estensione di un campo finito con gruppo di Galois isomorfo
a $S_3$?\medskip\bigskip\bigskip

\ \dotfill\ \bigskip\bigskip\bigskip

\vfil\eject

%Dimostrare che un estensione finita \`{e} necessariamente algebrica. Produrre
%un esempio di un estensione algebrica non finita.

\item{2.} Mostrare che un estensione di campi \`e finita se e solo se \`e algebrica e finitamente generata spiegando le nozioni di cui si
parla.

\vv


\item{3.} Sia $\alpha=\cos 9^\circ$ (il  coseno di nove gradi). Dopo aver dimostrato che \`e numero algebrico, se ne calcoli il polinomio 
minimo e lo si esprima, se possibile, in termini di radicali. 
\ve\ \vs

%Dopo aver verificato che \`e algebrico, calcolare
%il polinomio minimo di $\cos \pi/9$ su ${\bf Q}$.

\item{4.} Determinare i gruppi di Galois su ${\bf Q}$ dei seguenti polinomi $x^3+x+10$ e $x^4+2x^2+5$.\vv

\item{5.} Dopo aver dimostrato che $\sin2\pi/5$ \`e un numero algebrico, se ne calcoli il polinomio minimo.
\ve\ \vs

%--\item{6.} Descrivere la nozione di campo perfetto dimostrando che i campi finiti
%sono perfetti.

\item{6.} Si enunci e si dimostri il Lemma di Artin.\vskip 6cm\bigskip\bigskip\bigskip\vv\vv


\item{7.}  Dare un esempio di campo finito ${\bf F}_{27}$ con $27$
elementi determinando tutti i generatori del gruppo moltiplicativo
${\bf F}_{27}^*$.\vskip 3cm\bigskip\bigskip\bigskip\vv\vv

\item{8.} Enunciare e dimostrare il Teorema di costruibilit\`{a} dei poligoni regolari.\vv

\ \vst
 \bye
