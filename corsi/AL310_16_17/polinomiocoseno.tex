\documentclass[a4paper,10pt]{article}
\usepackage[utf8]{inputenc}
\usepackage[cm]{fullpage}

%opening
\title{Tre metodi per calcolare il polinomio $f_{\cos 2\pi/n}$}
\author{AL310 -- 2016/17}

\begin{document}

\maketitle

In questa nota, il polinomio $\Psi_n(X)=f_{2\cos 2\pi/n}(X)\in{\bf Z}[X]$ indica il polinomio minimo del
numero algebrico $2\cos2\pi/n=\zeta_n+\bar\zeta_n$ dove $\zeta_n=e^{2\pi i/n}$
è una radice primitiva $n$--esima dell'unità.

Inoltre indichiamo con $\Phi_n(X)=f_{\zeta_n}(X)$ l'$n$--esimo polinomio ciclotomico.

Fatti che utilizzeremo:
\begin{enumerate}
 \item  $\Phi_n(X)\in{\bf Z}[X]$ è monico, irriducibile e $\deg\Phi_n=\varphi(n)$;
 \item $\bar\zeta_n=\zeta_n^{n-1}=\zeta_n^{-1}$;
 \item $[{\bf Q}[\zeta_n]:{\bf Q}]=\deg\Phi_n=\varphi(n)$;
 \item $[{\bf Q}[\cos 2\pi/n]:{\bf Q}]=\deg\Psi_{n} =\frac12\varphi(n)$;
\item $2^{-\varphi(n)/2}\Psi_n(2X)=f_{2\cos 2\pi/n}(X).$
\end{enumerate}



\section{Il sistema lineare} 
Si parte dall'identità $\Phi_n(\zeta_n)=0$
che può essere usata come strumento per scrivere tutti gli elementi di ${\bf Q}[\zeta_n]$ nella forma
$$a_0+a_1\zeta_n+\cdots+a_{\varphi(n)-1}\zeta_n^{\varphi(n)-1}.$$
Vogliamo determinare i numeri interi $A_0,\ldots,A_M$ tali che 
$$\Psi_n(X)=A_0+A_1X+\cdots+A_{M}X^M,$$
dove $M=\frac12\varphi(n)$.
Si parte dall'identità:
$$\Psi_n(2\cos \frac{2\pi}n)=\Psi_n\left(\zeta_n+\zeta_n^{n-1}\right)=0.$$
che può essere scritta come
$$A_0+A_1\left(\zeta_n+\zeta_n^{\varphi(n)-1}\right)+\cdots+A_{M}\left(\zeta_n+\zeta_n^{\varphi(n)-1}\right)^M=0$$
e, in quanto elemento di ${\bf Q}[\zeta_n]$, la quantità sopra può essere riscritta nella forma seguente:
$$c_0+c_1\zeta_n+\cdots+c_{n-1}\zeta_n^{n-1}=0.$$
dove per ogni $j$, $0\le j<\varphi(n)$, $c_j=c_j(A_0,A_1,\ldots,A_M)$ è lineare (i.e. una combinazione lineare degli $A_j$) nelle variabili
$A_0,A_1,\ldots,A_M$. Risolvendo il sistema lineare
$$\left\{\begin{array}{rcl}
          c_0(A_0,A_1,\ldots,A_M)& =&0\\
c_1(A_0,A_1,\ldots,A_M)&=&0\\
&\vdots&\\
c_{\varphi(n)-1}(A_0,A_1,\ldots,A_M)&=&0
           \end{array}\right.$$
  si ottengono i coefficienti $A_0,A_1,\ldots,A_M$.\medskip
  
\noindent\textsc{ESEMPIO del calcolo di $\Psi_9$:}
Il nono polinomio ciclotomico è $\Phi_9(X)=\Phi_3(X^3)=X^6+X^3+1$ da cui deduciamo:
$\zeta_9^6=-1-\zeta_9^3.$
Il grado $\deg \Psi_9=3$. Pertanto$\Psi_9(X)=A_0+A_1X+A_2X^2+X^3.$ Sostituendo
$2\cos \frac{2\pi}9=\zeta_9+\zeta_9^8$, otteniamo
$$A_0+A_1\left(\zeta_9+\zeta_9^8\right)+
+A_2\left(\zeta_9+\zeta_9^8\right)^2+\left(\zeta_9+\zeta_9^8\right)^3.$$
Facendo i calcoli otteniamo:
$$\left(A_0+A_2-1\right)+
\left(A_1-A_2+3\right)\zeta_9+
\left(-A_1+A_2-8\right)\zeta_9^2+
\left(-A_2\right)\zeta_9^4+
\left(-A_1-3\right)\zeta_9^5.$$
Da cui otteniamo il sistema lineare:
$$\left\{\begin{array}{rcl}
          A_0+A_2-1& =&0\\
A_1-A_2+3&=&0\\
-A_1+A_2-3&=&0\\
-{A_2}&=&0\\
-A_1-3&=&0
           \end{array}\right.$$
che ha per soluzioni $A_0=1,A_1=-3,A_2=0$. Infine $\Psi(X)=X^3-3X+1.$



\section{il trucco ciclotomico}

Questo metodo consiste nell'utilizzo della seguente identità:
$$X^{M}\Psi\left(X+\frac1X\right)=\Phi_n(X).$$
Tale identità segue dal fatto che la quantità di sinitra è un polinomio monico a coefficienti razionali di grado $2M$.
Inoltre $\Psi_n\left(\zeta_n+\zeta_n^{-1}\right)=\Psi_n\left(2\cos\frac{2\pi}n\right)=0$.
Dall'unicità del polinomio minimo segue l'identità con il polinomio ciclotomico.
\medskip

Se scriviamo
$$\Phi_n(X)=\alpha_0+\alpha_1X+\cdots+\alpha_{2M-1}X^{2M-1}+X^{2M}\quad\textrm{e}\quad \Psi(X)=A_0+A_1X+\cdots+A_{M-1}X^{M-1}+X^{M},$$
allora
\begin{eqnarray*}
 X^{M}\Psi_{n}\left(X+\frac1X\right)&=&
A_0X^M+A_1 X^{M-1}\left(X^2+1\right)+\cdots+A_{M-1}X(X^2+1)^{M-1}+\left(X^2+1\right)^{M}\\
&=&b_0+b_1X+\cdots+b_{2M-1}X^{2M-1}+X^{2M}
\end{eqnarray*}
dove, per ogni $j$, $0\le j<2M$, $b_j=b_j(A_0,A_1,\ldots,A_M)$ è lineare (i.e. una combinazione lineare) nelle variabili
$A_0,A_1,\ldots,A_M$. Risolvendo il sistema lineare
$$\left\{\begin{array}{rcl}
          b_0(A_0,A_1,\ldots,A_M)& =&\alpha_0\\
b_1(A_0,A_1,\ldots,A_M)&=&\alpha_1\\
&\vdots&\\
b_{2M-1}(A_0,A_1,\ldots,A_M)&=&\alpha_{2M-1}
           \end{array}\right.$$
  si ottengono i coefficienti $A_0,A_1,\ldots,A_M$. Si noti che non è difficile verificare che $b_0=1$, $b_1=2A_{M-1}$, $b_2=M$
\medskip

\noindent\textsc{ESEMPIO del calcolo di $f_{2\cos\frac{2\pi}9}$}:  L'identità
$X^3\Psi_{3}\left(X+\frac1X\right)=\Phi_9(X)$ è equivalente a
$$X^6+X^3+1=X^6
 + A_{2} X^5
 + \left( A_{1}
 + 3\right)  X^4
 + \left( A_{0}
 +  A_{2}\right)  X^3
 +  \left( A_{1}
 + 3\right)  X^2
 +  A_{2} X
 + 1
$$
da cui deduciamo $A_2=0$, $A_1=-3$, $A_2=1$ e $f_{\cos \frac{2\pi}9}(X)=X^3-{3}X+1.$

\section{per ottenere un multiplo del polinomio minimo}

Se si sfrutta il fatto che $\zeta_n=\cos\frac{2\pi}n+i\sin\frac{2\pi}n$ e si considera l'identità
$\Phi_n(\zeta_n)=0$. Scriviamo
$$\Re\Phi_n(\cos\frac{2\pi}n+i\sin\frac{2\pi}n)=\Re\left(\alpha_0+\alpha_1\zeta_n+\cdots+\alpha_{2M-1}\zeta_n^{2M-1}+\zeta_n^{2M}\right)=0$$
e osserviamo che $\Re\zeta_n^k=\cos \frac{k\pi}n.$ 

Affermiamo che per ogni $k\in{\bf N}$ esistono due polinomi $g_k(X),h_k(X)\in{\bf Z}[X]$ tali che
$\cos k\alpha=g_k(\cos\alpha)$ e $\sin k\alpha=\sin\alpha h_k(\cos\alpha)$. Ciò è chiaro per $k=1$. Inoltre dalla formula di duplicazione per il coseno, si ottiene
$\cos2\alpha=2\cos^2\alpha-1$ e $\sin2\alpha=2\sin\alpha\cos\alpha$. In generale, 
\begin{eqnarray*}
\cos(k+1)\alpha &=& \cos\alpha\cos k\alpha-\sin\alpha\sin k\alpha=\cos\alpha\times g_k(\cos\alpha)-(1-\cos^2\alpha)h_k(\cos k\alpha)\\
\sin(k+1)\alpha &=& \sin\alpha\cos k\alpha+\cos\alpha\sin k\alpha=\sin\alpha\times \left(g_k(\cos\alpha)+\cos\alpha h_k(\cos k\alpha)\right). 
\end{eqnarray*}
Quindi $g_k$ e $h_k$ soddisfano le relazioni ricorsive:
$$\left\{\begin{array}{l}
          g_1(X)=X\\ g_{k+1}(X)=X\times g_k(X)-(1-X^2)h_k(X),
         \end{array}\right.\qquad
         \left\{\begin{array}{l}
          h_1(X)=X\\ h_{k+1}(X)=g_k(X)+X\times h_k(X).
         \end{array}\right.
         $$
E' anche facile dimostrare per induzione che $\deg g_k=k$ e $\deg h_k=k-1$.
         
         
Da questa osservazione deduciamo:
$$\alpha_0+\alpha_1g_1(\cos\frac{2\pi}{n})+\cdots+\alpha_{2M-1}g_{2M-1}(\cos\frac{2\pi}{n})+g_{2M}(\cos\frac{2\pi}{n})=0$$
Quindi, il polinomio 
$$U_n(X):=\alpha_0+\alpha_1g_1(X)+\cdots+\alpha_{2M-1}g_{2M-1}(X)+g_{2M}(X)\in{\bf Z}[X]$$ 
ha la proprietà di avere $\cos\frac{2\pi}{n}$ come radice. Pertanto $\Psi_{{n}}(X)$ divide $U_n(X)$.
\medskip
  
\noindent\textsc{ESEMPIO del calcolo di $f_{\cos\frac{2\pi}9}$}: Partendo dall'identità $\Phi_9(X)=X^6+X^3+1$, otteniamo
$$U_9(X)=1+g_3(X)+g_6(X)=1+(4X^3 - 3X)+(32X^6 - 48X^4 + 18X^2 - 1)$$
Quindi, fattorizzando $U_9$, si ottiene: 
$U_9(X)=32 X^6
 - 48 X^4
 + 4 X^3
 + 18 X^2
 - 3 X=X(4X^2-3)(8X^3 - 6X + 1).$ Sapendo che $\deg\Psi_{9}=3$, deduciamo che $\Psi_{9}(X/2)=X^3-\frac{3}{4}X+\frac{1}{8}.$

Si noti che dall'identità $\zeta_n^n=1$ si deduce anche che, per $n>1$,
$\Psi_{n}\mid\gcd(g_n,h_n)$.

\section{la trigonometria}

ANCORA DA SCRIVERE
\medskip
  
\noindent\textsc{ESEMPIO del calcolo di $f_{\cos\frac{2\pi}9}$}  
\end{document}
