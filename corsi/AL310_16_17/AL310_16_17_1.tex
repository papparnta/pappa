\nopagenumbers \font\title=cmti12
% \def\ve{\vfill\eject}
% \def\vv{\vfill}
% \def\vs{\vskip-2cm}
% \def\vss{\vskip10cm}
% \def\vst{\vskip13.3cm}

\def\ve{\bigskip\bigskip}
\def\vv{\bigskip\bigskip}
\def\vs{}
\def\vss{}
\def\vst{\bigskip\bigskip}

\hsize=19.5cm
\vsize=27.58cm
\hoffset=-1.6cm
\voffset=0.5cm
\parskip=-.1cm
\ \vs \hskip -6mm AL310 AA16/17\ (Teoria delle Equazioni)\hfill ESAME
DI MET\`{A} SEMESTRE \hfill Roma, 9 Novembre 2016 \hrule
\bigskip\noindent
{\title COGNOME}\  \dotfill\ {\title NOME}\ \dotfill {\title
MATRICOLA}\ \dotfill\
\smallskip  \noindent
Risolvere il massimo numero di esercizi accompagnando le risposte
con spiegazioni chiare ed essenziali. \it Inserire le risposte
negli spazi predisposti. NON SI ACCETTANO RISPOSTE SCRITTE SU
ALTRI FOGLI.
\rm 1 Esercizio = 4 punti. Tempo previsto: 2 ore. Nessuna domanda
durante la prima ora e durante gli ultimi 20 minuti.
\smallskip
\hrule\smallskip
\centerline{\hskip 6pt\vbox{\tabskip=0pt \offinterlineskip
\def \trl{\noalign{\hrule}}
\halign to277pt{\strut#& \vrule#\tabskip=0.7em plus 1em& \hfil#&
\vrule#& \hfill#\hfil& \vrule#& \hfil#& \vrule#& \hfill#\hfil&
\vrule#& \hfil#& \vrule#& \hfill#\hfil& \vrule#& \hfil#& \vrule#&
\hfill#\hfil& \vrule#& \hfil#& \vrule#& \hfill#\hfil& \vrule#&
\hfil#& \vrule#& \hfill#\hfil& \vrule#& \hfil#& \vrule#& \hfil#&
\vrule#\tabskip=0pt\cr\trl && FIRMA && 1 && 2 && 3 && 4 &&
5 && 6 && 7 && 8  &&  TOT. &\cr\trl && &&   &&
&&     &&   &&     &&   &&   &&    && &\cr &&
\dotfill &&       &&   &&   &&     &&   && && && &&
&\cr\trl }}}
\medskip

\item{1.} Rispondere alle seguenti domande fornendo una giustificazione di una riga:\bigskip\bigskip\bigskip


\itemitem{a.} E' vero che il grado di $F[\alpha]$ su $F$, se finito, \`e pari a $\deg f_\alpha$?\medskip\bigskip\bigskip

%\ \dotfill\ \bigskip\bigskip\bigskip\vfil

\itemitem{b.} E' vero che esistono campi con elementi algebrici sul sottocampo fondamentale il cui polinomio minimo \`e non separabile?\medskip\bigskip\bigskip

%\ \dotfill\ \bigskip\bigskip\bigskip\vfil

\itemitem{c.} Determinare il grado del campo ${\bf Q}(\cos(\pi/22))$.\medskip\bigskip\bigskip
 
%\ \dotfill\ \bigskip\bigskip\bigskip\vfil

\itemitem{d.} Fornire un esempio di estensione finita $E/F$ tale che $1<$\# Aut($E/F)<[E:F]$.\medskip\bigskip\bigskip

%\ \dotfill\ \bigskip\bigskip\bigskip


%\vfil\eject

%Dimostrare che un estensione finita \`{e} necessariamente algebrica. Produrre
%un esempio di un estensione algebrica non finita.

\item{2.} Calcolare il polinomio minimo di $1/\alpha$ e di $1/(\alpha-1)$ nel campo ${\bf Q}[\alpha],\alpha^4=\alpha+1$.\vv


\item{3.} Dopo aver definito la nozione di polinomio ciclotomico $\Phi_n(X)$, si dimostrino le seguenti propriet\`a:\smallskip
\itemitem{a.} Se $p$ \`e primo, $\Phi_p(X)=(X^p-1)/(X-1)$\smallskip
\itemitem{b.} Se $\alpha\ge1$, $\Phi_{p^\alpha}(X)=\Phi_p(X^{p^{\alpha-1}})$\smallskip
\itemitem{c.} Se $n$ \`e dispari, $\Phi_{2n}(X)=\Phi_n(-X)$


\ve\ \vs

%Dopo aver verificato che \`e algebrico, calcolare
%il polinomio minimo di $\cos \pi/9$ su ${\bf Q}$.

\item{4.} Dopo aver descritto tutti gli elementi di Aut(${\bf Q}(5^{1/3},\sqrt{-3})/{\bf Q})$, si determini l'ordine di ciascuno di essi.\vv

\item{5.} Determinare il campo di spezzamento su ${\bf Q}$ di $f(X)=(X^4-2)(X^2+1)((X-3)^2+6)\in{\bf Q}[X]$ e se ne determini il grado su ${\bf Q}$.
\ve\ \vs

%--\item{6.} Descrivere la nozione di campo perfetto dimostrando che i campi finiti
%sono perfetti.

\item{6.} Dopo aver definito la nozione di campo perfetto, si forniscano esempi di campi perfetti e di campi non perfetti.\vv\vv


\item{7.}   Dopo aver verificato che \`e algebrico, calcolare
il polinomio minimo di $\cos \pi/10$ su ${\bf Q}$. ({\it suggerimento: calcolare $\cos2\pi/5$ e poi usare le formule di duplicazione degli angoli - altri metodi potrebbero essere troppo lunghi})
\vv\vv


\item{8.} Dopo aver verificato che ${\bf Q}(\sqrt{2})\subset{\bf Q}(\sqrt3,\sqrt6)$, descrivere gli ${\bf Q}(\sqrt{2})$--omomorfismi del campo 
${\bf Q}(\sqrt3,\sqrt6)$ in ${\bf C}$.

\vv

\ \vst
 \bye
