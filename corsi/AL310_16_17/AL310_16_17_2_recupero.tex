\nopagenumbers \font\title=cmti12
\def\ve{\vfill\eject}
\def\vv{\vfill}
\def\vs{\vskip-2cm}
\def\vss{\vskip10cm}
\def\vst{\vskip13.3cm}

%\def\ve{\bigskip\bigskip}
%\def\vv{\bigskip\bigskip}
%\def\vs{}
%\def\vss{}
%\def\vst{\bigskip\bigskip}

\hsize=19.5cm
\vsize=27.58cm
\hoffset=-1.6cm
\voffset=0.5cm
\parskip=-.1cm
\ \vs \hskip -6mm AL310 AA16/17\ (Teoria delle Equazioni)\hfill ESAME
DI FINE SEMESTRE BIS\hfill Roma, 18 Gennaio 2017. \hrule
\bigskip\noindent
{\title COGNOME}\  \dotfill\ {\title NOME}\ \dotfill {\title
MATRICOLA}\ \dotfill\
\smallskip  \noindent
Risolvere il massimo numero di esercizi accompagnando le risposte
con spiegazioni chiare ed essenziali. \it Inserire le risposte
negli spazi predisposti. NON SI ACCETTANO RISPOSTE SCRITTE SU
ALTRI FOGLI. Scrivere il proprio nome anche nell'ultima pagina.
\rm 1 Esercizio = 5 punti. Tempo previsto: 2 ore. Nessuna domanda
durante la prima ora e durante gli ultimi 20 minuti.
\smallskip
\hrule\smallskip
\centerline{\hskip 6pt\vbox{\tabskip=0pt \offinterlineskip
\def \trl{\noalign{\hrule}}
\halign to277pt{\strut#& \vrule#\tabskip=0.7em plus 1em& \hfil#&
\vrule#& \hfill#\hfil& \vrule#& \hfil#& \vrule#& \hfill#\hfil&
\vrule#& \hfil#& \vrule#& \hfill#\hfil& \vrule#& \hfil#& \vrule#&
\hfill#\hfil& \vrule#& \hfil#& \vrule#& \hfill#\hfil& \vrule#&
\hfil#& \vrule#& \hfill#\hfil& \vrule#& \hfil#& \vrule#& \hfil#&
\vrule#\tabskip=0pt\cr\trl && FIRMA && 1 && 2 && 3 && 4 &&
5 && 6 && 7 && 8 &&  TOT. &\cr\trl && &&   &&
&&     &&   &&   &&   &&   &&    && &\cr &&
\dotfill    &&   &&   &&   &&     &&   && && && &&
&\cr\trl }}}
\medskip

\item{1.} Rispondere alle seguenti domande fornendo una giustificazione di una riga:\bigskip\bigskip\bigskip


\itemitem{a.} Quali sono i valori di $b\in{\bf C}$ tali che $[{\bf Q}[\sqrt{bi}]:{\bf Q}]=2$? \medskip\bigskip\bigskip

\ \dotfill\ \bigskip\bigskip\bigskip\vfil

\itemitem{b.} Scrivere una ${\bf Q}$--base del campo di spezzamento del polinomio $X^6-1\in{\bf Q}[X]$.\medskip\bigskip\bigskip

\ \dotfill\ \bigskip\bigskip\bigskip\vfil

\itemitem{c.}  \`E vero che se $K$ \`e il campo di spezzamento di $X^6+X^2+1\in{\bf F}_2[X]$, allora
$[K:{\bf F}_2]=3$?\medskip\bigskip\bigskip
 
\ \dotfill\ \bigskip\bigskip\bigskip\vfil

\itemitem{d.}\`E vero che le estensioni finite di campi finiti sono sempre cicliche?\medskip\bigskip\bigskip

\ \dotfill\ \bigskip\bigskip\bigskip


\vfil\eject

%Dimostrare che un estensione finita \`{e} necessariamente algebrica. Produrre
%un esempio di un estensione algebrica non finita.

\item{2.} Fornire un esempio concreto di un polinomio irriducibile di grado otto il cui gruppo di Galois \`e isomorfo a $D_4$.\vv


\item{3.} Dato un gruppo finito $H\subseteq S_p$ ($p$ primo), dimostrare che esiste un estensione di Galois $E/F$ tale che 
Gal$(E/F)\cong H$.
\ve\ \vs

%Dopo aver verificato che \`e algebrico, calcolare
%il polinomio minimo di $\cos \pi/9$ su ${\bf Q}$.

\item{4.} Enunciare e dimostrare una formula per il numero di polinomi irriducibili di grado $n$ su ${\bf F}_p$. 

\vv

\item{5.} Dimostrare che, fissato $N\in{\bf N}$, esistono infiniti campi, a due a due linearmente disgiunti che 
ammettono almeno $2^N$ sottocampi quadratici.
\ve\ \vs

%--\item{6.} Descrivere la nozione di campo perfetto dimostrando che i campi finiti
%sono perfetti.

\item{6.} Si enunci nella completa generalit\`a il Teorema di
corrispondenza di Galois.\vskip7cm\vv\vv


\item{7.} Descrivere tutti gli elementi del gruppo di Galois del polinomio $x^6-9\in{\bf Q}[x]$ e determinare il reticolo dei sottocampi
del campo di spezzamento.\vskip7cm\vv\vv

\item{8.} Determinare, dato un numero naturale $t$, un numero algebrico il cui polinomio minimo sui razionali ha un gruppo di 
Galois isomorfo a ${\bf Z}/{7^{t}}{\bf Z}$.

\vv



\ \vst
 \bye
