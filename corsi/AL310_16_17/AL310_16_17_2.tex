\nopagenumbers \font\title=cmti12
\def\ve{\vfill\eject}
\def\vv{\vfill}
\def\vs{\vskip-2cm}
\def\vss{\vskip10cm}
\def\vst{\vskip13.3cm}

%\def\ve{\bigskip\bigskip}
%\def\vv{\bigskip\bigskip}
%\def\vs{}
%\def\vss{}
%\def\vst{\bigskip\bigskip}

\hsize=19.5cm
\vsize=27.58cm
\hoffset=-1.6cm
\voffset=0.5cm
\parskip=-.1cm
\ \vs \hskip -6mm AL310 AA16/17\ (Teoria delle Equazioni)\hfill ESAME
DI FINE SEMESTRE \hfill Roma, 21 Dicembre 2016. \hrule
\bigskip\noindent
{\title COGNOME}\  \dotfill\ {\title NOME}\ \dotfill {\title
MATRICOLA}\ \dotfill\
\smallskip  \noindent
Risolvere il massimo numero di esercizi accompagnando le risposte
con spiegazioni chiare ed essenziali. \it Inserire le risposte
negli spazi predisposti. NON SI ACCETTANO RISPOSTE SCRITTE SU
ALTRI FOGLI. Scrivere il proprio nome anche nell'ultima pagina.
\rm 1 Esercizio = 5 punti. Tempo previsto: 2 ore. Nessuna domanda
durante la prima ora e durante gli ultimi 20 minuti.
\smallskip
\hrule\smallskip
\centerline{\hskip 6pt\vbox{\tabskip=0pt \offinterlineskip
\def \trl{\noalign{\hrule}}
\halign to277pt{\strut#& \vrule#\tabskip=0.7em plus 1em& \hfil#&
\vrule#& \hfill#\hfil& \vrule#& \hfil#& \vrule#& \hfill#\hfil&
\vrule#& \hfil#& \vrule#& \hfill#\hfil& \vrule#& \hfil#& \vrule#&
\hfill#\hfil& \vrule#& \hfil#& \vrule#& \hfill#\hfil& \vrule#&
\hfil#& \vrule#& \hfill#\hfil& \vrule#& \hfil#& \vrule#& \hfil#&
\vrule#\tabskip=0pt\cr\trl && FIRMA && 1 && 2 && 3 && 4 &&
5 && 6 && 7 && 8 &&  TOT. &\cr\trl && &&   &&
&&     &&   &&   &&   &&   &&    && &\cr &&
\dotfill    &&   &&   &&   &&     &&   && && && &&
&\cr\trl }}}
\medskip

\item{1.} Rispondere alle seguenti domande fornendo una giustificazione di una riga:\bigskip\bigskip\bigskip


\itemitem{a.} E' vero che il campo di spezzamento su ${\bf F}_p$ di $f(X)\in{\bf F}_p[X]$ ha sempre $p^{\deg f}$ elementi? \medskip\bigskip\bigskip

\ \dotfill\ \bigskip\bigskip\bigskip\vfil

\itemitem{b.} E' vero che per ogni $a\in{\bf Q}^*$, il grado $[{\bf Q}[a^{1/4}]:{\bf Q}]=4$?\medskip\bigskip\bigskip

\ \dotfill\ \bigskip\bigskip\bigskip\vfil

\itemitem{c.} Quali sono i valori di $b\in{\bf Q}^*$, per cui ${\bf Q}[b^{1/6}]/{\bf Q}$ \`e Galois.\medskip\bigskip\bigskip
 
\ \dotfill\ \bigskip\bigskip\bigskip\vfil

\itemitem{d.} \`E vero che tutti i gruppi di Galois dei polinomi di grado $5$ sono tutti sottogruppi di $S_5$?\medskip\bigskip\bigskip

\ \dotfill\ \bigskip\bigskip\bigskip


\vfil\eject

%Dimostrare che un estensione finita \`{e} necessariamente algebrica. Produrre
%un esempio di un estensione algebrica non finita.

\item{2.} Fornire un esempio di un polinomio irriducibile di grado sei il cui gruppo di Galois \`e isomorfo a $S_3$.\vv


\item{3.} Sia $p$ un numero primo. Sia $H\subset{\rm Gal}({\bf Q}[\zeta_p]/{\bf Q})$
l'unico sottogruppo con $(p-1)/2$ elementi. Si determini il periodo di Gauss $\eta_H$.
\ve\ \vs

%Dopo aver verificato che \`e algebrico, calcolare
%il polinomio minimo di $\cos \pi/9$ su ${\bf Q}$.

\item{4.} Dopo aver dimostrato che $X^2+3\in{\bf F}_5[X]$ \`e irriducibile, si consideri 
${\bf F}_{5^2}={\bf F}_5[\alpha], \alpha^2=2$ e si determinino i generatori di ${\bf F}_5[\alpha]^*$. 

\vv

\item{5.} Descrivere il reticolo dei sottocampi del campo ciclotomico ${\bf Q}[\zeta_{15}]$ menzionando i ciascun caso i generatori.
\ve\ \vs

%--\item{6.} Descrivere la nozione di campo perfetto dimostrando che i campi finiti
%sono perfetti.

\item{6.} Si enunci nella completa generalit\`a il Teorema di
corrispondenza di Galois.\vskip7cm\vv\vv


\item{7.} Dopo aver enunciato il Teorema dell'elemento primitivo, si consideri  $E={\bf Q}[\sqrt{3},\sqrt{-2},\sqrt{-6}]$. 
Determinare un elemento primitivo $\gamma\in E$ su ${\bf Q}$ e scriverne il polinomio minimo su ${\bf Q}$. Descrivere inoltre tutti 
i sottocampi di $E$.\vskip7cm\vv\vv

\item{8.} Determinare un numero algebrico il cui polinomio minimo sui razionali ha un gruppo di 
Galois isomorfo a $C_{24}$.

\vv



\ \vst
 \bye
