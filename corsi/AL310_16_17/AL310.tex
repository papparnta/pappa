\input programma.sty
\def\abbrcorso{AL310}
\def\titolocorso{Istituzioni di algebra superiore}
\def\sottotitolo{http://www.mat.uniroma3.it/users/pappa/CORSI/AL310$_-$16$_-$17/AL310.htm}
\def\docente{Prof. Francesco Pappalardi e Prof. Valerio Talamanca}
\def\crediti{7}
\def\semestre{I}
\def\esoneri{1}
\def\scrittofinale{1}
\def\oralefinale{1}
\def\altreprove{0}
\Intestazione \titoloparagr{Introduzione.} Equazioni di Cardano
per la risolubilit\`{a} delle equazioni di terzo grado,  anelli e
campi, la caratteristica di un campo, richiami sugli anelli di
polinomi, estensioni di campi, costruzione di alcune estensioni di
campi, il sottoanello generato da un sottoinsieme, il sottocampo
generato da un sottoinsieme, elementi algebrici e trascendenti,
campi algebricamente chiusi.

\titoloparagr{Campi di spezzamento.} Estensioni semplici e mappe
tra estensioni semplici, campi di spezzamento, esistenza del campo
di spezzamento, unicit\`{a} a meno di isomorfismi del campo di
spezzamento, radici multiple, derivate formali, polinomi
separabili e campi perfetti, polinomi minimi e loro
caratterizzazioni.

\titoloparagr{Il Teorema fondamentale della Teoria di Galois.}
Gruppo degli automorfismi di un campo, estensioni normali,
separabili e di Galois, caratterizzazioni di estensioni separabili,
{\bf Teorema fondamentale della corrispondenza di Galois}, esempi,
gruppo di Galois di un polinomio, Estensioni radicali, gruppi risolubili e il 
Teorema di Galois sulla risoluzione delle equazioni, Teorema dell'esistenza
dell'ele\-men\-to primitivo.

\titoloparagr{Il calcolo del gruppo di Galois.} Gruppi di Galois
come sottogruppi di $S_n$, sottogruppi transitivi di $S_n$,
caratterizzazione dell'irriducibilit\`{a} in termini della
transitivit\`{a}, polinomi con gruppi di Galois in $A_n$, Teoria dei
discriminanti, gruppi di Galois di polinomi di grado minore o uguale
a $4$, esempi di polinomi con gruppo di Galois $S_p$, Teorema di
Dedekind (solo enunciato). Applicazioni del Teorema di Dedekind,
come costruire un polinomio con gruppo di Galois $S_n$.

\titoloparagr{Campi ciclotomici.} Definizioni, gruppo di Galois,
sottocampi reali massimali, sottocampi quadratici, gruppi di
Galois, polinomi ciclotomici e loro propriet\`{a}, Teorema della
teoria inversa di Galois per gruppi abeliani.

\titoloparagr{Campi Finiti.} Esistenza e unicit\`{a} dei campi
finiti, gruppo di Galois di un campo finito, sottocampi di un
campo finito, enumerazione dei polinomi irriducibili su campi
finiti. Costruzione della chiusura algebrica di un campo finito con $p$ elementi.

\titoloparagr{Costruzioni con riga e compasso.} Definizione di
punti del piano costruibili, numeri reali costruibili,
caratterizzazione dei punti costruibili in termini di campi,
sottocampi costruibili e costruzione di numeri costruibili,
duplicazione del cubo, trisezione degli angoli, quadratura del
cerchio e Teorema di Gauss per la costruibilit\`{a} degli poligoni
regolari con riga e compasso. Numeri costruibili per origami. La piegatura di Beloch. Duplicazione del
cubo e trisezione di un angolo tramite la piegatura di Beloch.

\testi

\bib
\autore{J. S. Milne} \titolo{Fields and Galois Theory}
\editore{Course Notes} \annopub{2015}
\endbib

\bib
\autore{S. Gabelli} \titolo{Teoria delle Equazioni e Teoria di
Galois.} \editore{Unitext, Springer, ISBN: 978-88-470-0618-8 } 
\annopub{2008} 
\endbib

\altritesti

\bib
\autore{D. Dummit and R. Foote} \titolo{Abstract algebra}
\editore{Prentice Hall, Inc., Englewood Cliffs, NJ} \annopub{1991}
\endbib

\bib
\autore{M. Artin} \titolo{Algebra} \editore{Prentice Hall, Inc.,
Englewood Cliffs, NJ} \annopub{1991}
\endbib

\bib
\autore{T. W. Hungerford} \titolo{Algebra} \editore{Reprint of the
1974 original. Graduate Texts in Mathematics, 73. Springer-Verlag,
New York-Berlin} \annopub{1980}
\endbib

\bib
\autore{N. Jacobson} \titolo{Lectures in abstract algebra. III.
Theory of fields and Galois theory} \editore{Second corrected
printing. Graduate Texts in Mathematics, No. 32. Springer-Verlag,
New York-Heidelberg} \annopub{1975}
\endbib


\bib
\autore{S. Lang} \titolo{Algebra} \editore{Revised third edition.
Graduate Texts in Mathematics, 211. Springer-Verlag, New York}
\annopub{2002}
\endbib

\bib
\autore{J. Rotman} \titolo{Galois theory} \editore{Universitext.
Springer-Verlag, New York} \annopub{1998}
\endbib

\bib
\autore{I. Stewart} \titolo{Galois theory} \editore{Second
edition. Chapman and Hall, Ltd., London} \annopub{1989}
\endbib

\bib
\autore{J. Stillwell} \titolo{Elements of algebra}
\editore{Undergraduate Texts in Mathematics. Springer-Verlag, New
York} \annopub{1994}
\endbib

\esami

Gli studenti sono incoraggiati a richiedere un colloquio orale a
completamento dell'esame. Il docente ha la facolt\`a
di chiedere allo studente di svolgere un colloquio orale di
discussione del compito.\bye
