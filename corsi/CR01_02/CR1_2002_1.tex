\nopagenumbers
\font\title=cmti12
\hsize=20cm
\vsize=28cm
\hoffset=-2.2cm
\voffset=-0.5cm
\parskip=-.1cm
\ \vskip -2cm \hskip -6mm CR1 AA01/02\ (Crittografia a chiave
pubblica)\hfill ESAME DI MET\`{A} SEMESTRE \hfill Roma, 5 Aprile
2002. \hrule
\bigskip\noindent
{\title COGNOME}\  \dotfill\  {\title NOME}\ \dotfill
{\title MATRICOLA}\ \dotfill\
\smallskip  \noindent
Risolvere gli esercizi accompagnando le risposte con spiegazioni
chiare ed essenziali. \it Inserire le risposte negli spazi
predisposti.\hfill\break NON SI ACCETTANO RISPOSTE SCRITTE SU
ALTRI FOGLI. Scrivere il proprio nome anche nell'ultima pagina\rm
Ogni esercizio vale 3 punti.
\smallskip
\hrule
\medskip
\item{1.}
Se $n\in{\bf N}$, sia $\sigma(n)$ la somma dei divisori di $n$. Supponiamo che sia nota
la fattorizzazione (unica) di $n=p_1^{\alpha_1}\cdots p_s^{\alpha_s}$. Calcolare il
numero di operazioni bit necessarie per calcolare $\sigma(n)$. (\it Suggerimento: Usare il
fatto che $\sigma$ \`{e} una funzione moltiplicativa e calcolare una formula per $\sigma(p^\alpha)$ \rm)
\vfill
\item{2.} Mostrare che le moltiplicazioni nell'anello
quoziente ${\bf Z}/n{\bf Z}[x]/(x^d)$ si possono calcolare in
O($\log^2 n^d$) operazioni bit mentre le addizioni in O($\log
n^d$) operazioni bit. \vfill \item{3.} Dato il numero binario
$n=(10011100101)_2$, calcolare $[\sqrt{n}]$ usando l'algoritmo
delle approssimazioni successive (Non passare a base 10 e  non
usare la calcolatrice!) \vfill\eject \ \vskip-2cm \item{4.}
Calcolare il massimo comun divisore tra $240$ e $180$ utilizzando
sia l'algoritmo euclideo che quello binario. Calcolare anche
l'identit\`{a} di Bezout. \vfill
\item{5.} Dimostrare che se $n=p_1\cdots p_{20}$ \`{e} un intero privo di fattori quadratici, e $f(x)
\in{\bf Z}/n{\bf Z}[x]$ ha grado 10, allora la congruenza $f(x)\equiv0\bmod n$ \`{e} risolvibile
se e solo se lo sono le $20$ congruenze
$\cases{f(x)\equiv0\bmod p_1\cr \vdots \cr f(x)\equiv0\bmod p_{20}.}$
Dedurre che la prima congruenza $f(x)\equiv0\bmod n$ ha al pi\`{u} $10^{20}$ soluzioni. Sapreste dare
un esempio in cui le soluzioni sono esattamente $10^{20}$?
\vfill
\item{6.} Illustrare l'algoritmo dei quadrati successivi in un gruppo analizzandone la complessit\`{a}. Fare anche
un esempio.
\vfill\eject
\ \vskip-2cm
\item{7.} Mettere in ordine di priorit\`{a} e spiegare il significato di ciascuna delle
seguenti operazioni:
$$x\!\sim\hskip1cm x\wedge y\hskip1cm x\& y\hskip1cm x\!+\!+\hskip1cm x\backslash y\hskip1cm
 x=y\hskip1cm x\% y\hskip1cm x|y\hskip1cm x\ll n$$
\vfill
\item{8.} Si dia la definizione di pseudo primo forte in base $2$ e si mostri che
se $n=2^\alpha+1$ \`{e} pseudo primo forte in base $2$, allora
$2^{2^\beta}\equiv -1\bmod n$ per qualche $\beta<\alpha$. \vfill
\item{9.} Scrivere un programma in Pari che produca due vettori $v$ e $w$. In cui
$v$ contiene i primi $100$ {\it pseudo-primi composti} in base $2$ e il secondo i primi $100$
{\it pseudo primi di Eulero composti} in base $2$.
\vfill\eject
\ \vskip-2cm
\item{10.}
Implementare RSA utilizzando il sistema Pari e creando tre funzioni distinte (una per
generare le chiavi, una per cifrare e una per decifrare).
\vfill\vfill
\ \vskip 25cm
\centerline{\hskip 8pt\vbox{\tabskip=0pt \offinterlineskip
\def \trl{\noalign{\hrule}}
\halign to580pt{\strut#& \vrule#\tabskip=1em plus 2em&
\hfil#& \vrule#& \hfill#\hfil& \vrule#&
\hfil#& \vrule#& \hfill#\hfil& \vrule#&
\hfil#& \vrule#& \hfill#\hfil& \vrule#&
\hfil#& \vrule#& \hfill#\hfil& \vrule#&
\hfil#& \vrule#& \hfill#\hfil& \vrule#&
\hfil#& \vrule#&
\hfil#& \vrule#\tabskip=0pt\cr\trl
&& NOME E COGNOME && 1 && 2 && 3 && 4 && 5 && 6 && 7 && 8 && 9 && 10 && TOTALE &\cr\trl
&& &&   &&   &&   &&   &&   &&   &&   &&   &&   &&    &&     &\cr
&& \dotfill &&   &&   &&   &&   &&   &&   &&   &&   &&   &&    &&     &\cr\trl
}}}
 \bye
