\nopagenumbers
\font\title=cmti12
\hsize=20cm
\vsize=28cm
\hoffset=-2.2cm
\voffset=-0.5cm
\parskip=-.1cm
\ \vskip -2cm \hskip -6mm CR1 AA01/02\ (Crittografia a chiave
pubblica)\hfill ESAME FINALE \hfill Roma, 9 Settembre 2002. \hrule
\bigskip\noindent
{\title COGNOME}\  \dotfill\  {\title NOME}\ \dotfill
{\title MATRICOLA}\ \dotfill\
\smallskip  \noindent
Risolvere gli esercizi accompagnando le risposte con spiegazioni
chiare ed essenziali. \it Inserire le risposte negli spazi
predisposti.\hfill\break NON SI ACCETTANO RISPOSTE SCRITTE SU
ALTRI FOGLI. Scrivere il proprio nome anche nell'ultima pagina.\
\rm Ogni esercizio vale 3 punti.
\smallskip
\hrule
\medskip

\item{(1)} Si dia una stima (in funzione del parametro $t$) per il
numero di operazioni bit necessarie al calcolo del determinante di
una matrice $3\times3$ a coefficienti interi in cui gli elementi
della prima colonna sono in valore assoluto minori di $t$, quelli
della seconda colonna sono in valore assoluto minori di $e^t$ e
quelli della terza sono in valore assoluto minori di $t^t$.\vfill


\item{(2)} Si descriva un'algoritmo per calcolare in tempo
polinomiale $2^m(\bmod\ m+1)$. Si stimi anche il numero di
operazioni bit necessarie.\vfill



\item{(3)} Si enunci e dimostri la formula per il numero di
polinomi irriducibili di grado $8$ su ${\bf F}_p$. Quale \`{e} la
probabilit\`{a} che un polinomio di grado 4 monico su ${\bf F}_3$
e che non ammette zeri in ${\bf F}_3$ risulti irriducubile su
${\bf F}_3$? \vfill\eject \ \vskip-2cm

\item{(4)} Descrivere il test di primalit\`{a} di Miller Rabin
spiegandone gli aspetti probabilistici.
  \vfill

\item{(5)} Si descrivano gli ordini degli elementi del campo di spezzamento
del polinomio $x^4+x+1$ su ${\bf F}_2$.
  \vfill

\item{(6)} Spiegare il funzionamento del crittosistema
Massey--Omura e simularne un'applicazione in un campo con 32
elementi.

\vfill\eject \ \vskip-2cm

\item{(7)} Dopo aver verificato che si tratta di una curva
ellittica, determinare l'ordine e la struttura del gruppo dei
punti razionali della curva ellittica su ${\bf F}_{7}$
$$E: y^2=x^3-x+1.$$
determinando l'ordine di ciascun punto.\vfill

\vfill \item{(8)} Sia $m$ un intero tale che $m\equiv17\bmod28$.
Si calcoli (se \`{e} ben definito) il seguente simbolo di Jacobi: $\left({5m+7 \over m^5}\right)$.

 \vfill

\item{(9)} Si scriva un programma Pari che verifichi se un
polinomio di grado $3$ a coefficienti in ${\bf F}_5$ \`{e}
o meno primitivo.\vfill\eject
 \ \vskip-2cm
\item{(10)} Scrivere un programma in pari che implementi il metodo di
fattorizzazione $\rho$ di Pollard.
 \vfill\vfill \
\vskip 25cm \centerline{\hskip 8pt\vbox{\tabskip=0pt
\offinterlineskip
\def \trl{\noalign{\hrule}}
\halign to580pt{\strut#& \vrule#\tabskip=1em plus 2em&
\hfil#& \vrule#& \hfill#\hfil& \vrule#&
\hfil#& \vrule#& \hfill#\hfil& \vrule#&
\hfil#& \vrule#& \hfill#\hfil& \vrule#&
\hfil#& \vrule#& \hfill#\hfil& \vrule#&
\hfil#& \vrule#& \hfill#\hfil& \vrule#&
\hfil#& \vrule#&
\hfil#& \vrule#\tabskip=0pt\cr\trl
&& NOME E COGNOME && 1 && 2 && 3 && 4 && 5 && 6 && 7 && 8 && 9 && 10 && TOTALE &\cr\trl
&& &&   &&   &&   &&   &&   &&   &&   &&   &&   &&    &&     &\cr
&& \dotfill &&   &&   &&   &&   &&   &&   &&   &&   &&   &&    &&     &\cr\trl
}}}
 \bye
