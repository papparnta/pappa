\nopagenumbers

\font\title=cmti12
\hsize=20cm
\vsize=28cm
\hoffset=-2.2cm
\voffset=-0.5cm
\parskip=-.1cm
\ \vskip -2cm \hskip -6mm CR1 AA01/02\ (Crittografia a chiave
pubblica)\hfill SECONDO ESAME \hfill Roma, 5 Giugno 2002. \hrule
\bigskip\noindent
{\title COGNOME}\  \dotfill\  {\title NOME}\ \dotfill
{\title MATRICOLA}\ \dotfill\
\smallskip  \noindent
Risolvere gli esercizi accompagnando le risposte con spiegazioni
chiare ed essenziali. \it Inserire le risposte negli spazi
predisposti.\hfill\break NON SI ACCETTANO RISPOSTE SCRITTE SU
ALTRI FOGLI. Scrivere il proprio nome anche nell'ultima pagina.\
\rm Ogni esercizio vale 3 punti.
\smallskip
\hrule
\medskip
\item{1.}
 Quale \`{e} la probabilit\`{a} che un polinomio
irriducibile $f$ di grado $8$ su ${\bf F}_{7}$ risulti primitivo?
\vfill

\item{2.} Spiegare il metodo di fattorizzazione $p-1$.
 \vfill

 \item{3.} Fissare una radice primitiva di ${\bf F}_{5^2}$ ed
 utilizzarla per simulare un scambio chiavi alla Diffie--Hellmann

  \vfill\eject \ \vskip-2cm

  \item{4.} Fattorizzare $f(x)=(x^{10}+3x^5+1)(x^2+2)(x^2+1)$ su ${\bf
  F}_5$ e dopo averne fissato un campo di spezzamento ${\bf F}$,
  si scrivano tutte le radici di $f(x)$ in ${\bf F}$.
  \vfill

\item{5.} Spiegare il funzionamento del crittosistema
Massey--Omura sul gruppo dei punti razionali di una curva
ellittica.

\vfill
\item{6.} Dopo aver verificato che si tratta di una curva ellittica, determinare (giustificando la risposta)
l'ordine e la struttura del gruppo dei punti razionali della curva ellittica su
${\bf F}_7$
$$y^2=x^3-x+5.$$
\vfill\eject \ \vskip-2cm

\item{7.} Spiegare l'algoritmo di Berlekamp.

\vfill \item{8.} Spiegare il significato delle seguenti funzioni di Pari:

ispseudoprimes$(n)$;\ \
znprimroot$(n)$;\ \
znstar$(n)$;\ \ znorder$(x)$;\ \
ffinit$(p, n, x)$.

 \vfill

\item{9.} Implementare in pari
il crittosistema di El Gamal.


\vfill\eject
\ \vskip-2cm
\item{10.} Dato un gruppo ciclico $G$, sia $g$ un suo generatore e $p$ un primo tale che $p^4\| \#G$. Supponiamo
che $X$ denoti il logaritmo discreto di $\alpha\in G$. Si scriva
uno pseudo codice per calcolare $X\bmod p^4$.
\vfill\vfill
\ \vskip 25cm
\centerline{\hskip 8pt\vbox{\tabskip=0pt \offinterlineskip
\def \trl{\noalign{\hrule}}
\halign to580pt{\strut#& \vrule#\tabskip=1em plus 2em&
\hfil#& \vrule#& \hfill#\hfil& \vrule#&
\hfil#& \vrule#& \hfill#\hfil& \vrule#&
\hfil#& \vrule#& \hfill#\hfil& \vrule#&
\hfil#& \vrule#& \hfill#\hfil& \vrule#&
\hfil#& \vrule#& \hfill#\hfil& \vrule#&
\hfil#& \vrule#&
\hfil#& \vrule#\tabskip=0pt\cr\trl
&& NOME E COGNOME && 1 && 2 && 3 && 4 && 5 && 6 && 7 && 8 && 9 && 10 && TOTALE &\cr\trl
&& &&   &&   &&   &&   &&   &&   &&   &&   &&   &&    &&     &\cr
&& \dotfill &&   &&   &&   &&   &&   &&   &&   &&   &&   &&    &&     &\cr\trl
}}}
 \bye
