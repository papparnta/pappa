\nopagenumbers
\font\title=cmti12
\hsize=20cm
\vsize=28cm
\hoffset=-2.2cm
\voffset=-0.5cm
\parskip=-.1cm
\ \vskip -2cm \hskip -6mm CR1 AA01/02\ (Crittografia a chiave
pubblica)\hfill ESAME FINALE \hfill Roma, 5 Giugno 2002. \hrule
\bigskip\noindent
{\title COGNOME}\  \dotfill\  {\title NOME}\ \dotfill
{\title MATRICOLA}\ \dotfill\
\smallskip  \noindent
Risolvere gli esercizi accompagnando le risposte con spiegazioni
chiare ed essenziali. \it Inserire le risposte negli spazi
predisposti.\hfill\break NON SI ACCETTANO RISPOSTE SCRITTE SU
ALTRI FOGLI. Scrivere il proprio nome anche nell'ultima pagina.\
\rm Ogni esercizio vale 3 punti.
\smallskip
\hrule
\medskip

\item{1.} Si stimi il numero di operazioni bit necessarie a
calcolare la derivata di un polinomio di grado $n^2$ in cui tutti
i coefficienti sono minori di $n$. \vfill


\item{2.} Si risolva il seguente sistema di equazioni di
congruenze $$\cases{x^3\equiv 1\bmod 7 \cr x^2\equiv 1\bmod
5}.$$\vfill



\item{3.} Quale \`{e} la probabilit\`{a} che un polinomio
irriducibile $f$ di grado $8$ su ${\bf F}_{7}$ risulti primitivo?
\vfill\eject \ \vskip-2cm

\item{4.} Si illustri la nozione di pseudo primo di eulero e si
indichi la sua applicazione in crittografia.

  \vfill

  \item{5.} Fattorizzare $f(x)=(x^{10}+3x^5+1)(x^2+2)(x^2+1)$ su ${\bf
  F}_5$ e dopo aver fissato un campo di spezzamento ${\bf F}$ per $f$,
  si scrivano tutte le radici di $f(x)$ in ${\bf F}$.

  \vfill

\item{6.} Spiegare il funzionamento del metodo dello scambio delle
chiavi Diffie--Hellman  sul gruppo dei punti razionali di una
curva ellittica.

\vfill\eject \ \vskip-2cm

\item{7.} Dopo aver verificato che si tratta di una curva ellittica, determinare l'ordine e la
struttura del gruppo dei punti razionali della curva ellittica su
${\bf F}_7$
$$y^2=x^3-x+5.$$
\vfill

\vfill
\item{8.} Si calcoli il seguente simbolo di Jacobi:
$\left({234564 \atop 134431}\right)$.

 \vfill

\item{9.}Scrivere in una sola riga il codice (in PARI) per
ottenere: \itemitem{a}Numero di cifre binarie di $x$; \itemitem{b}
L' inverso aritmetico di $a\in({\bf Z}/{n\bf Z})$*; \itemitem{c}
Un primo con al massimo $m$ cifre binarie. \vfill\eject

\ \vskip-2cm \item{10.} Scrivere un programma in PARI per ottenere
un vettore contenente i numeri minori di $10^{20}$ che sono
pseudoprimi forti, per almeno una base random.

 \vfill\vfill \
\vskip 25cm \centerline{\hskip 8pt\vbox{\tabskip=0pt
\offinterlineskip
\def \trl{\noalign{\hrule}}
\halign to580pt{\strut#& \vrule#\tabskip=1em plus 2em&
\hfil#& \vrule#& \hfill#\hfil& \vrule#&
\hfil#& \vrule#& \hfill#\hfil& \vrule#&
\hfil#& \vrule#& \hfill#\hfil& \vrule#&
\hfil#& \vrule#& \hfill#\hfil& \vrule#&
\hfil#& \vrule#& \hfill#\hfil& \vrule#&
\hfil#& \vrule#&
\hfil#& \vrule#\tabskip=0pt\cr\trl
&& NOME E COGNOME && 1 && 2 && 3 && 4 && 5 && 6 && 7 && 8 && 9 && 10 && TOTALE &\cr\trl
&& &&   &&   &&   &&   &&   &&   &&   &&   &&   &&    &&     &\cr
&& \dotfill &&   &&   &&   &&   &&   &&   &&   &&   &&   &&    &&     &\cr\trl
}}}
 \bye
