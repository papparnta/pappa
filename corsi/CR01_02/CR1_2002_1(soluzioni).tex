\documentclass{article}
\usepackage[italian]{babel}
\textwidth=6.5in
\oddsidemargin=0.1cm
\evensidemargin=-0.9cm
\textheight=9.8in
\topmargin=-2cm
\hfuzz=6pt

\begin{document}

\centerline{{\Large Soluzioni del compito di Crittografia a chiave
pubblica (CR1) }}

\centerline{{\large 5 Aprile 2001 (2 ore)}}

\begin{itemize}
\item[1.] \textbf{
Se $n\in{\bf N}$, sia $\sigma(n)$ la somma dei divisori di $n$.
Supponiamo che sia nota la fattorizzazione (unica) di
$n=p_1^{\alpha_1}\cdots p_s^{\alpha_s}$. Calcolare il numero di
operazioni bit necessarie per calcolare $\sigma(n)$. (\it
Suggerimento: Usare il fatto che $\sigma$ \`{e} una funzione
moltiplicativa e calcolare una formula per $\sigma(p^\alpha)$
\rm)}
\medskip

\noindent\emph{Soluzione.} Dalla moltiplicativit\`{a} di $\sigma$
segue che
$\sigma(n)=\sigma(p_1^{\alpha_1})\cdots\sigma(p_1^{\alpha_1})$.
Inoltre $\sigma(p^a)=\sum_{i\leq a}p^i=\frac{p^{a+1}-1}{p-1}$.
Notare che {\bf Time}$(\sigma(p^a))=O(a^2\log^2 p)$ infatti
l'operazione che richiede il maggior numero di operazioni bit
\`{e} il calcolo di $p^{a+1}$. Questo implica che per calcolare
$\sigma(p_1^{\alpha_1}),\ldots, \sigma(p_s^{\alpha_s})$ sono
necessarie $O(\sum_{i\leq s}\alpha_i^2\log^2p_i)= O(\log^2 n)$
operazioni bit. Quindi, calcolando $\sigma(n)$ moltiplicando gli
$s$ numeri $\sigma(p_1^{\alpha_1}),\ldots, \sigma(p_s^{\alpha_s})$
che sono tutti minori di $n^2$, otteniamo (usando la formula
standard) \textbf{Time}$(\sigma(n))=O(s^2\log^2 n)$. Infine da
$s=O(\log s)$ si ha \textbf{Time}$(\sigma(n))=O(\log^4 n)$. Con un
p\`{o} di lavoro in pi\`{u} si pu\`{o} anche mostrare che
\textbf{Time}$(\sigma(n))=O(\log^3 n)$.
\bigskip

\item[2.] \textbf{Mostrare che le moltiplicazioni nell'anello
quoziente ${\bf Z}/n{\bf Z}[x]/(x^d)$ si possono calcolare in
$O(\log^2 n^d$) operazioni bit mentre le addizioni in $O(\log
n^d$) operazioni bit.}\medskip


\noindent\emph{Soluzione.} Scriviamo
$$f_1=\sum_{j=0}^{d-1}a_jx^j,\ f_2=\sum_{j=0}^{d-1}b_jx^j\in
{\bf Z}/n{\bf Z}[x]/(x^d).$$ Allora
$f_1+f_2=\displaystyle\sum_{j=0}^{d-1}(a_j+b_j)x^j.$ Pertanto
\textbf{Time}$(f_1+f_2)$=\textbf{Time}$(a_0+b_0,\ldots,a_{d-1}+b_{d-1})=O(d\log
n)$. Invece se scriviamo $f_1f_2=\sum_{k=0}^{d-1}c_kx^k$, allora
$c_k=\sum_{i+j=k}a_ib_j$. Inoltre per ogni $k=0,\ldots,d-1$,
\textbf{Time}($c_k))=O(d\log^2n)$ (infatti si tratta di $k$
moltiplicazioni e $k-1$ somme in ${\bf Z}/n{\bf Z}$. Infine
$$\textbf{Time}(f_1f_2)=\textbf{Time}(c_0,\ldots,c_{d-1}))=O(d^2\log^2n).$$
\bigskip


\item[3.] \textbf{Dato il numero binario $n=(10011100101)_2$,
calcolare $[\sqrt{n}]$ usando l'algoritmo delle approssimazioni
successive (Non passare a base 10 e  non usare la
calcolatrice!)}\medskip


\noindent\emph{Soluzione.} Applicando l'algoritmo delle approssimazioni successive
otteniamo:
\begin{center}
\begin{tabular}{ll}
$x_0=(100000)=2^5$ & \\
$x_1=(100000)$     & $x_0+2^4>n$;\\
$x_2=(100000)$    & $x_1+2^3>n$;\\
$x_3=(100000)$    & $x_2+2^2>n$;\\
$x_4=(100010)$    & $x_3+2<n$;\\
$x_5=(100011)$    & $x_4+1<n$;\\
\end{tabular}\end{center}

Quindi $[\sqrt{n}]=100011$.
\bigskip

\item[4.] \textbf{Calcolare il massimo comun divisore tra $240$ e
$180$ utilizzando sia l'algoritmo euclideo che quello binario.
Calcolare anche l'identit\`{a} di Bezout.}\medskip


\noindent\emph{Soluzione.} L'esecuzione dell'algoritmo di Euclide per calcolare $(240,180)$ \`{e}
$$240=180+60;\ \ 180=3\cdot60+0;\ \ (240,180)=60.$$
mentre quella di quello binario \`{e}
$$(240,180)=4(60,45)=4(15,45)=4((45-15)/2,15)=4(15,15)=4(0,15)=4\cdot15=60.$$
L'identit\`{a} di Bezout \`{e} $60=240-180.$
\bigskip

\item[5.] \textbf{Dimostrare che se $n=p_1\cdots p_{20}$ \`{e} un
intero privo di fattori quadratici, e $f(x)\in{\bf Z}/n{\bf Z}[x]$
ha grado 10, allora la congruenza $f(x)\equiv0\bmod n$ \`{e}
risolvibile se e solo se lo sono le $20$ congruenze
$\cases{f(x)\equiv0\bmod p_1\cr \vdots \cr f(x)\equiv0\bmod
p_{20}.}$ Dedurre che la prima congruenza $f(x)\equiv0\bmod n$ ha
al pi\`{u} $10^{20}$ soluzioni. Sapreste dare un esempio in cui le
soluzioni sono esattamente $10^{20}$?}\medskip


\noindent\emph{Soluzione.} \emph{(Se)} Per ogni $i=1,\ldots,20$,
sia $\alpha_i$ una soluzione di $f(x)\equiv0\bmod p_i$ e sia
$\alpha\in{\bf Z}/n{\bf Z}$ una soluzione del sistema di
congruenze
$$\left\{
\begin{array}{l}
X\equiv \alpha_1\bmod p_1\\
\vdots\\
X\equiv \alpha_{20}\bmod p_{20}
\end{array}
\right..$$ Si ha che $f(\alpha)\equiv f(\alpha_i)\equiv 0\bmod
p_i$. Quindi $f(\alpha)$ \`{e} divisibile per  $p_1,\ldots,
p_{20}$ e quindi per $n$.\\
\emph{(Solo se)} Se $\alpha\in{\bf Z}/n{\bf Z}$ \`{e} una
soluzione di $f(x)\equiv 0\bmod n$, allora $\alpha\bmod{p_i}$
\`{e} una soluzione di $f(x)\equiv 0\bmod p_i.$ per ogni
$i=1,\ldots,20$.

Se $f(x)=x(x-1)(x-2)\ldots(x-9)$ e $n=11\cdot13\cdots 89$ (il
prodotto dei 20 primi tra 11 e 89). Allora  $f$ ha esattamente
$10^{20}$ soluzioni modulo $n$. Infatti per ciascuna delle
$10^{20}$ scelte $i_1,\ldots,i_{20}\in\{0,\ldots,9\}$, il sistema
di congruenza
$$\left\{\begin{array}{l}
X\equiv i_1\bmod p_1\\
\vdots\\
X\equiv  i_{20}\bmod p_{20}
\end{array}\right.$$
da luogo (per il Teorema cinese dei resti) a esattamente una
soluzione modulo $n$.
\bigskip

\item[6.] \textbf{Illustrare l'algoritmo dei quadrati successivi
in un gruppo analizzandone la complessit\`{a}. Fare anche un
esempio.}\medskip


\noindent\emph{Soluzione.} Sia $G$ un gruppo (o anche un monoide)
abeliano indicato moltiplicativamente e sia $g\in G$. Per ogni
$k\in{\mathbf{N}}$, idichiamo con ${\texttt{QS}(g,k)}$ l'elemento
$g^k\in G$. Allora scriviamo l'espansione binaria
$k=\sum_{j=0}^t\epsilon_j2^j$ e

\begin{center}
\texttt{
\begin{tabular}{rcl}
QS$(a,k)$ & $=$ & $A=g, B=1$\\
 && for $j=0$ to $t$ do \\
 && if $\epsilon_j=1$ then $B=B\cdot A$ fi\\
 && $A=A^2$\\
 && rof\\
 && Print($B$).
\end{tabular}
}\end{center}

Se $T_G$ \`{e} il massimo numero di operazioni bit necessarie per
moltiplicare due elementi di $G$, allora
\textbf{Time}(${\texttt{QS}(g,k)}$)=$O(tT_G)$.

\emph{Esempio:} $G={\bf Z}/5{\bf Z}$, allora $2^5\bmod
5=2\cdot(2^2)^2\bmod 5=2$ ($2^2=-1, (2^2)^2=-1$).
\bigskip

\item[7.] \textbf{Mettere in ordine di priorit\`{a} e spiegare il significato di ciascuna delle
seguenti operazioni:
$$x\!\sim,\ \  x\wedge y,\ \  x\& y,\ \  x\!+\!+,\ \  x\backslash y,\ \
 x=y,\ \  x\% y,\ \  x|y,\ \  x\ll n.$$}\medskip


\noindent\emph{Soluzione.} Ordine di priorit\`a: 1 massima, 5
minima.
\begin{enumerate}
\item $x++$: $x=x+1$;
\item $x=y$: assegna il valore dell'espressione $y$ ad $x$;
\item $x\;\tilde{ }\;$: trasposta di $x$, vettore o matrice;\\
$x\;\hat{}\;y$: $x$ elevato ad $y$;
\item $x\backslash y$: restituisce $q$ t.c. $x=q * y+r$ (quoziente euclideo);\\
$x\%y$: restituisce $r$ t.c. $x=q* y+r$ (resto euclideo);\\
$x<<n$: shift sinistro di $x$ di $n$ bits, pari a $x*2^n$;
\item $x\&y$: $x$ AND $y$;\\
$x|y$: $x$ OR $y$;
\end{enumerate}
\bigskip

\item{8.} \textbf{Si dia la definizione di pseudo primo forte in
base $2$ e si mostri che se $n=2^\alpha+1$ \`{e} pseudo primo
forte in base $2$, allora $2^{2^\beta}\equiv -1\bmod n$ per
qualche $\beta<\alpha$.}\medskip


\noindent\emph{Soluzione.} $n\in{\bf N}$ si dice
\emph{pseudo--primo forte in base $2$} se, posto
$n-1=2^\alpha\cdot t$ con $t$ dispari, si ha che
$$(2^t,2^{2t},\ldots,2^{2^\alpha t})\equiv_n\left\{
\begin{array}{l}
(1,1,\ldots,1)\\
(2^t,2^{2t},\ldots,2^{2^\beta t},-1,1,\ldots,1)\end{array}
\right.$$ Nel caso in cui $n=2^\alpha+1$, allora $t=1$ e quindi la
prima possibilit\`{a} non si presenta e quindi $\exists\beta<1$
t.c. $2^{2^\beta}\equiv -1\bmod n.$
\bigskip

\item{9.} \textbf{Scrivere un programma in Pari che produca due
vettori $v$ e $w$. In cui $v$ contiene i primi $100$ {\it
pseudo-primi composti} in base $2$ e il secondo i primi $100$ {\it
pseudo primi di Eulero composti} in base $2$.}\medskip

\noindent\emph{Soluzione.}
\begin{center}
\texttt{
\begin{tabular}{lll}
n=1;\\
v=VECTOR(100,i,&UNTIL(k!=1 \& & QUADSUCC(2,n-1,n)==1,\\
                              & &n+=2;\\
                              & &k=MATSIZE(FACTOR(n))[1]);\\
         &n);\\
\\
n=1;\\
w=VECTOR(100,i,&UNTIL(k!=1 \& &(Mod(QUADSUCC(2,(n-1)$>>$1,n),n)== Mod(JAC(2,n),n)),\\
                              & &n+=2;\\
                              & &k=MATSIZE(FACTOR(n))[1]);\\
         &n);\\
\end{tabular}
}
\end{center}
\bigskip

\item{10.} \textbf{Implementare RSA utilizzando il sistema Pari e
creando tre funzioni distinte (una per generare le chiavi, una per
cifrare e una per decifrare).}\medskip


\noindent\emph{Soluzione.}

\texttt{KEYGEN} $(n)$
\begin{itemize}
\item INPUT: $n=$numero di bits del modulo pubblico; \item OUTPUT:
stampa le chiavi pubbliche $(e,m)$, ritorna la chiave privata $d$;
\end{itemize}
\texttt{CIFRA} $(P,e,M)$
\begin{itemize}
\item INPUT: $P=$messaggio in chiaro (formato numerico), $(e,M)$
chiave pubblica; \item OUTPUT: $C=$messaggio cifrato (formato
numerico);
\end{itemize}
\texttt{DECIFRA} $(C,d,M)$
\begin{itemize}
\item INPUT: $C=$messaggio cifrato (formato numerico), $(d,M)$
chiave privata; \item OUTPUT: $P=$messaggio in chiaro (formato
numerico);
\end{itemize}

%\begin{center}
\texttt{
\begin{tabular}{lll}
KEYGEN(n,p,q,M,e,d,phi\_M) &=\\
&p=NEXTPRIME(RANDOM(1$<<$(n$>>$1)));\\
&q=NEXTPRIME(RANDOM(1$<<$(n$>>$1)));\\
&M=p*q;\\
&phi\_M=EULERPHI(M);\\
&UNTIL(GCDBINARIO(e,phi\_M)==1,e=RANDOM(phi\_M));\\
&d=LIFT(Mod(e,phi\_M)$\;\hat{}\;$(-1));\\
&PRINT(e,M);\\
&RETURN(d);\\
\\
CIFRA(P,e,M)=\\
&RETURN(QUADSUCC(P,e,M));\\
\\
DECIFRA(C,d,M)=\\
&RETURN(QUADSUCC(C,d,M));\\
\end{tabular}
}

\bigskip

\end{itemize}
\end{document}
