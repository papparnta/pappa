\nopagenumbers
\font\title=cmti12
\hsize=20cm
\vsize=28cm
\hoffset=-2.2cm
\voffset=-0.5cm
\parskip=-.1cm
\ \vskip -2cm \hskip -6mm CR1 AA01/02\ (Crittografia a chiave
pubblica)\hfill ESAME FINALE \hfill Roma, 9 Luglio 2002. \hrule
\bigskip\noindent
{\title COGNOME}\  \dotfill\  {\title NOME}\ \dotfill
{\title MATRICOLA}\ \dotfill\
\smallskip  \noindent
Risolvere gli esercizi accompagnando le risposte con spiegazioni
chiare ed essenziali. \it Inserire le risposte negli spazi
predisposti.\hfill\break NON SI ACCETTANO RISPOSTE SCRITTE SU
ALTRI FOGLI. Scrivere il proprio nome anche nell'ultima pagina.\
\rm Ogni esercizio vale 3 punti.
\smallskip
\hrule
\medskip

\item{(1)} Sia $m$ un intero dispari. Dopo aver dimostrato che
ammette soluzione, si stimi il numero di operazioni bit necessarie
a risolvere il seguente sistema
$$\cases{X\equiv 1\bmod m\cr X\equiv 2\bmod m+1\cr X\equiv 3 \bmod m+2.}$$ \vfill


\item{(2)} Si descriva un'algoritmo per calcolare $[\sqrt{m}]$,
dove $m\in{\bf N}$ in tempo polinomiale\vfill



\item{(3)} Si descrivano i valori di $a\in{\bf F}_p$ per cui
$x^2+a\in{\bf F}_p[x]$ \`{e} irriducibile e si dimostri che non
\`{e} mai primitivo.\vfill\eject \ \vskip-2cm

\item{(4)} Si dimostri che se $m$ \`{e} un intero dispari
composto, allora esiste sempre un base $a\in U({\bf Z}/m{\bf Z})$
rispetto a cui $m$ non \`{e} pseudo primo di Eulero. Quale \`{e}
l'applicazione di questa propriet\`{a} nei test di primalit\`{a}?

  \vfill

\item{(5)} Fattorizzare
$f(x)=(x^{12}+3x^{4}+1)(x^2+x+2)(x^{10}+x^2+1)$ su ${\bf
  F}_2$ e determinare il numero di elementi del campo di spezzamento di $f$.

  \vfill

\item{(6)} Spiegare il funzionamento del crittosistema RSA e
simularne un'applicazione con un modulo RSA di esattamente tre
cifre.

\vfill\eject \ \vskip-2cm

\item{(7)} Dopo aver verificato che si tratta di una curva
ellittica, determinare l'ordine e la struttura del gruppo dei
punti razionali della curva ellittica su ${\bf F}_{11}$
$$E: y^2=x^3-1.$$
Quale \`{e} l'ordine del punto $(5,5)$ in $E({\bf F}_{11})?$
\vfill

\vfill \item{(8)} Siano $m,n$ interi tali che $m\equiv3\bmod4$,
che $m\equiv2\bmod n$ e che $n\equiv1\bmod8$. Si calcoli il
seguente simbolo di Jacobi: $\left({(5m+n)^3 \over m}\right)$.

 \vfill

\item{(9)} Si scriva un programma Pari che implementi il metodo di
fattorizzazione di Pollard. \vfill\eject

\ \vskip-2cm \item{(10)} Scrivere un programma in pari che
verifichi se un numero $n$ di cui \`{e} nota la fattorizzazione in
primi ($n=p_1\cdots p_t$) \`{e} o meno un numero di Carmichael.

 \vfill\vfill \
\vskip 25cm \centerline{\hskip 8pt\vbox{\tabskip=0pt
\offinterlineskip
\def \trl{\noalign{\hrule}}
\halign to580pt{\strut#& \vrule#\tabskip=1em plus 2em&
\hfil#& \vrule#& \hfill#\hfil& \vrule#&
\hfil#& \vrule#& \hfill#\hfil& \vrule#&
\hfil#& \vrule#& \hfill#\hfil& \vrule#&
\hfil#& \vrule#& \hfill#\hfil& \vrule#&
\hfil#& \vrule#& \hfill#\hfil& \vrule#&
\hfil#& \vrule#&
\hfil#& \vrule#\tabskip=0pt\cr\trl
&& NOME E COGNOME && 1 && 2 && 3 && 4 && 5 && 6 && 7 && 8 && 9 && 10 && TOTALE &\cr\trl
&& &&   &&   &&   &&   &&   &&   &&   &&   &&   &&    &&     &\cr
&& \dotfill &&   &&   &&   &&   &&   &&   &&   &&   &&   &&    &&     &\cr\trl
}}}
 \bye
