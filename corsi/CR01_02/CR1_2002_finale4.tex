\nopagenumbers
\font\title=cmti12
\hsize=20cm
\vsize=28cm
\hoffset=-2.2cm
\voffset=-0.5cm
\parskip=-.1cm
\ \vskip -2cm \hskip -6mm CR1 AA01/02\ (Crittografia a chiave
pubblica)\hfill ESAME FINALE \hfill Roma, 19 Febbraio 2003. \hrule
\bigskip\noindent
{\title COGNOME}\  \dotfill\  {\title NOME}\ \dotfill
{\title MATRICOLA}\ \dotfill\
\smallskip  \noindent
Risolvere gli esercizi accompagnando le risposte con spiegazioni
chiare ed essenziali. \it Inserire le risposte negli spazi
predisposti.\hfill\break NON SI ACCETTANO RISPOSTE SCRITTE SU
ALTRI FOGLI. Scrivere il proprio nome anche nell'ultima pagina.\
\rm Ogni esercizio vale 3 punti.
\smallskip
\hrule
\medskip

\item{1.} Si stimi il numero di operazioni bit necessarie a
calcolare l'integrale a di un polinomio di grado $n$ in cui tutti
i coefficienti sono minori di $e^n$. \vfill


\item{2.} Si risolva il seguente sistema di equazioni di
congruenze $$\cases{x^2\equiv 4\bmod 11 \cr x^3\equiv 2\bmod
5}.$$\vfill



\item{3.} Quale \`{e} la probabilit\`{a} che un polinomio
irriducibile $f$ di grado $6$ su ${\bf F}_{11}$ risulti primitivo?
\vfill\eject \ \vskip-2cm

\item{4.} Si illustri la nozione di pseudo primo forte e se ne
indichi l' applicazione in crittografia.

  \vfill

  \item{5.} Fattorizzare $f(x)=(x^{14}+3x^7+1)(x^2+2)(x^2+1)$ su ${\bf
  F}_7$ e dopo aver fissato un campo di spezzamento ${\bf F}$ per $f$,
  si scrivano tutte le radici di $f(x)$ in ${\bf F}$.

  \vfill

\item{6.} Spiegare il funzionamento del metodo dello scambio delle
chiavi Diffie--Hellman e simularne un applicazione in un campo
finito con $19$ elementi.

\vfill\eject \ \vskip-2cm

\item{7.} Dopo aver verificato che si tratta di una curva
ellittica, determinare l'ordine e la struttura del gruppo dei
punti razionali della curva ellittica su ${\bf F}_5$
$$y^2=x^3-x+1.$$
\vfill

\vfill
\item{8.} Si calcoli il seguente simbolo di Jacobi:
$\left({983932 \atop 72637}\right)$.

 \vfill

\item{9.}Scrivere in una sola riga il codice (in PARI) per
ottenere: \itemitem{a} Numero di cifre decimali di $x$; \itemitem{b}
Il resto della divisione euclidea di $a$ per $b$; \itemitem{c}
Un pi\`{u} piccolo numero primo con $100$ cifre binarie. \vfill\eject

\ \vskip-2cm \item{10.} Scrivere un programma in PARI per ottenere
un vettore contenente i numeri minori di $10^{20}$ che sono
pseudoprimi di Eulero, per almeno una base random.

 \vfill\vfill \
\vskip 25cm \centerline{\hskip 8pt\vbox{\tabskip=0pt
\offinterlineskip
\def \trl{\noalign{\hrule}}
\halign to580pt{\strut#& \vrule#\tabskip=1em plus 2em&
\hfil#& \vrule#& \hfill#\hfil& \vrule#&
\hfil#& \vrule#& \hfill#\hfil& \vrule#&
\hfil#& \vrule#& \hfill#\hfil& \vrule#&
\hfil#& \vrule#& \hfill#\hfil& \vrule#&
\hfil#& \vrule#& \hfill#\hfil& \vrule#&
\hfil#& \vrule#&
\hfil#& \vrule#\tabskip=0pt\cr\trl
&& NOME E COGNOME && 1 && 2 && 3 && 4 && 5 && 6 && 7 && 8 && 9 && 10 && TOTALE &\cr\trl
&& &&   &&   &&   &&   &&   &&   &&   &&   &&   &&    &&     &\cr
&& \dotfill &&   &&   &&   &&   &&   &&   &&   &&   &&   &&    &&     &\cr\trl
}}}
 \bye
