\documentclass[italian,a4paper,11pt]
{article}
\usepackage{babel,amsmath,amssymb,amsbsy,amsfonts,latexsym,exscale,
amsthm,epsf,colordvi,enumerate}

\usepackage[latin1]{inputenc}
\usepackage[all]{xy}
\usepackage{textcomp}
\usepackage{graphicx} 


\newcommand{\Q}{\mathbb{Q}}
\newcommand{\Z}{\mathbb Z}
\newcommand{\R}{\mathbb{R}}
\newcommand{\PP}{\mathbb{P}}
\newcommand{\A}{\mathbb{A}}
\newcommand{\I}{\mathcal{I}}

\newcommand{\F}{\mathbb{F}}
\newcommand{\N}{\mathbb{N}}
\newcommand{\C}{\mathbb{C}}
\newcommand{\T}{\mathcal{T}}
\newcommand{\Zeri}{\mathcal{Z}}
\newcommand{\U}{\mathcal{U}}
\newcommand{\p}{\mathfrak{p}}
\newcommand{\ga}{\mathfrak{a}}
\newcommand{\gb}{\mathfrak{b}}

\newcommand{\q}{\mathfrak{q}}
\newcommand{\m}{\mathfrak{m}}
\newcommand{\X}{\mathbf{X}}

\newcommand{\D}{\mbox{\rm{\textbf{Dom}}}}
\newcommand{\Ze}{\mbox{\rm{\textbf{Rie}}}}

\newcommand{\esse}{\mbox{\rm{\textbf{Spec}}}}
\newcommand{\Ci}{\mathbf{C}}
\newcommand{\Ex}{\textbf{Esercizio}}


\newcommand{\Sse}{\Longleftrightarrow}
\newcommand{\sse}{\Leftrightarrow}
\newcommand{\implica}{\Rightarrow}

\newcommand{\frecdl}{\longrightarrow}
\newcommand{\frecd}{\rightarrow}
\newcommand{\st}{\scriptstyle}
\newcommand{\svol}{\textbf{Svolgimento:}}
\newcommand{\cvd}{\begin{flushright} \qed \end{flushright}}
\newcommand{\acc}{\`}
\begin{document}
\begin{center}

\textbf{Universit\`a degli Studi Roma Tre}\\

\textbf{Corso di Laurea in Matematica, a.a. 2009/2010}\\

\textbf{AL2 - Algebra 2: Gruppi, Anelli e Campi}\\

\textbf{Prof. F. Pappalardi}\\

\textbf{Tutorato 8 - 2 Dicembre 2009}\\

\textbf{Matteo Acclavio, Luca Dell'Anna}\\

www.matematica3.com\\
\end{center}



\vspace{0.4cm}

\vspace{0.4 cm}
\noindent
\begin{Ex}\textbf{ 1.}\\
Si consideri l'ideale $(X)$ in $\Z[X]$. Stabilire se $(X)$ \acc e primo o massimale.

\end{Ex}

\vspace{0.4 cm}
\noindent
\begin{Ex}\textbf{ 2.}\\
Dimostrare che $(X)$ \acc e massimale (primo) in $A[X]$ se e solo se $A$ \acc e un campo (dominio).
\end{Ex}

\vspace{0.4 cm}
\noindent
\begin{Ex}\textbf{ 3.}\\
Sia $d$ un UFD e $a,b,c\in D$. Mostrare che:
\begin{itemize}
\item $a|bc$ e $a$ \acc e irriducibile allora $a|b$ o $a|c$
\item $MCD(a,b)=1$ e $a|bc$ allora $a|c$
\end{itemize}
\end{Ex}

\vspace{0.4 cm}
\noindent
\begin{Ex}\textbf{ 4.}\\
Sia $A$ un anello commutativo unitario e $A[X]$ il relativo anello di polinomi. Dato $I$ ideale di $A$, definiamo $I[X]$ come il sottoinsieme di $A[X]$ costituito da tutti i polinomi a coefficienti in $I$. Mostrare che:
\begin{itemize}
\item $I[X]$ \acc e un ideale di $A[X]$
\item $A[X]/I[X]$ \acc e isomorfo a $(A/I)[X]$
\item $I[X]$ non \acc e massimale
\end{itemize}
\end{Ex}


\vspace{0.4 cm}
\noindent
\begin{Ex}\textbf{ 5.}\\
Si consideri in $\R[X]$ l'insieme $I:=\{f(x)\in \R \ | \ f(\sqrt2)=0 \ e \ f(\sqrt5)=0\}$
\begin{itemize}
\item Provare che $I$ \acc e un ideale di $\R[X]$
\item Provare che $I$ non \acc e un ideale primo
\item Descrivere tutti gli ideali primi che contengono $I$. Sono anche massimali?
\end{itemize}

\end{Ex}


\end{document}
