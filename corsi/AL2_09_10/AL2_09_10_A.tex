\nopagenumbers \font\title=cmti12
\def\ve{\vfill\eject}
\def\vv{\vfill}
\def\vs{\vskip-2cm}
\def\vss{\vskip10cm}
\def\vst{\vskip13.3cm}

%\def\ve{\bigskip\bigskip}
%\def\vv{\bigskip\bigskip}
%\def\vs{}
%\def\vss{}
%\def\vst{\bigskip\bigskip}

\hsize=19.5cm
\vsize=27.58cm
\hoffset=-1.6cm
\voffset=0.5cm
\parskip=-.1cm
\ \vs \hskip -6mm AL2 AA09/10\ (Algebra: gruppi, anelli e campi)\hfill APPELLO A \hfill Roma, 11 Gennaio 2010. \hrule
\bigskip\noindent
{\title COGNOME}\  \dotfill\ {\title NOME}\ \dotfill {\title
MATRICOLA}\ \dotfill\
\smallskip  \noindent
Risolvere il massimo numero di esercizi accompagnando le risposte
con spiegazioni chiare ed essenziali. \it Inserire le risposte
negli spazi predisposti. NON SI ACCETTANO RISPOSTE SCRITTE SU
ALTRI FOGLI. Scrivere il proprio nome anche nell'ultima pagina.
\rm 1 Esercizio = 4 punti. Tempo previsto: 2 ore. Nessuna domanda
durante la prima ora e durante gli ultimi 20 minuti.
\smallskip
\hrule\smallskip
\centerline{\hskip 6pt\vbox{\tabskip=0pt \offinterlineskip
\def \trl{\noalign{\hrule}}
\halign to300pt{\strut#& \vrule#\tabskip=0.7em plus 1em& \hfil#&
\vrule#& \hfill#\hfil& \vrule#& \hfil#& \vrule#& \hfill#\hfil&
\vrule#& \hfil#& \vrule#& \hfill#\hfil& \vrule#& \hfil#& \vrule#&
\hfill#\hfil& \vrule#& \hfil#& \vrule#& \hfill#\hfil& \vrule#&
\hfil#& \vrule#& \hfill#\hfil& \vrule#& \hfil#& \vrule#& \hfil#&
\vrule#\tabskip=0pt\cr\trl && FIRMA && 1 && 2 && 3 && 4 &&
5 && 6 && 7 && 8 && 9 &&  TOT. &\cr\trl && &&   &&
&&     &&   &&   &&   &&   &&   &&    && &\cr &&
\dotfill &&     &&   &&   &&   &&     &&   && && && &&
&\cr\trl }}}
\medskip

\item{1.} Descrivere tutti i sottogruppi normali di $D_4$.\vv

\item{2.} Fornire la definizione di centro $Z(G)$ di un gruppo $G$ e dimostrare che se $\psi\in{\rm Aut}(G)$, allora
$\psi(Z(G))=Z(G)$.\ve\vs

\item{3.} Dimostrare che l'unico omomorfismo $\phi:{\bf Z}_5\mapsto S_4$ \`e quello banale (i.e. $\psi(x)=(1)\ \forall x\in{\bf Z}_5$.\vv

\item{4.} Descrivere il gruppo $D_6$ e dimostrare che \`e isomorfo a un sottogruppo di $S_6$. Dimostrare che non \`e
un sottogruppo normale.\vv

\item{5.} Applicare il Teorema di classificazione dei gruppi abeliani finiti per determinare
tutte le classi di isomorfismo dei gruppi abeliani con $36$ elementi.\ve\vs

\item{6.} Stabilire se l'anello ${\bf Z}_2\times{\bf Z}_4\times{\bf Z}$ \`e unitario e determinare i suoi divisori 
dello zero e le sue unit\`a.
\vv

\item{7.} Definire la nozione di Anello Euclideo e dimostrare che ${\bf Z}$ \`e Euclideo.
\ve \vs

\item{8.} Dimostrare che il polinomio $f(X)=X^4 - 4X^3 + 6X^2 + X + 1\in{\bf Q}[X]$ \`e irriducibile (\it suggerimento:
considerare $f(X+1)$\rm).
\vv\vv

\item{9.} Determinare esplicitamente un campo finito con $27$ elementi (\it i.e. realizzarlo come quoziente tra un anello di
polinomi e un suo ideale\rm ).
\ \vst

 \bye
