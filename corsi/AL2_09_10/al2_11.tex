\documentclass[italian,a4paper,11pt]
{article}
\usepackage{babel,amsmath,amssymb,amsbsy,amsfonts,latexsym,exscale,
amsthm,epsf,colordvi,enumerate}

\usepackage[latin1]{inputenc}
\usepackage[all]{xy}
\usepackage{textcomp}
\usepackage{graphicx} 


\newcommand{\Q}{\mathbb{Q}}
\newcommand{\Z}{\mathbb Z}
\newcommand{\R}{\mathbb{R}}
\newcommand{\PP}{\mathbb{P}}
\newcommand{\A}{\mathbb{A}}
\newcommand{\I}{\mathcal{I}}

\newcommand{\F}{\mathbb{F}}
\newcommand{\N}{\mathbb{N}}
\newcommand{\C}{\mathbb{C}}
\newcommand{\T}{\mathcal{T}}
\newcommand{\Zeri}{\mathcal{Z}}
\newcommand{\U}{\mathcal{U}}
\newcommand{\p}{\mathfrak{p}}
\newcommand{\ga}{\mathfrak{a}}
\newcommand{\gb}{\mathfrak{b}}

\newcommand{\q}{\mathfrak{q}}
\newcommand{\m}{\mathfrak{m}}
\newcommand{\X}{\mathbf{X}}

\newcommand{\D}{\mbox{\rm{\textbf{Dom}}}}
\newcommand{\Ze}{\mbox{\rm{\textbf{Rie}}}}

\newcommand{\esse}{\mbox{\rm{\textbf{Spec}}}}
\newcommand{\Ci}{\mathbf{C}}
\newcommand{\Ex}{\textbf{Esercizio}}


\newcommand{\Sse}{\Longleftrightarrow}
\newcommand{\sse}{\Leftrightarrow}
\newcommand{\implica}{\Rightarrow}

\newcommand{\frecdl}{\longrightarrow}
\newcommand{\frecd}{\rightarrow}
\newcommand{\st}{\scriptstyle}
\newcommand{\svol}{\textbf{Svolgimento:}}
\newcommand{\cvd}{\begin{flushright} \qed \end{flushright}}
\newcommand{\acc}{\`}
\begin{document}
\begin{center}

\textbf{Universit\`a degli Studi Roma Tre}\\

\textbf{Corso di Laurea in Matematica, a.a. 2009/2010}\\

\textbf{AL2 - Algebra 2: Gruppi, Anelli e Campi}\\

\textbf{Prof. F. Pappalardi}\\

\textbf{Tutorato 11 - 4 Gennaio 2010}\\

\textbf{Matteo Acclavio, Luca Dell'Anna}\\

www.matematica3.com\\
\end{center}



\vspace{0.4cm}



\vspace{0.4 cm}
\noindent
\begin{Ex}\textbf{ 1.}\\
Nell'anello dei polinomi $K[X]$ si consideri il polinomio $f(X) = X^2 + X + 1$
\begin{itemize}
\item Decomporre il polinomio nei casi in cui $K = \Z_2, \Z_3,\C,\R.$
\item Detto $I$ l'ideale generato da $f(X)$, si dica quando $K[X]/I$ \acc e un campo.
\end{itemize}
\end{Ex}

\vspace{0.4 cm}
\noindent
\begin{Ex}\textbf{ 2.}\\
Effettuare la divisione euclidea tra $13 + 18i$ e $5 + 3i$ in $\Z[i]$.\\ Mostrare che i possibili
quozienti (ed i rispettivi resti) sono quattro.
\end{Ex}


\vspace{0.4 cm}
\noindent
\begin{Ex}\textbf{ 3.}\\
Dimostrare che se $x$ \acc e irriducibile (primo) allora anche tutti i suoi elementi
associati sono irriducibili (primi).
\end{Ex}


\vspace{0.4 cm}
\noindent
\begin{Ex}\textbf{ 4.}\\
Sia $R$ un anello commutativo ed unitario. Siano $I,J$ due suoi ideali.\\
Sia $I + J := \{x + y$ con $x \in I, y \in J\}$. Sia $\phi : R\longrightarrow R/I \times R/J$, l'applicazione definita come $\phi(r):= (r + I, r + J)$ per ogni $r\in R$.
\begin{itemize}
\item Si dimostri che $I + J$ \acc e un ideale di $R$.
\item Si dimostri che $\phi$ \acc e un omomorfismo unitario di anelli.
\item Si dimostri che $\phi$ \acc e suriettivo se e solo se $I + J = R$.
\item Si dimostri che il nucleo di $\phi$ \acc e $I\cap J$.
\item Nel caso $R = \Z$, $I = 5\Z, J = 12\Z$, si dimostri che\\ $\Z/60\Z = \Z/5\Z \times \Z/12\Z$.
\end{itemize}
\end{Ex}

\vspace{0.4 cm}
\noindent
\begin{Ex}\textbf{ 5.}\\

Si consideri il polinomio $f(X) = X^3 + X + 1 \in \Q[X]$.
\begin{itemize}
\item Verificare che $f(X)$ \acc e irriducibile in $\Q[X]$.
\item Sia $\theta$ una radice reale di $f(X)$ (dire perch\'e esiste); \\ si consideri
l'estensione $\Q(\theta)$ di $\Q$; esprimere ciascuno dei seguenti elementi
attraverso la base $\{1, \theta, \theta ^2\}$:\\
$\theta ^4,\ \ \ \  \theta ^5,\ \ \ \  3\theta ^5 - \theta ^4 + 2,\ \ \ \  (\theta ^2 + 2\theta + 2)^{-1}$.
\end{itemize}

\end{Ex}


\vspace{0.4 cm}
\noindent
\begin{Ex}\textbf{ 6.}\\
Sia $f(X) := 2X^3 + X^2 + 1$ e $A := \Z_3[X]/(f(X))$ .
\begin{itemize}
\item Mostrare che $A$ ha zero divisori;
\item Mostrare che $\alpha := X^3 + (f(X))$ \acc e invertibile in $A$ e determinare il
suo inverso.
\end{itemize}
\end{Ex}


\vspace{0.4 cm}
\noindent
\begin{Ex}\textbf{ 7.}\\
Siano $f_a(X) = X^3 + X^2 + X + a \in  \Z_3[X]$ ed $I_a = (f_a(X))$.
\begin{itemize}
\item Determinare per quali valori di $a $ in $\ \Z_3$ l'anello quoziente\\ 
$R_a = \Z_3[X]/I_a$ \acc e un campo.
\item Mostrare che $(X^5 - X^4) + I_2 \ $ \acc e invertibile in $R_2$ e calcolare il suo
inverso.
\end{itemize}
\end{Ex}

\vspace{0.4 cm}
\noindent
\begin{Ex}\textbf{ 8.}\\
Sia $K$ un campo e consideriamo l'anello $A = K[X;Y]/(X^2;Y^2)$.
\begin{itemize}
\item Dette $x$ e $y$ le classi di $A$ determinate da $X$ e $Y$ , provare che ogni
elemento di $A$ si pu\acc o esprimere in un unico modo nella forma:\\ $axy + bx + cy + d$ con $a, b, c, d \in K$
\item Calcolare il prodotto tra due elementi di $A$ generici.
\item Determinare gli zero divisori di $A$.
\item Determinare gli invertibili di $A$.
\end{itemize}
\end{Ex}

\end{document}
