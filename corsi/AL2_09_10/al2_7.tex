\documentclass[italian,a4paper,11pt]
{article}
\usepackage{babel,amsmath,amssymb,amsbsy,amsfonts,latexsym,exscale,
amsthm,epsf,colordvi,enumerate}

\usepackage[latin1]{inputenc}
\usepackage[all]{xy}
\usepackage{textcomp}
\usepackage{graphicx} 


\newcommand{\Q}{\mathbb{Q}}
\newcommand{\Z}{\mathbb Z}
\newcommand{\R}{\mathbb{R}}
\newcommand{\PP}{\mathbb{P}}
\newcommand{\A}{\mathbb{A}}
\newcommand{\I}{\mathcal{I}}

\newcommand{\F}{\mathbb{F}}
\newcommand{\N}{\mathbb{N}}
\newcommand{\C}{\mathbb{C}}
\newcommand{\T}{\mathcal{T}}
\newcommand{\Zeri}{\mathcal{Z}}
\newcommand{\U}{\mathcal{U}}
\newcommand{\p}{\mathfrak{p}}
\newcommand{\ga}{\mathfrak{a}}
\newcommand{\gb}{\mathfrak{b}}

\newcommand{\q}{\mathfrak{q}}
\newcommand{\m}{\mathfrak{m}}
\newcommand{\X}{\mathbf{X}}

\newcommand{\D}{\mbox{\rm{\textbf{Dom}}}}
\newcommand{\Ze}{\mbox{\rm{\textbf{Rie}}}}

\newcommand{\esse}{\mbox{\rm{\textbf{Spec}}}}
\newcommand{\Ci}{\mathbf{C}}
\newcommand{\Ex}{\textbf{Esercizio}}


\newcommand{\Sse}{\Longleftrightarrow}
\newcommand{\sse}{\Leftrightarrow}
\newcommand{\implica}{\Rightarrow}

\newcommand{\frecdl}{\longrightarrow}
\newcommand{\frecd}{\rightarrow}
\newcommand{\st}{\scriptstyle}
\newcommand{\svol}{\textbf{Svolgimento:}}
\newcommand{\cvd}{\begin{flushright} \qed \end{flushright}}
\newcommand{\acc}{\`}
\begin{document}
\begin{center}

\textbf{Universit\`a degli Studi Roma Tre}\\

\textbf{Corso di Laurea in Matematica, a.a. 2009/2010}\\

\textbf{AL2 - Algebra 2: Gruppi, Anelli e Campi}\\

\textbf{Prof. F. Pappalardi}\\

\textbf{Tutorato 7 - 25 Novembre 2009}\\

\textbf{Matteo Acclavio, Luca Dell'Anna}\\

www.matematica3.com\\
\end{center}



\vspace{0.4cm}

\vspace{0.4 cm}
\noindent
\begin{Ex}\textbf{ 1.}\\
Si considerino in $(\Z,+,\cdot)$ gli insiemi $(3):=\{3z \ | \ z\in\Z\}$, $(7):=\{7h \ | \ h\in\Z\}$, $(9):=\{9l \ | \ l\in\Z\}$, $(21):=\{21f \ | \ f\in\Z\}$. 
\begin{itemize}
\item  Verificare che gli insiemi descritti sono ideali di $\Z$
\item Stabilire quali di essi sono ideali primi
\item Stabilire quali di essi sono ideali massimali
\item Determinare $(21)\cap(9)$, $(3)\cap(7)$, $(3)+(9)$, $(3)+(7)$
\item Descrivere i relativi quozienti determinandone le propriet\acc a (i.e. se sono domini, campi...), calcolare la caratteristica di ogni anello quoziente e descriverne gli ideali
\end{itemize}
\end{Ex}

\vspace{0.4 cm}
\noindent
\begin{Ex}\textbf{ 2.}\\
Si consideri l'applicazione $\varphi:\Z\longrightarrow \Z_7\times\Z_5=:B$,  definito come $\varphi(x):=([x]_7,[x]_5)$. Dimostrare che $\varphi$ \acc e un omomorfismo di anelli. Dire se \acc e iniettivo e/o suriettivo e descriverne il nucleo e l'immagine. Dimostrare inoltre che tutti e soli gli ideali primi di $B$ sono $I:=\{[0]_7\}\times \Z_5$ e $J:=\Z_7\times\{[0]_5\}$. Determinare gli ideali primi $P:=\varphi^{-1}(I)$ e $Q:=\varphi^{-1}(J)$. Descrivere infine $\varphi^{-1}(([5]_7,[2]_5))$
\end{Ex}

\vspace{0.4 cm}
\noindent
\begin{Ex}\textbf{ 3.}\\
Sia $A=\{{m\over{10^t}}\in \Q \ | \ m,t\in Z, \ t\geq 0\}$. 
\begin{itemize}
\item Verificare che $A$ \acc e un sottoanello di $\Q$
\item Determinare gli elementi invertibili di $A$
\item Se $I$ \acc e un ideale di $A$ provare che $I\cap\Z$ \acc e un ideale di $\Z$
\item Provare che se $I\neq J$ sono ideali di $A$ allora $I\cap\Z \neq J\cap\Z$
\item Provare che se $I$ \acc e primo o massimale allora $I\cap\Z$ \acc e primo o massimale
\item Provare che per ogni $p\neq 2,5$, con $p$ primo, $pA$ \acc e un ideale massimale in $A$ e $(p)\cap \Z$ \acc e un ideale primo di $\Z$
\end{itemize}
\end{Ex}

\vspace{0.4 cm}
\noindent
\begin{Ex}\textbf{ 4.}\\
Sia $A$ un anello commutativo unitario e siano $I,J$ ideali di $A$.
\begin{itemize}
\item Si dimostri che $IJ:=\{\sum_{i=1}^na_ib_i \ | \ a_i\in I, \ b_i\in J\}$ \acc e un ideale di $A$ contenuto in $I\cap J$
\item Dimostrare che se $I$ e $J$ sono ideali primi allora $IJ$ \acc e primo se e solo se $J\subseteq I$ oppure $I\subseteq J$ 
\end{itemize}
\end{Ex}


\vspace{0.4 cm}
\noindent
\begin{Ex}\textbf{ 5.}\\
Sia $$A:=\left\{\left(\begin{matrix} a & b \\ 0 & c \end{matrix} \right) \ | \ a,b,c\in \Z \right\}.$$
\begin{itemize}
\item Si dimostri che $A$ \acc e un sottoanello delle matrici quadrate $2\times2$ a coefficienti in $\Z$
\item Dato $n\geq2$ e $$I:=\left\{\left(\begin{matrix} a & b \\ 0 & c \end{matrix} \right) \ | \ a,b,c\in n\Z \right\},$$ dimostrare che $I$ \acc e un ideale bilatero di $A$ e trovare un omomorfismo di anelli $\varphi:A\longrightarrow\Z/n\Z$ tale che $Ker(\varphi)=I$
\item Descrivere gli ideali di $A/I$
\end{itemize}
\end{Ex}


\end{document}
