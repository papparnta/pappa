\nopagenumbers \font\title=cmti12
\def\ve{\vfill\eject}
\def\vv{\vfill}
\def\vs{\vskip-2cm}
\def\vss{\vskip10cm}
\def\vst{\vskip13.3cm}

%\def\ve{\bigskip\bigskip}
%\def\vv{\bigskip\bigskip}
%\def\vs{}
%\def\vss{}
%\def\vst{\bigskip\bigskip}

\hsize=19.5cm
\vsize=27.58cm
\hoffset=-1.6cm
\voffset=0.5cm
\parskip=-.1cm
\ \vs \hskip -6mm AL2 AA09/10\ (Algebra: gruppi, anelli e campi)\hfill ESAME DI FINE SEMESTRE \hfill Roma, 7 Gennaio 2010. \hrule
\bigskip\noindent
{\title COGNOME}\  \dotfill\ {\title NOME}\ \dotfill {\title
MATRICOLA}\ \dotfill\
\smallskip  \noindent
Risolvere il massimo numero di esercizi accompagnando le risposte
con spiegazioni chiare ed essenziali. \it Inserire le risposte
negli spazi predisposti. NON SI ACCETTANO RISPOSTE SCRITTE SU
ALTRI FOGLI. Scrivere il proprio nome anche nell'ultima pagina.
\rm 1 Esercizio = 4 punti. Tempo previsto: 2 ore. Nessuna domanda
durante la prima ora e durante gli ultimi 20 minuti.
\smallskip
\hrule\smallskip
\centerline{\hskip 6pt\vbox{\tabskip=0pt \offinterlineskip
\def \trl{\noalign{\hrule}}
\halign to300pt{\strut#& \vrule#\tabskip=0.7em plus 1em& \hfil#&
\vrule#& \hfill#\hfil& \vrule#& \hfil#& \vrule#& \hfill#\hfil&
\vrule#& \hfil#& \vrule#& \hfill#\hfil& \vrule#& \hfil#& \vrule#&
\hfill#\hfil& \vrule#& \hfil#& \vrule#& \hfill#\hfil& \vrule#&
\hfil#& \vrule#& \hfill#\hfil& \vrule#& \hfil#& \vrule#& \hfil#&
\vrule#\tabskip=0pt\cr\trl && FIRMA && 1 && 2 && 3 && 4 &&
5 && 6 && 7 && 8 && 9 &&  TOT. &\cr\trl && &&   &&
&&     &&   &&   &&   &&   &&   &&    && &\cr &&
\dotfill &&     &&   &&   &&   &&     &&   && && && &&
&\cr\trl }}}
\medskip

\item{1.} Dimostrare che il nucleo di un omomorfismo di anelli (non necessariamente commutativi)
\`e un ideale bilaterale del dominio.\vv

\item{2.} Considerare l'insieme $A=\{n+2m\sqrt{2}| m,n\in{\bf Z}\}$. Dimostrare che $A$
\`e un anello commutativo con unit\`a.\ve\vs

\item{3.} Dimostrare che l'anello ${\bf Z}[\sqrt{-6}]$ non \`e a fattorizzazione unica (suggerimento:
cercare elementi di norma $25$).\vv

\item{4.} Considerare l'applicazione $\Psi: {\bf Z}[X]\rightarrow {\bf Z}_8, f(X)\mapsto f(3)\bmod 8$.
Dopo aver verificato che si tratta di un omomorfismo, se ne calcoli il nucleo e l'immagine. Verificare
se il nucleo \`e un ideale primo di ${\bf Z}[X]$.\vv

\item{5.} Sia $A=\{{n\over 2^\alpha}|\ n\in{\bf Z}, \alpha\in{\bf N}\}$. Dopo aver dimostrato che $A$ \`e un
anello, dimostrare che il suo campo dei quozienti \`e ${\bf Q}$.\ve\vs

\item{6.} Dimostrare che il prodotto di due anelli ha sempre divisori dello zero.
\vv

\item{7.} Dimostrare che l'ideale $\langle 2,X\rangle\subset{\bf Z}[X]$ non \`e principale. Dedurre che ${\bf Z}[X]$ 
non \`e un anello Euclideo.
\ve \vs

\item{8.} Dimostrare che il polinomio $X^4+6X+12\in{\bf Q}[X]$ \`e irriducibile.
\vv\vv

\item{9.} Si consideri il campo con $7$ elementi ${\bf F}_7$ e il  polinomio $f(X)=X^2+1\in{\bf F}_7[X]$. Dimostrare che l'anello
quoziente ${\bf F}_7[x]/(f(X))$ \`e un campo e se ne calcoli il numero di elementi.
\ \vst

 \bye
