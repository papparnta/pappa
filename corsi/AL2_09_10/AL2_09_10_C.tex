\nopagenumbers \font\title=cmti12
\def\ve{\vfill\eject}
\def\vv{\vfill}
\def\vs{\vskip-2cm}
\def\vss{\vskip10cm}
\def\vst{\vskip13.3cm}

%\def\ve{\bigskip\bigskip}
%\def\vv{\bigskip\bigskip}
%\def\vs{}
%\def\vss{}
%\def\vst{\bigskip\bigskip}

\hsize=19.5cm
\vsize=27.58cm
\hoffset=-1.6cm
\voffset=0.5cm
\parskip=-.1cm
\ \vs \hskip -6mm AL2 AA09/10\ (Algebra: gruppi, anelli e campi)\hfill APPELLO C \hfill Roma, 7 Giugno 2010. \hrule
\bigskip\noindent
{\title COGNOME}\  \dotfill\ {\title NOME}\ \dotfill {\title
MATRICOLA}\ \dotfill\
\smallskip  \noindent
Risolvere il massimo numero di esercizi accompagnando le risposte
con spiegazioni chiare ed essenziali. \it Inserire le risposte
negli spazi predisposti. NON SI ACCETTANO RISPOSTE SCRITTE SU
ALTRI FOGLI. Scrivere il proprio nome anche nell'ultima pagina.
\rm 1 Esercizio = 4 punti. Tempo previsto: 2 ore. Nessuna domanda
durante la prima ora e durante gli ultimi 20 minuti.
\smallskip
\hrule\smallskip
\centerline{\hskip 6pt\vbox{\tabskip=0pt \offinterlineskip
\def \trl{\noalign{\hrule}}
\halign to300pt{\strut#& \vrule#\tabskip=0.7em plus 1em& \hfil#&
\vrule#& \hfill#\hfil& \vrule#& \hfil#& \vrule#& \hfill#\hfil&
\vrule#& \hfil#& \vrule#& \hfill#\hfil& \vrule#& \hfil#& \vrule#&
\hfill#\hfil& \vrule#& \hfil#& \vrule#& \hfill#\hfil& \vrule#&
\hfil#& \vrule#& \hfill#\hfil& \vrule#& \hfil#& \vrule#& \hfil#&
\vrule#\tabskip=0pt\cr\trl && FIRMA && 1 && 2 && 3 && 4 &&
5 && 6 && 7 && 8 && 9 &&  TOT. &\cr\trl && &&   &&
&&     &&   &&   &&   &&   &&   &&    && &\cr &&
\dotfill &&     &&   &&   &&   &&     &&   && && && &&
&\cr\trl }}}
\medskip

\item{1.} Determinare tutti i sottogruppi del gruppo dei quaternioni ${\bf H}$.\vv

\item{2.} Sia $(G,+)$ un gruppo abeliano e sia $\psi: G\times G \rightarrow G, (x,y)\mapsto x-y$. Dopo 
aver dimostrato che $\psi$ \`e un omomorfismo, se ne calcoli il nucleo e l'immagine.\ve\vs.

\item{3.} Dimostrare che ogni gruppo ciclico con $n$ elementi \`e isomorfo a ${\bf Z}/n{\bf Z}$.\vv

\item{4.} Dimostrare che $S_4$ constiene due sottogruppi con $4$ elementi non isomorfi.\vv

\item{5.} Dopo aver definito la nozione di ideale primo e di ideale massimale per un anello
commutativo, si fornisca un esempio di un anello e di un suo ideale primo
ma non  massimale.\ve\vs

\item{6.} Considerare l'ideale $I=(5+5i,6)$ di ${\bf Z}[i]$. Dimostrare che $I$ \`e principale
e determinarne un generatore.
\vv

\item{7.} Sia $A$ un anello commutativo ed unitario; un elemento $a\in A$ si dice
nilpotente se esiste $n > 0$ tale che $a^n = 0$. Sia $N(A)$ l’insieme degli
elementi nilpotenti di $A$. Dimostrare che $N(A)$ \`e un ideale di $A$ (detto nilradicale)
e che \`e contenuto in ogni ideale primo di $A$.
\ve \vs

\item{8.} Determinare gli elementi invertibili rispetto al prodotto di ${\bf Z}/6{\bf Z}[X]$.
\vv\vv

\item{9.} Considerare $f(x)=X^3+2X^2+X+2\in{\bf Z}/3{\bf Z}[X]$. Dimostrare che ${\bf Z}/3{\bf Z}[X]/f(X)$
non \`e un campo esibendo un elemento che non \`e invertibile. Quanti elementi ha ${\bf Z}/3{\bf Z}[X]/f(X)$?
\ \vst

 \bye
