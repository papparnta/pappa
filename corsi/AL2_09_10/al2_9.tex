\documentclass[italian,a4paper,11pt]
{article}
\usepackage{babel,amsmath,amssymb,amsbsy,amsfonts,latexsym,exscale,
amsthm,epsf,colordvi,enumerate}

\usepackage[latin1]{inputenc}
\usepackage[all]{xy}
\usepackage{textcomp}
\usepackage{graphicx} 


\newcommand{\Q}{\mathbb{Q}}
\newcommand{\Z}{\mathbb Z}
\newcommand{\R}{\mathbb{R}}
\newcommand{\PP}{\mathbb{P}}
\newcommand{\A}{\mathbb{A}}
\newcommand{\I}{\mathcal{I}}

\newcommand{\F}{\mathbb{F}}
\newcommand{\N}{\mathbb{N}}
\newcommand{\C}{\mathbb{C}}
\newcommand{\T}{\mathcal{T}}
\newcommand{\Zeri}{\mathcal{Z}}
\newcommand{\U}{\mathcal{U}}
\newcommand{\p}{\mathfrak{p}}
\newcommand{\ga}{\mathfrak{a}}
\newcommand{\gb}{\mathfrak{b}}

\newcommand{\q}{\mathfrak{q}}
\newcommand{\m}{\mathfrak{m}}
\newcommand{\X}{\mathbf{X}}

\newcommand{\D}{\mbox{\rm{\textbf{Dom}}}}
\newcommand{\Ze}{\mbox{\rm{\textbf{Rie}}}}

\newcommand{\esse}{\mbox{\rm{\textbf{Spec}}}}
\newcommand{\Ci}{\mathbf{C}}
\newcommand{\Ex}{\textbf{Esercizio}}


\newcommand{\Sse}{\Longleftrightarrow}
\newcommand{\sse}{\Leftrightarrow}
\newcommand{\implica}{\Rightarrow}

\newcommand{\frecdl}{\longrightarrow}
\newcommand{\frecd}{\rightarrow}
\newcommand{\st}{\scriptstyle}
\newcommand{\svol}{\textbf{Svolgimento:}}
\newcommand{\cvd}{\begin{flushright} \qed \end{flushright}}
\newcommand{\acc}{\`}
\begin{document}
\begin{center}

\textbf{Universit\`a degli Studi Roma Tre}\\

\textbf{Corso di Laurea in Matematica, a.a. 2009/2010}\\

\textbf{AL2 - Algebra 2: Gruppi, Anelli e Campi}\\

\textbf{Prof. F. Pappalardi}\\

\textbf{Tutorato 9 - 9 Dicembre 2009}\\

\textbf{Matteo Acclavio, Luca Dell'Anna}\\

www.matematica3.com\\
\end{center}



\vspace{0.4cm}


\vspace{0.4 cm}
\noindent
\begin{Ex}\textbf{ 1.}\\
Dire quali dei seguenti quozienti $\frac{K[x]}{(f(x))}$ sono integri o campi
\begin{itemize}
\item $K=\Z, \Q, \R$						$\ \ \ \ \ \ f(x)=2x^2-2$
\item $K=\Z, \Q, \Z_3$ 					$\ \ \ \ \ f(x)=x^3-3x^2+2x+2$
\item $K=\Z, \Q$ 								$\ \ \ \ \ \ \ \ \ f(x)=2x^3+3x^2-4x+1$
\item $K=\Z$										$\ \ \ \ \ \ \ \ \ \ \ \ f(x)=3x^5+2x^4-6x^3-2x^2+4x+14$
\item $K=\Z, \R, \C$ 						$\ \ \ \ \ f(x)=2x^4+2$
\end{itemize}
\end{Ex}


\vspace{0.4 cm}
\noindent
\begin{Ex}\textbf{ 2.}\\
Sia $\alpha\in \Z$ e sia $\varphi_{\alpha}:=\Z[x] \longrightarrow \C$ t.c. $\varphi_{\alpha}(f(x))= f(\sqrt{\alpha})$,dimostrare che:
\begin{itemize}
\item $\varphi$ \'e un omomorfismo e $Ker(\varphi)=(x^2-\alpha)$
\item Sia $Im(\varphi_{\alpha})=\Z[\sqrt{\alpha}]:=\{a+b\sqrt{\alpha}\ \ \  t.c. \ \ \  a,b\in \Z\}$ dimostrare che $Im(\varphi_{\alpha})=\Z \iff \alpha$ \acc e un quadrato perfetto
\item Dimostrare che $\Z[\sqrt{6}]=\frac{\Z[x]}{Ker(\varphi_{6})}$ e non \acc e UFD
\end{itemize}
\end{Ex}

\vspace{0.4 cm}
\noindent
\begin{Ex}\textbf{ 3.}\\
Sia $K$ un campo, $\alpha \in K$ e $f(x) \in K[x]$, dimostrare che:
\begin{itemize}
\item $\varphi: K[x] \longrightarrow K $ t.c. $\varphi (f(x))=f(\alpha)$ \acc e un ben definito omomorfismo, determinarne il nucleo e immagine
\item Stabilire per quali $I=\left(f(x)\right)$ il quoziente $A=\frac{K[x]}{I}$ \acc e integro
\item Dimostrare che $A$ ammette elementi nilpotenti non banali (ovvero $Nil(A)\neq \{0\}$) $\iff $ $MCD(f(x),f'(x))\neq 1$
\end{itemize}
\end{Ex}
\end{document}
