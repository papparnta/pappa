\nopagenumbers \font\title=cmti12
\def\ve{\vfill\eject}
\def\vv{\vfill}
\def\vs{\vskip-2cm}
\def\vss{\vskip10cm}
\def\vst{\vskip13.3cm}

%\def\ve{\bigskip\bigskip}
%\def\vv{\bigskip\bigskip}
%\def\vs{}
%\def\vss{}
%\def\vst{\bigskip\bigskip}

\hsize=19.5cm
\vsize=27.58cm
\hoffset=-1.6cm
\voffset=0.5cm
\parskip=-.1cm
\ \vs \hskip -6mm AL2 AA09/10\ (Algebra: gruppi, anelli e campi)\hfill APPELLO X \hfill Roma, 13 Settembre 2010. \hrule
\bigskip\noindent
{\title COGNOME}\  \dotfill\ {\title NOME}\ \dotfill {\title
MATRICOLA}\ \dotfill\
\smallskip  \noindent
Risolvere il massimo numero di esercizi accompagnando le risposte
con spiegazioni chiare ed essenziali. \it Inserire le risposte
negli spazi predisposti. NON SI ACCETTANO RISPOSTE SCRITTE SU
ALTRI FOGLI.
\rm 1 Esercizio = 4 punti. Tempo previsto: 2 ore. Nessuna domanda
durante la prima ora e durante gli ultimi 20 minuti.
\smallskip
\hrule\smallskip
\centerline{\hskip 6pt\vbox{\tabskip=0pt \offinterlineskip
\def \trl{\noalign{\hrule}}
\halign to277pt{\strut#& \vrule#\tabskip=0.7em plus 1em& \hfil#&
\vrule#& \hfill#\hfil& \vrule#& \hfil#& \vrule#& \hfill#\hfil&
\vrule#& \hfil#& \vrule#& \hfill#\hfil& \vrule#& \hfil#& \vrule#&
\hfill#\hfil& \vrule#& \hfil#& \vrule#& \hfill#\hfil& \vrule#&
\hfil#& \vrule#& \hfill#\hfil& \vrule#& \hfil#& \vrule#& \hfil#&
\vrule#\tabskip=0pt\cr\trl && FIRMA && 1 && 2 && 3 && 4 &&
5 && 6 && 7 && 8  &&  TOT. &\cr\trl && &&   &&
&&     &&   &&     &&   &&   &&    && &\cr &&
\dotfill &&       &&   &&   &&     &&   && && && &&
&\cr\trl }}}
\medskip

\item{1.} Rispondere alle sequenti domande fornendo una giustificazione di una riga:\bigskip\bigskip\bigskip


\itemitem{a.} \`E vero che se $A$ \`e un anello commutativo e $I\subset A$ \`e un ideale, allora $A/I$ \`e
un anello commutativo?\medskip\bigskip\bigskip

\ \dotfill\ \bigskip\bigskip\bigskip\vfil

\itemitem{b.} E' vero che se $G$ \`e un gruppo abeliano finito e $n$ divide $|G|$, allora $G$
contiene un elemento di ordine $n$?\medskip\bigskip\bigskip

\ \dotfill\ \bigskip\bigskip\bigskip\vfil

\itemitem{c.} \'E vero che se
$F$ \`e un campo, allora $F[X,Y]$ \`e un anello a ideali principali?\medskip\bigskip\bigskip
 
\ \dotfill\ \bigskip\bigskip\bigskip\vfil

\itemitem{d.} \'E vero che esiste un campo con $81$ elementi.\medskip\bigskip\bigskip

\ \dotfill\ \bigskip\bigskip\bigskip


\vfil\eject

\item{2.} Dopo aver fornito la definizione di gruppo ciclico, si dimostri che un gruppo ciclico con $n$ elementi
contiene esattamente $\varphi(n)$ generatori (Qui $\varphi$ indica la funzione di Eulero).\vv

\item{3.} Dimostrare che se $G_1$, $G_2$ e $G_3$ sono gruppi tali che $G_1\le G_2\le G_3$ e $G_1$ \`e normale
in $G_3$, allora $G_1$ \`e normale in $G_2$. Fornire un esempio non banale in cui non vale il viceversa (cio\`e un
esempio di gruppi non banali tali che $G_1$ \`e normale in $G_2$ ma $G_1$ non \`e normale in $G_3$).\ve\vs

\item{4.} Determinare tutti i sottogruppi di $S_3\times {\bf Z}_2$ e per ciascuno stabilire se \`e o meno
normale.\vv

\item{5.} Dopo aver definito la nozione di anello a ideali principali, dimostrare che se $\psi: A\rightarrow B$
\`e un omomorfismo suriettivo di anelli e $A$ \`e a ideali principali, allora anche $B$ \`e a ideali
principali.\ve\vs

\item{6.} Sia $A$ un anello e $I$ un ideale. Dimostrare che $\tilde I=\{a\in A\ {\rm tali\ che\ } \exists n\in{\bf N}, a^n\in I\}$
\`e un ideale contenente $I$. Determinare $\tilde I$ nel caso in cui $A={\bf Z}$ e $I=16{\bf Z}$. 
\vv

\item{7.} Sia $A={\bf Z}[X]$ e sia $S$ l'insieme dei polinomi $f$ di $A$ della forma $f(X)=a_0+a_1X^5+a_2X^{10}+\cdots a_mX^{5m}$.
Dimostrare che $S$ \`e un sottoanello di $A$ e stabilire se \`e un ideale.\vv


\item{8.} Determinare tutti i divisori dello zero dell'anello  $({\bf Z}/5{\bf Z})[X]/(x^2-1)$.
\ve \vs

 \bye

\item{9.} Considerare $f(x)=X^3+2X^2+X+2\in{\bf Z}/3{\bf Z}[X]$. Dimostrare che ${\bf Z}/3{\bf Z}[X]/f(X)$
non \`e un campo esibendo un elemento che non \`e invertibile. Quanti elementi ha ${\bf Z}/3{\bf Z}[X]/f(X)$?
\ \vst
