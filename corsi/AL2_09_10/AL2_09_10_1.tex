\nopagenumbers \font\title=cmti12
\def\ve{\vfill\eject}
\def\vv{\vfill}
\def\vs{\vskip-2cm}
\def\vss{\vskip10cm}
\def\vst{\vskip13.3cm}

%\def\ve{\bigskip\bigskip}
%\def\vv{\bigskip\bigskip}
%\def\vs{}
%\def\vss{}
%\def\vst{\bigskip\bigskip}

\hsize=19.5cm
\vsize=27.58cm
\hoffset=-1.6cm
\voffset=0.5cm
\parskip=-.1cm
\ \vs \hskip -6mm AL2 AA09/10\ (Algebra: gruppi, anelli e campi)\hfill ESAME DI MET\`{A} SEMESTRE \hfill Roma, 6 Novembre 2009. \hrule
\bigskip\noindent
{\title COGNOME}\  \dotfill\ {\title NOME}\ \dotfill {\title
MATRICOLA}\ \dotfill\
\smallskip  \noindent
Risolvere il massimo numero di esercizi accompagnando le risposte
con spiegazioni chiare ed essenziali. \it Inserire le risposte
negli spazi predisposti. NON SI ACCETTANO RISPOSTE SCRITTE SU
ALTRI FOGLI. Scrivere il proprio nome anche nell'ultima pagina.
\rm 1 Esercizio = 4 punti. Tempo previsto: 2 ore. Nessuna domanda
durante la prima ora e durante gli ultimi 20 minuti.
\smallskip
\hrule\smallskip
\centerline{\hskip 6pt\vbox{\tabskip=0pt \offinterlineskip
\def \trl{\noalign{\hrule}}
\halign to300pt{\strut#& \vrule#\tabskip=0.7em plus 1em& \hfil#&
\vrule#& \hfill#\hfil& \vrule#& \hfil#& \vrule#& \hfill#\hfil&
\vrule#& \hfil#& \vrule#& \hfill#\hfil& \vrule#& \hfil#& \vrule#&
\hfill#\hfil& \vrule#& \hfil#& \vrule#& \hfill#\hfil& \vrule#&
\hfil#& \vrule#& \hfill#\hfil& \vrule#& \hfil#& \vrule#& \hfil#&
\vrule#\tabskip=0pt\cr\trl && FIRMA && 1 && 2 && 3 && 4 &&
5 && 6 && 7 && 8 && 9 &&  TOT. &\cr\trl && &&   &&
&&     &&   &&   &&   &&   &&   &&    && &\cr &&
\dotfill &&     &&   &&   &&   &&     &&   && && && &&
&\cr\trl }}}
\medskip

\item{1.} Dimostrare che il numero di trasposizioni in $S_n$ \`e $\left({n\over2}\right)$
e che il numero di $k$--cicli \`e $(k-1)!\left({n\over k}\right)$.\vv

\item{2.} Calcolare il numero di elementi di ordine $10$ di $D_4\times{\bf Z}_{5}$ e dire quale \`e
il massimo degli ordini di tutti gli elementi.\ve\vs

\item{3.} Sia $G$ un gruppo. Dimostrare che se $G/Z(G)$ \`e ciclico, allora $G$ \`e abeliano.\vv

\item{4.} Dimostrare che ${\bf Z}_{n^2}$ e ${\bf Z}_n\times{\bf Z}_n$ non sono isomorfi.\vv

\item{5.} Sia $G$ un gruppo ciclico con $20$ elementi e sia $\phi: G\rightarrow G, g\mapsto g^7$.
Dimostrare che $\phi\in{\rm Aut}(G)$ e calcolarne l'ordine (in ${\rm Aut}(G)$).
\ve\vs

\item{6.} Calcolare il centro di $D_4\times D_4$.
\vv

\item{7.} Determinare tutti i sottogruppi di ${\bf Z}_{12}$.
\ve \vs

\item{8.} In $S_7$ sia $a=(1\ 3\ 5\ 2)$ e $b=(2\ 3)$. Dimostrare che $H=\langle a,b\rangle$ ha
$8$ elementi e stabilire se $H$ \`e un sottogruppo normale di $S_7$.
\vv\vv

\item{9.} Sia $G={\rm GL}_3({\bf F}_3)$ e $H=\left\{\left(\matrix{1&a&b\cr0&1&c\cr0&0&1}\right)\in G\right\}$.
Determinare il numero di elementi di $H$ dopo aver mostrato che \`e un sottogruppo e calcolare $Z(H)$.
\ \vst

 \bye
