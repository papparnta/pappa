\nopagenumbers \font\title=cmti12
\def\ve{\vfill\eject}
\def\vv{\vfill}
\def\vs{\vskip-2cm}
\def\vss{\vskip10cm}
\def\vst{\vskip13.3cm}

%\def\ve{\bigskip\bigskip}
%\def\vv{\bigskip\bigskip}
%\def\vs{}
%\def\vss{}
%\def\vst{\bigskip\bigskip}

\hsize=19.5cm
\vsize=27.58cm
\hoffset=-1.6cm
\voffset=0.5cm
\parskip=-.1cm
\ \vs \hskip -6mm CR410 AA13/14 (Crittografia a chiave pubblica)\hfill ESAME DI FINE SEMESTRE \hfill Roma, 26 Maggio, 2014. \hrule
\bigskip\noindent
{\title Cognome}\  \dotfill\ {\title Nome}\ \dotfill {\title
Matricola}\ \dotfill\
\smallskip  \noindent
Risolvere il massimo numero di esercizi fornendo spiegazioni chiare e sintetiche. \it Inserire le risposte negli spazi
predisposti. NON SI ACCETTANO RISPOSTE SCRITTE SU ALTRI FOGLI.
\rm 1 Esercizio = 4 punti. Tempo previsto: 2 ore. Nessuna domanda durante le prima ora e durante gli ultimi 20 minuti.
\smallskip
\hrule\smallskip
\centerline{\hskip 6pt\vbox{\tabskip=0pt \offinterlineskip
\def \trl{\noalign{\hrule}}
\halign to248pt{\strut#& \vrule#\tabskip=0.7em plus 1em& \hfil#&
\vrule#& \hfill#\hfil& \vrule#& \hfil#& \vrule#& \hfill#\hfil&
\vrule#& \hfil#& \vrule#& \hfill#\hfil& \vrule#& \hfil#& \vrule#&
\hfill#\hfil& \vrule#& \hfil#& \vrule#& \hfill#\hfil& \vrule#&
\hfil#& \vrule#& \hfill#\hfil& \vrule#& \hfil#& \vrule#& \hfil#&
\vrule#\tabskip=0pt\cr\trl && 1 && 2 && 3 && 4 &&
5 && 6 && 7 && 8  && 9 && TOT. &\cr\trl  &&   &&
&&  &&   &&   &&     &&  &&    &&  && &\cr &&       &&   &&      &&   && && && &&
 && && &\cr\trl }}}
\medskip

\item{1.} Rispondere alle seguenti domande con una giustificazione di 1 riga:\bigskip
\itemitem{a.} E' possibile calcolare i simboli di Jacobi senza fattorizzare?\bigskip

\ \dotfill\ \vfil\bigskip\bigskip

\itemitem{b.} I simboli di Jacobi hanno applicazioni in crittografia?\bigskip

\ \dotfill\ \vfil\bigskip\bigskip

\itemitem{c.} E' possibile implementare RSA con un esponente di cifratura pari? perch\`e?\bigskip
 
\ \dotfill\ \vfil\bigskip\bigskip

\itemitem{d.} Dare un esempio di curva ellittica $E/{\bf F}_p$ in cui $\#E({\bf F}_p)$ \`e dispari.\bigskip

\ \dotfill\ \bigskip\bigskip\bigskip

\item{2.} Spiegare il funzionamento del crittosistema RSA. 
\vfill\eject\ \vskip-2cm

\item{3.} Definire la nozione di pseudo primo di Miller Rabin e dimostrare che $91$ \`e pseudo primo di Miller Rabin in base $10$ e in base $22$.\vfill

\item{4.} Calcolare il simbolo di Jacobi $\left({m\over n}\right)$ sapendo che $n\equiv 7\bmod 4m$ e che $m\equiv 3\bmod 28$.\vfill\eject\ \vskip-2cm


\item{5.} Dopo aver definito la nozione di numero di Carmichael, si enunci e dimostri il criterio di Korselt.
\vfill

\item{6.} Sia $E: y^2=x^3+Ax+B$ una curva ellittica su un campo ${\bf F}_p$ di caratteristica maggiore di $3$. 
Dimostrare che se $P=(\alpha,\beta)\in E({\bf F}_p)$ \`e un punto di ordine tre, allora $\alpha$ \`e una radice del polinomio:
$$\Psi_3(X)=3X^4+6AX^2+12BX-A^2$$
\vfill\eject\ \vskip-2cm

\item{7.} Sapendo che una curva ellittica $E$ su ${\bf F}_{101}$ ha un punto di ordine $50$, cosa possiamo dire su $\#E({\bf F}_{101})$?
\vfill

\item{8.} Si determini $\#E({\bf F}_{5^{20}})$ quando $E: y^2=x^3+2x-3$.\vfill

\item{9.} Si dimostri la legge di reciprocit\`a quadratica.

\vfill \eject\bye
