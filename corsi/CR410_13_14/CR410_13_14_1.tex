\nopagenumbers \font\title=cmti12
\def\ve{\vfill\eject}
\def\vv{\vfill}
\def\vs{\vskip-2cm}
\def\vss{\vskip10cm}
\def\vst{\vskip13.3cm}

%\def\ve{\bigskip\bigskip}
%\def\vv{\bigskip\bigskip}
%\def\vs{}
%\def\vss{}
%\def\vst{\bigskip\bigskip}

\hsize=19.5cm
\vsize=27.58cm
\hoffset=-1.6cm
\voffset=0.5cm
\parskip=-.1cm
\ \vs \hskip -6mm CR410 AA13/14 (Crittografia a chiave pubblica)\hfill ESAME DI MET\`A SEMESTRE \hfill Roma, 4 Aprile, 2014. \hrule
\bigskip\noindent
{\title Cognome}\  \dotfill\ {\title Nome}\ \dotfill {\title
Matricola}\ \dotfill\
\smallskip  \noindent
Risolvere il massimo numero di esercizi fornendo spiegazioni chiare e sintetiche. \it Inserire le risposte negli spazi
predisposti. NON SI ACCETTANO RISPOSTE SCRITTE SU ALTRI FOGLI.
\rm 1 Eesrcizio = 4 punti. Tempo previsto: 2 ore. Nessuna domanda durante le prima ora e durante gli ultimi 20 minuti.
\smallskip
\hrule\smallskip
\centerline{\hskip 6pt\vbox{\tabskip=0pt \offinterlineskip
\def \trl{\noalign{\hrule}}
\halign to225pt{\strut#& \vrule#\tabskip=0.7em plus 1em& \hfil#&
\vrule#& \hfill#\hfil& \vrule#& \hfil#& \vrule#& \hfill#\hfil&
\vrule#& \hfil#& \vrule#& \hfill#\hfil& \vrule#& \hfil#& \vrule#&
\hfill#\hfil& \vrule#& \hfil#& \vrule#& \hfill#\hfil& \vrule#&
\hfil#& \vrule#& \hfill#\hfil& \vrule#& \hfil#& \vrule#& \hfil#&
\vrule#\tabskip=0pt\cr\trl && 1 && 2 && 3 && 4 &&
5 && 6 && 7 && 8  && TOT. &\cr\trl  &&   &&
&&     &&   &&     &&  &&    &&  && &\cr &&       &&   &&      &&   && && && &&
 && &\cr\trl }}}
\medskip

\item{1.} Rispondere alle seguenti domande che forniscono una giustificazione di 1 riga:\bigskip
\itemitem{a.} Esistono campi finiti con 48 elementi?\medskip\bigskip\bigskip

\ \dotfill\ \vfil

\itemitem{b.} E' vero che non esistono identit\`a di Bezout con coefficienti a segno discorde?\medskip\bigskip\bigskip

\ \dotfill\ \vfil

\itemitem{c.} Fornire un esempio di campi finiti diversi con 16 elementi.\medskip\bigskip\bigskip
 
\ \dotfill\ \vfil

\itemitem{d.} Scrivere tutti i polinomi primitivi in ${\bf F}_2[x]$ di grado minore uguale a $4$.\medskip\bigskip\bigskip

\ \dotfill\ \bigskip\bigskip\bigskip

\item{2.} Enunciare e dimostrare il Teorema di struttura dei sottocampi di ${\bf F}_{p^n}$. Lo si utilizzi per costruire
un esempio di campo finito con esattamente $6$ sottocampi. 
\vfill\eject\ \vskip-2cm

\item{3.} Determinare tutti le radici primitive di ${\bf F}_{5}[\tau], \tau^2=2$.\vfill

\item{4.} Spiegare il funzionamento di alcuni sistemi crittografici che basano la propria sicurezza sul problema del 
logaritmo discreto.\vfill\eject\ \vskip-2cm


\item{5.} Spiegare in dettaglio if funzionamento dell'Algoritmo Pohlig--Hellman.
\vfill

\item{6.} Si applichi l'algoritmo delle approssimazioni successive per calcolare la parte intera del numero binario $\sqrt{101011101}$
\vfill\eject\ \vskip-2cm

\item{7.} Si determini il grado del campo di spezzamento su ${\bf F}_3$ del sequente polinomio $(x^{3^{11}}+6x-x^9+30)(x^6+1)(x^9+15x-1)$
\vfill

\item{8.} Calcolare il massimo comun divisore  $\gcd(273,130)$ utilizzando sia l'algoritmo binario che quello esteso di Euclide. Utilizzare l'algoritmo di Euclide anche per 
calcolare un identit\`a di Bezout.
\vfill \eject\bye
