\nopagenumbers \font\title=cmti12
\def\ve{\vfill\eject}
\def\vv{\vfill}
\def\vs{\vskip-2cm}
\def\vss{\vskip10cm}
\def\vst{\vskip13.3cm}

% \def\ve{\bigskip\bigskip}
% \def\vv{\bigskip\bigskip}
% \def\vs{}
% \def\vss{}
% \def\vst{\bigskip\bigskip}

\hsize=19.5cm
\vsize=27.58cm
\hoffset=-1.6cm
\voffset=0.5cm
\parskip=-.1cm
\ \vs \hskip -6mm CR410 AA13/14\ (Crittografia 1)\hfill APPELLO A \hfill Roma, 3 GIUGNO 2014. \hrule
\bigskip\noindent
{\title COGNOME}\  \dotfill\ {\title NOME}\ \dotfill {\title
MATRICOLA}\ \dotfill\
\smallskip  \noindent
Risolvere il massimo numero di esercizi accompagnando le risposte
con spiegazioni chiare ed essenziali. \it Inserire le risposte
negli spazi predisposti. NON SI ACCETTANO RISPOSTE SCRITTE SU
ALTRI FOGLI. Scrivere il proprio nome anche nell'ultima pagina.
\rm 1 Esercizio = 4 punti. Tempo previsto: 2 ore. Nessuna domanda
durante la prima ora e durante gli ultimi 20 minuti.
\smallskip
\hrule\smallskip
\centerline{\hskip 6pt\vbox{\tabskip=0pt \offinterlineskip
\def \trl{\noalign{\hrule}}
\halign to400pt{\strut#& \vrule#\tabskip=0.7em plus 2em& \hfil#&
\vrule#& \hfill#\hfil& \vrule#& \hfil#& \vrule#& \hfill#\hfil&
\vrule#& \hfil#& \vrule#& \hfill#\hfil& \vrule#& \hfil#& \vrule#&
\hfill#\hfil& \vrule#& \hfil#& \vrule#& \hfill#\hfil& \vrule#&
\hfil#& \vrule#& \hfill#\hfil& \vrule#& \hfil#& \vrule#& \hfil#&
\vrule#\tabskip=0pt\cr\trl && FIRMA && 1 && 2 && 3 && 4 &&
5 && 6 && 7 && 8 && 9 &&  TOT&\cr\trl && &&   &&
&&     &&   &&   &&   &&   &&   &&    && &\cr &&
\dotfill &&     &&   &&   &&   &&     &&   && && && &&
&\cr\trl }}}
\medskip


%\item{1.}
%Se $n\in{\bf N}$, sia $\varphi(n)$ la funzione di Eulero. Supponiamo che sia nota
%la fattorizzazione (unica) di $n=p_1^{\alpha_1}\cdots p_s^{\alpha_s}$. Stimare il
%numero di operazioni bit necessarie per calcolare $\varphi(n)$.
%\vv
%\item{2.} Stimare in termini di $k$ il numero di operazioni bit necessarie per calcolare $\left[\sqrt{2^{k^k}\bmod 3^k}
%\right]$.
%\vv

\item{1.} Si descriva un algoritmo per calcolare in tempo polinomiale la parte 
          intera di $m^{1/2}$ per ogni intero positivo $m$.\vv

\item{2.} Supponiamo che $e=5$ sia la chiave di cifratura di un crittosistema RSA con modulo $n=53\cdot43$. Si calcoli la chiave $d$ di 
decifratura.\ve\vs

\item{3.} Dimostrare che in ${\bf F}_p$ l'equazione $x^m\equiv1\bmod p$ 
          ammette $\gcd(p-1,m)$ soluzioni. Quante ne ammette in ${\bf Z}/(101\cdot 103){\bf Z}$?\vv

\item{4.} Definire il simbolo di Jacobi ed illustrare un algoritmo polinomiale per calcolarlo.\vv

\item{5.} Spiegare il funzionamento dei protocolli crittografici incontrati nel corso.\ve\vs

\item{6.} Si determini la probabilit\`a che un polinomio irriducibile 
          su ${\bf F}_2$ di grado $8$ risulti primitivo.\vv

\item{7.} Determinare tutti i generatori di ${\bf F}_{5}[\tau], \tau^2=2$ e di ciascuno determinare il polinomio minimo.\ve \vs

\item{8.} Determinare la struttura del gruppo dei punti razionali di una curva ellittica definita su ${\bf F}_{101}$ sapendo che
ha un punto $P$ di ordine $41$.\vv\vv

\item{9.} Siano $E_1: y^2=x^3+x+1$ e $E_2: y^2=x^3+x+4$ due curve definite su ${\bf F}_5$. Dopo aver verificato se sono ellittiche determinarne ka struttura della
          gruppo dei punti razionali su ${\bf F}_5$ e si ${\bf F}_{5^2}$.
 
\ \vst\bye
