\nopagenumbers \font\title=cmti12
\def\ve{\vfill\eject}
\def\vv{\vfill}
\def\vs{\vskip-2cm}
\def\vss{\vskip10cm}
\def\vst{\vskip13.3cm}

% \def\ve{\bigskip\bigskip}
% \def\vv{\bigskip\bigskip}
% \def\vs{}
% \def\vss{}
% \def\vst{\bigskip\bigskip}

\hsize=19.5cm
\vsize=27.58cm
\hoffset=-1.6cm
\voffset=0.5cm
\parskip=-.1cm
\ \vs \hskip -6mm CR410 AA13/14\ (Crittografia 1)\hfill APPELLO C \hfill Roma, 30 GENNAIO 2015 \hrule
\bigskip\noindent
{\title COGNOME}\  \dotfill\ {\title NOME}\ \dotfill {\title
MATRICOLA}\ \dotfill\
\smallskip  \noindent
Risolvere il massimo numero di esercizi accompagnando le risposte
con spiegazioni chiare ed essenziali. \it Inserire le risposte
negli spazi predisposti. NON SI ACCETTANO RISPOSTE SCRITTE SU
ALTRI FOGLI. Scrivere il proprio nome anche nell'ultima pagina.
\rm 1 Esercizio = 3 punti. Tempo previsto: 2 ore. Nessuna domanda
durante la prima ora e durante gli ultimi 20 minuti.
\smallskip
\hrule\smallskip
\centerline{\hskip 6pt\vbox{\tabskip=0pt \offinterlineskip
\def \trl{\noalign{\hrule}}
\halign to560pt{\strut#& \vrule#\tabskip=0.7em plus 2em& \hfil#&
\vrule#& \hfill#\hfil& \vrule#& \hfil#& \vrule#& \hfill#\hfil&
\vrule#& \hfil#& \vrule#& \hfill#\hfil& \vrule#& \hfil#& \vrule#&
\hfill#\hfil& \vrule#& \hfil#& \vrule#& \hfill#\hfil& \vrule#&
\hfil#& \vrule#& \hfill#\hfil& \vrule#& \hfil#& \vrule#& \hfil#&
\vrule#\tabskip=0pt\cr\trl && FIRMA && 1 && 2 && 3 && 4 &&
5 && 6 && 7 && 8 && 9 &&  10 &\cr\trl && &&   &&
&&     &&   &&   &&   &&   &&   &&    && &\cr &&
\dotfill &&     &&   &&   &&   &&     &&   && && && &&
&\cr\trl }}}
\medskip


%\item{1.}
%Se $n\in{\bf N}$, sia $\varphi(n)$ la funzione di Eulero. Supponiamo che sia nota
%la fattorizzazione (unica) di $n=p_1^{\alpha_1}\cdots p_s^{\alpha_s}$. Stimare il
%numero di operazioni bit necessarie per calcolare $\varphi(n)$.
%\vv
%\item{2.} Stimare in termini di $k$ il numero di operazioni bit necessarie per calcolare $\left[\sqrt{2^{k^k}\bmod 3^k}
%\right]$.
%\vv

\item{-} Si descrivano:
\itemitem{-1-} L'algoritmo di Euclide (per l'identita di Bezout) e suo il tempo di esecuzione. ;\vv

\itemitem{-2-} Gli algoritmi per la moltiplicazione degli interi a la loro complessit\`a;\vv

\itemitem{-3-} L'algoritmo Baby Steps Giant Steps per il calcolo dell'ordine di una curva ellittica su un campo  finito;
 \ve\vs
 
\itemitem{-4-} L'algorimo di Pholig--Hellman per il calcolo dei logaritmi discreti; 
\vv

\itemitem{-5-} La varie definizioni di pseudo primi e le loro principali propriet\`a.
\ve\vs

\item{6.} Determinare ordine e struttura di $E({\bf F}_7)$ dove $E: y^2=x^3-1$.\vv

\item{7.} Dopo aver descritto quali sono i fattori irriducibili in ${\bf F}_11[x]$ di $x^{11^6}-x$ ($p$ primo), 
si determini il numero di tali fattori che sono primitivi.\vv

\item{8.} Dopo aver fornito la definizione di numero di Carmichael, si enuncino e dimostrino le principali propriet\`a
dei numeri di Carmichael fornendone esempi.\ve \vs

\item{9.} Dimostrare che su ${\bf F}_q, q$ dispari, c'\`e sempre una curva ellittica con gruppo dei punti razionali non ciclico.
\vv\vv

\item{10.} Si descrivano i principali algoritmi di cifratura e decifratura.
\ \vst

 \bye
