\documentclass{article}

\begin{document}

\centerline{9 Luglio 1999 - ORE 15:00}
\centerline{ESAME DI MATEMATICA APPLICATA 2}
\centerline{Alberto Berretti e Francesco Pappalardi}\bigskip

\begin{enumerate}

\item Sia $f(x)=\sin(x)$.
\begin{enumerate}
\item Si stimi il tempo necessario per calcolare
il valore del polinomio di Taylor di grado 3 intorno a $0$ in
un valore intero $x=n$.
\item  Si stimi il tempo necessario per calcolare
il valore del polinomio di Taylor di grado $k$ intorno a $0$ in
un valore intero $x=n$.
\end{enumerate}

\item Si determini un numero intero positivo $x$ nell'
intervallo $[60, 120]$ tale che
$$\left\{\begin{array}{l}
x\equiv 1\pmod 3 \\
x\equiv 4 \pmod 5 \\
x\equiv 2 \pmod 4 \\
\end{array}\right.$$ \bigskip\bigskip

\item (SIMULAZIONE DI RSA). Sia $p=29$, $q=31$, $n=pq$. Assumiamo
che la chiave (pubblica) di codifica sia $e=13$.
\begin{enumerate}
\item Calcolare la chiave (segreta) di decodifica $d$.
\item Crittografare la parola "{\bf ciao}". (Usare 4 messaggi).
\item Dire se \`e possibile scegliere $e=5$ come
chiave pubblica
\end{enumerate} \bigskip\bigskip

\item Sia $\bf F_3$ il campo finito con 3 elementi.
\begin{enumerate}
\item Determiare tutti i polinomi irriducibili di grado 3 su
  $\bf F_3$.
\item Determinare tutti i polinomi primitivi di grado 3 su $\bf F_3$
\item Si scelga un polinomio irriducibile $f(x)$ non primitivo del
punto (a) e sia $\alpha$ una sua radice primitiva. Determinare
tutte le radici primitive di $\bf F_3(\alpha)$.
\end{enumerate} \bigskip\bigskip
\end{enumerate}

{\bf N.B. \`E consentito l'uso di una calcolatrice non scientifica.}
Tempo concesso 120 minuti.
\end{document}
