\documentclass{article}

\begin{document}
\thispagestyle{empty}
\centerline{1 Settembre 1999 -- ORE 15:30}
\centerline{ESAME DI MATEMATICA APPLICATA 2}
\centerline{Alberto Berretti e Francesco Pappalardi}\bigskip

\begin{enumerate}

\item {\bf (10 punti)} Si dia una stima per il numero di operazioni
bit necessarie al calcolo del determinante di 
una matrice $3\times3$ a coefficienti interi in cui
gli elementi della prima colonna sono in valore 
assoluto minori di $M$, quelli della seconda 
colonna sono in valore assoluto minori di $N$ e
quelli della terza sono in valore assoluto
minori di $L$.

\item {\bf (15 punti)} 
Calcolare la parte intera di $\sqrt{1101010001101}$ utilizzando
l'algo\-ritmo delle approssimazioni successive. (si tratta di un
numero binario.)
\bigskip\bigskip

\item  {\bf (10 punti)} Si determini un numero intero $y$ nell'
intervallo $[-80,0]$ tale che
$$\left\{\begin{array}{l}
y\equiv 2\pmod 3 \\
y\equiv 2 \pmod 7 \\
y\equiv 4 \pmod{11} \\
\end{array}\right.$$
\bigskip\bigskip

\item {\bf (30 punti)} Supponiamo si voler utilizzare RSA per
spedire il messaggio $PERA$ utilizzando un alfabeto di
$22$ lettere (compreso lo spazio).

\begin{enumerate}
\item Scegliere due numeri primi $p$ e $q$ in modo che sia
possibile spedire il messaggio utilizzando un'unica trasmissione.
\item Dopo aver calcolato $n=p\cdot q$ e $\varphi(n)$, si scelga
come esponente di codifica $e=2$ e si codifichi il messaggio.
\item Si scriva il messaggio crittografato in termini dell'
alfabeto.
\end{enumerate}
 \bigskip\bigskip

\item {\bf (35 punti)} Si costruisca un polinomio $f$ di grado $4$ irriducibile
su $\bf F_2$. Si indichi con $\alpha$ una radice di $f$ e con
$\bf F_2[\alpha]$ il campo di spezzamento di $f$.
\begin{enumerate}
\item Si calcolino tutte le radici primitive di $\bf F_2[\alpha]$. 
\item 
Si calcoli il logaritmo discreto di $\alpha^3+\alpha$ in base $\alpha^2+1$.
\item 
Quanti elementi pu\`o avere il campo di spezzamento
di un generico polinomio di grado $4$ su $\bf F_2$?
\end{enumerate} \bigskip\bigskip
\end{enumerate}

{\bf N.B. \`E consentito l'uso di una calcolatrice non scientifica.}
Tempo concesso 120 minuti.
\end{document}
