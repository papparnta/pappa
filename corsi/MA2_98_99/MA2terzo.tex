\documentclass{article}

\begin{document}

\centerline{26 Luglio 1999 - ORE 15:30}
\centerline{ESAME DI MATEMATICA APPLICATA 2}
\centerline{Alberto Berretti e Francesco Pappalardi}\bigskip

\begin{enumerate}

\item Si dia una stima per il numero di operazioni
bit necessarie al calcolo della parte intera 
della norma di un vettore di $\bf R^n$ assumendo
che tutte le coordinate sono minori di $1000000$.

\item 
Calcolare la parte intera di $\sqrt{10110101011}$ utilizzando
l'algoritmo delle approssimazioni successive. (si tratta di un
numero binario.)
\bigskip\bigskip

\item Dimostrare che $n^7-n$ \`e sempre divisibile per $42$.
\bigskip\bigskip

\item Sia $q=37$ e $p=41$. 
\begin{enumerate}
\item Utilizzare $n=pq$ per spedire il 
messaggio ``$MAI$'' utilizzando RSA. Fare le cose in modo che
sia necessario spedire il numero minimo di trasmissioni.
(Scegliere a caso il valore di $e$)
\item decodificare il messaggio ``$NO$'' utilizzando le notazioni precedenti 
e sapendo che il messaggio \`e stato ottenuto utilizzando una
sola trasmissione.
\end{enumerate}
 \bigskip\bigskip

\item Si costruisca un polinomio di grado $3$ irriducibile
su $\bf F_7$.
\begin{enumerate}
\item Si illustri il metodo per calcolare tutti i polinomi
di grado tre irriducibili su $\bf F_7$.
\item Si dica quanti sono i polinomi primitivi su $\bf F_7$
spiegando la ragione della risposta.
\item Quanti elementi pu\`o avere il campo di spezzamento
di un generico polinomio di grado $3$ su $\bf F_7$?
\end{enumerate} \bigskip\bigskip
\end{enumerate}

{\bf N.B. \`E consentito l'uso di una calcolatrice non scientifica.}
Tempo concesso 120 minuti.
\end{document}
