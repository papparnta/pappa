\documentclass{article}
\usepackage{amsmath}
\usepackage{amssymb}
\usepackage{amsfonts}

\begin{document}
\thispagestyle{empty} 
\centerline{30 Settembre 1999 -- ORE 14:00} \centerline{ESAME DI 
MATEMATICA APPLICATA 2} \centerline{Alberto Berretti e Francesco Pappalardi}\bigskip 

\begin{enumerate}

\item {\bf (20 punti)}
Si dia una stima per il numero di operazioni bit necessarie per calcolare
l'inversa di una matrice $4\times4$ in cui tutte le componenti sono $\leq N$.
\bigskip

\item {\bf (20 punti)} 
Si calcoli 
$$62400^{5461} \pmod{61345}$$
utilizzando il metodo dei quadrati successivi.
%RISPOSTA 39205
\bigskip\bigskip

\item {\bf (30 punti)} Supponiamo si voglia implementare 
RSA per spedire il messaggio $FINEMA$ utilizzando una
sola trasmissione.
\begin{enumerate} 
\item Quale \`{e} il valore minimo del modulo RSA $n$ che \`{e}
necessario scegliere?

\item Dare un esempio di una possibile scelta di $n$.

\item Implementare RSA utilizzando come esponente di codifica $e=33$
(Questo implica che $n$ dovr\`{a} essere scelto in modo opportuno).

\item Calcolare l'esponente di decodifica $d$.
\end{enumerate}
 \bigskip\bigskip

\item {\bf (30 punti)} Si fattorizzi il polinomio
$$f(x)=x^9+x^3+x$$
su $\mathbb F_2$.
\begin{enumerate}
\item Quanti elementi ha il campo di spezzamento $\mathbb K$ di $f(x)$.

\item Determinare un polinomio irriducibile $g(x)\in\mathbb F_2[x]$
tale che il suo campo di spezzamento sia $\mathbb K$. 

\item Sia $\alpha$ una radice di $g(x)$. Calcolare se esiste il 
logaritmo discreto in base $\alpha$ di $\alpha+1$.
\end{enumerate} \bigskip\bigskip
\end{enumerate}

{\bf N.B. \`E consentito l'uso di una calcolatrice non scientifica.} Tempo concesso 120 
minuti. 
\end{document}
