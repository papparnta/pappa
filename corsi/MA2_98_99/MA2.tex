\input programma.sty     
\def\abbrcorso{MA2}  
\def\titolocorso{Matematica Applicata: Laboratorio 2}
\def\sottotitolo{Introduzione alla Crittografia} 
\def\docente{Prof. Alberto Berretti e Prof. Francesco Pappalardi}  
\def\crediti{7.5}   
\def\semestre{II}
\def\esoneri{0}
\def\scrittofinale{1}
\def\oralefinale{0}
\def\altreprove{1}

\Intestazione   
\titoloparagr{Terminologia e Crittografia classica.}

Algoritmi di sostituzione e trasposizione. Algoritmi di XOR,
Vigenere, Vernam. Teoria di Shannon, entropia.

\titoloparagr{Introduzione ai protocolli crittografici.}

Algoritmi a chiave privata e a chiave pubblica. Firma digitale 
e funzioni di Hash. Considerazioni sulla lunghezza delle chiavi.

\titoloparagr{Algoritmi a chiave privata (o simmetrica).}

DES (Data Encryption Standard). Modalit\`a di uso del DES
e degli altri algoritmi a chiave privata: triplo DES, ECB, 
CBC, CFB. Algoritmi a blocchi vs. algoritmi di crittografia
di flussi (stream ciphers). RC4 (ARCFOUR).

\titoloparagr{Argomenti di Teoria dei numeri elementare.}

Il concetto di operazione bit tipo somma o sottrazione. Stima 
del numero di operazioni bit (tempo macchina) per eseguire 
le operazioni fondamentali. Algoritmi che convergono in tempo 
esponenziale o polinomiale. Divisibilit\`a. Algoritmo di Euclide
(identit\'a di Bezout) e suo tempo di esecuzione. Congruenze.
Teorema cinese dei resti.

\titoloparagr{RSA. L'algoritmo di Adleman, Shamir e Rivest.}

Formulazione dell'algoritmo e analisi del suo tempo di esecuzione.
Esempi concreti non realistici. Metodi di implementazione pratica.
Errori frequenti nell'implementazione di RSA: Modulo RSA con un
fattore troppo piccolo, Modulo RSA con fattori troppo vicini,
Pubblicazione della chiave di decodifica. Uso di RSA per la firma
digitale.

\titoloparagr{Campi finiti.}

Fatti fondamentali di teoria dei campi. Teorema dell'elemento generatore.
Esistenza e unicit\`a dei campi finiti (campi di spezzamento). Polinomi 
irriducibili e primitivi. Aritmetica in tempo polinomiale sui
campi finiti. Esempi.

\titoloparagr{Logaritmi discreti.}

Funzioni a trappola. Metodo di Diffie Hellman per lo scambio delle
chiavi. Metodo di Massey Omura per la trasmissione dei messaggi.
Il crittositema di ElGamal. DSS (Digital Signature Standard).
Algoritmi per il calcolo dei logaritmi discreti nei campi finiti. 

\titoloparagr{Altri Algoritmi.}

Metodo dello zainetto. Dimostrazioni a conoscenza zero.
isomorfismi di grafi. prove di identit\'a a conoscenza zero.
Algoritmo di Feige-Fiat-Shamir.

\titoloparagr{Aspetti pratici.}

Framework X.509. Autenticit\`a certificazioni. PGP.

\testi  

\bib
\autore{Bruce Schneier}
\titolo{Applied Criptography}
\editore{John Wiley \& Sons, Inc.}
\annopub{1996}
\altro{seconda edizione}
\endbib

\bib
\autore{Neal Koblitz} 
\titolo{A Course in Number Theory and Cryptography} 
\editore{Springer}
\annopub{1994}
\altro{Graduate Texts in Mathematics, No 114}
\endbib

\bib                                             
\autore{Douglas R. Stinson}
\titolo{Cryptography: Theory and Practice} 
\editore{CRC Pr}
\annopub{1995}
\endbib

\altritesti  

\bib
\autore{Jan C. A. van der Lubbe} 
\titolo{Basic Mathods of Cryptography} 
\editore{Cambridge University Press}
\annopub{1988}  
\endbib
 
\bib
\autore{Neal Koblitz} 
\titolo{Algebraic Aspects of Cryptography} 
\editore{Springer}
\annopub{1998}
\altro{Algorithms and Computation in Mathematics, Vol 3}
\endbib

\esami 

Gli studenti sono invitati a svolgere un progetto che consiste nella 
redazione di un programma con un linguaggio di programmazione a
scelta che implementi un algoritmo tra quelli svolti durante il
corso.
\bye

