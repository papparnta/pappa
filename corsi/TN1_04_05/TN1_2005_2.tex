\nopagenumbers \font\title=cmti12
%\def\ve{\vfill\eject}
%\def\vv{\vfill}
%\def\vs{\vskip-2cm}
%\def\vss{\vskip10cm}
%\def\vst{\vskip13.3cm}

\def\ve{\medskip}
\def\vv{\medskip}
\def\vs{}
\def\vss{\medskip}
\def\vst{}
\hsize=19cm
\vsize=27.58cm
\hoffset=-1.6cm
\voffset=0.5cm
\parskip=-.1cm
\ \vs \hskip -6mm TN1 AA04/05\ (Teoria dei Numeri)\hfill ESAME DI
FINE SEMESTRE \hfill Roma, 1 Giugno 2005. \hrule
\bigskip\noindent
{\title COGNOME}\  \dotfill\  {\title NOME}\ \dotfill {\title
MATRICOLA}\ \dotfill\ \smallskip  \noindent Risolvere il massimo
numero di esercizi accompagnando le risposte con spiegazioni chiare
ed essenziali. \it Inserire le risposte negli spazi predisposti. NON
SI ACCETTANO RISPOSTE SCRITTE SU ALTRI FOGLI. Scrivere il proprio
nome anche nell'ultima pagina. \rm 1 Esercizio = 3 punti. Tempo
previsto: 2 ore. Nessuna domanda durante la prima ora e durante gli
ultimi 20 minuti.\smallskip\hrule\medskip\bigskip\bigskip
\item{1.} Spiegare come usare la nozione di simbolo di Jacobi per calcolare il simbolo di Legendre.\vv
\item{2.} Dimostrare che esistono infinite coppie di interi positivi $(a,n)$ con $a<n$ e tali che
il simbolo di Jacobi $\big({a\over n}\big)=1$ ma la congruenza
$X^2\equiv a\bmod n$ non ammette soluzione.\ve\ \vs
\item{3.} Assumere la legge di reciprocit\`{a} quadratica per simboli di Legendre e la si dimostri
per simboli di Jacobi.\vv
\item{4.} Mostrare che se $n$ \`{e} privo di fattori quadratici e
$f$ \`{e} una funzione aritmetica moltiplicativa allora
$$(f*f*f)(n)=f(n)\cdot(\tau*{1\!\!1})(n).$$\ve\ \vs
\item{5.} Enunciare e dimostrare la formula di inversione di M\"obius.\vv
\item{6.} Calcolare $(\varphi*\varphi*\varphi)(2^4)$.\ve\ \vs
\item{7.} Mostrare che se $(x,y,z)$ \`{e} una terna pitagorica, allora $60\mid xyz$.\vv
\item{8.} Enunciare il Teorema di classificazione delle terne pitagoriche positive, primitive e normali.\ve\ \vs
\item{9.} Trovare tutte le 10 soluzioni di $x^2+2y^2=162$.\vv
\item{10.} Scrivere $13940$ come somma di due quadrati in almeno $9$ modi diversi.\ve\ \vs
\item{11.} Mostrare che i numeri della forma $7\cdot 4^e$ non si possono scrivere come somme di tre
quadrati e scriverli come somma di quattro quadrati.\vss
\item{12.} Enunciare e dimostrare il Teorema dei quattro quadrati.\vv\ \vst\vskip-8mm
%\centerline{\hskip 6pt\vbox{\tabskip=0pt \offinterlineskip
%\def \trl{\noalign{\hrule}}
%\halign to500pt{\strut#& \vrule#\tabskip=0.7em plus 1em& \hfil#&
%\vrule#& \hfill#\hfil& \vrule#& \hfil#& \vrule#& \hfill#\hfil&
%\vrule#& \hfil#& \vrule#& \hfill#\hfil& \vrule#& \hfil#& \vrule#&
%\hfill#\hfil& \vrule#& \hfil#& \vrule#& \hfill#\hfil& \vrule#&
%\hfil#& \vrule#& \hfill#\hfil& \vrule#& \hfil#& \vrule#& \hfil#&
%\vrule#\tabskip=0pt\cr\trl && NOME E COGNOME && 1 && 2 && 3 && 4 &&
%5 && 6 && 7 && 8 && 9 && 10 && 11 && 12 &&  TOT. &\cr\trl && &&   &&
%&&     &&   &&   &&   &&   &&   &&    &&   &&   &&  && &\cr &&
%\dotfill &&     &&   &&   &&   &&   &&   &&    &&  &&   && && && &&
%&\cr\trl }}}
\bye
