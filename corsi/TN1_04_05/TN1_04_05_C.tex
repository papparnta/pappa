\nopagenumbers \font\title=cmti12
%\def\ve{\vfill\eject}
%\def\vv{\vfill}
%\def\vs{\vskip-2cm}
%\def\vss{\vskip10cm}
%\def\vst{\vskip13.3cm}

\def\ve{\bigskip}
\def\vv{\bigskip}
\def\vs{}
\def\vss{}
\def\vst{\bigskip}

\hsize=19cm
\vsize=27.58cm
\hoffset=-1.6cm
\voffset=0.5cm
\parskip=-.1cm
\ \vs \hskip -6mm TN1 AA04/05\ (Teoria dei Numeri)\hfill APPELLO C
ESAME SCRITTO \hfill Roma, 16 Settembre 2005. \hrule
\bigskip\noindent
{\title COGNOME}\  \dotfill\  {\title NOME}\ \dotfill {\title
MATRICOLA}\ \dotfill\
\smallskip  \noindent
Risolvere il massimo numero di esercizi accompagnando le risposte
con spiegazioni chiare ed essenziali. \it Inserire le risposte
negli spazi predisposti. NON SI ACCETTANO RISPOSTE SCRITTE SU
ALTRI FOGLI. Scrivere il proprio nome anche nell'ultima pagina.
\rm 1 Esercizio = 3 punti. Tempo previsto: 2 ore. Nessuna domanda
durante la prima ora e durante gli ultimi 20 minuti.
\smallskip
\hrule
\medskip

\item{1.} Costruire tutte le soluzione dell'equazione diofantea
$3X+9Y+6Z=-6$.

\vv \item{2.} Mostrare che se $a\in{\bf Z}$, $m_1,m_2,n\in{\bf N}$
sono tali che $m_1\equiv m_2(\bmod\varphi(n))$ e $n$ non ha fattori
quadratici, allora $a^{m_1}\equiv a^{m_2}(\bmod n)$.

\ve\ \vs \item{3.} Determinare le soluzioni di
$X^4+3X^2+X\equiv0\bmod9$.

\vv

\item{4.} Enunciare e dimostrare il Lemma di Gauss per il calcolo del simbolo di Legendre.

\ve\ \vs

\item{5.} Dimostrare che se $p$ e $q$ sono primi dispari distinti,
allora non esiste una radice primitiva modulo $pq$.

\vv \item{6.} Determinare le soluzioni (se esistono) della
congruenza polinomiale $X^4\equiv 8\bmod 31$.
 \ve\ \vs

\item{7.} Calcolare il seguente simbolo di Jacobi/Legendre:
$\Big({1731\over 2431}\Big)$.

\vv \item{8.} Mostrare che se $n,m\in{\bf N}$ sono tali che
$(n,m)=1$ e se $f\in{\bf Z}[X]$, allora $\#{\cal N}(f,nm)= \#{\cal
N}(f,n)\cdot\#{\cal N}(f,m)$ dove ${\cal N}(f,m)=\{z\in{\bf Z}\ |\
f(z)\equiv0(\bmod m), z\in[0,m)\}$.

\ve\ \vs\item{9.} Enunciare e dimostrare la formula di inversione di
M\"obius e usarla per dimostrare che $\mu*1\!\!1=u$.

\vv

\item{10.} Mostrare che  $x^4+y^4=z^2$ non ha soluzioni non banali.

\ve\ \vs\item{11.} Siano $p$ e $q$ primi distinti tali che $q\equiv
p\equiv1\bmod4$. Mostrare che l'equazione $pq=x^2+y^2$ ammette
almeno due soluzioni distinte $(x,y)\in{\bf N}^2$ con $x\leq y$.

\vss

\item{12.} Mostrare che se $e,k\in{\bf N}$, allora $4^e(7+8k)$ non si pu\`{o}
scrivere come somma di tre quadrati.

%\vv
%
%\ \vst\vskip-8mm
%
%\centerline{\hskip 6pt\vbox{\tabskip=0pt \offinterlineskip
%\def \trl{\noalign{\hrule}}
%\halign to500pt{\strut#& \vrule#\tabskip=0.7em plus 1em&
%\hfil#& \vrule#& \hfill#\hfil& \vrule#&
%\hfil#& \vrule#& \hfill#\hfil& \vrule#&
%\hfil#& \vrule#& \hfill#\hfil& \vrule#&
%\hfil#& \vrule#& \hfill#\hfil& \vrule#&
%\hfil#& \vrule#& \hfill#\hfil& \vrule#&
%\hfil#& \vrule#& \hfill#\hfil& \vrule#&
%\hfil#& \vrule#& \hfil#& \vrule#\tabskip=0pt\cr\trl
%&& NOME E COGNOME && 1 && 2 && 3 && 4 && 5 && 6 && 7 && 8 && 9 && 10 && 11 && 12 &&  TOT. &\cr\trl
%&& &&   &&   &&     &&   &&   &&   &&   &&   &&    &&   &&   &&  && &\cr
%&& \dotfill &&     &&   &&   &&   &&   &&   &&    &&  &&   && && && && &\cr\trl
%}}}
 \bye
