\nopagenumbers \font\title=cmti12
\def\ve{\vfill\eject}
\def\vv{\vfill}
\def\vs{\vskip-2cm}
\def\vss{\vskip10cm}
\def\vst{\vskip13.3cm}

%\def\ve{\bigskip}
%\def\vv{\bigskip}
%\def\vs{}
%\def\vss{}
%\def\vst{\bigskip}

\hsize=19cm
\vsize=27.58cm
\hoffset=-1.6cm
\voffset=0.5cm
\parskip=-.1cm
\ \vs \hskip -6mm TN1 AA04/05\ (Teoria dei Numeri)\hfill ESAME SCRITTO
\hfill Roma, 7 Giugno 2005. \hrule
\bigskip\noindent
{\title COGNOME}\  \dotfill\  {\title NOME}\ \dotfill {\title
MATRICOLA}\ \dotfill\
\smallskip  \noindent
Risolvere il massimo numero di esercizi accompagnando le risposte
con spiegazioni chiare ed essenziali. \it Inserire le risposte
negli spazi predisposti. NON SI ACCETTANO RISPOSTE SCRITTE SU
ALTRI FOGLI. Scrivere il proprio nome anche nell'ultima pagina.
\rm 1 Esercizio = 3 punti. Tempo previsto: 2 ore. Nessuna domanda
durante la prima ora e durante gli ultimi 20 minuti.
\smallskip
\hrule
\medskip

\item{1.} Costruire tutte le soluzioni dell'equazione diofantea
$10X+11Y+12Z=3$.

\vv \item{2.} Enunciare e dimostrare il Teorema Cinese dei Resti e
spiegare come utilizzarlo per ridurre il problema dello studio delle
congruenze polinomiali modulo $n$ a quello delle congruenze
polinomiali modulo una potenza di un numero primo.


\ve\ \vs \item{3.} Determinare il numero di soluzioni di
$2X^3+2X+4\equiv0(\bmod16)$.

\vv

\item{4.} Enunciare e dimostrare il Teorema di Eulero per il calcolo
del simbolo di Legendre.

\ve\ \vs

\item{5.} Determinare (se esistono) tutte le radici primitive di ${\bf Z}/34{\bf
Z}$ e  ${\bf Z}/21{\bf Z}$.


\vv \item{6.} Dato un primo $p$ e un intero $m$, per quanti valori
di $a\in\{0,1,2,\ldots, 2p-1\}$ la congruenza $X^m\equiv a(\bmod p)$
\`{e} risolubile? Giustificare la risposta.
 \ve\ \vs

\item{7.} Calcolare il seguente simbolo di Jacobi/Legendre:
$\Big({2222\over 3137}\Big)$.

\vv \item{8.} Supponiamo che $n$ sia un intero i cui fattori primi
sono tutti congruenti a $1$ modulo $4$ e che $n$ abbia $5$ fattori
primi distinti. Quante soluzioni ha la congruenza $X^2+1\equiv0\bmod
n$?

\ve\ \vs\item{9.} Dimostrare che $\tau^{-1}=\mu*\mu$ giustificando
ogni passaggio.

\vv

\item{10.} Dopo aver definito precisamente la nozione di terna pitagorica
positiva, primitiva e normale (tpppn), enunciare il Teorema di
classificazione delle tpppn.


\ve\ \vs\item{11.} Dato un intero $e$, scrivere $153^e$ come somma
di due quadrati.

\vss

\item{12.} Sia $g(n)$ il minimo intero $k$ tale che ogni intero pu\`{o}
essere scritto come la somma di $k$ $n$--esime potenze. Mostrare che
$g(4)\geq19$. Quanto vale $g(2)$?

\vv

\ \vst\vskip-8mm

\centerline{\hskip 6pt\vbox{\tabskip=0pt \offinterlineskip
\def \trl{\noalign{\hrule}}
\halign to500pt{\strut#& \vrule#\tabskip=0.7em plus 1em&
\hfil#& \vrule#& \hfill#\hfil& \vrule#&
\hfil#& \vrule#& \hfill#\hfil& \vrule#&
\hfil#& \vrule#& \hfill#\hfil& \vrule#&
\hfil#& \vrule#& \hfill#\hfil& \vrule#&
\hfil#& \vrule#& \hfill#\hfil& \vrule#&
\hfil#& \vrule#& \hfill#\hfil& \vrule#&
\hfil#& \vrule#& \hfil#& \vrule#\tabskip=0pt\cr\trl
&& NOME E COGNOME && 1 && 2 && 3 && 4 && 5 && 6 && 7 && 8 && 9 && 10 && 11 && 12 &&  TOT. &\cr\trl
&& &&   &&   &&     &&   &&   &&   &&   &&   &&    &&   &&   &&  && &\cr
&& \dotfill &&     &&   &&   &&   &&   &&   &&    &&  &&   && && && && &\cr\trl
}}}
 \bye
