\nopagenumbers \font\title=cmti12
\def\ve{\vfill\eject}
\def\vv{\vfill}
\def\vs{\vskip-2cm}
\def\vss{\vskip10cm}
\def\vst{\vskip13.3cm}

%\def\ve{\bigskip}
%\def\vv{\bigskip}
%\def\vs{}
%\def\vss{}
%\def\vst{\bigskip}

\hsize=19cm
\vsize=27.58cm
\hoffset=-1.6cm
\voffset=0.5cm
\parskip=-.1cm
\ \vs \hskip -6mm TN1 AA04/05\ (Teoria dei Numeri)\hfill ESAME
SCRITTO \hfill Roma, 21 Luglio 2005. \hrule
\bigskip\noindent
{\title COGNOME}\  \dotfill\  {\title NOME}\ \dotfill {\title
MATRICOLA}\ \dotfill\
\smallskip  \noindent
Risolvere il massimo numero di esercizi accompagnando le risposte
con spiegazioni chiare ed essenziali. \it Inserire le risposte
negli spazi predisposti. NON SI ACCETTANO RISPOSTE SCRITTE SU
ALTRI FOGLI. Scrivere il proprio nome anche nell'ultima pagina.
\rm 1 Esercizio = 3 punti. Tempo previsto: 2 ore. Nessuna domanda
durante la prima ora e durante gli ultimi 20 minuti.
\smallskip
\hrule
\medskip

\item{1.} Descrivere in dettaglio il metodo risolutivo dell'equazione diofantea
$AX+BY+CZ=D$ dove $A,B,C,D\in{\bf Z}$.

\vv \item{2.} Trovare tutte le soluzioni del sistema di congruenze
$\cases{x\equiv 2\bmod 3\cr x\equiv 1\bmod 5\cr 2x\equiv 1\bmod 7 }$
negli intervalli $[100,300]$.


\ve\ \vs \item{3.} Si enunci e dimostri il Teorema del sollevamento
per congruenze polinomiali.

\vv

\item{4.} Dimostrare che se $p$ \`e primo, allora $\Big({-1\over p}\Big)=(-1)^{(p-1)/2}$.

\ve\ \vs

\item{5.} Dimostrare che per ogni primo $p$, esiste una radice primitiva modulo $p$.


\vv \item{6.} Determinare le soluzioni (se esistono della congruenza
polinomiale $X^5\equiv 2\bmod 31$.
 \ve\ \vs

\item{7.} Calcolare il seguente simbolo di Jacobi/Legendre:
$\Big({1919\over 7231}\Big)$.

\vv \item{8.} Mostrare che l'insieme delle funzioni aritmetiche
formano un anello rispetto alla somma naturale e al prodotto di
convoluzione. Determinarne le unit\`{a}.

\ve\ \vs\item{9.} Calcolare $\tau*\tau*\mu(30)$.

\vv

\item{10.} Mostrare che se $x^2+y^2=z^2$ con $x,y,z\in{\bf Z}$, allora $60|xyz$.

\ve\ \vs\item{11.} Dopo aver determinato i valori di $e$ per cui
$35^e$ si pu\`o scrivere come somma di due quadrati, determinare
(per ogni $e$) interi $a$ e $b$ tali che $35^e=a^2+b^2$.

\vss

\item{12.} Enunciare il Teorema di caratterizzazione degli interi
che si possono scrivere come somma di due quadrati. Dare cenni della
dimostrazione.

\vv

\ \vst\vskip-8mm

\centerline{\hskip 6pt\vbox{\tabskip=0pt \offinterlineskip
\def \trl{\noalign{\hrule}}
\halign to500pt{\strut#& \vrule#\tabskip=0.7em plus 1em&
\hfil#& \vrule#& \hfill#\hfil& \vrule#&
\hfil#& \vrule#& \hfill#\hfil& \vrule#&
\hfil#& \vrule#& \hfill#\hfil& \vrule#&
\hfil#& \vrule#& \hfill#\hfil& \vrule#&
\hfil#& \vrule#& \hfill#\hfil& \vrule#&
\hfil#& \vrule#& \hfill#\hfil& \vrule#&
\hfil#& \vrule#& \hfil#& \vrule#\tabskip=0pt\cr\trl
&& NOME E COGNOME && 1 && 2 && 3 && 4 && 5 && 6 && 7 && 8 && 9 && 10 && 11 && 12 &&  TOT. &\cr\trl
&& &&   &&   &&     &&   &&   &&   &&   &&   &&    &&   &&   &&  && &\cr
&& \dotfill &&     &&   &&   &&   &&   &&   &&    &&  &&   && && && && &\cr\trl
}}}
 \bye
