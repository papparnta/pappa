\nopagenumbers \font\title=cmti12
%\def\ve{\vfill\eject}
%\def\vv{\vfill}
%\def\vs{\vskip-2cm}
%\def\vss{\vskip10cm}
%\def\vst{\vskip13.3cm}

\def\ve{\bigskip}
\def\vv{\bigskip}
\def\vs{}
\def\vss{}
\def\vst{\bigskip}

\hsize=19cm
\vsize=27.58cm
\hoffset=-1.6cm
\voffset=0.5cm
\parskip=-.1cm
\ \vs \hskip -6mm TN1 AA04/05\ (Teoria dei Numeri)\hfill APPELLO X
\hfill Roma, 13 Gennaio 2006. \hrule
\bigskip\noindent
{\title COGNOME}\  \dotfill\  {\title NOME}\ \dotfill {\title
MATRICOLA}\ \dotfill\
\smallskip  \noindent
Risolvere il massimo numero di esercizi accompagnando le risposte
con spiegazioni chiare ed essenziali. \it Inserire le risposte
negli spazi predisposti. NON SI ACCETTANO RISPOSTE SCRITTE SU
ALTRI FOGLI. Scrivere il proprio nome anche nell'ultima pagina.
\rm 1 Esercizio = 3 punti. Tempo previsto: 2 ore. Nessuna domanda
durante la prima ora e durante gli ultimi 20 minuti.
\smallskip
\hrule
\medskip

\item{1.} Costruire tutte le soluzioni dell'equazione diofantea
$X-5Y+6Z=-3$.

\vv \item{2.} Enunciare e dimostrare il Teorema Cinese dei Resti e
spiegare come utilizzarlo per ridurre il problema dello studio delle
congruenze polinomiali modulo $n$ a quello delle congruenze
polinomiali modulo una potenza di un numero primo.


\ve\ \vs \item{3.} Determinare il numero di soluzioni di
$32X^{20}+2X^2+8X+4\equiv0(\bmod16)$.

\vv

\item{4.} Enunciare e dimostrare il Teorema di Eulero per il calcolo
del simbolo di Legendre.

\ve\ \vs

\item{5.} Determinare (se esistono) tutte le radici primitive di ${\bf Z}/18{\bf
Z}$ e  ${\bf Z}/15{\bf Z}$.


\vv \item{6.} Dimostrare che in $({\bf Z}/p{\bf Z})^*$ \`{e} un
gruppo ciclico.
 \ve\ \vs

\item{7.} Calcolare il seguente simbolo di Jacobi/Legendre:
$\Big({3335\over 3137}\Big)$. %-1

\vv \item{8.} Supponiamo che $n$ sia un intero i cui fattori primi
sono tutti congruenti a $1$ modulo $4$ e che $n$ abbia $5$ fattori
primi distinti. Quante soluzioni ha la congruenza $X^2+4\equiv0\bmod
n$?

\ve\ \vs\item{9.} Calcolare $(\tau*\mu*\phi)(2340)$ giustificando
ogni passaggio.

\vv

\item{10.} Enunciare de dimostrare il Teorema di classificazione delle terne pitagoriche.

\ve\ \vs\item{11.} Scrivere $2340$ in tutti i modi possibili come
somma di due quadrati di interi positivi. % (6,48), (24,42)

\vss

\item{12.} Enunciare il Teorema dei quattro quadrati e illustrare
cenni della dimostrazione.

 \vv

\ \vst\vskip-8mm

\centerline{\hskip 6pt\vbox{\tabskip=0pt \offinterlineskip
\def \trl{\noalign{\hrule}}
\halign to500pt{\strut#& \vrule#\tabskip=0.7em plus 1em&
\hfil#& \vrule#& \hfill#\hfil& \vrule#&
\hfil#& \vrule#& \hfill#\hfil& \vrule#&
\hfil#& \vrule#& \hfill#\hfil& \vrule#&
\hfil#& \vrule#& \hfill#\hfil& \vrule#&
\hfil#& \vrule#& \hfill#\hfil& \vrule#&
\hfil#& \vrule#& \hfill#\hfil& \vrule#&
\hfil#& \vrule#& \hfil#& \vrule#\tabskip=0pt\cr\trl
&& NOME E COGNOME && 1 && 2 && 3 && 4 && 5 && 6 && 7 && 8 && 9 && 10 && 11 && 12 &&  TOT. &\cr\trl
&& &&   &&   &&     &&   &&   &&   &&   &&   &&    &&   &&   &&  && &\cr
&& \dotfill &&     &&   &&   &&   &&   &&   &&    &&  &&   && && && && &\cr\trl
}}}
 \bye
