\nopagenumbers \font\title=cmti12
%\def\ve{\vfill\eject}
%\def\vv{\vfill}
%\def\vs{\vskip-2cm}
%\def\vss{\vskip10cm}
%\def\vst{\vskip13.3cm}

\def\ve{\medskip}
\def\vv{\medskip}
\def\vs{}
\def\vss{\medskip}
\def\vst{}
%\hsize=19cm
%\vsize=27.58cm
%\hoffset=-1.6cm
%\voffset=0.5cm
%\parskip=-.1cm
\ \vs \hskip -6mm TN1 AA04/05\ (Teoria dei Numeri)\hfill ESAME DI
MET\`{A} SEMESTRE \hfill Roma, 15 Aprile 2005. \hrule
\bigskip\noindent
%{\title COGNOME}\  \dotfill\  {\title NOME}\ \dotfill {\title
%MATRICOLA}\ \dotfill\
\smallskip  \noindent
%Risolvere il massimo numero di esercizi accompagnando le risposte
%con spiegazioni chiare ed essenziali. \it Inserire le risposte
%negli spazi predisposti. NON SI ACCETTANO RISPOSTE SCRITTE SU
%ALTRI FOGLI. Scrivere il proprio nome anche nell'ultima pagina.
\rm 1 Esercizio = 3 punti. Tempo previsto: 2 ore. Nessuna domanda
durante la prima ora e durante gli ultimi 20 minuti.
\smallskip
\hrule
\medskip\bigskip\bigskip

\item{1.} Si determinino tutte le soluzioni intere della seguente equazione: $4X+6Y-10Z=100.$ \vv

\item{2.} Enunciare e dimostrare il Teorema di Wilson.\ve\ \vs

\item{3.} Calcolare il numero delle soluzioni modulo $27$ della seguente congruenza polinomiale:\hfill\break ${(X-1)^2(X^2+X+1)}\equiv0\bmod27.$ \vv

\item{4.} Enunciare il Teorema del sollevamento per congruenze polinomiali.\ve\ \vs

\item{5.} Calcolare le soluzioni del sistema di congruenze: $\cases{2X\equiv 3 \bmod 9\cr X\equiv 3\bmod 5}$ nell'intervallo $[50,150].$ \vv

\item{6.} Dimostrare il Teorema cinese dei resti.\ve\ \vs

\item{7.} Quante e quali soluzioni ha la congruenza $X^{4}\equiv4\bmod 17?$\vv

\item{8.} Quale \`{e} il massimo possibile valore per l'ordine di un intero modulo 35? Giustificare la risposta.\ve\ \vs

\item{9.} Siano $p$ e $q$ primi dispari tali che $p\equiv2\bmod q$ e $p+q\equiv0\bmod 4$. Dimostrare che $\big({q\over p}\big)=1$\vv

\item{10.} Calcolare tutte le radici primitive modulo 25.\ve\ \vs

\item{11.} Calcolare il seguente simbolo di Legendre: $\big({1212\over2213}\big)$.\vss %-1

\item{12.} Sia $p$ un primo, $a$ un intero non divisibile per $p$ e $g$ una radice modulo $p$. Dimostrare che $\big({a\over p}\big)=(-1)^{{\rm ind}_g(a)}$.\vv\ \vst\vskip-8mm

%\centerline{\hskip 6pt\vbox{\tabskip=0pt \offinterlineskip
%\def \trl{\noalign{\hrule}}
%\halign to500pt{\strut#& \vrule#\tabskip=0.7em plus 1em&
%\hfil#& \vrule#& \hfill#\hfil& \vrule#&
%\hfil#& \vrule#& \hfill#\hfil& \vrule#&
%\hfil#& \vrule#& \hfill#\hfil& \vrule#&
%\hfil#& \vrule#& \hfill#\hfil& \vrule#&
%\hfil#& \vrule#& \hfill#\hfil& \vrule#&
%\hfil#& \vrule#& \hfill#\hfil& \vrule#&
%\hfil#& \vrule#& \hfil#& \vrule#\tabskip=0pt\cr\trl
%&& NOME E COGNOME && 1 && 2 && 3 && 4 && 5 && 6 && 7 && 8 && 9 && 10 && 11 && 12 &&  TOT. &\cr\trl
%&& &&   &&   &&     &&   &&   &&   &&   &&   &&    &&   &&   &&  && &\cr
%&& \dotfill &&     &&   &&   &&   &&   &&   &&    &&  &&   && && && && &\cr\trl
%}}}
 \bye
