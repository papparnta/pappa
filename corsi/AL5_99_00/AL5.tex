\input programma.sty
\def\abbrcorso{AL5}
\def\titolocorso{Istituzioni di Algebra Superiore (2$^0$ Modulo)}
\def\sottotitolo{Teoria dei gruppi finiti}
\def\docente{Prof. Francesco Pappalardi}
\def\crediti{7}
\def\semestre{II}
\def\esoneri{1}
\def\scrittofinale{1}
\def\oralefinale{0}
\def\altreprove{1}
\Intestazione

\titoloparagr{Introduzione}

 Esempi: gruppi ciclici, gruppi simmetrici, gruppi diedrali,
gruppi di matrici su campi finiti, tutti i gruppi con $8$ elementi, Teorema di Lagrange,
isomorfismi, sottogruppi normali, centro, equazione delle classi,
centralizzanti, gruppi semplici, serie di decomposizione, Teorema
di Jordan H\"older, programma di H\"older per la classificazione
dei gruppi finiti, sottogruppi coniugati, normalizzanti, sottogruppo
derivato e sue propriet\`a.

\titoloparagr{Azioni dei gruppi sugli insiemi}

Definizione di azione, nucleo di un Azione, esempi, azioni fedeli, orbite,
azioni transitive, stabilizzatori, azioni di $p$--gruppi.

\titoloparagr{Gruppi di permutazioni}

Richiami, gruppi alterni, decomposizione in cicli, generatori,
trasposizioni, parit\`a, il derivato di $S_n$,
semplicit\`a di $A_n$ ($n\geq 5$), applicazioni.

\titoloparagr{Teoremi di Sylow}

Enunciato e dimostrazione dei Teoremi di Sylow, esempi,
$p$--gruppi, criteri di non semplicit\`a,
classificazione dei gruppi semplice con ordine minore di 200.

\titoloparagr{Prodotti semidiretti di Gruppi}

Definizione, diverse caratterizzazioni, gruppi diedrali generalizzati,
gruppi metaciclici, automorfismi dei gruppi diedrali, gruppi di ordine $pq$,
classificazione di gruppi con 12 elementi.

\titoloparagr{Gruppi risolubili e nilpotenti}

Enunciato del
Teorema di struttura per gruppi Abeliani, definizioni, serie centrale
ascendente, serie derivata, esempi, sottogruppi e quozienti
di gruppi risolubili e nilpotenti, enunciato del Teorema di P. Hall.

\titoloparagr{Semplicit\`a dei gruppi di Matrici}

Trasvezioni, le trasvezioni generano $SL$, Azioni $2$--transitive,
criteri di semplicit\`a per gruppi con azioni due transitive,
il sottogruppo derivato di $GL$, $SL$ e $PSL$, semplicit\`a
di $PSL$.

\testi


\bib
\autore{A. Mach\'i}
\titolo{Introduzione alla Teoria dei Gruppi}
\editore{Feltrinelli--Milano}
\annopub{1974}
\endbib

\bib
\autore{Hungeford}
\titolo{Algebra GTM 73}
\editore{Springer}
\annopub{1974}
\endbib

\bib
\autore{F. Pappalardi}
\altro{Note sulla semplicit\`e nei gruppi di matrici}
\endbib

\esami

Oltre ai due test proposti durante il semestre, gli studenti
devono produrre una tesina scritta in \TeX\ di circa 5 pagine
in cui affrontano un argomento suggerito dal docente.
\bye






