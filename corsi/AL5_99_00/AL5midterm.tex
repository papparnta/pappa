\magnification 1400
\nopagenumbers
\centerline{\bf TEST DI TEORIA DEI GRUPPI, AL5 -  20 Aprile 2000}
\centerline{\it Francesco Pappalardi}
\noindent Tutti gli esercizi tranne il primo valgono 10 punti oggi, 5 fino a Pasqua e 1 dopo Pasqua, il primo vale 40 punti oggi, 20 prima di Pasqua e 8 dopo
Pasqua.


\item{1)} Sia $q$ una potenza di un numero primo e sia $n>1$ un numero intero.
Denotiamo con ${\cal T}(n,q)$ l'insieme delle matrici di GL$(n,q)$ triangolari
superiori. cio\`e
$${\cal T}(n,q)=\left\{A\in{\rm GL}(n,q)\ {\rm tali\ che}\ A=
\pmatrix{* & 0 & 0 & \cdots & 0\cr
         * & * & 0 & \cdots & 0 \cr
         * & * & * & \cdots & 0 \cr
         \vdots & \vdots & \ddots & \cdots & \vdots\cr
         * & * & * & \cdots & *}\right\}$$
\itemitem{a.} Mostrare che ${\cal T}(n,q)$ \`e un gruppo
finito rispetto al prodotto di matrici.
\itemitem{b.} Calcolare l'ordine di ${\cal T}(n,q)$.
\itemitem{c.} Dire se ${\cal T}(n,q)$ \`e abeliano.
\itemitem{d.} Determinare un $l$--Sylow di ${\cal T}(3,5)$ per ogni
divisore dell'ordine.
\itemitem{e.} Generalizzare l'esercizio precedente a ${\cal T}(n,q)$.
\itemitem{f.} Dire per quali valori di $l$, gli $l$--Sylow di 
${\cal T}(n,q)$ sono abeliani.
\itemitem{g.} Calcolare l'ordine di $S{\cal T}(n,q)=\{A\in {\cal T}(n,q)\ |
\det(A)=1\}$.
\itemitem{h.} Calcolare il centro e il derivato dei seguenti gruppi 
$${\cal T}(2,3),\ \ {\cal T}(3,3),\ \ S{\cal T}(2,3)\ \ \rm{e}\  S{\cal T}(3,3).$$\medskip

\item{2)} Dimostrare che un gruppo con $1000=2^3\cdot 5^3$ elementi
oppure con $1500=3\cdot2^2\cdot5^3$ elementi non \`e semplice.\medskip

\item{3)} Dimostrare che GL$(n,q)$ \`e il prodotto semidiretto di 
$H$ per SL$(n,q)$ dove $H$ \`e un opportuno sottogruppo di GL$(n,q)$.\medskip

\item{4)} Quale \`e il massimo ordine di un elemento in $S_{11}$?\medskip

\item{5)} Quanti $7$--Sylow ha $A_7$?\medskip

\item{6)} Mostrare che tutti i $p$--Sylow di $S_p$ e di $A_p$ sono ciclici.
\medskip

\item{7)} Si scriva una serie di composizione per il $p$--Sylow di GL$(3,p)$.
\medskip

\item{8)} A quale gruppo (ben noto) \`e isomorfo un qualsiasi $2$--Sylow
di GL$(3,2)$?

\item{9)} Si dia un esempio di un valore di $n$ e uno di $l$ per cui un $l$--Sylow di $S_n$ non \`e abeliano. 
\medskip


\noindent{\it Suggerimento:} Cominciare da quelli pi\`u facili!!



\bye
