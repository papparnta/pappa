%Ragazzi,
%
%vi spedisco un elenco di esercizi. Penso che forniscano un ottima prova
%generale per il compito del 18 Aprile. 
%
%Vi suggerisco si riunirvi (ad esempio Martedi' 4 Aprile alle 15:45)
%per cercare di lavorarci insieme. Se alcuni non vi vengono, provate a
%scrivermi. Inoltre penso che potreste incontrarvi anche Giovedi' 6,
%Martedi' 11 e Giovedi' 13 se non riusciste a finirli in una volta.
%Infine, se li finite e ne volete altri, vi basta scrivermi. 
%Se mi spedite delle soluzioni per email ne saro' felice e cerchero' 
%di correggerli.
%Per mettervi d'accordo, potreste comunicare facendo un reply collettivo
%a questo messaggio.
%
%buon lavoro
%
%Francesco
%
\magnification 1000
\nopagenumbers
\centerline{\bf PROBLEMI DI AL5 (AA 1999/2000).}
\centerline{\it Francesco Pappalardi}
\bigskip\bigskip

\item{1.} Sia $p\equiv1(\bmod5)$. Dimostrare che il gruppo O$(2,p)$ delle matrici quadrate ortogonali
$2\times2$ a coefficienti in ${\bf F}_p$ ha $2(p-1)$ elementi.\par
{\it suggerimento: Mostrare che l'equazione $x^2+y^2=1$ in ${\bf F}_p$ ammette esattamente
$p-1$ soluzioni sfruttando il fatto che $\sqrt{-1}\in{\bf F}_p$.}

\item{2.} O$(2,p)$ \`{e} abeliano?

\item{3.} Quanti elementi ha SO$(2,p)$? SO$(2,p)$ \`{e} abeliano?

\item{4.} Determinare tutti i Sylow di SO$(2,12)$ e la sua struttura come gruppo
abeliano (cio\`{e} lo si scriva come il prodotto di $p$--gruppi ciclici).

\item{5.} Si determini il centro e il derivato di un $p$-Sylow di GL$(n,p)$.

\item{6.} Si mostri che PSL$(4,2)$ e PSL$(3,4)$ hanno lo stesso numero
di elementi ma non sono isomorfi perch\`{e} i loro $2$--Sylow hanno centri 
diversi (spiegare bene come si pu\`{o} applicare l'esercizio 4. a questo
caso) \par 
{\sevenrm N.B. questo \`{e} l'unico esempio che conosco di due gruppi
semplici non isomorfi con lo stesso ordine.}

\item{7.} Dimostrare che un gruppo di ordine $400$ non \`{e} semplice.

\item{8.} Sia G un gruppo nel quale il sottogruppo derivato ha meno di 
due elementi. Dimostrare che se $x_1, x_2$ e $x_3$ sono tre elementi di
$G$, allora 
$$x_1x_2x_3=x_{\sigma(1)}x_{\sigma(2)}x_{\sigma(3)}$$
dove $\sigma$ \`{e} una permutazione non identica di $1, 2, 3$.

\item{9.} Sia $G$ un gruppo e $H$ un sottogruppo normale di $G$ tale che
\itemitem{i)} Ogni automorfismo di $H$ \`{e} interno;
\itemitem{ii)} Il centro di $H$ \`{e} banale.
Dimostrare che $G\cong H\times C_G(H)$ (prodotto diretto).

\item{10.} Si consideri l'azione di GL$(n,q)$ su ${\bf F}_q^n$. Quanti elementi
ha lo stabilizzatore di un elemento? Quale \`{e} lo stabilizzatore dell'elemento
$\underline{e}_1$ della base canonica di ${\bf F}_q^n$?

\item{11.} Dire se \`{e} vero che $D_{2n}\cong D_n\times C_2$ (prodotto diretto).

\item{12.} Si determini un $2$--Sylow di $S_6$ e uno di $A_6$. 

\item{13.} Per ciascun primo $p$ che divide $6!$ si determini il numero di $n_p$
dei $p$--Sylow di $S_6$.

\item{14.} Per ciascun primo $p$ che divide $6!/2$ si determini il numero di $n_p$
dei $p$--Sylow di $A_6$.

\item{15.} Sia $G$ il prodotto semidiretto 
$U({\bf Z}/9{\bf Z})\times_\varphi {\bf Z}/9{\bf Z}$ dove 
$$\varphi: U({\bf Z}/9{\bf Z})
\rightarrow Aut({\bf Z}/9{\bf Z})$$
\`{e} l'omomorfismo naturale. Si determini un $3$--Sylow di $G$, si mostri che
\`{e} normale e si dica se \`{e} abeliano.

\item{16.} Si determini $Aut(C_2\times C_2\times C_2)$.

\item{17.} Siano $G$ e $H$ due gruppi e sia $G^H$ l'insieme delle applicazioni da
$H$ a $G$. Sia $G\wr H$ l'insieme delle coppie $(f,h)$ con $f\in G^H$ e $h\in H$. Se 
$(f_1,h_1)$ e $(f_2,h_2)$ sono elementi di $G\wr H$ definiamo il prodotto
$$(f_1,h_1)(f_2,h_2)=(g,h_1h_2).$$
Dove $g:\ H\rightarrow G, h\mapsto g(h)=f_1(h)f_2(hh_1)$.
Si dimostri che rispetto a questa operazione $G\wr H$ \`{e} un gruppo indicandone
l'elemento neutro e l'inverso del generico elemento $(f,h)$\par (si chiama il
{\bf prodotto a corona di $G$ e $H$}). Quanti elementi ha $G\wr H$?

\item{18.} Mostrare che $H$ si pu\`{o} immergere in $G\wr H$ in modo
naturale.

\item{19.} A quale dei $5$ gruppi con $8$ elementi che conosciamo \`{e} isomorfo 
$C_2\wr C_2$?

\item{20.} Sia $G_n={\bf Z}/n{\bf Z}\times{\bf Z}/n{\bf Z}\times{\bf Z}/n{\bf Z}$,
su $G$ definiamo la seguente operazione:
$$(a,b,c)(a',b',c')=(a+a',b+b'+ac',c+c')$$
\itemitem{i)} Mostrare che rispetto a questa operazione $G_n$ \`{e} un gruppo.
\itemitem{ii)} Quale \`{e} il centro di $G_n$?
\itemitem{iii)} Quale \`{e} il derivato di $G_n$?
\itemitem{iv)} A quale dei 5 gruppi di ordine 8 che conosciamo \`{e} isomorfo $G_2$?
\itemitem{v)} Mostrare che $G_p$ \`{e} isomorfo a un $p$--Sylow di GL$(3,p)$.
\bye


