\input programma.sty
\def\abbrcorso{AL420}
\def\titolocorso{Teoria algebrica dei numeri}
\def\sottotitolo{http://www.mat.uniroma3.it/users/pappa/CORSI/AL420$_-$18$_-$19/AL420.htm}
\def\docente{Prof. Francesco Pappalardi}
\def\crediti{7}
\def\semestre{II}
\def\esoneri{0}
\def\scrittofinale{0}
\def\oralefinale{1}
\def\altreprove{1}
\Intestazione 

\titoloparagr{Introduzione.} 
Richiami sui Campi numerici. Norme Tracce e Discriminanti. Anelli degli interi.

\titoloparagr{Algebra Commutativa.}
Anelli Noetheriani e anelli di Dedekind. La funzione $\zeta$ di Dedekind.

\titoloparagr{Algebra.} Gruppi finitamente generati e richiami di Teoria di Galois. Reticoli.

\titoloparagr{Discriminanti e Ramificazione.} Il Teorema di Minkowsi. I Teorema di Dirichlet. Il Gruppo delle classi e la finitezza del gruppo delle classi.

\titoloparagr{La formula del numeri di classe.} Solo enunciato

\testi

\bib
\autore{Schoof, R.} \titolo{Algebraic Number Theory}
\editore{dispense Universit\`a di Roma Tor Vergata,\hfill \break
http://www.mat.uniroma2.it/\~{}eal/moonen.pdf} \annopub{2003}
\endbib

\bib
\autore{Milne, J.} 
\titolo{Algebraic Number Theory} 
\editore{Lecture Notes,\hfill\break http://www.jmilne.org/math/CourseNotes/ANT.pdf} 
\annopub{2017} 
\endbib

\bib
\autore{Marcus, D} \titolo{Number fields, 3rd Ed} 
\editore{Springer-Verlag}
\annopub{1977}
\endbib

\altritesti

\bib \autore{Samuel, P.}
\titolo{Th\'eorie alg\'ebrique des nombres} \editore{Hermann, Paris}
\annopub{1971}
\endbib


\bib
\autore{Ono, T.}
\titolo{An introduction to algebraic number theory}
\editore{ Plenum Press, New York} \annopub{1990}
\endbib

\bib
\autore{M. Artin} \titolo{Algebra} \editore{Prentice Hall, Inc.,
Englewood Cliffs, NJ} \annopub{1991}
\endbib


\esami

Gli studenti dovranno risolvere degli esercizi proposti dal docente e consegnare
la soluzione secondo un calendazio prestabilito. Inoltre dovranno presentare
un seminario su argomenti pertinenti al programma.

\bye
