\nopagenumbers \font\title=cmti12
\def\ve{\vfill\eject}
\def\vv{\vfill}
\def\vs{\vskip-2cm}
\def\vss{\vskip10cm}
\def\vst{\vskip13.3cm}

% \def\ve{\bigskip\bigskip}
% \def\vv{\bigskip\bigskip}
% \def\vs{}
% \def\vss{}
% \def\vst{\bigskip\bigskip}

\hsize=19cm
\vsize=27.58cm
\hoffset=-1.6cm
\voffset=0.5cm
\parskip=-.1cm
\ \vs \hskip -6mm TN410 AA14/15\ (Elementary Number Theory)\hfill Session X \hfill Rome, September 24, 2015. \hrule
\bigskip\noindent
{\title Family Name}\  \dotfill\ {\title Name}\ \dotfill {\title
Student ID(Matricola):}\ \dotfill\
\smallskip  \noindent
Solve the problems adding to the replies short and essential explenations. 
\it Please write the solutions in the designed areas. NO EXTRA SHEETS WILL BE ACCEPTED. 
\rm 1 Problem = 4 marks. Duration: 2 hours. No questions allowed in the first hour and in 
the last 20 minutes.
\smallskip
\hrule\smallskip
\centerline{\hskip 6pt\vbox{\tabskip=0pt \offinterlineskip
\def \trl{\noalign{\hrule}}
\halign to371.7pt{
\strut#& \vrule#\tabskip=1.1em plus 2.9em& \hfil#
       & \vrule#& \hfill#\hfil
       & \vrule#& \hfil#
       & \vrule#& \hfill#\hfil
       & \vrule#& \hfil#
       & \vrule#& \hfill#\hfil
       & \vrule#& \hfil#
       & \vrule#& \hfill#\hfil
       & \vrule#& \hfil#
       & \vrule#& \hfill#\hfil
       & \vrule#&\hfil#
       & \vrule#&\hfil#
       & \vrule#& \hfil#
       &\vrule#\tabskip=0pt\cr
\trl && 1 && 2 && 3a && 3b && 3c && 3d && 4 && 5 && 6 && TOTAL&\cr
\trl && &&   &&     &&   &&   &&  &&   &&    && && &\cr 
&&   &&   &&   &&     &&   && && && && && &\cr
\trl }}}
\medskip


\item{1.} Dimostrare che
$$n!=\prod_{\ell\ {\rm primo}}\prod_{\alpha=0}^\infty\ell^{\left[{n\over\ell^{\alpha}}\right]}$$.\vv

\item{2.} Calcolare la frazione continua di $\sqrt{2}-1$\ve\vs

%\item{3.} \vv

\item{3.}   Sia $\varphi$ la funzione di Eulero e $\sigma$ la funzione ``somma dei divisori''.
\itemitem{a.}  Trovare tutti gli interi $n$ tali che $\varphi(n)=6$.\vv

\itemitem{b.} Dimostrare che se $n$ \`e un intero tale che $\varphi(n)=\varphi(2n)$, allora $n$ \`e dispari.
E' vero anche il viceversa?\vv

\itemitem{c.} Dimostrare che $\sigma(n)\ge n+1$ e che l'uguaglianza vale se e solo se $n$ \`e primo.\vv
\itemitem{d.} Determinare tutti gli interi $n$ tali che $\sigma(n)=12$.\ve\vs

\item{4.} Enunciare e dimostrare la formula delle somme parziali e utilizzarla per dimostrare che se $N$ \`e pari, allora
$\sum_{n\le N}{(-1)^n \over n}=\int_{1}^N{A(t)\over t^2}dt$ dove $A(t)=0$ se $[t]$ \`e pari e $A(t)=-1$ se $[t]$ \`e dispari.\vv

\item{5.} Sia $d,n,m\in{\bf Z}$. Dimostrare che se esistono $x,y,z, t\in{\bf Z}$ tali che $n=x^2+dy^2$ e $m=z^2+dt^2$, allora esistono
$u,v\in{\bf Z}$ tali che $nm=u^2+dv^2$. Usare questo fatto per esprimere $5548$ nella forma $Q^2+3P^2$\ve \vs

\item{6.} Dimostrare che 
$$\left({17 \over p}\right)_{\rm J}=\cases{1 &se $p\equiv\pm1,\pm2,\pm4,\pm8\bmod 17$\cr 0 &if $p=17$\cr -1 &if $p\equiv\pm3,\pm5,\pm6,\pm7\bmod 17.$}$$
\ \vst\bye
