\nopagenumbers \font\title=cmti12
\def\ve{\vfill\eject}
\def\vv{\vfill}
\def\vs{\vskip-2cm}
\def\vss{\vskip10cm}
\def\vst{\vskip13.3cm}

% \def\ve{\bigskip\bigskip}
% \def\vv{\bigskip\bigskip}
% \def\vs{}
% \def\vss{}
% \def\vst{\bigskip\bigskip}

\hsize=19cm
\vsize=27.58cm
\hoffset=-1.6cm
\voffset=0.5cm
\parskip=-.1cm
\ \vs \hskip -6mm TN410 AA14/15\ (Elementary Number Theory)\hfill Session C \hfill Rome, January 13, 2016. \hrule
\bigskip\noindent
{\title Family Name}\  \dotfill\ {\title Name}\ \dotfill {\title
Student ID(Matricola):}\ \dotfill\
\smallskip  \noindent
Solve the problems adding to the replies short and essential explenations. 
\it Please write the solutions in the designed areas. NO EXTRA SHEETS WILL BE ACCEPTED. 
\rm 1 Problem = 4 marks. Duration: 2 hours. No questions allowed in the first hour and in 
the last 20 minutes.
\smallskip
\hrule\smallskip
\centerline{\hskip 6pt\vbox{\tabskip=0pt \offinterlineskip
\def \trl{\noalign{\hrule}}
\halign to371.7pt{
\strut#& \vrule#\tabskip=1.1em plus 2.9em& \hfil#
       & \vrule#& \hfill#\hfil
       & \vrule#& \hfil#
       & \vrule#& \hfill#\hfil
       & \vrule#& \hfil#
       & \vrule#& \hfill#\hfil
       & \vrule#& \hfil#
       & \vrule#& \hfill#\hfil
       & \vrule#& \hfil#
       & \vrule#& \hfill#\hfil
       & \vrule#&\hfil#
       & \vrule#&\hfil#
       & \vrule#& \hfil#
       &\vrule#\tabskip=0pt\cr
\trl && 1 && 2 && 3a && 3b && 3c && 3d && 4 && 5 && 6 && TOTAL&\cr
\trl && &&   &&     &&   &&   &&  &&   &&    && && &\cr 
&&   &&   &&   &&     &&   && && && && && &\cr
\trl }}}
\medskip


\item{1.} Suppose that $n\in{\bf N}$ and that $p$ is a prime. Show that the largest integer
$k$ such that $p^k\mid n!$ is given by 
$$k=\sum_{j=1}^\infty\left[{n\over p^j}\right]$$.\vv

\item{2.} Compute the continued fraction of $7-\sqrt{3}$.\ve\vs

%\item{3.} \vv

\item{3.}   Let $k$ be a positive integer. 
The Jordan's totient function $J_k(n)$ of a positive integer $n$ is the number of $k$-tuples of positive integers $a_1,\ldots, a_k$ all less than or equal 
to $n$ such that $\gcd(n,a_1,\ldots, a_k)=1$.
This is a generalisation of Euler's totient function, which is $J_1$. 
\itemitem{a.}  Show that the Jordan's totient function is multiplicative and that 
$$J_k(n)=n^k \prod_{p|n}\left(1-{1\over p^k}\right).$$\vv

\itemitem{b.} Show that $\displaystyle\sum_{d | n } J_k(d) = n^k.$\vv

\itemitem{c.} Show that the identity $J_k(n) = \mu(n) \star n^k$ where $\star$ denotes the Dirichlet convolution product.\vv

\itemitem{d.} Show that the identity $\displaystyle\sum_{n\ge 1}{J_k(n)\over n^s} = {\zeta(s-k)\over\zeta(s)}.$\ve\vs

\item{4.} Let $\gamma$ be the Euler Mascheroni Constant. Prove the identity: 
$$\gamma=1-\int_{1}^\infty{\{u\}du\over u^2}.$$\vv

\item{5.} Describe the integers that cannot be written as sum of three squares and prove that there are infinitely many
such integers.\ve \vs

\item{6.} After having stated Gau\ss\ Theorem on the existence of primitive roots, compute all primitive roots modulo $250$.
\ \vst\bye
