\nopagenumbers \font\title=cmti12
\def\ve{\vfill\eject}
\def\vv{\vfill}
\def\vs{\vskip-2cm}
\def\vss{\vskip10cm}
\def\vst{\vskip13.3cm}

% \def\ve{\bigskip\bigskip}
% \def\vv{\bigskip\bigskip}
% \def\vs{}
% \def\vss{}
% \def\vst{\bigskip\bigskip}

\hsize=19cm
\vsize=27.58cm
\hoffset=-1.6cm
\voffset=0.5cm
\parskip=-.1cm
\ \vs \hskip -6mm TN410 AA14/15\ (Elementary Number Theory)\hfill Session B \hfill Rome, July 15, 2015. \hrule
\bigskip\noindent
{\title Family Name}\  \dotfill\ {\title Name}\ \dotfill {\title
Student ID(Matricola):}\ \dotfill\
\smallskip  \noindent
Solve the problems adding to the replies short and essential explenations. 
\it Please write the solutions in the designed areas. NO EXTRA SHEETS WILL BE ACCEPTED. 
\rm 1 Problem = 4 marks. Duration: 2 hours. No questions allowed in the first hour and in 
the last 20 minutes.
\smallskip
\hrule\smallskip
\centerline{\hskip 6pt\vbox{\tabskip=0pt \offinterlineskip
\def \trl{\noalign{\hrule}}
\halign to371.7pt{
\strut#& \vrule#\tabskip=1.1em plus 2.9em& \hfil#
       & \vrule#& \hfill#\hfil
       & \vrule#& \hfil#
       & \vrule#& \hfill#\hfil
       & \vrule#& \hfil#
       & \vrule#& \hfill#\hfil
       & \vrule#& \hfil#
       & \vrule#& \hfill#\hfil
       & \vrule#& \hfil#
       & \vrule#& \hfill#\hfil
       & \vrule#&\hfil#
       & \vrule#&\hfil#
       & \vrule#& \hfil#
       &\vrule#\tabskip=0pt\cr
\trl && 1 && 2 && 3a && 3b && 3c && 3d && 4 && 5 && 6 && TOTAL&\cr
\trl && &&   &&     &&   &&   &&  &&   &&    && && &\cr 
&&   &&   &&   &&     &&   && && && && && &\cr
\trl }}}
\medskip


\item{1.} Compute the $11$--adic valuation $v_{11}(100!)$.\vv

\item{2.} Prove that if a real number has a periodic continued fraction, then it is of the form $a+b\sqrt{d}$ with $a,b\in{\bf Q}$, $d\in{\bf Z}$.\ve\vs

%\item{3.} \vv

\item{3.}   A  positive  integer $d$ is  said  to  be a {\it unitary} divisor of $n\in{\bf N}$ if $d|n$
 and $\gcd(d,n/d)=1$. The {\it number\ of unitary divisors} function $d^*(n):=\#\{d\in{\bf N}: d\rm{\ is\ a\ unitary\ 
 divisor\ of\ }n\}$.
\itemitem{a.}  Show that if $d^*$ is multiplicative function.\vv

\itemitem{b.} Compute a formula for $d^*(p^a)$ for all $p$ prime and $a\in{\bf N^{\ge0}}$ and deduce that, if $n$ is square
free, $d^*(n)=d(n)$. Provide an example for which $d^*(n)\neq d(n)$\vv

\itemitem{c.} Consider the function $\sigma^*_k(d)=\displaystyle\sum_{{d\mid n\atop d{ \rm\ unitary\ divisor\ of\ }n}}d^k$ and prove that $\sigma^*_k$ 
is multiplicative for all $k\in{\bf Z}$.\vv
\itemitem{d.} Compute a formula for $\sigma^*_k(p^a)$ for all $p$ prime and $a\in{\bf N^{\ge0}}$ and compute $(\sigma^*_{-3}*\mu)(324)$.\ve\vs

\item{4.} Use the partial summation formula to produce an asymptotic formula for $\sum_{n\le T}\log^3 n$\vv

\item{5.} State all formulas that allow to express an integer as the sum of two squares, provide some ideas of their proof and apply them 
to compute the number of way to write $5500$ as the sum of two squares .\ve \vs

\item{6.} Justifying every step, prove that 
$$\left({13 \over p}\right)_{\rm J}=\cases{1 &if $p\equiv\pm1,\pm3,\pm4\bmod 13$\cr 0 &if $p=13$\cr -1 &if $p\equiv\pm2,\pm5,\pm6\bmod 13.$}$$
\ \vst\bye
