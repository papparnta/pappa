\documentclass[a4paper,11pt]{article}
\usepackage{amsmath,amsfonts,amssymb}
\usepackage[utf8]{inputenc}
\usepackage[cm]{fullpage}
%opening
\title{Elementary Number Theory (TN410)}
\author{Exercises: Sheet \#3}
\date{April 30, 2015}

\begin{document}

\maketitle
\thispagestyle{empty}

\begin{enumerate}
 \item Find a simple continued fraction expansion for the following numbers:
$$\frac{32}{19},\quad\frac{22}9,\quad\sqrt{6},\quad\sqrt{35},\quad\sqrt{21}.$$
\item Find the first 30 terms of the continued fraction of $e$ and of $\pi$.\\
\phantom{\ }\ \hfill {\tiny(\textit{hint: use a computer and an appropriate code})}
\item Find all positive integers $x$, $y$ and $z$ such that:
    $$x+\frac{1}{y+\frac1z}=N$$
    where $N=\frac{49}{11}, \frac{43}{36}, \frac{41}{33}$.
    \item Compute the exact value of the following continued fractions 
    $$[1;\overline{3,4}],\quad[2;\overline{5,1}],\quad[1;\overline{1,1,3}],\quad[5;3,\overline{3,4}].$$
  \item  Show that for $N=8/5$ it is impossible to find positive integer values of $x$, $y$ and $z$ for which the 
  above identity is satisfied. Find some more rational numbers $N$ for which the same is true.
 \item compute the number of representations as sum of two squares of the following integers:
 $$90,\qquad 999,\qquad,110500\qquad (4!)^4$$ 
 \item write a computer code in any language that given a positive integer $n$ as input, determines all possible representations
 of $n$ as sum of two squares.
 \item Let $p$ be a prime number. Prove the following statement:
 \begin{itemize}
  \item $\exists x,y\in\mathbb Z: p=x^2+2y^2\quad\Longleftrightarrow\quad p\equiv 1 \text{ or } 3(\bmod 8)$
   \item $\exists x,y\in\mathbb Z: p=x^2+3y^2\quad\Longleftrightarrow\quad p\equiv 1(\bmod 3)$
  \end{itemize}
 \item Determine an asymptotic formula with an error term for the average number of ways to write an integer as the sum of two squares.
 \item Determine the natural density of the following sets of integers:
 \begin{enumerate}
  \item $\mathbb P$, the set of prime numbers
  \item $\mathbb S_k$, the set of $k$--free numbers
  \item $a+q\mathbb N$, the set of integers $n$: $n\equiv a\bmod q$
  \item The set of integers that can be written as the sum of three squares
 \end{enumerate}
 \item Let $k\in\mathbb N$ with $1\le k\le 9$. Prove that the set of integers with most significan digit equal to $k$ do not admit 
 a natural density.
 \item Show that it is not true in general that the product of integers that can be written as the sum of three squares can also be written as the 
 sum of three squares.
 \end{enumerate} 
\end{document}
