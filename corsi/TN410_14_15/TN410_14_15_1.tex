\nopagenumbers \font\title=cmti12
\def\ve{\vfill\eject}
\def\vv{\vfill}
\def\vs{\vskip-2cm}
\def\vss{\vskip10cm}
\def\vst{\vskip13.3cm}

% \def\ve{\bigskip\bigskip}
% \def\vv{\bigskip\bigskip}
% \def\vs{}
% \def\vss{}
% \def\vst{\bigskip\bigskip}

\hsize=19cm
\vsize=27.58cm
\hoffset=-1.6cm
\voffset=0.5cm
\parskip=-.1cm
\ \vs \hskip -6mm TN410 AA14/15\ (Elementary Number Theory)\hfill Midterm Exam \hfill Rome, April 1, 2015. \hrule
\bigskip\noindent
{\title Family Name}\  \dotfill\ {\title Name}\ \dotfill {\title
Student ID(Matricola):}\ \dotfill\
\smallskip  \noindent
Solve the problems adding to the replies short and essential explenations. 
\it Please write the solutions in the designed areas. NO EXTRA SHEETS WILL BE ACCEPTED. 
\rm 1 Problem = 4 marks. Duration: 2 hours. No questions allowed in the first hour and in 
the last 20 minutes.
\smallskip
\hrule\smallskip
\centerline{\hskip 6pt\vbox{\tabskip=0pt \offinterlineskip
\def \trl{\noalign{\hrule}}
\halign to320.5pt{
\strut#& \vrule#\tabskip=1.1em plus 2.6em& \hfil#
       & \vrule#& \hfill#\hfil
       & \vrule#& \hfil#
       & \vrule#& \hfill#\hfil
       & \vrule#& \hfil#
       & \vrule#& \hfill#\hfil
       & \vrule#& \hfil#
       & \vrule#& \hfill#\hfil
       & \vrule#& \hfil#
       & \vrule#& \hfill#\hfil
       & \vrule#&\hfil#
       & \vrule#& \hfil#
       &\vrule#\tabskip=0pt\cr
\trl && 1 && 2 && 3 && 4 && 5 && 6 && 7 && 8 && TOTAL&\cr
\trl && &&   &&     &&   &&   &&  &&   &&    && &\cr 
&&   &&   &&   &&     &&   && && && && &\cr
\trl }}}
\medskip


\item{1.} Compute $\gcd(1380,1110)$ using the Extended Euclidean Algorithm and deduce a Bezout Identity.\vv

\item{2.} Compute the $7$--adic valuation $v_7(100!)$.\ve\vs

\item{3.} Let $\mu$ be the M\"obius function and denote by $*$ the Dirichlet convolution of arithmetic functions. Prove that $k$--folded
iterated convolution of $\mu$ satisfies:
$$\left(\mu*\mu*\cdots*\mu\right)(n)=\prod_{p}\left({k \atop v_p(n)}\right)(-1)^{v_p(n)}$$
where for $a\in{\bf Z}$ and $b\in{\bf N}$, $\left({a \atop b}\right)={a(a-1)\cdots(a-b+1)\over b!}$ is the binomial coefficient. \hfill\break
\qquad\qquad \hfil{\it (suggestion: try first to prove the formula for $k=2, 3, ...$)}\vv

\item{4.} After having stated Gauss Theorem of existence of primitive roots modulo integers, 
compute all primitive roots modulo $686$.\ve\vs

\item{5.} Find all integers $X$ in the interval $[-10,200]$ such that $\cases{X\equiv 3\bmod 4\cr X\equiv 2\bmod 5\cr X\equiv 4\bmod 7.}$\vv

\item{6.} After having stated the important properties of the Legendre--Jacobi Symbols, compute $\left({3073 \over 2919}\right)_{\rm J}$.\ve \vs

\item{7.} Prove that 
$$\left({-7 \over p}\right)_{\rm J}=\cases{1 &if $p\equiv1,2,4\bmod 7$\cr 0 &if $p=7$\cr -1 &if $p\equiv3,5,6\bmod 7.$}$$\vv

\item{8.} Let $p\ge3$ be a prime and let $k\in{\bf N}$. Prove that \smallskip
\itemitem{i)} the equation $X^k\equiv1\bmod p$ has $\gcd(k,p-1)$ solutions,\smallskip
\itemitem{ii)} the equation $X^k\equiv1\bmod p^\alpha$ has $\gcd(k,p-1)$ solutions if $p\not{\mid} k$.
 
\ \vst\bye
