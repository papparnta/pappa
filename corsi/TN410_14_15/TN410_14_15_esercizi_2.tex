\documentclass[a4paper,12pt]{article}
\usepackage{amsmath,amsfonts,amssymb}
\usepackage[utf8]{inputenc}
\usepackage[cm]{fullpage}
%opening
\title{Elementary Number Theory (TN410)}
\author{Exercises: Sheet \#2}
\date{March 21, 2015}

\begin{document}

\maketitle

% \noindent\textsc{Gli studenti volontari che, durante le esercitazioni del corso, 
% presenteranno la soluzione di uno (o più) esercizi (tra il numero $4$ e il numero $10$ sotto),
% otterranno un bonus di un punto sul voto finale per ciascun esercizio svolto alla lavagna.}

% \noindent\textsc{Volunteer students who, during problem sessions, will
% present the solution of one (or more) exercises (between number $4$ and number $10$ below),
% will get a bonus of one point on the final grade for each exercise solved at the blackboard.}

%\thispagestyle{empty}
\begin{enumerate}
 \item Let $(a_n)_{n\in\mathbb N}$ be a sequence of real or complex numbers and $\phi (x)$ a function of class $\mathcal{C}^1$. Prove that
$$\sum_{1 \le n \le x} a_n \phi(n) = A(x)\phi(x) - \int_1^x A(u)\phi'(u) \, \mathrm{d}u \,\qquad\text{where}\qquad
A(x):= \sum_{1 \le n \le x} a_n \,$$ and that, more generally, we have
$\displaystyle\sum_{x< n\le y} a_n\phi(n) = A(y)\phi(y) - A(x)\phi(x) -\int_x^y A(u)\phi'(u)\,\mathrm{d}u \,.$
 \item Find all integer solutions of  in the interval $[-500,500]$ of the following linear congruences:
$$5X\equiv 10(\bmod 35),\qquad 12X\equiv 14(\bmod 106),\qquad 12X\equiv 12(\bmod 42),\qquad 36X\equiv 18(\bmod 60).$$
 \item Find all integer solutions in the interval $[-500,500]$ of
$$\begin{cases} x \equiv 2  \pmod{3} \\ x \equiv 3  \pmod{4} \\ x \equiv 4  \pmod{5} \end{cases}\quad 
\begin{cases} x \equiv 3  \pmod{6} \\ x \equiv 4  \pmod{7} \\ x \equiv 7  \pmod{11} \end{cases}\quad
\begin{cases} x \equiv 1  \pmod{11} \\ x \equiv 2  \pmod{21} \\ x \equiv 3  \pmod{10} \end{cases}\quad
\begin{cases} x \equiv 5  \pmod{31} \\ x \equiv 3  \pmod{27} \\ x \equiv 4  \pmod{8}. \end{cases}$$
\item (\textit{Extended Chinese remainder Theorem.})Let $m_1,\ldots,m_s$ be positive integers and let $a_1,\ldots,a_s\in\mathbb Z$. Prove that
$$\begin{cases}X\equiv a_1\bmod m_1\\ \vdots\\ X\equiv a_s\bmod m_s 
 \end{cases}$$
has a solution exists if and only if $a_i\equiv a_j\mod \gcd(m_i,m_j)$ for all $i,j$. Moreover, 
in the case when has a solution exists, any two solutions differ by some common multiple of $m_1,\ldots,m_n$. 
 \item  Compute all primitive roots modulo $50$, $54$, $81$, $162$ and $250$.
 \item  Let $\alpha\in\mathbb N$, $\alpha\ge3$. 
 Prove that for any $a\in\mathbb Z$ odd, there exists $\nu\in\{0,1\}$ and $\mu\in\{0,\ldots,2^{\alpha-2}-1\}$ such that
 $$a\equiv (-1)^\nu\cdot 5^\mu\bmod 2^\alpha.$$
 Deduce that \qquad $\displaystyle U(\mathbb Z/2^\alpha\mathbb Z)\cong \mathbb Z/2\mathbb Z\times \mathbb Z/2^{\alpha-2}\mathbb Z.$ 
 \item (\textit{The Lifting Solutions Lemma}) Let $f(X)$ be a polynomial with integer coefficients and with degree $n$, let $p$ be a prime and let $a\in\mathbb N$. 
 Prove the following:
 \begin{enumerate}
  \item Suppose that $\zeta\in\mathbb Z$ is a solution of $f(X)\equiv0\bmod p^{a+1}$ 
  then $\zeta=\xi+sp^{a}$ where $\xi$ is a solution of $f(X)\equiv0\bmod p^{a}$ and $s\in\{0,\ldots,p-1\}$.
  \item If $\xi\in\mathbb Z$ is a solution of $f(X)\equiv0\bmod p^{a}$ such that the derivative $f'(\xi)\not\equiv0\bmod p$, 
  then there exists a unique integer $s\in\{0,\ldots,p-1\}$ such that $\xi+sp^{a}$ is a solution of $f(X)\equiv0\bmod p^{a+1}$.
  \item If $\xi\in\mathbb Z$ is a solution of $f(X)\equiv0\bmod p^{a}$ such that the derivative $f'(\xi)\equiv0\bmod p$, 
  then for all $s\in\{0,\ldots,p-1\}$ either \begin{enumerate}
        \item $\xi+sp^{a}$ is always a solution of $f(X)\equiv0\bmod p^{a+1}$ or
        \item $\xi+sp^{a}$ is never a solution of $f(X)\equiv0\bmod p^{a+1}$.
       \end{enumerate}
 \end{enumerate}
{  \small{(\textit{\textbf{hint:} 
 Use the Taylor expansion of $f$})}}
 \item For any polynomial with integer coefficients $f$ and any $m\in\mathbb N$, we set $N_f(m)$ to be the number of solutions in 
 any complete set of residues modulo $m$  of the congruence $f(X)\equiv 0\bmod m$. Prove that $N_f(m)=\prod_{p\mid m}N_f(p^{v_p(m)}).$
 \item Let $m$ be an odd integer and $a\in\mathbb Z$. Prove that if $X^2\equiv a\bmod m$ is solvable then $\left(\frac am\right)_{\!\!\!J}=1$.
 Give an example of $a$ and $m$ where the opposite does not hold.
 \item Compute a formula for the number of square roots of an integer $a$ modulo $m$ in terms of the Legendre symbols $\left(\frac ap\right)_{\!\!\!L}$
 where $p\mid m$.
 \item Prove that, for $p\ge3$, 
 $$\left(\frac{-3}p\right)_{\!\!\!L}=\begin{cases}1&\text{if }p\equiv1\bmod3\\ 0&\text{if }p=3\\-1&\text{if }p\equiv2\bmod3.    
   \end{cases}\text{\quad and that\quad}\left(\frac{3}p\right)_{\!\!\!L}=\begin{cases}1&\text{if }p\equiv\pm1\bmod 12\\ 0&\text{if }p=3\\-1&\text{if }p\equiv\pm5\bmod12.    
   \end{cases}$$
   \item Compute the following Jacobi symbols without ever factoring the odd integers involved:
   $$\left(\frac{2725}{9473}\right)_{\!\!\!J},\quad\left(\frac{5811}{1013}\right)_{\!\!\!J},\quad\left(\frac{7269}{573}\right)_{\!\!\!L},
   \quad\left(\frac{7307}{5809}\right)_{\!\!\!J},
    \quad\left(\frac{1269}{7231}\right)_{\!\!\!J}\quad\left(\frac{89439}{20259}\right)_{\!\!\!J}
 \quad\left(\frac{57599}{5557}\right)_{\!\!\!J}.$$
  \end{enumerate}
\end{document}
