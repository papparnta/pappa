\nopagenumbers \font\title=cmti12
\def\ve{\vfill\eject}
\def\vv{\vfill}
\def\vs{\vskip-2cm}
\def\vss{\vskip10cm}
\def\vst{\vskip13.3cm}

% \def\ve{\bigskip\bigskip}
% \def\vv{\bigskip\bigskip}
% \def\vs{}
% \def\vss{}
% \def\vst{\bigskip\bigskip}

\hsize=19cm
\vsize=27.58cm
\hoffset=-1.6cm
\voffset=0.5cm
\parskip=-.1cm
\ \vs \hskip -6mm TN410 AA14/15\ (Elementary Number Theory)\hfill Final Exam \hfill Rome, May 29, 2015. \hrule
\bigskip\noindent
{\title Family Name}\  \dotfill\ {\title Name}\ \dotfill {\title
Student ID(Matricola):}\ \dotfill\
\smallskip  \noindent
Solve the problems adding to the replies short and essential explenations. 
\it Please write the solutions in the designed areas. NO EXTRA SHEETS WILL BE ACCEPTED. 
\rm 1 Problem = 4 marks. Duration: 2 hours. No questions allowed in the first hour and in 
the last 20 minutes.
\smallskip
\hrule\smallskip
\centerline{\hskip 6pt\vbox{\tabskip=0pt \offinterlineskip
\def \trl{\noalign{\hrule}}
\halign to371.7pt{
\strut#& \vrule#\tabskip=1.1em plus 2.9em& \hfil#
       & \vrule#& \hfill#\hfil
       & \vrule#& \hfil#
       & \vrule#& \hfill#\hfil
       & \vrule#& \hfil#
       & \vrule#& \hfill#\hfil
       & \vrule#& \hfil#
       & \vrule#& \hfill#\hfil
       & \vrule#& \hfil#
       & \vrule#& \hfill#\hfil
       & \vrule#&\hfil#
       & \vrule#&\hfil#
       & \vrule#& \hfil#
       &\vrule#\tabskip=0pt\cr
\trl && 1 && 2 && 3a && 3b && 3c && 3d && 4 && 5 && 6 && TOTAL&\cr
\trl && &&   &&     &&   &&   &&  &&   &&    && && &\cr 
&&   &&   &&   &&     &&   && && && && && &\cr
\trl }}}
\medskip


\item{1.} Calculate the continued fraction expansion of $\sqrt{87}$\vv

\item{2.} An irrational number has continued fraction expansion $[\overline{2,5}]$. Compute it.\ve\vs

%\item{3.} \vv

\item{3.} Solve the following problems:
\itemitem{a.}  Show that if $p$ is a prime number such that 
$p=x^2+5y^2$ for
suitable $x,y\in{\bf Z}$, then either $p\equiv1\bmod 20$ or $p\equiv 
9\bmod 
20$. \hfill{\bf hint:\it study the identity modulo 5 and modulo 4. Then apply 
Chinese Remainder Theorem}\vv

\itemitem{b.} Prove that for any prime $p$, there exists $k\in\{1, 2, 
3, 4, 5\}$ such 
that $kp=x^2+5y^2$ for some $x, y\in{\bf Z}$.\hfill\break {\ }\hskip 2cm\hfill{\bf hint:\it \ apply 
the pigeon holes principle}\vv

\itemitem{c.} prove that, if $x, y\in{\bf Z}$, then $x^2+5y^2\not\equiv 2,3,7, 18\bmod 20$ and deduce that
if $p$ is prime with  $p\equiv1\bmod 20$ or $p\equiv 
9\bmod 20$ then either  $p=x^2+5y^2$ or $4p=x^2+5y^2$.
\hfill\break {\ }\hskip 2cm\hfill{\bf hint:\it \ first do some computation and then apply 3.b observing that if $5\mid x^2+5y^2$, then $5\mid x$.}\vv
 
\itemitem{d.} prove that if $4\mid x^2+5y^2$, then $2\mid \gcd(x,y)$. Finally deduce that if $p$ is prime, 
$$p\equiv 1, 9\bmod 20\qquad\Longleftrightarrow\qquad p=x^2+5y^2,\ \exists x, y\in{\bf Z}.$$\ve\vs

\item{4.} Show that if $\alpha\in{\bf R}, 0\le\alpha\le1$, then it exists a set $S\subset{\bf N}$ which has natural density
$\alpha$.\hfill\break {\bf hint:\it Consider the sequence $([\beta n])_{n\in{\bf N}}$ for a suitable $\beta\in{\bf R}$.}\vv

\item{5.} Let $a,b\in{\bf N}$. Compute the number of ways to express $6^a\cdot 65^b$ as the sum of two squares.\ve \vs

\item{6.} State Merten's Theorems on the distribution of primes and give some ideas on their proofs.
\ \vst\bye
