\input programma.sty    
\def\abbrcorso{GE7}
\def\titolocorso{Geometria Superiore}
\def\sottotitolo{Rappresentazioni di Gruppi finiti}
\def\docente{Prof. Francesco Pappalardi}
\def\crediti{7}
\def\semestre{I}
\def\esoneri{1}
\def\scrittofinale{0}
\def\oralefinale{1}
\def\altreprove{0}
\Intestazione  

\titoloparagr{RAPPRESENTAZIONI DEI GRUPPI FINITI}

I gruppi lineari classici. Rappresentazioni complesse. Riducibilit\`a.
Completa riducibilit\`a delle rappresentazioni dei gruppi finiti.
Omomorfismi di
rappresentazioni. Lemma di Schur. Caratteri. Le relazioni di
ortogonalit\`a tra i
caratteri irriducibili di un gruppo finito. 
Prodotti tensoriali di rappresentazioni. Rappresentazioni indotte.
Esempi: gruppi ciclici
finiti, gruppi abeliani, $S_n$ $A_n$, $n\le 5$, $D_{2n}$.

\titoloparagr{RAPPRESENTAZIONI DEL GRUPPO LINEARE FINITO $GL_2({\bf
F}_q)$}

Rappresentazioni 1-dimensionali di $GL_n({\bf F}_q)$. Azione sullo spazio
proiettivo e rappresentazione standard. Sottogruppi classici di
$GL_n({\bf F}_q)$.
Classi di coniugazione in $GL_2({\bf F}_q)$. Tavola dei caratteri di
$GL_2({\bf F}_q)$.


\titoloparagr{CARATTERI DEI GRUPPI SIMMETRICI}

Invarianti. Polinomi simmetrici elementari. Il teorema fondamentale per
i polinomi
simmetrici. Polinomi di Newton. Polinomi alterni. Vandermonde. Polinomi di Schur. 
Diagrammi e tabelle di Young. Sottogruppi di Young.
L'identit\`a di
Cauchy. Il teorema
di Frobenius. Le dimensioni delle rappresentazioni irriducibili di
$S_n$. Formula delle lunghezze dei ganci. 

\testi 

\bib
\autore{Artin M.}
\titolo{ Algebra}
\editore{Bollati Boringhieri}
\annopub{1997}
\endbib

\bib
\autore{Fulton W., Harris J.}
\titolo{Representation Theory}
\editore{Springer Verlag}
\annopub{1991}
\endbib

\bib
\autore{Macdonald I.G.}
\titolo{Symmetric functions and Hall polynomials}
\editore{Clarendon Press}
\annopub{1979}

\bib
\autore{Naimark M., Stern A.}
\titolo{Teoria delle rappresentazioni dei gruppi finiti}
{\eightrm Editori Riuniti.}

\bib
\autore{Serre J.P.}
\titolo{Representation theory of finite groups}
{\eightrm Springer Verlag.}
\endbib


\esami 

Per superare l'esame gli studenti devono superare i due esoneri previsti 
durante l'anno. 

Per coloro che non  hanno superato gli esoneri, \`{e} previsto un esame orale.
\bye
