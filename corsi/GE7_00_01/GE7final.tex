\magnification 1500
\nopagenumbers
\centerline{\bf Geometria Superiore (I modulo) --- GE7}
\centerline{\bf Esame di fine semestre, Gioved\`{\i}\ 22 Dicembre 2000}
\centerline{\bf Polinomi Simmetrici e caratteri di $S_n$}
\rm\bigskip

\item{1.} Si scriva una base per lo spazio vettoriale dei polinomi simmetrici
omogenei di grado 7 in 3 variabili $x_1, x_2, x_3$. \hfill\break Sia $F(x_1,x_2,x_3)$ il 
polinomio della base trovata che ha come uno dei suoi monomi $x_1x_2^2x_3^4$. Si esprima 
$F(x_1,x_2,x_3)$ come polinomio nelle funzioni simmetriche elementari.\bigskip 

\item{2.} Si dimostri la seguente identit\`{a}:
$$\sum_{} \left({\prod_{i<j}(l_i-l_j)\over l_1!l_2!\cdots l_n!}\right)^2={1\over n!}$$ 
dove la somma \`{e} estesa a tutte le $n$-uple di interi $$2n>l_1\geq l_2\geq \cdots 
l_n>0$$ tali che $\sum_{i=1}^n l_i={n^2+n\over 2}$. \bigskip 

\item{3.} Si determini la dimensione del carattere di $S_{16}$ associato alla
partizione $\underline{\lambda}=(5,4,3,3,1)$. Quale \`{e} il sottogruppo di Young 
associato a $\underline{\lambda}$? Si dica quale \`{e} il massimo ordine di un elemento 
in tale sottogruppo.\bigskip 

\item{4.} Sia $n\in{\bf N}$ e consideriamo la partizione $\underline{\lambda}=(n-2,2)$.
Dimostrare che $$\chi_{\underline{\lambda}}(C_{\mu})={1\over2}(i_1-1)(i_2-1)+i_2-1$$ dove 
$C_{\mu}$ \`{e} una classe di coniugazione di permutazioni di $S_n$ e se $\sigma\in 
C_{\mu}$, $i_1$ e $i_2$ sono rispettivamente il numero di $1$--cicli e di $2$--cicli 
nella decomposizione in cicli di $\sigma$. 

\bigskip\bigskip

\noindent{\bf Regole.} Ogni esercizio vale 7.5 punti. Tempo concesso 120 minuti. 
\`{E} vietato consultare libri e appunti. 
\`{E} vietato comunicare con altri studenti. Ogni esercizio deve essere svolto
su una e una sola facciata di un foglio.
\bye
