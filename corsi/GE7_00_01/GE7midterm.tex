\magnification 1100
\nopagenumbers
\centerline{\bf Geometria Superiore (I modulo) --- GE7}
\centerline{\bf Esame di Met\`a semestre, Gioved\`{\i}\ 9 Nov 2000}
\centerline{\bf Rappresentazioni e Caratteri di gruppi finiti}
\rm\bigskip

\item{1.} Scrivere la tavola dei caratteri irriducibili del gruppo dei quaternioni:
$${\bf H}=\{ \pm 1,\pm i, \pm j, \pm k\}.$$
\item{2.} Una rappresentazione $\rho: G \rightarrow {\rm GL}(V)$
si dice {\it rappresentazione permutazione} se esiste una base
$\cal B$ di $V$ tale che per ogni $g\in G$ e per ogni $\underline{e}\in
\cal B$, $\rho_g(\underline{e})\in \cal B$.
\itemitem{i.} Mostrare che se $\rho$ \`e una rappresentazione permutazione,
allora il suo carattere \`{e}
$$\chi_\rho(g)=\#\{\underline{e}\in {\cal B}\ |\ g\underline{e}=\underline{e}\}.$$
\itemitem{ii.} Mostrare che ogni {rappresentazione permutazione} ammette
la rappresentazione banale come sotto--rap\-pre\-sen\-ta\-zio\-ne.
\itemitem{iii.} Mostrare che la molteplicit\`{a} della rappresentazione
banale come sotto--rap\-pre\-sen\-ta\-zio\-ne di $\rho$ \`e pari al numero di orbite
dell'azione di $G$ sulla base $\cal B$.
\item{3.} Siano $\chi_1$ e $\chi_2$ i due caratteri irriducibili di $S_4$ di
dimensione 3. Sia
$\chi_3=\chi_1\otimes \chi_2$ il prodotto tensoriale (come
carattere di $S_4$). Scrivere $\chi_3$ come somma di caratteri
irriducibili.
\item{4.} Sia $G$ il gruppo meta--abeliano cos\`\i\ presentato:
$$G=\langle a,b | a^5=b^4=1, b^{-1}ab=a^2\rangle.$$
Calcolare la tavola dei caratteri irriducibili di $G$. {\it sugg: cercare i caratteri
1--di\-men\-sio\-na\-li tra quelli di $\langle b\rangle$. Dimostrare che ci sono solo
$5$ classi di coniugazione}
\item{5.} Calcolare la tavola dei caratteri di $S_3\times S_3$.
\item{6.} Sia $G$ un gruppo e $N$ un suo sottogruppo normale. 
Se $V=\langle N\rangle_{\bf C}$ \`e lo spazio vettoriale libero
generato da $N$ allora possiamo considerare la rappresentazione:
$\rho:G\rightarrow {\rm GL}(V), g\mapsto \rho_G$ con
$$\rho_g(\sum_{n\in N}\alpha_n\cdot n)=\sum_{n\in N}\alpha_n\cdot g^{-1}ng.$$
\itemitem{i.} Mostrare che $\rho$ \`{e} una rappresentazione.
\itemitem{ii.} Calcolare il carattere $\chi$ di $\rho$ nel caso in
cui 
$$G=S_4\ \ \ {\rm e}\ \ \  N=\{(1),(12)(34),(13)(24),(14)(23)\}.$$
\itemitem{iii.} Scrivere $\chi$ come somma di caratteri irriducibili.
\item{7.} Sia $\rho$ la rappresentazione standard di $S_n$. Calcolare
la rappresentazione determinantale di $\rho$.
\item{8.} Sia $G={\rm GL}_2({\bf F}_5)$:
\itemitem{i.} Quanti sono gli elementi e i caratteri irriducibili di $G$?
\itemitem{ii.} Descrivere tutti i caratteri $1$-dimensionali.
\item{9.} Mostrare che se $G$ \`e un gruppo che ha un unico
carattere $1$--dimensionale e $\rho: G\rightarrow{\rm GL}(V)$
\`e una qualsiasi rappresentazione, allora $\rho_g\in{\rm SL}(V)$
per ogni $g\in G$.
\item{10.} Calcolare le tavole dei caratteri irriducibili
di $S_5$ e $A_5$.
\item{11.} Mostrare che i caratteri di $S_n$ sono sempre reali.
\`E vero anche per $A_n$?\bigskip

\noindent{\bf Regole.} Ogni esercizio vale 10 punti. Tempo concesso 120 minuti. Seconda
consegna ore 00:00 - 10.11.00 (nella buca delle lettere di F. Pappalardi). 
Per la seconda consegna ogni esercizio vale 5 punti. \`{E} vietato consultare libri e
appunti. \`{E} vietato comunicare con altri studenti. Ogni esercizio deve essere svolto
su una e una sola facciata di un foglio.
\bye
