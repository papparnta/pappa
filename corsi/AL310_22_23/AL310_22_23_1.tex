\nopagenumbers \font\title=cmti12
\def\ve{\vfill\eject}
\def\vv{\vfill}
\def\vs{\vskip-2cm}
\def\vss{\vskip10cm}
\def\vst{\vskip13.3cm}

% \def\ve{\bigskip\bigskip}
% \def\vv{\bigskip}
% \def\vs{}
% \def\vss{}
% \def\vst{\bigskip}

\hsize=19.5cm
\vsize=27.58cm
\hoffset=-1.6cm
\voffset=0.5cm
\parskip=-.1cm
\ \vs \hskip -6mm AL310 AA22/23\ (Teoria delle Equazioni)\hfill ESAME
DI MET\`{A} SEMESTRE \hfill Roma, 11 Novembre 2022 \hrule
\bigskip\noindent
{\title COGNOME}\  \dotfill\ {\title NOME}\ \dotfill {\title
MATRICOLA}\ \dotfill\
\smallskip  \noindent
Risolvere il massimo numero di esercizi accompagnando le risposte
con spiegazioni chiare ed essenziali. \it Inserire le risposte
negli spazi predisposti. NON SI ACCETTANO RISPOSTE SCRITTE SU
ALTRI FOGLI.
\rm 1 Esercizio = 4 punti. Tempo previsto: 2 ore. Nessuna domanda
durante la prima ora e durante gli ultimi 20 minuti.
\smallskip
\hrule\smallskip
\centerline{\hskip 6pt\vbox{\tabskip=0pt \offinterlineskip
\def \trl{\noalign{\hrule}}
\halign to277pt{\strut#& \vrule#\tabskip=0.7em plus 1em& \hfil#&
\vrule#& \hfill#\hfil& \vrule#& \hfil#& \vrule#& \hfill#\hfil&
\vrule#& \hfil#& \vrule#& \hfill#\hfil& \vrule#& \hfil#& \vrule#&
\hfill#\hfil& \vrule#& \hfil#& \vrule#& \hfill#\hfil& \vrule#&
\hfil#& \vrule#& \hfill#\hfil& \vrule#& \hfil#& \vrule#& \hfil#&
\vrule#\tabskip=0pt\cr\trl && FIRMA && 1 && 2 && 3 && 4 &&
5 && 6 && 7 && 8  &&  TOT. &\cr\trl && &&   &&
&&     &&   &&     &&   &&   &&    && &\cr &&
\dotfill &&       &&   &&   &&     &&   && && && &&
&\cr\trl }}}
\medskip

\item{1.} Rispondere alle seguenti domande fornendo una giustificazione di una riga:
%\bigskip\bigskip

\itemitem{a.} E' vero che per ogni $q\in{\bf Q}$, $\cos(q\pi)$ \`e algebrico su ${\bf Q}$?\medskip
 \bigskip\bigskip\bigskip

 \ \dotfill\ \bigskip\bigskip\vfil

\itemitem{b.} E' vero che l'estensione ${\bf Q}[\pi]/{\bf Q}[\pi^4]$ \`e trascendente?\medskip
 \bigskip\bigskip\bigskip

 \ \dotfill\ \bigskip\bigskip\vfil

\itemitem{c.} Determinare il grado del campo ${\bf Q}[5^{1/m},5^{1/n}]$ su ${\bf Q}$ al variare di $m$ e $n$ in ${\bf N}$.\medskip

 \bigskip\bigskip\bigskip
 
 \ \dotfill\ \bigskip\bigskip\vfil

\itemitem{d.} E' vero che ${\bf Q}(\pi)$ \`e isomorfo a ${\bf Q}(\sqrt\pi)$?

 \medskip
 
 \bigskip\bigskip\bigskip

 \ \dotfill\ \bigskip\bigskip


 \vfil\eject

\item{2.} Sia $F$ un campo di caratteristica $0$ e $E/F$ un estensione di grado 2 allora $E=F[\alpha]$ dove $\alpha^2\in F$. E' vero l'analogo della stessa affermazione per estensioni di grado 3?\vv


\item{3.} Dopo aver descritto tutti gli elementi di Aut(${\bf Q}(2^{1/4},i)/{\bf Q})$, si determini l'ordine di ciascuno di essi. Dopo aver mostrato che ${\bf Q}(\zeta_8)\subset{\bf Q}(2^{1/4},i)/{\bf Q})$, si descrivano gli automorfismi sopra che fissano ${\bf Q}(\zeta_8)$.\vv

% 
% 
% Dopo aver dimostrato che $2\cos(2\pi/15)$ \`e un numero algebrico, se ne calcoli il polinomio minimo su ${\bf Q}$.


\ve\ \vs

%Dopo aver verificato che \`e algebrico, calcolare
%il polinomio minimo di $\cos \pi/9$ su ${\bf Q}$.

\item{4.} Data un estensione $E/F$, si dica cosa significa che $\alpha\in E$ \`e algebrico su $F$ e cosa \`e il polinomio minimo di $\alpha$ su $F$ dimostrando che \`e irriducibile.
\vv


\item{5.} Determinare il grado del campo di spezzamento su ${\bf Q}$, su ${\bf R}$ e su ${\bf F}_5$ di $f(X)=(X^4-1)(X^4-2)((X-3)^2+2)$.
\ve\ \vs

%--\item{6.} Descrivere la nozione di campo perfetto dimostrando che i campi finiti
%sono perfetti.

\item{6.} Si consideri $E ={\bf F}_5[\alpha]$ dove $\alpha$ \`e una radice del polinomio $X^2 + 2$. Determinare il polinomio minimo su ${\bf F}_5$ di $1/(\alpha + 3)$.
% Dopo aver definito la nozione di campo perfetto, si forniscano esempi di campi perfetti e di campi non perfetti.
\vv\vv

\item{7.} Dopo aver mostrato che $x^3-3x-6\in{\bf Q}[x]$ \`e irriducibile, si consideri il campo ${\bf Q}[\omega], \omega^3=3\omega+6$. 

\itemitem{a.} Determininare $a,b,c\in{\bf Q}$ tali che $\omega^{-3}=a+b\omega+c\omega^2.$ 

\itemitem{b.} Calcolare il polinomio minimo su ${\bf Q}$ di $\omega^2$. \vv\vv

\item{8.} 
\itemitem{a.} calcolare il polinomio minimo di $3^{1/6}$ su ${\bf Q}[\sqrt{3}]$.
\itemitem{b.} verificare che ${\bf Q}[3^{1/6}]\subset{\bf Q}[\sqrt{3},3^{1/9}]$; 
\itemitem{c.} descrivere i ${\bf Q}[3^{1/6}]$--omomorfismi del campo 
${\bf Q}[\sqrt{3},3^{1/9}]$ in ${\bf C}$;


\vv

\ \vst
 \bye
