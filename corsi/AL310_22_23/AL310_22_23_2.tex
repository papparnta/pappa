\nopagenumbers
\font\title=cmti12
\vfuzz=280pt
\def\ve{\vfill\eject}
\def\vv{\vfill}
\def\vs{\vskip-2cm}
\def\vss{\vskip10cm}
\def\vst{\vskip13.3cm}

\def\ve{\bigskip\bigskip}
\def\vv{\bigskip\bigskip}
\def\vs{}
\def\vss{}
\def\vst{\bigskip\bigskip}

\hsize=19.5cm
\vsize=27.58cm
\hoffset=-1.6cm
\voffset=0.5cm
\parskip=-.1cm
\ \vs \hskip -6mm AL310 AA22/23\ (Teoria delle Equazioni)\hfill ESAME
DI FINE SEMESTRE \hfill  - Roma, 20 Dicembre 2022 \hrule
\bigskip\noindent
{\title COGNOME}\  \dotfill\  {\title NOME}\ \dotfill {\title
MATRICOLA}\ \dotfill\
\smallskip  \noindent
Risolvere il massimo numero di esercizi accompagnando le risposte
con spiegazioni chiare ed essenziali. \it Inserire le risposte
negli spazi predisposti. NON SI ACCETTANO RISPOSTE SCRITTE SU
ALTRI FOGLI. Scrivere il proprio nome anche nell'ultima pagina.
\rm 1 Esercizio = 3 punti. Tempo previsto: 2 ore. Nessuna domanda
durante la prima ora e durante gli ultimi 20 minuti.
\smallskip
\hrule
\medskip

\item{1.} Fornire un esempio esplicito di un polinomio il cuo gruppo di Galois su ${\bf Q}$
\`e isomorfo a $A_3$.%

\vv \item{2.} Enunciare e dimostrare il Teorema dell'elemento primitivo. %

\ve\ \vs \item{3.} Determinare tutti i sottocampi $K$ di ${\bf
Q}(\zeta_{24})$ tali che $[{\bf
Q}(\zeta_{24}):K]=2$. %

\vv

\item{4.} Fornire un esempio di un polinomio irriducibile di grado sei il cui gruppo di Galois \`e isomorfo al gruppo $S_3$.\ve\ \vs %

\item{5.} Calcolare il gruppo di Galois si ${\bf Q}$ del polinomio
$x^4+2x^2+2$. %%%%%%%%%

\vv \item{6.}Mostrare the se
$f(x)=\displaystyle{\prod_{i=1}^m\big(x-\alpha_i\big)}\in F[x],$
allora il discriminante $D(f)$ soddisfa:
$D(f)=\displaystyle{(-1)^{{m(m-1)\over2}}\prod_{i=1}^mf'(\alpha_i)}$. \ve\ \vs  %%%%%%%%%%

\item{7.} Sia $E={\bf Q}[\sqrt{7},\sqrt{11},\sqrt{77}]$. Determinare un elemento primitivo $\gamma\in E$ su ${\bf Q}$ e scriverne il
polinomio minimo su ${\bf Q}$. Descrivere tutti 
i sottocampi di $E$.

\vv \item{8.} Si enunci nella completa generalit\`{a} il Teorema
di corrispondenza di Galois. %%%%
\ve\ \vs

\item{9.} Si calcoli il numero di elementi nel campo di spezzamento
del polinomio
$(x^{81}-3x^4-x)(x^3+x^2+2)(x^9+x^3+1)$ su ${\bf F}_3$.\vv %%%%%%%%

\item{10.} Dimostrare che $\Phi_{p^2}(x)$ il $p^2$--esimo polinomio ciclotomico ($p\ge3$ primo)) \`e $(x^{p^2}-1)/(x^p-1)$ e usare questa identit\`a
per verificare che il suo discriminante \`e pari a $\pm p^{p(2p-3)}.$
\hfil\break{\it Suggerimento: mostrare che se $\zeta_{p^k}=e^{2\pi i/p^k}$, 
allora $\Phi_{p^2}'(\zeta_{p^2})=p^2/(\zeta_{p^2}(\zeta_p-1))$. Quindi
usare una formula nota.}
\ve\ \vs

\item{11.} Determinare un numero algebrico il cui polinomio minimo sui razionali ha un gruppo di Galois isomorfo
a $C_2 \times C_{51}$.\vss

\item{12.} Definire la nozione di numero costruibile e mostrare che l'insieme
dei numeri costruibili \`{e} un campo.\vv %%%%%%%%%%%
\ \vst

\centerline{\hskip 6pt\vbox{\tabskip=0pt \offinterlineskip
\def \trl{\noalign{\hrule}}
\halign to500pt{\strut#& \vrule#\tabskip=0.7em plus 1em&
\hfil#& \vrule#& \hfill#\hfil& \vrule#&
\hfil#& \vrule#& \hfill#\hfil& \vrule#&
\hfil#& \vrule#& \hfill#\hfil& \vrule#&
\hfil#& \vrule#& \hfill#\hfil& \vrule#&
\hfil#& \vrule#& \hfill#\hfil& \vrule#&
\hfil#& \vrule#& \hfill#\hfil& \vrule#&
\hfil#& \vrule#& \hfil#& \vrule#\tabskip=0pt\cr\trl
&& NOME E COGNOME && 1 && 2 && 3 && 4 && 5 && 6 && 7 && 8 && 9 && 10 && 11 && 12 &&  TOT. &\cr\trl
&& &&   &&   &&     &&   &&   &&   &&   &&   &&    &&   &&   &&  && &\cr
&& \dotfill &&     &&   &&   &&   &&   &&   &&    &&  &&   && && && && &\cr\trl
}}}
 \bye
