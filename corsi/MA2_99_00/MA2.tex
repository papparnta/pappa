\input programma.sty     
\def\abbrcorso{MA2}  
\def\titolocorso{Matematica Applicata: Laboratorio 2}
\def\sottotitolo{Crittografia a chiave pubblica} 
\def\docente{Prof. Francesco Pappalardi}  
\def\crediti{7.5}   
\def\semestre{II}
\def\esoneri{1}
\def\scrittofinale{1}
\def\oralefinale{0}
\def\altreprove{1}

\Intestazione   

\titoloparagr{Argomenti di Teoria dei numeri elementare.}

Il concetto di operazione bit tipo somma o sottrazione. Stima 
del numero di operazioni bit (tempo macchina) per eseguire 
le operazioni fondamentali. Algoritmi che convergono in tempo 
esponenziale o polinomiale. Divisibilit\`a. Algoritmo di Euclide
(identit\'a di Bezout) e suo tempo di esecuzione. Congruenze. 
Teorema cinese dei resti. Esempi di fattorizzazione.

\titoloparagr{RSA. L'algoritmo di Adleman, Shamir e Rivest.}

Formulazione dell'algoritmo e analisi del suo tempo di esecuzione.
Esempi concreti non realistici. Costruzione di numeri primi (grandi):
Simboli di Legendre e simboli di Jacobi.
Legge di reciprocit\`{a} quadratica generale (senza dimostrazione) --
algoritmo polinomiale per il calcolo del simbolo di Jacobi. 
pseudo--primi di Eulero e pseudo--primi forti.
Algoritmi montecarlo. Il test di Solovay--Strassen. il test di
Miller--Rabin. 
Accortezza nell'implementazione di RSA: Modulo RSA con un
fattore troppo piccolo, Modulo RSA con fattori troppo vicini,
Pubblicazione del'esponente di decodifica -- Algoritmi Las--Vegas
per fattorizzazione del Modulo RSA. Il crittosistema di Rabin.
Il metodo di fattorizzazione $p-1$.

\titoloparagr{DES.}

L'algoritmo a chiave privata Data Encryption Standard. 
Modalit\`a di uso del DES (triplo DES, ECB, 
CBC, CFB): cenni.

\titoloparagr{Campi finiti.}

Fatti fondamentali di teoria dei campi. Teorema dell'elemento primitivo
in un campo finito. Esistenza e unicit\`a dei campi finiti 
(campi di spezzamento). Esempi. Polinomi irriducibili e primitivi. Enumerazione dei 
polinomi irriducibili e primitivi.
 Aritmetica in tempo polinomiale sui campi finiti. Esempi.
Fattorizzazione nell'anello dei polinomi ${\bf F}_{p}[x]$: Gli algoritmi di Berlekamp 
(campi piccoli e campi grandi), Esempi, calcolo delle radici dei polinomi. 
\vfill\eject

\titoloparagr{Logaritmi discreti.}

Funzioni a trappola. Il problema del logaritmo discreto in un gruppo 
ciclico astratto. Metodo di Diffie Hellman per lo scambio delle
chiavi. Metodo di Massey Omura per la trasmissione dei messaggi.
Il crittositema di ElGamal. Esempi. Algoritmi per il calcolo dei 
logaritmi discreti nei campi finiti: L' algoritmo di Shanks, L'algoritmo di
Pohlig--Hellman, esempi.

\titoloparagr{Altri Algoritmi.}

Metodo dello zainetto, Crittosistema di Merkle--Hellman (esempio), Crittosistema di Chor--Rivest. 
Crittosistemi Ellittici: Generalit\`{a} sulle curve ellittiche,
definizione di addizione sui punti razionali di una curva ellittica, Il gruppo di Mordell--Weil,
Teorema di Struttura del gruppo di Mordell Weil di una curva ellittica su un campo campo finito
(solo enunciato), Teorema di Hasse (solo enunciato), esempio. Il crittosistema di ElGamal su
$E(\bf{F}_p)$. Il crittosistema di Menezes--Vanstone.


\titoloparagr{Sistema Pari GP.} Facoltativo: Rudimenti del Sistema Pari per cacolare con numeri
a precisione arbitraria e campi finiti.\bigskip

\testi  

\bib
\autore{Neal Koblitz} 
\titolo{A Course in Number Theory and Cryptography} 
\editore{Springer}
\annopub{1994}
\altro{Graduate Texts in Mathematics, No 114}
\endbib

\bib                                             
\autore{Douglas R. Stinson}
\titolo{Cryptography: Theory and Practice} 
\editore{CRC Pr}
\annopub{1995}
\endbib

\bib
\autore{Rudolf Lidl, Harald Niederreiter}
\titolo{Finite Fields}
\editore{Cambridge University Press}
\annopub{1997}
\endbib

\bib
\autore{C. Batut, K. Belabas, D. Bernardi, H. Cohen, M. Olivier}
\titolo{Pari--GP (2.014)}\par \hskip 6mm
\editore{{http://pari.home.ml.org}}
\annopub{1998}
\endbib

%\altritesti  
%
%\bib
%\autore{Jan C. A. van der Lubbe} 
%\titolo{Basic Mathods of Cryptography} 
%\editore{Cambridge University Press}
%\annopub{1988}  
%\endbib
% 
%\bib
%\autore{Neal Koblitz} 
%\titolo{Algebraic Aspects of Cryptography} 
%\editore{Springer}
%\annopub{1998}
%\altro{Algorithms and Computation in Mathematics, Vol 3}
%\endbib

%
%\bib
%\autore{Bruce Schneier}
%\titolo{Applied Cryptography}
%\editore{John Wiley \& Sons, Inc.}
%\annopub{1996}
%\altro{seconda edizione}
%\endbib

\esami 

Oltre agli 8 compiti svoltisi durante il corso (oppure oltre allo scritto 
finale), gli studenti che aspirano ad alzare il loro voto di due punti
possono realizzare:
\item{1.} Una simulazione del Crittosistema di ElGamal in un campo finito che
non sia un campo primo e abbia ordine $\geq 3^4$.
\item{2.} Una simulazione del Crittosistema di Rivest--Chor in un campo finito
del tipo ${\bf F}_{p^n}$ con $p, n\geq 5$
\bye 

