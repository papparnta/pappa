\nopagenumbers
\font\title=cmti12
\def\ve{\vfill\eject}
\def\vv{\vfill}
\def\vs{\vskip-2cm}
\def\vss{\vskip10cm}
\def\vst{\vskip15cm}

%\def\ve{\bigskip\bigskip}
%\def\vv{\bigskip\bigskip}
%\def\vs{}
%\def\vss{}
%\def\vst{\bigskip\bigskip}

\hsize=20cm
\vsize=28cm
\hoffset=-2.2cm
\voffset=-0.5cm
\parskip=-.1cm
\ \vs \hskip -6mm CR1 AA02/03\ (Crittografia a chiave
pubblica)\hfill ESAME DI MET\`{A} SEMESTRE \hfill Roma, 16 Aprile
2003. \hrule
\bigskip\noindent
{\title COGNOME}\  \dotfill\  {\title NOME}\ \dotfill
{\title MATRICOLA}\ \dotfill\
\smallskip  \noindent
Risolvere il massimo numero di esercizi accompagnando le risposte
con spiegazioni chiare ed essenziali. \it Inserire le risposte
negli spazi predisposti. NON SI ACCETTANO RISPOSTE
SCRITTE SU ALTRI FOGLI. Scrivere il proprio nome anche nell'ultima
pagina. \rm 1 Esercizio = 3 punti. Tempo previsto: 2 ore. Nessuna
domanda durante la prima ora e durante gli ultimi 20 minuti.
\smallskip
\hrule
\medskip
\item{1.}
Se $n\in{\bf N}$, sia $\varphi(n)$ la funzione di Eulero. Supponiamo che sia nota
la fattorizzazione (unica) di $n=p_1^{\alpha_1}\cdots p_s^{\alpha_s}$. Stimare il
numero di operazioni bit necessarie per calcolare $\varphi(n)$.
\vv
\item{2.} Stimare in termini di $k$ il numero di operazioni bit necessarie per calcolare $\left[\sqrt{2^{k^k}\bmod 3^k}
\right]$.
\vv
\item{3.} Dato il numero binario
$n=(111001011101)_2$, calcolare $[\sqrt{n}]$ usando l'algoritmo
delle approssimazioni successive (Non passare a base 10 e  non
usare la calcolatrice!) \ve \vs

\item{4.}
Calcolare il massimo comun divisore tra $1235$ e $1800$ utilizzando
l'algoritmo binario. \vv
\item{5.} Calcolare tutte le soluzioni in $[-300,200]$, del sistema di congruenze $\cases{x^3\equiv 1 \bmod 5\cr
x^4\equiv 1\bmod 8}$?
\vv
\item{6.} Illustrare l'algoritmo dei quadrati successivi in un gruppo analizzandone la complessit\`{a}. $b=2^4+2^2+1$,
quante moltiplicazioni in ${\bf Z}/m{\bf Z}$ sono necessarie per calcolare $2^b\bmod m$?
\ve
\ \vs
\item{7.} Spiegare come usare l'algoritmo di Euclide per calcolare gli inversi in ${\bf Z}/m{\bf Z}$.
\vv
\item{8.} Si spieghi cosa \`{e} un algoritmo Montecarlo polarizzato con probabilit\`{a} di errore
pari a $\delta$. \vv
\item{9.} Dopo aver definito i numeri di Carmichael, si dimostri che i quadrati dei numeri primi
non sono numeri di Carmichael.
\ve
\ \vs
\item{10.} Definire un sistema di chiavi rsa in modo tale che: i) l'esponente di cifratura sia $5$;
ii) Sia possibile spedire messaggi con pacchetti di 3 lettere alla volta.
\vv

\item{11.} Calcololare il seguente simbolo di Jacobi senza fattorizzare: $\left({22348\over65431}\right)$.

\ve


\ \vs
\item{12.} Definire le nozioni di pseudo primo, pseduo primo di Eulero e pseudo primo forte. E spiegare le
connessioni tra le tre nozioni.

\vv


\item{13.} Carlo scopre il valore di $\varphi(n)$ dove $n$ \`{e} il modulo RSA che Alice e Bernardo
stanno usando per comunicare. Come pu\`{o} usare questa informazione per decifrare i messaggi?

\ve
\ \vs
\item{14.} Quante iterazioni sono necessarie utilizzando l'algoritmo di Miller Rabin su un numero
$m\leq 10^{400}$ per essere certi di avere una probabilit\`{a} di errore $< 10^{-4}$?

\ \vss


\item{15.} Spiegare brevemente il funzionamento del metodo di fattorizzazione $\rho$ di Pollard.


\ \vst
\centerline{\hskip 6pt\vbox{\tabskip=0pt \offinterlineskip
\def \trl{\noalign{\hrule}}
\halign to580pt{\strut#& \vrule#\tabskip=0.7em plus 1em&
\hfil#& \vrule#& \hfill#\hfil& \vrule#&
\hfil#& \vrule#& \hfill#\hfil& \vrule#&
\hfil#& \vrule#& \hfill#\hfil& \vrule#&
\hfil#& \vrule#& \hfill#\hfil& \vrule#&
\hfil#& \vrule#& \hfill#\hfil& \vrule#&
\hfil#& \vrule#& \hfill#\hfil& \vrule#&
\hfil#& \vrule#& \hfill#\hfil& \vrule#&
\hfil#& \vrule#& \hfill#\hfil& \vrule#&
\hfil#& \vrule#\tabskip=0pt\cr\trl
&& NOME E COGNOME && 1 && 2 && 3 && 4 && 5 && 6 && 7 && 8 && 9 && 10 && 11 && 12 && 13 && 14 && 15 && TOT. &\cr\trl
&& &&   &&   &&   &&   &&   &&   &&   &&   &&   &&    &&   &&   &&  && && && &\cr
&& \dotfill &&   &&   &&   &&   &&   &&   &&   &&   &&   &&    &&  &&   && && && && &\cr\trl
}}}
 \bye
