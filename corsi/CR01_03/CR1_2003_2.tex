\nopagenumbers
\font\title=cmti12
\def\ve{\vfill\eject}
\def\vv{\vfill}
\def\vs{\vskip-2cm}
\def\vss{\vskip10cm}
\def\vst{\vskip15cm}
%\mathbb{F}
%\def\ve{\bigskip\bigskip}
%\def\vv{\bigskip\bigskip}
%\def\vs{}
%\def\vss{}
%\def\vst{\bigskip\bigskip}
\def\F{\matbb{F}}
\hsize=20cm
\vsize=28cm
\hoffset=-2.2cm
\voffset=-0.5cm
\parskip=-.1cm
\ \vs \hskip -6mm CR1 AA02/03\ (Crittografia a chiave
pubblica)\hfill ESAME DI FINE SEMESTRE \hfill Roma, 9 giugno 2003.
\hrule
\bigskip\noindent
{\title COGNOME}\  \dotfill\  {\title NOME}\ \dotfill
{\title MATRICOLA}\ \dotfill\
\smallskip  \noindent
Risolvere il massimo numero di esercizi accompagnando le risposte
con spiegazioni chiare ed essenziali. \it Inserire le risposte
negli spazi predisposti. NON SI ACCETTANO RISPOSTE
SCRITTE SU ALTRI FOGLI. Scrivere il proprio nome anche nell'ultima
pagina. \rm 1 Esercizio = 3 punti. Tempo previsto: 2 ore. Nessuna
domanda durante la prima ora e durante gli ultimi 20 minuti.
\smallskip
\hrule
\medskip
\item{1.}
Definire il concetto di  curva ellittica su un campo finito ed il
gruppo di Mordell-Weyll associato. \vv

\item{2.} Consideriamo la curva $y^2 = x^3 +x + 6$ sul campo
 ${\bf F}_{7}$. Descrivere $E({\bf F}_{7})$.
\vss \ve \ \vs

\item{3.} Descrivere l'algoritmo dello scambio delle chiavi di Diffie-Helmann sulle
curve
 ellittiche. \vst

\item{4.}
Sia $g$ una radice primitiva di ${\bf F}_p$. Dimostrare le seguenti
propriet\`{a} del logaritmo discreto:

\qquad $ \log_g(a\cdot b) \; = \log_g(a) + \log_g(b),$ \qquad
$\log_g(a^n) = n \log_g(a)$ \qquad $(\bmod\;p-1)$. \vst \ve \ \vs


\item{5.} Consideriamo il campo base ${\bf F}_5$. Determinare il numero dei
polinomi monici irriducibili di grado 12. Qual \`{e} la
probabilit\`{a} che, preso random un polinomio monico di grado 12,
esso sia irriducibile?  \vv




\item{6.} Consideriamo il polinomio $f = x^3 + x +1 \in {\bf F}_5[x]$. Dopo aver
verificato che \`{e} irriducibile, trovare una radice primitiva
per il campo:
$${\bf F}_{5}[x]/(f) \; = \{a + b\theta + c\theta^2\; |\;
\theta^3 = - \theta -1 \}.$$ \ve \ \vs





\item{7.} Calcolare il $\log_3(13)$ nel campo ${\bf F}_{31}$, utilizzando un 
qualsiasi metodo. \vv



\item{8.} Dimostrare che il polinomio $x^2+1$ \`{e} irriducibile su tutti i
campi finiti ${\bf F}_p$ con $p \equiv 3$ $(\bmod\;4)$.  \ve \ \vs

\item{9.} Si descrivino
tutti i valori di $a \in {\bf F}_p$ per cui il polinomio: $ f(x) \; =\;
x^2 + a $ \`{e} irriducibile, e si dimostri che tale polinomio non
pu\`{o} mai essere primitivo. \vv


\item{10.} Mostrare che tutti i polinomi del tipo $x^p +x +1$ sono riducibili su 
${\bf F}_p$, con $p \geq 3$. \ve \ \vs

\item{11.} Dopo aver scritto tutti i polinomi irriducibili di
grado $\leq 4$ su ${\bf F}_2$, mostrare che il polinomio $f(x) =
x^5+x^2+1$ \`{e} irriducibile. \vst



\item{12.} Descrivere tutti i sottocampi di ${\bf F}_{5^{30}}$.

\ \vss
 \centerline{\hskip 6pt\vbox{\tabskip=0pt \offinterlineskip
\def \trl{\noalign{\hrule}}
\halign to580pt{\strut#& \vrule#\tabskip=0.7em plus 1em& \hfil#&
\vrule#& \hfill#\hfil& \vrule#& \hfil#& \vrule#& \hfill#\hfil&
\vrule#& \hfil#& \vrule#& \hfill#\hfil& \vrule#& \hfil#& \vrule#&
\hfill#\hfil& \vrule#& \hfil#& \vrule#& \hfill#\hfil& \vrule#&
\hfil#& \vrule#& \hfill#\hfil& \vrule#& \hfil#& \vrule#&
\hfill#\hfil& \vrule#& \hfil#& \vrule#& \hfill#\hfil& \vrule#&
\hfil#& \vrule#\tabskip=0pt\cr\trl && NOME E COGNOME && 1 && 2 &&
3 && 4 && 5 && 6 && 7 && 8 && 9 && 10 && 11 && 12  && TOT.
&\cr\trl && &&   &&   &&   &&   &&   &&   &&   &&   &&   &&    &&
&&   &&   &\cr && \dotfill &&   &&   &&   &&   &&   &&   &&   &&
&&   &&    &&  &&    && &\cr\trl }}}
 \bye
