\nopagenumbers
\font\title=cmti12
%\def\ve{\vfill\eject}
%\def\vv{\vfill}
%\def\vs{\vskip-2cm}
%\def\vss{\vskip10cm}
%\def\vst{\vskip7cm}

\def\ve{\bigskip\bigskip}
\def\vv{\bigskip\bigskip}
\def\vs{}
\def\vss{}
\def\vst{\bigskip\bigskip}

\hsize=20cm
\vsize=28cm
\hoffset=-2.2cm
\voffset=-0.5cm
\parskip=-.1cm
\ \vs \hskip -6mm CR1 AA02/03\ (Crittografia a chiave
pubblica)\hfill ESAME SCRITTO \hfill Roma, 19 Settembre 2003.
\hrule
\bigskip\noindent
{\title COGNOME}\  \dotfill\  {\title NOME}\ \dotfill
{\title MATRICOLA}\ \dotfill\
\smallskip  \noindent
Risolvere il massimo numero di esercizi accompagnando le risposte
con spiegazioni chiare ed essenziali. \it Inserire le risposte
negli spazi predisposti. NON SI ACCETTANO RISPOSTE
SCRITTE SU ALTRI FOGLI. Scrivere il proprio nome anche nell'ultima
pagina. \rm 1 Esercizio = 3 punti. Tempo previsto: 2 ore. Nessuna
domanda durante la prima ora e durante gli ultimi 20 minuti.
\smallskip
\hrule
\medskip
\item{1.} Dimostrare che il numero di operazioni bit necessarie a
moltiplicare due interi con al pi\`{u} $k$ cifre binarie \`{e}
$O(k^2)$ spiegando tutti i passaggi.

\vv \item{2.} Dati due numeri primi distinti $p$ e $q$, si spieghi
come risolvere il seguente sistema di equazioni di congruenze
$$\cases{x^3\equiv 1\bmod p \cr x^4\equiv 1\bmod q}$$ fornendo un
esempio in cui il sistema ammette $12$ soluzioni.

%$p=13,q=5$

\vv \item{3.} Supponiamo che $n$ sia una chiave RSA e che sia noto
il valore di $\varphi(n)$. Descrivere un algoritmo con
complessit\`{a} quadratica per fattorizzare $n$.

 \ve \vs

\item{4.} Si dimostri che se $m$ \`{e} un intero dispari composto,
allora esiste sempre un base $a\in U({\bf Z}/m{\bf Z})$ rispetto a
cui $m$ non \`{e} pseudo primo di Eulero. Quale \`{e}
l'applicazione di questa propriet\`{a} nei test di primalit\`{a}?
\vv

\item{5.} Enunciare un algoritmo per calcolare il simbolo di
Jacobi di due interi con tempo di esecuzione polinomiale. %%%%%%%%%%%%%

\vv \item{6.} Descrivere in dettaglio il crittosistema El-Gamal
facendo un esempio nel caso del gruppo moltiplicativo di un campo
finito.

\ve \ \vs \item{7.} Simulare uno scambio delle chiavi alla
Diffie--Hellmann in un campo finito con $32$ elementi

\vv \item{8.} Dimostrare che $x^{p^h}-x+1$ non ammette mai radici
in un campo finito con ${\bf F}_{p^h}$ elementi. E' sempre
irriducibile?
%$$(x^2+x+1)(x^{6} + x^{5} + x^{3} +x^2 + 1)$$

 \vv
\item{9.} Calcolare la probabilit\`{a} che un polinomio
irriducibile di grado 6 su ${\bf F}_{11}$ risulti primitivo. Dare
un esempio di polinomio irriducibile e non primitivo.


\ve\ \vs

\item{10.} Enunciare l'algoritmo Pohlig--Hellmann per calcolare i
logaritmi discreti in un gruppo ciclico finito dimostrandone la validit\`{a}.
%%%%%%%%%%%

\vv\vst

\item{11.} Dopo aver dimostrato che \`{e} una curva ellittica si
${\bf F}_7$, calcolare la struttura del gruppo dei punti razionali
di $y^2=x^3+x$.%%%%%%%%%%%%%

\vv\vst

\item{12.} Enunciare le formule per la duplicazione di un punto
razionale su una curva ellittica e spiegare come si ottengono. %%%%%%%%%%
\vv\vst


\centerline{\hskip 6pt\vbox{\tabskip=0pt \offinterlineskip
\def \trl{\noalign{\hrule}}
\halign to447pt{\strut#& \vrule#\tabskip=0.7em plus 1em&
\hfil#& \vrule#& \hfill#\hfil& \vrule#&
\hfil#& \vrule#& \hfill#\hfil& \vrule#&
\hfil#& \vrule#& \hfill#\hfil& \vrule#&
\hfil#& \vrule#& \hfill#\hfil& \vrule#&
\hfil#& \vrule#& \hfill#\hfil& \vrule#&
\hfil#& \vrule#& \hfill#\hfil& \vrule#&
\hfil#& \vrule#& \hfill#\hfil& \vrule#&
\hfil#& \vrule#& \hfill#\hfil& \vrule#&
\hfil#& \vrule#\tabskip=0pt\cr\trl
&& NOME E COGNOME && 1 && 2 && 3 && 4 && 5 && 6 && 7 && 8 && 9 && 10 && 11 && 12 && TOT. &\cr\trl
&& &&   &&   &&   &&   &&   &&   &&   &&   &&   &&    &&   &&   && &\cr
&& \dotfill &&   &&   &&   &&   &&   &&   &&   &&   &&   &&    &&  &&   && &\cr\trl
}}}
 \bye
