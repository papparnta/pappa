\nopagenumbers
\font\title=cmti12
%\def\ve{\vfill\eject}
%\def\vv{\vfill}
%\def\vs{\vskip-2cm}
%\def\vss{\vskip10cm}
%\def\vst{\vskip7cm}

\def\ve{\bigskip\bigskip}
\def\vv{\bigskip\bigskip}
\def\vs{}
\def\vss{}
\def\vst{\bigskip\bigskip}

\hsize=20cm
\vsize=28cm
\hoffset=-2.2cm
\voffset=-0.5cm
\parskip=-.1cm
\ \vs \hskip -6mm CR1 AA02/03\ (Crittografia a chiave
pubblica)\hfill ESAME SCRITTO \hfill Roma, 13 Giugno 2003. \hrule
\bigskip\noindent
{\title COGNOME}\  \dotfill\  {\title NOME}\ \dotfill
{\title MATRICOLA}\ \dotfill\
\smallskip  \noindent
Risolvere il massimo numero di esercizi accompagnando le risposte
con spiegazioni chiare ed essenziali. \it Inserire le risposte
negli spazi predisposti. NON SI ACCETTANO RISPOSTE
SCRITTE SU ALTRI FOGLI. Scrivere il proprio nome anche nell'ultima
pagina. \rm 1 Esercizio = 3 punti. Tempo previsto: 2 ore. Nessuna
domanda durante la prima ora e durante gli ultimi 20 minuti.
\smallskip
\hrule
\medskip
\item{1.} Determinare una stima per il numero di operazioni bit necessarie
a moltiplicare due matrici $n\times n$ i cui coefficienti sono minori di $e^n$.

\vv
\item{2.} Calcolare il numero di soluzioni della seguente equazione
$$x^5+x^2+x+1\bmod 2\cdot 3\cdot 5.$$

\vv
\item{3.} Dopo aver enunciato le propriet\`{a} dei numeri di Carmichael, dimostrare che $75361$
\`{e} un numero di Carmichael.

 \ve \vs

\item{4.} Spiegare il funzionamento del test di primalit\`{a} di Miller Rabin introducendo le
nozioni necessarie.

\vv

\item{5.} Calcolare il seguente simbolo di Jacobi senza fattorizzare $\left({33331\over44447}\right)$. %-1

\vv
\item{6.} Descrivere in dettaglio il crittosistema Massey Omura facendo un
esempio nel caso del gruppo moltiplicativo di un campo finito.

\ve
\ \vs
\item{7.} Simulare uno scambio delle chiavi alla Diffie--Hellmann in un campo finito con
$49$ elementi \hfill {\it suggerimento: Usare il polinomio $x^2+1$}

\vv
\item{8.} Dimostrare che $x^p+a$ \`{e} riducibile in ogni campo finito ${\bf F}_p$.

 \vv
\item{9.} Calcolare la probabilit\`{a} che un polinomio irriducibile di grado 11 su ${\bf F}_{7}$
risulti primitivo. Dare un esempio di polinomio irriducibile e non primitivo.

\ve
\ \vs
\item{10.} Su ${\bf F}_{31}$ calcolare i seguenti logaritmi discreti: $\log_3(11)$
e $\log_{17}(11)$.

\vv\vst



\item{11.} Dopo aver dimostrato che \`{e} una curva ellittica si ${\bf F}_7$,
calcolare la struttura del gruppo dei punti razionali di $y^7=x^3+x+3$.

\vv\vst

\item{12.} Descrivere l'algoritmo {\it Baby steps - Giant steps} (Shanks) per calcolare il
numero dei punti razionali del gruppo dei punti razionali di una curva ellittica
definita su in campo finito.
\vv\vst


\centerline{\hskip 6pt\vbox{\tabskip=0pt \offinterlineskip
\def \trl{\noalign{\hrule}}
\halign to447pt{\strut#& \vrule#\tabskip=0.7em plus 1em&
\hfil#& \vrule#& \hfill#\hfil& \vrule#&
\hfil#& \vrule#& \hfill#\hfil& \vrule#&
\hfil#& \vrule#& \hfill#\hfil& \vrule#&
\hfil#& \vrule#& \hfill#\hfil& \vrule#&
\hfil#& \vrule#& \hfill#\hfil& \vrule#&
\hfil#& \vrule#& \hfill#\hfil& \vrule#&
\hfil#& \vrule#& \hfill#\hfil& \vrule#&
\hfil#& \vrule#& \hfill#\hfil& \vrule#&
\hfil#& \vrule#\tabskip=0pt\cr\trl
&& NOME E COGNOME && 1 && 2 && 3 && 4 && 5 && 6 && 7 && 8 && 9 && 10 && 11 && 12 && TOT. &\cr\trl
&& &&   &&   &&   &&   &&   &&   &&   &&   &&   &&    &&   &&   && &\cr
&& \dotfill &&   &&   &&   &&   &&   &&   &&   &&   &&   &&    &&  &&   && &\cr\trl
}}}
 \bye
