
\documentclass[italian,a4paper,10pt]{report}
\usepackage{babel,amsmath,amssymb,amsbsy,amsfonts,latexsym,exscale,
amsthm,epsf,colordvi}
\newtheorem{Proposition}{Proposition}[section]
\newtheorem{Theo}{Teorema}[section]
\newtheorem{Lemma}{Lemma}[section]
\newtheorem{Es}{Esercizio}[section]
\newtheorem{Corollary}{Corollary}[section]
\newcommand{\om}{\Omega_t}
\newcommand{\la}{\lambda}
\newcommand{\fine}{$\blacksquare$}
\newcommand{\e}{\`e\;}
\newcommand{\E}{\`E\;}
\newcommand{\R}{\mathbb{R}}
\newcommand{\F}{\mathbb{F}}
\newcommand{\Z}{\mathbb{Z}}
\newcommand{\El}{\mathcal{E}}
\newcommand{\N}{\mathbb{N}}
\newcommand{\Lap}{\triangle}
\newcommand{\sol}{{\bf Soluzione.}\quad }
\newcommand{\displayfrac}[2]{\frac{\displaystyle #1}{\displaystyle #2}}
\renewcommand{\baselinestretch}{1,2}
\parindent 0pc
%\parskip 6pt
\overfullrule=0pt
\begin{document}
\centerline{{\bf Universit\`a degli Studi Roma Tre}}
\centerline{{\bf Corso di Laurea in Matematica, A.A. 2002/2003}}
\centerline{{\bf Soluzione del II Esonero di CR1 - Crittografia
1}} \centerline{A cura di Andrea Susa} \vspace{15mm}

1. Definire il concetto di  curva ellittica su un campo finito ed
il gruppo di Mordell-Weyll associato. \vspace{3mm}

2. Consideriamo la curva $y^2 = x^3 +x + 6$ sul campo
 $F_{7}$. Descrivere $\El(F_{7})$.

{\bf Soluzione}

Per prima cosa verifichiamo che la curva ellittica sia ben
definita, cio\e valga la relazione:
$$
4a^3 + 27b^2 \not\equiv 0  \qquad (\bmod\;7).
$$
Nel nostro caso:
$$
4 + 27(-1)^2 \equiv 5 \not\equiv 0\qquad (\bmod\;7).
$$
Determiniamo i punti finiti di $\mathcal{E}(\F_7)$:
\begin{center}
\begin{tabular}{|c|c|c|c|}
  \hline
  $x$ & $x^3+x+6 \;(\bmod\;7)$ & RQ & $y$ \\
 \hline
  0 & 6 & no &  \\
   \hline
  1 & 1 & si & $\pm 1$ \\
 \hline
  2 & 2 & si & $\pm 3$ \\
\hline
  3 & 1 & si & $\pm 1$ \\
\hline
  4 & 4 & si & $\pm 2$ \\
  \hline
  5 & 3 & no &  \\ \hline
 6 & 4 & si & $\pm 2$ \\ \hline
\end{tabular}
\end{center}

Quindi $\El(\F_7) \simeq \F_{11}$.



 3. Descrivere l'algoritmo dello scambio delle chiavi di Diffie-Helmann sulle
curve
 ellittiche. \vspace{5mm}

4. Sia $g$ una radice primitiva di $F_p$. Dimostrare le seguenti
propriet\`{a} del logaritmo discreto:
\begin{align*}
\log_g(a\cdot b) \; &= \log_g(a) + \log_g(b), \qquad &&(\bmod\;p-1)\\
\log_g(a^n) &= n \log_g(a) \qquad &&(\bmod\;p-1).
\end{align*}

{\bf Soluzione}

Siano $g^r = a$, $g^s = b$, $g^n = ab$. Quindi $\log_g (ab) = n$.
Ma
$$g^n = ab =(g^r)(g^s) = g^{r+s} \quad \Longrightarrow \quad n
\equiv r+s \quad (\bmod\;p-1).
$$
Allora:
$$
\log_g (ab) = n \equiv r + s = log_g(a) + \log_g(b) \qquad
(\bmod\;p-1).
$$

Per induzione su $n$:

se $n=2$ per la precedente $\log_g(a^2) = \log_g(a) + \log_g(a) =
2\log_g(a)$.

Per ipotesi induttiva $\log_g(a^{n-1}) = (n-1)\log_g(a)$.

Allora se $n \geq 3$, $\log_g(a^{n}) = \log_g(a) + \log_g(a^{n-1})
= n\log_g(a)$. \vspace{5mm}

5. Consideriamo il campo base $F_5$. Determinare il numero dei
polinomi monici irriducibili di grado 12. Qual \`{e} la
probabilit\`{a} che, preso random un polinomio monico di grado 12,
esso sia irriducibile?  \vspace{5mm}

{\bf Soluzione}

Dobbiamo calcolare:
$$
5^{12} = \sum_{d|12}\, dN_d(5).
$$
Abbiamo che:
$$
N_1(5) = 5, \qquad N_2(5) = 10, \qquad N_3(5) = 40
$$
Mentre:
\begin{align*}
N_4(5) &=\frac{1}{4} \left( 5^4 - N_1(5) - 2N_2(5) \right) = 150\\
\\
N_6(5) &= \frac{1}{6} \left( 5^6 - N_1(5) - 2N_2(5) - 3N_3(5)
\right) = 2580\\
\end{align*}
Quindi:
\begin{align*}
N_{12}(5) &= \frac{1}{12} (5^{12} - 5 - 20 - 120 - 4\cdot 150 -
6\cdot 2580) =\\
&=20343700.
\end{align*}

Calcoliamo la probabilit\`{a}: se indichiamo con $T_{k}(p)=p^k$ il
numero dei polinomi monici di grado $k$ su $\F_p$, allora
$$
\frac{N_{12}(5)}{T_{12}(5)} = \frac{20343700}{5^{12}} = 0.08332
$$ \vspace{5mm}


6. Consideriamo il polinomio $f = x^3 + x +1 \in F_5[x]$. Dopo
aver verificato che \`{e} irriducibile, trovare una radice
primitiva per il campo:
$$F_{5}[x]/(f) \; = \{a + b\theta + c\theta^2\; |\;
\theta^3 = - \theta -1 \}.$$ \vspace{2mm}

{\bf Soluzione}

Il polinomio \e irriducibile in quanto ha grado $3$, e quindi se
fosse riducibile dovrebbe avere una radice sul campo. Ma si
calcola che $f(0) = 1$, $f(\pm 1) = (\pm 1) + (\pm 1)+1 \ne 0$,
$f(\pm 2) = (\pm 8) + (\pm 2)+1 \ne 0$. Quindi $f$ riducibile.

Sia $\theta$ una radice formale. Allora $\theta^2+1$ \e primitiva.
\vspace{5mm}









7. Calcolare il $\log_3(13)$ nel campo $\F_{31}$, utilizzando un
qualsiasi metodo. \vspace{5mm}

{\bf Soluzione}

Utiliziamo l'algortimo di Shanks:

$m= [\sqrt{31} ]= 5$. Otteniamo le seguenti liste:
\begin{center}
\begin{tabular}{|c|c|c|c|c|c|}
  \hline
  j & 0 & 1 & 2 & 3 & 4 \\
  \hline
  $g^{mj}$ & 1 & 26 & 25 & 30 & 5 \\ \hline
\end{tabular}
\end{center}
\begin{center}
\begin{tabular}{|c|c|c|c|c|c|}
  \hline
  i& 0 & 1 & 2 & 3 & 4 \\
  \hline
 $ 13\cdot g^{-i}$ & 13 & 25 & 29 & 20 &17 \\ \hline
\end{tabular}
\end{center}

Le coppie sono: $(2,25)$ e $(1,25)$. Quindi $\log_3(13) = mj + i =
5 \cdot 2 + 1=11$. \vspace{5mm}

8. Dimostrare che il polinomio $x^2+1$ \`{e} irriducibile su tutti
i campi finiti $F_p$ con $p \equiv 3$ $(\bmod\;4)$. \vspace{5mm}

{\bf Soluzione} Il polinomio $f=x^2+1$ \e irriducibile su $\F_p$
se e soltanto se non ha radici su $\F_p$, cio\e se $\left(
\frac{-1}{p} \right) = -1$. Ma questo accade se e soltanto se $p
\equiv 3$ $(\bmod\;4)$. \vspace{5mm}


9. Si descrivino tutti i valori di $a \in F_p$ per cui il
polinomio: $ f(x) \; =\; x^2 + a $ \`{e} irriducibile, e si
dimostri che tale polinomio non pu\`{o} mai essere primitivo.

{\bf Soluzione} $f$ \e irriducibile per ogni $a\in F_p$ tale che
$\left( \frac{-a}{p}\right) = -1$.

Se per assurdo $f$ fosse primitivo, allora avremmo che la sua
radice $\theta$ \e tale che:
$$
ord_p(\theta^2) = ord_p(-a) \leq p-1.
$$
Quindi: $ord_p(\theta) = p^2 - 1 = \frac{1}{2} ord_p(\theta^2)
\leq \frac{p-1}{2}$ e questo \e assurdo. \vspace{5mm}





10. Mostrare che tutti i polinomi del tipo $x^p +x +1$ sono
riducibili su $\F_p$, per $p \geq 3$.

{\bf Soluzione} I polinomi del tipo $x^p + x+1$ hanno sempre una
radice sul campo $\F_p$. Infatti $(p- 2)^*$, cio\e l'inverso
aritmetico di $-2$ $(\bmod\;p)$ \e sempre soluzione del polinomio:
$$
(p-2^*)^p + (p-2^*) +1 \equiv (-2^*) + (-2^*) +1 \equiv -2 \cdot
(2^*) + 1 \equiv 0 \qquad (\bmod\;p).
$$
\vspace{5mm}




11. Dopo aver scritto tutti i polinomi irriducibili di grado $\leq
4$ su $F_2$, mostrare che il polinomio $f(x) = x^5+x^2+1$ \`{e}
irriducibile. \vspace{5mm}

{\bf Soluzione} La seguente tabella mostra tutti i polinomi
irriducibili di grado $\leq 4$.
\begin{center}
\begin{tabular}[c]{|c|c|c|c|}
    \hline
    $n=1$         &$x$ &$ x+1$ &$$\\ \hline
    $n=2$      &$x^2 + x + 1$ &$$ &$$\\ \hline
    $n=3$        &$x^3+x^2+1$&$x^3+x+1 $ &$$\\ \hline
    $n=4$        &$x^4+x+1$&$ x^4+x^3+1$&$x^4+x^3+x^2+x+1$ \\
    \hline
 \end{tabular}
\end{center}

Si verifica che $f$ non ha radici sul campo, quindi se fosse
riducibile dovrebbe essere il prodotto di un polinomio di grado
tre con un polinomio di grado due.

Ma svolgendo la divisione di $f$ con l'unico polinomio polinomio
irriducibile di grado 2 otteniamo:
$$
f = (x^2+x+1)(x^3+x^2) + 1
$$
quindi $f$ \e irriducibile.

12. Descrivere tutti i sottocampi di $F_{5^{30}}$.\vspace{5mm}

{\bf Soluzione} Esiste un sottocampo per ogni divisore proprio di
30.


\end {document}
