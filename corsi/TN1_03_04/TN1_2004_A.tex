\nopagenumbers \font\title=cmti12
%\def\ve{\vfill\eject}
%\def\vv{\vfill}
%\def\vs{\vskip-2cm}
%\def\vss{\vskip10cm}
%\def\vst{\vskip13.3cm}

\def\ve{\bigskip}
\def\vv{\bigskip}
\def\vs{}
\def\vss{}
\def\vst{\bigskip}

\hsize=19cm
\vsize=27.58cm
\hoffset=-1.6cm
\voffset=0.5cm
\parskip=-.1cm
\ \vs \hskip -6mm TN1 AA03/04\ (Teoria dei Numeri)\hfill ESAME SCRITTO 
\hfill Roma, 7 Giugno 2004. \hrule
\bigskip\noindent
{\title COGNOME}\  \dotfill\  {\title NOME}\ \dotfill {\title
MATRICOLA}\ \dotfill\
\smallskip  \noindent
Risolvere il massimo numero di esercizi accompagnando le risposte
con spiegazioni chiare ed essenziali. \it Inserire le risposte
negli spazi predisposti. NON SI ACCETTANO RISPOSTE SCRITTE SU
ALTRI FOGLI. Scrivere il proprio nome anche nell'ultima pagina.
\rm 1 Esercizio = 3 punti. Tempo previsto: 2 ore. Nessuna domanda
durante la prima ora e durante gli ultimi 20 minuti.
\smallskip
\hrule
\medskip

\item{1.} Descrivere tutte le soluzioni dell'equazione diofantea
$3x+y=7$.

\vv \item{2.} Trovare tutte le soluzioni intere di
$\cases{X\equiv3(\bmod5)\cr X\equiv5(\bmod3)}$ in $[100,200)$.


\ve\ \vs \item{3.} Enunciare e dimostrare il Teorema di Lagrange
per il numero di soluzioni di una congruenza del tipo
$f(X)\equiv0(\bmod p)$.

\vv

\item{4.} Determinare il numero di soluzioni di $X^3+X+10\equiv0(\bmod40)$.

\ve\ \vs

\item{5.} Determinare tutte le radici primitive di ${\bf Z}/26{\bf
Z}$ e  ${\bf Z}/20{\bf Z}$.


\vv \item{6.} Mostrare che per ogni $d|(p-1)$, in ${\bf Z}/p{\bf
Z}^*$ ci sono esattamente $\varphi(d)$ elementi di ordine $d$.

 \ve\ \vs

\item{7.} Calcolare il seguente simbolo di Jacobi
$\Big({1111\over5433}\Big)$.

\vv \item{8.} Mostrare che $\sum_{a=1}^p\Big({a\over p}\Big)=0.$

\ve\ \vs

\item{9.} Calcolare $(\sigma*\tau*\mu)(2^8\cdot5)$.

\vv

\item{10.} Enunciare il teorema di classificazione per le terne
pitagoriche primitive positive.


\ve\ \vs

\item{11.} Scrivere $3036285$ come somma di due quadrati.

\vss

\item{12.} Dimostrare che esistono infiniti numeri interi che non
si possono scrivere come somma di tre quadrati.

\vv

\ \vst\vskip-8mm

\centerline{\hskip 6pt\vbox{\tabskip=0pt \offinterlineskip
\def \trl{\noalign{\hrule}}
\halign to500pt{\strut#& \vrule#\tabskip=0.7em plus 1em&
\hfil#& \vrule#& \hfill#\hfil& \vrule#&
\hfil#& \vrule#& \hfill#\hfil& \vrule#&
\hfil#& \vrule#& \hfill#\hfil& \vrule#&
\hfil#& \vrule#& \hfill#\hfil& \vrule#&
\hfil#& \vrule#& \hfill#\hfil& \vrule#&
\hfil#& \vrule#& \hfill#\hfil& \vrule#&
\hfil#& \vrule#& \hfil#& \vrule#\tabskip=0pt\cr\trl
&& NOME E COGNOME && 1 && 2 && 3 && 4 && 5 && 6 && 7 && 8 && 9 && 10 && 11 && 12 &&  TOT. &\cr\trl
&& &&   &&   &&     &&   &&   &&   &&   &&   &&    &&   &&   &&  && &\cr
&& \dotfill &&     &&   &&   &&   &&   &&   &&    &&  &&   && && && && &\cr\trl
}}}
 \bye
