\nopagenumbers \font\title=cmti12
\def\ve{\vfill\eject}
\def\vv{\vfill}
\def\vs{\vskip-2cm}
\def\vss{\vskip10cm}
\def\vst{\vskip13.3cm}

%\def\ve{\bigskip}
%\def\vv{\bigskip}
%\def\vs{}
%\def\vss{}
%\def\vst{\bigskip}

\hsize=19cm
\vsize=27.58cm
\hoffset=-1.6cm
\voffset=0.5cm
\parskip=-.1cm
\ \vs \hskip -6mm TN1 AA03/04\ (Teoria dei Numeri)\hfill ESAME DI
FINE SEMESTRE \hfill Roma, 31 Maggio 2004. \hrule
\bigskip\noindent
{\title COGNOME}\  \dotfill\  {\title NOME}\ \dotfill {\title
MATRICOLA}\ \dotfill\
\smallskip  \noindent
Risolvere il massimo numero di esercizi accompagnando le risposte
con spiegazioni chiare ed essenziali. \it Inserire le risposte
negli spazi predisposti. NON SI ACCETTANO RISPOSTE SCRITTE SU
ALTRI FOGLI. Scrivere il proprio nome anche nell'ultima pagina.
\rm 1 Esercizio = 3 punti. Tempo previsto: 2 ore. Nessuna domanda
durante la prima ora e durante gli ultimi 20 minuti.
\smallskip
\hrule
\medskip

\item{1.} Calcolare per quali valori di $\alpha\in{{\bf Z}/13 {\bf
Z}}$ la congruenza $$X^2+X+\alpha\equiv0\bmod 13$$ \`{e} risolubile.


\vv \item{2.} Enunciare e dimostrare il criterio di Eulero per il calcolo del
simbolo di Legendre.


\ve\ \vs \item{3.} Si calcoli il simbolo di Legendre
$\big({1755\over3001}\big).$


\vv

\item{4.} Dopo aver definito la nozione di residuo quadratico, dimostrare che il numero di
residui quadratici in ${\bf Z}/p{\bf Z}^*$ \`{e} $(p-1)/2$.

 \ve\ \vs

\item{5.} Mostrare che se $p\equiv 9\bmod 28$, allora
$\big({7\over p}\big)=1$.


\vv \item{6.} Sia $\omega(n)$ il numero di divisori primi distinti
dell'intero $n$. Mostrare che per ogni numero complesso $z$, la
funzione $f_z(n):=z^{\omega(n)}$ \`{e} moltiplicativa. Nel caso in
cui $z=i$, calcolare $(f_z*\mu)(60)$.

\ve\ \vs

\item{7.} Enunciare e dimostrare la formula di inversione di M\"obius.

\vv \item{8.} Elencare tutte le terne pitagoriche primitive e
positive $(x,y,z)$ con $x,y,z\leq85$.

\ve\ \vs

\item{9.} Enunciare il teorema di caratterizzazione per i numeri
che si possono esprimere come somma di due quadrati.

\vv

\item{10.} Esprimere $5^s13^t$ per ogni $s,t\in{\bf N}$ come somma di due quadrati.

\ve\ \vs

\item{11.} Dopo aver espresso $3$ e $5$ come somma di tre quadrati, mostrare che non \`{e} detto che
se due interi si esprimono come somma di tre quadrati, allora anche il loro prodotto si
esprime come somma di tre quadrati. Fornire pi\`{u} di un contro esempio.

\vss

\item{12.} Scrivere $47$ come somma del minor numero possibile di quadrati.

\vv

\ \vst\vskip-8mm

\centerline{\hskip 6pt\vbox{\tabskip=0pt \offinterlineskip
\def \trl{\noalign{\hrule}}
\halign to500pt{\strut#& \vrule#\tabskip=0.7em plus 1em&
\hfil#& \vrule#& \hfill#\hfil& \vrule#&
\hfil#& \vrule#& \hfill#\hfil& \vrule#&
\hfil#& \vrule#& \hfill#\hfil& \vrule#&
\hfil#& \vrule#& \hfill#\hfil& \vrule#&
\hfil#& \vrule#& \hfill#\hfil& \vrule#&
\hfil#& \vrule#& \hfill#\hfil& \vrule#&
\hfil#& \vrule#& \hfil#& \vrule#\tabskip=0pt\cr\trl
&& NOME E COGNOME && 1 && 2 && 3 && 4 && 5 && 6 && 7 && 8 && 9 && 10 && 11 && 12 &&  TOT. &\cr\trl
&& &&   &&   &&     &&   &&   &&   &&   &&   &&    &&   &&   &&  && &\cr
&& \dotfill &&     &&   &&   &&   &&   &&   &&    &&  &&   && && && && &\cr\trl
}}}
 \bye
