\nopagenumbers \font\title=cmti12
%\def\ve{\vfill\eject}
%\def\vv{\vfill}
%\def\vs{\vskip-2cm}
%\def\vss{\vskip10cm}
%\def\vst{\vskip13.3cm}

\def\ve{\bigskip}
\def\vv{\bigskip}
\def\vs{}
\def\vss{}
\def\vst{\bigskip}

\hsize=19cm
\vsize=27.58cm
\hoffset=-1.6cm
\voffset=0.5cm
\parskip=-.1cm
\ \vs \hskip -6mm TN1 AA03/04\ (Teoria dei Numeri)\hfill ESAME SCRITTO 
\hfill Roma, 14 Gennaio 2005. \hrule
\bigskip\noindent
{\title COGNOME}\  \dotfill\  {\title NOME}\ \dotfill {\title
MATRICOLA}\ \dotfill\
\smallskip  \noindent
Risolvere il massimo numero di esercizi accompagnando le risposte
con spiegazioni chiare ed essenziali. \it Inserire le risposte
negli spazi predisposti. NON SI ACCETTANO RISPOSTE SCRITTE SU
ALTRI FOGLI. Scrivere il proprio nome anche nell'ultima pagina.
\rm 1 Esercizio = 3 punti. Tempo previsto: 2 ore. Nessuna domanda
durante la prima ora e durante gli ultimi 20 minuti.
\smallskip
\hrule
\medskip

\item{1.} Descrivere tutte le soluzioni dell'equazione diofantea
$3x+y-7z=70$.

\vv \item{2.} Determinare i valore del parametro $\tau$ per cui is
sistema di congruenze sotto non ammette un unica soluzione:
$$\cases{X-\tau Y\equiv3(\bmod13)\cr X+\tau^2Y\equiv5(\bmod13)}.$$

\ve\ \vs \item{3.} Enunciare e dimostrare il Piccolo Teorema di
Fermat e il Teorema di Wilson.\vv

\item{4.} Calcolare il numero delle soluzioni modulo 125 della seguente congruenza polinomiale:
$X^3 + 11X^2 + 24X + 14 \equiv 0 \bmod 125$. \ve\ \vs

\item{5.} Si enunci il Teorema del sollevamento per soluzioni di congruenze polinomiali.

\vv \item{6.} Dopo aver definito la nozione di radice primitiva,
mostrare direttamente che non esiste una radice primitiva modulo 12.
 \ve\ \vs

\item{7.} Calcolare il seguente simbolo di Jacobi
$\Big({3333\over4567}\Big)$.

\vv \item{8.}Quante e quali soluzioni ha la congruenza $X^{15}\equiv
5 \bmod 93$? \ve\ \vs

\item{9.} Enunciare e dimostrare la formula di inversione di M\"obius.
\vv

\item{10.} Enunciare il teorema di classificazione per le terne
pitagoriche primitive positive. \ve\ \vs

\item{11.} Scrivere (se \`e possibile) $12182625$ come somma di due quadrati.
\vss

\item{12.} Enunciare il teorema di caratterizzazione per i
numeri che si possono esprimere come somma di due quadrati.\vv

\ \vst\vskip-8mm

\centerline{\hskip 6pt\vbox{\tabskip=0pt \offinterlineskip
\def \trl{\noalign{\hrule}}
\halign to500pt{\strut#& \vrule#\tabskip=0.7em plus 1em&
\hfil#& \vrule#& \hfill#\hfil& \vrule#&
\hfil#& \vrule#& \hfill#\hfil& \vrule#&
\hfil#& \vrule#& \hfill#\hfil& \vrule#&
\hfil#& \vrule#& \hfill#\hfil& \vrule#&
\hfil#& \vrule#& \hfill#\hfil& \vrule#&
\hfil#& \vrule#& \hfill#\hfil& \vrule#&
\hfil#& \vrule#& \hfil#& \vrule#\tabskip=0pt\cr\trl
&& NOME E COGNOME && 1 && 2 && 3 && 4 && 5 && 6 && 7 && 8 && 9 && 10 && 11 && 12 &&  TOT. &\cr\trl
&& &&   &&   &&     &&   &&   &&   &&   &&   &&    &&   &&   &&  && &\cr
&& \dotfill &&     &&   &&   &&   &&   &&   &&    &&  &&   && && && && &\cr\trl
}}}
 \bye
