\nopagenumbers \font\title=cmti12
%\def\ve{\vfill\eject}
%\def\vv{\vfill}
%\def\vs{\vskip-2cm}
%\def\vss{\vskip10cm}
%\def\vst{\vskip13.3cm}

\def\ve{\bigskip}
\def\vv{\bigskip}
\def\vs{}
\def\vss{}
\def\vst{\bigskip}

\hsize=19cm
\vsize=27.58cm
\hoffset=-1.6cm
\voffset=0.5cm
\parskip=-.1cm
\ \vs \hskip -6mm TN1 AA03/04\ (Teoria dei Numeri)\hfill ESAME SCRITTO 
\hfill Roma, 15 Settembre 2004. \hrule
\bigskip\noindent
{\title COGNOME}\  \dotfill\  {\title NOME}\ \dotfill {\title
MATRICOLA}\ \dotfill\
\smallskip  \noindent
Risolvere il massimo numero di esercizi accompagnando le risposte
con spiegazioni chiare ed essenziali. \it Inserire le risposte
negli spazi predisposti. NON SI ACCETTANO RISPOSTE SCRITTE SU
ALTRI FOGLI. Scrivere il proprio nome anche nell'ultima pagina.
\rm 1 Esercizio = 3 punti. Tempo previsto: 2 ore. Nessuna domanda
durante la prima ora e durante gli ultimi 20 minuti.
\smallskip
\hrule
\medskip

\item{1.} Descrivere tutte le soluzioni dell'equazione diofantea
$x+y+z=1$ con $\max\{|x|,|y|,|z|\}\leq 5$.

\vv \item{2.} Per quali valori del parametro $\lambda$, il seguente sistema
di congruenze non ammette un unica soluzione?

$$\cases{2x-y\equiv0(\bmod11)\cr 3x+\lambda^2y\equiv1(\bmod11).}$$


\ve\ \vs \item{3.} Enunciare e dimostrare il Teorema cinese dei resti.

\vv

\item{4.} Calcolare il numero delle soluzioni modulo 125 della seguente congruenza 
polinomiale $55x^3+120x+50\equiv0(\bmod125)$.

\ve\ \vs

\item{5.} Descrivere l'algoritmo di Gau\ss\ per calcolare radici primitive.


\vv \item{6.} Determinare il numero di radici primitive modulo 125 giustificando la risposta.

 \ve\ \vs

\item{7.} Calcolare il seguente simbolo di Jacobi
$\Big({1999\over2003}\Big)$.

\vv \item{8.} Usare il lemma di Gaus\ss\ per dimostrare che $\left({2\over p}\right)=(-1)^{(p^2-1)/8}.$

\ve\ \vs

\item{9.} Stabilire quando una funzione aritmetia \`{e} invertibile rispetto al 
prodotto di Dirichlet fornendo una formula per l'inversa.

\vv

\item{10.} Calcolare $\mu*\tau*\mu(360)$.


\ve\ \vs

\item{11.} Scrivere $8330$ come somma di due quadrati.

\vss

\item{12.} Mostrare che l'insieme dei numeri che si possono scrivere come la somma di un 
quadrato e il triplo di un quadrato \`{e} chiuso rispetto al prodotto (cio\`{e} che se 
$n=x^2+3y^2$ e $m=a^2+3b^2$, allora esistono interi $r, s$ tali che $nm=r^2+3s^2$).

\vv

\ \vst\vskip-8mm

\centerline{\hskip 6pt\vbox{\tabskip=0pt \offinterlineskip
\def \trl{\noalign{\hrule}}
\halign to500pt{\strut#& \vrule#\tabskip=0.7em plus 1em&
\hfil#& \vrule#& \hfill#\hfil& \vrule#&
\hfil#& \vrule#& \hfill#\hfil& \vrule#&
\hfil#& \vrule#& \hfill#\hfil& \vrule#&
\hfil#& \vrule#& \hfill#\hfil& \vrule#&
\hfil#& \vrule#& \hfill#\hfil& \vrule#&
\hfil#& \vrule#& \hfill#\hfil& \vrule#&
\hfil#& \vrule#& \hfil#& \vrule#\tabskip=0pt\cr\trl
&& NOME E COGNOME && 1 && 2 && 3 && 4 && 5 && 6 && 7 && 8 && 9 && 10 && 11 && 12 &&  TOT. &\cr\trl
&& &&   &&   &&     &&   &&   &&   &&   &&   &&    &&   &&   &&  && &\cr
&& \dotfill &&     &&   &&   &&   &&   &&   &&    &&  &&   && && && && &\cr\trl
}}}
 \bye
