\nopagenumbers \font\title=cmti12
%\def\ve{\vfill\eject}
%\def\vv{\vfill}
%\def\vs{\vskip-2cm}
%\def\vss{\vskip10cm}
%\def\vst{\vskip13.3cm}

\def\ve{\bigskip}
\def\vv{\bigskip}
\def\vs{}
\def\vss{}
\def\vst{\bigskip}

\hsize=19cm
\vsize=27.58cm
\hoffset=-1.6cm
\voffset=0.5cm
\parskip=-.1cm
\ \vs \hskip -6mm TN1 AA03/04\ (Teoria dei Numeri)\hfill ESAME SCRITTO 
\hfill Roma, 23 Luglio 2004. \hrule
\bigskip\noindent
{\title COGNOME}\  \dotfill\  {\title NOME}\ \dotfill {\title
MATRICOLA}\ \dotfill\
\smallskip  \noindent
Risolvere il massimo numero di esercizi accompagnando le risposte
con spiegazioni chiare ed essenziali. \it Inserire le risposte
negli spazi predisposti. NON SI ACCETTANO RISPOSTE SCRITTE SU
ALTRI FOGLI. Scrivere il proprio nome anche nell'ultima pagina.
\rm 1 Esercizio = 3 punti. Tempo previsto: 2 ore. Nessuna domanda
durante la prima ora e durante gli ultimi 20 minuti.
\smallskip
\hrule
\medskip

\item{1.} Descrivere tutte le soluzioni dell'equazione diofantea
$x-2y+3z=2$.


 \vv \item{2.} Trovare tutte le soluzioni intere di
$\cases{X+Y\equiv3(\bmod7)\cr X+3Y\equiv0(\bmod7)}$ con $X,Y\in
[0,20)$.


\ve\ \vs \item{3.} Enunciare e dimostrare il Teorema del
sollevamento.

\vv

%
\item{4.} Determinare il numero di soluzioni di $X^3\equiv2(\bmod32)$.

\ve\ \vs

%
\item{5.} Definire la nozione di radice primitiva e descrivere quale
sono glie elementi di ${\bf N}$ che ammettono una radice primitiva.



\vv \item{6.} Dimostrare che ${\bf Z}/36{\bf Z}$ non ammette una
radice primitiva determinando l'ordine di ciascuno dei suoi elementi
invertibili.
 \ve\ \vs

\item{7.} Calcolare il seguente simbolo di Jacobi
$\Big({2004\over1999}\Big)$.


\vv \item{8.} Dopo aver definito la nozione di simbolo di Legendre,
se ne enuncino e dimostrino le propriet\`{a} principali\ve\ \vs


\item{9.} Enunciare e dimostrare la formula di inversione di M\"obius.

\vv


\item{10.} Scrivere $127$ come la somma del minor numero di quadrati.

\ve\ \vs


\item{11.} Scrivere $1611090$ come somma di due quadrati.

\vss

\item{12.} Sia $n$ un numero dispari che si pu\`{o} scrivere
come $n=x^2+2y^2$. Dimostrare che necessariamente si deve avere
$n\equiv1\bmod8$ oppure $n\equiv3\bmod8$.
 \vv

\ \vst\vskip-8mm

%\centerline{\hskip 6pt\vbox{\tabskip=0pt \offinterlineskip
%\def \trl{\noalign{\hrule}}
%\halign to500pt{\strut#& \vrule#\tabskip=0.7em plus 1em&
%\hfil#& \vrule#& \hfill#\hfil& \vrule#&
%\hfil#& \vrule#& \hfill#\hfil& \vrule#&
%\hfil#& \vrule#& \hfill#\hfil& \vrule#&
%\hfil#& \vrule#& \hfill#\hfil& \vrule#&
%\hfil#& \vrule#& \hfill#\hfil& \vrule#&
%\hfil#& \vrule#& \hfill#\hfil& \vrule#&
%\hfil#& \vrule#& \hfil#& \vrule#\tabskip=0pt\cr\trl
%&& NOME E COGNOME && 1 && 2 && 3 && 4 && 5 && 6 && 7 && 8 && 9 && 10 && 11 && 12 &&  TOT. &\cr\trl
%&& &&   &&   &&     &&   &&   &&   &&   &&   &&    &&   &&   &&  && &\cr
%&& \dotfill &&     &&   &&   &&   &&   &&   &&    &&  &&   && && && && &\cr\trl
%}}}
 \bye
