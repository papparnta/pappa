\nopagenumbers \font\title=cmti12
%\def\ve{\vfill\eject}
%\def\vv{\vfill}
%\def\vs{\vskip-2cm}
%\def\vss{\vskip10cm}
%\def\vst{\vskip13.3cm}

\def\ve{\bigskip\bigskip}
\def\vv{\bigskip\bigskip}
\def\vs{}
\def\vss{}
\def\vst{\bigskip\bigskip}

\hsize=19cm
\vsize=27.58cm
\hoffset=-1.6cm
\voffset=0.5cm
\parskip=-.1cm
\ \vs \hskip -6mm TN1 AA03/04\ (Teoria dei Numeri)\hfill
ESAME DI MET\`{A} SEMESTRE \hfill Roma, 15 Aprile 2004. \hrule
\bigskip\noindent
{\title COGNOME}\  \dotfill\  {\title NOME}\ \dotfill {\title
MATRICOLA}\ \dotfill\
\smallskip  \noindent
Risolvere il massimo numero di esercizi accompagnando le risposte
con spiegazioni chiare ed essenziali. \it Inserire le risposte
negli spazi predisposti. NON SI ACCETTANO RISPOSTE SCRITTE SU
ALTRI FOGLI. Scrivere il proprio nome anche nell'ultima pagina.
\rm 1 Esercizio = 3 punti. Tempo previsto: 2 ore. Nessuna domanda
durante la prima ora e durante gli ultimi 20 minuti.
\smallskip
\hrule
\medskip

\item{1.} Si determinino tutte le soluzioni intere della seguente equazione:
$2X+3Y+5Z=100.$


\vv \item{2.} Per quali valori del parametro $\lambda$ il seguente sistema di
congruenze ammette un unica soluzione?
$\cases{2x-4y\equiv 0 \bmod 7\cr 3x+\lambda^2 y \equiv 1 \bmod 7.}$


\ve\ \vs \item{3.} Dimostrare il piccolo Teorema di Fermat.

\vv


\item{4.} Dimostrare che per ogni primo $p$ la seguente congruenza \`{e} verificata:
$(p-4)!\equiv 6^* \bmod p$
dove $6^*$ \`{e} l'inverso aritmetico modulo $p$.

 \ve\ \vs

\item{5.} Calcolare il numero delle soluzioni modulo $125$ della seguente congruenza polinomiale:
${X^{3} - {11}X^{2} + {24}X - {14}}
\equiv0\bmod 125. $


\vv \item{6.} Calcolare le soluzioni del sistema di congruenze:
$\cases{X\equiv 4 \bmod 5\cr X\equiv 3\bmod 7}$
nell'intervallo $[100,250].$

 \ve\ \vs

\item{7.} Si enunci il Teorema del sollevamento per soluzioni di congruenze polinomiali.


\vv \item{8.} Sia $p$ un primo dispari tale che $q=2p+1$ \`{e} anche primo. Mostrare che se un intero $a$, $2\leq a\leq p-2$
\`{e} tale che $a^p\equiv-1\bmod q$ se e solo se $a$ \`{e} una radice primitiva modulo $q$.

\ve\ \vs

\item{9.} Quante e quali soluzioni ha la congruenza
$X^{15}\equiv 5 \bmod 93?$
\hfill {\it Suggerimento: lavorare modulo primi}

 \vv

\item{10.} Mostrare direttamente che non esiste una radice primitiva modulo $24$.

\ve\ \vs


\item{11.} Illustrare l'algoritmo di Gauss per il calcolo di una radice primitiva.


 \vss

\item{12.} Usare una radice primitiva per mostrare che se $p$ \`{e} primo e $m$ \`{e}
un intero, allora
$$1^m+2^m+\cdots+ (p-1)^m\equiv \cases{0 \bmod p & se $(p-1)\not| m$\cr -1 \bmod p & se $p-1| m$.}$$


%\vv
%
%\ \vst\vskip-8mm
%
%\centerline{\hskip 6pt\vbox{\tabskip=0pt \offinterlineskip
%\def \trl{\noalign{\hrule}}
%\halign to500pt{\strut#& \vrule#\tabskip=0.7em plus 1em&
%\hfil#& \vrule#& \hfill#\hfil& \vrule#&
%\hfil#& \vrule#& \hfill#\hfil& \vrule#&
%\hfil#& \vrule#& \hfill#\hfil& \vrule#&
%\hfil#& \vrule#& \hfill#\hfil& \vrule#&
%\hfil#& \vrule#& \hfill#\hfil& \vrule#&
%\hfil#& \vrule#& \hfill#\hfil& \vrule#&
%\hfil#& \vrule#& \hfil#& \vrule#\tabskip=0pt\cr\trl
%&& NOME E COGNOME && 1 && 2 && 3 && 4 && 5 && 6 && 7 && 8 && 9 && 10 && 11 && 12 &&  TOT. &\cr\trl
%&& &&   &&   &&     &&   &&   &&   &&   &&   &&    &&   &&   &&  && &\cr
%&& \dotfill &&     &&   &&   &&   &&   &&   &&    &&  &&   && && && && &\cr\trl
%}}}
 \bye
