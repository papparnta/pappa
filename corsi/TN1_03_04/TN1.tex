\input programma.sty
\def\square{\hbox{\rlap{$\sqcap$}$\sqcup$}}
\def\abbrcorso{TN1}
\def\titolocorso{introduzione alla teoria dei numeri}
\def\sottotitolo{{\rm http://www.mat.uniroma3.it/users/pappa/CORSI/TN1$_{-}$03$_{-}$04/TN1.htm}}
\def\docente{Prof. Francesco Pappalardi}
\def\crediti{7.5}
\def\semestre{II}
\def\esoneri{1}
\def\scrittofinale{1}
\def\oralefinale{1}
\def\altreprove{0}

\Intestazione

\titoloparagr{Teoria della congruenze}

Propriet\`{a} elementari delle congruenze, criteri elementari di
divisibilit\`{a}, congruenze lineari ed equazioni diofantee lineari,
il piccolo Teorema di Fermat ($a^p\equiv a\bmod p$) e il Teorema di
Eulero ($a^{\varphi(m)}\equiv1\bmod m\Leftarrow(a,m)=1$), Teorema di
Wilson ($(p-1)!\equiv-1\bmod p$), Teorema cinese dei resti,
l'algoritmo dei quadrati successivi, generalit\`{a} sulle congruenze
polinomiali, derivate formali, Teorema del sollevamento, Teorema di
Lagrange, radici primitive dell'unit\`{a}, Teorema di Gauss sull
esistenza di radici primitiva ($\exists g\in{\bf Z}$ t.c. $\langle
g\bmod n\rangle =({\bf Z}/n{\bf Z})^*\Longleftrightarrow
n=2,4,p^k,2p^k (k\in{\bf N},p\geq3)$), algoritmo di Gauss per il
calcolo di radici primitive, congruenze del tipo $X^m\equiv a(\bmod
n)$, congruenze quadratiche, simboli di Legendre e propriet\`{a},
criterio di Eulero, Lemma di Gauss ($\big({2\over
q}\big)=(-1)^{(p^2-1)/8}$), legge di reciprocit\`{a} ($\big({p\over
q}\big)\cdot\big({q\over p}\big) =(-1)^{(p-1)(q-1)/4}$), simboli di
Jacobi e calcolo dei simboli di Legendre senza fattorizzare.

\titoloparagr{Funzioni aritmetiche}

Generalit\`{a}, funzioni aritmetiche moltiplicative e totalmente
moltiplicative, esempi: $\varphi, \tau, \sigma, \sigma_k,
1\hskip-1.2mm1, e, u$, prodotto di Dirichlet di funzioni
aritmetiche, gruppo delle funzioni aritmetiche moltiplicative,
formule di inversione di M\"{o}bius.

\titoloparagr{Terne pitagoriche}

Generalit\`{a}, terne pitagoriche primitive, terne pitagoriche
positive, teorema di classificazione delle terne pitagoriche,
discesa di Fermat e l'equazione diofantea $X^4+Y^4=Z^2$.

\titoloparagr{Interi che sono somme di due quadrati}

Generalit\`{a}, Teorema di Fermat sui primi che possono essere
scritti come somma di due quadrati
($p=\square+\square\Leftrightarrow p\not\equiv3\bmod4$), Teorema di
caratterizzazione sui numeri che possono essere scritti come somma
di due quadrati ($n=\square+\square\Longleftrightarrow\forall p|n,
\big(p\equiv3(\bmod 4)\Rightarrow v_p(n)=2\alpha \big)$), numero di
espressioni.

\titoloparagr{Interi che sono somma di quattro quadrati}

Generalit\`{a}, enunciato del Teorema di Legendre Gauss sui numeri
che si possono scrivere come somma di 3 quadrati
($n=\square+\square+\square\Longleftrightarrow n\neq 4^e(8k+7)
\forall e,k\in{\bf N}$), Formula di Eulero
($(\square+\square+\square+\square)\cdot(\square+\square+\square+\square)=\square+\square+\square+\square$),
Teorema di Lagrange ($n=\square+\square+\square+\square$), enunciato
del problema di Waring.


\testi
\bib\autore{M. Fontana} \altro{Note -
 http://www.mat.uniroma3.it/users/fontana/}
\endbib
\bib \autore{G.H. Hardy -- E.M. Wright} \titolo{An introduction to the
theory of numbers} \editore{The Clarendon Press, Oxford University Press,
New York} \annopub{1979} \altro{(5a Ed.)}
\endbib
\bib \autore{H. Davenport} \titolo{Aritmetica superiore.
Un'introduzione alla teoria dei numeri} \editore{Zanichelli}
\annopub{1994} \endbib

\altritesti
\bib \autore{K.H. Rosen} \titolo{Elementary number theory and its
applications} \editore{Addison Wesley} \annopub{1985} \endbib
\bib \autore{Z.I. Borevich -- I.R. Shafarevich} \titolo{Number theory}
\editore{Academic Press} \annopub{1964} \endbib
\bib \autore{Autori Vari} \titolo{Dispense Online}
\altro{\hfill\break http://www.mat.uniroma3.it/ntheory/lecture$_{-}$notes.html} \endbib
\esami\bye
