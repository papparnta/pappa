\nopagenumbers \font\title=cmti12
\def\ve{\vfill\eject}
\def\vv{\vfill}
\def\vs{\vskip-2cm}
\def\vss{\vskip10cm}
\def\vst{\vskip13.3cm}

% \def\ve{\bigskip\bigskip}
% \def\vv{\bigskip\bigskip}
% \def\vs{}
% \def\vss{}
% \def\vst{\bigskip\bigskip}

\hsize=19cm
\vsize=27.58cm
\hoffset=-1.6cm
\voffset=0.5cm
\parskip=-.1cm
\ \vs \hskip -6mm TN410 AA14/15\ (Elementary Number Theory)\hfill Final Exam \hfill Rome, May 27, 2015. \hrule
\bigskip\noindent
{\title Family Name}\  \dotfill\ {\title Name}\ \dotfill {\title
Student ID(Matricola):}\ \dotfill\
\smallskip  \noindent
Solve the problems adding to the replies short and essential explenations. 
\it Please write the solutions in the designed areas. NO EXTRA SHEETS WILL BE ACCEPTED. 
\rm 1 Problem = 4 marks. Duration: 2 hours. No questions allowed in the first hour and in 
the last 20 minutes.
\smallskip
\hrule\smallskip
\centerline{\hskip 6pt\vbox{\tabskip=0pt \offinterlineskip
\def \trl{\noalign{\hrule}}
\halign to371.7pt{
\strut#& \vrule#\tabskip=1.1em plus 2.9em& \hfil#
       & \vrule#& \hfill#\hfil
       & \vrule#& \hfil#
       & \vrule#& \hfill#\hfil
       & \vrule#& \hfil#
       & \vrule#& \hfill#\hfil
       & \vrule#& \hfil#
       & \vrule#& \hfill#\hfil
       & \vrule#& \hfil#
       & \vrule#& \hfill#\hfil
       & \vrule#&\hfil#
       & \vrule#&\hfil#
       & \vrule#& \hfil#
       &\vrule#\tabskip=0pt\cr
\trl && 1 && 2 && 3a && 3b && 3c && 3d && 4 && 5 && 6 && TOTAL&\cr
\trl && &&   &&     &&   &&   &&  &&   &&    && && &\cr 
&&   &&   &&   &&     &&   && && && && && &\cr
\trl }}}
\medskip


\item{1.} Calculate the continued fraction expansion of $\sqrt{39}$\vv

\item{2.} An irrational number has continued fraction expansion $[3;\overline{1,4}]$. Compute it.\ve\vs

%\item{3.} \vv

\item{3.} State some of the important facts of the Theory of continues fractions.\vv

\item{4.}   Show that $p\ne3$ is a prime number such that 
$p=x^2+3y^2$ if and only if $p\equiv1\bmod3$.\vv

\item{5.} Show that if an integer $m$ has the form $m=2(7+8k)$, the $m$ does not have the form $m=x^2+y^2+2z^2$.\ve\vs

\item{6.} Prove that there are infinitely many integers that can be written in exactly $20$ distict ways as the sum of $2$ squares.\vv
 
\item{7.} Let $n$ and $m$ be integers such that  $5n\equiv 3 \bmod 8m$ and $m\equiv13 \bmod 60$. Compute the Jacobi symbol $\left({m\over n}\right)$.
\ve\vs

\item{8.} Compute the density of the integers that are odd and with last two digits in base $5$ equal to $22$.\vv

\item{9.} Prove the following asymptotic formula:
$$\sum_{p\le T}{\log p\over p}=\log T + O(1).$$
{\ }\hskip 2cm\hfill{\bf hint:\it \ first show that $\sum_{m\le T}{\Lambda(m)}m=\log T + O(1)$ using a formula proven in class. 
Then estimate the contribution of prime powers.}
\ \vst\bye
