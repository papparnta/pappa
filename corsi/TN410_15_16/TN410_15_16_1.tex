\nopagenumbers \font\title=cmti12
\def\ve{\vfill\eject}
\def\vv{\vfill}
\def\vs{\vskip-2cm}
\def\vss{\vskip10cm}
\def\vst{\vskip13.3cm}

% \def\ve{\bigskip\bigskip}
% \def\vv{\bigskip\bigskip}
% \def\vs{}
% \def\vss{}
% \def\vst{\bigskip\bigskip}

\hsize=19cm
\vsize=27.58cm
\hoffset=-1.6cm
\voffset=0.5cm
\parskip=-.1cm
\ \vs \hskip -6mm TN410 AA15/16\ (Elementary Number Theory)\hfill Midterm Exam \hfill Rome, April 12, 2016 \hrule
\bigskip\noindent
{\title Family Name}\  \dotfill\ {\title Name}\ \dotfill {\title
Student ID(Matricola):}\ \dotfill\
\smallskip  \noindent
Solve the problems adding to the replies short and essential explenations. 
\it Please write the solutions in the designed areas. NO EXTRA SHEETS WILL BE ACCEPTED. 
\rm 1 Problem = 4 marks. Duration: 2 hours. No questions allowed in the first hour and in 
the last 20 minutes.
\smallskip
\hrule\smallskip
\centerline{\hskip 6pt\vbox{\tabskip=0pt \offinterlineskip
\def \trl{\noalign{\hrule}}
\halign to320.5pt{
\strut#& \vrule#\tabskip=1.1em plus 2.6em& \hfil#
       & \vrule#& \hfill#\hfil
       & \vrule#& \hfil#
       & \vrule#& \hfill#\hfil
       & \vrule#& \hfil#
       & \vrule#& \hfill#\hfil
       & \vrule#& \hfil#
       & \vrule#& \hfill#\hfil
       & \vrule#& \hfil#
       & \vrule#& \hfill#\hfil
       & \vrule#&\hfil#
       & \vrule#& \hfil#
       &\vrule#\tabskip=0pt\cr
\trl && 1 && 2 && 3 && 4 && 5 && 6 && 7 && 8 && TOTAL&\cr
\trl && &&   &&     &&   &&   &&  &&   &&    && &\cr 
&&   &&   &&   &&     &&   && && && && &\cr
\trl }}}
\medskip


\item{1.} Compute $\gcd(1332,1406)$ using the Extended Euclidean Algorithm and deduce a Bezout Identity.\vv

\item{2.} After having stated and proved a formula to compute the $p-$adic valuation of $n!$, apply it to compute the $5$--adic valuation $v_5(121!)$.\ve\vs

\item{3.} 
\itemitem{a} Show that a positive integer has an odd number of positive divisors if and only if it is a perfect square
\itemitem{b} Show that 
$2\varphi(n)=\cases{\varphi(2n)& if $n$ is even \cr \varphi(4n) & if $n$ is odd}$
\itemitem{c} Determine all integers $n$ such that $\sigma(n)=7$.\vv

\item{4.} After having stated Gauss Theorem of existence of primitive roots modulo integers, 
compute all primitive roots modulo $4394$.\ve\vs

\item{5.} Find all solutions of $x^3-x+1\equiv 0 (\bmod 140)$ by using the Chinese remainder Theorem.\vv

\item{6.} After having stated the important properties of the Legendre--Jacobi Symbols, compute $\left({1332 \over 1407}\right)_{\rm J}$.\ve \vs

\item{7.} Prove that if $n$ is an odd positive integer, then 
$$\left({13 \over n}\right)_{\rm J}=\cases{1 &if $n\equiv\pm1,\pm3,\pm4\bmod 13$\cr 0 &if $13\mid n$\cr -1 &if $n\equiv\pm2,\pm5,\pm6\bmod 13.$}$$\vv

\item{8.} Let $p$ be a prime, let $a\in\{0,\cdots,p-1\}$ and let $N_3(p,a)$ be the number of integers $x\in\{0,\ldots, p-1\}$ such that 
$x^3\equiv a\bmod p$. Show that 
\itemitem{a)} If $p\equiv2\bmod 3$, then $N_3(p,a)=1$
\itemitem{b)} If $p\equiv1\bmod 3$, then $N_3(p,a)\in\{0,3\}$
\itemitem{c)} $N_3(3,a)=1$
 
\ \vst\bye
p