\nopagenumbers \font\title=cmti12
\def\ve{\vfill\eject}
\def\vv{\vfill}
\def\vs{\vskip-2cm}
\def\vss{\vskip10cm}
\def\vst{\vskip13.3cm}

%\def\ve{\bigskip\bigskip}
%\def\vv{\bigskip\bigskip}
%\def\vs{}
%\def\vss{}
%\def\vst{\bigskip\bigskip}

\hsize=19.5cm
\vsize=27.58cm
\hoffset=-1.6cm
\voffset=0.5cm
\parskip=-.1cm
\ \vs \hskip -6mm AL310 AA11/12\ (Teoria delle Equazioni)\hfill APPELLO B (Scritto) \hfill Roma, 2 Febbraio 2012. \hrule
\bigskip\noindent
{\title COGNOME}\  \dotfill\ {\title NOME}\ \dotfill {\title
MATRICOLA}\ \dotfill\
\smallskip  \noindent
Risolvere il massimo numero di esercizi accompagnando le risposte
con spiegazioni chiare ed essenziali. \it Inserire le risposte
negli spazi predisposti. NON SI ACCETTANO RISPOSTE SCRITTE SU
ALTRI FOGLI. Scrivere il proprio nome anche nell'ultima pagina.
\rm 1 Esercizio = 5 punti. Tempo previsto: 2 ore. Nessuna domanda
durante la prima ora e durante gli ultimi 20 minuti.
\smallskip
\hrule\smallskip
\centerline{\hskip 6pt\vbox{\tabskip=0pt \offinterlineskip
\def \trl{\noalign{\hrule}}
\halign to277pt{\strut#& \vrule#\tabskip=0.7em plus 1em& \hfil#&
\vrule#& \hfill#\hfil& \vrule#& \hfil#& \vrule#& \hfill#\hfil&
\vrule#& \hfil#& \vrule#& \hfill#\hfil& \vrule#& \hfil#& \vrule#&
\hfill#\hfil& \vrule#& \hfil#& \vrule#& \hfill#\hfil& \vrule#&
\hfil#& \vrule#& \hfill#\hfil& \vrule#& \hfil#& \vrule#& \hfil#&
\vrule#\tabskip=0pt\cr\trl && FIRMA && 1 && 2 && 3 && 4 &&
5 && 6 && 7 && 8 &&   TOT. &\cr\trl && &&   &&
&&     &&   &&   &&   &&   &&    && &\cr &&
\dotfill &&     &&   &&   &&     &&   && && && &&
&\cr\trl }}}
\medskip

\item{1.} Rispondere alle sequenti domande fornendo una giustificazione di una riga (giustificazioni
incomplete o poco chiare comportano punteggio nullo):\bigskip\bigskip\bigskip


\itemitem{a.} \`E vero che ogni gruppo abeliano finito \`e il gruppo 
di Galois di qualche polinomio irriducibile in ${\bf Q}[X]$?\medskip\bigskip\bigskip

\ \dotfill\ \bigskip\bigskip\bigskip\vfil

\itemitem{b.} Scrivere una ${\bf Q}[i]$--base del campo di spezzamento del polinomio $X^4-2\in{\bf Q}[i][X]$.\medskip\bigskip\bigskip

\ \dotfill\ \bigskip\bigskip\bigskip\vfil

\itemitem{c.} \`E vero che se $K$ ha caratteristica $p$ allora il polinomio $X^p-X+1$
 non ha radici in $K$?\medskip\bigskip\bigskip
 
\ \dotfill\ \bigskip\bigskip\bigskip\vfil

\itemitem{d.} \`E vero che se $E/F$ \`e un estensione finita, allora il numero
di $F$--endomorfismi \`e sempre al pi\`u $[E:F]$?\medskip\bigskip\bigskip

\ \dotfill\ \bigskip\bigskip\bigskip

\itemitem{e.} Sia $E/{\bf Q}$ un estensione di Galois tale che $\#{\rm Gal}(E/{\bf Q})=2^\alpha$ ($\alpha\in{\bf N}$).
\`E vero che $E\cap{\bf R}$ contiene solo numeri costruibili?\medskip\bigskip\bigskip

\ \dotfill\ \bigskip\bigskip\bigskip

\vfil\eject

%Dimostrare che un estensione finita \`{e} necessariamente algebrica. Produrre
%un esempio di un estensione algebrica non finita.

\item{2.} Dato un gruppo finito $G$, dimostrare che esiste una estensione di campi $E/F$ opportuna tale che
Gal$(E(F)\cong G$.\hfill\break 
{\it Suggerimento:} Usare il Teorema di Cayley, il fatto che l'enunciato \`e vero
per $G=S_n$ e il Teorema di Corrispondenza.

\vv


\item{3.} Dimostrare che se $p>5$ \`e primo tale che $(p-1)/2$ \`e primo, allora 
${\bf Q}[\zeta_{p^\alpha}]$ ammette esattamente $4\alpha$ sottocampi. Dedurne la struttura del
reticolo dei sottocampi di ${\bf Q}[\zeta_{49}]$.
\ve\ \vs

%Dopo aver verificato che \`e algebrico, calcolare
%il polinomio minimo di $\cos \pi/9$ su ${\bf Q}$.

\item{4.} Calcolare il gruppo di Galois del polinomio $(X^3-5)(X^3-2)(X^3-7)\in{\bf Q}[X]$. \vv

\item{5.} Dimostrare che se $p\geq3$ \`e primo, allora il discriminante di $X^p-2$ \`e $(-1)^{(p-1)/2}2^{p-1}p^p$.
\hfill\break {\it Suggerimento:} Usare la formula per il discriminante che ha a che fare con la derivata prima.
\ve\ \vs

%--\item{6.} Descrivere la nozione di campo perfetto dimostrando che i campi finiti
%sono perfetti.

\item{6.} Si enunci nella completa generalit\`a il Teorema di
corrispondenza di Galois.\vskip 6cm\bigskip\bigskip\bigskip\vv\vv


\item{7.} Scrivere tutti i fattori irriducibili del polinomio $(X^{2^3}-X)(X^{2^2}-X)\in{\bf F}_2[X]$.
\vskip 6cm\bigskip\bigskip\bigskip\vv\vv

\item{8.} Dato $f\in{\bf Q}[X]$ irriducibile di grado $n$, dimostrare che se $\pmatrix{a&b\cr c&d}\in {\bf GL}_2({\bf Q})$,
allora $(cX+d)^nf((aX+b)/(cX+d))$ ha lo stesso campo di spezzamento di $f(X)$.

\vv

\ \vst
\bye
