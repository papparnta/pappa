\nopagenumbers \font\title=cmti12
\def\ve{\vfill\eject}
\def\vv{\vfill}
\def\vs{\vskip-2cm}
\def\vss{\vskip10cm}
\def\vst{\vskip13.3cm}

%\def\ve{\bigskip\bigskip}
%\def\vv{\bigskip\bigskip}
%\def\vs{}
%\def\vss{}
%\def\vst{\bigskip\bigskip}

\hsize=19.5cm
\vsize=27.58cm
\hoffset=-1.6cm
\voffset=0.5cm
\parskip=-.1cm
\ \vs \hskip -6mm AL310 AA11/12\ (Teoria delle Equazioni)\hfill ESAME
DI FINE SEMESTRE \hfill Roma, 21 Dicembre 2011. \hrule
\bigskip\noindent
{\title COGNOME}\  \dotfill\ {\title NOME}\ \dotfill {\title
MATRICOLA}\ \dotfill\
\smallskip  \noindent
Risolvere il massimo numero di esercizi accompagnando le risposte
con spiegazioni chiare ed essenziali. \it Inserire le risposte
negli spazi predisposti. NON SI ACCETTANO RISPOSTE SCRITTE SU
ALTRI FOGLI. Scrivere il proprio nome anche nell'ultima pagina.
\rm 1 Esercizio = 5 punti. Tempo previsto: 2 ore. Nessuna domanda
durante la prima ora e durante gli ultimi 20 minuti.
\smallskip
\hrule\smallskip
\centerline{\hskip 6pt\vbox{\tabskip=0pt \offinterlineskip
\def \trl{\noalign{\hrule}}
\halign to277pt{\strut#& \vrule#\tabskip=0.7em plus 1em& \hfil#&
\vrule#& \hfill#\hfil& \vrule#& \hfil#& \vrule#& \hfill#\hfil&
\vrule#& \hfil#& \vrule#& \hfill#\hfil& \vrule#& \hfil#& \vrule#&
\hfill#\hfil& \vrule#& \hfil#& \vrule#& \hfill#\hfil& \vrule#&
\hfil#& \vrule#& \hfill#\hfil& \vrule#& \hfil#& \vrule#& \hfil#&
\vrule#\tabskip=0pt\cr\trl && FIRMA && 1 && 2 && 3 && 4 &&
5 && 6 && 7 && 8 &&  TOT. &\cr\trl && &&   &&
&&     &&   &&   &&   &&   &&    && &\cr &&
\dotfill    &&   &&   &&   &&     &&   && && && &&
&\cr\trl }}}
\medskip

\item{1.} Rispondere alle seguenti domande fornendo una giustificazione di una riga:\bigskip\bigskip\bigskip


\itemitem{a.} \`E vero che esistono infiniti $n$ tali che l'$n$--agono regolare 
\`e costruibile con riga e compasso?\medskip\bigskip\bigskip

\ \dotfill\ \bigskip\bigskip\bigskip\vfil

\itemitem{b.} E' vero che il gruppo di Galois di un qualsiasi polinomio in ${\bf F}_p[X]$ \`e
abeliano?\medskip\bigskip\bigskip

\ \dotfill\ \bigskip\bigskip\bigskip\vfil

\itemitem{c.} \`E vero che tutte le estensioni di ${\bf F}_p(X^p,Y^p)$ sono non semplici?\medskip\bigskip\bigskip
 
\ \dotfill\ \bigskip\bigskip\bigskip\vfil

\itemitem{d.} \`E vero che alcuni polinomi irriducibili di grado $5$ sono risolubili per radicali?\medskip\bigskip\bigskip

\ \dotfill\ \bigskip\bigskip\bigskip


\vfil\eject

%Dimostrare che un estensione finita \`{e} necessariamente algebrica. Produrre
%un esempio di un estensione algebrica non finita.

\item{2.} Dopo averne calcolato il gruppo di Galois, determinare il reticolo dei sottocampi del campo di spezzamento 
del polinomio $(X^2+3)(X^3-2)\in{\bf Q}[x]$.\vv


\item{3.} Determinare il campo di spezzamento di $X^8 + 2X^4 + 2$ su ${\bf Q}$ e dimostrare
che \`e contenuto in un'estensione risolubile.
\ve\ \vs

%Dopo aver verificato che \`e algebrico, calcolare
%il polinomio minimo di $\cos \pi/9$ su ${\bf Q}$.

\item{4.} Calcolare le radici di $X^3+X+1$ nel campo $({\bf F}_2[\alpha], \alpha^3=1+\alpha^2$). \vv

\item{5.} Dimostrare che $\Psi_{p^2}(x)$ il $p^2$--esimo polinomio ciclotomico ($p\ge3$ primo)) \`e $(x^{p^2}-1)/(x^p-1)$ e usare questa identit\`a
per verificare che il suo discriminante \`e pari a $\pm p^{p(2p-3)}.$ \hfil\break{\it Suggerimento: mostrare che se $\zeta_{p^k}=e^{2\pi i/p^k}$, 
allora $\Psi_{p^2}'(\zeta_{p^2})=p^2/(\zeta_{p^2}(\zeta_p-1))$. Quindi
usare una formula nota.}
\ve\ \vs

%--\item{6.} Descrivere la nozione di campo perfetto dimostrando che i campi finiti
%sono perfetti.

\item{6.} Si enunci nella completa generalit\`a il Teorema di
corrispondenza di Galois.\vskip7cm\vv\vv


\item{7.} Sia $E={\bf Q}[\sqrt{3},\sqrt{5}]$. Determinare un elemento primitivo $\gamma\in E$ su ${\bf Q}$ e scriverne il
polinomio minimo su ${\bf Q}$. Descrivere tutti 
i sottocampi di $E$.\vskip7cm\vv\vv

\item{8.} Determinare un numero algebrico il cui polinomio minimo sui razionali ha un gruppo di 
Galois isomorfo a $C_3\times C_{9} \times C_{27}$.

\vv



\ \vst
 \bye
