\nopagenumbers \font\title=cmti12
% \def\ve{\vfill\eject}
% \def\vv{\vfill}
% \def\vs{\vskip-2cm}
% \def\vss{\vskip10cm}
\def\vst{\vskip13.3cm}

\def\ve{\bigskip\bigskip}
\def\vv{\bigskip\bigskip}
\def\vs{}
\def\vss{}
\def\vst{\bigskip\bigskip}

\hsize=19.5cm
\vsize=27.58cm
\hoffset=-1.6cm
\voffset=0.5cm
\parskip=-.1cm
\ \vs \hskip -6mm AL310 AA11/12\ (Teoria delle Equazioni)\hfill ESAME
DI MET\`{A} SEMESTRE \hfill Roma, 31 Ottobre 2011. \hrule
\bigskip\noindent
{\title COGNOME}\  \dotfill\ {\title NOME}\ \dotfill {\title
MATRICOLA}\ \dotfill\
\smallskip  \noindent
Risolvere il massimo numero di esercizi accompagnando le risposte
con spiegazioni chiare ed essenziali. \it Inserire le risposte
negli spazi predisposti. NON SI ACCETTANO RISPOSTE SCRITTE SU
ALTRI FOGLI.
\rm 1 Esercizio = 4 punti. Tempo previsto: 2 ore. Nessuna domanda
durante la prima ora e durante gli ultimi 20 minuti.
\smallskip
\hrule\smallskip
\centerline{\hskip 6pt\vbox{\tabskip=0pt \offinterlineskip
\def \trl{\noalign{\hrule}}
\halign to277pt{\strut#& \vrule#\tabskip=0.7em plus 1em& \hfil#&
\vrule#& \hfill#\hfil& \vrule#& \hfil#& \vrule#& \hfill#\hfil&
\vrule#& \hfil#& \vrule#& \hfill#\hfil& \vrule#& \hfil#& \vrule#&
\hfill#\hfil& \vrule#& \hfil#& \vrule#& \hfill#\hfil& \vrule#&
\hfil#& \vrule#& \hfill#\hfil& \vrule#& \hfil#& \vrule#& \hfil#&
\vrule#\tabskip=0pt\cr\trl && FIRMA && 1 && 2 && 3 && 4 &&
5 && 6 && 7 && 8  &&  TOT. &\cr\trl && &&   &&
&&     &&   &&     &&   &&   &&    && &\cr &&
\dotfill &&       &&   &&   &&     &&   && && && &&
&\cr\trl }}}
\medskip

\item{1.} Rispondere alle sequenti domande fornendo una giustificazione di una riga:\bigskip\bigskip\bigskip
\itemitem{a.} \`E vero che nei campi di caratteristica 0 i polinomi irriducibili hanno solo radici semplici?\medskip\bigskip\bigskip
\itemitem{b.} E' vero che esistono estensioni infinite e algebriche:\medskip\bigskip\bigskip
\itemitem{c.} E' vero che se $E$ \`e il campo di spezzamento di un polinomio di grado $n$, l'ordine del gruppo degli automorfismi Aut($E/F$) \`e minore di $n!$?\medskip\bigskip\bigskip
\itemitem{d.} Fornire un esempio di un polinomio in ${\bf Q}[X]$ di grado 6 il cui campo di spezzamento su ${\bf Q}$ ha grado 6.\medskip\bigskip\bigskip

\item{2.} Calcolare il polinomio minimo di $i+\sqrt5+\sqrt{3}$ sul campo
${\bf Q}[\zeta],\zeta^2+\zeta+1=0$.\vv


\item{3.} Enunciare e dimostrare il teorema della dimensione per estensioni di campi. Dedurne che se $E/F$
\`e un estensione di grado $41$, allora non esistono campi intermedi tra $E$ e $F$. 

\ve\ \vs

%Dopo aver verificato che \`e algebrico, calcolare
%il polinomio minimo di $\cos \pi/9$ su ${\bf Q}$.

\item{4.} Si consideri $E={\bf Q}[\alpha]$ dove $\alpha$ \`{e}
una radice del polinomio $X^3-X+1$. Determinare il polinomio minimo
su ${\bf Q}$ di $1/(2\alpha-1)$. \vv

\item{5.} Descrivere il gruppo Aut(${\bf Q}(\zeta_{24})/{\bf Q}$) indicandone
il numero di elementi e possibilimente la struttura. \hfill\break ({\it Suggerimento: provare
a calcolare $\zeta_{24}$ o alcune delle sue potenze})
\ve\ \vs

%--\item{6.} Descrivere la nozione di campo perfetto dimostrando che i campi finiti
%sono perfetti.

\item{6.} Si enunci nella completa generalit\`a il Teorema di
corrispondenza di Galois.\vv\vv


\item{7.}   Dopo aver verificato che \`e algebrico, calcolare
il polinomio minimo di $\cos 2\pi/9$ su ${\bf Q}$.
\vv\vv

\item{8.} Sia $\zeta_{16}$ una radice primitiva 16--esima
dell'unit\`a. Descrivere gli ${\bf Q}(\sqrt{-1})$--omomorfismi di
${\bf Q}(\zeta_{16})$ in ${\bf C}$.

\vv



\ \vst
 \bye
