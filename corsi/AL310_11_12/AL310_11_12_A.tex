\nopagenumbers \font\title=cmti12
\def\ve{\vfill\eject}
\def\vv{\vfill}
\def\vs{\vskip-2cm}
\def\vss{\vskip10cm}
\def\vst{\vskip13.3cm}

%\def\ve{\bigskip\bigskip}
%\def\vv{\bigskip\bigskip}
%\def\vs{}
%\def\vss{}
%\def\vst{\bigskip\bigskip}

\hsize=19.5cm
\vsize=27.58cm
\hoffset=-1.6cm
\voffset=0.5cm
\parskip=-.1cm
\ \vs \hskip -6mm AL310 AA11/12\ (Teoria delle Equazioni)\hfill APPELLO A (Scritto) \hfill Roma, 10 Gennaio 2012. \hrule
\bigskip\noindent
{\title COGNOME}\  \dotfill\ {\title NOME}\ \dotfill {\title
MATRICOLA}\ \dotfill\
\smallskip  \noindent
Risolvere il massimo numero di esercizi accompagnando le risposte
con spiegazioni chiare ed essenziali. \it Inserire le risposte
negli spazi predisposti. NON SI ACCETTANO RISPOSTE SCRITTE SU
ALTRI FOGLI. Scrivere il proprio nome anche nell'ultima pagina.
\rm 1 Esercizio = 5 punti. Tempo previsto: 2 ore. Nessuna domanda
durante la prima ora e durante gli ultimi 20 minuti.
\smallskip
\hrule\smallskip
\centerline{\hskip 6pt\vbox{\tabskip=0pt \offinterlineskip
\def \trl{\noalign{\hrule}}
\halign to277pt{\strut#& \vrule#\tabskip=0.7em plus 1em& \hfil#&
\vrule#& \hfill#\hfil& \vrule#& \hfil#& \vrule#& \hfill#\hfil&
\vrule#& \hfil#& \vrule#& \hfill#\hfil& \vrule#& \hfil#& \vrule#&
\hfill#\hfil& \vrule#& \hfil#& \vrule#& \hfill#\hfil& \vrule#&
\hfil#& \vrule#& \hfill#\hfil& \vrule#& \hfil#& \vrule#& \hfil#&
\vrule#\tabskip=0pt\cr\trl && FIRMA && 1 && 2 && 3 && 4 &&
5 && 6 && 7 && 8 &&   TOT. &\cr\trl && &&   &&
&&     &&   &&   &&   &&   &&    && &\cr &&
\dotfill &&     &&   &&   &&     &&   && && && &&
&\cr\trl }}}
\medskip

\item{1.} Rispondere alle sequenti domande fornendo una giustificazione di una riga (giustificazioni
incomplete o poco chiare comportano punteggio nullo):\bigskip\bigskip\bigskip


\itemitem{a.} \`E vero che esistono dei valori di $a\in{\bf C}$ tali che $[{\bf Q}[\sqrt{ai}]:{\bf Q}]=4$?\medskip\bigskip\bigskip

\ \dotfill\ \bigskip\bigskip\bigskip\vfil

\itemitem{b.} Scrivere una ${\bf Q}$--base del campo di spezzamento del polinomio $X^3-3\in{\bf Q}[X]$.\medskip\bigskip\bigskip

\ \dotfill\ \bigskip\bigskip\bigskip\vfil

\itemitem{c.} \`E vero che se $K$ \`e il campo di spezzamento di $X^4+X^2+1\in{\bf F}_2[X]$, allora
$[K:{\bf F}_2]=4$?\medskip\bigskip\bigskip
 
\ \dotfill\ \bigskip\bigskip\bigskip\vfil

\itemitem{d.} \`E vero che esistono campi finiti algebricamente chiusi?\medskip\bigskip\bigskip

\ \dotfill\ \bigskip\bigskip\bigskip

\itemitem{e.} \`E vero che il campo di spezzamento di un qualsiasi polinomio a coefficienti in un campo
di caratteristica zero \`e un estesione di Galois del campo dei coefficienti?\medskip\bigskip\bigskip

\ \dotfill\ \bigskip\bigskip\bigskip

\vfil\eject

%Dimostrare che un estensione finita \`{e} necessariamente algebrica. Produrre
%un esempio di un estensione algebrica non finita.

\item{2.} Sia ${\bf F}_p$ un campo finito con $p$ elementi. Dimostrare che 
$$\bigcup_{n=1}^\infty{\bf F}_{p^{n!}}$$
\`e un campo algebricamente chiuso.

\vv


\item{3.} Determinare tutti i sottocampi del campo di spezzamento di $X^7-1$ e dimostrare che se 
$E$ \`e un tale sottocampi allora $E$ \`e il campo di spezzamento di un opportuno polinomio in ${\bf Q}[X]$.
\ve\ \vs

%Dopo aver verificato che \`e algebrico, calcolare
%il polinomio minimo di $\cos \pi/9$ su ${\bf Q}$.

\item{4.} Calcolare il gruppo di Galois del polinomio $X^6-8\in{\bf Q}[X]$. \vv

\item{5.} Dopo aver definito la nozione di risolubilit\`a per radicali, dimostrare che $X^5-14X+7$
non \`e risolubile per radicali.
\ve\ \vs

%--\item{6.} Descrivere la nozione di campo perfetto dimostrando che i campi finiti
%sono perfetti.

\item{6.} Si enunci nella completa generalit\`a il Teorema di
corrispondenza di Galois.\vskip 6cm\bigskip\bigskip\bigskip\vv\vv


\item{7.} Costruire un campo finito con $9$ elementi e determinare l'ordine di ciascuno dei suoi elementi non nulli.\vskip 6cm\bigskip\bigskip\bigskip\vv\vv

\item{8.} Considerare l'estensione algebrica semplice ${\bf Q}[\gamma], \gamma^3=\gamma-1$. 
Determinare il polinomio minimo di $\gamma+{1\over\gamma}$ su ${\bf Q}$.


\vv

\ \vst
 \bye
