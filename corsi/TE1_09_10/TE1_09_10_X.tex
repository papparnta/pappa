\nopagenumbers \font\title=cmti12
\def\ve{\vfill\eject}
\def\vv{\vfill}
\def\vs{\vskip-2cm}
\def\vss{\vskip10cm}
\def\vst{\vskip13.3cm}

%\def\ve{\bigskip\bigskip}
%\def\vv{\bigskip\bigskip}
%\def\vs{}
%\def\vss{}
%\def\vst{\bigskip\bigskip}

\hsize=19.5cm
\vsize=27.58cm
\hoffset=-1.6cm
\voffset=0.5cm
\parskip=-.1cm
\ \vs \hskip -6mm TE1 AA09/10\ (Teoria delle Equazioni)\hfill APPELLO X (Scritto) \hfill Roma, 13 Settembre 2010. \hrule
\bigskip\noindent
{\title COGNOME}\  \dotfill\ {\title NOME}\ \dotfill {\title
MATRICOLA}\ \dotfill\
\smallskip  \noindent
Risolvere il massimo numero di esercizi accompagnando le risposte
con spiegazioni chiare ed essenziali. \it Inserire le risposte
negli spazi predisposti. NON SI ACCETTANO RISPOSTE SCRITTE SU
ALTRI FOGLI. Scrivere il proprio nome anche nell'ultima pagina.
\rm 1 Esercizio = 4 punti. Tempo previsto: 2 ore. Nessuna domanda
durante la prima ora e durante gli ultimi 20 minuti.
\smallskip
\hrule\smallskip
\centerline{\hskip 6pt\vbox{\tabskip=0pt \offinterlineskip
\def \trl{\noalign{\hrule}}
\halign to277pt{\strut#& \vrule#\tabskip=0.7em plus 1em& \hfil#&
\vrule#& \hfill#\hfil& \vrule#& \hfil#& \vrule#& \hfill#\hfil&
\vrule#& \hfil#& \vrule#& \hfill#\hfil& \vrule#& \hfil#& \vrule#&
\hfill#\hfil& \vrule#& \hfil#& \vrule#& \hfill#\hfil& \vrule#&
\hfil#& \vrule#& \hfill#\hfil& \vrule#& \hfil#& \vrule#& \hfil#&
\vrule#\tabskip=0pt\cr\trl && FIRMA && 1 && 2 && 3 && 4 &&
5 && 6 && 7 && 8 &&   TOT. &\cr\trl && &&   &&
&&     &&   &&   &&   &&   &&    && &\cr &&
\dotfill &&     &&   &&   &&     &&   && && && &&
&\cr\trl }}}
\medskip

\item{1.} Rispondere alle sequenti domande fornendo una giustificazione di una riga:\bigskip\bigskip\bigskip


\itemitem{a.} \`E vero che se un campo contiene un elemento trascendente allora ne contiene infiniti?\medskip\bigskip\bigskip

\ \dotfill\ \bigskip\bigskip\bigskip\vfil

\itemitem{b.} \`E vero che ogni campo finito con $q=3^k$ elementi \`e isomorfo
al quoziente di un opportuno anello di polinomi?\medskip\bigskip\bigskip

\ \dotfill\ \bigskip\bigskip\bigskip\vfil

\itemitem{c.} \'E vero che ogni gruppo abeliano finito \`e il gruppo di Galois su ${\bf Q}$ di un sottocampo di 
un campo ciclotomico?
\medskip\bigskip\bigskip
 
\ \dotfill\ \bigskip\bigskip\bigskip\vfil

\itemitem{d.} Dare un esempio di un polinomio a coefficienti razionali di grado $4$ avente gruppo di Galois
con $4$ elementi e non ciclico.\medskip\bigskip\bigskip

\ \dotfill\ \bigskip\bigskip\bigskip


\vfil\eject

%Dimostrare che un estensione finita \`{e} necessariamente algebrica. Produrre
%un esempio di un estensione algebrica non finita.

\item{2.} Dimostrare che se $E/F$ \`e un estensione algebrica e ogni polinomio in $F[X]$ si decompone completamente in
$E[X]$, allora $E$ \`e algebricamente chiuso.

\vv


\item{3.} Dopo aver definito la nozione di numero reale costruibile, dimostrare che il campo di spezzamento del polinomio
$X^4-16X^2+4$ contiene solo numeri costruibili.

\ve\ \vs

%Dopo aver verificato che \`e algebrico, calcolare
%il polinomio minimo di $\cos \pi/9$ su ${\bf Q}$.

\item{4.} Determinare gruppo di Galois e campo di spezzamento del polinomio $f(X)=(x^2-3)(x^2+7)(x^{21}-1) \in{\bf Q}[X]$. \vv

\item{5.} Dimostrare che il gruppo di Galois di un polinomio irriducibile di grado 3 in ${\bf Q}[X]$ \`e ciclico se
e solo se il suo discriminante \`e un quadrato perfetto.
\ve\ \vs

%--\item{6.} Descrivere la nozione di campo perfetto dimostrando che i campi finiti
%sono perfetti.

\item{6.} Si enunci e dimostri il Lemma di Artin.\vv\vv


\item{7.} Determinare il reticolo dei sottocampi del campo di spezzamento del polinomio 
$(X^{2^{12}}-X^{2^{4}})(X^{24}+X^{16}+1)(X^{5}+X^2+X+5)\in{\bf F}_2[X].$\vv\vv

\item{8.} Calcolare il polinomio minimo di $\sqrt{1+\cos(6\pi/7)}$.

\ \vst
 \bye
