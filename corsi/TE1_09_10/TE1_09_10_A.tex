\nopagenumbers \font\title=cmti12
\def\ve{\vfill\eject}
\def\vv{\vfill}
\def\vs{\vskip-2cm}
\def\vss{\vskip10cm}
\def\vst{\vskip13.3cm}

%\def\ve{\bigskip\bigskip}
%\def\vv{\bigskip\bigskip}
%\def\vs{}
%\def\vss{}
%\def\vst{\bigskip\bigskip}

\hsize=19.5cm
\vsize=27.58cm
\hoffset=-1.6cm
\voffset=0.5cm
\parskip=-.1cm
\ \vs \hskip -6mm TE1 AA09/10\ (Teoria delle Equazioni)\hfill APPELLO A (Scritto) \hfill Roma, 18 Giugno 2010. \hrule
\bigskip\noindent
{\title COGNOME}\  \dotfill\ {\title NOME}\ \dotfill {\title
MATRICOLA}\ \dotfill\
\smallskip  \noindent
Risolvere il massimo numero di esercizi accompagnando le risposte
con spiegazioni chiare ed essenziali. \it Inserire le risposte
negli spazi predisposti. NON SI ACCETTANO RISPOSTE SCRITTE SU
ALTRI FOGLI. Scrivere il proprio nome anche nell'ultima pagina.
\rm 1 Esercizio = 4 punti. Tempo previsto: 2 ore. Nessuna domanda
durante la prima ora e durante gli ultimi 20 minuti.
\smallskip
\hrule\smallskip
\centerline{\hskip 6pt\vbox{\tabskip=0pt \offinterlineskip
\def \trl{\noalign{\hrule}}
\halign to277pt{\strut#& \vrule#\tabskip=0.7em plus 1em& \hfil#&
\vrule#& \hfill#\hfil& \vrule#& \hfil#& \vrule#& \hfill#\hfil&
\vrule#& \hfil#& \vrule#& \hfill#\hfil& \vrule#& \hfil#& \vrule#&
\hfill#\hfil& \vrule#& \hfil#& \vrule#& \hfill#\hfil& \vrule#&
\hfil#& \vrule#& \hfill#\hfil& \vrule#& \hfil#& \vrule#& \hfil#&
\vrule#\tabskip=0pt\cr\trl && FIRMA && 1 && 2 && 3 && 4 &&
5 && 6 && 7 && 8 &&   TOT. &\cr\trl && &&   &&
&&     &&   &&   &&   &&   &&    && &\cr &&
\dotfill &&     &&   &&   &&     &&   && && && &&
&\cr\trl }}}
\medskip

\item{1.} Rispondere alle sequenti domande fornendo una giustificazione di una riga:\bigskip\bigskip\bigskip


\itemitem{a.} \`E vero che $\cos\pi/40$ \`e un numero costruibile?\medskip\bigskip\bigskip

\ \dotfill\ \bigskip\bigskip\bigskip\vfil

\itemitem{b.} E' vero che tutti i polinomi di grado 4 a coefficienti reali sono risolubili per radicali?\medskip\bigskip\bigskip

\ \dotfill\ \bigskip\bigskip\bigskip\vfil

\itemitem{c.} \`E vero che i campi finiti sono sempre estensioni normali dei loro sottocampi fondamentali?\medskip\bigskip\bigskip
 
\ \dotfill\ \bigskip\bigskip\bigskip\vfil

\itemitem{d.} \`E vero che esistono estensioni di grado infinito ma algebriche?\medskip\bigskip\bigskip

\ \dotfill\ \bigskip\bigskip\bigskip


\vfil\eject

%Dimostrare che un estensione finita \`{e} necessariamente algebrica. Produrre
%un esempio di un estensione algebrica non finita.

\item{2.} Dopo aver dato la definizione di chiusura algebrica di un campo, dimostrare che una chiusura 
algebrica di un qualsiasi campo \`e sempre un campo infinito.

\vv


\item{3.} Sia $q$ un numero primo tale che $p=2q+1$ \`e  primo. Dimostrare che il campo ciclotomico
${\bf Q}(\zeta_p)$ ammette esclusivamente due sottocampi non banali. Determinarli.

\ve\ \vs

%Dopo aver verificato che \`e algebrico, calcolare
%il polinomio minimo di $\cos \pi/9$ su ${\bf Q}$.

\item{4.} Calcolare il gruppo di Galois del polinomio $f(X)=X^4-2X^3+2X^2+2 \in{\bf Q}[X]$. \vv

\item{5.} Dimostrare che il Gruppo di Galois di un polinomio irriducibile a coefficienti razionali \`e 
costituito da permutazioni pari delle radici se e solo se il discriminante del polinomio \`e un
quadrato perfetto.
\ve\ \vs

%--\item{6.} Descrivere la nozione di campo perfetto dimostrando che i campi finiti
%sono perfetti.

\item{6.} Si enunci nella completa generalit\`a il Teorema di
corrispondenza di Galois.\vv\vv


\item{7.} Calcolare il grado del campo di spezzamento di 
$(X^{2^{11}}-X)(X^{16}+X^8+1)\in{\bf F}_2[X].$\vv\vv

\item{8.} Determinare il polinomio minimo di $\sqrt{2+\sqrt{5}}$ su ${\bf Q}[\sqrt{5}]$.


\vv



\ \vst
 \bye
