\nopagenumbers \font\title=cmti12
\def\ve{\vfill\eject}
\def\vv{\vfill}
\def\vs{\vskip-2cm}
\def\vss{\vskip10cm}
\def\vst{\vskip13.3cm}

%\def\ve{\bigskip\bigskip}
%\def\vv{\bigskip\bigskip}
%\def\vs{}
%\def\vss{}
%\def\vst{\bigskip\bigskip}

\hsize=19.5cm
\vsize=27.58cm
\hoffset=-1.6cm
\voffset=0.5cm
\parskip=-.1cm
\ \vs \hskip -6mm TE1 AA09/10\ (Teoria delle Equazioni)\hfill ESAME
DI FINE SEMESTRE \hfill Roma, 4 Giugno 2010. \hrule
\bigskip\noindent
{\title COGNOME}\  \dotfill\ {\title NOME}\ \dotfill {\title
MATRICOLA}\ \dotfill\
\smallskip  \noindent
Risolvere il massimo numero di esercizi accompagnando le risposte
con spiegazioni chiare ed essenziali. \it Inserire le risposte
negli spazi predisposti. NON SI ACCETTANO RISPOSTE SCRITTE SU
ALTRI FOGLI. Scrivere il proprio nome anche nell'ultima pagina.
\rm 1 Esercizio = 4 punti. Tempo previsto: 2 ore. Nessuna domanda
durante la prima ora e durante gli ultimi 20 minuti.
\smallskip
\hrule\smallskip
\centerline{\hskip 6pt\vbox{\tabskip=0pt \offinterlineskip
\def \trl{\noalign{\hrule}}
\halign to300pt{\strut#& \vrule#\tabskip=0.7em plus 1em& \hfil#&
\vrule#& \hfill#\hfil& \vrule#& \hfil#& \vrule#& \hfill#\hfil&
\vrule#& \hfil#& \vrule#& \hfill#\hfil& \vrule#& \hfil#& \vrule#&
\hfill#\hfil& \vrule#& \hfil#& \vrule#& \hfill#\hfil& \vrule#&
\hfil#& \vrule#& \hfill#\hfil& \vrule#& \hfil#& \vrule#& \hfil#&
\vrule#\tabskip=0pt\cr\trl && FIRMA && 1 && 2 && 3 && 4 &&
5 && 6 && 7 && 8 && 9 &&  TOT. &\cr\trl && &&   &&
&&     &&   &&   &&   &&   &&   &&    && &\cr &&
\dotfill &&     &&   &&   &&   &&     &&   && && && &&
&\cr\trl }}}
\medskip

\item{1.} Rispondere alle sequenti domande fornendo una giustificazione di una riga:\bigskip\bigskip\bigskip


\itemitem{a.} \`E vero che il numbero $3+\sqrt{\sqrt{2}+\sqrt{7}+5^{1/4}}$ \`e costruibile?\medskip\bigskip\bigskip

\ \dotfill\ \bigskip\bigskip\bigskip\vfil

\itemitem{b.} E' vero che un qualsiasi polinomio di grado 5 con esattamente 3 radici reali ha gruppo di Galois isomorfo a
$S_5$?\medskip\bigskip\bigskip

\ \dotfill\ \bigskip\bigskip\bigskip\vfil

\itemitem{c.} \`E vero che le estensioni finite di campi finiti sono sembre abeliane?\medskip\bigskip\bigskip
 
\ \dotfill\ \bigskip\bigskip\bigskip\vfil

\itemitem{d.} \`E vero che alcuni polinomi di grado $7$ sono risolibili per radicali?\medskip\bigskip\bigskip

\ \dotfill\ \bigskip\bigskip\bigskip


\vfil\eject

%Dimostrare che un estensione finita \`{e} necessariamente algebrica. Produrre
%un esempio di un estensione algebrica non finita.

\item{2.} Dimostrare che se $G$ \`e un gruppo finito con $n$ elementi e $E/F$ \`e
un estensione di Galois tale che Gal$(E/F)\cong S_n$, allora esiste un campo intermedio
$M$, $F\subseteq M\subseteq E$ tale che Gal$(E/M)\cong G$.\vv


\item{3.} Calcolare il 24 esimo polinomio ciclotomico. 

\ve\ \vs

%Dopo aver verificato che \`e algebrico, calcolare
%il polinomio minimo di $\cos \pi/9$ su ${\bf Q}$.

\item{4.} Dimostrare che $F[\alpha,\beta]/F$ \`e un estensione algebrica e $\beta$ \`e
separabile su $F$, allora $F[\alpha,\beta]=F[\alpha+c\beta]$ per un opportuno $c\in F$. \vv

\item{5.} Dimostrare che il discriminante del polinomio $x^4+ax+b$ \`e $-3^3a^4+4^4b^3$.
\ve\ \vs

%--\item{6.} Descrivere la nozione di campo perfetto dimostrando che i campi finiti
%sono perfetti.

\item{6.} Si enunci nella completa generalit\`a il Teorema di
corrispondenza di Galois.\vv\vv


\item{7.} Determinare il gruppo di Galois di $x^4+3x^2+1\in{\rm Q}[X]$ e $x^4+3x^2+1\in{\bf F}_2[X]$. \vv\vv

\item{8.} Determinare un numero algebrico il cui polinomio minimo sui razionali ha un gruppo di 
Galois isomorfo a $C_5\times C_{15} \times C_{30}$.

\vv



\ \vst
 \bye
