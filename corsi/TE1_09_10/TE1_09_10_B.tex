\nopagenumbers \font\title=cmti12
\def\ve{\vfill\eject}
\def\vv{\vfill}
\def\vs{\vskip-2cm}
\def\vss{\vskip10cm}
\def\vst{\vskip13.3cm}

%\def\ve{\bigskip\bigskip}
%\def\vv{\bigskip\bigskip}
%\def\vs{}
%\def\vss{}
%\def\vst{\bigskip\bigskip}

\hsize=19.5cm
\vsize=27.58cm
\hoffset=-1.6cm
\voffset=0.5cm
\parskip=-.1cm
\ \vs \hskip -6mm TE1 AA09/10\ (Teoria delle Equazioni)\hfill APPELLO A (Scritto) \hfill Roma, 18 Giugno 2010. \hrule
\bigskip\noindent
{\title COGNOME}\  \dotfill\ {\title NOME}\ \dotfill {\title
MATRICOLA}\ \dotfill\
\smallskip  \noindent
Risolvere il massimo numero di esercizi accompagnando le risposte
con spiegazioni chiare ed essenziali. \it Inserire le risposte
negli spazi predisposti. NON SI ACCETTANO RISPOSTE SCRITTE SU
ALTRI FOGLI. Scrivere il proprio nome anche nell'ultima pagina.
\rm 1 Esercizio = 4 punti. Tempo previsto: 2 ore. Nessuna domanda
durante la prima ora e durante gli ultimi 20 minuti.
\smallskip
\hrule\smallskip
\centerline{\hskip 6pt\vbox{\tabskip=0pt \offinterlineskip
\def \trl{\noalign{\hrule}}
\halign to277pt{\strut#& \vrule#\tabskip=0.7em plus 1em& \hfil#&
\vrule#& \hfill#\hfil& \vrule#& \hfil#& \vrule#& \hfill#\hfil&
\vrule#& \hfil#& \vrule#& \hfill#\hfil& \vrule#& \hfil#& \vrule#&
\hfill#\hfil& \vrule#& \hfil#& \vrule#& \hfill#\hfil& \vrule#&
\hfil#& \vrule#& \hfill#\hfil& \vrule#& \hfil#& \vrule#& \hfil#&
\vrule#\tabskip=0pt\cr\trl && FIRMA && 1 && 2 && 3 && 4 &&
5 && 6 && 7 && 8 &&   TOT. &\cr\trl && &&   &&
&&     &&   &&   &&   &&   &&    && &\cr &&
\dotfill &&     &&   &&   &&     &&   && && && &&
&\cr\trl }}}
\medskip

\item{1.} Rispondere alle sequenti domande fornendo una giustificazione di una riga:\bigskip\bigskip\bigskip


\itemitem{a.} \`E vero che se l'$n$--agono \`e costruibile allora 
lo \`e anche il $16\cdot n$--agono?\medskip\bigskip\bigskip

\ \dotfill\ \bigskip\bigskip\bigskip\vfil

\itemitem{b.} E' vero che dati due campi finiti $F_1$ e $F_2$ con lo stesso sottocampo fondamentale, \`e sempre
possibile costruire un campo finito che contiene due sottocampi $L_1$ e $L_2$ con $L_1$ isomorfo a $F_1$ 
e $L_2$ isomorfo a $F_2$?\medskip\bigskip\bigskip

\ \dotfill\ \bigskip\bigskip\bigskip\vfil

\itemitem{c.} \'E vero che esistono estensioni di grado arbitrariamente elevato che sono risolubili?\medskip\bigskip\bigskip
 
\ \dotfill\ \bigskip\bigskip\bigskip\vfil

\itemitem{d.} Dare un esempio di estensione separabile e non normale e un esempio di estensione 
normale e non separabile.\medskip\bigskip\bigskip

\ \dotfill\ \bigskip\bigskip\bigskip


\vfil\eject

%Dimostrare che un estensione finita \`{e} necessariamente algebrica. Produrre
%un esempio di un estensione algebrica non finita.

\item{2.} Dimostrare che se un dominio di integrit\`a $R$ contiene un campo $F$ e $\dim_FR$ \`e finita, allora $R$ \`e un campo.

\vv


\item{3.} Enunciare e dimostrare il Teorema di esistenza di un campo di spezzamento di un polinomio a coefficienti in un campo.

\ve\ \vs

%Dopo aver verificato che \`e algebrico, calcolare
%il polinomio minimo di $\cos \pi/9$ su ${\bf Q}$.

\item{4.} Descrivere gli elementi del gruppo di Galois del polinomio $f(X)=(x+3)^5+1 \in{\bf Q}[X]$. \vv

\item{5.} Calcolare il gruppo di Galois del polinomio: $(x^4-2)(x^3-1)\in{\bf Q}[X]$.
\ve\ \vs

%--\item{6.} Descrivere la nozione di campo perfetto dimostrando che i campi finiti
%sono perfetti.

\item{6.} Si enunci nella completa generalit\`a il Teorema di
corrispondenza di Galois.\vv\vv


\item{7.} Calcolare il grado del campo di spezzamento di 
$(X^{3^{11}}-X^{3^{4}})(X^{54}+X^{27}+1)(X^{54}+1)\in{\bf F}_3[X].$\vv\vv

\item{8.} Dimostrare che il campo ${\bf Q}[\sqrt{5}+\sqrt{7}]$ \`e contenuto in un estensione
ciclotomica di ${\bf Q}$.


\vv



\ \vst
 \bye
