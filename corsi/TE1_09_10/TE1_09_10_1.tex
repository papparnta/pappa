\nopagenumbers \font\title=cmti12
\def\ve{\vfill\eject}
\def\vv{\vfill}
\def\vs{\vskip-2cm}
\def\vss{\vskip10cm}
\def\vst{\vskip13.3cm}

%\def\ve{\bigskip\bigskip}
%\def\vv{\bigskip\bigskip}
%\def\vs{}
%\def\vss{}
%\def\vst{\bigskip\bigskip}

\hsize=19.5cm
\vsize=27.58cm
\hoffset=-1.6cm
\voffset=0.5cm
\parskip=-.1cm
\ \vs \hskip -6mm TE1 AA09/10\ (Teoria delle Equazioni)\hfill ESAME
DI MET\`{A} SEMESTRE \hfill Roma, 16 Aprile 2010. \hrule
\bigskip\noindent
{\title COGNOME}\  \dotfill\ {\title NOME}\ \dotfill {\title
MATRICOLA}\ \dotfill\
\smallskip  \noindent
Risolvere il massimo numero di esercizi accompagnando le risposte
con spiegazioni chiare ed essenziali. \it Inserire le risposte
negli spazi predisposti. NON SI ACCETTANO RISPOSTE SCRITTE SU
ALTRI FOGLI. Scrivere il proprio nome anche nell'ultima pagina.
\rm 1 Esercizio = 4 punti. Tempo previsto: 2 ore. Nessuna domanda
durante la prima ora e durante gli ultimi 20 minuti.
\smallskip
\hrule\smallskip
\centerline{\hskip 6pt\vbox{\tabskip=0pt \offinterlineskip
\def \trl{\noalign{\hrule}}
\halign to300pt{\strut#& \vrule#\tabskip=0.7em plus 1em& \hfil#&
\vrule#& \hfill#\hfil& \vrule#& \hfil#& \vrule#& \hfill#\hfil&
\vrule#& \hfil#& \vrule#& \hfill#\hfil& \vrule#& \hfil#& \vrule#&
\hfill#\hfil& \vrule#& \hfil#& \vrule#& \hfill#\hfil& \vrule#&
\hfil#& \vrule#& \hfill#\hfil& \vrule#& \hfil#& \vrule#& \hfil#&
\vrule#\tabskip=0pt\cr\trl && FIRMA && 1 && 2 && 3 && 4 &&
5 && 6 && 7 && 8 && 9 &&  TOT. &\cr\trl && &&   &&
&&     &&   &&   &&   &&   &&   &&    && &\cr &&
\dotfill &&     &&   &&   &&   &&     &&   && && && &&
&\cr\trl }}}
\medskip

\item{1.} Rispondere alle sequenti domande fornendo una giustificazione di una riga:\bigskip\bigskip\bigskip


\itemitem{a.} E' vero che ogni estensione di un campo in caratteristica $0$ \`e normale?\medskip\bigskip\bigskip

\ \dotfill\ \bigskip\bigskip\bigskip\vfil

\itemitem{b.} Fornire un esempio di estensione algebrica dei razionali di grado infinito:\medskip\bigskip\bigskip

\ \dotfill\ \bigskip\bigskip\bigskip\vfil

\itemitem{c.} E' vero che l'ordine del gruppo di Galois di un polinomio di grado $n$ divide $n!$?\medskip\bigskip\bigskip
 
\ \dotfill\ \bigskip\bigskip\bigskip\vfil

\itemitem{d.} ${\bf Q}[2^{1/3}]/{\bf Q}$ \`e un estensione di Galois?\medskip\bigskip\bigskip

\ \dotfill\ \bigskip\bigskip\bigskip


\vfil\eject

%Dimostrare che un estensione finita \`{e} necessariamente algebrica. Produrre
%un esempio di un estensione algebrica non finita.

\item{2.} Descrivere gli elementi del gruppo di Galois del
polinomio $(x^2-2)(x^4-4)\in{\bf Q}[x]$ determinando anche \it alcuni \rm sottocampi
 del campo di spezzamento.\vv


\item{3.} Dimostrare che se $F[\alpha]$ \`e un estensione algebrica semplice
di un campo $F$ tale che $[F[\alpha]:F]$ \`e dispari, allora $F[\alpha]=F[\alpha^2]$. 

\ve\ \vs

%Dopo aver verificato che \`e algebrico, calcolare
%il polinomio minimo di $\cos \pi/9$ su ${\bf Q}$.

\item{4.} Si consideri $E={\bf Q}[\alpha]$ dove $\alpha$ \`{e}
una radice del polinomio $X^2-X+1$. Determinare il polinomio minimo
su ${\bf Q}$ di $1/(\alpha+2)$. \vv

\item{5.} Descrivere il reticolo dei sottocampi di ${\bf Q}(\zeta_{24})$.
\ve\ \vs

%--\item{6.} Descrivere la nozione di campo perfetto dimostrando che i campi finiti
%sono perfetti.

\item{6.} Si enunci nella completa generalit\`a il Teorema di
corrispondenza di Galois.\vv\vv


\item{7.} Sia $n\in{\bf N}$, sia $K={\bf Q}[\zeta_n]$ e $G={\rm Gal}(K/{\bf Q})$. Per ogni sottogruppo
$H\leq G$, sia $\eta_H=\sum_{\sigma\in H}\sigma(\zeta_n)$. Dimostrare che ${\bf Q}(\eta_H)\subseteq K^H$. \vv\vv

\item{8.} Dimostrare che ${\bf Q}[\zeta_{13}]$ ha 6 sottocampi (compresi ${\bf Q}$ e ${\bf Q}[\zeta_{13}]$).
Per ciascuno si calcoli il grado su ${\bf Q}$ e si determinino eventuali inclusioni fra di essi.

\vv



\ \vst
 \bye
