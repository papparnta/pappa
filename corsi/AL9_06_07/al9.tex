\input programma.sty
\def\abbrcorso{AL9}
\def\titolocorso{Introduzione alla Teoria dei Gruppi}
\def\sottotitolo{http://www.mat.uniroma3.it/users/pappa/CORSI/AL9$_-$06$_-$07/AL9.htm}
\def\docente{Prof. Francesco Pappalardi}
\def\crediti{6}
\def\semestre{II}
\def\esoneri{0}
\def\scrittofinale{0}
\def\oralefinale{1}
\def\altreprove{1}
\Intestazione \titoloparagr{Introduzione.}  Generalit\`a e richiami sui gruppi. Teorema di classificazione dei gruppi abeliani finiti. Gruppi ciclici, Diedrali, quaternioni. Elenco dei gruppi di ordine minore di 16. Enumerazione dei gruppi. Teorema di Cayley. Matrici permutazione. Laterali. Teorema di Lagrange. Sottogruppi normali, propriet\`a. Gruppi semplici, classificazione dei gruppi semplici. Teorema di omomorfismo per gruppi, Teorema di isomorfismo per gruppi, Teorema di Corrispondenza. Automorfismi, Centro, automorfismi interni. Gruppi completi. $S_n$ \`e completo se $n\neq2,6$. Il centro di $S_n$ \`e banale se $n>2$. Gruppi di automorfismi, sottogruppi caratteristici.

\titoloparagr{Gruppi liberi e presentazioni.}
Propriet\`a universale della proiezione canonica sui quozienti. Semigruppi liberi, parole, Gruppi Liberi. Propriet\`a univerali dei gruppi liberi. Sottogruppi normali generati da sottoinsiemi, sottoinsiemi normali. Presentazioni di gruppi, esempi, gruppi finitamente presentati.
Propriet\`a universale delle presentazioni, presentazioni dei gruppi ciclici, abeliani, diedrali e quaternioni generalizzati. Ogni gruppo finito \`e finitamente presentabile. Il problema delle parole, Il problema di Burnside. Computer e Presentazioni.

\titoloparagr{Prodotti.} Prodotti diretti, prodotti diretti di sottogruppi.  Prodotti semi-diretti Propriet\`a dei prodotti semi-diretti. Caratterizzazione dei prodotti semi-diretti . Esempi: $C_3\times\!\!\!\!|\ C_4$.
Classificazione dei gruppi con $2p$ elementi. Classificazione dei gruppi con $p^3$ elementi. Isomorfismi di prodotti semidiretti.

\titoloparagr{Ampliamenti di Gruppi.} Successioni esatte corte di gruppi.
Ampliamenti di gruppi, esempi. Serie di composizione per gruppi finiti. Teorama di Jordan Hoelder sull'unicit\`a dei fattori di composizione. Programma di Hoelder per la classificazione dei gruppi finiti.


\titoloparagr{Azioni di gruppi sugli insiemi.} Esempi fondamentali, Orbite, Azioni transitive, $k$--volte transitive, stabilizzatori. Sotto insiemi stabili, propriet\`a degli stabilizzatori, Propriet\`a delle azioni transitive, azioni fedeli e libere, L'equazione delle classi. Il Teorema di Cauchy. Propriet\`a dei $p$-gruppi, I $p$-gruppi hanno centro non banale e un sottogruppo normale per ogni divisore dell'ordine. Gruppi con $p^2$ elementi. Azioni sui laterali di un sottogruppo. Criteri di non semplicit\`a. Gruppi con $99$ elementi. Azioni primitive. Blocchi. Caratterizzazione delle azioni primitive in termini della massimalit\`a degli stabilizzatori.

\titoloparagr{Gruppi di permutazioni.} Generalit\`a sulle permutazioni. Segno di una permutazione. Decomposizione in cicli, classi di coniugazioni, generatori, partizioni, esempi. Semplicit\`a di $A_n$, $n>4$. I sottogruppi normali di $S_n, n>4$.

\titoloparagr{Teoremi di Sylow.} Primo Teorema di Sylow. Classificazione dei gruppi con 99 e 30 elementi. Gruppi con $pq$ elementi. Secondo Teorema di Sylow. $A_5$ \`e l'unico gruppo semplice con 60 elementi.

\titoloparagr{Condizioni sulle catene.} Catene di composizione. Gruppi risolubili, Catene derivate. Indice di risolubilit\`a. Enunciato del Teorema di Feit Thompson. Sottogruppi e quozienti di gruppi risolubili sono risolubili. Estensioni di gruppi risolubili sono risolubili. Gruppi nilpotenti. Propriet\`a e esempi di gruppi nilpotenti. Sottogruppi e quozienti di gruppi nilpotenti sono nilpotenti. grado di nilpotenza. Serie centrali ascendenti. Caratterizzazione dei gruppi nilpotenti in termini del prodotto diretto dei Sylow.

\testi
\bib
\autore{J. S. Milne} \titolo{Group Theory}
\editore{Course Notes} \annopub{2003}
\endbib

\bib
\autore{T. Mach\`\i\ } \titolo{Dispense del Corso di Teoria dei Gruppi}.
\endbib

\altritesti

\bib
\autore{M. Artin} \titolo{Algebra} \editore{Prentice Hall, Inc.,
Englewood Cliffs, NJ} \annopub{1991}
\endbib

\bib
\autore{D. Dummit and R. Foote} \titolo{Abstract algebra}
\editore{Prentice Hall, Inc., Englewood Cliffs, NJ} \annopub{1991}
\endbib


\bib
\autore{T. W. Hungerford} \titolo{Algebra} \editore{Reprint of the
1974 original. Graduate Texts in Mathematics, 73. Springer-Verlag,
New York-Berlin} \annopub{1980}
\endbib

\bib
\autore{N. Jacobson} \titolo{Lectures in abstract algebra. III.
Theory of fields and Galois theory} \editore{Second corrected
printing. Graduate Texts in Mathematics, No. 32. Springer-Verlag,
New York-Heidelberg} \annopub{1975}
\endbib


\bib
\autore{S. Lang} \titolo{Algebra} \editore{Revised third edition.
Graduate Texts in Mathematics, 211. Springer-Verlag, New York}
\annopub{2002}
\endbib

\bib
\autore{J. Rotman} \titolo{Galois theory} \editore{Universitext.
Springer-Verlag, New York} \annopub{1998}
\endbib

\bib
\autore{I. Stewart} \titolo{Galois theory} \editore{Second
edition. Chapman and Hall, Ltd., London} \annopub{1989}
\endbib

\bib
\autore{J. Stillwell} \titolo{Elements of algebra}
\editore{Undergraduate Texts in Mathematics. Springer-Verlag, New
York} \annopub{1994}
\endbib

\esami

Gli studenti devono presentare uno o pi\`{u} seminari su argomenti avanzati del corso e
risolvere degli esercizi a casa che saranno poi discussi in presenza del docente. \bye
