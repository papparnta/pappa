\nopagenumbers
\font\title=cmti12
\vfuzz=280pt
%\def\ve{\vfill\eject}
%\def\vv{\vfill}
%\def\vs{\vskip-2cm}
%\def\vss{\vskip10cm}
%\def\vst{\vskip13.3cm}

\def\ve{\bigskip\bigskip}
\def\vv{\bigskip\bigskip}
\def\vs{}
\def\vss{}
\def\vst{\bigskip\bigskip}

\hsize=19.5cm
\vsize=27.58cm
\hoffset=-1.6cm
\voffset=0.5cm
\parskip=-.1cm
\ \vs \hskip -6mm TE1 AA02/03\ (Teoria delle Equazioni)\hfill
ESAME SCRITTO \hfill Roma, 18 Luglio 2003. \hrule
\bigskip\noindent
{\title COGNOME}\  \dotfill\  {\title NOME}\ \dotfill {\title
MATRICOLA}\ \dotfill\
\smallskip  \noindent
Risolvere il massimo numero di esercizi accompagnando le risposte
con spiegazioni chiare ed essenziali. \it Inserire le risposte
negli spazi predisposti. NON SI ACCETTANO RISPOSTE SCRITTE SU
ALTRI FOGLI. Scrivere il proprio nome anche nell'ultima pagina.
\rm 1 Esercizio = 3 punti. Tempo previsto: 2 ore. Nessuna domanda
durante la prima ora e durante gli ultimi 20
minuti.\smallskip\hrule\medskip

\item{1.} Calcolare il polinomio minimo su ${\bf Q}$, di $\zeta_{9}$. \vv %%%
\item{2.} Calcolare la dimensione $[{\bf Q}(5^{1/3},5^{1/2},5^{1/6}):{\bf Q}]$.\ve \vs %%
\item{3.} Dopo aver dimostrato che \`{e} un estensione di Galois di ${\bf Q}$, determinare tutti i sottocampi di ${\bf Q}(\zeta_{16})$.\vv%%%
\item{4.} Elencare tutti i polinomi irriducibili (monici) di grado minore di $5$ su ${\bf F}_{2}$.\ve\ \vs%%%
\item{5.} Calcolare il gruppo di Galois del polinomio $(x^2+2)(x^2+3)(x^2+5)$ elencando tutti i sottocampi del campo di spezzamento.\vv%%%
\item{6.} Dato un polinomio $f$, se ne definisca il discriminante e se ne illustrino le propriet\`{a} principali. \ve\ \vs%
\item{7.} Calcolare il gruppo di Galois del polinomio $(X+3)^4+5$.\vv %
\item{8.} Si enunci nella completa generalit\`{a} il Teorema di corrispondenza di Galois. \ve\ \vs%%
\item{9.} Dopo aver definito la nozione di campo perfetto, si mostri che un campo di caratteristica zero \`{e} perfetto.\vv %%%%%%%%
\item{10.} Enunciare e dimostrare il Teorema di esistenza e unicit\`{a} per campi finiti.\ve\ \vs %%%%%
\item{11.} Si calcoli il numero di elementi nel campo di spezzamento del polinomio $(x^3+x+1)(x^3+x^2+1)(x^{15}+x^{10}+1)(x^{25}+5x^{3}+1)$ su ${\bf F}_5$.\vss %%%%%
\item{12.} Dimostrare che non \`{e} possibile costruire con riga e compasso un cerchio di area unitaria.\vv %%%%%%%%%%%
\ \vst

\centerline{\hskip 6pt\vbox{\tabskip=0pt \offinterlineskip
\def \trl{\noalign{\hrule}}
\halign to500pt{\strut#& \vrule#\tabskip=0.7em plus 1em&
\hfil#& \vrule#& \hfill#\hfil& \vrule#&
\hfil#& \vrule#& \hfill#\hfil& \vrule#&
\hfil#& \vrule#& \hfill#\hfil& \vrule#&
\hfil#& \vrule#& \hfill#\hfil& \vrule#&
\hfil#& \vrule#& \hfill#\hfil& \vrule#&
\hfil#& \vrule#& \hfill#\hfil& \vrule#&
\hfil#& \vrule#& \hfil#& \vrule#\tabskip=0pt\cr\trl
&& NOME E COGNOME && 1 && 2 && 3 && 4 && 5 && 6 && 7 && 8 && 9 && 10 && 11 && 12 &&  TOT. &\cr\trl
&& &&   &&   &&     &&   &&   &&   &&   &&   &&    &&   &&   &&  && &\cr
&& \dotfill &&     &&   &&   &&   &&   &&   &&    &&  &&   && && && && &\cr\trl
}}}
 \bye
