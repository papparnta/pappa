\nopagenumbers
\font\title=cmti12
\vfuzz=280pt
%\def\ve{\vfill\eject}
%\def\vv{\vfill}
%\def\vs{\vskip-2cm}
%\def\vss{\vskip10cm}
%\def\vst{\vskip13.3cm}

\def\ve{\bigskip\bigskip}
\def\vv{\bigskip\bigskip}
\def\vs{}
\def\vss{}
\def\vst{\bigskip\bigskip}

\hsize=19.5cm
\vsize=27.58cm
\hoffset=-1.6cm
\voffset=0.5cm
\parskip=-.1cm
\ \vs \hskip -6mm TE1 AA02/03\ (Teoria delle Equazioni)\hfill
ESAME SCRITTO \hfill Roma, 19 Settembre 2003. \hrule
\bigskip\noindent
{\title COGNOME}\  \dotfill\  {\title NOME}\ \dotfill {\title
MATRICOLA}\ \dotfill\
\smallskip  \noindent
Risolvere il massimo numero di esercizi accompagnando le risposte
con spiegazioni chiare ed essenziali. \it Inserire le risposte
negli spazi predisposti. NON SI ACCETTANO RISPOSTE SCRITTE SU
ALTRI FOGLI. Scrivere il proprio nome anche nell'ultima pagina.
\rm 1 Esercizio = 3 punti. Tempo previsto: 2 ore. Nessuna domanda
durante la prima ora e durante gli ultimi 20 minuti.
\smallskip
\hrule
\medskip

\item{1.} Quali possono essere tutti i possibili gruppi di Galois
dei polinomi di grado tre su ${\bf Q}$ e su ${\bf F}_2$?
 \vv

\item{2.} Dopo aver definito le nozioni di estensioni algebriche e
trascendenti, dimostrare che un estensione di campi \`{e} finita
solo se \`{e} algebrica.

\ve\ \vs

\item{3.} Dopo aver dimostrato che \`{e} un estensione di Galois
di ${\bf Q}$, determinare tutti i sottocampi di ${\bf
Q}(\zeta_{13})$. \vv

\item{4.} Calcolare quanti sono i polinomi irriducibili (monici)
di grado $6$ su ${\bf F}_{7}$.\ve\ \vs

\item{5.} Calcolare il gruppo di Galois del polinomio $x^4+7$.

\vv \item{6.} Definire la nozione di discriminante di un polinomio
e spiegare come \`{e} possibile usare la derivata prima per
calcolarlo. \ve\ \vs

\item{7.} Spiegare come si fa a costruire un polinomio il cui
gruppo di Galois ciclico con $n$ elementi.

\vv \item{8.} Si enunci nella completa generalit\`{a} il Teorema
di corrispondenza di Galois.

\ve\ \vs

\item{9.} Calcolare il gruppo di Galois del polinomio $(x^3+x+1)$
sul campo finito ${\bf F}_3$.\vv

\item{10.} Fornire due esempi distinti di campi finiti ${\bf
F}_{9}$ con $9$ elementi e costruire un isomorfismo tra i due.\ve\
\vs


\item{11.} Definire la nozione di sottogruppo transitivo di $S_n$
ed elencare tutti i sottogruppi transitivi di $S_3$ e $S_4$.

\vss

\item{12.} Dopo aver enunciato la nozione di numero costruibile,
dare dei cenni della dimostrazione che l'insieme dei numeri
costruibili sono un campo.\vv

\ \vst

\centerline{\hskip 6pt\vbox{\tabskip=0pt \offinterlineskip
\def \trl{\noalign{\hrule}}
\halign to500pt{\strut#& \vrule#\tabskip=0.7em plus 1em&
\hfil#& \vrule#& \hfill#\hfil& \vrule#&
\hfil#& \vrule#& \hfill#\hfil& \vrule#&
\hfil#& \vrule#& \hfill#\hfil& \vrule#&
\hfil#& \vrule#& \hfill#\hfil& \vrule#&
\hfil#& \vrule#& \hfill#\hfil& \vrule#&
\hfil#& \vrule#& \hfill#\hfil& \vrule#&
\hfil#& \vrule#& \hfil#& \vrule#\tabskip=0pt\cr\trl
&& NOME E COGNOME && 1 && 2 && 3 && 4 && 5 && 6 && 7 && 8 && 9 && 10 && 11 && 12 &&  TOT. &\cr\trl
&& &&   &&   &&     &&   &&   &&   &&   &&   &&    &&   &&   &&  && &\cr
&& \dotfill &&     &&   &&   &&   &&   &&   &&    &&  &&   && && && && &\cr\trl
}}}
 \bye
