\nopagenumbers
\font\title=cmti12
\vfuzz=280pt
\def\ve{\vfill\eject}
\def\vv{\vfill}
\def\vs{\vskip-2cm}
\def\vss{\vskip10cm}
\def\vst{\vskip13.3cm}

%\def\ve{\bigskip\bigskip}
%\def\vv{\bigskip\bigskip}
%\def\vs{}
%\def\vss{}
%\def\vst{\bigskip\bigskip}

\hsize=19.5cm
\vsize=27.58cm
\hoffset=-1.6cm
\voffset=0.5cm
\parskip=-.1cm
\ \vs \hskip -6mm TE1 AA02/03\ (Teoria delle Equazioni)\hfill
ESAME DI FINE SEMESTRE \hfill Roma, 22 Maggio 2003. \hrule
\bigskip\noindent
{\title COGNOME}\  \dotfill\  {\title NOME}\ \dotfill {\title
MATRICOLA}\ \dotfill\
\smallskip  \noindent
Risolvere il massimo numero di esercizi accompagnando le risposte
con spiegazioni chiare ed essenziali. \it Inserire le risposte
negli spazi predisposti. NON SI ACCETTANO RISPOSTE SCRITTE SU
ALTRI FOGLI. Scrivere il proprio nome anche nell'ultima pagina.
\rm 1 Esercizio = 3 punti. Tempo previsto: 2 ore. Nessuna domanda
durante la prima ora e durante gli ultimi 20 minuti.
\smallskip
\hrule
\medskip

\item{1.} Dopo aver dato la definizione di sottogruppo transitivo di $S_n$, si elenchino
i sottogruppi transitivi di $S_4$ descrivendone gli elementi come permutazioni.

\vv \item{2.} Descrivere gli elementi del gruppo di Galois del polinomio $x^5-2$ mostrando
che ha $20$ elementi.

\ve\ \vs \item{3.} Dimostrare che il gruppo di Galois de polinomio
(che si pu\`{o} assumere irriducibile) $x^5+x^4+x^3+2x^2+3x+4$ non
ha $20$ elementi n\'{e} $10$ mostrando che contiene un $3$
ciclo.\hfill ({\it Pensare al numero primo $2$})

\vv


\item{4.} Calcolare quanti sono i polinomi irriducibili (monici) di grado $8$ su
${\bf F}_2$.\ve\ \vs

\item{5.} Calcolare il gruppo di Galois del polinomio $x^4+8x+12$
(assumendo che \`{e} irriducibile).

\vv \item{6.} Calcolare una formula per il discriminante di
$x^4+ax+b$. \ve\ \vs

\item{7.} Spiegare come si fa a costruire un polinomio il cui
gruppo di Galois ha $13$ elementi. \hfill {\it Pensare al numero
primo $53$.}

\vv \item{8.} Si enunci nella completa
generalit\`{a} il Teorema di corrispondenza di Galois.


\ve\ \vs

\item{9.} Definire il discriminante di un polinomio ${\bf F}[x]$ e
dimostrare che \`{e} un elemento di ${\bf F}$.\vv

\item{10.} Quali sono le radici di $x^{16}+x^{12}+1$ in ${\bf
F}_2[\alpha]$, con $\alpha^4=\alpha+1$? \hfill {\it Provare con
$\alpha^3+1$.}

\ve\ \vs


\item{11.} Si calcoli il numero di elementi nel campo di
spezzamento del polinomio $(x^4+x+1)(x^4+x^3+1)(x^2+x+1)(x^3+x+1)$
su ${\bf F}_2$ . \vss

\item{12.} Dopo aver enunciato il teorema di caratterizzazione per
i numeri reali costruibili, si dimostri che
$\sqrt{1+\sqrt{3-2^{1/8}}}$ esibendone una costruzione nel senso
della teoria dei campi. Dimostrare anche che $2^{1/5}$ non \`{e}
costruibile.\vv

\ \vst

\centerline{\hskip 6pt\vbox{\tabskip=0pt \offinterlineskip
\def \trl{\noalign{\hrule}}
\halign to500pt{\strut#& \vrule#\tabskip=0.7em plus 1em&
\hfil#& \vrule#& \hfill#\hfil& \vrule#&
\hfil#& \vrule#& \hfill#\hfil& \vrule#&
\hfil#& \vrule#& \hfill#\hfil& \vrule#&
\hfil#& \vrule#& \hfill#\hfil& \vrule#&
\hfil#& \vrule#& \hfill#\hfil& \vrule#&
\hfil#& \vrule#& \hfill#\hfil& \vrule#&
\hfil#& \vrule#& \hfil#& \vrule#\tabskip=0pt\cr\trl
&& NOME E COGNOME && 1 && 2 && 3 && 4 && 5 && 6 && 7 && 8 && 9 && 10 && 11 && 12 &&  TOT. &\cr\trl
&& &&   &&   &&     &&   &&   &&   &&   &&   &&    &&   &&   &&  && &\cr
&& \dotfill &&     &&   &&   &&   &&   &&   &&    &&  &&   && && && && &\cr\trl
}}}
 \bye
