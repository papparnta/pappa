\magnification 900
\hsize=19.5cm
\vsize=27.58cm
\hoffset=-1.6cm
\voffset=0.5cm
\parskip=1mm
\noindent  TE1 AA02/03\ (Teoria delle
Equazioni)\hfill Roma, 24 Aprile 2003. \hrule
\medskip
\smallskip

\centerline{SOLUZIONI DELL' ESAME DI MET\`{A} SEMESTRE}\medskip

 \hfill \hrule
\medskip\medskip\medskip

\item{1.} Data un estensione $E/F$, si dica cosa significa che
$\alpha\in E$ \`{e} algebrico su $F$ e cosa \`{e} il polinomio
minimo di $\alpha$ su $F$ dimostrando che \`{e} irriducibile.
\medskip

{\it $\alpha\in E$ si dice {\bf algebrico} su $F$ se l'omomorfismo di anelli $\varphi: F[x]\rightarrow
 E, f(x) \mapsto f(\alpha)$ non \`{e} iniettivo.\smallskip

  In tal caso, siccome $F[x]$ \`{e} un anello
 a ideali principali, l'ideale Ker$\varphi\subset F[x]$ \`{e} principale.\smallskip

 Pertanto esiste $f_\alpha\in F[x]$
monico tale che Ker$\varphi = \langle f_\alpha \rangle$. Tale $f_\alpha$ si chiama {\bf polinomio minimo}
di $\alpha$ su $F$.\smallskip

 Infine, siccome $F(\alpha)\cong F[x]/\langle f_\alpha \rangle \hookrightarrow E$
\`{e} un campo, $\langle f_\alpha \rangle$ risulta un ideale
massimale e $f_\alpha$ irriducibile.}\medskip\hrule\medskip

 \item{2.} Descrivere gli elementi del gruppo di Galois del
polinomio $(x^2-2)(x^2+3)$ determinando anche tutti i sottocampi
del campo di spezzamento.
\medskip

{\it Sia $f(x)=(x^2-2)(x^2+3)$. Allora il campo di spezzamento di
$f$ su ${\bf Q}$ \`{e} ${\bf Q}_f={\bf Q}(\sqrt{2},
\sqrt{-3})$.\smallskip

Dal fatto che $[{\bf Q}_f:{\bf Q}]=4$ deduciamo che gli elementi del gruppo di Galois Gal$({\bf Q}_f/{\bf Q})$
sono i seguenti:
$$\left\{ {\sqrt{2}\mapsto \sqrt{2}\atop \sqrt{-3}\mapsto \sqrt{-3},}
{\sqrt{2}\mapsto \sqrt{2}\atop \sqrt{-3}\mapsto -\sqrt{-3},}
{\sqrt{2}\mapsto -\sqrt{2}\atop \sqrt{-3}\mapsto \sqrt{-3},}
{\sqrt{2}\mapsto -\sqrt{2}\atop \sqrt{-3}\mapsto -\sqrt{-3},}
\right\}.$$

Ciascuno degli ultimi tre elementi del gruppo di Galois genera un sottogruppo con indice $2$ che corrisponde
a uno dei tre sottocampi quadratici di ${\bf Q}_f$
$${\bf Q}(\sqrt{-3}),\quad {\bf Q}(\sqrt{2}),\quad {\bf Q}(\sqrt{-6})$$

che oltre a ${\bf Q}$ e ${\bf Q}_f$ sono tutti e soli i sottocampi
di ${\bf Q}_f$. }\medskip\hrule\medskip

\item{3.} Dopo aver verificato che \`{e} algebrico, calcolare il
polinomio minimo di $\cos \pi/12$ su ${\bf Q}$.\medskip

{\it Dal fatto $\cos
\pi/12={1\over2}\left(\zeta_{24}+\overline{\zeta}_{24}\right)$ e
dal fatto che $\zeta_{24}$ e $\overline{\zeta}_{24}$ entrambi
soddisfano il polinomio $x^{24}-1$ deduciamo che $\cos \pi/12$
\`{e} algebrico su ${\bf Q}$ in quanto somma di numeri
algebrici.\medskip

Si ha inoltre che

$$\partial f_{\cos{\pi\over12}}=[{\bf Q}(\cos \pi/12):{\bf Q}]= {1\atop2} [{\bf Q}(\zeta_{24}):{\bf Q}]=
\varphi(24)/2=4.$$

Quindi $\cos \pi/12$ soddisfa un polinomio di grado $4$ in ${\bf Q}[x]$. Scriviamo $\zeta=\zeta_{24}$
$$\cos^4 {\pi\over12}+A\cos^3{\pi\over12}+B\cos^2{\pi\over12}+C\cos{\pi\over12}+D=
{\left((\zeta+\overline{\zeta})^4+2A(\zeta+\overline{\zeta})^3+
4B(\zeta+\overline{\zeta})^2+8C(\zeta+\overline{\zeta})+16D\right)\over16}.$$

Sfruttando le identit\`{a} $\zeta^8-\zeta^4+1=0$ e $\zeta^{12}=-1$
e facendo i calcoli otteniamo che $A=C=0$, $B=-1$ e $D=1/16$.\medskip

Quindi \quad\quad\quad$f_{\cos \pi/12}(x)=x^4-x^2+1/16.$
}\medskip\hrule\medskip

\item{4.} Quanti elementi ha il campo di spezzamento di
$(x^2+x+1)(x^3+x^2+1)$ su ${\bf F}_2$?\smallskip

{\it Ne ha $2^6$. Infatti $x^2+x+1$ e $x^3+x^2+1$ sono entrambi irriducibili
e il campo di spezzamento $E$ ha come sottocampi ${\bf F}_2(\alpha),
\alpha^2=\alpha+1$ e ${\bf F}_2(\beta), \beta^3=\beta^2$+1. Pertanto
sia $2$ che $3$ dividono $[E:{\bf F}_2]$.\smallskip

Infine ${\bf F}_2(\alpha,\beta)$ \`{e} il campo di spezzamento infatti
$(x^2+x+1)(x^3+x^2+1)=(x+\alpha)(x+\alpha^2)(x+\beta)(x+\beta^2)(x+\beta^4).$\smallskip

Quindi $6=[E:{\bf F}_2]$ e $|E|=2^6.$}\medskip\hrule\medskip

\item{5.} Dimostrare che se $p$ \`{e} primo, $\cos 2\pi/p^2$
soddisfa un polinomio di grado $p(p-1)/2$ su ${\bf Q}$ con gruppo
di Galois ciclico.\medskip

{\it \centerline{$[{\bf Q}(\cos{2\pi\over
p^2}):{\bf Q}]={[{\bf Q}(\zeta_{p^2}:{\bf Q}(\cos{2\pi\over
p^2})]\over [:{\bf Q}(\zeta_{p^2}):{\bf Q}(\cos{2\pi\over p^2})]}=
{\varphi(p^2)\over 2}={p(p-1)\over 2}.$}\smallskip

Infine $\cos 2\pi/p^2$ ha un polinomio minimo con grado
${p(p-1)\over 2}$ il cui gruppo di Galois \`{e} ciclico in quanto
sottogruppo del gruppo ciclico $U({\bf Z}/p^2{\bf
Z}).$}\medskip\hrule\medskip

 \item{6.} Mostrare che un estensione di campi \`{e}  finita se
e solo se \`{e} algebrica e finitamente generata spiegando le
nozioni di cui si parla.\medskip

{\it Un estensione $E/F$ si dice {\bf algebrica} se ogni elemento
di $E$ soddisfa un polinomio a coefficienti in $F$ non nullo; si
dice {\bf finita} se $E$ \`{e} uno spazio vettoriale di dimensione
finita e si dice finitamente generata se esistono
$\alpha_1,\ldots,\alpha_h\in E$ tali che
$E=F(\alpha_1,\ldots,\alpha_h)$.
Se $E/F$ \`{e} finita, allora per ogni $\alpha\in E$, se $\alpha$ non fosse algebrico,
la famiglia $1,\alpha,\alpha^2,\ldots$ darebbe luogo ad una famiglia infinita di vettori
linearmente indipendenti. Se inoltre consideriamo la catena di sottocampi:
$$F\subset F[\alpha_1] \subset F[\alpha_1,\alpha_2] \subset \ldots \subset E$$
dove $\alpha_i$ \`{e} scelto in modo tale che $\alpha_i\in E\setminus F[\alpha_1,\ldots,\alpha_{i-1}]$,
Allora la catena non pu\`{o} crescere all'infinito cio\`{e} $E=F[\alpha_1,\ldots,\alpha_h]$ \`{e}
finitamente generato.\medskip

Se invece $E/F$ \`{e} algebrica e finitamente generata, allora.
$$E=F[\alpha_1,\ldots,\alpha_h]=F(\alpha_1,\ldots,\alpha_h)\quad\quad
{\rm e}\quad\quad
[E:F]=\prod_{j=1}^{h-1}[F[\alpha_1,\ldots,\alpha_{j-1}]:F[\alpha_1,\ldots,\alpha_j]].$$
Ciascun fattore \`{e} finito perch\`{e} si tratta di un estensione semplice con un elemento algebrico.
Quindi $[E:F]$ \`{e} finito.
}\medskip\hrule\medskip

\item{7.} Descrivere gli elementi del gruppo di Galois del campo
di spezzamento di $x^n-1$.\medskip

{\it Il campo di spezzamento del polinomio \`{e} il campo
ciclotomico ${\bf Q}(\zeta_m)$. Inoltre se
\quad $\sigma_j: {\bf Q}(\zeta_m) \rightarrow {\bf Q}(\zeta_m), \zeta_m
\mapsto \zeta_m^j,$\smallskip

Allora Gal$({\bf Q}(\zeta_m)/{\bf Q})=\{\sigma_j\ |\ j\in U({\bf
Z}/m{\bf Z})\}$.}\medskip\hrule\medskip

\item{8.} Si enunci nella completa generalit\`{a} il Teorema di
corrispondenza di Galois.\medskip

\noindent{\bf Teorema.} {\it Sia $E/F$ un estensione di Galois
(cio\`{e} $E$ \`{e} il campo di spezzamento di un polinomio
separabile in $F[x]$) e sia $G=$Gal$(E/F)$. Allora c'\`{e} una
corrispondenza biunivoca tra ti sottogruppi di $G$ e i sottocampi
di $E$ che contengono $F$. Se $H\leq G$ e $F\subseteq M\subseteq
E$, allora la corrispondenza \`{e} data da:
$$H\mapsto E^H,\quad M\mapsto {\rm Gal}(E/M).$$
Inoltre \item{i} $G$ corrisponde a $F$ e $\{1\}$ corrisponde a
$E$; \item{ii} $H_1\leq H_2 \Leftrightarrow E^{H_1}\supseteq
E^{H_2}.$ \item{iii} Per ogni $\sigma\in G$,
    \itemitem{-} $E^{\sigma H\sigma^{-1}}=\sigma E^H$;
    \itemitem{-} $Gal(E/\sigma M)=\sigma Gal(E/M)\sigma^{-1}$.
\item{iv} $H\triangleleft G \Leftrightarrow E^H/F$ \`{e} un
estensione normale. In tal caso inoltre $Gal(E^H/F)\cong G/H.$}
\medskip\hrule
\medskip

\item{9.} Calcolare la dimensione su ${\bf Q}$ del campo di
spezzamento del seguente polinomio $x^3+x+10$.\medskip

{\it La dimensione \`{e} $2$. Infatti $x^3+x+10=(x+2)(x^2-2x+5)$ e
quindi il campo di spezzamento \`{e} ${\bf Q}(\sqrt{-1})$.}\medskip\hrule\medskip

\item{10.} Dare un esempio di polinomio non separabile, un esempio
de estensione algebrica normale e non separabile e di una
separabile e non normale.\medskip

{\it \itemitem{-} $x^p-t\in{\bf F}_p(t)[x]$ \`{e} un polinomio non
separabile;\smallskip

 \itemitem{-} ${\bf F}_p(t)/{\bf F}_p(t^p)$ \`{e} un
estensione algebrica normale non separabile;\smallskip

 \itemitem{-}
${\bf Q}(2^{1/3})$ \`{e} un estensione normale non separabile.
}\medskip\hrule\medskip

 \item{11.} Descrivere gli ${\bf Q}(\sqrt{-1})$--omomorfismi di
${\bf Q}(\sqrt{-3},\sqrt{3})$ in ${\bf C}$.\medskip

{\it Sappiamo che
$${\bf Q}(\sqrt{-3},\sqrt{3})={\bf Q}(\sqrt{3},\sqrt{-1})=
{\bf Q}(\sqrt{-1})(\sqrt{3})$$ e che gli ${\bf
Q}(\sqrt{-1})$--omomorfismi sono in numero pari al numero di
radici del polinomio minimo di $\sqrt{3}$ in ${\bf C}$ cio\`{e}
due.

Si tratta quindi dell'identit\`{a} e l'omomorfismo: $$\sigma:{\bf
Q}(\sqrt{-3},\sqrt{3})\rightarrow {\bf C}, \sqrt{-3}\mapsto
-\sqrt{-3}, \sqrt{3}\mapsto -\sqrt{3}$$ che \`{e} un ${\bf
Q}(\sqrt{-1})$--omomorfismo in quanto
$$\sigma(\sqrt{-1})=\sigma({1\over3}\sqrt{3}\sqrt{-3})=\sqrt{-1}.$$
 }\hrule\medskip

\item{12.} Dimostrare che se $E_1$ e $E_2$ sono estensioni di
Galois di $F$ con $E_1\subset L$ e $E_2\subset L$, allora $E_1\cap
E_2$ e  $E_1E_2$ (il campo composto) sono estensioni di Galois. (
Per ulteriore punteggio mostrare che ${\rm Gal}(E_1E_2/F)\cong
{\rm Gal}(E_1/F)\times {\rm Gal}(E_2/F)$ se $E_1\cap E_2 = F$).
\medskip
{\it La soluzione \`{e} nelle dispense di Milne a pagina 37 nella
Proposizione 3.20.}

\medskip
\hrule

 \bye
