\nopagenumbers
\magnification 1200

\centerline{\bf ESERCIZI DI TEORIA DI GALOIS}
\bigskip
\bigskip\bigskip\bigskip

\item{1.} Per ciascuno dei seguenti polinomi si calcoli:
\itemitem{i.} Il campo di spezzamento ${\bf Q}(f)$.
\itemitem{ii.} Il gruppo di Galois Gal$({\bf Q}(f)/{\bf Q})$.
\itemitem{iii.} Tutti i sottocampi di ${\bf Q}(f)$ e per
ciascuno di essi si dica se sono estensioni normali di ${\bf Q}$.

\itemitem{a.} $f(t)=t^4-1$.\bigskip

\itemitem{b.} $f(t)=t^6-1$.\bigskip

\itemitem{c.} $f(t)=t^4+2t^2-15$.\bigskip

\itemitem{d.} $f(t)=t^6-27$.\bigskip

\itemitem{e.} $f(t)=t^3-5$.\bigskip

\itemitem{f.} $f(t)=t^4-2$.\bigskip

\itemitem{g.} $f(t)=t^6+2t^3-15$.\bigskip

\itemitem{h.} (difficile) $f(t)=t^a-2$, $a\in{\bf N}$.\bigskip\bigskip

\item{2.} Si costruisca il campo di spezzamento ${\bf K}(f)$ dei
seguenti polinomi con ${\bf K}={\bf Z}_2,{\bf Z}_3,{\bf Z}_5$:
\bigskip

\itemitem{a.} $f(t)=t^3-t+2$.\bigskip

\itemitem{b.} $f(t)=t^4+1$.\bigskip

\itemitem{c.} $f(t)=t^3+t-2+1$.
\bye
