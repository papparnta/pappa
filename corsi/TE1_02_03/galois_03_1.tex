\magnification 1200
%\nopagenumbers
\def\frac#1#2{{#1\over#2}}
\def\Q{{\bf Q}}
\def\Z{{\bf Z}}
\def\N{{\bf N}}
\def\F{{\bf F}}
\def\QQ{{\rm Q}}


\centerline{{\bf Esercizi di Teoria di Galois 1.}}\medskip

\centerline{Roma Tre, 6 Marzo 2003}\bigskip

\item{1.} Sia $E/F$ un estensione di campi e $S\subseteq E$ un
sottoinsieme: Dimostrare che
$$\Q(F[S])=F(S).$$\bigskip

\item{2.} In ciascuno dei seguenti casi, determinare (cio\`{e} esprimere come polinomi
nell'elemento che genera il campo) ove possibile l'inverso
degli elementi assegnati:
\itemitem{a.} $\Q(\alpha)$ con $\alpha^3-5\alpha-1=0$;
$$\frac1{\alpha+1},\hskip 2cm\frac1{\alpha^2+\alpha+1}\hskip 2cm \frac1{2+\alpha};$$
\itemitem{b.} $\Q(\lambda)$ con $\lambda^3-2\lambda-2=0$;
$$\frac1{20\lambda},\hskip 2cm\frac1{\lambda+3},\hskip 2cm \frac 1{\lambda^5}$$
\itemitem{c.} $\Q(\xi)$ con $\xi^2+\xi+1=0$:
$$\frac1{a+b\xi},\ \ \ \ a,b\in\Q, ab\neq0;$$
\itemitem{d.} $\F_{13}(\zeta)$, con $\zeta^4+\zeta^3+\zeta^2+\zeta+1=0$,
$$\frac1{\zeta^t},\ \ \ \ t\in\N.$$
\bigskip


\item{3.} Determinare il polinomio minimo di $\mu$ su $F$ in ciascuno dei seguenti casi:

\itemitem{a.} $E=\Q(\sqrt{5})$, $F=\Q$,\ \hfill \ $\mu=\frac{1+\sqrt{5}}{4-3\sqrt{5}}$;\smallskip
\itemitem{b.} $E=\Q(3^{1/4})$, $F=\Q$,\ \hfill \ $\mu=3^{1/4}+5\cdot3^{3/4}$;\smallskip
\itemitem{c.} $E=\Q(5^{1/6})$, $F=\Q(5^{1/2})$,\ \hfill \ $\mu=1+5^{1/6}+3\cdot5^{5/6}$;\smallskip
\itemitem{d.} $E=\Q(\tau)$ ($\tau^3=3\tau+2$), $F=\Q$,\ \hfill \ $\mu=2\tau^2-\tau+2$;\smallskip
\itemitem{e.} $E=\F_{7}(\rho)$ ($\rho^3=\rho+2$), $F=\F_7$,\ \hfill \ $\mu=1+\rho$.
\bigskip

\item{4.} Dire quali dei seguenti insiemi sono campi e quali no giustificando la risposta.

\itemitem{a.} $\Q[x]/(x^5+1);$\smallskip
\itemitem{b.} $\F_5[x]/(x^2+1);$\smallskip
\itemitem{c.} $\Z[x]/(x^3+x+1);$\smallskip
\itemitem{d.} $\Q(\sqrt{3})[x]/(x^2-3);$\smallskip
\itemitem{e.} $\Q[\pi][X]/(X+1)$.
\bigskip

\item{5.} In ciascuno dei seguenti casi calcolare $[E:F]$:

\itemitem{a.} $E=\Q(2^{1/2},2^{1/3})$,\ \ \  $F=\Q$;\smallskip
\itemitem{b.} $E=\Q(2^{1/2},2^{1/3},2^{1/4},\cdots,2^{1/20})$\ \ \  $F=\Q$;\smallskip
\itemitem{c.} $E=\Q(\sqrt{5},\zeta), \zeta^3+\zeta-1=0$, $F=\Q$ (giustificare la risposta);\smallskip
\itemitem{d.} $E=\F_3[\sqrt{-1}]$,\ \ \ $F=\F_3$;\smallskip
\itemitem{e.} $E=\F_5[\sqrt{-1}]$,\ \ \ $F=\F_5$;\smallskip
\itemitem{f.} $E=\F_{31}[\sqrt{2},\sqrt{3},\sqrt{5},\sqrt{6},\sqrt{15},\sqrt{10}]$\ \ \ $\F=\F_{31}(\sqrt{10})$.
\bigskip

\item{6.} Sia $E=\Q[3^{1/h}]$ dove $h\in\N$. Dimostrare direttamente (senza usare la teoria)
che comunque scelti $a_0,a_1\ldots,a_{h-1}\in\Q$ non tutti nulli, risulta
$$\frac 1{a_0+a_13^{1/h}\cdots+a_{h-1}3^{(h-1)/h}}\in\Q[3^{1/h}].$$
\bigskip

\item{7.} Dimostrare (o dimostrare che sono sbagliate) le uguaglianze dei seguenti campi:
\itemitem{a.} $\Q(\sqrt{2},\sqrt{5},\sqrt{6})=\Q(3\sqrt{2}-\sqrt{5}+5\sqrt{3})$;\smallskip
\itemitem{b.} $\Q(\sqrt{a^2-4b})=\Q(\sigma), \sigma^2+a\sigma+b=0$, $a,b\in\Q$;\smallskip
\itemitem{c.} $\Q(\sqrt{-3},\sqrt{3})\cap\Q(\sqrt{6},\sqrt{-6})=\Q(i)$;
\bigskip

\item{8.} Risolvere i problemi sulle note di Milne a pagina 23.
\vfill

\eject

\centerline{\it Alcune Risposte (Esercizi di Teoria di Galois 1. del 6/3/03):}

\item{2.}
\itemitem{(a.}\ ${4-\alpha^2+\alpha\over3}$, ${6-\alpha^2\over5}$, $1+2\alpha-\alpha^2$),\medskip
\itemitem{(b.}\ ${\gamma^2-2\over40}$, ${\gamma^2-3\gamma+7\over 23}$, ${2-\gamma\over 4}$),\medskip
\itemitem{(c.}\ ${b\over b^2+ab-a^2}\xi+{b-a\over b^2+ab-a^2}$),\medskip
\itemitem{(d.}\ 1 se $5|t$,\hfill\break
$\zeta$ se $t\equiv4\bmod5$,\hfill\break
$\zeta^2$ se $t\equiv3\bmod5$,\hfill\break
$\zeta^3$ se $t\equiv2\bmod5$ e \hfill\break
$-1-\zeta-\zeta^2-\zeta^3$ se $t\equiv1\bmod5$);\bigskip

\item{3.} \itemitem{(a.}\ $29x^2+38x+4$),\medskip \itemitem{(b.}\
$x^4-60x^2-16428$),\medskip \itemitem{(c.}\
$x^3-3x^2-42x+44-676\sqrt{5}$),\medskip \itemitem{(d.}\
$x^3-19x^2+105x-200$),\medskip \itemitem{(e.}\
$x^3-3x^2+2x-2$);\bigskip

\item{4.} Nessuno \`{e} campo;\bigskip

\item{5.}
\itemitem{(a.}\ 6),\medskip
\itemitem{(b.}\ 232792560),\medskip
\itemitem{(c.}\ 6),\medskip
\itemitem{(d.}\ 2),\medskip
\itemitem{(e.}\ 1),\medskip
\itemitem{(f.}\ 2).
\bye
