\magnification 900
\def\Q{{\bf Q}}
\hsize=19.5cm \vsize=27.58cm \hoffset=-1.6cm

\voffset=0.5cm
\parskip=1mm
\noindent  TE1 AA02/03\ (Teoria delle Equazioni)\hfill Roma, 22
Maggio 2003. \hrule
\medskip
\smallskip

\centerline{SOLUZIONI DELL' ESAME DI FINE SEMESTRE}\medskip

 \hfill \hrule
\medskip\medskip\medskip

\item{1.} Dopo aver dato la definizione di sottogruppo transitivo
di $S_n$, si elenchino i sottogruppi transitivi di $S_4$
descrivendone gli elementi come permutazioni.\medskip

{\it Un sottogruppo $H$ di $S_n$ si dice transitivo se per ogni
$i,j\in\{1,2,\ldots,n\}$ esiste $\sigma\in H$ tale che
$\sigma(i)=j$. I sottogruppi transitivi di $S_4$ sono isomorfi a
uno dei seguenti:
$$S_4,\ A_4, V=\{(1), (1\ 2)(3\ 4),(1\ 3)(2\ 4), (1\ 4)(2\ 3)\},$$
$$C_4=\{(1), (1\ 2\ 3\ 4), (1\ 3)(2\ 4), (1\ 4\ 2\ 3)\}\ \ \ {\it e}\ \ \
D_4=C_4\cup \{(1\ 2)(3\ 4), (1\ 3), (2\ 4),  (1\ 4)(2\ 3)\}.$$}
\smallskip
\hrule
\medskip

 \item{2.} Descrivere gli elementi del gruppo di Galois del polinomio $x^5-2$ mostrando
che ha $20$ elementi.\medskip

{\it Si osservi che il campo di spezzamento del polinomio \`{e}
$\Q(2^{1/5}, \zeta_52^{1/5}, \zeta_5^22^{1/5}, \zeta_5^32^{1/5}, \zeta_5^42^{1/5})=
\Q(2^{1/5}, \zeta_5).$ Quindi la dimensione
$[\Q(2^{1/5},
\zeta_5):\Q]=[\Q(2^{1/5}, \zeta_5):\Q(2^{1/5})][\Q(2^{1/5}):\Q]=20$ visto che
$4$ non divide $[\Q(2^{1/5}):\Q]$.
Ci\`{o} mostra che il gruppo di Galois del polinomio ha esattamente 20 elementi che sono esattamente:
$$\Q(2^{1/5}, \zeta_5)\longrightarrow \Q(2^{1/5}, \zeta_5),\
2^{1/5}\mapsto \zeta^j_52^{1/5},\ \ \zeta_5\mapsto \zeta_5^i, \ \
\ \ {\it con}\ \ i=1,2,3,4\ \ {\it e}\ \
j=1,2,3,4,5.$$}\smallskip\hrule
\medskip

  \item{3.} Dimostrare che il gruppo di Galois de polinomio
(che si pu\`{o} assumere irriducibile) $x^5+x^4+x^3+2x^2+3x+4$ non
ha $20$ elementi n\'{e} $10$ mostrando che contiene un $3$
ciclo.\medskip

{\it Il polinomio ottenuto riducendo il polinomio modulo $2$ \`{e}
$x^5+x^4+x^3+x=x(x^4+x^3+x^2+1)= x(x+1)(x^3+x^2+1)$. Osserviamo
che il terzo fattore \`{e} irriducibile perch\`{e} non ha radici e
ha grado tre. Possiamo applicare il Teorema di Dedekind che
implica che il gruppo di Galois contiene un elemento che ammette
una rappresentazione come permutazione di $S_5$ del tipo
$(c_1)(c_1)(c_3)$ cio\`{e} un $3$ ciclo. Infine sia un gruppo con
$20$ elementi che un gruppo con $10$ elementi non possono
contenere un elemento di ordine $3$.}
\medskip\hrule
\medskip

\item{4.} Calcolare quanti sono i polinomi irriducibili (monici)
di grado $8$ su ${\bf F}_2$.

{\it Partiamo dalla formula $\sum_{d\mid n}dN_d(q)=q^n$ (dove
$N_d(q)$ indica il numero dei polinomi irriducibili di grado $d$
in ${\bf F}_q[x]$). Nel caso $q=2$ e $n=8$ otteniamo:
$$8N_8(2)+4N_4(2)+2N_2(2)+N_1(2)=2^8=256.$$
D'altronde $N_1(2)=2$ ($x$ e $x+1$ sono i polinomi irriducibili di
grado $1$) $N_2(2)=1$ ($x^2+x+1$ \`{e} l'unico polinomio
irriducibile di grado $2$) e $N_4(2)=3$ ($x^4+x+1$, $x^4+x^3+1$ e
$x^4+x^3+x^2+x+1$ sono i polinomi irriducibili di grado $4$).
Quindi
$$N_8(2)={1\over8}\left(256-4N_4(2)-2N_2(2)-N_1(2)\right)={256-12-2-2\over8}=30.$$
}
\medskip\hrule
\medskip

\item{5.} Calcolare il gruppo di Galois del polinomio $x^4+8x+12$
(assumendo che \`{e} irriducibile).\medskip

{\it La risolvente cubica del polinomio \`{e} $x^3-48x-64=64(({x\over4})^3-3({x\over4})-1)$.
Pertanto la risolvente \`{e} irriducibile e il suo gruppo di Galois \`{e} lo
stesso di $y^3-3y-1$. Quest'ultimo polinomio ha discriminante pari a $43^3-27=3^4$ (un quadrato perfetto)
e quindi il suo gruppo di Galois \`{e} $A_3$. Infine il polinomio di grado $4$ di partenza ha gruppo
di Galois di typo $A_4$.}
\medskip\hrule
\medskip

 \item{6.} Calcolare una formula per il discriminante di
$x^4+ax+b$.\medskip

{\it La derivata del polinomio $f=x^4+ax+b$ \`{e} $f'=4x^3+a$.
Utilizzando la formula $D=$disc$(x^4+ax+b)=\prod_{i=1}^4
f'(\alpha_i)$ e il fatto che
$\gamma=(4\alpha^3+a)=-4(a+b/\alpha)+a$, otteniamo che
$\alpha=-4b/(\gamma+3a)$. Dunque $f(-4b/(\gamma+3a))=0$ e quindi
$256b^3-4a(\gamma+3a)^3+(\gamma+3a)^4=0$. Da cui il polinomio
minimo  $f_{\gamma}=256b^3-4a(x+3a)^3+(x+3a)^4=0$ e infine
$D=f_{\gamma}(0)=\gamma_1\cdot\gamma_2\cdot\gamma_3\cdot\gamma_4=-27a^4
+ 256b^3 $}
\medskip\hrule
\medskip

\item{7.} Spiegare come si fa a costruire un polinomio il cui
gruppo di Galois ha $13$ elementi. \hfill {\it Pensare al numero
primo $53$.}
\medskip

{\it Consideriamo il campo ciclotomico $\Q(\zeta_{53})$. Sappiamo
che il gruppo di Galois Gal$(\Q(\zeta_{53})/\Q)= U({\bf Z}/53{\bf
Z})\cong {\bf Z}/52{\bf Z}.$ Dal fatto che $52=13\cdot4$ segue che
il gruppo di Galois contiene un sottogruppo con indice $13$. Si
tratta del sottogruppo $\{{\rm id},\sigma_{-1},\sigma_{23},
\sigma_{-23}\}=\langle\sigma_{23}\rangle$. Poniamo
$\eta=\zeta_{53}+\zeta_{53}^{-1}+\zeta_{53}^{23}+\zeta_{53}^{-23}$.
Vogliamo prima mostrare che ${\bf Q}(\eta)={\bf
Q}(\zeta_{53})^{\langle\sigma_{23}\rangle}$ e poi calcolare il
polinomio minimo di $\eta$ su ${\bf Q}$ il cui gruppo di Galois
sar\`{a} ciclico con $13$ elementi (perch\`{e} grazie al Teorema
di corrispondenza di Galois \`{e} isomorfo a
$H=$Gal$(\Q(\zeta_{53})/\Q)\left/\langle\sigma_{23}\rangle.\right.$
che \`{e} ciclico con 13 elementi).

Per quanto riguarda la prima affermazione, si osservi che siccome
$\sigma_{23}(\eta)=\eta$, si ha ${\bf Q}(\eta)\subseteq{\bf
Q}(\zeta_{53})^{\langle\sigma_{23}\rangle}$. D'altronde se
l'inclusione sopra fosse propria, si avrebbe un sottogruppo che
contiene propriamente $\langle\sigma_{23}\rangle$ i cui elementi
fissano $\eta$. Ci\`{o} \`{e} impossibile perch\`{e} l'indice di
$\langle\sigma_{23}\rangle$ \`{e} 13 e quindi non ci sono
sottogruppi propri del gruppo di Galois che contengono
propriamente $\langle\sigma_{23}\rangle$.

Per quanto riguarda la seconda affermazione, osservare che
$\eta=2(\cos {2\pi\over53}+\cos {46\pi\over53})\approx
0.1556709575409969811512011452$  e che il polinomio minimo
$$f_\eta(x)=\prod_{\sigma\in H}(x-\sigma(\eta))=\prod_{j=1}^{13}\left(
x-\sigma_{2^j}(\eta)\right) =\prod_{j=1}^{13}\left(x-2\left(\cos
{2 \cdot2^j\pi\over53}+\cos
{46\cdot2^j\pi\over53}\right)\right)=$$
$$=x^{13} + x^{12} - {24}x^{11} - {19}x^{10} + {190}x^{9} +
{116}x^{8} - {601}x^{7 } - {246}x^{6} + {738}x^{5} + {215}x^{4} -
{291}x^{3} - {68}x^{2} + {10}x + 1
$$}
\medskip\hrule
\medskip

 \item{8.} Si enunci nella completa
generalit\`{a} il Teorema di corrispondenza di Galois.

\noindent{\bf Teorema.} {\it Sia $E/F$ un estensione di Galois
(cio\`{e} $E$ \`{e} il campo di spezzamento di un polinomio
separabile in $F[x]$) e sia $G=$Gal$(E/F)$. Allora c'\`{e} una
corrispondenza biunivoca tra ti sottogruppi di $G$ e i sottocampi
di $E$ che contengono $F$. Se $H\leq G$ e $F\subseteq M\subseteq
E$, allora la corrispondenza \`{e} data da:
$$H\mapsto E^H,\quad M\mapsto {\rm Gal}(E/M).$$
Inoltre \item{i} $G$ corrisponde a $F$ e $\{1\}$ corrisponde a
$E$; \item{ii} $H_1\leq H_2 \Leftrightarrow E^{H_1}\supseteq
E^{H_2}.$ \item{iii} Per ogni $\sigma\in G$,
    \itemitem{-} $E^{\sigma H\sigma^{-1}}=\sigma E^H$;
    \itemitem{-} $Gal(E/\sigma M)=\sigma Gal(E/M)\sigma^{-1}$.
\item{iv} $H\triangleleft G \Leftrightarrow E^H/F$ \`{e} un
estensione normale. In tal caso inoltre $Gal(E^H/F)\cong G/H.$}
\medskip\hrule
\medskip

\item{9.} Definire il discriminante di un polinomio ${\bf F}[x]$ e
dimostrare che \`{e} un elemento di ${\bf F}$.
\medskip

{\it La definizione in questione \`{e} nelle note di Milne a
pagina 42. Mentre la propriet\`{a} da dimostrare \`{e} il primo
punto del Corollario 4.2.}
\medskip\hrule
\medskip

\item{10.} Quali sono le radici di $x^{16}+x^{12}+1$ in ${\bf
F}_2[\alpha]$, con $\alpha^4=\alpha+1$?
\hfill {\it Provare con $\alpha^3+1$.}\medskip

{\it Si ha la seguente fattorizzazione unica in ${\bf F}_2[x]$:
$x^{16}+x^{12}+1=(x^4+x^3+1)^4$. Infine le radici del polinomio
$x^4+x^3+1$ in ${\bf F}_2[\alpha]$ sono: $\alpha^3+\alpha+1$,
$\alpha^3+1$, $\alpha^3+\alpha^2+1$ e $\alpha^3+\alpha^2+\alpha$. Quindi
$$x^{16}+x^{12}+1=(x+\alpha^3+\alpha+1)^4(x+\alpha^3+1)^4(x+\alpha^3+\alpha^2+1)^4
(x+\alpha^3+\alpha^2+\alpha)^4.$$}
\medskip\hrule
\medskip



\item{11.} Si calcoli il numero di elementi nel campo di
spezzamento del polinomio $(x^4+x+1)(x^4+x^3+1)(x^2+x+1)(x^3+x+1)$
su ${\bf F}_2$.\medskip

{\it Il numero di elementi del campo di spezzamento \`{e}
$2^{12}$. Infatti se $F$ \`{e} un campo di spezzamento, allora
$[F:{\bf F_2}]$ \`{e} divisibile per $4$ in quanto contiene un
sottocampo isomorfo a ${\bf F}_2[\tau], \tau^4=\tau+1$. \`{E}
anche divisibile per tre infatti contiene un sottocampo isomorfo a
${\bf F}_2[\rho], \rho^3=\rho+1$. Quindi $12\mid [F:{\bf F}_2]$.

Infine ogni campo con $2^{12}$ elementi contiene tutte le radici
del polinomio e quindi contiene un campo di
spezzamento.}\medskip\hrule
\medskip


\item{12.} Dopo aver enunciato il teorema di caratterizzazione per
i numeri reali costruibili, si dimostri che
$\sqrt{1+\sqrt{3-2^{1/8}}}$ esibendone una costruzione nel senso
della teoria dei campi. Dimostrare anche che $2^{1/5}$ non \`{e}
costruibile.\medskip

{\it {\bf Teorema.} $x\in {\bf R}$ \`{e} costruibile se e solo se
esiste un torre di campi $\Q=K_0\subset
K_1\subset\cdots\subset K_n$ tali che $[K_i:K_{i-1}]=2$ per ogni
$i=1,2,\ldots,n$ e $K_n=\Q(x)$.\smallskip

$\sqrt{1+\sqrt{3-2^{1/8}}}$ \`{e} costruibile in quanto una sua
\`{e} data da
$$\Q\subset\Q(\sqrt{2})\subset\Q(\sqrt{\sqrt{2}})
\subset\Q(\sqrt{\sqrt{\sqrt{2}}})\subset\Q(\sqrt{3-2^{1/8}})
\subset\Q(\sqrt{1+\sqrt{3-2^{1/8}}}).$$ Infine $2^{1/5}$ non \`{e}
costruibile perch\`{e} $[\Q(2^{1/5}):\Q]=5$ non \`{e} una potenza
di $2$ e pertanto \`{e} impossibile soddisfare il teorema di
caratterizzazione.\medskip\hrule}



 \bye
