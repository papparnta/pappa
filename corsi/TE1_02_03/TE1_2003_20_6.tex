\nopagenumbers
\font\title=cmti12
\vfuzz=280pt
%\def\ve{\vfill\eject}
%\def\vv{\vfill}
%\def\vs{\vskip-2cm}
%\def\vss{\vskip10cm}
%\def\vst{\vskip13.3cm}

\def\ve{\bigskip\bigskip}
\def\vv{\bigskip\bigskip}
\def\vs{}
\def\vss{}
\def\vst{\bigskip\bigskip}

\hsize=19.5cm
\vsize=27.58cm
\hoffset=-1.6cm
\voffset=0.5cm
\parskip=-.1cm
\ \vs \hskip -6mm TE1 AA02/03\ (Teoria delle Equazioni)\hfill
ESAME SCRITTO \hfill Roma, 20 Giugno 2003. \hrule
\bigskip\noindent
{\title COGNOME}\  \dotfill\  {\title NOME}\ \dotfill {\title
MATRICOLA}\ \dotfill\
\smallskip  \noindent
Risolvere il massimo numero di esercizi accompagnando le risposte
con spiegazioni chiare ed essenziali. \it Inserire le risposte
negli spazi predisposti. NON SI ACCETTANO RISPOSTE SCRITTE SU
ALTRI FOGLI. Scrivere il proprio nome anche nell'ultima pagina.
\rm 1 Esercizio = 3 punti. Tempo previsto: 2 ore. Nessuna domanda
durante la prima ora e durante gli ultimi 20 minuti.
\smallskip
\hrule
\medskip

\item{1.} Calcolare il grado del polinomio minimo su ${\bf Q}$, di
$\zeta_{13}+\zeta^3_{13}+\zeta^9_{13}$. (Se hai tempo calcola
anche il polinomio minimo). %%%%%%%%

\vv \item{2.} Dare la definizione di polinomio minimo di un numero
algebrico enunciando e dimostrando le varie caratterizzazioni. %%%%%%%%

\ve\ \vs \item{3.} Dopo aver dimostrato che \`{e} un estensione di
Galois di ${\bf Q}$, determinare tutti i sottocampi di ${\bf
Q}(\zeta_{16})$. %%%%%%%%%%%

\vv

\item{4.} Calcolare quanti sono i polinomi irriducibili (monici)
di grado $7$ su ${\bf F}_{11}$.\ve\ \vs %%%%

\item{5.} Calcolare il gruppo di Galois del polinomio
$x^4+8x^2+2$. %%%%%%%%%

\vv \item{6.} Calcolare una formula per il discriminante di
$x^n+ax+b$. \ve\ \vs  %%%%%%%%%%

\item{7.} Spiegare come si fa a costruire un polinomio il cui
gruppo di Galois ciclico con $n$ elementi. %%%%%

\vv \item{8.} Si enunci nella completa generalit\`{a} il Teorema
di corrispondenza di Galois. %%%%


\ve\ \vs

\item{9.} Dimostrare che ogni campo finito \`{e} un estensione di
Galois del suo sottocampo caratteristico e descriverne il gruppo
di Galois.\vv %%%%%%%%

\item{10.} Dare un esempio di campo finito ${\bf F}_{16}$ con $16$
elementi e sia ${\bf F}_{4}$ un suo sottocampo con $8$ elementi.
Costruire tutti gli ${\bf F}_{4}$--omomorfismi di ${\bf F}_{16}$.
%%%%%

\ve\ \vs


\item{11.} Si calcoli il numero di elementi nel campo di
spezzamento del polinomio
$(x^3+x+1)(x^3+x^2+1)(x^9+x^6+1)(x^{27}+5x^9+1)$ su ${\bf F}_3$.
%%%%%
\vss

\item{12.} Enunciare e dimostrare il Teorema di Gauss sulla
costruibilit\`{a} dei poligoni regolari.\vv %%%%%%%%%%%

\ \vst

\centerline{\hskip 6pt\vbox{\tabskip=0pt \offinterlineskip
\def \trl{\noalign{\hrule}}
\halign to500pt{\strut#& \vrule#\tabskip=0.7em plus 1em&
\hfil#& \vrule#& \hfill#\hfil& \vrule#&
\hfil#& \vrule#& \hfill#\hfil& \vrule#&
\hfil#& \vrule#& \hfill#\hfil& \vrule#&
\hfil#& \vrule#& \hfill#\hfil& \vrule#&
\hfil#& \vrule#& \hfill#\hfil& \vrule#&
\hfil#& \vrule#& \hfill#\hfil& \vrule#&
\hfil#& \vrule#& \hfil#& \vrule#\tabskip=0pt\cr\trl
&& NOME E COGNOME && 1 && 2 && 3 && 4 && 5 && 6 && 7 && 8 && 9 && 10 && 11 && 12 &&  TOT. &\cr\trl
&& &&   &&   &&     &&   &&   &&   &&   &&   &&    &&   &&   &&  && &\cr
&& \dotfill &&     &&   &&   &&   &&   &&   &&    &&  &&   && && && && &\cr\trl
}}}
 \bye
