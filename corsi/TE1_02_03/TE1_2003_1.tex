\nopagenumbers \font\title=cmti12
\def\ve{\vfill\eject}
\def\vv{\vfill}
\def\vs{\vskip-2cm}
\def\vss{\vskip10cm}
\def\vst{\vskip13.3cm}

%\def\ve{\bigskip\bigskip}
%\def\vv{\bigskip\bigskip}
%\def\vs{}
%\def\vss{}
%\def\vst{\bigskip\bigskip}

\hsize=19.5cm
\vsize=27.58cm
\hoffset=-1.6cm
\voffset=0.5cm
\parskip=-.1cm
\ \vs \hskip -6mm TE1 AA02/03\ (Teoria delle Equazioni)\hfill
ESAME DI MET\`{A} SEMESTRE \hfill Roma, 24 Aprile 2003. \hrule
\bigskip\noindent
{\title COGNOME}\  \dotfill\  {\title NOME}\ \dotfill {\title
MATRICOLA}\ \dotfill\
\smallskip  \noindent
Risolvere il massimo numero di esercizi accompagnando le risposte
con spiegazioni chiare ed essenziali. \it Inserire le risposte
negli spazi predisposti. NON SI ACCETTANO RISPOSTE SCRITTE SU
ALTRI FOGLI. Scrivere il proprio nome anche nell'ultima pagina.
\rm 1 Esercizio = 3 punti. Tempo previsto: 2 ore. Nessuna domanda
durante la prima ora e durante gli ultimi 20 minuti.
\smallskip
\hrule
\medskip

\item{1.} Data un estensione $E/F$, si dica cosa significa
che $\alpha\in E$ \`{e} algebrico su $F$ e cosa \`{e} il polinomio minimo
di $\alpha$ su $F$ dimostrando che \`{e} irriducibile.

\vv \item{2.} Descrivere gli elementi del gruppo di Galois del
polinomio $(x^2-2)(x^2+3)$ determinando anche tutti i sottocampi
del campo di spezzamento.

\ve\ \vs \item{3.} Dopo aver verificato che \`{e} algebrico,
calcolare il polinomio minimo di $\cos \pi/12$ su ${\bf Q}$.

\vv


\item{4.} Quanti elementi ha il campo di spezzamento di
$(x^2+x+1)(x^3+x^2+1)$ su ${\bf F}_2$? \ve\ \vs

\item{5.} Dimostrare che se $p$ \`{e} primo, $\cos 2\pi/p^2$
soddisfa un polinomio di grado $p(p-1)/2$ su ${\bf Q}$.

\vv \item{6.} Mostrare che un estensione di campi \`{e}  finita se
e solo se \`{e} algebrica e finitamente generata spiagando le
nozioni di cui si parla. \ve\ \vs

\item{7.} Descrivere gli elementi del gruppo di Galois del campo
di spezzamento di $x^n-1$. \vv \item{8.} Si enunci nella completa
generalit\`{a} il Teorema di corrispondenza di Galois.

\ve\ \vs

\item{9.} Calcolare la dimensione su ${\bf Q}$ del campo di
spezzamento del seguente polinomio $x^3+x+10$. \vv

\item{10.} Dare un esempio di polinomio non separabile, un esempio de estensione
algebrica normale e non separabile e di
una separabile e non normale.

\ve\ \vs


\item{11.} Descrivere gli ${\bf Q}(\sqrt{-1})$--omomorfismi di
${\bf Q}(\sqrt{-3},\sqrt{3})$ in ${\bf C}$.
\vss

\item{12.} Dimostrare che se $E_1$ e $E_2$ sono estensioni di
Galois di $F$ con $E_1\subset L$ e $E_2\subset L$, allora $E_1\cap
E_2$ e  $E_1E_2$ (il campo composto) sono estensioni di Galois. (Per
ulteriore punteggio mostrare che ${\rm Gal}(E_1E_2/F)\cong
{\rm Gal}(E_1/F)\times {\rm Gal}(E_2/F)$ se $E_1\cap E_2 = F$). \vv

\ \vst

\centerline{\hskip 6pt\vbox{\tabskip=0pt \offinterlineskip
\def \trl{\noalign{\hrule}}
\halign to500pt{\strut#& \vrule#\tabskip=0.7em plus 1em&
\hfil#& \vrule#& \hfill#\hfil& \vrule#&
\hfil#& \vrule#& \hfill#\hfil& \vrule#&
\hfil#& \vrule#& \hfill#\hfil& \vrule#&
\hfil#& \vrule#& \hfill#\hfil& \vrule#&
\hfil#& \vrule#& \hfill#\hfil& \vrule#&
\hfil#& \vrule#& \hfill#\hfil& \vrule#&
\hfil#& \vrule#& \hfil#& \vrule#\tabskip=0pt\cr\trl
&& NOME E COGNOME && 1 && 2 && 3 && 4 && 5 && 6 && 7 && 8 && 9 && 10 && 11 && 12 &&  TOT. &\cr\trl
&& &&   &&   &&     &&   &&   &&   &&   &&   &&    &&   &&   &&  && &\cr
&& \dotfill &&     &&   &&   &&   &&   &&   &&    &&  &&   && && && && &\cr\trl
}}}
 \bye
