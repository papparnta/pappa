\nopagenumbers \font\title=cmti12
\def\ve{\vfill\eject}
\def\vv{\vfill}
\def\vs{\vskip-2cm}
\def\vss{\vskip10cm}
\def\vst{\vskip13.3cm}

%\def\ve{\bigskip\bigskip}
%\def\vv{\bigskip\bigskip}
%\def\vs{}
%\def\vss{}
%\def\vst{\bigskip\bigskip}

\hsize=19.5cm
\vsize=27.58cm
\hoffset=-1.6cm
\voffset=0.5cm
\parskip=-.1cm
\ \vs \hskip -6mm AL310 AA18/19\ (Teoria delle Equazioni)\hfill ESAME
DI FINE SEMESTRE \hfill Roma, 19 Dicembre  2018. \hrule
\bigskip\noindent
{\title COGNOME}\  \dotfill\ {\title NOME}\ \dotfill {\title
MATRICOLA}\ \dotfill\
\smallskip  \noindent
Risolvere il massimo numero di esercizi accompagnando le risposte
con spiegazioni chiare ed essenziali. \it Inserire le risposte
negli spazi predisposti. NON SI ACCETTANO RISPOSTE SCRITTE SU
ALTRI FOGLI. Scrivere il proprio nome anche nell'ultima pagina.
\rm 1 Esercizio = 4 punti. Tempo previsto: 2 ore. Nessuna domanda
durante la prima ora e durante gli ultimi 20 minuti.
\smallskip
\hrule\smallskip
\centerline{\hskip 6pt\vbox{\tabskip=0pt \offinterlineskip
\def \trl{\noalign{\hrule}}
\halign to300pt{\strut#& \vrule#\tabskip=0.7em plus 1em& \hfil#&
\vrule#& \hfill#\hfil& \vrule#& \hfil#& \vrule#& \hfill#\hfil&
\vrule#& \hfil#& \vrule#& \hfill#\hfil& \vrule#& \hfil#& \vrule#&
\hfill#\hfil& \vrule#& \hfil#& \vrule#& \hfill#\hfil& \vrule#&
\hfil#& \vrule#& \hfill#\hfil& \vrule#& \hfil#& \vrule#& \hfil#&
\vrule#\tabskip=0pt\cr\trl && FIRMA && 1 && 2 && 3 && 4 &&
5 && 6 && 7 && 8 && 9 &&  TOT. &\cr\trl && &&   &&
&&     &&   &&   &&   &&   &&   &&    && &\cr &&
\dotfill &&     &&   &&   &&   &&     &&   && && && &&
&\cr\trl }}}
\medskip

\item{1.} Sia $f(x)=(x^4-2)(x^2+4)$.
\itemitem{-a-} Determinarne il gruppo di Galois;
\itemitem{-b-} Etichettarne le radici usando i numeri da uno a sei e descriverne esplicitamente il gruppo di
Galois come sottogruppo di $S_6$.

\vv\item{2.} Considerare ${\bf Q}(\zeta_{105})$.
\itemitem{-a-} Descriverne il gruppo di Galois e scrivendolo come prodotto di gruppi ciclici
\itemitem{-b-} Elencarne tutti i sottocampi quadratici

\ve\ \vs

\item{3.} Dimostrare che due campi finiti con lo stesso numero di elementi sono isomorfi.

\vv\item{4.} Sia $\alpha=\cos(\pi/32)$ 
\itemitem{-a-}  Dimostrare che $\alpha$ \`{e} costruibile;
\itemitem{-b-} Determinare esplicitamente una costruzione di ${\bf Q}(\alpha)$;
\itemitem{-c-} Scrivere una formula esplicita usando radicali per $\alpha$.\vv

\item{5.} Determinare un polinomio a coefficienti razionali con gruppo di Galois isomorfo a  $C_6\times S_{3}$.
\ve\ \vs

\item{6.} Sia $p\geq3$ un primo. Dimostrare che se $\zeta_p=e^{2\pi i/p}$, allora Gal(${\bf Q}(2^{1/p},\zeta_p)/{\bf Q}(\zeta_p))\cong {\bf Z}/p{\bf Z}.$

\vv \item{7.} Si enunci nella completa generalit\`a il Teorema di
corrispondenza di Galois e se ne dimostrino le parti salienti.

\ve\ \vs


\item{8.} Sia $f(x)=x^4+Ax+B\in{\bf Q}[x]$.
\itemitem{-a-} Fornire la definizione del discriminante di un polinomio a coefficienti razionali;
\itemitem{-b-} Dimostrare che il discriminante di $f$ \`e pari a $4^4B^{3}-3^3A^{4}$
\vv

\item{9.} Dopo aver determinato il numero di polinomi irriducibili di grado $4$ su ${\bf F}_2$, si determinino le radici di 
$x^4+x^3+x^2+x+1\in{\bf F}_2[x]$ nel campo ${\bf F}_2[\alpha], \alpha^4=\alpha+1$.

\ \vst
 \bye
