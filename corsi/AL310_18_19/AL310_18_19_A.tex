\nopagenumbers \font\title=cmti12
\def\ve{\vfill\eject}
\def\vv{\vfill}
\def\vs{\vskip-2cm}
\def\vss{\vskip10cm}
\def\vst{\vskip13.3cm}

%\def\ve{\bigskip\bigskip}
%\def\vv{\bigskip\bigskip}
%\def\vs{}
%\def\vss{}
%\def\vst{\bigskip\bigskip}

\hsize=19.5cm
\vsize=27.58cm
\hoffset=-1.6cm
\voffset=0.5cm
\parskip=-.1cm
\ \vs \hskip -6mm AL310 AA18/19\ (Teoria delle Equazioni)\hfill APPELLO A (Scritto) \hfill Roma, 24 Gennaio 2019. \hrule
\bigskip\noindent
{\title COGNOME}\  \dotfill\ {\title NOME}\ \dotfill {\title
MATRICOLA}\ \dotfill\
\smallskip  \noindent
Risolvere il massimo numero di esercizi accompagnando le risposte
con spiegazioni chiare ed essenziali. \it Inserire le risposte
negli spazi predisposti. NON SI ACCETTANO RISPOSTE SCRITTE SU
ALTRI FOGLI. Scrivere il proprio nome anche nell'ultima pagina.
\rm 1 Esercizio = 4 punti. Tempo previsto: 2 ore. Nessuna domanda
durante la prima ora e durante gli ultimi 20 minuti.
\smallskip
\hrule\smallskip
\centerline{\hskip 6pt\vbox{\tabskip=0pt \offinterlineskip
\def \trl{\noalign{\hrule}}
\halign to277pt{\strut#& \vrule#\tabskip=0.7em plus 1em& \hfil#&
\vrule#& \hfill#\hfil& \vrule#& \hfil#& \vrule#& \hfill#\hfil&
\vrule#& \hfil#& \vrule#& \hfill#\hfil& \vrule#& \hfil#& \vrule#&
\hfill#\hfil& \vrule#& \hfil#& \vrule#& \hfill#\hfil& \vrule#&
\hfil#& \vrule#& \hfill#\hfil& \vrule#& \hfil#& \vrule#& \hfil#&
\vrule#\tabskip=0pt\cr\trl && FIRMA && 1 && 2 && 3 && 4 &&
5 && 6 && 7 && 8 &&   9 &\cr\trl && &&   &&
&&     &&   &&   &&   &&   &&    && &\cr &&
\dotfill &&     &&   &&   &&     &&   && && && &&
&\cr\trl }}}
\medskip

\item{1.} Rispondere alle sequenti domande fornendo una giustificazione di una riga (giustificazioni
incomplete o poco chiare comportano punteggio nullo):\bigskip\bigskip\bigskip


\itemitem{a.} E' sempre vero che se $F$ \`e un campo e $\alpha$ \`e algebrico su $F$, allora $[F(\alpha):F]=\deg f_\alpha =\#{\rm Gal}(f_\alpha)$ (dove
${\rm Gal}(f)$ indica il gruppo di Galois del polinomio $f\in F[X]$?\medskip\bigskip\bigskip

\ \dotfill\ \bigskip\bigskip\bigskip\vfil

\itemitem{b.} Scrivere una ${\bf Q}$--base del campo ${\bf Q}[3^{1/4},2^{1/3}]$.\medskip\bigskip\bigskip

\ \dotfill\ \bigskip\bigskip\bigskip\vfil

\itemitem{c.} Quanti elementi ha il campo di spezzamento di $(X^{32}+7X+2)(X^8+3X^4+5)(X^{32}+X^4)\in{\bf F}_2[X]$?\medskip\bigskip\bigskip
 
\ \dotfill\ \bigskip\bigskip\bigskip\vfil

\itemitem{d.} \`E possibile costruire un esempio di estensione di un campo finito con gruppo di Galois abeliano e isomorfo ${\bf Z}/4{\bf Z}\times{\bf Z}/5{\bf Z}$?\medskip\bigskip\bigskip

\ \dotfill\ \bigskip\bigskip\bigskip

\vfil\eject

\item{2.} Fornire la definizione di {\it composto} di due sottocampi di un campo dato e fornire un esempio in cui l'unione di due sottocampi coincide con il composto.
\vv


\item{3.} Dimostrare che un polinomio a coefficienti razionali \`e irriducibile se e solo 
se il suo gruppo di Galois agisce transitivamente sulle sue radici.
\vv

\item{4.} Determinare i gruppi di Galois su ${\bf Q}$  e su ${\bf F}_5$ del seguente polinomio $x^6-3^6$.\ve\ \vs

\item{5.} Sia $\alpha=\cos2\pi/5+\cos2\pi/7+\cos2\pi/13+\cos2\pi/17$. Dopo aver mostrato che ${\bf Q}(\alpha)/{\bf Q}$ \`e Galois, si determini il grado del polinomio minimo su ${\bf Q}$ di $\alpha$.
\vv


\item{6.} Si enunci e si dimostri il Lemma di Artin e si spieghi il suo ruolo in Teoria di Galois.\vv


\item{7.} Dimostrare che un polinomio $f$ di grado $n\ge2$ a coefficienti in ${\bf F}_{7}$ \`e irriducibile se e solo se per ogni $k=1,\ldots,n$, si ha che $\gcd(f,x^{7^k}-x)=1$.\ve\ \vs

\item{8.} Determinare tutti i sottocampi cubici (cio\`e di grado tre su ${\bf Q}$) del campo di spezzamento di $(x^3-2)(x^3-3)$.\vv

\item{9.} Descrivere in dettagli is reticolo dei sottocampi del campo di spezzamento di $X^{24}-1\in{\bf Q}[X]$ indicando per ciascun sottocampo il polinomio minimo di un generatore.
\ \vst

 \bye
