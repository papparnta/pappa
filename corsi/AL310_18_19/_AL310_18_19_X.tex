\nopagenumbers \font\title=cmti12
\def\ve{\vfill\eject}
\def\vv{\vfill}
\def\vs{\vskip-2cm}
\def\vss{\vskip10cm}
\def\vst{\vskip13.3cm}

%\def\ve{\bigskip\bigskip}
%\def\vv{\bigskip\bigskip}
%\def\vs{}
%\def\vss{}
%\def\vst{\bigskip\bigskip}

\hsize=19.5cm
\vsize=27.58cm
\hoffset=-1.6cm
\voffset=0.5cm
\parskip=-.1cm
\ \vs \hskip -6mm AL310 AA18/19\ (Teoria delle Equazioni)\hfill APPELLO X (Scritto) \hfill Roma, 20 Settembre 2019. \hrule
\bigskip\noindent
{\title COGNOME}\  \dotfill\ {\title NOME}\ \dotfill {\title
MATRICOLA}\ \dotfill\
\smallskip  \noindent
Risolvere il massimo numero di esercizi accompagnando le risposte
con spiegazioni chiare ed essenziali. \it Inserire le risposte
negli spazi predisposti. NON SI ACCETTANO RISPOSTE SCRITTE SU
ALTRI FOGLI. Scrivere il proprio nome anche nell'ultima pagina.
\rm 1 Esercizio = 4 punti. Tempo previsto: 2 ore. Nessuna domanda
durante la prima ora e durante gli ultimi 20 minuti.
\smallskip
\hrule\smallskip
\centerline{\hskip 6pt\vbox{\tabskip=0pt \offinterlineskip
\def \trl{\noalign{\hrule}}
\halign to277pt{\strut#& \vrule#\tabskip=0.7em plus 1em& \hfil#&
\vrule#& \hfill#\hfil& \vrule#& \hfil#& \vrule#& \hfill#\hfil&
\vrule#& \hfil#& \vrule#& \hfill#\hfil& \vrule#& \hfil#& \vrule#&
\hfill#\hfil& \vrule#& \hfil#& \vrule#& \hfill#\hfil& \vrule#&
\hfil#& \vrule#& \hfill#\hfil& \vrule#& \hfil#& \vrule#& \hfil#&
\vrule#\tabskip=0pt\cr\trl && FIRMA && 1 && 2 && 3 && 4 &&
5 && 6 && 7 && 8 &&   9 &\cr\trl && &&   &&
&&     &&   &&   &&   &&   &&    && &\cr &&
\dotfill &&     &&   &&   &&     &&   && && && &&
&\cr\trl }}}
\medskip

\item{1.} Rispondere alle sequenti domande fornendo una giustificazione di una riga (giustificazioni
incomplete o poco chiare comportano punteggio nullo):\bigskip\bigskip\bigskip

\itemitem{a.} \`E vero che i gruppi non abeliani non sono gruppi di Galois di estensioni finite di campi finiti?\medskip\bigskip\bigskip

\ \dotfill\ \bigskip\bigskip\bigskip\vfil

\itemitem{b.} Scrivere una ${\bf Q}$--base del campo di spezzamento del polinomio $(X^3-1)(X^3-2)(X^3-3)\in{\bf Q}[X]$.\medskip\bigskip\bigskip

\ \dotfill\ \bigskip\bigskip\bigskip\vfil

\itemitem{c.} Scrivere un elemento primitivo dell'estensione ${\bf Q}(\sqrt{2},2^{1/4},2^{1/5})$?\medskip\bigskip\bigskip
 
\ \dotfill\ \bigskip\bigskip\bigskip\vfil

\itemitem{d.} Quanti elementi ha il gruppo di Galois di $X^9-2\in{\bf Q}[X]?$\medskip\bigskip\bigskip

\ \dotfill\ \bigskip\bigskip\bigskip


\vfil\eject


\item{2.} Dopo aver fornito la definizione di estensione {\it normale} e di estensione {\it separabile}, si fornisca l'esempio di un'estensione normale e non separabile e una separabile e non normale.
\vv


\item{3.} Sia $r\in{\bf N}$. Fornire un esempio di polinomio in ${\bf Q}[X]$ il cui gruppo di Galois
\`e isomorfo a $({\bf Z}/3{\bf Z})^2$.\ve\ \vs

\item{4.} Calcolare il gruppo di Galois del polinomio $X^4+2X^2+2X\in{\bf F}_5[X]$. \vv

\item{5.} Fornire la definizione di sottogruppo transitivo di $S_n$ e descrivere tutti i sottogruppi transitivi di $S_3$ e $S_4$.
\ve\ \vs

\item{6.} Dopo aver enunciato il Teorema di Gauss per la costruibilit\`a dei poligoni regolari, si scrivano tutti gli interi $2\le m\le 150$ per i quali l'$m$--agono regolare risulta construibile con riga e compasso.\vskip 6cm\bigskip\bigskip\bigskip\vv\vv

\item{7.} Costruire un campo finito con $27$ elementi e determinare l'ordine di ciascuno dei suoi elementi non nulli.\vskip 6cm\bigskip\bigskip\bigskip\vv\vv

% \item{8.} Dopo aver fornito la dimostrazione di numero costruibile, dimostrare che tutti gli elementi del campo di spezzamento del polinomio $x^4-2\in{\bf Q}[x]$
% sono costruibili.


\item{8.} Descrivere in dettagli is reticolo dei sottocampi del campo di spezzamento di $X^{30}-1\in{\bf Q}[X]$ indicando per ciascun sottocampo il polinomio minimo di un generatore.
\ \vst

 \bye
