\nopagenumbers \font\title=cmti12
\def\ve{\vfill\eject}
\def\vv{\vfill}
\def\vs{\vskip-2cm}
\def\vss{\vskip10cm}
\def\vst{\vskip13.3cm}

% \def\ve{\bigskip\bigskip}
% \def\vv{\bigskip\bigskip}
% \def\vs{}
% \def\vss{}
% \def\vst{\bigskip\bigskip}

\hsize=19.5cm
\vsize=27.58cm
\hoffset=-1.6cm
\voffset=0.5cm
\parskip=-.1cm
\ \vs \hskip -6mm AL310 AA18/19\ (Teoria delle Equazioni)\hfill MIDTERM EXAM \hfill Rome, November 8, 2018 \hrule
\bigskip\noindent
{\title FAMILY NAME}\  \dotfill\ {\title NAME}\ \dotfill {\title
MATRICOLA}\ \dotfill\
\smallskip  \noindent
Solve as many problems as possible by giving clear and essential explenations. \it Write each solution in the appropriate space. SOLUTIONS IN OTHER SHEETS WILL NOT BE ACCEPTED.
\rm 1 Problem = 4 points. Time: 2 hours.
\smallskip
\hrule\smallskip
\centerline{\hskip 6pt\vbox{\tabskip=0pt \offinterlineskip
\def \trl{\noalign{\hrule}}
\halign to277pt{\strut#& \vrule#\tabskip=0.7em plus 1em& \hfil#&
\vrule#& \hfill#\hfil& \vrule#& \hfil#& \vrule#& \hfill#\hfil&
\vrule#& \hfil#& \vrule#& \hfill#\hfil& \vrule#& \hfil#& \vrule#&
\hfill#\hfil& \vrule#& \hfil#& \vrule#& \hfill#\hfil& \vrule#&
\hfil#& \vrule#& \hfill#\hfil& \vrule#& \hfil#& \vrule#& \hfil#&
\vrule#\tabskip=0pt\cr\trl && SIGNATURE && 1 && 2 && 3 && 4 &&
5 && 6 && 7 && 8  &&  TOT. &\cr\trl && &&   &&
&&     &&   &&     &&   &&   &&    && &\cr &&
\dotfill &&       &&   &&   &&     &&   && && && &&
&\cr\trl }}}
\medskip

\item{1.} Let $p$ be a prime number, let ${\bf F}_{p^n}$ be a finite field with $p^n$ elements, let $f\in{\bf F}_p[x]$ and let $\alpha\in {\bf F}_{p^n}$
be a root of $f$.
\itemitem{a.} Show that also $\alpha^p$ is a root of $f$.
\itemitem{b.} Show that for every positive integer $k$, $\alpha^{p^k}$ is a root of $f$. 
\itemitem{c.} Show that if $f$ is irriducibile and $n=\deg f$, then $\alpha, \alpha^p,\cdots,\alpha^{p^{n-1}}$ are all distinct.
\itemitem{d.} Deduce that every finite field with $p^n$ elements is a normal extension of ${\bf F}_p$.
\vv

%Show that un estensione finita \`{e} necessariamente algebrica. Produrre
%un esempio of un estensione algebrica non finita.

\item{2.} Give the definition of an algebraic closed field and of the algebraic closure of a field.\ve\ \vs

\item{3.} Determine the degree of the splitting field of $(x^3-2)(x^3-5)(x^2+x+1)$ over ${\bf Q}$.\vv

%Dopo aver verificato che is algebrico, calcolare
%il polinomio minimo of $\cos \pi/9$ over ${\bf Q}$.

\item{4.} Show that if $(x,y)\in{\bf C}$ is costructible, then ${\bf Q}(x,y)/{\bf Q}$ is finite and $[{\bf Q}(x,y):{\bf Q}]$ is a power of $2$.\ve\ \vs

\item{5.} Let $K={\bf Q}(\sqrt3,\sqrt5)$
\itemitem{a.} Compute $[K:{\bf Q}]$ and show that $K={\bf Q}(\sqrt3+\sqrt5)$
\itemitem{a.} Compute minimum polynomial of $\sqrt3+\sqrt5$ over ${\bf Q}$ and over
${\bf Q}(\sqrt3)$
\itemitem{a.} After having shown that ${\bf Q}(\sqrt{15})\subseteq K$, describe the monomorphisms $K\rightarrow {\bf C}$ that fix ${\bf Q}(\sqrt{15})$. 
\vv

%--\item{6.} Descrivere la nozione of campo perfetto ofmostrando che i campi finiti
%sono perfetti.

\item{6.} Consider the cyclotomic field ${\bf Q}(\zeta_{15})$ ($\zeta_{15}=e^{2\pi/15}$).
\itemitem{a.} Compute the mimimal polynomial of $\zeta_{15}$ over ${\bf Q}$
\itemitem{b.} Compute the mimimal polynomial of $\zeta_{15}$ over ${\bf Q}(\zeta_3)$ and over ${\bf Q}(\zeta_5)$
\itemitem{c.} Determine all the automorphisms of ${\bf Q}(\zeta_{15})$ that fix ${\bf Q}(\zeta_3)$\ve\ \vs

\item{7.} After having shown that it is algebraic, compute the minimal polynomial of $\cos2\pi/15$ over ${\bf Q}$. \bf(hint: if $\theta=2\pi/15$, consider the $\cos(5\theta)$ and apply the classical formulas from trigonometry)\rm 
\vv

\item{8.} State and prove the ``multiplicativity of degrees Theorem'' (if $K\subseteq L\subseteq M$, then $[M:K]=[M:L][L:K]$).
\ve

 \bye
