\nopagenumbers \font\title=cmti12
% \def\ve{\vfill\eject}
% \def\vv{\vfill}
% \def\vs{\vskip-2cm}
% \def\vss{\vskip10cm}
% \def\vst{\vskip13.3cm}

\def\ve{\bigskip\bigskip}
\def\vv{\bigskip\bigskip}
\def\vs{}
\def\vss{}
\def\vst{\bigskip\bigskip}

\hsize=19.5cm
\vsize=27.58cm
\hoffset=-1.6cm
\voffset=0.5cm
\parskip=-.1cm
\ \vs \hskip -6mm AL310 18/19\ (Teoria delle Equazioni)\hfill Appello B (scritto) \hfill Roma, 12 Gennaio 2019. \hrule
\bigskip\noindent
{\title COGNOME}\  \dotfill\ {\title NOME}\ \dotfill {\title
MATRICOLA}\ \dotfill\
\smallskip  \noindent
Risolvere il massimo numero di esercizi accompagnando le risposte
con spiegazioni chiare ed essenziali. \it Inserire le risposte
negli spazi predisposti. NON SI ACCETTANO RISPOSTE SCRITTE SU
ALTRI FOGLI. Scrivere il proprio nome anche nell'ultima pagina.
\rm 1 Esercizio = 5 punti. Tempo previsto: 2 ore. Nessuna domanda
durante la prima ora e durante gli ultimi 20 minuti.
\smallskip
\hrule\smallskip
\centerline{\hskip 6pt\vbox{\tabskip=0pt \offinterlineskip
\def \trl{\noalign{\hrule}}
\halign to277pt{\strut#& \vrule#\tabskip=0.7em plus 1em& \hfil#&
\vrule#& \hfill#\hfil& \vrule#& \hfil#& \vrule#& \hfill#\hfil&
\vrule#& \hfil#& \vrule#& \hfill#\hfil& \vrule#& \hfil#& \vrule#&
\hfill#\hfil& \vrule#& \hfil#& \vrule#& \hfill#\hfil& \vrule#&
\hfil#& \vrule#& \hfill#\hfil& \vrule#& \hfil#& \vrule#& \hfil#&
\vrule#\tabskip=0pt\cr\trl && FIRMA && 1 && 2 && 3 && 4 &&
5 && 6 && 7 && 8 &&   TOT. &\cr\trl && &&   &&
&&     &&   &&   &&   &&   &&    && &\cr &&
\dotfill &&     &&   &&   &&     &&   && && && &&
&\cr\trl }}}
\medskip
 
 \item{1.} 
 \itemitem{(a)} Determinare il polinomio minimo di $2\cdot 3^{1/3} + 3\cdot 3^{-1/3}$ su ${\bf Q}$, dimostrando che si tratta del polinomio minimo.
 \vfil
 
 \itemitem{(b)} Dimostrare che  ${\bf Q}(2\cdot 3^{1/3} + 3\cdot 3^{-1/3})={\bf Q}(3^{1/3})$ e che ${\bf Q}(3^{1/3})\ne{\bf Q}(\sqrt{3})$
 \vfil
 

\item{2.} Sia $R$ un dominio (i.e. anello commutativo senza divisori dello zero) e supporre che $F$ \`e un campo contenuto in
$R$ (come sottoanello). 
Dimostrare che $\dim_FR$ \`e finita, allora $R$ \`e un campo. Mostrare che la condizione $\dim_FR<\infty$ \`e necessaria.

\vfil

\eject

\item{3.} Dimostrare il Teorema sulla transitivit\`a sulle estenzioni algebriche:se $F\subseteq K\subseteq L$ sono estensioni 
di campi tali che $K$ \`e algebrica su $F$ e  $L$ \`e algebrica su $K$, allora $L$ \`e algebrica su $F$.
\vfill

%Dopo aver verificato che \`e algebrico, calcolare
%il polinomio minimo di $\cos \pi/9$ su ${\bf Q}$.

\item{4.} Dimostrare che $K={\bf Q}(\sqrt{-6},\sqrt{2},\sqrt{6},5^{1/4}, 5^{1/3}$ \`e un estensione di Galois di ${\bf Q}$ e determinare un polinomio di grado 24 il cui campo di spezzamento su ${\bf Q}$ \`e $K$. \vv

\vfill\eject

\item{5.} Determinare il polinomio minimo di $\cos2\pi/15$.
\vfill
%--\item{6.} Descrivere la nozione di campo perfetto dimostrando che i campi finiti
%sono perfetti.

\item{6.} Enunciare i completa generalit\`a il Teorema di corrispondenza di Galois e spiegarne a grandi linee la dimostrazione.\vskip 6cm\bigskip\bigskip\bigskip\vv\vv

\vfil\eject

\item{7.} Dato un campo finito ${\bf F}_{q}$ ($q=p^n$), si consideri $\gamma\in{\bf F}_{q}^*$ e sia  $f_\gamma(X)\in{\bf F}_p[X]$ il 
polinomio minimo di $\gamma$ su ${\bf F}_p$.
\itemitem{a)} Mostrare che se  $m=\deg f_\gamma$, allora $\gamma, \gamma^p, \gamma^{p^2},\ldots, \gamma^{p^{m-1}}$ sono tutte e sole 
le radici di  $f_\gamma(X)$.\medskip

\itemitem{b)} Mostrare che se  $\gamma$ \`e un generatore del gruppo moltiplicativo ${\bf F}_{q}^*$, allora tutte le radici di $f_\gamma$ sono anche generatori.
\vfill

\item{8.} 
\itemitem{a)} Mostrare che per ogni numero razionale $q$, il numero reale $\cos(q\pi)$ \`e algebrico.\hfill{\it Suggerimento: considerare $e^{i\pi q}$}.\medskip
\itemitem{b)} Determinare il polinomio minimo di $3\cos(2\pi/5)+\cos(4\pi/5)$.

\vfill \bye
