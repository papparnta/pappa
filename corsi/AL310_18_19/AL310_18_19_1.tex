\nopagenumbers \font\title=cmti12
% \def\ve{\vfill\eject}
% \def\vv{\vfill}
% \def\vs{\vskip-2cm}
% \def\vss{\vskip10cm}
% \def\vst{\vskip13.3cm}

\def\ve{\bigskip\bigskip}
\def\vv{\bigskip\bigskip}
\def\vs{}
\def\vss{}
\def\vst{\bigskip\bigskip}

\hsize=19.5cm
\vsize=27.58cm
\hoffset=-1.6cm
\voffset=0.5cm
\parskip=-.1cm
\ \vs \hskip -6mm AL310 AA18/19\ (Teoria delle Equazioni)\hfill ESAME
DI MET\`{A} SEMESTRE \hfill Roma, 8 Novembre 2018 \hrule
\bigskip\noindent
{\title COGNOME}\  \dotfill\ {\title NOME}\ \dotfill {\title
MATRICOLA}\ \dotfill\
\smallskip  \noindent
Risolvere il massimo numero di esercizi accompagnando le risposte
con spiegazioni chiare ed essenziali. \it Inserire le risposte
negli spazi predisposti. NON SI ACCETTANO RISPOSTE SCRITTE SU
ALTRI FOGLI.
\rm 1 Esercizio = 4 punti. Tempo previsto: 2 ore. Nessuna domanda
durante la prima ora e durante gli ultimi 20 minuti.
\smallskip
\hrule\smallskip
\centerline{\hskip 6pt\vbox{\tabskip=0pt \offinterlineskip
\def \trl{\noalign{\hrule}}
\halign to277pt{\strut#& \vrule#\tabskip=0.7em plus 1em& \hfil#&
\vrule#& \hfill#\hfil& \vrule#& \hfil#& \vrule#& \hfill#\hfil&
\vrule#& \hfil#& \vrule#& \hfill#\hfil& \vrule#& \hfil#& \vrule#&
\hfill#\hfil& \vrule#& \hfil#& \vrule#& \hfill#\hfil& \vrule#&
\hfil#& \vrule#& \hfill#\hfil& \vrule#& \hfil#& \vrule#& \hfil#&
\vrule#\tabskip=0pt\cr\trl && FIRMA && 1 && 2 && 3 && 4 &&
5 && 6 && 7 && 8  &&  TOT. &\cr\trl && &&   &&
&&     &&   &&     &&   &&   &&    && &\cr &&
\dotfill &&       &&   &&   &&     &&   && && && &&
&\cr\trl }}}
\medskip

\item{1.} Sia $p$ un numero primo, sia ${\bf F}_{p^n}$ un campo finito
con $p^n$ elementi, sia $f\in{\bf F}_p[x]$ e sia $\alpha\in {\bf F}_{p^n}$
una radice di $f$.
\itemitem{a.} Dimostrare che anche $\alpha^p$ \`e una radice di $f$.
\itemitem{b.} Dimostrare che per ogni intero positivo $k$, $\alpha^{p^k}$ \`e una radice di $f$. 
\itemitem{c.} Dimostrare che se $f$ \`e irriducibile e $n=\deg f$, allora $\alpha, \alpha^p,\cdots,\alpha^{p^{n-1}}$ sono tutte distinte.
\itemitem{d.} Dedurre che ogni campo finito con $p^n$ elementi \`e un estensione normale di ${\bf F}_p$.
\vv

%Dimostrare che un estensione finita \`{e} necessariamente algebrica. Produrre
%un esempio di un estensione algebrica non finita.

\item{2.} Dare la definizione di campo algebricamente chiuso e di chiusura algebrica di un campo.\ve\ \vs

\item{3.} Determinare il grado del campo di spezzamento di $(x^3-2)(x^3-5)(x^2+x+1)$ su ${\bf Q}$.\vv

%Dopo aver verificato che \`e algebrico, calcolare
%il polinomio minimo di $\cos \pi/9$ su ${\bf Q}$.

\item{4.} Dimostrare che se $(x,y)\in{\bf C}$ \`e costruibile, allora ${\bf Q}(x,y)/{\bf Q}$ \`e finita e $[{\bf Q}(x,y):{\bf Q}]$ \`e una potenza di $2$.\ve\ \vs

\item{5.} Sia $K={\bf Q}(\sqrt3,\sqrt5)$
\itemitem{a.} Calcolare $[K:{\bf Q}]$ e dimostrare che $K={\bf Q}(\sqrt3+\sqrt5)$
\itemitem{a.} Calcolare il polinomio minimo di $\sqrt3+\sqrt5$ su ${\bf Q}$ e
su ${\bf Q}(\sqrt3)$
\itemitem{a.} Dopo aver mostrato che ${\bf Q}(\sqrt{15})\subseteq K$, descrivere i monomorfismi $K\rightarrow {\bf C}$ che fissano ${\bf Q}(\sqrt{15})$. 
\vv

%--\item{6.} Descrivere la nozione di campo perfetto dimostrando che i campi finiti
%sono perfetti.

\item{6.} Si consideri il campo ciclotomico ${\bf Q}(\zeta_{15})$ ($\zeta_{15}=e^{2\pi/15}$).
\itemitem{a.} Determinare il polinomio minimo di $\zeta_{15}$ su ${\bf Q}$
\itemitem{b.} Determinare il polinomio minimo di $\zeta_{15}$ su ${\bf Q}(\zeta_3)$ e su ${\bf Q}(\zeta_5)$
\itemitem{c.} Determinate gli automorfismi di ${\bf Q}(\zeta_{15})$ che fissano ${\bf Q}(\zeta_3)$\ve\ \vs

\item{7.} Dopo aver verificato che \`e algebrico, calcolare il polinomio minimo di $\cos2\pi/15$ su ${\bf Q}$. \bf(suggerimento, se $\theta=2\pi/15$, considerare il $\cos(5\theta)$ e applicare le formule classiche della trigonometria)\rm 
\vv

\item{8.} Enunciare e dimostrare il teorema del grado (se $K\subseteq L\subseteq M$, allora $[M:K]=[M:L][L:K]$).
\ve

 \bye
