\magnification 1200 \nopagenumbers
\def\frac#1#2{{#1\over#2}}
\def\Q{{\bf Q}}
\def\Z{{\bf Z}}
\def\N{{\bf N}}
\def\C{{\bf C}}
\def\F{{\bf F}}
\def\QQ{{\rm Q}}


\centerline{{\bf Esercizi di Teoria di Galois 3.}}\medskip

\centerline{Roma Tre, 8 Aprile 2003}\bigskip

\item{1.} Sia $E=\Q(\zeta_{13})$. Dimostrare che se $\eta=\zeta_{13}+
\zeta_{13}^3+\zeta_{13}^9$, allora il polinomio minimo $f_\eta$ di
$\eta$ su $\Q$ ha grado $4$. Dopo averne evidenziato le radici, mostrare (calcolando)
che
$$f_\eta(x)=x^4+x^3+2x^2-4x+3.$$
Qual'\`{e} la dimensione del campo di spezzamento di $f_\eta$ su $\Q$?

\  \hfill {\it Suggerimento: Usare il gruppo {\rm Gal}$(\Q(\zeta_{13})/\Q)$ e la corrispondenza
di Galois.}
\bigskip

\item{2.} Dimostrare $\Q(\zeta_p)$ (dove $p>2$ \`{e} primo) ha sempre esattamente un sottocampo
quadratico. Dedurre che ogni campo ciclotomico ammette sempre un sottocampo che \`{e}
un estensione quadratica di $\Q$.

\  \hfill {\it Suggerimento: Usare il gruppo {\rm Gal}$(\Q(\zeta_{p})/\Q)$ e la corrispondenza
di Galois.}
\bigskip

\item{3.} (per che soffre di insonnia) Mostrare la seguente identit\`{a}:
$$\sum_{j=1}^p\left({j\over p}\right)\zeta_{p}^j=\pm\sqrt{(-1)^{(p-1)/2}p}$$
(N.B. $\left({j\over p}\right)$ \`{e} il classico simbolo di Legendre).
Dedurre che ogni campo quadratico \`{e} sempre contenuto in un campo ciclotomico.
\bigskip

\item{4.} Si descrivano tutti i campi intermedi tra $E$ e $\Q$ in
ciascuno dei seguenti casi:
\itemitem{a.} $E=\Q(\zeta_{16})$\hfill {b.} $E=\Q(\zeta_{24})$
\itemitem{c.} $E=\Q_f$ il campo di spezzamento di $x^4-2$\hfill
{d.} $E=\Q(\zeta_{13})$
\itemitem{e.} $E=\Q_f$ il campo di spezzamento di $(x^2-2)(x^2-3)(x^2-5)$.

\  \hfill {\it Suggerimento: Usare la corrispondenza di Galois.}
\bigskip

\item{5.} Per ciascuno degli esercizi del punto 4. si descrivano gli
elementi del gruppo di Galois Gal$(E/F)$.
\bigskip

\item{6.} In cisacuno dei seguenti casi si dica se si tratta di estensioni separabili,
normali o di Galois (nel qual caso descrivere il gruppo di Galois):
\itemitem{{\it i.}} $\F_7(T)/\F_7(T^7)$;\hfill {\it ii.} $\Q(3^{1/5})/\Q$;
\itemitem{{\it iii.}} $\F_{11}(T)/\F_{11}$;\hfill {\it iv.} $\Q(3^{1/5},\zeta_{30})/\Q(\zeta_{30})$;
\itemitem{{\it v.}} $\Q(\sqrt{-1},5^{1/4})/\Q$;\hfill {\it vi.} $\Q(\pi,\sqrt{\pi})/\Q(\pi)$.
\bigskip

\item{7.} Mostrare che,
${\rm Gal}(\Q(\zeta_{n^2})/\Q(\zeta_{n}))\cong \Z/n\Z$
esibendo un isomorfismo esplicito. \hfill {\it (Sugg: considerare $\sigma_j:\zeta_{n^2}\mapsto \zeta_{n^2}^{nj+1}.$)}
\bigskip

 \item{8.} Risolvere i problemi sulle note di Milne a pagina
41. \bigskip

\item{9.} Sia $E\subseteq \C$ un estensione algebrica di $\Q$. Mostrare che esiste un unico sottocampo
$\overline{E}$ di $\C$ contenente $E$tale che:
\item{a.} $\overline{E}/\Q$  e di Galois;
\item{b.} $\overline{E}$ \`{e} contenuto in tutte le estensioni di Galois $L$ di $\Q$ tali che $E\subseteq L\subseteq \C$.

\ \hfill {\it Oss.} $\overline{E}$ si chiama {\it chiusura di Galois di $E$ in $\C$}.
\bye
