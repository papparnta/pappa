\magnification 1200 \nopagenumbers
\def\frac#1#2{{#1\over#2}}
\def\Q{{\bf Q}}
\def\Z{{\bf Z}}
\def\N{{\bf N}}
\def\C{{\bf C}}
\def\F{{\bf F}}
\def\QQ{{\rm Q}}


\centerline{{\bf Esercizi di Teoria di Galois 2.}}\medskip

\centerline{Roma Tre, 17 Marzo 2003}\bigskip

\item{1.} Dimostrare che se $q\in\Q$, allora $\cos(q\pi)$ \`{e} un
numero algebrico. Calcolare anche la dimensione
$$\left[\Q(\cos(q\pi)):\Q\right].$$ Si pu\`{o} dire la stessa cosa
di $\sin(q\pi)$?

\hfill {\it Suggerimento: Utilizzare (senza mostrarlo) il  fatto
che $\left[\Q(\zeta_m):\Q\right]=\varphi(m).$}
\bigskip

\item{2.} In ciascuno dei seguenti casi, determinare la dimensione
del campo di spezzamento del polinomio sul campo assegnato $F$:
\itemitem{a.} $f(x)=x^3$ \hfill $F=\Q$; \itemitem{b.}
$f(x)=(x^2-3)(x^2-27)(x^2-12)$ \hfill $F=\Q(3^{1/3});$
\itemitem{c.} $f(x)=x^8-4$\hfill $F=\Q$; \itemitem{d.}
$f(x)=x^h-3$\hfill $F=\Q(e^{2\pi i/h})$; \itemitem{f.}
$f(x)=x^3+30x+1$\hfill $F=\Q$; \itemitem{g.}
$f(x)=x^{15}+3x^5+1$\hfill $F=\F_5$; \itemitem{h.}
$f(x)=x^4-x^3-4x^2+1$\hfill $F=\Q$.
\itemitem{i.} $f(x)=x^{10}+x+1$\hfill $F=\F_2$. %x^3+x+1\ \ x^7+x^5+x^4+x+1
\bigskip


\item{3.} Descrivere gli $F$--omomorfismi di $E$ in $\C$ in
ciascuno dei seguenti casi: \itemitem{a.} $E=\Q(e^{\pi
i/8})$\hfill $F=\Q(e^{\pi i/2})$;
\smallskip
\itemitem{b.} $E=\Q(\sqrt{2},\sqrt{3},\sqrt{5})$ \hfill
$F=\Q(\sqrt{6})$;
\smallskip
\itemitem{c.} $E=\Q(\zeta_7)$\hfill $F=\Q(\cos2\pi/7)$
\smallskip
\itemitem{d.} $E=\Q(\sqrt{\sqrt{3}+1})$ \hfill $F=\Q(\sqrt{3})$;
\smallskip
Nel prossimo sostituire $\C$ con $\F_{7}(\beta)$,
$\beta^4+\beta+1=0$. \itemitem{e.} $E=\F_{7}(\alpha),
\alpha^4+5\alpha^2+3\alpha+1=0$\hfill $F=\F_7(\sqrt{-2})$.
\bigskip

\item{4.} In ciascuno dei seguenti numeri algebrici, si calcoli il
polinomio minimo?

\itemitem{a.} $e^{2\pi i/33}$; \ \ \ \ \ {b.} $\cos 2\pi/9$; \ \ \
\ \ {c.} $\cos 2\pi/11$; \medskip\itemitem{d.} $\cos 2\pi/13$;
 \ \ \ \ \ {e.} $\cos \pi/5$;
 \ \ \ \ \ {f.} $\sin \pi/7$.
\bigskip

\item{6.} Mostrare che se $f\in F[x]$ \`{e} un polinomio
irriducibile e char$F=p$, allora il campo di spezzamento di $f$ ha
grado $\partial f$.

\bigskip


\item{7.} Si mostri che $\Q(\sqrt{-7})\subseteq
\Q(\zeta_{7})$.

\  \hfil {\it Suggerimento: Considerare il numero
$\zeta_{7}+\zeta_{7}^2-\zeta_{7}^3+\zeta_{7}^4-\zeta_{7}^{5}+\zeta_{7}^6.$}\break
\bigskip

\item{8.} Mostrare che se $n\mid m$, allora $\Q(\zeta_n)\subset
\Q(\zeta_m).$\bigskip

 \item{9.} Risolvere i problemi sulle note di Milne a pagina
29 e 30. \vfill

\bye

\eject

\centerline{\it Alcune Risposte (Esercizi di Teoria di Galois 2. del 17/3/03):}
