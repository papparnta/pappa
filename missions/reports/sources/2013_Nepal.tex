\documentclass[12pt]{scrartcl}
\usepackage{amsfonts,amsmath}
\usepackage[utf8]{inputenc}
\usepackage[cm]{fullpage}
\newcommand{\Q}{\mathbb Q}
\newcommand{\Z}{\mathbb Z}
\newcommand{\C}{\mathbb C}
\renewcommand{\familydefault}{\sfdefault}
\title{Kathmandu (Nepal)\\ Report on the visit by}
\author{Francesco Pappalardi}
\date{December 14--20, 2013}
\begin{document}
\maketitle

I arrived in Kathmandu in the afternoon of December 14. On Sunday 15 and Monday 16, I was busy for private reasons as 
I attended the weeding of Nilakantha Paudel which was held in Dumre, 150 km from the capital. I returned to Kathmandu on Monday night.

\begin{enumerate}
\item On Sunday morning I had a meeting with the Vice-Chancellor of Tribhuvan University Professor Hira Bahadur Maharjan. 
We discussed the old project of signing a M.O.U. between TU and Roma Tre. He said that he was ready to sign the document during my visit in Nepal. 
So I wrote to the secretary of my rector and she sent me the M.O.U. immediately. Finally the TU V.C. signed it before my departure and I could bring 
it back to Italy. This document is important since it is one of the prerequisites for future collaborations. More precisely, now that an M.O.U. 
is signed, applications for credit transfers, teachers mobility, co--tutelages can be submitted. The V.C. also mentioned that there is an ongoing 
collaboration for Chemistry (if I remember correctly) with a Norwegian institution which also contemplates credits transfer and mobility 
of various kinds. I asked a copy of the agreement that was signed but there was not enough time to find them.

During the meeting with the V.C., he said that the most important need to the mathematics at TU would be to have basic courses for the master 
level students. I reminded of the organization of a \textbf{SPIM} program in Nepal and we all regretted that the project never took off.

\item On Wednesday morning December 18, I gave a seminar at the College of Information Technology \& Engineering (\textbf{CIT}). This is the first contact with 
this institution which is located in Kathmandu not far from the airport. \textbf{CIT} is associated to Purbanchal University, a public university situated in 
Biratnagar, the economic centre of Nepal. Nilakantha Paudel is a teacher at \textbf{CIT}. Although \textbf{CIT} does not offer any degree in mathematics, mathematics courses of various 
type are offered to its students of Engineering and Computer Sciences. I had lunch with various young faculties of \textbf{CIT}, among them there were the vice principal 
Chandra Bilash Bhurtel, the principal Madhur Singh and the vice director Sudhir Guragain. During the meal I mention the goal of organizing 
mathematical activities 
that could be reproduced every year and that could attract young students from various institutions. 
I said that, initially, the activities could be directed to students, 
not necessarily studying mathematics as a major, as long as they 
could be selected on the basis of their interests, their youth and avoiding a bias towards elder participants. The reaction looked 
positive and we agreed that I would keep on discussing the matter with Nilakantha Paudel that was also attending the lunch.

Later I had a chance to discuss the matter further with Nilakantha. The discussion was quite interesting. He mentioned the following aspects
\begin{enumerate}
\item The organization of a una--tantum event would not be a problem at \textbf{CIT}. However making a regular meeting would not be as easy.
\item Any Nepalese institution (including \textbf{CIT}) that would organize an event of the type \textbf{NSNTC}, \textbf{SPIM} or a CIMPA school, would 
give priorities to the participation to elder people (e.g. teachers) usually connected to it. 
\item There are two Computer Science organizations, namely the \textbf{Information Technology Society Nepal} (\texttt{http://itsn.org.np/}) and 
\textbf{Computer Association Nepal computer Association Nepal} (\texttt{http://www.can.org.np/}). 
They work dynamically and harmonically with each other and he has contacts with the first one. 
They would be the a good local partner to organize instructional activities in mathematics if they are oriented towards the applications to
Computer Science. I guess that in future visit to Nepal, there should be efforts to contact these organizations. 
\end{enumerate}
  

\item On Wednesday December 18, I had a meeting with Professor Bhadra Man Tuladhar. I had asked to meet him and Ajaya arranged the appointment. 
During the meeting I asked him to promote the organization at KU of a second \textbf{NSNTC} (National School in Number Theory and Cryptography)
which in my opinion had been a very successful event. During the meeting he has introduced me to Associate Professor D.L. Bahadur Gurung 
(\texttt{db\_gurung}@\texttt{ku.edu.np}), specialized in Mathematical Biology. They both said that they will pursue the project of a second
\textbf{NSNTC}, maybe in October 2016. They also showed interest in the possibility of presenting the project for a CIMPA school in applied 
mathematics. During the meeting they illustrated an ongoing project called \textbf{International Forum for Mathematical Modelling} 
(\texttt{http://science.cmb.ac.lk/ifmm/}). This project involves KU, the Colombo University (Sri Lanka) and the 
Institute of Technology Bandung (Indonesia) and it is supervised by the University Kaiserlautern Germany. I found the project interesting
and worth being studied further.

\item There were regrettable consequences to the Manila school of last July. 
During my visit in Kathmandu I have been told for the second time (the first time was by a Pakistani participant, immediately after the end of 
the CIMPA-ICTP school of last summer in Manila) that the funds that were given to the international participants to pay for their suppers were 
heavily insufficient. 
Some students had a really hard time in Manila and they even had to pay for the transportation back to the airport that was charged a lot more 
than expected. I take my share of responsibility for the event and I wish that this does not happen again. The voice that CIMPA promises 
local support and then does not honour its commitment has already spread in Nepal and I cannot exclude that possible participants to future 
schools would refrain to apply simply because they cannot afford to support themselves in foreign countries for two weeks. 

\item Regarding the November CIMPA school on Aspects of Dynamical Systems, together with Ajaya Singh, 
we prepared a realistic budget. Most of the logistic services (meals, housing for teachers and paying participants) will be handle by the 
Kirtipur Hill Side Hotel. 
Even the lecture hall will be inside the premises of the Hotel. I was initially negative on the matter as I felt that the lectures
should take place inside the University. However Ajaya explained that there are two counter-indications in holding all the 
activities inside TU. The first is the difficulty guaranteeing power supply for the projector (I had to stop my lecture there when
the power went off at 11AM) and the second is the fact that, if meals are served inside the University, there would be no way to
prevent several workers of TU from eating at the participant's buffet. This would force to double the number of lunches and would have 
a significant negative impact on the budget. On the other hand I suggested that some of the lectures could be held at TU and some
at the Kirtipur Hill Side Hotel. Ajaya will look into this possibility. When we organized the 2011 CIMPA school in KU, the identical problem
was mentioned by Kanhaya Jha. Furthermore it should be taken into account that the price of food is not cheap in Nepal. I also mentioned
to Ajaya the need that all participants (Nepalese and International) would have at least their lunch together. This need is in conflict
with the problem that Nepalese participants expect to pay much less for their lunch (maximum 200NPR) while the Hotel has offered a buffet
meal for approx 500NPR. Also this matter will be dealt with by Ajaya. The manager director of the Kirtipur Hill Side Hotel, Mr Vijaya Kumar 
Maharjan, is very helpful and flexible. He said that Kirtipur Hill Side Hotel may accept payments by International Bank Transfer and he can produce 
proforma invoices when ever they are needed. He has also offered help to act as inter mediator with other local companies 
(e.g. Hotel for international participants, transportation etc.). Other CIMPA schools have used travel agencies to deal with local
logistics so I feel that it would be no harm in using Mr Vijaya Kumar Maharjan services. 

My last comment on the issue is that it would be very important if the budget produced by Ajaya was studied, emended and approved
by the organisers of the November CIMPA school. For this and other matters he needs and deserves active feedbacks and collaboration.

\item I had the chance to meet students that would like to apply for fellowships for PhD studies in Europe. They were Bhimsen Khadka and Harish Chandra Bhandari.
Both gave me a good impression. I also met briefly Manoj Gyawali and Dhruba Bahadur Thapa. 
Ajaya will provide important comments that we can use to help them.
\bigskip

\vfill Rome, Jan 3, 2014\hfill Francesco Pappalardi
\end{enumerate}
\end{document}

