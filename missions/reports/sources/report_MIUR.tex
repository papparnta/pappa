\documentclass[12pt,a4paper]{letter}
\usepackage{wallpaper}
\usepackage{hyperref}

\signature{Francesco Pappalardi\\
\includegraphics[width=4cm]{franfirm.pdf}}

\begin{document}
\ThisURCornerWallPaper{1}{CIroma3.pdf}

\begin{letter}

\centerline{\textbf{RELAZIONE SULLA MISSIONE SVOLTA IN VIETNAM}}
\centerline{\textit{ DAL 30 Agosto al 10 Settembre 2015}}

La missione ha fatto parte di un'attivit\`a pi\`u ampia che ha avuto luogo in Asia nell'estate del 2015 con il seguente piano 
di lavoro:

\begin{itemize}
 \item[27/08 - 30/08] 
\textit{Royal University of Phnom Penh.} Visita accademica agli studenti del Master in Matematica.
\item[31/08 - 09/09] 
\textbf{Department of Computer Science - University of Science, Ho Chi Minh.}
Partecipazione alla \textit{South East Asian Mathematical Society (SEAMS) School 2015 
"Number Theory and Applications in Cryptography and Coding Theory"}
\item[5/09 - 6/09]  
\textbf{Can Tho University.}
Membro della commissione mista per l'attivazione di un accordo di collaborazione tra Can Tho University (Vietnam) e Royal University of Phnom Penh.
\item[10/08 - 13/08] 
\textit{Academy of Mathematics and System Sciences a Pechino.}
Visita accademica e presentazione seminario di ricerca.
\item[13/08 - 18/08]
\textit{National University of Mongolia a Ulan Bator - Mongolia.} Mini corso sulla \emph{crittografia delle curve ellittiche} e 
partecipazione all' incontro della \textit{Mongolian Mathematical Society}. 
 \end{itemize}

Il presente documento pu\`o essere trovato in

\small{\texttt{http://www.mat.uniroma3.it/users/pappa/missions/report$\_$MIUR.pdf}}
 
Segue la relazione sulle attivit\`a pertinenti alla parte delle missione svolta in Vietnam:\bigskip

\centerline{\Large{\textbf{University of Science, Ho Chi Minh}}}

La mia attivit\`a di insegnamento \`e stata svolta nell'ambito della
\textit{South East Asian Mathematical Society (SEAMS) School 2015 
''Number Theory and Applications in Cryptography and Coding Theory''}.

Uno degli principali obiettivi della scuola \`e stato quello di fornire una preparazione preliminare
e di base in Teoria dei Numeri e Crittografia a un gruppo di studenti provenienti sia dal Vietnam
che da altre nazioni dalla Regione sud--est asiatica.

Tali studenti sono potenzialmente interessati  a partecipare alla futura \textit{CIMPA-ICTP-
VIETNAM - Research School on Lattices and application to cryptography and
coding theory} che si terr\`a presso la University of Pedagogy a Ho Chi Minh City dall'1 al 12 Agosto 2016.

Durante la scuola appena terminata sono stati impartiti i seguenti corsi di circa 10 ore ciascuno:
\begin{enumerate}
 \item Introduzione alla teoria dei codici, (Michel Waldschmidt).
 \item Crittografia delle Curve ellittiche, (Francesco Pappalardi).
 \item RSA e sue varianti (Thuc D. Nguyen, Long D. Tran e Thu D. Tran).
 \item  Reticoli e Applicazioni, (Dung H. Duong, Khuong A. Nguyen, Ha Tran).
\end{enumerate}

Esiste ancora una pagina attiva dove possono essere trovate informazioni dettagliate.\\
\small{\texttt{https://www.math.uni-bielefeld.de/~dhoang/seams15/}}

Le lezioni del corso da me impartito sono cominciate Luned\`\i\ 31 Agosto, 2015. Le trasparenze sono disponibili in\\
\hspace*{-.5cm}\small{\texttt{http://www.mat.uniroma3.it/users/pappa/missions/slides/HCMC$\_$2015$\_$1.pdf}}\\
\hspace*{-.5cm}\small{\texttt{http://www.mat.uniroma3.it/users/pappa/missions/slides/HCMC$\_$2015$\_$2.pdf}}\\
\hspace*{-.5cm}\small{\texttt{http://www.mat.uniroma3.it/users/pappa/missions/slides/HCMC$\_$2015$\_$3.pdf}}\\
\hspace*{-.5cm}\small{\texttt{http://www.mat.uniroma3.it/users/pappa/missions/slides/HCMC$\_$2015$\_$4.pdf}}\\
\hspace*{-.5cm}\small{\texttt{http://www.mat.uniroma3.it/users/pappa/missions/slides/HCMC$\_$2015$\_$5.pdf}}

Il numero di docenti coinvolti nella scuola \`e stato pari a 8:
\begin{itemize}
\item sei docenti vietnamiti\vspace*{-2mm}
\begin{itemize} \item 1 da \textit{Hue}\vspace*{-2mm}
\item 5 da \textit{Ho Chi Minh City}\vspace*{-2mm}
\end{itemize}
\item due relatori europei (Michel Waldschmidt dell'Universit\`a Pierre e Marie Curie e il sottoscritto)\vspace*{-2mm}
\end{itemize}

Il numero totale degli studenti partecipanti \`e stato 74:
\begin{itemize}
 \item \textit{Vietnam} (32) \vspace*{-2mm}
 \begin{itemize}
 \item \textit{Lat} (2)\vspace*{-2mm}
 \item \textit{Dong Nai} (2)\vspace*{-2mm}
 \item \textit{Ha Noi} (11)\vspace*{-2mm}
 \item \textit{Sai Gon}) (17)\vspace*{-2mm}
 \end{itemize}
\item \textit{Nepal} (11)\vspace*{-2mm}
\item \textit{Thailandia} (7)\vspace*{-2mm}
\item \textit{Indonesia} (6)\vspace*{-2mm}
\item \textit{Filippine} (5)\vspace*{-2mm}
\item \textit{Cambogia} (3)\vspace*{-2mm}
\item \textit{Malesia} (2)\vspace*{-2mm}
\end{itemize}

\URCornerWallPaper{1}{CIroma3b.pdf}

%\textbf{Il Progetto della International Mathematical Union (IMU) per i matematici Nepalesi.}

La cerimonia di apertura si \`e svolta la mattina del secondo giorno, il primo settembre 2015. 
A presiedere la cerimonia sono stati invitati il Console Generale d'Italia a Ho Chi Minh, \textbf{Carlotta Colli}
e il Console Generale di Francia a Ho Chi Minh, 
\textbf{Emmanuel Ly-Batalla}. Entrambi hanno salutato i partecipanti augurando buon lavoro a tutti. 

E' importante ricordare che i due principali organizzatori, Dr. \textbf{Ha Tran} e Dr. \textbf{Dung H. Duong} 
sono entrambi ex alunni di universit\`a Italiane
dove hanno recentemente difeso tesi di dottorato in Teoria dei Numeri.

La presenza del Console Colli, a cui va un sincero ringraziamento, ha confermato il forte sostegno proveniente dall'Italia 
alla cooperazione nel campo della matematica con il Vietnam. Tale sostegno si concretizza anche nel fatto che
cinque tra i nove docenti del 2016 CIMPA Scuola verranno dall'Italia.

La scuola SEAMS ha anche avuto diversi effetti collaterali positivi. 
La \textit{commissione per i paesi in via di sviluppo (CDC)} dell'\textit{Unione Matematica internazionale (IMU)} 
ha finanziato completamente la partecipazione alla scuola SEAMS di undici leader matematici (di cui tre donne) 
dal Nepal.

Informazioni supplementari su tale sottoprogetto possono essere trovate in:\\
\small{\texttt{http://www.mathunion.org/cdc/grants/project-support/project-support-2015/nepal/}}

La partecipazione alla scuola SEAMS ha permesso loro di riorganizzarsi, fare piani per
futuro e anche di condividere le esigenze della
comunit\`a matematica  Nepalese con colleghi di altri paesi della Regione sud-est asiatica. 

Ci\`o ha facilitato l'attivazione di nuovi contatti A la valutazione di possibili progetti 
per il supporto regionale e internazionale dopo il terremoto che ha colpito il Nepal di recente.

Ritratti di tutti i partecipanti alla scuola possono essere trovate qui:

\small{\texttt{http://www.mat.uniroma3.it/users/pappa/missions/albums/SEAMS2015/}}

Sar\`a redatta una relazione separata sulla partecipazione della delegazione nepalese ed inoltrato all CDC dell'IMU.

Un ulteriore effetto collaterale positivo della scuola SEAMS \`e stata la possibilit\`a
per i matematici vietnamiti da Hanoi e Ho Chi Minh City di avere
scambi utili riguardanti possibili programmi futuri di cooperazione.

I principali organizzatori di questa Scuola SEAMS, Dr. Ha Tran e Dr. Dung H. Duong,
hanno fatto un lavoro superbo. La loro dedizione \`e stata l'ingrediente prinicipale per il successo indiscusso
di questa iniziativa. Ha e Dung hanno anche favorito la creazione di un ambiente accogliente. Tutti i partecipanti
conserveranno un caro ricordo di questa esperienza.\bigskip

\centerline{\Large{\textbf{Can Tho University}}}

Il soggiorno vietnamita ha fornito l'opportunit\`a di gettare le basi per un accordo
bilaterale tra 
l'\textit{Universit\`a di Cantho} in Vietnam e 
la \textit{Royal University of Phnom Penh} in Cambogia.

Nel mese di maggio 2015, il Dr. Nguyen Trung Kien dell'Universit\`a di Cantho ha visitato Phnom Penh e 
ha stabilito i primi contatti. In seguito ha invitato i tre partecipanti cambogiani alla Scuola SEMAS (Sok Lin, Mam Mareth
e Say Ol), 
Michel Waldschmidt, il sottoscritto e Mai Hoang Bien dell'\textit{Universit\`a nazionale di Ho Chi Minh}, 
a visitare la sua universit\`a durante il week-end della Scuola SEAMS.

Nguyen Trung Kien ha organizzato un seminario il 5 settembre durante il quale io ho tenuto una conferenza su 
un'introduzione l'Ipotesi di Riemann per i principianti, mentre Michel Waldschmidt ne ha presentato uno 
sulla equazione di Brahmagupta-Fermat-Pell. Nguyen Trung Kien anche organizzato una cena il Sabato sera e 
la visita al Delta del Mekong Domenica mattina.

Siamo grati a lui per il suo invito e la sua calda accoglienza. 
Abbiamo apprezzato anche durante questo fine settimana il gentile aiuto di Mai Hoang Bien, 
un collega di Bui Xuan Hai dell'Universit\`a nazionale di Ho Chi Minh.

\closing{Roma, 30 Settembre 2015}

\ps

\vfill \newpage

\centerline{\textbf{Foto di gruppo}}



\includegraphics[width=13cm]{foto1.jpg}


\includegraphics[width=13cm]{foto2.jpg}
\end{letter}
\end{document}
