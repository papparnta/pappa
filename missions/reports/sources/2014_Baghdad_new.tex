\documentclass[12pt,a4paper]{scrartcl}
\usepackage{amsfonts,amsmath,textcomp,enumerate}
\usepackage{graphicx}
%\usepackage{showframe}
\usepackage[utf8]{inputenc}
%\usepackage[cm]{fullpage}
\usepackage{wallpaper}
\textwidth = 533pt
\textheight = 702pt
\hoffset = 0pt
\footskip = 18pt
\marginparwidth = 15pt
\oddsidemargin = -35pt
\evensidemargin = -35pt
\marginparsep = 10pt
\newcommand{\Q}{\mathbb Q}
\newcommand{\Z}{\mathbb Z}
\newcommand{\C}{\mathbb C}
\renewcommand{\thefootnote}{\fnsymbol{footnote}}
\renewcommand{\familydefault}{\sfdefault}
\title{Baghdad (Iraq)\\ Report on the visit by}
\author{Francesco Pappalardi}
\date{March 29 - April 1, 2014}
\begin{document}
\ThisURCornerWallPaper{1}{CIroma3.pdf}
\maketitle


\section{Preamble}

I was invited to visit the College of Sciences for Women last October 2013 by 
Professor \textbf{Saad Naji}, Dr. \textbf{Hussain Ali Mohamad} and 
Professor \textbf{Iden Hasan Al-Kinany} that I met in Basrah
while I was attending the ``\textsl{Second International Conference 
of Mathematics and its Applications}''\footnote{see report \texttt{http://www.mat.uniroma3.it/users/pappa/missions/2013\_Basrah\_ICMA.pdf}}. 
 I accepted their kind invitation and agreed to visit Baghdad in early 2014. After I returned to 
Rome from Basrah in late 2013, I learned  that in that moment there were some ongoing contacts between my University (Roma3) and the 
Embassy of The Republic of Iraq in Italy. Professor \textbf{Manahil Ahmed Ali AL-Nawas}, Cultural Attach\'e and Professor \textbf{Mario Panizza}, Rector of
Roma3,
had just signed an agreement to plan future Cultural, Educational and Scientific cooperation.  

The agreement is being finalized in these days and will involve exchange of students (at the level of Masters and Doctorate) and of scientific staff.  
Certainly, once established, mathematics could dramatically benefit from it. 
It was Dr. \textbf{Claudia Stamini}
from the Ufficio Relazioni Internazionali that established the contact between Professor Manahel Al-Nawas, that I met at the Embassy
in Rome on March 18$^{\textrm{th}}$, and myself. During the meeting she was highly supportive, she gave me a variety of useful information and explained that her
origin home institution is the University of Baghdad itself where she is a Professor of English Literature at the College of Education for Women, next door with respect
to the College of Science for Women. I would like to express my gratitude to her, her staff and to the Ambassador of Iraq in Italy, His
Excellence Dr. \textbf{Saywan Sabir Mustafa Barzani} for the 
precious support.

\section{The visit}

I arrived in Baghdad in the early morning of March 30$^{\textrm{th}}$ from Istanbul. Dr. {Hussain Ali Mohamad} pick me up and drove me 
directly to the Department where I spent a few hours meeting some of the members of the Department of Mathematics (DM) of the 
College of Science for Women (CSW) of the University of Baghdad (UB). The Head of Department, Dr. \textbf{Na'mh Abdulla Abid} 
explained me what were the expectation from my visit:
\begin{enumerate}
 \item To report in details on the organization of the undergraduate program and the master program in my home university. Dr. Na'mh lamented that
 between 1991 and 2004 the Iraqi mathematical community, and in particular the DM of CSW, suffered isolation and within the DM there is a growing 
common feeling that a twin--ship
 with a European institution could help to review its curricula and possibly lead to improvements. I replied that I believed that my department, was, 
 on principle, interested in the possibility of establishing such a cooperation, that I would have reported on that request and that I would
 have worked on a follow up of that meeting. 

I had brought with me 10 copies of the booklet ``Benvenuto a Matematica'' which can
 be found in \texttt{http://www.mat.uniroma3.it/scuola\_orientamento/benvenuto.shtml}. I distributed them and, although in Italian, they provided
 a good basis for the informal seminar that I gave during the the afternoon where I explained to the members of DM some of the details of the
 organization of the Bachelor program (Laurea Triennale) in my university. 
 \item To meet Dr. \textbf{Saad Abbod Badday}, Number Theorist, graduate from the University of Glasgow. 
 Professor Saad Abbod is supervising
 a Master Thesis on Analytic Number Theory regarding analytic properties of the Divisors Function. He needed to access some of the more recent
 references. I tried to help him to download some of the reviews of recent publications and we agreed that we will get in touch soon to exchange more
 information.
 \item To meet two former students that have completed their Masters Thesis and that are now looking at the possibility to complete a PhD thesis abroad.
 They are:
 \begin{itemize}\itemsep2pt
  \item[-] \textbf{Ms. Najlaa Issa Tawfiq} (Birth 20/10/1976) \texttt{<njlaaaldaloo@yahoo.com>}. She has already an approved fellowship for a PhD funded by the Iraq
  Minister of Higher Education and Scientific Research. 
  \item[-] \textbf{Ms. Nagham Mousa Neamah Al-Behadili} (Birth 15/11/1975) \texttt{<naghan57@yahoo.com>}. 
  She is at an advanced stage of the process of application for a PhD
  fellowship from the Minister of Higher Education and Scientific Research.
 \end{itemize}
I collected their CV, their research projects and I committed myself to verify the possibility to find a supervisor for them in my Department. One of the problem
is due to the fact that the topic of their PhD thesis seems to be already decided with a negotiation with the DM of CWS. One of the topics is
Dynamical Systems which is covered by the research topics in my Department but the second is Fractional Calculus on which I am not aware of the 
competence that my Department can offer. However I suggested to have an open attitude toward Mathematics avoiding a rigid attitude regarding the thesis topic.
I wish that it would be possible to have them both in Rome to start a PhD program starting next October 2014. I will work to accomplish this project.

 \item To present three seminar in Number Theory.

 \item To address a group of faculties from various Department of the CSW and also from other Baghdad Institution on the nature of the Italian
 university system with emphasis to Roma3 and to the Mathematics programs. 
\end{enumerate}

\subsection{Sunday March 30$^{\textrm{th}}$}
I spent the day of Sunday discussing and exchanging information with the colleagues of the Department of Mathematics. Everybody there knew Professor Manahel
Ahmed Ali AL-Nawas, being her a colleague from the College of Education for Women which is located next-door.  The two Colleges, of Science for Women and of Education for Women
used to be united and have recently been separated.

Among other discussions, Dr. \textbf{Radhi Ibrahim Mohammed Ali}, Vice Head of Department, gave me an overview of the research specialities if the DM. They range over
a variety of topics: Mathematical Statistics, Dynamical Systems, Number Theory, Point Set Topology, Numerical Analysis, Commutative Algebra, Probability Theory, Fuzzy Mathematics,
Fractional Calculus,$\ldots$. Some of these topics are very popular also in Europe, some are less popular and, maybe, internationally less diffused. 
 
\URCornerWallPaper{1}{CIroma3b.pdf}
\subsection{Monday March 31$^{\textrm{st}}$}

On Monday I gave two lectures, one introductory and one regarding applications of Number Theory to Cryptography. I believe that they were both
well received. The program for my days in Baghdad was produced by Professor Saad Naji that has been my correspondent during the
preparation of the mission. The slides for my talks can be found in\hfill \\
\texttt{http://www.mat.uniroma3.it/users/pappa/missions/slides/RH\_Baghdad\_2014.pdf} and in

\noindent \texttt{http://www.mat.uniroma3.it/users/pappa/missions/slides/RSA\_31\_03\_2014.pdf}.

I also had a chance to meet two colleagues from the University of technology,
Applied Sciences/Applied Mathematics. They were Professor \textbf{Nadia Alsaid} %\texttt{<nadiamg08@gmail.com>}
and  Professor \textbf{Shatha Assaad Salman Al-Najjar}. %\texttt{<shatha.alnajar@yahoo.com, drshatha.alnajar@gamil.com>}. 
Professor Nadia Alsaid invited me to visit her Department. Unfortunately my visit was too short to be able to accept the 
invitation.


\subsection{Tuesday April 1$^{\textrm{st}}$}

In the morning the Department Head Dr. Nh'am took me to meet the Dean of the College of Science for Women, Professor \textbf{Dr. Ahlam M. Farhan}.
We had a pleasant and welcoming meeting. 

Later I gave two more lectures. The first regarding advanced topics in Number Theory and the second, open to a wider audience, regarding
the Italian university system and Roma3 in particular. I also gave a brief account of previous projects where Roma3 was involved in Iraq. 
Finally I also gave some information on the Doctorate program in my Department. Some of the slides can be found in  \hfill\\
\texttt{http://www.mat.uniroma3.it/users/pappa/missions/slides/2\_iraq\_and\_roma3.pdf} and in

\noindent \texttt{http://www.mat.uniroma3.it/users/pappa/missions/slides/3\_phd\_Roma\_tre.pdf}.

I also had the surprise to meet Dr. \textbf{Haytham Al-Ubadat} whom I had met about 10 years ago when he was a doctoral student
at the Universit\`a di Roma Tor Vergata.
%Ciclo XVIII - Academic Year 2002/2003  


\section{Final Remarks}

I only have good thing to say regarding this visit at CSW. The hospitality of Iraqis has been memorable. From a security point of view, I can
say that I always felt safe and all possible caution was taken to assure that my time was relaxed. I was taken to the airport by two officers 
from the minister that were authorized to escort me all the way to the check--in counter. 

All my discussions were based on a serious and sincere will to exchange questions, information and to plan future cooperation 
with the goal to disrupt the isolation of the Iraqi mathematical community. 
I wish that new branches of research can be introduced in Iraq. The mathematical community of CSW looked 
cohesive, enthusiastic, scientifically diversified, united, harmonious, competent and organized. If given the appropriate tools, I believe that
it can improve
rapidly its standard and reach an international level. 

I wish to be able to address some of the proposals that came out during the meetings. In particular I believe that the two goals:  
\begin{itemize}
 \item[\textborn] to receive two Doctorate students in Roma3 and 
 \item[\textborn] to start a pilot program to host Iraqi Masters student to write their thesis in Roma3 on a co--supervision scheme
\end{itemize}
are the first ones. For that I hope that both University and other institutions will be able to provide support.
\bigskip
\vfill

Rome, April 15$^{\textrm{th}}$, 2014\hfill Francesco Pappalardi
 \newpage

\section{Some of my Encounters from CSW}

During my visit of the School of Science for Women I met the following colleagues:
\begin{enumerate}
\item \textsl{Dr. Na'mh Abdulla Abid}, Head of Department \texttt{<namh\_abed@yahoo.com>} - Differential Equations;
\item \textsl{Dr. Radhi Ibrahim Mohammed Ali}, Vice Head of Department, \texttt{<Radhi52@yahoo.com>} Functional Analysis; 
\item \textsl{Dr. Saad Abbod Badday}, \texttt{<m\_alsaedi870@yahoo.com>} -  Number Theory; 
\item \textsl{Prof. Saad Naji Al-Azzawi\footnote{One of the three colleagues that I met in Basrah in October 2013 and has invited me}}, 
\texttt{<saad\_naji2007@yahoo.com>} - Fractional Differential Equations; 
\item \textsl{Dr. Hussain Ali Mohamad$^\dag$}, \texttt{<hu\_moha@yahoo.com>} - Delay Differential Equation;
\item \textsl{Prof. Iden Hasan Al-Kinany$^\dag$}, \texttt{<idenalkanany@yahoo.com>} - Mathematical Statistics;
\item \textsl{Dr. Ms. Muna Abbas Ahmad}, Commutative Algebra;
\item \textsl{Dr. Ms. Muna Mansoor Mustafa}, \texttt{<munamm05@gmail.com>} - Numerical Analysis;   
\item \textsl{Dr. Ah-Bayati Jelal Hatem}, Topology.
\end{enumerate}

\section{An incomplete list of mathematical specialities of the CSW}

\begin{enumerate}[I]
\item Analytic Number Theory; 
\item Fractional Differential Equations; 
\item Functional Analysis;
\item Fuzzy Linear Programming; 
\item Integral Equations;
\item Mathematical Statistics;
\item Module Theory;
\item Numerical Analysis;   
\item Oscillation and Delay Differential Equation;
\item Point set Topology.
\end{enumerate}
\vfill \newpage

\section{Some Photos in chronological order}



\includegraphics[width=6.5cm]{2014-03-30_1.jpg}\hspace{2.1cm}\begin{minipage}[t]{9cm}
\includegraphics[width=9cm]{2014-03-31_2.jpg}
                                          \end{minipage}
                                          
\small{\hspace*{.50cm} Iden, FP, Na'mh and Hussain\hspace{3.6cm} Radhi, Saad Naji, FP, Na'mh and Saad Abbod}\bigskip

\includegraphics[width=8.5cm]{2014-04-01_3.jpg}\hspace{.5cm}
\includegraphics[width=8.5cm]{2014-04-01_4.jpg}

\small{\hspace{1.9cm} Na'mh, FP and the Dean\hspace{5cm} Shatha, FP and Nadia}\bigskip

\includegraphics[width=8.5cm]{2014-04-01_5.jpg}\hspace{.5cm}
\includegraphics[width=8.5cm]{2014-04-01_6.jpg}

\small{\hspace{0.9cm} The College of Science for Women\hspace{3cm}  Eman Noori, Muna Abbas and FP}
\end{document}

