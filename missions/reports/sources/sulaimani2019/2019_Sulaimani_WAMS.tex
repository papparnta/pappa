\documentclass[12pt,a4paper]{scrartcl}
\usepackage{amsfonts,amsmath,textcomp,enumerate}
\usepackage{graphicx}
%\usepackage{showframe}
\usepackage[utf8]{inputenc}
%\usepackage[cm]{fullpage}
\usepackage{wallpaper}
\usepackage{lmodern}% http://ctan.org/pkg/lm
\textwidth = 533pt
\textheight = 702pt
\hoffset = 0pt
\footskip = 18pt
\marginparwidth = 15pt
\oddsidemargin = -35pt
\evensidemargin = -35pt
\marginparsep = 10pt
\newcommand{\Q}{\mathbb Q}
\newcommand{\Z}{\mathbb Z}
\newcommand{\C}{\mathbb C}
\renewcommand{\thefootnote}{\fnsymbol{footnote}}
\renewcommand{\familydefault}{\sfdefault}
\title{%Sulaimani, Kurdistan (Iraq)\\ 
Report on the WAMS school on
Representation Theory
College of Science, University of Sulaymaniyah, Sulaymaniyah/ Kurdistan Region/ Iraq}
\author{Francesco Pappalardi}
\date{\today}
\begin{document}
\ThisURCornerWallPaper{1}{CIroma3.pdf}
\maketitle

\section{Preamble}

This report covers three activities:
\begin{itemize}
 \item \textbf{Second International Conference of Mathematics (SICME-Erbil2019)}\\
Salahaddin University - Erbil (February 3 - 5, 2019)\\
\texttt{http://su.edu.krd/sicme2018/}
\item  Pre-school for the WAMS school on \textbf{Representation Theory},\\
 College of Science, University of Sulaimani (January 17-19, 2019) \\
\texttt{http://www.rnta.eu/Sulaimani2019/preschool.html} 
 \item WAMS school on
\textbf{Representation Theory},\\
College of Science, University of Sulaymaniyah (February 7 - 9, 2019)\\
\texttt{http://www.rnta.eu/Sulaimani2019/}
\end{itemize}

More detailed information regarding every single event can be found in the appropriate web pages above. This report can be downloaded at

\texttt{http://www.mat.uniroma3.it/users/pappa/missions/reports/2019\_Sulaimani\_WAMS.pdf}

The events reported here are to be considered as  ``followups'' of the WAMS research school Topics in algebraic number theory and Diophantine approximation that took place in Erbil in March 2017

\texttt{http://science.su.edu.krd/index.php/\\ \hspace*{4cm}\hfil news-and-events/302-proposal-for-a-wams-school-in-erbil}

\section{SICME-Erbil2019}

SICME took place at Salahaddin University. 
An International Scientific committee and a remarkable work by the Kurdish colleagues that formed an extremely efficient organizing committee, contributed to a very successful event.

There were 2 Invited speakers from Iran, 3 from France, 2 from Italy and 1 from Iraq.
The opening ceremony was attended by the French Consul General 
Dominique Mas and by the President of Salahaddin University Professor Doctor Ahmed Enwer Dezaye.
Also the Italian Consul, Mrs Serena MURONI, was invited but she was not present.

During the days of the conference there were two fruitful meetings:


\begin{itemize}
 \item The first was at the Minister of Education of the Kurdish Government with the 
 Honorable Minister of Education Pishtiwan Sadiq. During this meeting some proposals and plans for future collaboration  of Kurdish Institutions (including high schools) and foreign mathematicians. At the meeting were present some of the International participants to SICME2019 from France, Iran and Italy, including the Director of CIMPA Ludovic Rifford and myself.
\item The second was at Salahaddin University with the President of the University Prof. Dr. Ahmed Enwer Dezaye. The meeting was more concentrated on the interests of Salahaddin University and possibility of future actions including the recent application for a collaboration between Salahaddin University and the Universit\`a Roma TRE under the umbrella of Erasmus+ program. At the meeting were present some of the International participants to SICME2019 from France, Jordan and Italy, including Ludovic Rifford and myself. Some ideas regarding a future action with Jordan were proposed. One of the possibility is the  proposal of a WAMS school in Numerical Analysis.
\end{itemize}

At both meeting the mathematicians were lead by the Dean of the faculty of Sciences Prof. Dr. Herish Omar Abdullah who was also the main organizer of SICME2019.\bigskip
\URCornerWallPaper{1}{CIroma3b.pdf}

\centerline{\includegraphics[width=18cm]{tawaw2.jpg}}

The Italian participants were supported by the research groups of INDAM. In particular I was supported by GNSAGA. The initial program had a significant larger number of Italian participants. Unfortunately there was an extremely high number of cancellations. The cancellations were due in part to the wrong belief that there are security problems in Kurdistan and in part to the fact the some of the invited speakers had plans to visit the United States in the near future. Due to American regulation, people that have previously visited Iraq are not entitled to the ESTA U.S. and have to apply for an entry VISA. This complication has caused some of the cancellations by Italian speakers.

\section{Pre-school for the WAMS school.}

This activity was proposed by the local organizers and was meant as a preparation for the local participants
to the WAMS school.

I believe that the idea of a pre-school is highly successful and I feel that the suggestion to organize a pre-school might constitute a general recommendation to future WAMS schools.

The detailed schedule of the pre-school is the following:

\centerline{\includegraphics[width=18cm]{prep1.png}}

\centerline{\includegraphics[width=18cm]{prep2.png}}

\centerline{\includegraphics[width=18cm]{prep3.png}}

\section{The WAMS School}
\URCornerWallPaper{1}{CIroma3b.pdf}

During the school 5 courses were given:

\begin{itemize}


\item\textbf{INTRODUCTION TO GROUP THEORY,}
Giandomenico Boffi:\bigskip

Actions of groups on sets, Symmetric group and alternating group. Cayley's theorem. Direct products of groups, Sylow's theorem. Applications: classification of groups of small order, The alternating group is simple. Classification of finite abelian groups, finitely-generated abelian groups. 

\item\textbf{REPRESENTATION OF LIE GROUPS AND LIE ALGEBRAS,}
Mohammad Eftekhari:\bigskip

Introduction to Lie groups and Lie algebras (over R or C) Generalisation: Lie groups and algebras over finite fields (the so called finite groups of Lie type) Representations of finite groups of Lie type (Harish-Chandra approach) Representations of finite groups of Lie type (Deligne-Lusztig approach using l-adic cohomology) Computing character tables using perverse sheaves (the so called character sheaves) 

\item\textbf{TOPICS IN REPRESENTATION THEORY,}
Chwas Abbas Ahmed:\bigskip

Conjugacy classes of symmetric groups and parameterizations of simple modules. Counting standard tableaux of fixed shape: Young diagrams and tableaux, standard tableaux, Young--Frobenius formula, hook formula. Construction of fundamental modules for symmetric groups: Action of symmetric groups on tableaux, tabloids and polytabloids; permutation modules on Young subgroups. 

 \item \textbf{INTRODUCTION TO REPRESENTATION THEORY,}
Francesco Pappalardi:\bigskip

Modules over rings and algebras, simple modules, Schur's lemma. Actions of groups on vector spaces, representations. Group algebras, modules, complete reducibility, Wedderburn's theorem. Characters, orthogonality relations. Tensor product of representations. Restriction and induction.

\item \textbf{INTRODUCTION TO GALOIS THEORY,}
Michel Waldschmidt:\bigskip

Field extensions. Degree of extension. Algebraic numbers. Geometric constructions with ruler and compasses.The Galois group of an extension. The Galois correspondence between subgroups and intermediate fields.Splitting field for a polynomial. Transitivity of the Galois group on the zeros of an irreducible polynomial in a normal extension. Properties equivalent to normality. Galois groups of normal separable extensions. Properties of Galois correspondence for normal separable extensions. Normal subgroups and normal intermediate extensions. The Fundamental Theorem of Galois Theory. 
\end{itemize}

There was also a special lecture on representation theory given by Pierre Cartier.


The schedule of the lectures has been:

\centerline{\includegraphics[width=18cm]{schedule.png}}

The participants were more than 60. Among them there were 8 Iranians (6 from Zanjan e 2 from Teheran) and at least 8 Iraqis (not Kurdish). 

Given the special venue for the school, we feel that the turnout is quite satisfactory.
The classroom was very comfortable as was the rest of the campus of the University of Sulaimani which would be an ideal location for future CIMPA activities.

The local organizers, Dr. Chwas A. Ahmed and Dr. Ayad M. Ramadan did a remarkable job. 
Thanks to their efforts, we did not record any noticeable problem. The atmosphere was very pleasant
and all the schedules were always respected.


A ``trombinoscope'' of the participants of the school can be found in:

\centerline{\texttt{http://www.rnta.eu/Sulaimani2019/portraits/}}

\bigskip

\centerline{\includegraphics[width=18cm]{trombino1.png}}

\centerline{\includegraphics[width=18cm]{trombino2.png}}

\centerline{\includegraphics[width=18cm]{trombino3.png}}

\centerline{\includegraphics[width=18cm]{trombino3a.png}}

\centerline{\includegraphics[width=18cm]{trombino4.png}}


\subsection*{The budget}

The contribution from CIMPA was used almost completely. Regarding the International
participants, there were 12 applications from Iran (4 from Teheran and 8 from Zanjan), two applications
from Egypt and one from Per\`u.

The organizing committee decided to accept all the applications from Iran. Thanks to the contribution the Institute for Advanced Studies in Basic Sciences (IASBS) of Zanjan through Professor
Rashid Zaare-Nahandi, chairman of the Department of Mathematics, part of the travel expenses of the Zanjaninans
participants were covered by IASBS. Part was also covered by Salahaddin University. To three of the Teheraninans was offered to cover the travel expenses and
one of them cancelled. At the end in Sulaimani there were 6 Zanjaninans and 2 Teheraninans. The local expenses
were completely covered for all the participants (non from Sulaimani) and all the speakers. The travel expenses of the Teheraninans were covered completely by CIMPA.

Due to difficulties to obtain entry visa in Iraq, the organizing committee decided to not accept the other international applicants.

There was also a large number of applications from Iraqis (non Kurdish). They were all accepted but at the end, only a small number of them actually came to Sulaimani.

We believe that, given the success in international participation to this WAMS school, it is to be expected
that for future events in Kurdistan, the number of international applicants will grow significantly. For this reason, I believe that an higher financial investment might be justified. 

The possibility for satisfy all the requests to cover living expenses, including those from Iraqi (non Kurdish) participants, is due to an extremely good planning that was done by the local organizers that managed to negotiate good deals with the Hotel in Sulaimani.

Sulaimani University contributed financially to the school. I hope that for future events the contribution will be confirmed and also enlarged.\bigskip


\centerline{\includegraphics[width=18cm]{budget.png}}

 %\vfill\pagebreak
\URCornerWallPaper{1}{CIroma3b.pdf}

\section{Followups}

Sulaimani University would be an excellent destination for future CIMPA schools. I hope that there will be more
proposals for WAMS school and CIMPA schools to be hosted at the University of Sulaimani. Connections with local faculties and students might lead to further research collaboration and even to exchange of students.

There is also the plan to involve the local organizer Dr. Chwas Abbas Ahmed to a WAMS school in Armenia for May 2020. If fact, in the occasion of recent contacts with Prof. Dr. Yuri Movsisyan, I have been asked to consider the proposition of a WAMS school in Algebra at Yerevan State University. The possibility to involve colleagues from Sulaimani and from Iran in the enterprise would be highly desirable.


\end{document}

