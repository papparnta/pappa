\documentclass[12pt]{article}
\usepackage[cm]{fullpage}
\newcommand{\C}{\mathbb C}
\title{Dhaka (Bangladesh)\\ Report on the visit by}
\author{Francesco Pappalardi}
\date{December 20--22, 2013}
\begin{document}
\maketitle

The visit was initially planned in order to attend the 18$^{th}$ International Conference in Mathematics organized by the Bangladesh mathematical society that should have taken place in the period December 20--22. Due to the increasing 
luck of security in Dhaka the conference was cancelled on December 7$^{th}$ and will
be rescheduled sometimes in 2014. I decided to go to Bangladesh anyway and to take the
opportunity to meet some mathematicians.

My host and contact has been Habib Muzaffar,
I spend the day of December 21 with him and with Md Anwar Hossein. They took me
for a visit at the University of Dhaka, the largest and oldest of the 52 public academic institution of Bangladesh. Among these 52 universities, around 12 are technical universities. Recently private institutions have been established but, at the moment, they do not provide mathematics degrees.

I will report on my encounters with each Bangladesh mathematicians separately.


\begin{enumerate}
\item  Habib Bin Muzaffar is a number theorist with Phd from Carlton University in Ottawa.
He is 
Assistant Professor and Head of the
Department of Physical Sciences in the
School of Engineering \& Computer Science at the
Independent University, Bangladesh
(habibbin@secs.iub.edu.bd; habibbin@yahoo.com). He has been extremely helpful and introduced me to  Anwar Hossain. At the moment he teaches basic mathematics to engineers
and is not interested in cooperating with CIMPA to the organization of a CIMPA school.
However, his university, a private one, might start a degree in mathematics in the future. At that point he might reconsider the possibility of a collaboration with CIMPA.

\item Md Anwar Hossain, retired professor from the University of Dhaka, fellow of the Bangladesh academy of Sciences (the only mathematician in the academy) and president of the Bangladesh mathematical society (unique mathematical society in the country representing all Bangladesh institutions). He is the person the Michel Waldschmidt met
in Islamabad and that invited to participate in the 18$^{th}$ Bangladesh Mathematical Society meeting. 
He is an extremely lively, dynamic, kind and enthusiastic man. He has spend a long time explaining me the situation of Bangladesh Universities and their Mathematics. Him and the Bangladesh Mathematical Society are active in the organization on the mathematics Olympics. He explained that Bangladesh is not a member of IMU but the Bangladesh Mathematical Society in the only institution of hits kind in the country. He has had several students which are now academic in Bangladesh and in other countries. I wonder if he would be the right person to represent Bangladesh at the coming ICM.

Among several international contacts, including ICTP where he has been several times, he collaborates with Francine \& Marc Diener (University of Nice Sophia-Antipolis)	at the EMMAsia projects.

He is very interested in the possibility of organizing a CIMPA school in Fluid
Dynamics. I gave him some basic information on CIMPA schools and on other activities by CIMPA and pointed out that the starting point is the roadmap from the CIMPA website. We looked at it together and we noticed that in September 2013 a school on the topic of fluid dynamics took place in Senegal. He considered the possibility that the French responsible for a CIMPA school in Bangladesh could be Marc Diener.

\item Dr. Litan Kumar Saha, assistant professor at the University of Dhaka, PhD at
the Hokkaido University in Japan. He works in Fluid Dynamics. I met him briefly and could not talk to him.
\item Munibur Rahman Chowdhury, retired professor at the University of Dhaka,
 PhD in 1967 from G\"ottingen and student of Martin Keisner. He is a very knowledgeable
mathematician with a gentle and magnetic personality. He has authors books both in English and in Bengali. He gave be a copy of ``A textbook in Abstract Algebra'' and ``Essentials in Number Theory'' both really nice, in English, co-authored with Fatema Chowdhury and published by Bangladesh publishers. He mentioned that two of the topics 
important in Bangladesh mathematics research are Fluid Dynamics and Optimization Theory. He has been more then once at ICTP.

\item A. F. M. Khodadad Khan is the oldest living Bangladesh mathematician. His interests 
range from differential equations to fuzzy analysis. He was the PhD advisor of  Anwar Hossain. Finally Khodadad Khan is still lecturing at the Independent University.
\end{enumerate}\medskip

\centerline{\textbf{Final Remarks.}}

I had the impression that Bangladesh has an established mathematical tradition but
that it would appropriate and auspicable to pursue further relations with local mathematicians.

I am not sure if Number Theory could be part of immediate future actions of CIMPA while Fluid Mechanics, according to my limited and superficial experience, seems to be the most appropriate topic as a starting point.

I have informed the Bangladesh colleagues of the possibility of a visit of Michel Waldschmidt in the next months. They  all looked interested of the possibility of
his visit possibly when the security issue will be solved.

\vfill Dhaka 22 December 2013\hfill Francesco Pappalardi
\end{document}