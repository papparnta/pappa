\documentclass[10pt,handout]{beamer} %,hyperref={pdfpagelabels=false},draft,handout,handout
\usepackage[orientation=landscape,size=custom,width=16,height=9,scale=0.30,debug]{beamerposter} 
\usepackage[english]{babel}
\usepackage{lmodern}% http://ctan.org/pkg/lm
\usepackage[latin1]{inputenc}
\usepackage{times,hyperref,tikz,colortbl,yfonts,translator,marvosym,pifont}
\usepackage[T1]{fontenc}
 \newcommand{\Q}{\mathbb Q}
 \newcommand{\Z}{\mathbb Z}
 \newcommand{\N}{\mathbb N}
 \newcommand{\F}{\mathbb F}
 \newcommand{\C}{\mathbb C}
 \newcommand{\R}{\mathbb R}
\useoutertheme[height=0pt,width=2cm,right]{sidebar}
\usecolortheme{rose,sidebartab}
\useinnertheme{circles}
\usefonttheme[only large]{structurebold}
\theoremstyle{definition}
\newtheorem{Proposition}[theorem]{\translate{Proposition}}
\newtheorem{Note}[theorem]{\translate{Note}}
\lecture[4]{Elliptic curves over finite fields}{First Lecture}
\title[Elliptic curves over $\F_{q}$]{\insertlecture}
\setbeamercolor{formul}{fg=black,bg=pink}
\setbeamercolor{sidebar right}{bg=green!15}
\setbeamercolor{structure}{fg=black!120}
\setbeamercolor{postit}{fg=black,bg=yellow}
\setbeamercolor{greys}{fg=black,bg==black!25}
\setbeamerfont{title in sidebar}{series=\bfseries}
\setbeamerfont*{item}{series=}
\setbeamerfont{frametitle}{size=}
\setbeamerfont{block title}{size=\small}
\setbeamerfont{subtitle}{size=\normalsize,series=\normalfont}
\begin{document}

\begin{frame}
\includegraphics[width=1.6cm]{images/roma3.pdf}\hfill\includegraphics[width=1.9cm]{images/NUM2.jpeg}
\vfill

\begin{center}\begin{sc}
\begin{Large}

\textcolor{red}{Elliptic curves Cryptography}
\end{Large}\bigskip

\ {Francesco Pappalardi}\bigskip\bigskip

\begin{large}\begin{bf}\#3 - Elliptic curves Attacks.
\end{bf}\end{large}\medskip

September $16^{\text{th}}$ 2015\medskip
\vfill
\end{sc}\end{center}

%\includegraphics[width=1.6cm]{images/cimpalogo.pdf}\hfill
\begin{minipage}[b]{9.3cm}
\textbf{National University of Mongolia}\\  %Монгол Улсын Их Сургууль
Ulan Baatar, Mongolia\\
September 16, 2015
\end{minipage}\hfill
%\includegraphics[width=1.9cm]{images/seams.png}
\end{frame}


\section{Weil Pairing}
\begin{frame}\frametitle{Weil Pairing}
Let $E/K$ and $m\in\N$ s.t. $p\nmid m$. Then

\centerline{\begin{beamercolorbox}[rounded=true,shadow=true,wd=3cm,center]{formul}
$E[m]\cong C_m\oplus C_m$\end{beamercolorbox}}\pause

We set
\centerline{\begin{beamercolorbox}[rounded=true,shadow=true,wd=4cm,center]{postit}
$\mu_m:=\{x\in\bar{K}: x^m=1\}$\end{beamercolorbox}}\pause

$\mu_m$ is a cyclic group with $m$ elements(since $p\nmid m$)\pause
\begin{theorem}[Existence of Weil Pairing]
There exists a pairing \alert{$e_m:E[m]\times E[m]\rightarrow\mu_m$}
called \emph{Weil Pairing}, s.t. $\forall P, Q\in E[m]$\pause
\begin{enumerate}[<+-| alert@+>]
  \item $e_m(P+_EQ,R)=e_m(P,R)e_m(Q,R)$ (bilinearity)
  \item $e_m(P,R)=1\forall R\in E[m]\ \Rightarrow\ P=\infty$ (non degeneracy)
  \item $e_m(P,P)=1$
  \item $e_m(P,Q)=e_m(Q,P)^{-1}$
  \item $e_m(\sigma P,\sigma Q)=\sigma e_m(P,Q)\ \forall \sigma\in\operatorname{Gal}(\bar{K}/K)$ %   ($\sigma(A)=A$, $\sigma(B)=B$)
  \item $e_m(\alpha(P),\alpha(Q))=e_m(P,Q)^{\deg\alpha}\ \forall\alpha$ separable endomorphism
\end{enumerate}
\end{theorem}\vspace*{-4.2pt}\pause
\alert{{\scriptsize{The last one needs to be discussed further!!!}}}
\end{frame}

\begin{frame}
\frametitle{Properties of Weil pairing}

\centerline{
\begin{beamercolorbox}[shadow=true,center,rounded=true,wd=6cm]{postit}
$E[m]\cong C_m\oplus C_m\ \Rightarrow\  E[m]$ has a $\Z/m\Z$--\emph{basis}
\end{beamercolorbox}}\pause

i.e. 
$$\exists P,Q\in E[m]: \forall R\in E[m], \exists!\alpha,\beta\in\Z/m\Z, R=\alpha P+\beta Q$$\pause


\begin{Proposition}
  If $(P, Q)$ is a $\Z/m\Z$--basis, then $\zeta=e_m(P,Q)\in\mu_m$ is \emph{primitive} \hfill \hfill (i.e. $\operatorname{ord}\zeta=m$)
\end{Proposition}

\begin{proof}
 Let $d=\operatorname{ord}\zeta$. Then 
 $$1=e_m(P, Q)^d=e_m(P, dQ).$$\pause
 
$\forall R\in E[m]$ write $R=\alpha P+ \beta Q$. Hence\pause 

$$e_m(R,dQ)=e_m(P,dQ)^\alpha e_m(Q,Q)^{d\beta}=1$$\pause

So $dQ=\infty\ \Rightarrow\ m\mid d$.
\end{proof}


 \end{frame}

\begin{frame}
\frametitle{Properties of Weil pairing (continues)}\pause

\begin{Proposition}
   $$E[m]\subset E(K)\ \Rightarrow\ \mu_m\subset K$$ 
\end{Proposition}\pause

\begin{proof} 
Let $\sigma\in\operatorname{Gal}(\bar{K}/K)$. %$\sigma(A)=A$, $\sigma(B)=B$
Since the basis  $(P,Q)\subset E(K)$,
\centerline{$\sigma(P)=P, \sigma(Q)=Q.$}\pause
 
 Hence

 \centerline{$\zeta= e_m(P,Q)=e_m(\sigma P,\sigma Q)=\sigma e_m(P,Q)=\sigma\zeta$}\pause

So 
\centerline{$\zeta\in \bar{K}^{\operatorname{Gal}(\bar{K}/K)}=K\ \Rightarrow\ \mu_n=\langle\zeta\rangle\subset K^*$}
 \end{proof}\pause
 

 \begin{Corollary}
  $$E(\F_q)\cong C_n\oplus C_{kn}\ \Rightarrow q\equiv1\bmod n$$
  \end{Corollary}\pause
  
\begin{proof} 
 \centerline{$E[n]\subset E(\F_q)\Rightarrow \mu_n\subset\F_q^*\Rightarrow n\mid q-1$}
\end{proof}\pause

 \begin{beamercolorbox}[shadow=true,center,rounded=true,wd=\textwidth]{postit}
  If $E/\Q\ \Rightarrow\ E[m]\not\subseteq E(\Q)$ for $m\ge3$\end{beamercolorbox}

\end{frame}

\begin{frame}
 \frametitle{The MOV attack}\pause
 
First proposed by:
\textsc{Menezes, Alfred J.; Okamato, Tatsuaki; Vanstone, Scott A.} (1993). 
``\textit{Reducing Elliptic Curve Logarithms to Logarithms in a Finite Field}''. 
IEEE Transactions On Information Theory \textbf{39} (5).\bigskip\pause

 \begin{beamercolorbox}[shadow=true,center,rounded=true,wd=\textwidth]{postit}
  It allows to reduce the comutation of a DL in $E(\F_q)$ to a DL in $\F_{q^m}$ (for a suitable $m\in\N$).
 \end{beamercolorbox}\bigskip\pause
 
 \begin{itemize}[<+-| alert@+>]
  \item  Hence if $m<5$, there is a problem!
  \item we observed that DL in finite fields may be five times more unsafe then
 DL in elliptic curves
 \item  We shall discuss the case of supersingular curves where $m=2$
 \item  Hence, supersingular curves are NOT idoneous for ECC.
\item We assume that $E/\F_q$ is an elliptic curve
\item We shall also assume that the Weil pairing can be computed quickly (which is not obvious)
 \end{itemize}
\end{frame}

\begin{frame}
 \frametitle{The MOV attack}\pause
 \begin{itemize}[<+-| alert@+>]
  \item Assume that $P, Q\in E(\F_q)$ and that $N=\operatorname{ord}P$
  \item Also assume that $\gcd(q,N)=1$ so that $E[N]\cong \Z/N\Z\oplus\Z/N\Z$.
  \item We want to find $k$ such that
  $$Q=kP$$
  \item Such a $k$ may not exist!! However
 \end{itemize}\pause

 \begin{Proposition}
  There exists $k$ such that $Q=kP$ if and only if
  \begin{itemize}
   \item $NQ=\infty$
   \item $e_N(P,Q)=1$
  \end{itemize}
 \end{Proposition}\pause
 
 \begin{proof} (if): if $NQ=\infty$, then $Q\in E[N]$. \pause We choose $R\in E[N]$ in such a way that $\{R,P\}$ is basis for $E[N]$.\pause
 
 Then
 $$Q=a P + b R,\hfill \exists a,b\in\Z/N\Z$$\pause
 
 From basic properties of Weil pairing, $e_N(P,R)=\zeta$ is a primitive $N$--th root of unity.\pause Hence, if $e_N(P,Q)=1$, 
  $$1=e_N(P,Q)=e_N(P,P)^ae_N(P,R)^b=\zeta^b.$$\pause
 We deduce that $b\equiv 0\bmod N$. So $bR=\infty$ and $Q=aP$ as requested.\pause
 
 (only if): just note that $NQ=NkP=\infty$ and $e_N(P,Q)=e_N(P,P)^k=1$.
 \end{proof}
\end{frame}

\begin{frame}
 \frametitle{The MOV attack}
 \framesubtitle{the idea}\pause
 
 \begin{beamercolorbox}[shadow=true,center,rounded=true,wd=\textwidth]{postit}
 Given $E, P, Q$ and $N=\operatorname{ord}Q$, choose $m$ s.t. $E[N]\subset E(\F_{q^m})$.
 \end{beamercolorbox}\bigskip\pause

Note that 

 \begin{itemize}[<+-| alert@+>]
  \item such an $m$ exists since $E[N]\subset E(\overline{\F_q})$. So it is enough to choose $m$ such that $\F_{q^m}$ contains
all coordinates of all point in $E[N]$.
\item Since $\deg\phi_N=(N^2-1)/2$, we can find a suitable $m<((N^2-1)/2)!$ 
\item We shall do all our computation in $\F_{q^m}$
\end{itemize}\pause

\noindent\textbf{ALGORITHM:}
\texttt{
 \begin{enumerate}[<+-| alert@+>]
  \item Choose at random $T\in E(\F_{q^m})$
  \item Compute the order $M$ of $T$
  \item Let $d=\gcd(M,N)$, and let $T'=\frac{M}{d}T$. $T'$ has order $d$ which is a divisor of $N$. Hence $T'\in E[N]$
  \item Compute $\zeta_1=e_N(P,T')$ and $\zeta_2=e_N(Q,T_1)$. Then $\zeta_1,\zeta_2\in\mu_d\subset \F_{q^m}^*$
  \item Solve DL $\zeta_2=\zeta_1^k\in \F_{q^m}^*$. This will give $k\bmod d$.
  \item Repeat with random points untill the lcm of the $d$'s obtained in $N$. This determines $k$ modulo $N$.
  \end{enumerate}
}
\end{frame}

\begin{frame}
 \frametitle{The MOV attack}
 \framesubtitle{why does it work?}\pause

\noindent\textbf{ALGORITHM:}
\texttt{
 \begin{enumerate}
  \item Choose at random $T\in E(\F_{q^m})$
  \item Compute the order $M$ of $T$
  \item Let $d=\gcd(M,N)$, and let $T'=\frac{M}{d}T$. $T'$ has order $d$ which is a divisor of $N$. Hence $T'\in E[N]$
  \item Compute $\zeta_1=e_N(P,T')$ and $\zeta_2=e_N(Q,T_1)$. Then $\zeta_1,\zeta_2\in\mu_d\subset \F_{q^m}^*$
  \item Solve DL $\zeta_2=\zeta_1^k\in \F_{q^m}^*$. This will give $k\bmod d$.
  \item Repeat with random points untill the lcm of the $d$'s obtained in $N$. This determines $k$ modulo $N$.
  \end{enumerate}
}\pause

Let $k_d:=k\bmod d$ and note
$$\zeta_2=e_N(Q,T_1)=e_N(kP,T_1)=\zeta_1^k=\zeta_1^{k_d}$$
since $\zeta_1$ and $\zeta_2$ have both order $d$\bigskip\pause

If we compute $k_{d_1},\cdots,k_{d_s}$ with the property that 
$$\operatorname{lcm}(d_1,\cdots,d_s)=N.$$\pause Then, by the General Chinese
remainder Theorem, we can compute $k\bmod N$ which is the DL!\bigskip\pause

Once can verify that the probability that $d=1$ is quite small.

\end{frame}

\begin{frame}
 \frametitle{The MOV attack}
 \framesubtitle{Supersingular curves are unsuitable for EEC}\pause

 \begin{Definition}
  An elliptic curve is called \alert{supersingular} if, when we write
  $$E(\F_q)=q+1-a_E,$$
  we have
  $$a_E\equiv0\bmod p.$$
 \end{Definition}\pause
 
 \begin{Theorem}
  Suppose $E/\F_q$ is supersingular and that $a_E=0$. If $P\in E(\F_q)$ and $N=\operatorname{ord}P$.
  Then
  $$E[N]\subset E(\F_{q^2})$$
 \end{Theorem}\pause
 
 \begin{itemize}
  \item We shall prove the theorem now
  \item For other types of supersingular curves (i.e. with $a_E\equiv0\bmod p$ but $a_E\neq0$, it can be proven
  that If $P\in E(\F_q)$ and $N=\operatorname{ord}P$.
  Then
  $$E[N]\subset E(\F_{q^m})\qquad \text{with }m=3,4,6.$$
  \item Supersingular curves are not suitable for EEC
  
 \end{itemize}

\end{frame}

\subsection{Frobenius endomorphism}
\begin{frame}
\frametitle{The Frobenius endomorphism $\Phi_q$}\pause

\centerline{\begin{beamercolorbox}[shadow=true,left,rounded=true,center,wd=8cm]{postit}
$\Phi_q:\bar{\F}_q\rightarrow\bar{\F}_q, x\mapsto x^q$ is a field automorphism
\end{beamercolorbox}}\pause

Given $\alpha\in\bar{\F}_q$,
\centerline{\begin{beamercolorbox}[shadow=true,left,rounded=true,center,wd=6cm]{formul}
$\alpha\in\F_{q^n}\ \Leftrightarrow\ \Phi_q^n(\alpha)=\alpha^{q^n}=\alpha$\pause
\end{beamercolorbox}}\pause

Fixed points of powers of $\Phi_q$ are exactly elements of $\F_{q^n}$\pause

\centerline{\begin{beamercolorbox}[shadow=true,left,rounded=true,center,wd=8cm]{postit}
$\Phi_q:E(\bar{\F}_q)\rightarrow E(\bar{\F}_q), (x,y)\mapsto(x^q,y^q),\infty\mapsto\infty$\end{beamercolorbox}}\pause

\begin{block}{Properties of $\Phi_q$}
\begin{itemize}[<+-|alert@+>]
%\item $\Phi_q\in \operatorname{End}(E)$, it is not separable and has degree $q$
\item $\Phi_q(x,y)=(x,y)\ \Longleftrightarrow\ (x,y)\in E(\F_q)$
%\item $\operatorname{Ker}(\Phi_q-1)=E(\F_q)$
%\item $\#\operatorname{Ker}(\Phi_q-1)=\deg(\Phi_q-1)$ (since $\Phi_q-1$ is separable)
%\item if we can compute $\deg(\Phi_q-1)$, we can compute $\#E(\F_q)$
\item $\Phi_{q}^n(x,y)=(x^{q^n},y^{q^n})$ so  \alert{$\Phi_{q}^n(x,y)=(x,y)\Leftrightarrow(x,y)\in\F_{q^n}$}
%\item $\operatorname{Ker}(\Phi_q^n-1)=E(\F_{q^n})$
\item $\Phi_q$ satisfies the Carachteristic polynomial $T^2-a_ET+q$\\
i.e.
$$\forall (x,y)\in E(\overline{F_q}), (x^{q^2},y^{q^2})+_Eq(x,y)=a_E(x^q,y^q)$$
\item we write the above identity as 
$$\Phi_q^2-a_E\Phi_q+q=0.$$
\end{itemize}
\end{block}
\end{frame}
\begin{frame}
 \frametitle{The MOV attack}
 \framesubtitle{Supersingular curves are unsuitable for EEC}\pause
 
 \begin{Theorem}
  Suppose $E/\F_q$ is supersingular and that $a_E=0$. If $P\in E(\F_q)$ and $N=\operatorname{ord}P$.
  Then
  $$E[N]\subset E(\F_{q^2})$$
 \end{Theorem}\pause
 
\begin{proof} Since $a_E=0$, the Frobenius $\Phi_q$ satisfies
$$\Phi_q^2=-q$$\pause

Suppose that $P\in E(\F_q)$ has order $N$. Then $N\mid q+1$ (i.e. $q\equiv -1\bmod N$).\pause

Let $S\in E[N]$, Then
$$\Phi_{q^2}(S)=\Phi_q^2(S)=-qS=S.$$

This implies that $S\in E(\F_{q^2}).$ 
\end{proof}

\end{frame}

\begin{frame}
 \frametitle{Anomalous Curves}
 
 \begin{Definition}
  An elliptic curve is called \alert{anomalous} if, when we write
  $$\#E(\F_q)=q$$
 \end{Definition}\pause
 
\begin{itemize}[<+-|alert@+>]
\item In an anomalous curve points have order equal to a power of $p$. Hence the Weil pairing is not defined!!!
 \item  One may think that they are suitable for Cryptography for this reason. But this is not true!!
 \item There is an efficient algorithm to compute DL in anomalous curves
 \item If $E$ is anomalous, then $a_E=-1$
 \item The carachteristic polynomial of $E$ is $T^2-T+q$ with roots:
 $$\frac{1+\sqrt{1-4q}}{2}\quad\frac{1-\sqrt{1-4q}}{2}$$
 \item Hence \\
 \centerline{$\#E(\F_{q^n})=q^2+1-\frac1{2^n}\left((1+\sqrt{1-4q})^n+(1-\sqrt{1-4q})^n\right)$}
\item So $\#E(\F_{q^2})=q^2+2q$ and $E/\F_{q^2}$
\item An anamalous curve is not necessarily anomalous over field extensions but it still satisfies
$\Phi_q^2-\Phi_q+q=0$.
 \end{itemize}
\end{frame}

\begin{frame}
 \frametitle{Anomalous Curves}
 
 \begin{Definition}
  An elliptic curve is called \alert{anomalous} if, when we write
  $$\#E(\F_q)=q$$
 \end{Definition}\pause

 \begin{itemize}[<+-|alert@+>]
  \item Examples:
   \begin{enumerate}
    \item  $E': y^2+xy=x^3+x^2+1$ is anomalous over $\F_2$
    \item $E'': y^2=x^3 + x^2 - 1$ is anomalous over $\F_3$
   \end{enumerate}
 \item They are particularly suitable for Cryptography when considered over extensions
  \begin{enumerate}
   \item They group order can be computed very quickly
   \begin{itemize}
   \item $\#E'(\F_{2^{200}})=2^{200}+1-\frac{(1+\sqrt{-7})^{200}+(1-\sqrt{-7})^{00}}{2^{100}}=
   1606938044258990275541962092343697546215565682541130425732128$
   \item $\#E''(\F_{3^{150}})=3^{200}+1-\frac{(1+\sqrt{-11})^{150}+(1-\sqrt{-11})^{150}}{2^{150}}=
   369988485035126972924700782451696645401107717195926015868067750551938000$
   \end{itemize}
   \item Computations are fast on them
  \end{enumerate}
 \item From $\Phi_q^2-\Phi_q+q=0$ we deduce
 \item $\forall P=(x,y)\in E(\F_{q^n})$
 $$q(x,y)=(x^q,y^q)+(x^{q^2},-y^{q^2})$$
 \item Instead of computing $qP$ one can just compute $x^q,y^q, x^{q^2}, y^{q^2}$ which is fast in a finite field
 \item Especially if one uses \alert{normal basis}
 \end{itemize}
 \end{frame}

 \section{Normal basis on finite fields}
 \begin{frame}
  \frametitle{Normal basis on $\F_q$}
  
  \begin{Definition}
   Let $\F_{q^m}$ be a finite field extension of $\F_q$ and let $\beta\in\F_{q^m}^*$. We say that $\beta$ is \alert{normal} if
   $$ \mathcal B_\beta=\{ \beta, \beta^q, \beta^{q^2}, \ldots, \beta^{q^{m-1}} \} $$
   is an $\F_q$--basis of $\F_{q^m}$.
  \end{Definition}

 \begin{itemize}[<+-|alert@+>] 
\item $\mathcal B_\beta$ is called \alert{normal basis}
 \item  It is a classical result that every finite field admits a normal basis.
\item Given an  $\F_q$--normal basis of $\F_{q^m}$ and given $x\in\F_{q^m}^*$, we write
$$x=x_0\beta+x_1\beta^q+\cdots+x_{m-2}\beta^{q^{m-1}}$$
\item So 
$$x^p=x_0\beta^q+x_1\beta^{q^2}+\cdots+x_{m-2}\beta$$
\item Since $\beta^{q^m}=\beta$
\item There is no calculation in computing $x^q$ but just a circular rotation of the coefficients
\item Going back to anomalous curves:
 $$q(x,y)=(x^q,y^q)+(x^{q^2},-y^{q^2})$$
 \item implies that $q(x,y)$ can be computed in an anomalous curve at the cost of one addition in $E$
\end{itemize}
 \end{frame}

 
 
\section{Further reading}
\begin{frame}
\frametitle{Further Reading...}
\begin{scriptsize}
\begin{thebibliography}{99}
\bibitem{BSS} \textsc{Ian~F.~Blake,~Gadiel~Seroussi,~and~Nigel~P.~Smart},
Advances in elliptic curve cryptography, London Mathematical Society Lecture Note Series, vol. 317, Cambridge University Press, Cambridge, 2005.
 \bibitem{C} \textsc{J.~W.~S.~Cassels},
Lectures on elliptic curves, London Mathematical Society Student Texts, vol. 24, Cambridge University Press, Cambridge, 1991.
 \bibitem{CR} \textsc{John~E.~Cremona},
Algorithms for modular elliptic curves, 2nd ed., Cambridge University Press, Cambridge, 1997.
 \bibitem{Kn} \textsc{Anthony~W.~Knapp},
Elliptic curves, Mathematical Notes, vol. 40, Princeton University Press, Princeton, NJ, 1992.
 \bibitem{Ko} \textsc{Neal~Koblitz},
Introduction to elliptic curves and modular forms, Graduate Texts in Mathematics, vol. 97, Springer-Verlag, New York, 1984.
 %\bibitem{Po} \textsc{Poonen B} Elliptic curves (introduction)(19s) notes
 \bibitem{Sil} \textsc{Joseph~H.~Silverman},
The arithmetic of elliptic curves, Graduate Texts in Mathematics, vol. 106, Springer-Verlag, New York, 1986.
\bibitem{ST} \textsc{Joseph~H.~Silverman~and~John~Tate},
Rational points on elliptic curves, Undergraduate Texts in Mathematics, Springer-Verlag, New York, 1992.
\bibitem{washington} \textsc{Lawrence~C.~Washington},
Elliptic curves: Number theory and cryptography, 2nd ED. Discrete Mathematics and Its Applications, Chapman \& Hall/CRC, 2008.
\bibitem{Zimm} \textsc{Horst~G.~Zimmer},
Computational aspects of the theory of elliptic curves, Number theory and applications
(Banff, AB, 1988) NATO Adv. Sci. Inst. Ser. C Math. Phys. Sci., vol. 265, Kluwer Acad. Publ., Dordrecht, 1989, pp. 279--324.
\end{thebibliography}
\end{scriptsize}
\end{frame}

\end{document}


