\documentclass[handout]{beamer}%,[handout] %,hyperref={pdfpagelabels=false},draft,handout,handout
\usepackage[orientation=landscape,size=custom,width=16,
height=9,scale=0.42,debug]{beamerposter} 
\usepackage[english]{babel}
\usepackage{lmodern}% http://ctan.org/pkg/lm
\usepackage[latin1]{inputenc}
\usepackage{times,hyperref,tikz,colortbl,yfonts,translator}
\usepackage[T1]{fontenc}
 \newcommand{\Q}{\mathbb Q}
 \newcommand{\Z}{\mathbb Z}
 \newcommand{\N}{\mathbb N}
 \newcommand{\F}{\mathbb F}
 \newcommand{\C}{\mathbb C}
 \newcommand{\R}{\mathbb R}
%\useoutertheme[height=0pt,width=2cm,right]{sidebar}
\usecolortheme{rose,sidebartab}
\useinnertheme{circles}
\usefonttheme[only large]{structurebold}
\theoremstyle{definition}
\newtheorem{exercise}[theorem]{\translate{Exercise}}
\newtheorem{Note}[theorem]{\translate{Note}}
\lecture[4]{Elliptic curves over finite fields}{First Lecture}
\title[Elliptic curves over $\F_{q}$]{\insertlecture}
\setbeamercolor{formul}{fg=black,bg=pink}
\setbeamercolor{sidebar right}{bg=green!15}
\setbeamercolor{structure}{fg=black!120}
\setbeamercolor{postit}{fg=black,bg=yellow}
\setbeamercolor{greys}{fg=black,bg==black!25}
\setbeamerfont{title in sidebar}{series=\bfseries}
\setbeamerfont*{item}{series=}
\setbeamerfont{frametitle}{size=}
\setbeamerfont{block title}{size=\small}
\setbeamerfont{subtitle}{size=\normalsize,series=\normalfont}
\begin{document}

\begin{frame}
\includegraphics[width=1.6cm]{images/roma3.pdf}\hfill\includegraphics[width=1.9cm]{images/kku.jpeg}
\vfill

\begin{center}\begin{sc}
\begin{Large}

\textcolor{red}{Elliptic curves Cryptography}
\end{Large}\bigskip

\ {Francesco Pappalardi}\bigskip\bigskip

\begin{large}\begin{bf}\#1 - First Lecture.
\end{bf}\end{large}\medskip

June $16^{\text{th}}$ 2019\medskip
\vfill
\end{sc}\end{center}

%\includegraphics[width=1.6cm]{images/cimpalogo.pdf}\hfill
\begin{minipage}[b]{9.3cm}
\textsc{WAMS School:\\
Introductory topics in Number Theory\\ and Differential Geometry}\\
\textbf{King Khalid University}\\
Abha, Saudi Arabia
\end{minipage}\hfill
%\includegraphics[width=1.9cm]{images/seams.png}
\end{frame}

\section{Introduction}

\begin{frame}
 \frametitle{Three Lectures on Elliptic Curves Cryptography}

 \begin{Note}[Program of the Lectures]
  \begin{enumerate}[<+-| alert@+>]
   \item Generalities on Elliptic Curves over finite Fields
   \item Basic facts on Discrete Logarithms on finite groups, generic attacks (Pohlig--Hellmann, BSGS, Index Calculus)
   \item Elliptic curves Cryptography: pairing based Cryptography, MOV attacks, anomalous curves
  \end{enumerate}
 \end{Note} 
\end{frame}


\subsection{Fields}

\begin{frame}
 \frametitle{Notations}

\begin{alertblock}{Fields of characteristics 0}
 \begin{enumerate}[<+-| alert@+>]
 \item $\Q$ is the field of rational numbers
\item $\R$ and $\C$ are the fields of real and complex numbers
\item $K\subset\C$, $\dim_\Q K<\infty$ is a \emph{number field}
\begin{itemize}
\item $\Q[\sqrt{d}]$, $d\in\Q$
\item $\Q[\alpha]$, $f(\alpha)=0$, $f\in\Q[X]$
irreducible
\end{itemize}
\end{enumerate}
\end{alertblock}

\begin{exampleblock}{Finite fields}
 \begin{enumerate}[<+-| alert@+>]
 \item $\F_p=\{0,1,\ldots,p-1\}$ is the prime field;
 \item $\F_q$ is a finite field with $q=p^n$ elements
 \item $\F_q=\F_p[\xi]$, $f(\xi)=0$, $f\in\F_p[X]$
irreducible, $\partial f=n$
\item $\F_4=\F_2[\xi]$, $\xi^2=1+\xi$
\item $\F_8=\F_2[\alpha]$, $\alpha^3=\alpha+1$ but also $\F_8=\F_2[\beta]$, $\beta^3=\beta^2+1$, ($\beta=\alpha^2+1$)
\item $\F_{101^{101}}=\F_{101}[\omega], \omega^{101}=\omega+1$
\end{enumerate}
\end{exampleblock}

\end{frame}

\begin{frame}[label=current]
\frametitle{Notations}
%
 \begin{block}{Algebraic Closure of $\F_q$}\pause
 \begin{itemize}[<+-| alert@+>] % [<ballot@+-| visible@1-,+(1)>]
  \item $\C\supset\Q$ satisfies that \emph{Fundamental Theorem of Algebra}! (i.e. $\forall f\in\Q[x], \partial f>1, \exists\alpha\in\C,
 f(\alpha)=0)$
  \item We need a field that plays the role, for $\F_q$, that $\C$ plays for $\Q$. It will be $\overline{\F_q}$, called
\emph{algebraic closure of $\F_q$}
 \item[] \ \hfil 
 \begin{beamercolorbox}[rounded=true,shadow=true,wd=6.5cm]{postit}
         \begin{enumerate}
          \item $\forall n\in\N$, we fix an $\F_{q^n}$
          \item We also require that $\F_{q^n}\subseteq\F_{q^m}$ if $n\mid m$
          \item We let $\overline{\F_q}=\displaystyle\bigcup_{n\in\N}\F_{q^n}$
         \end{enumerate}
\end{beamercolorbox}
 \item
  \textbf{Fact:} $\overline{\F_q}$ is \emph{algebraically closed}\\ (i.e. $\forall f\in\F_q[x], \partial f>1, \exists\alpha\in\overline{\F_q},
 f(\alpha)=0)$
 \end{itemize}
 \end{block}
\begin{scriptsize}
If $F(x,y)\in\Q[x,y]$ a \emph{point of the curve $F=0$}, means $(x_0,y_0)\in\C^2$ s.t.
$F(x_0,y_0)=0$. \pause

If $F(x,y)\in\F_q[x,y]$ a \emph{point of the curve $F=0$}, means $(x_0,y_0)\in\overline{\F_q}^2$ s.t.
$F(x_0,y_0)=0$.
\end{scriptsize}
\end{frame}


\section{Weierstra\ss\ Equations}

\begin{frame}{The (general) Weierstra\ss\ Equation}

An elliptic curve $E$ over a $\F_q$ (finite field) is given by an equation
\centerline{\begin{beamercolorbox}[shadow=true,center,rounded=true,wd=8cm]{formul}
$E: y^2+a_1xy+a_3y=x^3+a_2x^2+a_4x+a_6$\end{beamercolorbox}}
where $a_1, a_3, a_2, a_4 ,a_6\in\F_q$ \pause

\begin{center}
 \includegraphics[width=60mm]{images/elliptic1.pdf}\pause
\llap{\includegraphics[width=60mm]{images/elliptic2.pdf}}\pause
\llap{\includegraphics[width=60mm]{images/elliptic3.pdf}}\pause
\llap{\includegraphics[width=60mm]{images/elliptic3b.pdf}}\pause
\llap{\includegraphics[width=60mm]{images/elliptic4.pdf}}\pause
\llap{\includegraphics[width=60mm]{images/elliptic5.pdf}}\pause
\llap{\includegraphics[width=60mm]{images/elliptic6.pdf}}\pause
\llap{\includegraphics[width=60mm]{images/elliptic7.pdf}}\pause
\llap{\includegraphics[width=60mm]{images/elliptic8.pdf}}\pause
\llap{\includegraphics[width=60mm]{images/elliptic9.pdf}}\pause
\llap{\includegraphics[width=60mm]{images/elliptic9b.pdf}}\pause
\llap{\includegraphics[width=60mm]{images/elliptic10.pdf}}\pause
\llap{\includegraphics[width=60mm]{images/elliptic10b.pdf}}\pause
\llap{\includegraphics[width=60mm]{images/elliptic6.pdf}}\pause
\end{center}

 \begin{beamercolorbox}[sep=1em,wd=7.5cm]{postit}
 The equation should not be \emph{singular}
 \end{beamercolorbox}
\end{frame}

\subsection{The Discriminant}

\begin{frame}
\frametitle{The Discriminant of an Equation}
\framesubtitle{The condition of absence of singular points in terms of $a_1, a_2, a_3, a_4, a_6$}
\pause

\begin{Definition} The \emph{discriminant} of a Weierstra\ss\ equation over $\F_q$,  $q=p^n$, $p\ge3$ is
\centerline{\begin{beamercolorbox}[shadow=true,center,rounded=true,wd=11cm]{formul}
\begin{align*}
{D}_E&:=\frac{1}{2^4}\left(-a_1^5 a_3 a_4 - 8 a_1^3 a_2 a_3 a_4 - 16 a_1 a_2^2 a_3 a_4 + 36 a_1^2 a_3^2 a_4 \right. \\
  &-a_1^4 a_4^2 - 8 a_1^2 a_2 a_4^2 - 16 a_2^2 a_4^2 + 96 a_1 a_3 a_4^2 +64 a_4^3 + \\
  & a_1^6 a_6 + 12 a_1^4 a_2 a_6 + 48 a_1^2 a_2^2 a_6 + 64 a_2^3 a_6 -36 a_1^3 a_3 a_6\\
  &\left. - 144 a_1 a_2 a_3 a_6 - 72 a_1^2 a_4 a_6 - 288 a_2 a_4 a_6 +
  432 a_6^2  \right)
 \end{align*}
\end{beamercolorbox}}\pause
\end{Definition}

\begin{Note}
 $E$ is \emph{non singular} if and only if ${D}_E\ne0$
\end{Note}

\end{frame}




\begin{frame}
\frametitle{Special Weierstra\ss\ equation of $E/\F_{p^\alpha}, p\neq2$}
\centerline{\begin{beamercolorbox}[shadow=true,center,rounded=true,wd=9cm]{formul}
$E: y^2+a_1xy+a_3y=x^3+a_2x^2+a_4x+a_6\quad a_i\in\F_{p^\alpha}$\end{beamercolorbox}}\pause

If ``complete the squares`` %by applying the transformation:%\begin{scriptsize}
%\centerline{
\begin{beamercolorbox}[shadow=true,center,rounded=true,wd=3.7cm]{postit}
        $\begin{cases}
  x\leftarrow x \\ y\leftarrow y -\frac{a_1x+a_3}2
 \end{cases}$            \end{beamercolorbox}%}%\end{scriptsize}
 \pause

 the Weierstra\ss\ equation becomes:
\ \ \begin{beamercolorbox}[shadow=true,center,rounded=true,wd=5.1cm]{formul}
$E': y^2=x^3+a'_2x^2+a'_4x+a'_6$
            \end{beamercolorbox}

            where $a'_2=a_2+\frac{a_1^2}4, a'_4= a_4+\frac{a_1a_3}2, a'_6= a_6+\frac{a_3^2}4$\pause

If $p\ge5$, we can also apply the transformation \pause
%\centerline{%\begin{scriptsize}
\begin{beamercolorbox}[shadow=true,center,rounded=true,wd=2.5cm]{postit}
$\begin{cases}
  x\leftarrow x-\frac{a'_2}{3} \\ y\leftarrow y
 \end{cases}$\end{beamercolorbox}
%\end{scriptsize}
%} 
obtaining the equations:\pause
\begin{beamercolorbox}[shadow=true,center,rounded=true,wd=4.2cm]{formul}
$E'': y^2=x^3+a''_4x+a''_6$
            \end{beamercolorbox}
            
\hfil\ \hspace*{-1.2cm} where $a''_4=a'_4-\frac{{a'_2}^2}3, a''_6= a'_6+\frac{2{a'_2}^3}{27}-\frac{a'_2a'_4}3$
\end{frame}

% \begin{frame}
% \frametitle{Special Weierstra\ss\ equation for $E/\F_{2^\alpha}$}
% \framesubtitle{Case $a_1\neq0$}
% \centerline{\begin{beamercolorbox}[shadow=true,left,rounded=true,wd=9cm]{formul}
% $E: y^2+a_1xy+a_3y=x^3+a_2x^2+a_4x+a_6\qquad a_i\in\F_{2^\alpha}$\hfill\\
% \ \hfill ${D}_E:=\frac{a_1^6 a_6+a_1^5 a_3 a_4+a_1^4 a_2 a_3^2+a_1^4 a_4^2+a_1^3 a_3^3+a_3^4}{a_1^6}$
% \end{beamercolorbox}}\pause
% 
% If we apply the affine transformation:\begin{scriptsize}
% \centerline{\begin{beamercolorbox}[shadow=true,left,rounded=true,wd=3.9cm]{postit}
%         $\begin{cases}
% x\longleftarrow a_1^2x+a_3/a_1\\\
% y\longleftarrow a_1^3y+(a_1^2a_4+a_3^2)/a_1^2
%   \end{cases}$\end{beamercolorbox}}\end{scriptsize}\pause
% 
% we obtain
% 
% \centerline{\begin{beamercolorbox}[shadow=true,center,rounded=true,wd=7cm]{formul}
% $E': y^2+xy=x^3+\left(\frac{a_2}{a_1^2}+\frac{a_3}{a_1^3}\right)x^2+\frac{{D}_E}{a_1^6}$\hfill \\
% \ \hfill Surprisingly ${D}_{E'}={D}_E/a_1^6$
% \end{beamercolorbox}}\pause\bigskip
% 
% With \texttt{Mathematica}
% 
% \begin{scriptsize}
% \ \hfill{\begin{beamercolorbox}[shadow=true,left,rounded=true,wd=8cm]{postit}
%       \texttt{El:=a6+a4x+a2x\^{ }2+x\^{ }3+a3y+a1xy+y\^{ }2;\\
% Simplify[PolynomialMod[ReplaceAll[El, \\
% \ \hfill \{x->a1\^{ }2 x+a3/a1, y->a1\^{ }3y+(a1\^{ }2a4+a3\^{ }2)/a1\^{ }3\}],2]]}
% \end{beamercolorbox}}\end{scriptsize}
% \end{frame}
% 
% 
\begin{frame}
% \frametitle{Special Weierstra\ss\ equation for $E/\F_{2^\alpha}$}
% \framesubtitle{Case $a_1=0$ and ${D}_E:=a_3\neq0$}
% \centerline{\begin{beamercolorbox}[shadow=true,center,rounded=true,wd=8cm]{formul}
% $E: y^2+a_1xy+a_3y=x^3+a_2x^2+a_4x+a_6\qquad a_i\in\F_{2^\alpha}$\\
% \end{beamercolorbox}}\pause
% 
% If we apply the affine transformation:\begin{scriptsize}
% \centerline{\begin{beamercolorbox}[shadow=true,center,rounded=true,wd=2.7cm]{postit}
%         $\begin{cases}
% x\longleftarrow x+a_2\\
% y\longleftarrow y
%   \end{cases}$\end{beamercolorbox}}\end{scriptsize}\pause
% 
% we obtain
% 
% \centerline{\begin{beamercolorbox}[shadow=true,center,rounded=true,wd=7cm]{formul}
% $E: y^2+a_3y=x^3+(a_4+a_2^2)x+(a_6+a_2a_4)$
% \end{beamercolorbox}}\pause\medskip
% 
% With \texttt{Mathematica}
% 
% \begin{scriptsize}
% \ \hfill{\begin{beamercolorbox}[shadow=true,left,rounded=true,wd=8.6cm]{postit}
%       \texttt{El:=a6+a4x+a2x\^{ }2+x\^{ }3+a3y+y\^{ }2;
%  Simplify[PolynomialMod[ReplaceAll[El,\{x->x+a2,y->y\}],2]]}
% \end{beamercolorbox}}\end{scriptsize}\pause
% 
%\begin{small}
\begin{Definition}
 Two Weierstra\ss\ equations over $\F_q$ are said (affinely) equivalent if there exists a (affine) change of variables that takes one
into the other
\end{Definition}
%\end{small}
\pause


\begin{Note}{The only affine transformation that take a Weierstrass equations in another Weierstrass equation have the form}
$$\begin{cases}
x\longleftarrow u^2 x+r\\
y\longleftarrow u^3 y+ u^2s x + t
  \end{cases} r,s,t,u\in\F_q$$
\end{Note}
\end{frame}

\begin{frame}
\frametitle{The Weierstra\ss\ equation}
\framesubtitle{Classification of simplified forms}

After applying a suitable affine transformation we can always assume that $E/\F_q (q=p^n)$
has a Weierstra\ss\ equation of the following form\pause

\begin{scriptsize}
 \begin{example}[Classification]
\centerline{\begin{tabular}{|l|c|l|}
\hline
 $E$ & $p$ & ${D}_E$\\
\hline
&&\\
 $y^2=x^3+Ax+B$ & $\ge5$ & $4A^3+27B^2$\\
&&\\
$y^2+xy=x^3+a_2x^2+a_6$ & $2$ & $a_6^2$\\
&&\\
 $y^2+a_3y=x^3+a_4x+a_6$  & $2$ & $a_3^4$\\
&&\\
 $y^2=x^3+Ax^2+Bx+C$ & $3$ & $
                               4A^3C-A^2B^2-18ABC+4B^3+27C^2$
                              \\
&&\\\hline
\end{tabular}}
\end{example}
\end{scriptsize}\pause

\vspace*{-1cm}
\begin{definition}[Elliptic curve] An elliptic curve is the data of a non
singular Weierstra\ss\ equation (i.e. ${D}_E\neq0$)
\end{definition}\pause

\centerline{\alert{\textbf{Note:} If $p\ge3, {D}_E\neq0\Leftrightarrow x^3+Ax^2+Bx+C$ has {no} double root}}
\end{frame}

\subsection{Elliptic curves \texorpdfstring{$/\F_2$}{F2}}
\begin{frame}
\frametitle{Elliptic curves over $\F_2$}

All possible Weierstra\ss\ equations over $\F_2$ are:\pause

\begin{beamerboxesrounded}[upper=block title example,lower=block body alerted,shadow=true]{Weierstra\ss\ equations over $\F_2$}
\begin{enumerate}
 \item $y^2+xy=x^3+x^2+1$
 \item$y^2+xy=x^3+1$
 \item$y^2+y=x^3+x$
 \item$y^2+y=x^3+x+1$
 \item$y^2+y=x^3$
 \item$y^2+y=x^3+1$
 \end{enumerate}
\end{beamerboxesrounded}
\pause

However the change of variables
$\begin{cases} x\leftarrow x+1\\ y\leftarrow y+x\end{cases}$ takes the sixth curve
into the fifth. Hence we can remove the sixth from the list.
\pause\bigskip

\begin{beamerboxesrounded}[upper=postit,lower=block body,shadow=true]{Fact:}
There are $5$ affinely inequivalent elliptic curves over $\F_2$
\end{beamerboxesrounded}
\end{frame}

\subsection{Elliptic curves \texorpdfstring{$/\F_3$}{F3}}
\begin{frame}
\frametitle{Elliptic curves in characteristic $3$}

Via a suitable transformation ($x\rightarrow u^2x+r, y\rightarrow u^3y+u^2sx+t$) over $\F_3$,  $8$ inequivalent
elliptic curves over $\F_3$ are found:\pause

\begin{beamerboxesrounded}[upper=block title example,lower=block body alerted,shadow=true]{Weierstra\ss\ equations over $\F_3$}
\begin{enumerate}
 \item $y^2=x^3+x$
 \item$y^2=x^3 - x$
 \item$y^2=x^3 - x +1$
 \item$y^2=x^3 - x -1$
 \item$y^2=x^3 + x^2 + 1$
 \item$y^2=x^3 + x^2 - 1$
 \item$y^2=x^3 - x^2 + 1$
 \item$y^2=x^3 - x^2 - 1$
 \end{enumerate}
\end{beamerboxesrounded}\pause

\vspace*{-3mm}\begin{beamerboxesrounded}[upper=postit,lower=block body,shadow=true]{Fact: let $\left(\frac{a}{q}\right)$ be the Kronecker symbol. The number of non--isomorphic (i.e. inequivalent) classes of elliptic c. over $\F_q$ is }
\centerline{$2q+3+\left(\frac{-4}{q}\right)+2\left(\frac{-3}{q}\right)$}
\end{beamerboxesrounded}



% p=5;S=0;for(a=0,p-1,for(b=0,p-1,if((4*a^3+27*b^2)%p>0,print1(S++" "x^3+a*x+b" 2-torsion "matsize(factormod(x^3+a*x+b,p))[1]);T=1;for(x=0,p-1,for(y=0,p-1,if((y^2-x^3-a*x-b)%p==0,T++)));print("  pts= "T))))

\end{frame}

\section{The sum of points}
\begin{frame}
\frametitle{The definition of $E(\F_q)$}
\begin{beamercolorbox}[shadow=true,left,rounded=true,wd=12cm]{formul}
Let $E/\F_q$ elliptic curve and consider a ``symbol'' $\infty$ (point at infinity). Set
$$E(\F_q)=\{(x,y)\in \F_q^2:\ y^2+a_1xy+a_3y=x^3+a_2x^2+a_4x+a_6\}\cup\{\infty\}$$
\end{beamercolorbox}\pause

\ \hfill \begin{beamercolorbox}[shadow=true,left,rounded=true,wd=9cm]{postit}
Hence\pause
\begin{itemize}
 \item<1-> $E(\F_q)\subset\F_q^2\cup\{\infty\}$
 \item<2-> If $\F_q\subset\F_{q^n}$, then $E(\F_q)\subset E(\F_{q^n})$
 \item<3-> We may think that $\infty$ sits on the top of the $y$--axis (``vertical direction'') 
\end{itemize}
\end{beamercolorbox}\pause

\begin{Definition}[line through points $P,Q\in E(\F_q)$]
$r_{P,Q}:\begin{cases}
                     \text{line through $P$ and }Q &\text{if }P\neq Q\\
                     \text{tangent line to $E$ at }P &\text{if }P=Q
                    \end{cases}$\hfill projective or affine
\end{Definition}\pause

\begin{itemize}[<+-| alert@+>]
\item if $\#(r_{P,Q}\cap E(\F_q))\ge2\ \Rightarrow\ \#(r_{P,Q}\cap E(\F_q))=3$
\hfill\scriptsize{\alert{if tangent line, contact point is counted with multiplicity}}  
\item $r_{\infty,\infty}\cap E(\F_q)=\{\infty,\infty,\infty\}$%\vspace*{-4.4pt}
\end{itemize}

\end{frame}

\begin{frame}
\frametitle{History (from \textsc{Wikipedia})}

\begin{columns}[c]
\begin{column}{6cm}
\begin{small}
\textbf{Carl Gustav Jacob Jacobi} (10/12/1804 -- 18/02/1851) was a German mathematician,
who made fundamental contributions to elliptic functions, dynamics, differential equations,
and number theory.
\end{small}\\
\centerline{\includegraphics[width=1.8cm]{images/Jacobi.jpg}}
%\centerline{\scriptsize{Carl Gustav Jacob Jacobi}}\\
\begin{scriptsize}\begin{block}{Some of His Achievements:}
\begin{itemize}
 \item Theta and elliptic function
 \item Hamilton Jacobi Theory
 \item Inventor of determinants
 \item Jacobi Identity
 \tiny{ $[A,[B,C]]+[B,[C,A]]+[C,[A,B]]=0$}
\end{itemize}
\end{block}\end{scriptsize}
\end{column}\pause
\begin{column}{6.5cm}\vspace*{-16.3pt}
\begin{center}
\includegraphics[width=5.5cm]{images/add1.pdf}\pause
\llap{\includegraphics[width=5.5cm]{images/add2.pdf}}\pause
\llap{\includegraphics[width=5.5cm]{images/add3.pdf}}\pause
\llap{\includegraphics[width=5.5cm]{images/add5.pdf}}\pause
\llap{\includegraphics[width=5.5cm]{images/add6.pdf}}\pause
\llap{\includegraphics[width=5.5cm]{images/add7.pdf}}\pause
\llap{\includegraphics[width=5.5cm]{images/add1.pdf}}\pause
\llap{\includegraphics[width=5.5cm]{images/add8.pdf}}\pause
\llap{\includegraphics[width=5.5cm]{images/add9.pdf}}\pause
\llap{\includegraphics[width=5.5cm]{images/ad10.pdf}}\pause
\llap{\includegraphics[width=5.5cm]{images/ad11.pdf}}\pause
\llap{\includegraphics[width=5.5cm]{images/ad12.pdf}}\pause
\llap{\includegraphics[width=5.5cm]{images/add1.pdf}}\pause
\llap{\includegraphics[width=5.5cm]{images/ad13.pdf}}\pause
\llap{\includegraphics[width=5.5cm]{images/ad14.pdf}}\pause
\llap{\includegraphics[width=5.5cm]{images/ad15.pdf}}\pause
\llap{\includegraphics[width=5.5cm]{images/add7.pdf}}\pause
\end{center}
\small{\vspace*{-2mm}
$r_{P,Q}\cap E(\F_q)=\{P,Q,R\}$\pause\\
$r_{R,\infty}\cap E(\F_q)=\{\infty,R,R'\}$}
%\centerline
{\begin{beamercolorbox}[shadow=true,center,wd=2cm]{formul}
$P+_E Q:=R'$\pause
            \end{beamercolorbox}}%\smallskip

 \small{$r_{P,\infty}\cap E(\F_q)=\{P,\infty,P'\}$}
 %\centerline
 {\begin{beamercolorbox}[shadow=true,center,wd=2cm]{formul}
             $-P:=P'$
            \end{beamercolorbox}}

\end{column}
\end{columns}
\end{frame}

\begin{frame}
\begin{beamercolorbox}[shadow=true,left,rounded=true,wd=12cm]{formul}
$E/\F_q$ elliptic curve ($D_E=D_E(a_1,a_2,a_3,a_4,a6)\neq0$)\\
$E(\F_q)=\{(x,y)\in \F_q^2:\ y^2+a_1xy+a_3y=x^3+a_2x^2+a_4x+a_6\}\cup\{\infty\}$
\end{beamercolorbox}

\includegraphics[width=7cm]{images/add7.pdf}
\end{frame}


\begin{frame}
\frametitle{Properties of the operation ``$+_E$''}

\begin{Theorem}
 The addition law on $E(\F_q)$ has the following
properties:
\begin{enumerate}[<+-| alert@+>][(a)]
 \item $P+_EQ\in E(\F_q)\hfill\forall P,Q\in E(\F_q)$
 \item  $P+_E\infty=\infty+_E P=P\hfill\forall P\in E(\F_q)$
 \item  $P+_E(-P)=\infty\hfill\forall P\in E(\F_q)$
 \item  $P+_E(Q +_E R)=(P+_E Q)+_E R\hfill\forall P,Q,R\in E(\F_q)$
 \item  $P+_E Q=Q +_E P\hfill\forall P,Q\in E(\F_q)$
\end{enumerate}
 \end{Theorem}\pause

\begin{itemize}[<+-| alert@+>]
%  \item By ``a point of $E/\F_q$ ($P\in E$)'' we mean $P\in E(\overline{\F_q})$
%  in analogy for $E/\Q$ where ``a point of $E$'' means  $P\in E(\C)$
 \item $\left(E(\F_q),+_E\right)$  \alert{commutative group}
 \item All group properties are easy except \alert{associative law (d)}
 \item Geometric proof of associativity uses \emph{Pappo's Theorem}
% \item We shall comment on how to do it by explicit computation
 \item can substitute $\F_q$ with any field $K$; Theorem holds for $\left(E(K),+_E\right)$
\item $-P=-(x_1,y_1)=(x_1,-a_1x_1-a_3-y_1)$
%In particular, if $E/\F_q$, can consider the groups $E(\overline{\F_q})$ or $E(\F_{q^n})$
\end{itemize}
\end{frame}
\end{document}


