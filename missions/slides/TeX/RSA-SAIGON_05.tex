\documentclass[landscape]{powersem} %,display
\usepackage{fancybox,marvosym,graphicx,amsmath,amssymb,pifont}
\usepackage[bookmarksopen,colorlinks,urlcolor=red,pdfpagemode=FullScreen]{hyperref}
\usepackage{fixseminar}
\usepackage{color}
\usepackage[latin1]{inputenc}
\usepackage[coloremph,colormath,colorhighlight,lightbackground]{texpower}
\IfFileExists{ucs.sty}
{\usepackage[utf8x,utf8]{vietnam}}
{\usepackage[utf8]{vietnam}}

\hfuzz=30pt
\vfuzz=30pt
\setlength{\slidewidth}{25cm}
\setlength{\slideheight}{17.5cm} \slideframe{}
\def\slideitemsep{.5ex plus .3ex minus .2ex}
\renewcommand{\slidetopmargin}{10mm}
\renewcommand{\slidebottommargin}{15mm}
\renewcommand{\slideleftmargin}{5mm}
\renewcommand{\sliderightmargin}{5mm}
\newcommand{\psd}{\pause}%\addtocounter{slide}{-1}}
\newcommand{\Ccal}{{\mathcal{C}}}
\newcommand{\F}{{\mathbb{F}}}
\newcommand{\C}{{\mathbb C}}
\newcommand{\Q}{{{\mathbb Q}}}
\newcommand{\Z}{{\mathbb Z}}
\newcommand{\N}{{\mathbb N}}
\newcommand{\manorossa}{\textcolor{conceptcolor}{\ding{43}}}
\newcommand{\matitablu}{\textcolor{altemcolor}{\ding{46}}}
\newcommand{\verde}{\textcolor{black}}
\newcommand{\heading}[1]{%
 \begin{center}
  \large\bf
  \Ovalbox{{\textcolor{conceptcolor}{#1}}}%
 \end{center}
 \vspace{1ex minus 1ex}}
\definecolor{verdescu}{rgb}{0,0.6,0.6}
\definecolor{rossoscu}{rgb}{1,0,0.2}
\definecolor{Salmon}{cmyk}{0,0.53,0.38,0}
\definecolor{MidnightBlue}{cmyk}{0.98,0.13,0,0.43}
\definecolor{BurntOrange}{cmyk}{0,0.51,1,0}
\definecolor{OliveGreen}{cmyk}{0.64,0,0.95,0.40}
\definecolor{Emerald}{cmyk}{1,0,0.51,0}
\definecolor{SpringGreen}{cmyk}{0.26,0,0.76,0} % PANTONE 381
\backgroundstyle[startcolor=white,
                   endcolor=grey,%firstgradprogression=3,
            rightpanelwidth=-7\semcm,,rightpanelcolor=pagecolor]{hgradient}%
%%%%%%%%%%%%% DATI DEL SEMINARIO IN QUESTIONE %%%%%%%%%%%%

\newpagestyle{327}%
 {\textcolor{codecolor}{\tiny{\textit{RSA cryptosystem}}} \hspace{\fill}\rightmark
\hspace{0.3cm}\thepage}
 {\includegraphics[width=4mm]{dipmat.pdf}\hspace{\fill}\tiny{\textcolor{codecolor}{\sc Universit\`a Roma Tre}}
 \hspace{\fill}\includegraphics[width=5mm]{roma3.pdf}}%%
\pagestyle{327}
%\markright{\textcolor{conceptcolor}{\tiny{\textit{University of
%Arkansas, November 17, 2005}}}}
%\markright{\textcolor{conceptcolor}{\tiny{\textit{University of
%Kathmandu, December 15, 2005}}}}
\markright{\textcolor{conceptcolor}{\tiny{\textit{\DJ\d AI H\d OC
S\UHORN\ PH\d AM TP H\`\OCIRCUMFLEX\ CH\'I MINH, December 12,
2005}}}}


%\title{\textcolor{underlcolor}{\textsc{\ \\ Factoring integers \\ Producing primes\\
%and\\ the RSA cryptosystem}}}
%\author{\verde{Francesco Pappalardi}}
%\date{Conference on Zeta Functions
%in honor of\\
% Prof. K. Ramachandra on his 70$^{\mathrm{th}}$ birthday\\
%\ \\
%\present{\doublebox{\includegraphics[width=1.5cm]{logo.pdf} \
%\begin{minipage}[c]{7cm} \vspace*{-1cm}\textsc{National Institute of Advanced Studies}\\
%\hspace*{3cm}\textbf{\emph{NIAS}}\end{minipage}\
%\includegraphics[width=1.5cm]{logo.pdf}}}\\
%\ \\
%\textcolor{red}{Fukuoka ottobre, 2004}}
\begin{document}

%\begin{slide}\pageTransitionWipe{30}
%\maketitle
%\end{slide}

\begin{slide}\pageTransitionWipe{30}\pagestyle{empty}
\addtocounter{slide}{-1}
\includegraphics[width=1.3cm]{crypto.jpg}\ \hfill \includegraphics[width=1.3cm]{crypto.jpg}
\vfil

\begin{sc}\begin{center}
\begin{Large}


\textcolor{underlcolor}{Factoring integers, Producing primes and the RSA cryptosystem}
\end{Large}
\bigskip

\textcolor{blue}{University of Pedagogy}\\
\textcolor{red}{Ho Chi Minh City}
\bigskip

%\present{\doublebox{\includegraphics[width=10cm]{UAHOME_banner.jpg}
%\present{\doublebox{\includegraphics[width=5cm]{index_banner.jpg} \
%\present{\doublebox{\includegraphics[width=10cm]{tieude6.jpg} \
\present{\noindent\includegraphics[height=1cm]{dhsplogo.jpg}\includegraphics[height=1cm]{title.pdf}}\
\\
\bigskip

December 12, 2005
\end{center}
\end{sc}
\vfill

\includegraphics[width=1.3cm]{crypto.jpg}\ \hfill \includegraphics[width=1.3cm]{crypto.jpg}
\end{slide}

\begin{slide}\pageTransitionWipe{30}

\begin{center}
\begin{small}
\ $RSA_{2048}$ = 25195908475657893494027183240048398571429282126204
032027777137836043662020707595556264018525880784406918290641249
515082189298559149176184502808489120072844992687392807287776735
971418347270261896375014971824691165077613379859095700097330459
748808428401797429100642458691817195118746121515172654632282216
869987549182422433637259085141865462043576798423387184774447920
739934236584823824281198163815010674810451660377306056201619676
256133844143603833904414952634432190114657544454178424020924616
515723350778707749817125772467962926386356373289912154831438167
899885040445364023527381951378636564391212010397122822120720357
\end{small}\end{center}\psd
\bigskip

\centerline{$RSA_{2048}$ is a 617 (decimal) digit number}\psd

\bigskip

\heading{\begin{small}\texttt{http://www.rsasecurity.com/rsalabs/node.asp?id=2093 }\end{small}}
\end{slide}

\begin{note}
Devi spiegare come hai ottenuto questo number e cosa si trova nel
sito RSA dove hai preso il number

"RSA-XXXX", where XXXX is the number's length, in bits. The values
are presented as decimal strings, with the most significant digit
first. Also listed are the number of digits, the decimal sum of
the digits and the dollar amount to be awarded for a successful
factorization.

Each challenge number may be downloaded as an ASCII text file. The
entire challenge list may be downloaded, in ASCII text format,
using the link below.

\end{note}



% ----------------------------------------------------------------

\begin{slide}\pageTransitionWipe{30}

\centerline{$RSA_{2048}$=$p\cdot q$,\ \ \ \ $p,q\approx
10^{308}$}\psd


\heading{{\bf PROBLEM:} {\it Compute $p$ and $q$}}\psd

\centerline{\textcolor{black}{\textsc{Price:}}
 200.000 US\$
  ($\sim 15,894.00$ VND)!!
%  ($\sim 13,948,300.17$ NPR)!!
 }
\bigskip\psd

\begin{center}
\begin{tabular}{|c|}
\hline \textbf{\textcolor{red}{Theorem.}} If $a\in{\mathbb N}$ \ \ $ \exists!\  p_1<p_2<\cdots<p_k$\ \textit{primes}\\
$ \ \ \textrm{s.t.} \ \ a=p_1^{\alpha_1}\cdots
p_k^{\alpha_k}$\\\hline\end{tabular}
\end{center}\psd
\bigskip

\textbf{Regrettably:} RSAlabs believes that factoring in one year
requires: \center{\begin{tabular}{|c|c|c|}\hline
number & computers & memory \\
$RSA_{1620}$ & $1.6\times10^{15}$&  $120$ Tb\\
$RSA_{1024}$ & $342,000,000$ &  $170$ Gb\\
$RSA_{760}$  & 215,000  & $4$Gb.\\ \hline
\end{tabular}}

\end{slide}

\begin{note}
$RSA$--2048 {\`e} il prodotto di due numeri primes con circa 250
cifre.\bigskip


Fattorizzare (cio{\`e} calcolare le cifre decimali di $p$ and
$q$).

Per la soluzione RSA-labs offre il premio

RSA--760 ha 228 CIFRE DECIMALI Si potrebbe fare!!
\end{note}

\begin{slide}\pageTransitionWipe{30}


\heading{\small{http://www.rsasecurity.com/rsalabs/node.asp?id=2093 }}\psd


\begin{center}
\begin{tabular}{|c|c|}\hline
 \textcolor{blue}{Challenge Number} & \textcolor{blue}{Prize (\$US)}  \\
$RSA_{576}$ &  \$10,000   \\
$RSA_{640}$ &  \$20,000    \\
$RSA_{704}$ &     \$30,000 \\
$RSA_{768}$ &     \$50,000 \\
$RSA_{896}$ &     \$75,000 \\
$RSA_{1024}$ &     \$100,000 \\
$RSA_{1536}$ &  \$150,000 \\
$RSA_{2048}$ &     \$200,000   \\
\hline
\end{tabular}
\end{center}
\end{slide}

\begin{slide}\pageTransitionWipe{30}
\addtocounter{slide}{-1}

\heading{\small{http://www.rsasecurity.com/rsalabs/node.asp?id=2093 }}


\begin{center}
\begin{tabular}{|c|c|c|}\hline
 \textcolor{blue}{Challenge Number} & \textcolor{blue}{Prize (\$US)} & \textcolor{blue}{Status} \\
$RSA_{576}$ &  \$10,000  & Factored December 2003\\
$RSA_{640}$ &  \$20,000   & Not Factored\\
$RSA_{704}$ &     \$30,000&  Not Factored\\
$RSA_{768}$ &     \$50,000&   Not Factored\\
$RSA_{896}$ &     \$75,000& Not Factored\\
$RSA_{1024}$ &     \$100,000&  Not Factored\\
$RSA_{1536}$ &  \$150,000 & Not Factored\\
$RSA_{2048}$ &     \$200,000 &  Not Factored\\
\hline
\end{tabular}
\end{center}
\end{slide}


\begin{slide}\pageTransitionWipe{30}
\heading{History of the ``\emph{Art of Factoring}''}\psd
\begin{itemize}
\item[\textcolor{black}{\ding{243}}] 220 BC Greeks (Eratosthenes of Cyrene )\psd
\item[\textcolor{black}{\ding{243}}] 1730 Euler $2^{2^5}+1=641\cdot 6700417 $\psd
\item[\textcolor{black}{\ding{243}}] 1750--1800 Fermat, Gauss (Sieves - Tables)\psd
\item[\textcolor{black}{\ding{243}}] 1880 Landry \& Le Lasseur:
$2^{2^6}+1= 274177 \times 67280421310721$\psd
\item[\textcolor{black}{\ding{243}}] 1919 Pierre and Eug\`ene Carissan (Factoring Machine)\psd
\item[\textcolor{black}{\ding{243}}] 1970 Morrison \& Brillhart
$2^{2^7}+1=
59649589127497217 \times 5704689200685129054721$\psd
\item[\textcolor{black}{\ding{243}}] 1980, Richard Brent and John Pollard
$2^{2^8}+1=
1238926361552897 \times  93461639715357977769163558199606896584051237541638188580280321$\psd
\item[\textcolor{black}{\ding{243}}] 1982 Quadratic Sieve \textbf{QS}
(Pomerance)\hfil $\rightsquigarrow$\hfil Number Fields Sieve \textbf{NFS}\psd
\item[\textcolor{black}{\ding{243}}] 1987 Elliptic curves factoring \textbf{ECF} (Lenstra)
\end{itemize}
\end{slide}

\begin{slide}\pageTransitionWipe{30}
\heading{Carissan's ancient Factoring Machine}\psd


\begin{figure}
  \centering
\includegraphics[width=5cm]{images/cari.jpg}
  \caption{Conservatoire Nationale des Arts et M\'etiers in Paris}
\end{figure}\psd

\heading{\begin{small}\texttt{http://www.math.uwaterloo.ca/~shallit/Papers/carissan.html
}\end{small}}

\end{slide}

\begin{slide}\pageTransitionWipe{30}
\begin{figure}
  \centering \includegraphics[width=3cm]{Caris2.jpg}
  \caption{Lieutenant Eug\`ene Carissan}\end{figure}\psd
\centerline{\begin{tabular}{rcl}
$225058681=229\times982789$ & &{2 minutes}\\
$3450315521=1409\times 2418769$ & & {3 minutes}\\
$3570537526921=841249\times4244329$ & & {18 minutes}\\
\end{tabular}}

\end{slide}

\begin{note}
Fattorizzare con i crivelli
$N=x^2-y^2=(x-y)(x+y)$
\end{note}

\begin{slide}\pageTransitionWipe{30}
\heading{Contemporary Factoring 1/2}\vspace{-4mm}\psd

\begin{itemize}
  \item[\textcolor{blue}{\ding{182}}] 1994, Quadratic Sieve (QS): (8 months, 600 voluntaries, 20 countries)\\
  D.Atkins, M. Graff, A. Lenstra, P. Leyland
\begin{tiny}\begin{tabular}{l}
\hspace*{-1cm}$  RSA_{129} = 114381625757888867669235779976146612010218296721242362562561842935706$\\
\hspace*{5mm}$935245733897830597123563958705058989075147599290026879543541=$\\
$        = 3490529510847650949147849619903898133417764638493387843990820577 \times$\\
$
32769132993266709549961988190834461413177642967992942539798288533
$\end{tabular}\end{tiny}\psd

  \item[\textcolor{blue}{\ding{183}}] (February 2  1999), Number Fields Sieve (NFS): (160 Sun, 4 months)
 \begin{tiny}\begin{tabular}{c}
\hspace*{-1cm}$ RSA_{155} = 109417386415705274218097073220403576120037329454492059909138421314763499842$\\
$88934784717997257891267332497625752899781833797076537244027146743531593354333897=$\\
$=102639592829741105772054196573991675900716567808038066803341933521790711307779
\times$\\
$106603488380168454820927220360012878679207958575989291522270608237193062808643
$\end{tabular}\end{tiny}\psd

  \item[\textcolor{blue}{\ding{184}}] (December 3, 2003) (NFS): J. Franke et al. (174 decimal digits)
 \begin{tiny}\begin{tabular}{c}
\hspace*{-1cm}$ RSA_{576} = 1881988129206079638386972394616504398071635633794173827007633564229888597152346$\\
$65485319060606504743045317388011303396716199692321205734031879550656996221305168759307650257059=$\\
$=398075086424064937397125500550386491199064362342526708406385189575946388957261768583317\times$\\
$472772146107435302536223071973048224632914695302097116459852171130520711256363590397527
$\end{tabular}\end{tiny}\psd

\item[\textcolor{blue}{\ding{185}}] (May 9,2005) (NFS): F. Bahr, et
al (663 binary digits)
 \begin{tiny}\hspace*{-1.5cm}\begin{tabular}{c}
$ RSA_{200} = 279978339112213278708294676387226016210704467869554285375600099293261284001076093456710529553608$\\
$56061822351910951365788637105954482006576775098580557613579098734950144178863178946295187237869221823983=$\\
$3532461934402770121272604978198464368671197400197625023649303468776121253679423200058547956528088349\times$\\
$7925869954478333033347085841480059687737975857364219960734330341455767872818152135381409304740185467$
\end{tabular}\end{tiny}
\end{itemize}

\end{slide}

\begin{slide}\pageTransitionWipe{30}
\heading{Contemporary Factoring 2/2}\psd
\vspace{-4mm} Elliptic curves factoring (ECM) H. Lenstra (1985) - small factors (50 digits)\psd
\begin{itemize}
\item[\textcolor{blue}{\ding{187}}] (1993) A. Lenstra, H. Lenstra, Jr., M. Manasse, and J. Pollard
$2^{2^9}+1 = 2424833\times 7455602825647884208337395736200454918783366342657 \times p99$\psd
\item[\textcolor{blue}{\ding{187}}] (April 6, 2005) (ECM) B. Dodson
$3^{466}+1$ is divisible by $709601635082267320966424084955776789770864725643996885415676682297$;
\psd
\item[\textcolor{blue}{\ding{188}}] (Sept. 5, 2005) (ECM) K. Aoki \& T. Shimoyama
$10^{311}-1$ is divisible by
$4344673058714954477761314793437392900672885445361103905548950933$
\psd
\end{itemize}
%\centerline{\textcolor{red}{All: "\emph{sub--exponential running
%time}"}}\psd
For updates see Paul Zimmerman's ``\textsl{Integer Factoring Records}'':
\vspace{-3mm}\heading{\begin{small}\texttt{http://www.loria.fr/~zimmerma/records/factor.html }\end{small}}\psd
\vspace{-3mm}More infoes about fatroring in
\vspace{-3mm}\heading{\begin{small}\texttt{http://www.crypto-world.com/FactorWorld.html }\end{small}}\psd
\vspace{-3mm}Update on ``factorization of Fermat Numbers'':
\vspace{-3mm}\heading{\begin{small}\texttt{http://www.prothsearch.net/fermat.html }\end{small}}
\end{slide}

\begin{slide}\pageTransitionWipe{30}
\heading{Last Minute News}\vspace{-4mm}\psd
\begin{scriptsize}
\begin{tt}
\noindent
 Date: Thu, 10 Nov 2005 22:07:26 -0500\\
 From: Jens Franke <franke@math.uni-bonn.de>\\
 To: NMBRTHRY@LISTSERV.NODAK.EDU

We have factored RSA640 by GNFS. The factors are

16347336458092538484431338838650908598417836700330\
92312181110852389333100104508151212118167511579

and

19008712816648221131268515739354139754718967899685\
15493666638539088027103802104498957191261465571

We did lattice sieving for most special q between 28e7 and 77e7
using factor base bounds of 28e7 on the algebraic side and 15e7 on
the rational side. The bounds for large primes were 2\^\ 34. This
produced 166e7 relations. After removing duplicates 143e7 relations
remained. A filter job produced a matrix with 36e6 rows and columns,
having 74e8 non-zero entries. This was solved by Block-Lanczos.

Sieving has been done on 80 2.2 GHz Opteron CPUs and took 3 months.
The matrix step was performed on a cluster of 80 2.2 GHz Opterons
connected via a Gigabit network and took about 1.5 months.

Calendar time for the factorization (without polynomial selection)
was 5 months.

More details will be given later.

F. Bahr, M. Boehm, J. Franke, T. Kleinjung
\end{tt}\end{scriptsize}

\end{slide}

\begin{slide}\pageTransitionWipe{30}

\centerline{\Large{\textcolor{blue}{RSA}}}

\centerline{\includegraphics[width=9cm]{rsa-photo.jpg}}

\centerline{Adi Shamir, Ron L. Rivest, Leonard Adleman (1978)}
\end{slide}

\begin{slide}\pageTransitionWipe{30}
\heading{The RSA cryptosystem}\psd

1978 R. L. Rivest, A. Shamir, L. Adleman (Patent expired in
1998)\psd

\textbf{\textcolor{black}{Problem:}} \emph{Alice wants to send the
message $\mathcal P$ to Bob so that Charles cannot read it}\psd

\begin{center}\begin{large}
\begin{tabular}{|lcr|}\hline
  $A$ (\textsl{Alice}) & $-\!\!\!-\!\!\!-\!\!\!-\!\!\!-\!\!\!-\!\!\!\rightarrow$ & $B$ (\textsl{Bob})\\
   & $\uparrow$ &  \\
   & $C$ (\textsl{Charles}) &  \\ \hline\end{tabular}
\end{large}\end{center}\psd

\parstepwise{\begin{itemize}
  \item[\textcolor{Salmon}{\ding{182}}] \step{\textsc{Key generation} \hspace{3cm} Bob has to do it}
  \item[\textcolor{Salmon}{\ding{183}}]  \step{\textsc{Encryption}\ \ \  \hspace{3cm} Alice has to do it}
  \item[\textcolor{Salmon}{\ding{184}}]  \step{\textsc{Decryption}\ \ \  \hspace{3cm} Bob has to do it}
  \item[\textcolor{Salmon}{\ding{185}}]  \step{\textsc{Attack}\ \ \ \ \ \ \hspace{3cm}
  Charles would like to do it}
\end{itemize}}
\end{slide}

\begin{slide}\pageTransitionWipe{30}
\heading{\textsc{Bob: Key generation}}\psd

\parstepwise{\begin{itemize}
  \item[\textcolor{Emerald}{\ding{45}}] \step{\underline{He chooses} \emph{\textcolor{red}{randomly}}  $p$ and $q$ primes \ \ \ \ \ \ \ ($p,q\approx 10^{100}$)}
  \item[\textcolor{Emerald}{\ding{45}}] \step{\underline{He computes} \ \ $M=p\times q$, $\varphi(M)=(p-1)\times(q-1)$}
  \item[\textcolor{Emerald}{\ding{45}}] \step{\underline{He chooses} an integer $e$  s.t.}
\item[\ ]\step{\ \ \ \ \ \ $0\leq e\leq \varphi(M)\ \
\textrm{and}\ \  \gcd(e,\varphi(M))=1$} \item[\ ]\step{\ \ \ \ \ \
\scriptsize\textsc{Note.} One could take  $e=3$ and $p\equiv
q\equiv 2\bmod 3$} \item[\ ]\step{\ \ \ \ \ \ \hspace{4cm}\ \ \
Experts recommend $e=2^{16}+1$}
  \item[\textcolor{Emerald}{\ding{45}}] \step{\underline{He computes} arithmetic inverse $d$ of  $e$ modulo $\varphi(M)$}
\item[\ ] \step{\ \ \ \ \ \ {(i.e. $d\in\mathbb N$ (unique $\leq \varphi(M)$) s.t. $e\times d\equiv 1 \pmod{\varphi(M)}$)}}
  \item[\textcolor{Emerald}{\ding{45}}] \step{\underline{Publishes} $(M,e)$ \textbf{\textcolor{Emerald}{public key}} and hides
  \textbf{\textcolor{red}{secret key}} $d$}
\end{itemize}}\psd\bigskip
\center{\textbf{\textcolor{black}{Problem:}} \emph{How does Bob do
all this?}- We will go came back to it!}

\end{slide}

\begin{note}
escludi a voce tutti tranne il prime.

per il penultimo ricorda a voce l'algoritmo di Euclide and la sua
importanza

spiega imprecisamente cosa \`{e} un number casuale confessando che
non sai esattamente cosa significa.

\end{note}

%%%%%%%%%%%%%%%
\begin{slide}\pageTransitionWipe{30}
\heading{Alice: Encryption}\psd

Represent the message ${\mathcal P}$ as an element of  ${\mathbb Z}/M{\mathbb Z}$\psd

(\textcolor{blue}{for example})\fbox{ $A\leftrightarrow 1\ \ \
B\leftrightarrow2\ \ \ C\leftrightarrow3\ \ \ldots\ \
Z\leftrightarrow26\ \ \ AA\leftrightarrow 27 \ldots$}\psd
\begin{scriptsize}
% A B C D E F G H I  J  K  L  M  N  O  P  Q  R  S  T  U  V  W  X  Y  Z
% 1 2 3 4 5 6 7 8 9 10 11 12 13 14 15 16 17 18 19 20 21 22 23 24 25 26=0
% A  R  K  A  N  S  A  S            | S  A  I  G  O  N  | N  E  P  A  L
% 1 18 11  1 14 19  1 19            |19  1  9  7 15  14 |14  5 16  1 12
% e = 65537
% M = 79537397720925283289
% D = lift(Mod(A,M)^e)
%
% PLAINTEXT ARKANSAS
% 26^7+18*26^6+11*26^5+26^4+14*26^3+19*26^2+26+19=A=13723705209
% ENCRYPTED ARKANSAS
% Mod(2089374373918965676, 79537397720925283289)
% D=2089374373918965676;for(J=1,30,B=D%26;print1(B" ");D=(D-B)/26)
% 10 25 2 12 1 17 15 9 17 18 6 23 21
%  J  Y B  L A  Q  O I  Q  R F  W  U
%
% PLAINTEXT SAIGON
% 19*26^5+26^4+9*26^3+7*26^2+15*26+14 =A= 226366440
% ENCRYPTED SAIGON
% Mod(71502481501746956206, 79537397720925283289)
% D=71502481501746956206;for(J=1,30,B=D%26;print1(B" ");D=(D-B)/26)
% 0 17 15 25 23 24 0 25 4 14 3 7 21 2 1
% Z  P  O  Y  W  X Z  X D  N C G  U B A  = ZPOYWXZXDNCGUBA
%
% PLAINTEXT NEPAL
% 14*26^4+5*26^3+16*26^2+26+12 =A= 6496398
% ENCRYPTED NEPAL
% Mod(68059003759328352940, 79537397720925283289)
% D=68059003759328352940;for(J=1,30,B=D%26;print1(B" ");D=(D-B)/26)
% 0 12 21 6 1 14 5 18 6 16 24 4 11 1 1
% Z  K  U F A  N E  R F  P  X D  K A A = ZKUFANERFPXDKAA


$$\texttt{\textcolor{red}{SAIGON}}\leftrightarrow
19\cdot26^5+26^4+9\cdot26^3+7\cdot26^2+15\cdot26+14=
                          226366440$$ Note. Better if texts are not too short. Otherwise one performs
some  \textsl{padding}
\end{scriptsize}\psd

\begin{Large}
\begin{center}\begin{tabular}{|c|}\hline
${\mathcal C}=E({\mathcal P})={\mathcal P}^e\pmod M$\\\hline\end{tabular}\end{center}
\end{Large}\psd

\scriptsize{ \textcolor{blue}{Example}: $p=9049465727$,
$q=8789181607$, $M=79537397720925283289 $, $e=2^{16}+1=65537$,
${\mathcal P}=\texttt{\textcolor{red}{SAIGON}}$:\psd
$$E(\texttt{\textcolor{red}{SAIGON}})=226366440^{65537}\,(\bmod 79537397720925283289)$$
$$=71502481501746956206={\mathcal C}=\texttt{\textcolor{black}{ZPOYWXZXDNCGUBA}}\hfill$$}
\end{slide}

\begin{note}
Spiega che la rappresentazione del testo come elemento di ${\mathbb Z}/M\mathbb Z$
in realt\`{a} una faccenda MOLTO delicata. Dietro c'\`{e} la questione
dei PKCS (Public Key Cryptography Standards).

\end{note}

\begin{slide}\pageTransitionWipe{30}
\heading{Bob: Decryption}\psd

\begin{Large}
\begin{center}\begin{tabular}{|c|}\hline
${\mathcal P}=D({\mathcal C})={\mathcal C}^d\pmod M$\\\hline\end{tabular}\end{center}
\end{Large}\psd

\textbf{\textcolor{blue}{Note.}} Bob decrypts because he is the
only one that knows $d$.\psd

\centerline{\begin{tabular}{|c|}\hline
\textbf{\textcolor{red}{Theorem. (Euler)}} If $a,m\in {\mathbb N}$, $\gcd(a,m)=1$, \\
$a^{\varphi(m)}\equiv 1\pmod m.$\\
If $n_1\equiv n_2\bmod\varphi(m)$ then $a^{n_1}\equiv a^{n_2}\bmod m$.\\
\hline
\end{tabular}}\psd

Therefore ($ed\equiv1\bmod\varphi(M)$)\medskip

\centerline{
\begin{tabular}{|c|}\hline
{\Large $D(E({\mathcal P}))={\mathcal P}^{ed}\equiv {\mathcal P}\bmod M$}\\
\hline
\end{tabular}}\psd

\scriptsize{\textcolor{blue}{Example}(cont.):$d=65537^{-1}\bmod \varphi(9049465727\cdot8789181607)=57173914060643780153$\\
\hspace*{2mm}$D(\texttt{\textcolor{black}{ZPOYWXZXDNCGUBA}})=$\\
\hspace*{2mm}$71502481501746956206
^{57173914060643780153}(\bmod79537397720925283289)=\texttt{\textcolor{red}{SAIGON}}$}

\end{slide}

\begin{slide}\pageTransitionWipe{30}

\centerline{\Large{\textcolor{blue}{RSA at work}}}\bigskip\bigskip

\centerline{\includegraphics[width=7cm]{rsa.jpg}}

\end{slide}

\begin{slide}\pageTransitionWipe{30}%1/Mod(65537,9049465726*8789181606)
\heading{Repeated squaring algorithm}\psd

\textbf{\textcolor{black}{Problem:}} How does one compute $a^b\bmod
c$?\psd \centerline{$71502481501746956206^{57173914060643780153
}(\bmod 79537397720925283289)$}\psd

\parstepwise{\begin{itemize}
\item[\textcolor{blue}{\ding{45}}]
\step{\underline{Compute the binary expansion}  $b=\displaystyle{\sum_{j=0}^{[\log_2b]}\epsilon_j2^j}$}
\item[\ ]
\step{\scriptsize{$57173914060643780153$}=\tiny{$1100011001011100101000101111101010111100110110001001
00011000111001$}}
\item[\textcolor{blue}{\ding{45}}] \step{\underline{Compute recursively}  $a^{2^j}\bmod c, j=1,\ldots,[\log_2b]$:}
\item[\ ]\step{\ \ \ \ \ $\displaystyle{a^{2^j}\bmod c=\left(a^{2^{j-1}}\bmod c\right)^2\bmod c}$}
\item[\textcolor{blue}{\ding{45}}] \step{\underline{Multiply} the $a^{2^j}\bmod c$ with $\epsilon_j=1$}%\vspace{-2mm}
\item[\ ]\step{\ \ \ \ \ ${a^b\bmod c=\left(\prod_{j=0,\epsilon_j=1}^{[\log_2b]}a^{2^j}\bmod c\right)\bmod c}$}
\end{itemize}}

\end{slide}

\begin{slide}\pageTransitionWipe{30}
\heading{$\#\{\textrm{oper. in\ }  {\mathbb Z}/c{\mathbb Z}\
\textrm{to compute } a^b\bmod c\} \leq 2\log_2b$}\psd

\texttt{\textcolor{black}{ZPOYWXZXDNCGUBA}} is decrypted with $131$
operations in {\small$${\mathbb Z}/79537397720925283289\mathbb Z$$}
\medskip\psd

\textsc{\textcolor{blue}{Pseudo code:}} $e_c(a,b)=a^b\bmod c$\psd

\begin{center}
\texttt{\begin{tabular}{|lclcll|}\hline
$e_c(a,b)$ &= &  if   & $b=1$  & then   & $a \bmod c$\\
              &  &  if   &   $2|b$  & then       & $e_c(a,\frac{b}{2})^2 \bmod c$\\
              &  &   else   &        &            & $a*e_c(a,\frac{b-1}{2})^2 \bmod c$\\
              \hline
\end{tabular}}
\end{center}\psd\bigskip

To encrypt with $e=2^{16}+1$, only $17$ operations  in ${\mathbb Z}/M{\mathbb Z}$ are enough
\end{slide}

\begin{slide}\pageTransitionWipe{30}
\heading{Key generation}\psd\bigskip

\textcolor{black}{\textbf{Problem.}} Produce a random prime
$p\approx 10^{100}$

\begin{center}\texttt{
\begin{tabular}{|cl|}\hline
&\textbf{\textcolor{blue}{Probabilistic algorithm (type Las Vegas)}}\\
\hline
  1.& Let $p=\textsc{Random}(10^{100})$\\
  2.& If \textsc{isprime}($p$)=1 then \textsc{Output}=$p$ else goto 1\\
\hline
\end{tabular}
}\end{center}\psd\medskip

\textcolor{black}{\textbf{subproblems:}}\psd

\textsl{\textcolor{red}{A.}} How many iterations are necessary?
\centerline{(i.e. how are primes distributes?)}\psd

\noindent\textsl{\textcolor{red}{B.}} How does one check if $p$ is
prime? \centerline{(i.e. how does one compute
${{\texttt{\textsc{isprime}}(p)}}$?) $\rightsquigarrow$ Primality test}\psd
\bigskip

\centerline{\fbox{\scriptsize{\textit{\textcolor{red}{False Metropolitan Legend:}
 \textcolor{blue}{Check primality is equivalent to factoring}}}}}

\end{slide}


\begin{slide}\pageTransitionWipe{30}
\heading{\textsl{A.} Distribution of prime numbers}\psd

$$\pi(x)=\#\{p\leq x\ \textrm{t. c. } p \textrm{ is prime}\}$$\psd

\begin{center}
\begin{tabular}{|c|}\hline
\textbf{\textcolor{red}{Theorem.}} (Hadamard - de la vallee Pussen - 1897)\\
 $\displaystyle{\pi(x)\sim \frac{x}{\log x}}$\\
\hline\end{tabular}
\end{center}\psd

Quantitative version:

\small{
\begin{center}
\begin{tabular}{|c|}\hline
\textbf{\textcolor{red}{Theorem.}} (Rosser - Schoenfeld) if $x\geq 67$\\
 $\displaystyle{\frac{x}{\log x-1/2}< \pi(x)< \frac{x}{\log x-3/2}}$\\
\hline\end{tabular}
\end{center}}\psd

Therefore

$$ 0.0043523959267
 < Prob\left((\texttt{\textsc{Random}}(10^{100})=\texttt{prime}\right)< 0.004371422086
$$\end{slide}


\begin{slide}\pageTransitionWipe{30}
If $P_{k}$ is the probability that among $k$ random numbers$\leq
10^{100}$ there is a prime one, then\psd
$$P_k=1-\left(1-\frac{\pi(10^{100})}{10^{100}}\right)^k$$\psd

Therefore
$$0.663942
<P_{250}<
 0.66554440
$$
\bigskip\psd

\textcolor{red}{To speed up the process}: One can
consider only odd random numbers not divisible by
$3$ nor by $5$.\psd

Let
$$\Psi(x,30)=\#\left\{n\leq x\ \textrm{s.t.} \gcd(n,30)=1\right\}$$

%1-(1-1/(100*log(10)-3/2))^50
\end{slide}

\begin{slide}\pageTransitionWipe{30}
\textcolor{red}{To speed up the process}: One can
consider only odd random numbers not divisible by
$3$ nor by $5$.\psd

Let
$$\Psi(x,30)=\#\left\{n\leq x\ \textrm{s.t.} \gcd(n,30)=1\right\}$$
then\psd
$$\frac{4}{15}x-4<\Psi(x,30)<\frac{4}{15}x+4$$\psd

Hence, if $P_k'$ is the probability that among $k$ random numbers $\leq 10^{100}$
coprime with 30, there is a prime one, then\psd
$$P'_k=1-\left(1-\frac{\pi(10^{100})}{\Psi(10^{100},30)}\right)^k$$\psd
and
$$0.98365832
<P'_{250}<
 0.98395199$$
\end{slide}

%1-(1-10^100/(100*log(10)-3/2)/(4*10^100/15+4))^250

\begin{slide}\pageTransitionWipe{30}
\heading{\textsl{B.} Primality test}\psd

%\hspace*{2.5cm}$\rightsquigarrow$ \textsc{\textcolor{blue}{Pseudo
%Primi and Pseudo Primi Forti}}

\begin{center}
\begin{tabular}{|c|}
\hline \textbf{\textcolor{red}{Fermat Little Theorem.}} If $p$ is prime, $p\nmid a\in{\mathbb N}$\\
$a^{p-1}\equiv1\bmod p$
\\\hline\end{tabular}
\end{center}\psd
\bigskip

\textbf{\textcolor{blue}{NON-primality test}}
$$M\in{\mathbb Z},\ \ 2^{M-1}\not\equiv 1\bmod M =\!\!\!=\!\!\!>\ \  M \textrm{composite!}$$\psd

\textsc{\textcolor{blue}{Example}:} $2^{RSA_{2048}-1}\not\equiv1\bmod RSA_{2048}$\\
\centerline{Therefore $RSA_{2048}$ is composite!}\psd

Fermat little Theorem does not invert. Infact\psd

$$2^{93960}\equiv 1\pmod{93961}\ \ \ \ \textrm{but}\ \ \ \ 93961
=7\times31\times433$$
 \end{slide}


\begin{slide}\pageTransitionWipe{30}
\heading{Strong pseudo primes}\psd

From now on $m\equiv3\bmod4$ (just to simplify the notation)\psd

{\textbf{\textcolor{red}{Definition.}}} $m\in\mathbb N$,
$m\equiv3\bmod4$, composite is said \emph{\textcolor{black}{strong pseudo prime} (SPSP)} in base $a$ if
$$a^{(m-1)/2}\equiv \pm1\pmod{m}.$$\psd

\textbf{\textcolor{blue}{Note.}} If $p>2$ prime $=\!\!\!=\!\!\!>\ \ a^{(p-1)/2}\equiv \pm1\pmod{p}$\\
%\ \ \ ($x=a^{(p-1)/2}$ is root of the polynomial $x^2-1$ $=\!\!\!=\!\!\!>\ \ x=\pm1$).
Let \fbox{${\mathcal S}=\{a\in {\mathbb Z}/m{\mathbb Z}\ \textrm{s.t.}\ \gcd(m,a)=1, a^{(m-1)/2}\equiv \pm1\pmod{m}\}$}%\\
\psd %$=\!\!\!=\!\!\!>$
\parstepwise{\begin{itemize}
  \item[\textcolor{Emerald}{\ding{192}}] \step{${\mathcal S}\subseteq ({\mathbb Z}/m{\mathbb Z})^*$ subgroup}
  \item[\textcolor{Emerald}{\ding{193}}] \step{If $m$ is composite $=\!\!\!=\!\!\!>$ proper subgroup}
  \item[\textcolor{Emerald}{\ding{194}}] \step{If $m$ is composite $=\!\!\!=\!\!\!>\ \ \ \#{\mathcal S}\leq \frac{\varphi(m)}{4}$}
  \item[\textcolor{Emerald}{\ding{195}}] \step{If $m$ is composite $=\!\!\!=\!\!\!>\ \ \ Prob(m\ \textrm{SPSP in base }a)\leq 0,25$}
\end{itemize}}
\end{slide}

\begin{slide}\pageTransitionWipe{30}
\heading{Miller--Rabin primality test}\psd

Let $m\equiv 3\bmod 4$\psd

\center{\texttt{\begin{tabular}{|l|}\hline
  % after \\: \hline or \cline{col1-col2} \cline{col3-col4} ...
  \textsc{\textcolor{blue}{Miller Rabin algorithm with $k$ iterations}} \\
\hline
$N=(m-1)/2$\\
for $j=0$ to $k$ do\ \
  $a=$Random$(m)$ \\
if $a^N\not\equiv\pm1\bmod m$ then OUPUT=($m$ composite): END \\
endfor OUTPUT=($m$ prime) \\\hline
\end{tabular}}}\psd

\emph{\textcolor{red}{Monte Carlo primality test}}\psd

$Prob($Miller Rabin says \texttt{$m$ prime} and $m$ is composite)
$\lessapprox\frac{1}{4^k}$\psd

In the real world, software uses Miller Rabin with $k=10$
\end{slide}


\begin{slide}\pageTransitionWipe{30}
\heading{Deterministic primality tests}\psd

\textbf{\textcolor{blue}{Theorem. (Miller, Bach)}} If $m$ is
composite, then \centerline{\textbf{{GRH}} $=\!\!\!=\!\!\!>\
\exists a\leq 2\log^2 m$ s.t.\ \ $a^{(m-1)/2}\not\equiv
\pm1\pmod{m}$.}\\ \ \hfill\textcolor{rossoscu}{(i.e. $m \textit{
is not SPSP in base } a.$)}\psd

\small{\textbf{\textcolor{blue}{Consequence:}} ``Miller--Rabin
\emph{de--randomizes on GRH}'' ($m\equiv3\bmod4$)}\psd

\center{\texttt{\begin{tabular}{|lll|}\hline
  % after \\: \hline or \cline{col1-col2} \cline{col3-col4} ...
  %\textsc{\textcolor{blue}{Miller Rabin algorithm with $k$ iterations}} \\
%\hline
for & $a=2$ to $2\log^2 m$ & do\ \ \\
    & if $a^{(m-1)/2}\not\equiv\pm1\bmod m$ & then \\ & & OUPUT=($m$ composite): END \\
endfor & & OUTPUT=($m$ prime) \\\hline
\end{tabular}}}\psd

\emph{\textcolor{red}{Deterministic Polynomial time algorithm}}\psd

\textcolor{blue}{It runs in $O(\log^5m)$ operations in ${\mathbb
Z}/m\mathbb Z$.}

%\textcolor{rossoscu}{(i.e. $m\equiv3\bmod4$ is prime if and
%only if: \fbox{$a^{(m-1)/2}\equiv \pm1 \bmod m\ \ \ \forall a\leq
%2\log^2 m$})}}

\end{slide}


\begin{slide}\pageTransitionWipe{30}
\heading{Certified prime records}\psd

Top 10 Largest primes:\psd
\begin{center}
\begin{scriptsize}
\parstepwise{
\begin{tabular}{|c|l|l|l|l|l|l|}
\hline
1 & $2^{25964951}-1$ & 7816230 & Nowak & 2005 & Mersenne & 42?\\
2 & $2^{24036583}-1$ & 7235733 & Findley & 2004 & Mersenne & 41?\\
3 & $2^{20996011}-1$ & 6320430 & Shafer & 2003 & Mersenne & 40?\\
4 & $2^{13466917}-1$ & 4053946 & Cameron & 2001 & Mersenne & 39\\
5 & $27653\times2^{9167433}+1$ & 2759677 & Gordon & 2005 & &\\
6 & $28433\times2^{7830457}+1$ & 2357207 & SB7 & 2004 & &\\
7 & $2^{6972593}-1$ & 2098960 & Hajratwala & 1999 & Mersenne & 38\\
8 & $5359\times2^{5054502}+1$ & 1521561 & Sundquist & 2003 & &\\
9 & $4847\times2^{3321063}+1$ & 999744 & Hassler & 2005 & &\\
10 & $2^{3021377}-1$ & 909526 & Clarkson & 1998 & Mersenne & 37\\
\hline
\end{tabular}}\end{scriptsize}\end{center}\psd

\textcolor{BurntOrange}{\ding{45}} Mersenne's Numbers:$M_p=2^p-1$\psd


\textcolor{BurntOrange}{\ding{45}} For more see
\heading{\begin{small}\texttt{http://primes.utm.edu/primes/ }\end{small}}

\end{slide}



\begin{slide}\pageTransitionWipe{30}
\heading{The AKS deterministic primality test}\psd

\centerline{Department of Computer Science \& Engineering,}
\centerline{I.I.T. Kanpur, Agost 8, 2002.}\psd

\centerline{\includegraphics[width=4cm]{primality-group.jpg}}

\centerline{Nitin Saxena, Neeraj Kayal and Manindra Agarwal}\psd

\centerline{\textcolor{blue}{\emph{New deterministic,
polynomial--time, primality test.}}}\psd

Solves $\# 1$ open question in computational number theory\psd


\heading{http://www.cse.iitk.ac.in/news/primality.html}
\end{slide}

\begin{slide}\pageTransitionWipe{30}
\heading{How does the AKS work?}\psd

\noindent\textcolor{blue}{\textbf{Theorem. (AKS)}} \emph{Let
$n\in{\mathbb N}$. Assume $q, r$ primes, $S\subseteq {\mathbb N}$
finite:
\begin{itemize}
    \item $q|r-1$;
    \item $n^{(r-1)/q}\bmod r\not\in\{0,1\}$;
    \item $\gcd(n,b-b')=1,\ \ \forall b,b'\in S$ (distinct);
    \item $\binom{q+\#S-1}{\#S}\geq n^{2\lfloor\sqrt{r}\rfloor}$;
    \item $(x+b)^n=x^n+b$ in ${\mathbb Z}/n{\mathbb Z}[x]/(x^r-1),\ \ \forall b\in
    S$;
\end{itemize}
Then $n$ is a power of a prime} \ \hfill \textcolor{black}{{\tiny
Bernstein formulation}}\psd

 \textcolor{red}{Fouvry Theorem (1985)} $=\!\!\!=\!\!\!>\ \ \ \exists
r\approx\log^6n, s\approx\log^4n$\psd

\hspace*{3.65cm} $=\!\!\!=\!\!\!>\ \ \ $ \textcolor{blue}{AKS runs
in $O(\log^{17}n)$\\  \hspace*{4.7cm}operations in ${\mathbb
Z}/n{\mathbb Z}$.}\psd

Many simplifications and improvements: \textcolor{red}{Bernstein,
Lenstra, Pomerance.....}
\end{slide}

\begin{note}
Fouvry THM was originally used to show that FTL holds for
infinitely many exponents.


The running time $\log^{17} n$ is calculated using naive
arithmetics. Fast Fourier transform brings it down to $\log^{11}n$

\end{note}
%%%%%%%%%%%%%%%%%%%%%%%%%%%%
%%%%%%%%%%%%%%%%%%%%%%%%%%%%%
%%%%%%%%%%%%%%%%%%%%%%%%%%%%
%%%%%%%%%%%%%%%%%%%%%%%%%%%%%
%%%%%%%%%%%%%%%%%%%%%%%%%%%%
%%%%%%%%%%%%%%%%%%%%%%%%%%%%%
%%%%%%%%%%%%%%%%%%%%%%%%%%%%
%%%%%%%%%%%%%%%%%%%%%%%%%%%%%


\begin{slide}\pageTransitionWipe{30}
\heading{Why is RSA safe?}\psd

\parstepwise{\begin{itemize}

\item[\textcolor{black}{\ding{43}}] \step{It is clear that if
Charles can factor $M$,} \step{then he can also compute
$\varphi(M)$ and then also $d$ so to decrypt messages}

\item[\textcolor{black}{\ding{43}}] \step{Computing $\varphi(M)$ is
equivalent to completely factor $M$. In fact}
\item[\ ] \step{\ \ \ \ \ \ $\displaystyle{p,q=\frac{M-\varphi(M)+1\pm\sqrt{(M-\varphi(M)+1)^2-4M}}{2}}$}
\item[\textcolor{black}{\ding{43}}] \step{\textbf{\textcolor{blue}{RSA
Hypothesis.}} The only way to compute efficiently}
\item[\ ] \step{\ \ \ \ \ \ $\displaystyle{x^{1/e}\bmod M,\ \ \ \forall x\in{\mathbb Z}/M\mathbb Z}$}
\item[\ ] \step{\ \ \ \ \ \ (i.e. decrypt messages) is to factor $M$}
\item[\ ] \step{\ \ \ \ \ \ In other words}
\item[\ ] \step{\ \ \ \ \ \ \ \emph{\textcolor{black}{The two problems are
polynomially equivalent}}}
\end{itemize}}
\end{slide}

\begin{slide}\pageTransitionWipe{30}

\heading{Two kinds of Cryptography}\psd

\begin{itemize}
  \item[\textcolor{Emerald}{\ding{43}}] {\bf Private key (or symmetric)}
\begin{itemize}
  \item[\textcolor{red}{\ding{46}}] Lucifer
  \item[\textcolor{red}{\ding{46}}] DES
  \item[\textcolor{red}{\ding{46}}] AES
\end{itemize}\psd

  \item[\textcolor{Emerald}{\ding{43}}] {\bf Public key}
\begin{itemize}
  \item[\textcolor{red}{\ding{46}}] RSA
  \item[\textcolor{red}{\ding{46}}] Diffie--Hellmann
  \item[\textcolor{red}{\ding{46}}] Knapsack
  \item[\textcolor{red}{\ding{46}}] NTRU
\end{itemize}
\end{itemize}

\end{slide}



\end{document}
