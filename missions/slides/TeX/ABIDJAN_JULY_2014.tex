

\documentclass[10pt,final]{beamer} %,hyperref={pdfpagelabels=false},draft,handout,handout
%\let\Tiny=\tiny
\hfuzz=3pt
\vfuzz=3pt
\usefonttheme{professionalfonts} % using non standard fonts for beamer
\usefonttheme{serif} % default family is serif

\usepackage[english]{babel}
\usepackage{lmodern}
\usepackage[latin1]{inputenc}
\usepackage{times}
\usepackage{amsthm}
\usepackage{hyperref}
%\usepackage[T1]{fontenc}
\usepackage{tikz}
\usepackage{colortbl}
\usepackage{yfonts}
\usepackage{pifont}
\usepackage{translator} % comment this, if not available
\mode<article>
{
  \usepackage{times}
  \usepackage{mathptmx}
  \usepackage[left=1.5cm,right=6cm,top=1.5cm,bottom=3cm]{geometry}
}

 \newcommand{\Q}{\mathbb Q}
 \newcommand{\Z}{\mathbb Z}
 \newcommand{\N}{\mathbb N}
 \newcommand{\F}{\mathbb F}
 \newcommand{\C}{\mathbb C}
 \newcommand{\R}{\mathbb R}
% Common theorem-like environments

\theoremstyle{definition}
\newtheorem{exercise}[theorem]{\translate{Exercise}}
\newtheorem{rem}[theorem]{\translate{Remark}}
\newtheorem{conj}[theorem]{\translate{Conjecture}}
\newtheorem{proposition}[theorem]{\translate{Proposition}}
\newtheorem{notation}[theorem]{\translate{Notation}}
\newtheorem{Note}[theorem]{\translate{Note}}
\newtheorem{Block}[theorem]{\translate{}}


% New useful definitions:

\lecture[1]{Introduction to Elliptic Cryptosystems\\ \ \ \\
\small{An invitation to Elliptic curves}}
{Elliptic curves}
\date{}
\title[{Dipartim. Mat. \& Fis.}]{\insertlecture}
\subtitle{\ }
\author[\ \hspace{-2mm} Universit\`a Roma Tre]{Francesco Pappalardi}
\institute{Dipartimento di Matematica e Fisica\\
  Universit\`a Roma Tre}

% Beamer version theme settings

\useoutertheme[height=0pt,width=2cm,right]{sidebar}
\usecolortheme{rose,sidebartab}
\useinnertheme{circles}
\usefonttheme[only large]{structurebold}

\setbeamercolor{formul}{fg=black,bg=pink}
\setbeamercolor{sidebar right}{bg=black!15}
\setbeamercolor{structure}{fg=green!50!black}
\setbeamercolor{author}{parent=structure}
\setbeamercolor{postit}{fg=black,bg=yellow}
\setbeamercolor{greys}{fg=black,bg==black!25}
\setbeamerfont{title}{series=\normalfont,size=\LARGE}
\setbeamerfont{title in sidebar}{series=\bfseries}
\setbeamerfont{author in sidebar}{series=\bfseries}
\setbeamerfont*{item}{series=}
\setbeamerfont{frametitle}{size=}
\setbeamerfont{block title}{size=\small}
\setbeamerfont{subtitle}{size=\normalsize,series=\normalfont}
\setbeamertemplate{navigation symbols}{}
\setbeamertemplate{bibliography item}[book]
\setbeamertemplate{sidebar right}
{
  {\usebeamerfont{title in sidebar}%
    \vskip1.5em%
    \hskip3pt%
    \usebeamercolor[fg]{title in sidebar}%
    \insertshorttitle[width=2.1cm,respectlinebreaks]\par%   left,
    \vskip1.25em%
  }%
  {%
    \hskip3pt%
    \usebeamercolor[fg]{author in sidebar}%
    \usebeamerfont{author in sidebar}%
    \insertshortauthor[width=2cm,center,respectlinebreaks]\par%
    \vskip1em%
  }%
  \hbox to2cm{\hss\insertlogo\hss}
  \vskip1em%
  \insertverticalnavigation{2cm}%
  \vfill
  \hbox to 2cm{\hfill\usebeamerfont{subsection in
      sidebar}\strut\usebeamercolor[fg]{subsection in
      sidebar}\insertframenumber\hskip5pt}%
  \vskip3pt%
}%

\setbeamertemplate{title page}
{
  \vbox{}
  \vskip1em
  %{\huge Lecture \insertshortlecture\par}
  {\usebeamercolor[fg]{title}\usebeamerfont{title}\inserttitle\par}%
  \ifx\insertsubtitle\@empty%
  \else%
    \vskip0.25em%
    {\usebeamerfont{subtitle}\usebeamercolor[fg]{subtitle}\insertsubtitle\par}%
  \fi%
  \vskip1em\par
   \textbf{\Large{Journ\'ee d'Aritm\'etique}}\\
\textsl{Universit\`e de Cocody -- UFR Mat\'ematique et Informatique}\\ Abidjan\ \emph{Juillet 24, 2014},\par
  \vskip0pt plus1filll
  \leftskip=0pt plus1fill\insertauthor\par
  \insertinstitute\vskip1em
}

\logo{\includegraphics[width=1cm]{images/roma3.pdf}}

% Article version layout settings

\mode<article>

\makeatletter
\def\@listI{\leftmargin\leftmargini
  \parsep 0pt
  \topsep 5\p@   \@plus3\p@ \@minus5\p@
  \itemsep0pt}
\let\@listi=\@listI


\setbeamertemplate{frametitle}{\paragraph*{\insertframetitle\
    \ \small\insertframesubtitle}\ \par
}
\setbeamertemplate{frame end}{%
  \marginpar{\scriptsize\hbox to 1cm{\sffamily%
      \hfill\strut\insertframenumber}\hrule height .2pt}}
\setlength{\marginparwidth}{1cm}
\setlength{\marginparsep}{4.5cm}

\def\@maketitle{\makechapter}

\def\makechapter{
  \newpage
  \null
  \vskip 2em%
  {%
    \parindent=0pt
    \raggedright
    \sffamily
    \vskip8pt
    {\fontsize{36pt}{36pt}\selectfont Kapitel \insertshortlecture \par\vskip2pt}
    {\fontsize{24pt}{28pt}\selectfont \color{blue!50!black} \insertlecture\par\vskip4pt}
    {\Large\selectfont \color{blue!50!black} \insertsubtitle\par}
    \vskip10pt
  }
  \par
  \vskip 1.5em%
}

\let\origstartsection=\@startsection
\def\@startsection#1#2#3#4#5#6{%
  \origstartsection{#1}{#2}{#3}{#4}{#5}{#6\normalfont\sffamily\color{blue!50!black}\selectfont}}

\makeatother

\mode
<all>

% Typesetting Listings

\usepackage{listings}
\lstset{language=Java}

\alt<presentation>
{\lstset{%
  basicstyle=\footnotesize\ttfamily,
  commentstyle=\slshape\color{green!50!black},
  keywordstyle=\bfseries\color{blue!50!black},
  identifierstyle=\color{blue},
  stringstyle=\color{orange},
  escapechar=\#,
  emphstyle=\color{red}}
}
{
  \lstset{%
    basicstyle=\ttfamily,
    keywordstyle=\bfseries,
    commentstyle=\itshape,
    escapechar=\#,
    emphstyle=\bfseries\color{red}
  }
}

\begin{document}

\begin{frame}
\titlepage
\end{frame}

\section{Introduction}
\subsection{History}

\begin{frame}\frametitle{Proto--History (from \textsc{Wikipedia})}

\begin{columns}[c]
\begin{column}{6cm}\begin{small}
Giulio Carlo, Count Fagnano, and Marquis de Toschi (December 6, 1682 -- September 26, 1766)
was an Italian mathematician. He was probably the first to direct attention to the theory of
\emph{elliptic integrals}. Fagnano was born in Senigallia.\medskip

He made his higher studies at the \emph{Collegio Clementino} in Rome and there won great distinction,
except in mathematics, to which his aversion was extreme. Only after his college course he took up the study of
mathematics.\medskip

Later, without help from any teacher, he mastered mathematics from its foundations.\end{small}
\begin{block}{Some of His Achievements:}
\begin{itemize}
 \item $\pi=2i\log\frac{1-i}{1+1}$
 \item Length of \emph{Lemniscate}
\end{itemize}
\end{block}
\end{column}
\begin{column}{4cm}
\includegraphics[width=2.5cm]{images/fagnano.jpg}\\
\scriptsize{Carlo Fagnano}
\bigskip

\includegraphics[width=2.5cm]{images/ColegioClementino.jpg}\\
\scriptsize{Collegio Clementino}
\bigskip

\includegraphics[width=2.5cm]{images/lemniscate.pdf}\\
\scriptsize{\quad Lemniscate $(x^2+y^2)^2=2a^2(x^2-y^2)$
$\ell=4\int_0^a\frac{a^2dr}{\sqrt{a^4-r^4}}=\frac{a \sqrt\pi\Gamma(\frac54)}{\Gamma(\frac34)}$}
\end{column}
\end{columns}
\end{frame}

\subsection{length of ellipses}
\begin{frame}\frametitle{Length of Ellipses}
\begin{columns}[c]
\begin{column}{4cm}<1->\begin{small}
\centerline{
\begin{beamercolorbox}[shadow=true,center,rounded=true,wd=\textwidth]{formul}
$\mathcal{E}: \frac{x^2}4+\frac{y^2}{16}=1$
\end{beamercolorbox}}
\centerline{\includegraphics[width=2.3cm]{images/ellipse.pdf}}\end{small}\pause
\begin{scriptsize}
\begin{block}
{\scriptsize{The length of the arc of a plane curve $y=f(x)$, $f:[a,b]\rightarrow\R$
is:}}
$$\ell=\int_a^b\sqrt{1+(f'(t))^2}dt$$
\end{block}
\pause
\end{scriptsize}
\end{column}
\begin{column}{6cm}<2->
\begin{scriptsize}
Applying this formula to $\mathcal{E}$:
\centerline{\begin{beamercolorbox}[shadow=true,center,rounded=true,wd=\textwidth]{formul}
 \begin{align*}
\ell(\mathcal{E})&=4\int_0^4\sqrt{1+\left(\frac{d\sqrt{16(1-t^2/4)}}{dt}\right)^2}dt\\
&=4\int_0^1\sqrt{\frac{1+3x^2}{1-x^2}}dx\qquad x=t/2
                    \end{align*}
\end{beamercolorbox}}
\pause

If $y$ is the integrand, then we have the identity:

\centerline{\begin{beamercolorbox}[shadow=true,center,rounded=true,wd=\textwidth]{formul}
$y^2(1-x^2)=1+3x^2$
\end{beamercolorbox}}\pause

Apply the invertible change of variables:


 \centerline{\begin{beamercolorbox}[shadow=true,center,rounded=true,wd=\textwidth]{formul}
 $\begin{cases}
   x = 1-2/t\\ y=\frac u{t-1}
  \end{cases}
$\end{beamercolorbox}}

Arrive to

\centerline{\begin{beamercolorbox}[shadow=true,center,rounded=true,wd=\hsize]{formul}
$u^2 = t^3 - 4 t^2 + 6 t -3$
\end{beamercolorbox}}
\end{scriptsize}
\end{column}
\end{columns}
\end{frame}

\subsection{why Elliptic curves?}
\begin{frame}\frametitle{What are Elliptic Curves?}
\framesubtitle{Reasons to study them}\pause

Elliptic Curves
\begin{enumerate}[<+-| alert@+>]
\item are curves and finite groups at the same time
\item are non singular projective curves of \emph{genus} 1
\item have important applications in Algorithmic Number Theory and Cryptography
\item are the topic of the \alert{Birch and Swinnerton-Dyer conjecture} (one of the seven Millennium Prize Problems)
\item have a group law that is a consequence of the fact that they intersect every line in
exactly three points (in the projective plane over $\C$ and counted with multiplicity)
\item represent a mathematical world in itself ... Each of them does!!
\end{enumerate}
\end{frame}


\section{Weierstra\ss\ Equations}

\begin{frame}{The (general) Weierstra\ss\ Equation}

An elliptic curve $E$ over a $\F_q$ (finite field) is given by an equation
\centerline{\begin{beamercolorbox}[shadow=true,center,rounded=true,wd=\textwidth]{formul}
$E: y^2+a_1xy+a_3y=x^3+a_2x^2+a_4x+a_6$\end{beamercolorbox}}
where $a_1, a_3, a_2, a_4 ,a_6\in\F_q$ \pause

\begin{center}
 \includegraphics[width=60mm]{images/elliptic1.pdf}\pause
\llap{\includegraphics[width=60mm]{images/elliptic2.pdf}}\pause
\llap{\includegraphics[width=60mm]{images/elliptic3.pdf}}\pause
\llap{\includegraphics[width=60mm]{images/elliptic3b.pdf}}\pause
\llap{\includegraphics[width=60mm]{images/elliptic4.pdf}}\pause
\llap{\includegraphics[width=60mm]{images/elliptic5.pdf}}\pause
\llap{\includegraphics[width=60mm]{images/elliptic6.pdf}}\pause
\llap{\includegraphics[width=60mm]{images/elliptic7.pdf}}\pause
\llap{\includegraphics[width=60mm]{images/elliptic8.pdf}}\pause
\llap{\includegraphics[width=60mm]{images/elliptic9.pdf}}\pause
\llap{\includegraphics[width=60mm]{images/elliptic9b.pdf}}\pause
\llap{\includegraphics[width=60mm]{images/elliptic10.pdf}}\pause
\llap{\includegraphics[width=60mm]{images/elliptic10b.pdf}}\pause
\llap{\includegraphics[width=60mm]{images/elliptic6.pdf}}\pause
\end{center}

 \begin{beamercolorbox}[sep=1em,wd=6.5cm]{postit}
 The equation should not be \emph{singular}
 \end{beamercolorbox}
\end{frame}


\subsection{The Discriminant}

\begin{frame}
\frametitle{The Discriminant of an Equation}
\framesubtitle{The condition of absence of singular points in terms of $a_1, a_2, a_3, a_4, a_6$}
\pause
\begin{Definition}[The discriminant of a Weierstra\ss equation is the following quantity]
\centerline{\begin{beamercolorbox}[shadow=true,center,rounded=true,wd=\textwidth]{formul}
\begin{align*}
\Delta'_E&:=\frac{1}{2^43^3}\left(-a_1^5 a_3 a_4 - 8 a_1^3 a_2 a_3 a_4 - 16 a_1 a_2^2 a_3 a_4 + 36 a_1^2 a_3^2 a_4 \right. \\
  &-a_1^4 a_4^2 - 8 a_1^2 a_2 a_4^2 - 16 a_2^2 a_4^2 + 96 a_1 a_3 a_4^2 +64 a_4^3 + \\
  & a_1^6 a_6 + 12 a_1^4 a_2 a_6 + 48 a_1^2 a_2^2 a_6 + 64 a_2^3 a_6 -36 a_1^3 a_3 a_6\\
  &\left. - 144 a_1 a_2 a_3 a_6 - 72 a_1^2 a_4 a_6 - 288 a_2 a_4 a_6 +
  432 a_6^2  \right)
 \end{align*}
\end{beamercolorbox}}
 \end{Definition}
\pause

\begin{Definition}
 Two Weierstra\ss\ equations over $\F_q$ are said (affinely) equivalent if there exists a (affine) 
 of the following form

 $$\begin{cases}
x\longleftarrow u^2 x+r\\
y\longleftarrow u^3 y+ u^2s x + t
  \end{cases} r,s,t,u\in\F_q$$
\end{Definition}

\end{frame}

\begin{frame}
\frametitle{The Weierstra\ss\ equation}
\framesubtitle{Classification of simplified forms}

After applying a suitable affine transformation we can always assume that $E/\F_q (q=p^n)$
has a Weierstra\ss\ equation of the following form\pause

\begin{scriptsize}
 \begin{example}[Classification]
\centerline{\begin{tabular}{|l|c|l|}
\hline
 $E$ & $p$ & $\Delta_E$\\
\hline
&&\\
 $y^2=x^3+Ax+B$ & $\ge5$ & $4A^3+27B^2$\\
&&\\
$y^2+xy=x^3+a_2x^2+a_6$ & $2$ & $a_6^2$\\
&&\\
 $y^2+a_3y=x^3+a_4x+a_6$  & $2$ & $a_3^4$\\
&&\\
 $y^2=x^3+Ax^2+Bx+C$ & $3$ & $\!\begin{array}{l}
                               4A^3C-A^2B^2-18ABC\\+4B^3+27C^2
                              \end{array}$\\
&&\\\hline
\end{tabular}}
\end{example}
\end{scriptsize}\pause

\begin{definition}[Elliptic curve] An elliptic curve is the data of a non
singular Weierstra\ss\ equation (i.e. $\Delta_E\neq0$)
\end{definition}\pause

\small{
\alert{\textbf{Note:} If $p\ge3, \Delta_E\neq0\Leftrightarrow x^3+Ax^2+Bx+C$ has {no} double root}}
\end{frame}

\subsection{Elliptic curves \texorpdfstring{$/\F_2$}{F2}}
\begin{frame}
\frametitle{Elliptic curves over $\F_2$}

All possible Weierstra\ss\ equations over $\F_2$ are:\pause

\begin{beamerboxesrounded}[upper=block title example,lower=block body alerted,shadow=true]{Weierstra\ss\ equations over $\F_2$}
\begin{enumerate}
 \item $y^2+xy=x^3+x^2+1$
 \item$y^2+xy=x^3+1$
 \item$y^2+y=x^3+x$
 \item$y^2+y=x^3+x+1$
 \item$y^2+y=x^3$
 \item$y^2+y=x^3+1$
 \end{enumerate}
\end{beamerboxesrounded}
\pause

However the change of variables
$\begin{cases} x\leftarrow x+1\\ y\leftarrow y+x\end{cases}$ takes the sixth curve
into the fifth. Hence we can remove the sixth from the list.
\pause\bigskip

\begin{beamerboxesrounded}[upper=postit,lower=block body,shadow=true]{Fact:}
There are $5$ affinely inequivalent elliptic curves over $\F_2$
\end{beamerboxesrounded}
\end{frame}

\subsection{Elliptic curves \texorpdfstring{$/\F_3$}{F3}}
\begin{frame}
\frametitle{Elliptic curves in characteristic $3$}

Via a suitable transformation ($x\rightarrow u^2x+r, y\rightarrow u^3y+u^2sx+t$) over $\F_3$,  $8$ inequivalent
elliptic curves over $\F_3$ are found:\pause

\begin{beamerboxesrounded}[upper=block title example,lower=block body alerted,shadow=true]{Weierstra\ss\ equations over $\F_3$}
\begin{enumerate}
 \item $y^2=x^3+x$
 \item$y^2=x^3 - x$
 \item$y^2=x^3 - x +1$
 \item$y^2=x^3 - x -1$
 \item$y^2=x^3 + x^2 + 1$
 \item$y^2=x^3 + x^2 - 1$
 \item$y^2=x^3 - x^2 + 1$
 \item$y^2=x^3 - x^2 - 1$
 \end{enumerate}
\end{beamerboxesrounded}\pause

\begin{block}{Observations}
\begin{enumerate}[<+-| alert@+>]
\item Over  $\F_5$ there are 12 elliptic curves
          \item Over $\F_p$ there are approximately $2p$
         \end{enumerate}
\end{block}

% 1 x^3 + 2*x         2-torsion 2  pts= 2
% 2 x^3 -   x + 2     2-torsion 1  pts= 3
% 3 x^3 + x + 2       2-torsion 2  pts= 4
% 4 x^3 + x           2-torsion 3  pts= 4
% 5 x^3 - 2*x + 2     2-torsion 1  pts= 5
% 6 x^3 + 1           2-torsion 2  pts= 6
% 7 x^3 + 2           2-torsion 2  pts= 6
% 8 x^3 +2*x + 1      2-torsion 1  pts= 7
% 9 x^3 -  x          2-torsion 3  pts= 8
% 10 x^3 - x + 1      2-torsion 2  pts= 8
% 11 x^3 + x + 1      2-torsion 1  pts= 9
% 12 x^3 - 2*x        2-torsion 2  pts= 10

% p=5;S=0;for(a=0,p-1,for(b=0,p-1,if((4*a^3+27*b^2)%p>0,print1(S++" "x^3+a*x+b" 2-torsion "matsize(factormod(x^3+a*x+b,p))[1]);T=1;for(x=0,p-1,for(y=0,p-1,if((y^2-x^3-a*x-b)%p==0,T++)));print("  pts= "T))))

\end{frame}

% 
\section{The sum of points}
\begin{frame}
\frametitle{The definition of $E(\F_q)$}
\centerline{\begin{beamercolorbox}[shadow=true,left,rounded=true,wd=\textwidth]{formul}
Let $E/\F_q$ elliptic curve, $\infty:=[0,1,0]$. Set\\
\ \\
\scriptsize{$E(\F_q)=\{[X,Y,Z]\in\mathbb P_2(\F_q):\ Y^2Z+a_1XYZ+a_3YZ^2=X^3+a_2X^2Z+a_4XZ^2+a_6Z^3\}$}\\
\ \\
or equivalently\\
\ \\
\qquad $E(\F_q)=\{(x,y)\in \F_q^2:\ y^2+a_1xy+a_3y=x^3+a_2x^2+a_4x+a_6\}\cup\{\infty\}$
\end{beamercolorbox}}\pause

\ \hfill \begin{beamercolorbox}[shadow=true,left,rounded=true,wd=9cm]{postit}
 We can think either\pause
\begin{itemize}
 \item<1-> $E(\F_q)\subset\mathbb P_2(\F_q)$   \pause       \hfil$\dashrightarrow$ geometric advantages
 \item<2-> $E(\F_q)\subset\F_q^2\cup\{\infty\}$\pause \hfil$\dashrightarrow$ algebraic advantages
\end{itemize}\pause
\ \hfill$\infty$ might be though as the ``vertical direction''
\end{beamercolorbox}\pause

\begin{Definition}[line through points $P,Q\in E(\F_q)$]
$r_{P,Q}:\begin{cases}
                     \text{line through $P$ and }Q &\text{if }P\neq Q\\
                     \text{tangent line to $E$ at }P &\text{if }P=Q
                    \end{cases}$\hfill projective or affine
\end{Definition}\pause

\begin{itemize}[<+-| alert@+>]
\item if $\#(r_{P,Q}\cap E(\F_q))\ge2\ \Rightarrow\ \#(r_{P,Q}\cap E(\F_q))=3$\\
\hfill\scriptsize{\alert{if tangent line, contact point is counted with multiplicity}}  \item $r_{\infty,\infty}\cap E(\F_q)=\{\infty,\infty,\infty\}$%\vspace*{-4.4pt}
 % $\#(r_{P_1,P_1}\cap E(\F_q))=2$
 \item $r_{P,Q}: aX+bZ=0$ (vertical) $\Rightarrow \infty=[0,1,0]\in r_{P,Q}$
                    \vspace*{-4.4pt}
\end{itemize}

\end{frame}

\begin{frame}
\frametitle{History (from \textsc{Wikipedia})}

\begin{columns}[c]
\begin{column}{4.5cm}
\begin{small}
\textbf{Carl Gustav Jacob Jacobi} (10/12/1804 -- 18/02/1851) was a German mathematician,
who made fundamental contributions to elliptic functions, dynamics, differential equations,
and number theory.
\end{small}\\
\centerline{\includegraphics[width=1.8cm]{images/Jacobi.jpg}}
%\centerline{\scriptsize{Carl Gustav Jacob Jacobi}}\\
\begin{scriptsize}\begin{block}{Some of His Achievements:}
\begin{itemize}
 \item Theta and elliptic function
 \item Hamilton Jacobi Theory
 \item Inventor of determinants
 \item Jacobi Identity\\
 \tiny{ $[A,[B,C]]+[B,[C,A]]+[C,[A,B]]=0$}
\end{itemize}
\end{block}\end{scriptsize}
\end{column}\pause
\begin{column}{5.5cm}\vspace*{-16.3pt}
\begin{center}
\includegraphics[width=5.5cm]{images/add1.pdf}\pause
\llap{\includegraphics[width=5.5cm]{images/add2.pdf}}\pause
\llap{\includegraphics[width=5.5cm]{images/add3.pdf}}\pause
\llap{\includegraphics[width=5.5cm]{images/add5.pdf}}\pause
\llap{\includegraphics[width=5.5cm]{images/add6.pdf}}\pause
\llap{\includegraphics[width=5.5cm]{images/add7.pdf}}\pause
\llap{\includegraphics[width=5.5cm]{images/add1.pdf}}\pause
\llap{\includegraphics[width=5.5cm]{images/add8.pdf}}\pause
\llap{\includegraphics[width=5.5cm]{images/add9.pdf}}\pause
\llap{\includegraphics[width=5.5cm]{images/ad10.pdf}}\pause
\llap{\includegraphics[width=5.5cm]{images/ad11.pdf}}\pause
\llap{\includegraphics[width=5.5cm]{images/ad12.pdf}}\pause
\llap{\includegraphics[width=5.5cm]{images/add1.pdf}}\pause
\llap{\includegraphics[width=5.5cm]{images/ad13.pdf}}\pause
\llap{\includegraphics[width=5.5cm]{images/ad14.pdf}}\pause
\llap{\includegraphics[width=5.5cm]{images/ad15.pdf}}\pause
\llap{\includegraphics[width=5.5cm]{images/add7.pdf}}\pause
\end{center}
\small{
$r_{P,Q}\cap E(\F_q)=\{P,Q,R\}$\\
$r_{R,\infty}\cap E(\F_q)=\{\infty,R,R'\}$}
\centerline{\begin{beamercolorbox}[shadow=true,center,rounded=true,wd=2cm]{formul}
$P+_E Q:=R'$\pause
            \end{beamercolorbox}}\smallskip

 \small{$r_{P,\infty}\cap E(\F_q)=\{P,\infty,P'\}$}\\
 \centerline{\begin{beamercolorbox}[shadow=true,center,rounded=true,wd=2cm]{formul}
             $-P:=P'$
            \end{beamercolorbox}}

\end{column}
\end{columns}
\end{frame}

\begin{frame}
\frametitle{Properties of the operation ``$+_E$''}

\begin{Theorem}
 The addition law on $E(\F_q)$ has the following
properties:
\begin{enumerate}[<+-| alert@+>][(a)]
 \item $P+_EQ\in E(\F_q)\hfill\forall P,Q\in E(\F_q)$
 \item  $P+_E\infty=\infty+_E P=P\hfill\forall P\in E(\F_q)$
 \item  $P+_E(-P)=\infty\hfill\forall P\in E(\F_q)$
 \item  $P+_E(Q +_E R)=(P+_E Q)+_E R\hfill\forall P,Q,R\in E(\F_q)$
 \item  $P+_E Q=Q +_E P\hfill\forall P,Q\in E(\F_q)$
\end{enumerate}
 \end{Theorem}\pause

\begin{itemize}[<+-| alert@+>]
%  \item By ``a point of $E/\F_q$ ($P\in E$)'' we mean $P\in E(\bar{\F}_q)$
%  in analogy for $E/\Q$ where ``a point of $E$'' means  $P\in E(\C)$
 \item $\left(E(\F_q),+_E\right)$  \alert{commutative group}
 \item All group properties are easy except \alert{associative law (d)}
 \item Geometric proof of associativity uses \emph{Pappo's Theorem}
 \item We shall comment on how to do it by explicit computation
 \item can substitute $\F_q$ with any field $K$; Theorem holds for $\left(E(K),+_E\right)$
\item In particular, if $E/\F_q$, can consider the groups $E(\overline{\F}_q)$ or $E(\F_{q^n})$
\end{itemize}
\end{frame}

\begin{frame}
\frametitle{Formulas for Addition on $E$ (Summary)}
\centerline{\begin{beamercolorbox}[shadow=true,center,rounded=true,wd=\textwidth]{formul}
$E: y^2+a_1xy+a_3y=x^3+a_2x^2+a_4x+a_6$\end{beamercolorbox}}\pause
$P_1 = (x_1, y_1), P_2 = (x_2, y_2)\in E(\F_q)\setminus\{\infty\}$,
\begin{beamerboxesrounded}[upper=block title example,lower=block body alerted,shadow=true]{Addition Laws for the sum of affine points}
\begin{itemize}[<+-| alert@+>]
 \item If $P_1\neq P_2$
\begin{itemize}
 \item $x_1 = x_2\ \hfill\Rightarrow\hfil$\ \ \
\begin{beamercolorbox}[shadow=true,center,rounded=true,wd=2cm]{formul}$P_1 +_E P_2 = \infty$
\end{beamercolorbox}
 \item $x_1 \neq x_2$\\
\centerline{\begin{beamercolorbox}[shadow=true,center,wd=4cm]{postit}
             $\lambda=\frac{y_2-y_1}{x_2-x_1}\qquad \nu=\frac{y_1x_2-y_2x_1}{x_2-x_1}$
            \end{beamercolorbox}}
 \end{itemize}
\item If $P_1 = P_2$
\begin{itemize}
 \item $2y_1+a_1x+a_3 = 0\ \hfill\Rightarrow\hfil$\ \ \
\begin{beamercolorbox}[shadow=true,center,rounded=true,wd=3cm]{formul}$P_1 +_E P_2 = 2P_1 = \infty$\end{beamercolorbox}
\item $2y_1+a_1x+a_3\neq 0$\\
\centerline{\begin{beamercolorbox}[shadow=true,center,wd=7cm]{postit}
$\lambda=\frac{3x_1^2+2a_2x_1+a_4-a_1y_1}{2y_1+a_1x+a_3}, \nu=-\frac{a_3y_1+x_1^3-a_4x_1-2a_6}{2y_1+a_1x_1+a_3}$
            \end{beamercolorbox}}
\end{itemize}
\end{itemize}\pause

Then\\
\centerline{\begin{beamercolorbox}[shadow=true,center,rounded=true,wd=10cm]{formul}
\scriptsize{$P_1 +_E P_2 = ({\color[cmyk]{0,1,1,0.5}\lambda^2-a_1\lambda-a_2-x_1-x_2},
{\color[cmyk]{1,0,1,0.5}-\lambda^3-a_1^2\lambda+(\lambda+a_1)(a_2+x_1+x_2)-a_3-\nu})$}
            \end{beamercolorbox}}
\end{beamerboxesrounded}
\end{frame}

\begin{frame}
\frametitle{Formulas for Addition on $E$ (Summary for special equation)}
\centerline{\begin{beamercolorbox}[shadow=true,center,rounded=true,wd=\textwidth]{formul}
$E: y^2=x^3+Ax+B$\end{beamercolorbox}}
$P_1 = (x_1, y_1), P_2 = (x_2, y_2)\in E(\F_q)\setminus\{\infty\}$,
\begin{beamerboxesrounded}[upper=block title example,lower=block body alerted,shadow=true]{Addition Laws for  the sum of affine points}
\begin{itemize}
 \item If $P_1\neq P_2$
\begin{itemize}
 \item $x_1 = x_2\ \hfill\Rightarrow\hfil$\ \ \
\begin{beamercolorbox}[shadow=true,center,rounded=true,wd=2cm]{formul}$P_1 +_E P_2 = \infty$
\end{beamercolorbox}
 \item $x_1 \neq x_2$\\
\centerline{\begin{beamercolorbox}[shadow=true,center,wd=4cm]{postit}
             $\lambda=\frac{y_2-y_1}{x_2-x_1}\qquad \nu=\frac{y_1x_2-y_2x_1}{x_2-x_1}$
            \end{beamercolorbox}}
 \end{itemize}
\item If $P_1 = P_2$
\begin{itemize}
 \item $y_1 = 0\ \hfill\Rightarrow\hfil$\ \ \
\begin{beamercolorbox}[shadow=true,center,rounded=true,wd=3cm]{formul}$P_1 +_E P_2 = 2P_1 = \infty$\end{beamercolorbox}
\item $y_1\neq 0$\\
\centerline{\begin{beamercolorbox}[shadow=true,center,wd=7cm]{postit}
$\lambda=\frac{3x_1^2+A}{2y_1}, \nu=-\frac{x_1^3-Ax_1-2B}{2y_1}$
            \end{beamercolorbox}}
\end{itemize}
\end{itemize}

Then\\
\centerline{\begin{beamercolorbox}[shadow=true,center,rounded=true,wd=7cm]{formul}
\small{$P_1 +_E P_2 = ({\color[cmyk]{0,1,1,0.5}\lambda^2-x_1-x_2},
{\color[cmyk]{1,0,1,0.5}-\lambda^3+\lambda(x_1+x_2)-\nu})$}
            \end{beamercolorbox}}
\end{beamerboxesrounded}

\end{frame}

\section{Examples}
\subsection{Structure of \texorpdfstring{$E(\F_2)$}{E(F2)}}
\begin{frame}
\frametitle{EXAMPLE: Elliptic curves over $\F_2$}

From our previous list:
\begin{block}{Groups of points}
\begin{tabular}{|l|c|l|}
\hline
 $E$ & $E(\F_2)$ & $|E(\F_2)|$\\
\hline
&&\\
 $y^2+xy=x^3+x^2+1$ & $\{\infty,(0,1)\}$& $2$\\
&&\\
$y^2+xy=x^3+1$ & $\{\infty,(0,1),(1,0),(1,1)\}$ & $4$\\
&&\\
$y^2+y=x^3+x$&$\{\infty,(0,0),(0,1),$ &\\ &$(1,0),(1,1)\}$&$5$\\
&&\\
 $y^2+y=x^3+x+1$ &$\{\infty\}$&$1$\\
&&\\
$y^2+y=x^3$ & $\{\infty,(0,0), (0,1)\}$ & $3$ \\
&&\\\hline
\end{tabular}
\end{block}
\pause
So for each curve $E(\F_2)$ is cyclic except possibly for the second for which we need to distinguish between
$C_4$ and $C_2\oplus C_2$.\pause

\ \hfill \begin{beamercolorbox}[center,wd=9cm]{postit}
Note: each $C_i, i=1,\ldots,5$ is represented by a curve $/\F_2$
            \end{beamercolorbox}
\end{frame}


\subsection{Structure of \texorpdfstring{$E(\F_3)$}{E(F3)}}
\begin{frame}
\frametitle{EXAMPLE: Elliptic curves over $\F_3$}
From our previous list:

\begin{block}{Groups of points}%\begin{center}
\begin{tabular}{|l|r|c|c|}
\hline
$i$ & $E_i$ & $E_i(\F_3)$ &$|E_i(\F_3)|\!$\\
\hline
$1$& $y^2=x^3+x$ & \scriptsize{$\{\infty,(0,0),(2,1),(2,2)\}$}& $4$\\
\hline
$2$&$y^2=x^3 - x$ & \scriptsize{$\{\infty,(1,0),(2,0),(0,0)\}$} & $4$\\
\hline
$3$&$y^2=x^3 - x +1$&\tiny{$\begin{array}{r}\{\infty,(0,1),(0,2),(1,1),(1,2),(2,1),\\(2,2)\}\end{array}$} & $7$\\
\hline
$4$&$y^2=x^3 - x -1$  &\scriptsize{$\{\infty\}$}&$1$\\
\hline
$5$&$y^2=x^3 + x^2 - 1$ & \scriptsize{$\{\infty,(1,1), (1,2)\}$} & $3$ \\
\hline
$6$&$y^2=x^3 + x^2 + 1$ & \Tiny{$\{\infty,(0,1), (0,2), (1,0),(2,1), (2,2)\}$} & $6$ \\
\hline
$7$&$y^2=x^3 - x^2 + 1$ & \scriptsize{$\{\infty,(0,1), (0,2), (1,1), (1,2),\}$} & $5$ \\
\hline
$8$&$y^2=x^3 - x^2 - 1$ & \scriptsize{$\{\infty,(2,0))\}$} & $2$ \\
\hline
\end{tabular}
%\end{center}
\end{block}
\pause
Each $E_i(\F_3)$ is cyclic except possibly for $E_1(\F_3)$ and $E_2(\F_3)$ that could be either
$C_4$ or $C_2\oplus C_2$. We shall see that:\pause

\centerline{\begin{beamercolorbox}[shadow=true,center,rounded=true,wd=7cm]{formul}
$E_1(\F_3)\cong C_4\qquad\text{and}\qquad E_2(\F_3)\cong C_2\oplus C_2$
\end{beamercolorbox}}\pause

\ \hfill \begin{beamercolorbox}[center,wd=9cm]{postit}
Note: each $C_i, i=1,\ldots,7$ is represented by a curve $/\F_3$
            \end{beamercolorbox}


\end{frame}


\section{Points of finite order}

\subsection{Points of order 2}
\begin{frame}\frametitle{Determining points of order $2$}
Let $P=(x_1,y_1)\in E(\F_q)\setminus\{\infty\},$\\ \pause
\centerline{
 \begin{beamercolorbox}[rounded=true,shadow=true,wd=9cm,center]{formul}
$P$ has order $2\ \Longleftrightarrow\ 2P=\infty\ \Longleftrightarrow\ P=-P$
\end{beamercolorbox}}\pause
So
\centerline{\small{
 \begin{beamercolorbox}[rounded=true,shadow=true,wd=10cm,center]{formul}
$-P=(x_1,-a_1x_1-a_3-y_1)=(x_1,y_1)=P\ \pause \Longrightarrow\ 2y_1=-a_1x_1-a_3$\end{beamercolorbox}}}\pause\medskip

If $p\neq2$, can assume $E: y^2=x^3+Ax^2+Bx+C$\pause

\centerline{\small{
 \begin{beamercolorbox}[rounded=true,shadow=true,wd=10cm,center]{formul}
$-P=(x_1,-y_1)=(x_1,y_1)=P\ \pause \Longrightarrow\ y_1=0,
x_1^3+Ax_1^2+Bx_1+C=0$\hfill
\end{beamercolorbox}}}\pause\medskip

\begin{Note}
\begin{itemize}[<+-| alert@+>]
 \item the number of points of order $2$ in $E(\F_q)$ equals the number of roots of $X^3+Ax^2+Bx+C$ in $\F_q$
 \item roots are distinct since discriminant $\Delta_E\neq0$
 \item $E(\F_{q^6})$ has always $3$ points of order $2$ if $E/\F_q$
 \item $E[2]:=\{P\in E(\bar{\F}_q): 2P=\infty\}\cong C_2\oplus C_2$
\end{itemize}
\end{Note}

% $\begin{cases}
%    2y=-a_1x-a_3\\
%    y^2+a_1xy+a_3y=x^3+a_2x^2+a_4x+a_6
%   \end{cases}\longrightarrow$, $\begin{cases}
%    2y=-a_1x-a_3\\
%    x^3+(a_2+a_1^2/4)x^2+(a_4+a_1a_3/2)x+a_6+a_3^2/4=0
%   \end{cases}$

\end{frame}

\begin{frame}\frametitle{Determining points of order $2$ (continues)}

\begin{itemize}[<+-| alert@+>]
\item If $p=2$ and $E: y^2+a_3y=x^3+a_2x^2+a_6$\pause

 \begin{beamercolorbox}[rounded=true,shadow=true,wd=10cm,center]{formul}
$-P=(x_1,a_3+y_1)=(x_1,y_1)=P\ \pause \Longrightarrow\ a_3=0$\end{beamercolorbox}\pause\medskip

Absurd ($a_3=0$) and there are no points of order $2$.
%$\begin{cases}
 %  x=a_3/a_1\\
  % y^2+a_1xy+a_3y+x^3+a_2x^2+a_4x+a_6=0
  %\end{cases}\longrightarrow$,
\item If $p=2$ and $E: y^2+xy=x^3+a_4x+a_6$\pause

 \begin{beamercolorbox}[rounded=true,shadow=true,wd=10cm,center]{formul}
$-P=(x_1,x_1+y_1)=(x_1,y_1)=P\ \pause \Longrightarrow\ x_1=0,y_1^2=a_6$\end{beamercolorbox}\pause\medskip

So there is exactly one point of order $2$ namely $(0,\sqrt{a_6})$
\end{itemize}\pause

\begin{Definition}{$2$--torsion points}
$$E[2]=\{P\in E: 2P=\infty\}.$$
\end{Definition}
In conclusion
$$E[2]\cong \begin{cases}
C_2\oplus C_2 &\text{if }p>2\\
C_2           &\text{if }p=2, E: y^2+xy=x^3+a_4x+a_6\\
\{\infty\}    &\text{if }p=2, E: y^2+a_3y=x^3+a_2x^2+a_6
\end{cases}
$$

\end{frame}

\begin{frame}
\frametitle{Elliptic curves over $\F_2, \F_3$ and $\F_5$}
\begin{small}
\begin{block}{Each curve $/\F_2$ has cyclic $E(\F_2)$.}
\begin{tabular}{|l|c|l|}
\hline
 $E$ & $E(\F_2)$ & $|E(\F_2)|$\\
\hline
 $y^2+xy=x^3+x^2+1$ & $\{\infty,(0,1)\}$& $2$\\
\hline
$y^2+xy=x^3+1$ & $\{\infty,(0,1),(1,0),(1,1)\}$ & $4$\\
\hline
$y^2+y=x^3+x$&$\{\infty,(0,0),(0,1),(1,0),(1,1)\}$&$5$\\
\hline
$y^2+y=x^3+x+1$ &$\{\infty\}$&$1$\\
\hline
$y^2+y=x^3$ & $\{\infty,(0,0), (0,1)\}$ & $3$ \\
\hline
\end{tabular}
\end{block}\end{small}
\pause
\begin{itemize}
 \item $E_1: y^2=x^3+x\qquad\qquad E_2:  y^2=x^3-x$\\
\centerline{\begin{beamercolorbox}[shadow=true,center,rounded=true,wd=7.5cm]{formul}
$E_1(\F_3)\cong C_4\qquad\text{and}\qquad E_2(\F_3)\cong C_2\oplus C_2$
\end{beamercolorbox}}
\item $E_3: y^2=x^3+x\qquad\qquad E_4: y^2=x^3+x+2$\\
\centerline{\begin{beamercolorbox}[shadow=true,center,rounded=true,wd=7.5cm]{formul}
$E_3(\F_5)\cong C_2\oplus C_2\qquad\text{and}\qquad E_4(\F_5)\cong C_4$
\end{beamercolorbox}}
\item $E_5: y^2=x^3+4x\qquad\qquad E_6: y^2=x^3+4x+1$\\
\centerline{\begin{beamercolorbox}[shadow=true,center,rounded=true,wd=7.5cm]{formul}
$E_5(\F_5)\cong C_2\oplus C_4\qquad\text{and}\qquad E_6(\F_5)\cong C_8$
\end{beamercolorbox}}
\end{itemize}
\end{frame}

\subsection{Points of order 3}
\begin{frame}\frametitle{Determining points of order $3$}
Let  $P=(x_1,y_1)\in E(\F_q)$
\centerline{
 \begin{beamercolorbox}[rounded=true,shadow=true,wd=9cm,center]{formul}
$P$ has order $3\ \Longleftrightarrow\ 3P=\infty\ \Longleftrightarrow\ 2P=-P$
\end{beamercolorbox}}\pause\smallskip

So, if $p>3$ and $E: y^2=x^2+Ax+B$\\
 \begin{beamercolorbox}[rounded=true,shadow=true,wd=9cm,center]{postit}
$2P=(x_{2P},y_{2P})=2(x_1,y_1)=({\color[cmyk]{0,1,1,0.5}\lambda^2-2x_1},
{\color[cmyk]{1,0,1,0.5}-\lambda^3+2\lambda x_1-\nu})$
\end{beamercolorbox}\pause\smallskip
\hfill where
$\lambda=\frac{3x_1^2+A}{2y_1}, \nu=-\frac{x_1^3-Ax_1-2B}{2y_1}$.\pause

 \begin{beamercolorbox}[rounded=true,shadow=true,wd=5.5cm,center]{formul}
$P$ has order $3\ \Longleftrightarrow\ x_{2P}=x_1$
\end{beamercolorbox}\pause%\smallskip

\centerline{
 \begin{beamercolorbox}[rounded=true,shadow=true,wd=9cm,center]{formul}
Substituting $\lambda$,\quad \pause\ $x_{2P}-x_1=\frac{-3x_1^4-6Ax_1^2-12Bx_1+A^2}{4(x_1^3+Ax_1+4B)}=0$
\end{beamercolorbox}}\pause

\begin{Note}
\begin{itemize}[<+-| alert@+>]
 \item $\psi_3(x):= 3x^4+6Ax^2+12Bx-A^2$ the $3^{\text{rd}}$ \emph{division} polynomial
 \item $(x_1,y_1)\in E(\F_q)$ has order $3\quad \Rightarrow \psi_3(x_1)=0$
 \item $E(\F_q)$ has at most $8$ points of order $3$
 \item If $p\neq 3$, $E[3]:=\{P\in E: 3P=\infty\}\cong C_3\oplus C_3$
\end{itemize}
 \end{Note}
\end{frame}

\begin{frame}\frametitle{Determining points of order $3$ (continues)}

\begin{Note} Let $E: y^2=x^3+Ax^2+Bx+C, A,B,C\in\F_{3^n}$. If $P=(x_1,y_1)\in E(\F_{3^n})$
has order $3$, then
\begin{enumerate}[<+-| alert@+>]
 \item $Ax_1^3+AC-B^2=0$
 \item $E[3]\cong C_3$ if $A\neq0$ and $E[3]=\{\infty\}$ otherwise
\end{enumerate}
\end{Note}\pause

\begin{example}
If $E: y^2=x^3+x+1$, then $\#E(\F_5)=9$.\pause
$$\psi_3(x)=(x + 3)(x + 4)(x^2 + 3x + 4)$$
Hence
\centerline{$E[3]=\left\{
\infty,(2,\pm1),(1,\pm\sqrt{3}),(1\alert{\pm}2\sqrt{3},\pm(1\alert{\pm}\sqrt{3}))\right\}$}\pause
\begin{enumerate}[<+-| alert@+>]
 \item $E(\F_5)=\{\infty,(2,\pm1),(0,\pm1),(3,\pm1),(4,\pm2)\}\cong C_9$
 \item Since $\F_{25}=\F_5[\sqrt{3}]\quad\Rightarrow\quad  E[3]\subset E(\F_{25})$
 \item $\#E(\F_{25})=27\quad\Rightarrow\quad E(\F_{25})\cong C_3\oplus C_9$
\end{enumerate}


\end{example}
\end{frame}

\begin{frame}\frametitle{Determining points of order $3$ (continues)}

\begin{scriptsize}
\begin{block}{Inequivalent curves $/\F_7$ with $\#E(\F_7)=9$.}
\begin{tabular}{|l|c|c|c|}
\hline
 $E$ & $\psi_3(x)$ & $E[3]\cap E(\F_7)$ & $\!\!\!E(\F_7)\cong\!\!\!$\\
\hline
 $\!\!y^2=x^3+2\!\!$ & $x(x + 1)(x + 2)(x + 4)$ &\tiny{$\!\!\!\left\{\!\!\!\begin{array}{l}
\infty,(0,\pm3),(-1,\pm1),\!\!\! \\ (5,\pm1),(3,\pm1)\end{array}\!\!\!\!\right\}\!\!$}
& $\!\!\!C_3\oplus C_3\!\!\!$\\
\hline
$\!\!y^2=x^3+3x+2\!\!$ & $\!\!(x + 2)(x^3 + 5x^2 + 3x + 2)\!\!$ & $\{\infty,(5,\pm3)\}$ & $C_9$ \\
\hline
$\!\!y^2=x^3+5x+2\!\!$ & $\!\!(x + 4)(x^3 + 3x^2 + 5x + 2)\!\!$ & $\{\infty,(3,\pm3)\}$ & $C_9$ \\
\hline
$\!\!y^2=x^3+6x+2\!\!$ & $\!\!(x + 1)(x^3 + 6x^2 + 6x + 2)\!\!$ & $\{\infty,(6,\pm3)\}$ & $C_9$ \\
\hline
\end{tabular}
\end{block}\end{scriptsize}\pause

\begin{block}
{Can one count the number of inequivalent $E/\F_q$ with $\#E(\F_q)=r$?}
%\pause \ \hfill \alert{Answer:} \pause Next Time!!
\end{block}

\begin{example}[A curve over $\F_4=\F_2(\xi), \xi^2=\xi+1;\qquad E: y^2+y=x^3$]\pause
 We know $E(\F_2)=\{\infty, (0,0), (0,1)\}\subset E(\F_4).$\pause\\
 \begin{scriptsize}$E(\F_4)=\{\infty,(0,0),(0,1),(1,\xi),(1,\xi+1),(\xi,\xi),(\xi,\xi+1),
 (\xi+1,\xi),(\xi+1,\xi+1)\}$\end{scriptsize} \pause

\begin{small}\centerline{
\begin{beamercolorbox}[rounded=true,shadow=true,wd=\textwidth,center]{postit}
$\psi_3(x)=x^4+x=x(x+1)(x+\xi)(x+\xi+1)\Rightarrow E(\F_4)\cong C_3\oplus C_3$
\end{beamercolorbox}}
\end{small}
\end{example}

\begin{Note}[Suppose $(x_0,y_0)\in E/\F_{2^n}$ has order $3$. Then]
\begin{enumerate}[<+-| alert@+>]
  \item $E: y^2+a_3y=x^3+a_4x+a_6\ \Rightarrow\ x_0^4+a_3^2x_0+(a_4a_3)^2=0$
  \item $E: y^2+xy=x^3+a_2x^2+a_6\ \Rightarrow\ x_0^4+x_0^3+a_6=0$
\end{enumerate}
\end{Note}
\end{frame}

\subsection{Points of finite order}

\begin{frame}\frametitle{Determining points of order (dividing) $m$}\pause
\begin{definition}[$m$--torsion point] Let $E/K$ and let $\bar{K}$ an \emph{algebraic closure of $K$}.

\centerline{\begin{beamercolorbox}[rounded=true,shadow=true,wd=5cm,center]{postit}
$E[m]=\{P\in E(\bar{K}):\ mP=\infty\}$\end{beamercolorbox}}
\end{definition}\pause

\begin{theorem}[Structure of Torsion Points]
Let $E/K$  and $m\in\N$. If $p=\operatorname{char}(K)\nmid m$,\pause

\centerline{\begin{beamercolorbox}[rounded=true,shadow=true,wd=3.5cm,center]{formul}
$E[m]\cong C_m\oplus C_m$\end{beamercolorbox}}

If $m=p^rm', p\nmid m'$,

\centerline{\begin{beamercolorbox}[rounded=true,shadow=true,wd=8cm,center]{formul}
$E[m]\cong C_m\oplus C_{m'}\qquad\text{or}\qquad E[m] \cong C_{m'}\oplus C_{m'}$\end{beamercolorbox}}
\end{theorem}\pause

\begin{block}\ \hfill
$E/\F_p$ is called $\begin{cases} \text{\emph{ordinary}} &\text{ if }E[p]\cong C_p\\
                \text{\emph{supersingular}} &\text{ if }E[p]=\{\infty\}
                    \end{cases}$\end{block}
\end{frame}

\subsection{The group structure}
\begin{frame}\frametitle{Group Structure of $E(\F_q)$}

\begin{corollary} Let $E/\F_q$. $\exists n,k\in\mathbb N$ are such that
\centerline{\begin{beamercolorbox}[rounded=true,shadow=true,wd=6cm,center]{formul}
$$E(\F_q)\cong C_n\oplus C_{nk}$$\end{beamercolorbox}}
\end{corollary}\pause

\begin{proof}
From classification Theorem of finite abelian group\\
\centerline{$E(\F_q)\cong  C_{n_1}\oplus C_{n_2}\oplus\cdots\oplus C_{n_r}$}
with $n_i|n_{i+1}$ for $i\ge1$.\pause

Hence $E(\F_q)$ contains $n_1^r$ points of order dividing $n_1$. From
\emph{Structure of Torsion Theorem}, $\#E[n_1]\le n_1^2$.
So $r\le2$\end{proof}\pause

\begin{theorem}[Corollary of Weil Pairing]  Let $E/\F_q$ and $n,k\in\mathbb N$ s.t.
$E(\F_q)\cong C_n\oplus C_{nk}.$
Then $n\mid q-1$.
\end{theorem}\pause
\end{frame}

\section{Important Results}
\subsection{Hasse's Theorem}
\begin{frame}
\begin{theorem}[Hasse]
Let $E$ be an elliptic curve over the finite field $\F_q$. Then the order of $E(\F_q)$
satisfies
$$\left|q+1-\#E(\F_q)\right|\le 2\sqrt q.$$
\end{theorem}\pause

So \alert{$\#E(\F_q)\in [(\sqrt q -1)^2, (\sqrt q+1)^2]$} the \emph{Hasse interval} ${\mathcal I}_q$

\begin{tiny}
 \begin{example}[Hasse Intervals]
\centerline{\begin{tabular}{|l|l|}
\hline
 $q$ & ${\mathcal I}_q$\\
\hline
$2$ & $\{1, 2, 3, 4, 5\}$\\
$3$ & $\{1, 2, 3, 4, 5, 6, 7\}$\\
$4$ & $\{1, 2, 3, 4, 5, 6, 7, 8, 9 \}$\\
$5$ & $\{2, 3, 4, 5, 6, 7, 8, 9, 10\}$\\
$7$ & $\{3, 4, 5, 6, 7, 8, 9, 10, 11, 12, 13\}$\\
$8$ & $\{4, 5, 6, 7, 8, 9, 10, 11, 12, 13, 14\}$\\
$9$ & $\{4, 5, 6, 7, 8, 9, 10, 11, 12, 13, 14, 15, 16\}$\\
$11$ & $\{6, 7, 8, 9, 10, 11, 12, 13, 14, 15, 16, 17, 18\}$\\
$13$ & $\{7, 8, 9, 10, 11, 12, 13, 14, 15, 16, 17, 18, 19, 20, 21\}$\\
$16$ & $\{9, 10, 11, 12, 13, 14, 15, 16, 17, 18, 19, 20, 21, 22, 23, 25\}$\\
$17$ & $\{10, 11, 12, 13, 14, 15, 16, 17, 18, 19, 20, 21, 22, 23, 24, 25, 26\}$\\
$19$ & $\{12, 13, 14, 15, 16, 17, 18, 19, 20, 21, 22, 23, 24, 25, 26, 27, 28\}$\\
$23$ & $\{15, 16, 17, 18, 19, 20, 21, 22, 23, 24, 25, 26, 27, 28, 29, 30, 31, 32,
 33\}$\\
$25$ & $\{16, 17, 18, 19, 20, 21, 22, 23, 24, 25, 26, 27, 28, 29, 30, 31, 32, 33,
 34, 35, 36\}$\\
$27$ & $\{18, 19, 20, 21, 22, 23, 24, 25, 26, 27, 28, 29, 30, 31, 32, 33, 34, 35,
 36, 37, 38\}$\\
$29$ & $\{20, 21, 22, 23, 24, 25, 26, 27, 28, 29, 30, 31, 32, 33, 34, 35, 36, 37,
 38, 39, 40\}$\\
$31$ & $\{21, 22, 23, 24, 25, 26, 27, 28, 29, 30, 31, 32, 33, 34, 35, 36, 37, 38,
 39, 40, 41, 42, 43 \}$\\
$32$ & $\{22, 23, 24, 25, 26, 27, 28, 29, 30, 31, 32, 33, 34, 35, 36, 37, 38, 39,
 40, 41, 42, 43, 44\}$\\  \hline
\end{tabular}}
\end{example}
\end{tiny}
\end{frame}

\subsection{Waterhouse's Theorem}
\begin{frame}[label=current]
\begin{theorem}[Waterhouse]\pause
\label{lem:Water}
 Let $q=p^n$ and let $N = q + 1-a$.
 $$\exists E/\F_q\text{ s.t.}\#E(\F_q) = N\Leftrightarrow|a|\le 2\sqrt q\text{ and}$$
 one of the following is satisfied:\pause
\begin{itemize}[<+-| alert@+>]
\item[(i)] $\gcd(a, p) = 1$;
\item[(ii)] $n$ even and one of the following is satisfied:
\begin{enumerate}
  \item $a=\pm 2\sqrt q$;
  \item $p\not\equiv 1 \pmod 3$, and $a = \pm\sqrt q$;
  \item $p\not\equiv 1 \pmod 4$, and $a = 0$;
\end{enumerate}
\item[(iii)] $n$ is odd, and one of the following is satisfied:
 \begin{enumerate}
   \item $p = 2$ or $3$, and $a = \pm p^{(n+1)/2}$;
   \item $a = 0$.
 \end{enumerate}
 \end{itemize}
\end{theorem}

%\setbeamercovered{transparent}
\begin{tiny}
\begin{example}[$q$ prime $\forall N\in I_q,\exists E/\F_q, \#E(\F_q)=N.$ $q$ not prime:]
\centerline{\begin{tabular}{|l|l|}
\hline
 $q$ & $a\in$\\
\hline\vspace*{-3.12pt}
\!\!$4=2^2$\!\! &\!\!\!\! $\{{\color<5->{green}-4},{\color<3->{green}-3},{\color<6->{green}-2},{\color<3->{green}-1},{\color<7->{green}0},{\color<3->{green}1},{\color<6->{green}2}, {\color<3->{green}3}, {\color<5->{green}4}\}$\\
\!\!$8=2^3$\!\! &\!\!\!\! $\{{\color<3->{green}-5},{\color<9->{green}-4},{\color<3->{green}-3},-2,{\color<3->{green}-1},{\color<10->{green}0},{\color<3->{green}1},2,{\color<3->{green}3}, {\color<9->{green}4},{\color<3->{green}5}\}$\\
\!\!$9=3^2$\!\! &\!\!\!\! $\{{\color<5->{green}-6},{\color<3->{green}-5},{\color<3->{green}-4},{\color<6->{green}-3},{\color<3->{green}-2},{\color<3->{green}-1},{\color<7->{green}0},{\color<3->{green}1},{\color<3->{green}2}, {\color<6->{green}3},{\color<3->{green}4},{\color<3->{green}5},{\color<5->{green}6}\}$\\
\!\!$16=2^4$\!\! &\!\!\!\! $\{{\color<5->{green}-8},{\color<3->{green}-7},-6,{\color<3->{green}-5},{\color<6->{green}-4},{\color<3->{green}-3},-2,{\color<3->{green}-1},{\color<7->{green}0},{\color<3->{green}1},2,{\color<3->{green}3}, {\color<6->{green}4},{\color<3->{green}5}, 6,{\color<3->{green}7},{\color<5->{green}8}\}$\\
\!\!$25=5^2$\!\! &\!\!\!\! $\{{\color<5->{green}-10},{\color<3->{green}-9},{\color<3->{green}-8},{\color<3->{green}-7},{\color<3->{green}-6},{\color<6->{green}-5},{\color<3->{green}-4},{\color<3->{green}-3},{\color<3->{green}-2},{\color<3->{green}-1},0,{\color<3->{green}1},{\color<3->{green}2}, {\color<3->{green}3}, {\color<3->{green}4},{\color<6->{green}5},{\color<3->{green}6},{\color<3->{green}7}, {\color<3->{green}8},{\color<3->{green}9}, {\color<3->{green}10}\}$\\
\!\!$27=3^3$\!\! &\!\!\!\! $\{{\color<3->{green}-10},{\color<9->{green}-9},{\color<3->{green}-8},{\color<3->{green}-7},-6,{\color<3->{green}-5},{\color<3->{green}-4},-3,{\color<3->{green}-2},{\color<3->{green}-1},{\color<10->{green}0},{\color<3->{green}1},{\color<3->{green}2}, 3, {\color<3->{green}4},{\color<3->{green}5},6,{\color<3->{green}7},{\color<3->{green}8},{\color<9->{green}9},{\color<3->{green} 10}\}$\!\!\!\!\\
\!\!$32=2^5$\!\!&\!\!\!\! $\{{\color<3->{green}-11},-10,{\color<3->{green}-9},{\color<9->{green}-8},{\color<3->{green}-7},-6,{\color<3->{green}-5},-4,{\color<3->{green}-3},-2,{\color<3->{green}-1},{\color<10->{green}0},{\color<3->{green}1},2, {\color<3->{green}3}, 4,{\color<3->{green}5}, 6, {\color<3->{green}7}, {\color<9->{green}8}, {\color<3->{green}9},10,{\color<3->{green}11}\}$\!\!\!\!\\  \hline
\end{tabular}}
\end{example}
\end{tiny}

\end{frame}

\subsection{R\"uck's Theorem}
\begin{frame}
\begin{theorem}[R\"uck]
Suppose $N$ is a possible order of an elliptic curve $/\F_q$,  $q=p^n$.  Write

\centerline{
$N = p^e n_1 n_2,\quad p\nmid n_1 n_2\quad\text{and}\quad n_1\mid n_2\ (\text{possibly }n_1 = 1).$}

There exists $E/\F_q$ s.t.
$$E(\F_q)\cong C_{n_1}\oplus C_{n_2p^e}$$
if and only if
\begin{enumerate}[<+-| alert@+>]
\item $n_1 = n_2$ in the case~(ii).1 of Waterhouse's Theorem;
\item $n_1 |q - 1$ in all other cases of  Waterhouse's Theorem.
\end{enumerate}
\end{theorem}\pause

\begin{example}
\begin{itemize}[<+->]
\item If $q=p^{2n}$ and $\#E(\F_q)=q+1\pm2\sqrt{q}=(p^n\pm1)^2$, then

\alert{\centerline{$E(\F_q)\cong C_{p^n\pm1}\oplus C_{p^n\pm1}.$}}
\item Let $N=100$ and $q=101\ \Rightarrow\ \exists E_1, E_2, E_3, E_4/\F_{101}$ s.t.

\alert{\centerline{$E_1(\F_{101})\cong C_{10}\oplus C_{10}\qquad E_2(\F_{101})\cong C_{2}\oplus C_{50}$}}

\alert{\centerline{$E_3(\F_{101})\cong C_{5}\oplus C_{20}\qquad E_4(\F_{101})\cong C_{100}$}}

\end{itemize}
\end{example}
\end{frame}


\subsection{Weil Pairing}
\begin{frame}\frametitle{Weil Pairing}
Let $E/K$ and $m\in\N$ s.t. $p\nmid m$. Then

\centerline{\begin{beamercolorbox}[rounded=true,shadow=true,wd=3cm,center]{formul}
$E[m]\cong C_m\oplus C_m$\end{beamercolorbox}}\pause

We set
\centerline{\begin{beamercolorbox}[rounded=true,shadow=true,wd=4cm,center]{postit}
$\mu_m:=\{x\in\bar{K}: x^m=1\}$\end{beamercolorbox}}\pause

$\mu_m$ is a cyclic group with $m$ elements(since $p\nmid m$)\pause
\begin{theorem}[Existence of Weil Pairing]
There exists a pairing \alert{$e_m:E[m]\times E[m]\rightarrow\mu_m$}
called \emph{Weil Pairing}, s.t. $\forall P, Q\in E[m]$\pause
\begin{enumerate}[<+-| alert@+>]
  \item $e_m(P+_EQ,R)=e_m(P,R)e_m(Q,R)$ (bilinearity)
  \item $e_m(P,R)=1\forall R\in E[m]\ \Rightarrow\ P=\infty$ (non degeneracy)
  \item $e_m(P,P)=1$
  \item $e_m(P,Q)=e_m(Q,P)^{-1}$
  \item $e_m(\sigma P,\sigma Q)=\sigma e_m(P,Q)\ \forall \sigma\in\operatorname{Gal}(\bar{K}/K)$ %   ($\sigma(A)=A$, $\sigma(B)=B$)
  \item $e_m(\alpha(P),\alpha(Q))=e_m(P,Q)^{\deg\alpha}\ \forall\alpha$ separable endomorphism
\end{enumerate}
\end{theorem}\vspace*{-4.2pt}\pause
\alert{{\scriptsize{The last one needs to be discussed further!!!}}}
\end{frame}

\section{Further reading}
\begin{frame}
\frametitle{Further Reading...}
\begin{scriptsize}
\begin{thebibliography}{99}
\bibitem{BSS} \textsc{Ian~F.~Blake,~Gadiel~Seroussi,~and~Nigel~P.~Smart},
Advances in elliptic curve cryptography, London Mathematical Society Lecture Note Series, vol. 317, Cambridge University Press, Cambridge, 2005.
 \bibitem{C} \textsc{J.~W.~S.~Cassels},
Lectures on elliptic curves, London Mathematical Society Student Texts, vol. 24, Cambridge University Press, Cambridge, 1991.
 \bibitem{CR} \textsc{John~E.~Cremona},
Algorithms for modular elliptic curves, 2nd ed., Cambridge University Press, Cambridge, 1997.
 \bibitem{Kn} \textsc{Anthony~W.~Knapp},
Elliptic curves, Mathematical Notes, vol. 40, Princeton University Press, Princeton, NJ, 1992.
 \bibitem{Ko} \textsc{Neal~Koblitz},
Introduction to elliptic curves and modular forms, Graduate Texts in Mathematics, vol. 97, Springer-Verlag, New York, 1984.
 %\bibitem{Po} \textsc{Poonen B} Elliptic curves (introduction)(19s) notes
 \bibitem{Sil} \textsc{Joseph~H.~Silverman},
The arithmetic of elliptic curves, Graduate Texts in Mathematics, vol. 106, Springer-Verlag, New York, 1986.
\bibitem{ST} \textsc{Joseph~H.~Silverman~and~John~Tate},
Rational points on elliptic curves, Undergraduate Texts in Mathematics, Springer-Verlag, New York, 1992.
\bibitem{washington} \textsc{Lawrence~C.~Washington},
Elliptic curves: Number theory and cryptography, 2nd ED. Discrete Mathematics and Its Applications, Chapman \& Hall/CRC, 2008.
\bibitem{Zimm} \textsc{Horst~G.~Zimmer},
Computational aspects of the theory of elliptic curves, Number theory and applications
(Banff, AB, 1988) NATO Adv. Sci. Inst. Ser. C Math. Phys. Sci., vol. 265, Kluwer Acad. Publ., Dordrecht, 1989, pp. 279--324.
\end{thebibliography}
\end{scriptsize}
\end{frame}




\end{document}


