\documentclass[10pt,handout]{beamer} %,hyperref={pdfpagelabels=false},draft,handout,handout
%\let\Tiny=\tiny
\hfuzz=3pt
\vfuzz=3pt
\usefonttheme{professionalfonts} % using non standard fonts for beamer
\usefonttheme{serif} % default family is serif

\usepackage[english]{babel}
\usepackage{lmodern}
\usepackage[latin1]{inputenc}
\usepackage{times}
\usepackage{amsthm}
\usepackage{amssymb}
\usepackage{hyperref}
%\usepackage[T1]{fontenc}
\usepackage{tikz}
\usepackage{colortbl}
\usepackage{yfonts}
\usepackage{pifont}
\usepackage{translator} % comment this, if not available
\mode<article>
{
  \usepackage{times}
  \usepackage{mathptmx}
  \usepackage[left=1.5cm,right=6cm,top=1.5cm,bottom=3cm]{geometry}
}

 \newcommand{\Q}{\mathbb Q}
 \newcommand{\Z}{\mathbb Z}
 \newcommand{\N}{\mathbb N}
 \newcommand{\F}{\mathbb F}
 \newcommand{\C}{\mathbb C}
 \newcommand{\R}{\mathbb R}
% Common theorem-like environments

\theoremstyle{definition}
\newtheorem{exercise}[theorem]{\translate{Exercise}}
\newtheorem{rem}[theorem]{\translate{Remark}}
\newtheorem{conj}[theorem]{\translate{Conjecture}}
\newtheorem{proposition}[theorem]{\translate{Proposition}}
\newtheorem{notation}[theorem]{\translate{Notation}}
\newtheorem{Note}[theorem]{\translate{Note}}
\newtheorem{Block}[theorem]{\translate{}}


% New useful definitions:

\lecture[1]{Introduction to Galois Representations\\
\small{Applications}}
{Galois Representations}
\date{September 2$^\textrm{nd}$, 2014}
\title[{Dipartim. Mat. \& Fis.}]{\insertlecture}
\subtitle{\ }
\author[\ \hspace{-2mm} Universit\`a Roma Tre]{Francesco Pappalardi}
\institute{Dipartimento di Matematica e Fisica\\
  Universit\`a Roma Tre}

% Beamer version theme settings

\useoutertheme[height=0pt,width=2cm,right]{sidebar}
\usecolortheme{rose,sidebartab}
\useinnertheme{circles}
\usefonttheme[only large]{structurebold}

\setbeamercolor{formul}{fg=black,bg=pink}
\setbeamercolor{sidebar right}{bg=black!15}
\setbeamercolor{structure}{fg=green!50!black}
\setbeamercolor{author}{parent=structure}
\setbeamercolor{postit}{fg=black,bg=yellow}
\setbeamercolor{greys}{fg=black,bg==black!25}
\setbeamerfont{title}{series=\normalfont,size=\LARGE}
\setbeamerfont{title in sidebar}{series=\bfseries}
\setbeamerfont{author in sidebar}{series=\bfseries}
\setbeamerfont*{item}{series=}
\setbeamerfont{frametitle}{size=}
\setbeamerfont{block title}{size=\small}
\setbeamerfont{subtitle}{size=\normalsize,series=\normalfont}
\setbeamertemplate{navigation symbols}{}
\setbeamertemplate{bibliography item}[book]
\setbeamertemplate{sidebar right}
{
  {\usebeamerfont{title in sidebar}%
    \vskip1.5em%
    \hskip3pt%
    \usebeamercolor[fg]{title in sidebar}%
    \insertshorttitle[width=2.1cm,respectlinebreaks]\par%   left,
    \vskip1.25em%
  }%
  {%
    \hskip3pt%
    \usebeamercolor[fg]{author in sidebar}%
    \usebeamerfont{author in sidebar}%
    \insertshortauthor[width=2cm,center,respectlinebreaks]\par%
    \vskip1em%
  }%
  \hbox to2cm{\hss\insertlogo\hss}
  \vskip1em%
  \insertverticalnavigation{2cm}%
  \vfill
  \hbox to 2cm{\hfill\usebeamerfont{subsection in
      sidebar}\strut\usebeamercolor[fg]{subsection in
      sidebar}\insertframenumber\hskip5pt}%
  \vskip3pt%
}%

\setbeamertemplate{title page}
{
  \vbox{}
  \vskip1em
  %{\huge Lecture \insertshortlecture\par}
  {\usebeamercolor[fg]{title}\usebeamerfont{title}\inserttitle\par}%
  \ifx\insertsubtitle\@empty%
  \else%
    \vskip0.25em%
    {\usebeamerfont{subtitle}\usebeamercolor[fg]{subtitle}\insertsubtitle\par}%
  \fi%
  \vskip1em\par
   \textbf{\Large{NATO ASI, Ohrid 2014}}\\
\textsl{Arithmetic of Hyperelliptic Curves}\\ August 25 - September 5, 2014\\ 
\emph{Ohrid, the former Yugoslav Republic of Macedonia},\par
  \vskip0pt plus1filll
  \leftskip=0pt plus1fill\insertauthor\par
  \insertinstitute\vskip1em
}

\logo{\includegraphics[width=1cm]{images/roma3.pdf}}

% Article version layout settings

\mode<article>

\makeatletter
\def\@listI{\leftmargin\leftmargini
  \parsep 0pt
  \topsep 5\p@   \@plus3\p@ \@minus5\p@
  \itemsep0pt}
\let\@listi=\@listI


\setbeamertemplate{frametitle}{\paragraph*{\insertframetitle\
    \ \small\insertframesubtitle}\ \par
}
\setbeamertemplate{frame end}{%
  \marginpar{\scriptsize\hbox to 1cm{\sffamily%
      \hfill\strut\insertframenumber}\hrule height .2pt}}
\setlength{\marginparwidth}{1cm}
\setlength{\marginparsep}{4.5cm}

\def\@maketitle{\makechapter}

\def\makechapter{
  \newpage
  \null
  \vskip 2em%
  {%
    \parindent=0pt
    \raggedright
    \sffamily
    \vskip8pt
    {\fontsize{36pt}{36pt}\selectfont Kapitel \insertshortlecture \par\vskip2pt}
    {\fontsize{24pt}{28pt}\selectfont \color{blue!50!black} \insertlecture\par\vskip4pt}
    {\Large\selectfont \color{blue!50!black} \insertsubtitle\par}
    \vskip10pt
  }
  \par
  \vskip 1.5em%
}

\let\origstartsection=\@startsection
\def\@startsection#1#2#3#4#5#6{%
  \origstartsection{#1}{#2}{#3}{#4}{#5}{#6\normalfont\sffamily\color{blue!50!black}\selectfont}}

\makeatother

\mode
<all>

% Typesetting Listings

\usepackage{listings}
\lstset{language=Java}

\alt<presentation>
{\lstset{%
  basicstyle=\footnotesize\ttfamily,
  commentstyle=\slshape\color{green!50!black},
  keywordstyle=\bfseries\color{blue!50!black},
  identifierstyle=\color{blue},
  stringstyle=\color{orange},
  escapechar=\#,
  emphstyle=\color{red}}
}
{
  \lstset{%
    basicstyle=\ttfamily,
    keywordstyle=\bfseries,
    commentstyle=\itshape,
    escapechar=\#,
    emphstyle=\bfseries\color{red}
  }
}

\begin{document}

\begin{frame}
\titlepage
\end{frame}

\section{Plan for today}
\begin{frame}
\frametitle{Plan for today}\pause

\begin{block}{Topics}\pause
\begin{itemize}[<+-| alert@+>]
 \item Short summery of Tuesday's Lecture
 \item Facts about Elliptic curves over finite fields
 \item Serre's Cyclicity Conjecture
 \item Lang--Trotter Conjecture for fixed traces
 \item Lang--Trotter Conjecture for primitive points
 \item Artin primitive roots Conjecture
\end{itemize}
\end{block}
\end{frame}

\begin{frame}

\frametitle{Elliptic curves}\pause

\textsc{Weierstra\ss\ Equation:} \hfill \textcolor{blue}{$E: Y^2=X^3+aX+b$}, \quad  $a, b\in\Z$;\bigskip\pause


\begin{block}{\textsc{Discriminant of $E$:}\hfill $\Delta_E=4a^3-27b^2$}\pause
\begin{itemize}[<+-| alert@+>]
\item $\Delta_E=(\alpha_1-\alpha_2)^2(\alpha_3-\alpha_2)^2(\alpha_3-\alpha_1)^2$ \\
\hspace*{2cm} ($\alpha_1$, $\alpha_2$, $\alpha_3$ {roots of $X^3+aX+b$});
\item $\Delta_E=0 \Longleftrightarrow X^3+aX+b$  has a double
root!
\end{itemize}
\end{block}\pause

\begin{definition}
  if $\Delta_E\neq0\
\Longrightarrow\ E$ is called \textcolor{red}{\textsc{Elliptic
Curve}}
\end{definition}\pause

\begin{block}{Group of Rational Points}
 If $K/\Q$ is an extension. Then
 $$E(K)=\{(x,y)\in K^2: y^2=x^3+ax+b\}\cup\{\infty\}$$\pause
\end{block}

\end{frame}


\begin{frame}
\frametitle{The $n$-torsion subgroups}\pause

\begin{block}
{If $n\in\N$\hspace{2cm}$E[n]:=\{P\in E(\overline{\Q})\ |\ nP=\infty\}$ }\pause
\begin{itemize}[<+-| alert@+>]
\item $E[n]\subset E(\overline{\Q})\cong \overline{\Q}/\Z\times \overline{\Q}/\Z$ is a subgroup
\item $E[n]\cong C_n\oplus C_n$
%\item $E[2]=\left\{P\in E(\overline{\Q})\ |\ r(P,P)\bigcap E(\overline{\Q})=\{P,\infty\} \right\}$
\item $E[2]=\{ (\alpha_1,0), (\alpha_2,0), (\alpha_3,0), \infty\}$\\
($\alpha_1, \alpha_2, \alpha_3$ roots of $x^3+ax+b$)
\item $E[3]$ is the set of inflection points
\item If $n$ is odd, $P=(\alpha,\beta)\in E[n]\ \ \Longrightarrow \psi_n(\alpha)=0$,\\
$\psi_n$ is $n$--division polynomials ($\partial \psi_n=(n^2-1)/2$ if $n$ odd)
\item
$E: y^3=x^3-2x\Longrightarrow E[2]=\{(0,0), (\sqrt{2},0), (-\sqrt{2},0), \infty\}$
\end{itemize}\pause
\end{block}
\end{frame}

\begin{frame}
\frametitle{Representation on $n$-torsion points}


\begin{block}{The $n$--\textcolor{blue}{torsion field}:
\hfill $\Q(E[n])=\displaystyle{\bigcap_{K^2 \supset E[n]\setminus\{\infty\}}} K$}\pause
\begin{itemize}[<+-| alert@+>]
%\item $\Q(E[n])$ is the splitting field of $\_n$ (division polynomials)
\item $\Q(E[n])$ is Galois over $\Q$
\item $\operatorname{Gal}(\Q(E[n])/\Q)\subseteq\operatorname{Aut}(E[n])\cong \operatorname{GL}_2(\Z/n\Z)$
$$\operatorname{Gal}(\Q(E[n])/\Q)\hookrightarrow \operatorname{GL}_2(\Z/n\Z)$$
$$ \sigma\mapsto \{(x,y)\mapsto (\sigma(x),\sigma(y))\}$$
\item[] Injective representation
\end{itemize}
\end{block}\pause

\begin{theorem}[Serre]\pause If $E/\Q$ is not CM. Then
$\operatorname{Gal}(\Q(E[\ell])/\Q)\neq \operatorname{GL}_2(\F_\ell)$
only for finitely many $\ell$. 
\end{theorem}\pause

\begin{conj}[$\ell\le 37$]
 \end{conj}

\end{frame}

\begin{frame}
\frametitle{Reducing modulo primes}

\begin{block}{Facts about elliptic curves over finite fields}\pause
\begin{itemize}[<+-| alert@+>]
\item $p$ prime, $p\nmid\Delta_E$

\item $E(\F_p)=\{\!(X,Y)\in\F_p^2\ |\ Y^2=X^3+aX+b\}\!\cup\{\infty\}$
 
%\item $E(\F_p)$ is a finite group (excellent for Cryptography)

\item $E(\F_p)\cong C_k\oplus C_{nk}$ for some $k\mid p-1$

\item $k=1$ above iff $E(\F_p)$ is cyclic

\item $\#E(\F_p)=p+1-a_p$ ($a_p$ is the \textcolor{blue}{\textsc{Trace of Frobenius)}}

\item \textcolor{blue}{\textsc{Hasse bound:}} \ \ \ \  $|a_p|\leq 2\sqrt{p}$;

\item $\Psi_p: E(\overline{\F_p})\rightarrow E(\overline{\F_p}), (x,y)\mapsto (x^p,y^p)$ 
\item[] it is an endomorphism of $E/\F_p$
 
\item $\Psi_p\in\operatorname{End}(E)$ satisfies $T^2-a_pT+p$
 
 \item $\Z[\Psi_p]\subset \operatorname{End}(E)$
 
 \item If the equality hold above, we say that $E$ is \textit{ordinary} at $p$. Otherwise we say
 that it is \textit{supersingular}

 \item $E/\F_p$ is supersingular \hfill $\Longleftrightarrow$ $E[p]=\{\infty\}$
 \item[] \hfill $\Longleftrightarrow$\qquad $a_p=0$
 \end{itemize}\pause
 \end{block}
%\centerline{\small Prob$(a_p(E)=r)\approx \frac 1{2\sqrt{p}}\ \ \
%=\!\!=\!\!=\!\!=\!\!\!> \  \ \ \pi_E^r(x)\approx \sum_{p\leq
%x}\frac1{2\sqrt{p}} \sim{\sqrt{x}\over \log x}. $ }
\end{frame}

\section{Serre's Cyclicity Conjecture}

\begin{frame}
\frametitle{Serre's Cyclicity Conjecture}
 Let $E/\Q$ and set\pause
$$\pi_E^{\text{cyclic}}(x)=\#\{p\le x: E(\F_p)\text{ is cyclic}\}.$$\pause

\begin{conj}[Serre]
The following asymptotic formula holds
$$\pi_E^{\text{cyclic}}(x)\sim\delta_E^{\text{cyclic}}\frac x{\log x}\qquad x\rightarrow\infty$$\pause
where \vspace*{-3pt}
$$\delta_E^{\text{cyclic}}=\sum_{n=1}^\infty\frac{\mu(n)}{\#\operatorname{Gal}(\Q(E[n])/\Q)}$$
 \end{conj}\pause

\begin{itemize}[<+-| alert@+>]
 \item Since $E(\F_p)\cong C_k\oplus C_{kn}$
 \item[] and $E[\ell]\cong C_\ell\oplus C_\ell$ for all $\ell\ne p$
 \item[] $E(\F_p)$ is cyclic iff $E[\ell]\nsubseteq E(\F_p) \forall \ell$ prime  $\ell\ne p$
 \item So we may rewrite \vspace*{-3pt}
$$\pi_E^{\text{cyclic}}(x)=\#\{p\le x: E[\ell]\nsubseteq E(\F_p) \forall \ell\text{ prime }, \ell\ne p\}.$$
 \end{itemize}\pause
\end{frame}


\begin{frame}
\frametitle{Serre's Cyclicity Conjecture}

We can apply inclusion exclusion principle:\pause

\begin{align*}
 \pi_E^{\text{cyclic}}(x)&=&\#\{p\le x: E[\ell]\nsubseteq E(\F_p) \forall \ell\text{ prime }, \ell\ne p\}\\
&=&\pi(x)-\sum_{\ell\text{ prime}}\pi_{E,\ell}(x)+\sum_{\ell_1,\ell_2\text{ primes}}\pi_{E,\ell_1\ell_2}(x)-\cdots
 \end{align*}\pause
 
where $\pi(x):=\#\{p\le x\}$ and if $k\in\N$,
$$\pi_{E,k}(x):=\#\{p\le x: E[k]\subseteq E(\F_p)\}$$\pause

Hence, if $\mu$ is the M\"obius function, then
$$\pi_E^{\text{cyclic}}(x)=\sum_{k\in\N}\mu(k)\pi_{E,k}(x)$$\pause

We will study $\pi_{E,k}(x)$ by mean of the Chebotarev density Theorem.
\end{frame}

\begin{frame}
\frametitle{Chebotarev Density Theorem  (from tuesday)} 


If $K/\Q$ be Galois and $p$ is prime unramified in $K$, the \emph{Artin Symbol}\pause
$$\left[\frac{K/\Q}p\right]:=\left\{\sigma\in\operatorname{Gal}(K/\Q): 
\begin{array}{l}
\exists\mathfrak p\text{ prime of $K$ above $p$ s.t. }\\
\sigma\alpha\equiv \alpha^{N\mathfrak p}\bmod \mathfrak p\ \forall \alpha\in\mathcal{O}
\end{array}\right\}$$\pause

Note that $\left[\frac{K/\Q}p\right]=\{id\}$ then $p$ splits completely in $K/\Q$ 
\\ \pause (i.e $p\mathcal O\subset\mathcal O$ is the product of $[K:\Q]$ prime ideals)\pause


\begin{theorem}[Chebotarev Density Theorem]
Let $K/\Q$ be finite and Galois, and let 
$\mathcal C\subset\operatorname{Gal}(K/\Q)$ be a union of conjugation classes. \pause
Then the density of the primes $p$ such that 
$\left[\frac{K/\Q}p\right]\subset \mathcal C$ equals $\frac{\#\mathcal C}{\#\operatorname{Gal}(K/\Q)}.$ \pause

In particular, if $\mathcal{C}=\{\operatorname{id}\}$, then the density of the primes $p$ such that 
$\left[\frac{K/\Q}p\right]=\{\operatorname{id}\}$ equals $\frac1{\#\operatorname{Gal}(K/\Q)}.$
\end{theorem}\pause

If $K=\Q(E[n])$, then
$$E[n]\subset {E}(\F_p)\quad\Longleftrightarrow \quad \left[\frac{\Q(E[n])/\Q}p\right]=\{\operatorname{id}\}$$

\end{frame}

\begin{frame}
\frametitle{Chebotarev Density Theorem and Serre's Cyclicity Conj.} 

If $K=\Q(E[n])$, then
$$E[n]\subset {E}(\F_p)\quad\Longleftrightarrow \quad \left[\frac{\Q(E[n])/\Q}p\right]=\{\operatorname{id}\}$$
\pause
Also recall that $\pi_{E,k}(x):=\#\{p\le x: E[k]\subseteq E(\F_p)\}$\pause
\begin{eqnarray*}
\pi_E^{\text{cyclic}}(x)&=&\sum_{k\in\N}\mu(k)\pi_{E,k}(x)\\
                        &=&\sum_{k\in\N}\mu(k)\#\left\{p\le x: \left[\frac{\Q(E[n])/\Q}p\right]=\{\operatorname{id}\}\right\}
\end{eqnarray*}
\pause

To proceed we need  a quantitative versions of the Chebotarev Density Theorem. Let
$$\pi_{\mathcal C/\mathcal G}(x):=\#\left\{p\le x: \left[\frac{K/\Q}p\right]\subset \mathcal C\right\}.$$
\end{frame}

\begin{frame}
\frametitle{The quantitative Chebotarev Density Theorem}

 Let
$$\pi_{\mathcal C/\mathcal G}(x):=\#\left\{p\le x: \left[\frac{K/\Q}p\right]\subset \mathcal C\right\}.$$
\pause

\begin{theorem}[Chebotarev, Lagarias, Odlyzko, Serre, Murty, Saradha]
The Generalized Riemann Hypothesis implies
$$\pi_{\mathcal C/\mathcal G}(x)=
\frac{\#\mathcal C}{\#\mathcal G}\int_2^x\frac{dt}{\log t}+O\left(\sqrt{\#\mathcal C}\sqrt{x}\log(xM\#\mathcal G)\right)$$
where $M$ is the product of primes numbers that ramify in $K/\Q$.
\end{theorem}\pause

In the case of $K=\Q(E[k])$ and $k$ is square free, the above specializes to 
$$\pi_{E,k}(x)=\frac1{\#\operatorname{Gal}(\Q(E[k])/\Q)}\int_2^x\frac{dt}{\log t}+
O\left(\sqrt{x}\log(xk)\right)$$
\end{frame}

\begin{frame}
\frametitle{The quantitative Chebotarev Density Theorem and Serre's Conj}
In the case of $K=\Q(E[k])$ and $k$ is square free, the above specializes to 
$$\pi_{E,k}(x)=\frac1{\#\operatorname{Gal}(\Q(E[k])/\Q)}\int_2^x\frac{dt}{\log t}+
O\left(\sqrt{x}\log(xk)\right)$$\pause
Hence
$$\pi_E^{\text{cyclic}}(x)=\sum_{k\in\N}\frac{\mu(k)}{\#\operatorname{Gal}(\Q(E[k])/\Q)}\int_2^x\frac{dt}{\log t}
+\operatorname{ERROR}$$\pause

The error can be estimated by standard analytic number theory\pause

Finally
$$\delta_E^{\text{cyclic}}=\sum_{k=1}^\infty\frac{\mu(k)}{\#\operatorname{Gal}(\Q(E[k])/\Q)}.$$
\end{frame}

\begin{frame}
\frametitle{The state of the Art on Serre's Cyclicity Conjecture}

\begin{itemize}[<+-| alert@+>]
\item Serre (1976): \emph{GRH }$\Rightarrow \pi_E^{\text{cyclic}}(x)\sim \delta_E^{\text{cyclic}}\frac{x}{\log x}$
\item Murty (1979): \emph{$E/\Q$ CM} $\Rightarrow \pi_E^{\text{cyclic}}(x)\sim \delta_E^{\text{cyclic}}\frac{x}{\log x}$
\item Gupta \& Murty (1990): $\pi_E^{\text{cyclic}}(x)\gg\frac{x}{(\log x)^2}$ iff $E[2]\nsubseteq E[\Q]$
 \item Cojocaru (2003): \emph{Simple proof and explicit error term for CM curves}
 \item Cojocaru \& Murty (2004): \emph{improved error terms depending on GRH}
 \item Serre: $\delta_E^{\text{cyclic}}$ is a rational multiple of 
 $$C=\prod_\ell\left(
 1-\frac1{\ell(\ell-1)^2(\ell+1)}\right)= 0.81375190610681571\cdots$$
 \item Lenstra, Moree \& Stevenhagen (2013): \emph{If $E/\Q$ is a Serre curve then:}
$$\delta_E^{\text{cyclic}}=
C\times\left(1+\prod_{\ell\mid 2\operatorname{disc}(\Q(\sqrt{\Delta_E}))}\frac{-1}{(\ell^2-1)(\ell^2-\ell)-1}\right)$$
 \end{itemize}\pause
\end{frame}


\section{Lang Trotter Conjecture for trace of Frobenius}

\begin{frame}
\frametitle{Lang Trotter Conjecture for trace of Frobenius} 
 Let $E/\Q$, $r\in\Z$  and set\pause
$$\pi_E^r(x)=\#\{p\le x: p\nmid\Delta_E\text{ and } \#\overline E(\F_p)=p+1-r\}\vspace*{-7pt}$$
\pause
\begin{conj}[Lang -- Trotter (1970)]
If either $r\ne0$ or if $E$ has no CM, \pause then
the following asymptotic formula holds
$$\pi_E^r(x)\sim C_{E,r}\frac{\sqrt{x}}{\log x}\qquad x\rightarrow\infty$$\pause
where $C_{E,r}$ is the  \emph{Lang--Trotter constant}
$$C_{E,r}=\frac2\pi\frac{m_E\#\operatorname{Gal}(\Q(E[m_E])/\Q)_{\text{tr}=r}}{\#\operatorname{Gal}(\Q(E[m_E])/\Q)}
\times\prod_{\ell\nmid m_E}\frac{\ell\#\operatorname{GL}_2(\F_\ell)_{\text{tr}=r}}{\#\operatorname{GL}_2(\F_\ell)}$$\pause
and $m_E$ is the \emph{Serre's conductor} of $E$\pause
 \end{conj}\pause

\begin{itemize}[<+-| alert@+>]
 \item If $E$ is a Serre's curve, then $m_E=[2,\operatorname{disc}(\Q(\sqrt{\Delta_E}))]$
 \item $\#\operatorname{GL}_2(\F_\ell)_{\text{tr}=r}=\begin{cases}
                   \ell^2(\ell-1) & \text{if } r=0\\
                   \ell(\ell^2-\ell-1) & \text{otherwise.}
                  \end{cases}
$
\end{itemize}\pause
\end{frame}

\subsection{state of the Art}

\begin{frame}
\frametitle{Lang Trotter Conjecture for trace of Frobenius} 
\framesubtitle{An application of $\ell$--adic representations and of the Chebotarev density Theorem}\pause

\begin{theorem}[Serre]
Assume that $E/\Q$ is not CM or that $r\neq0$ and that the Generalized Riemann 
Hypothesis holds. Then\pause
$$\pi_E^r(x)\ll\begin{cases} x^{7/8}(\log x)^{-1/2}&\text{if}\ r\ne0\\ x^{3/4}&\text{if}\ r=0.\end{cases}$$  
\end{theorem}\pause

\begin{itemize}[<+-| alert@+>]
 \item If $E/\Q$ is CM and $r=0$. It is classical
 $$\pi_E^0(x)\sim\frac12\frac{x}{\log x}\qquad x\rightarrow\infty$$
 \item Murty, Murty and Sharadha: If $r\ne0$, on GRH, $\pi_E^r(x)\ll x^{4/5}/(\log x)^{-1/5}$
 \item Elkies $\pi_E^0(x)\rightarrow\infty\quad x\rightarrow\infty$
 \item Elkies \& Murty: GRH $\Longrightarrow\pi_E^0(x)\gg\log\log x$
 \item Average Versions later
 \end{itemize}\pause

\end{frame}

\begin{frame}
\frametitle{Lang Trotter Conjecture for trace of Frobenius}

\begin{block}{Unvonditional Stetements}\pause
\begin{itemize}[<+-| alert@+>]
\item \textcolor{blue}{\textit{J. P. Serre (1981)}},\pause  
%\textcolor{blue}{\textit{Elkies, Kaneko, K. Murty, R. Murty, N. Saradha, Wan (1988):}}
$$\pi_{E,r}(x) \ll \left\{ \begin{array}{ll}
\frac{x(\log\log x)^2}{\log^2 x}& \mbox{if $r \neq 0$} \\  \\%x^{4/5}\log^{-1/5}{x}
{x^{3/4}} & \text{if }r = 0\text{ and}\\ & \text{ $E$ not CM} \end{array} \right.$$\pause
\item \textcolor{blue}{\textit{N. Elkies, E. Fouvry, R. Murty (1996)}}
$$\pi_{E,0}(x)\gg \log\log\log x/(\log\log\log\log
x)^{1+\epsilon}$$
\end{itemize}\pause
\end{block}
\end{frame}


\subsection{Serre's upperbound}

\begin{frame}
\frametitle{Chebotarev Density Theorem and Serre's Theorem on fixed traces} 

Let $\ell$ be sufficiently large such that 
$$\mathcal G=\operatorname{Gal}(\Q(E[\ell])/\Q)\cong\operatorname{GL}_2(\F_\ell)$$\pause

Set
$\mathcal C=\operatorname{GL}_2(\F_\ell)_{\text{tr}=r}=\{\sigma\in\operatorname{GL}_2(\F_\ell):\operatorname{tr}\sigma=t\}$
\pause

So that
$$\#\operatorname{GL}_2(\F_\ell)=(\ell^2-1)(\ell^2-\ell)$$
\pause
and\pause
$$\#\operatorname{GL}_2(\F_\ell)_{\text{tr}=r}=\begin{cases}
                   \ell^2(\ell-1) & \text{if } r=0\\
                   \ell(\ell^2-\ell-1) & \text{otherwise.}
                  \end{cases}$$
\pause

Then by Chebotarev Density Theorem on GRH, 
\begin{align*}
 \pi_{\mathcal C/\mathcal G}(x)&=
\frac{\#\mathcal C}{\#\mathcal G}\int_2^x\frac{dt}{\log t}+O\left(\sqrt{\#\mathcal C}\sqrt{x}\log(xM\#\mathcal G)\right)\\\pause
&\ll \frac{1}{\ell}\frac{x}{\log x}+\ell^{3/2}\sqrt{x}\log(x\ell)   
\end{align*}
\end{frame}
\begin{frame}
\frametitle{Chebotarev Density Theorem and Serre's Theorem on fixed traces} 

Finally recall (from tuesday) that if $\Phi_p$ is the Frobenius endomorphism,\pause
$$\# E(\F_p)=p+1-r\quad\Longleftrightarrow\quad\operatorname{Tr}(\Phi_p)\equiv r$$\pause
Hence for all $\ell$ sufficiently large,\pause
\begin{align*}\pi_E^r(x)&=\#\{p\le x: p\nmid\Delta_E\text{ and } \# E(\F_p)=p+1-r\}\\
 &\le \#\{p\le x: p\nmid\Delta_E\text{ and } \operatorname{Tr}(\Phi_p)\equiv r\bmod p\}\\
 &=\pi_{\mathcal C/\mathcal G}(x)\\
 &\ll  \frac{1}{\ell}\frac{x}{\log x}+\ell^{3/2}\sqrt{x}\log(x\ell)
\end{align*}\pause

It is enough to choose $\ell=x^{1/5}(\log x)^{-4/5}$\pause

To conclude that
$$\pi_E^r(x)\ll x^{4/5}(\log x)^{-1/5}$$
\end{frame}

\subsection{Average Lang Trotter Conjecture}

\begin{frame}
\frametitle{Average Lang Trotter Conjecture}\pause

\begin{theorem}[David, F. P. (1997)]\pause
Let
$${\mathcal C}_{x}=\{E: Y^2=X^3+aX+b\ :4a^3+27b^2\ne0\text{ and }\ |a|,|b|\leq x\log x\}$$\pause
Then\vspace*{-5pt}
$$\frac{1}{|\mathcal C_x|}\sum_{E\in\mathcal C_x}\pi_{E,r}(x)\sim
c_r\frac{\sqrt{x}}{\log x}\ \ \textrm{as}\ x\rightarrow\infty
$$\pause
where
$$
c_r 
%\frac{2}{\pi}
%\prod_{l|r}\left(1-\frac{1}{l^2}\right)^{-1} \prod_{l \nmid
%r}\frac{l(l^2-l-1)}{(l-1)(l^2-1)}
=\frac{2}{\pi}\prod_{l}
\frac{\ell|\operatorname{GL}_2(\F_\ell)^{\operatorname{tr}=r}|}{|\operatorname{GL}_2(\F_\ell)|}.
$$
\end{theorem}\pause

\begin{theorem}[N. Jones (2004)]
Let \vspace*{-5pt}$$\mathcal C_x^{\text{Serre}}:=\{E\in \mathcal{C}_x: E\text{ is a Serre curve}\}$$\pause
Then\vspace*{-5pt}
$$\lim_{x\rightarrow\infty}\frac{|\mathcal C_x^{\text{Serre}}|}{|\mathcal C_x|}=1$$\pause
In this sense \emph{almost all elliptic curves are Serre's curves}
\end{theorem}
\end{frame}

\begin{frame}
\frametitle{The General Lang--Trotter Conjecture}\pause

\begin{definition}[\textit{General Lang--Trotter function}]\pause
Let $K/\Q$ be a number field, Let $E/K$ be an elliptic curve and set $f\mid [K:\Q]$. Define
$$\pi_E^{r,f}(x)=\#\left\{p\leq x\ |\ \deg_K(p)=f,\ \exists \mathfrak p|p, a_E(\mathfrak p)=r\right\}\vspace*{-13pt}$$
 \end{definition}\pause


\begin{conj}[The General Lang-Trotter Conjecture for Fixed Trace]\pause
$\exists c_{E,r,f}\in\R^{\geq0}$ such that\vspace*{-5pt}
$$\displaystyle{\pi_E^{r,f}(x)\sim c_{E,r,f} \begin{cases} \frac{x}{\log x}& \textrm{if}\ E\
\textrm{has CM and } r=0\\ \\
\frac{\sqrt{x}}{\log x}  & \textrm{if}\ f=1 \\ \\
\log\log x & \textrm{if}\ f=2\\ \\
1 & \textrm{otherwise.}
\end{cases}}$$\vspace*{-5pt}
 \end{conj}\pause
 

{\bf Example.} $K=\Q(i)$: $\pi^{r,1}$ counts split primes $\equiv1\bmod4$;\\
\hspace*{3.2cm}$\pi^{r,2}$ counts inert primes $\equiv3\bmod4$
\end{frame}

\begin{frame}
\frametitle{Another Average result}

\begin{theorem}[C. David \& F.P. (2004)]\pause Let
$K=\Q(i)$, $r\in\mathbb Z,$ $r\neq 0$ and for $\alpha,\beta\in\Z[i]$, set
$E_{\alpha,\beta}: Y^2=X^3+\alpha X+\beta$. \pause Further let
$${\cal C}_x= \left\{ E_{\alpha,\beta}:  \ \left|\
\begin{array}{l}\alpha=a_1+a_2i,\beta=b_1+b_2i\in{\bf Z}[i],\\
 4\alpha^3-27\beta^2\neq0 \\
\max\{|a_1|,|a_2|,|b_1|,|b_2|\}<x\log x
\end{array}
    \right.\right\}$$ \pause    
Then
$$\frac1{|\textcolor{black}{{\cal
C}_x}|}\sum_{{E\in{\cal C}_x}}
\textcolor{black}{\pi_E^{r,2}(x)}{\sim}\textcolor{black}{
c_{r}\log\log x}.$$\pause
where
$$c_{r}=\frac1{3\pi}
\prod_{\ell>2}\frac{\ell(\ell-1-\left(\frac{-r^2}{\ell}\right))}{
(\ell-1)(\ell-\left({-1}{\ell}\right))}$$
\end{theorem}

Extended to the Average of the General Lang-Trotter function by Kevin James and Ethan Smith in 2011


\end{frame}

\subsection{Some ideas on Average results proofs}

\begin{frame}
\frametitle{Sketch of proof. 1/4}

\begin{definition}[Kronecker--Hurwitz class numbers] \pause
Let $d\in\Z$, $d\equiv0,1\bmod 4$. Then 
$$H(d)=2\sum_{f^2\mid d}\frac{h\left(\frac d{f^2}\right)}{w\left(\frac d{f^2}\right)}$$\pause
where \begin{itemize}
       \item $h(D)=$ class number
       \item $w(D)$ is number of units in $\Z[D+\sqrt{D}]\subset{\Q}(\sqrt{d})$
      \end{itemize}
\end{definition}\pause

\begin{theorem}[Deuring's Theorem]\pause
Let $q=p^n$, $r$ odd (simplicity) with $r^2-4q<0$.\pause

$$\#\left\{\begin{array}{l}
            \F_q-\text{isomorphism classes of } E/\F_q \\
            \text{ with }a_q(E)=r
           \end{array}\right\}= H(r^2-4q).$$
\end{theorem}
 
\end{frame}

\begin{frame}
\frametitle{Sketch of proof.  2/4}

\begin{block}{\textcolor{red}{\textit{Step 1:}} switch the order of summation}\pause
\begin{eqnarray*}
 \frac{1}{|\mathcal C_x|}\sum_{E\in\mathcal C_x}\pi_{E,r}(x) &=&\frac{1}{|\mathcal C_x|}\sum_{E\in\mathcal C_x}\sum_{\substack{ p\le x\\ a_p(E)=r}}1 \\
 &=&\sum_{p\le x}\frac{|\{E\in\mathcal C_x: a_p(E)=r\}}{|\mathcal C_x|}\\
 &=& \frac12 \sum_{\substack{p\leq x}}
 \frac{H(r^2-4p)}{p}+O(1)
\end{eqnarray*} 
\end{block}\pause

\begin{theorem}[Dirichlet Class Number Formula]\pause
 Let  $\chi_d(n)=\left(\frac dn\right)$ and let $L(s,\chi_{d})$ be the \textcolor{blue}{Dirichlet $L$--function}. \pause Then the class
 number
$$h(d)=\frac{\omega(d)|d|^{1/2}}{2\pi}L(1,\chi_{d})\vspace*{-12pt}$$
 \end{theorem}\pause

Next we use the definition of the \alert{Kronecker--Hurwitz class number}
 \end{frame}

\begin{frame}
\frametitle{Sketch of proof. 3/4}

 \begin{block}
 {\textcolor{red}{\textit{Step 2.}} applying the class number formula}\pause Let $d=(r^2-4p)/f^2$. Then \pause
$$\frac12\!\!\sum_{p\leq x}
\!\frac{H(r^2-4p)}{p}=\frac2\pi\sum_{\begin{subarray}{l} f\leq 2x\\ (f,2r)=1\end{subarray}}\!\frac 1f\!\!\!
\sum_{\begin{subarray}{c}p\leq x\\
4p\equiv r^2\bmod f^2
\end{subarray}}\!\!\!\!\frac{L(1,\chi_{d})}{p}+O(1)$$
 \end{block}\pause

 So the problem is reduced to a special $L$--function value average. \pause Analytic tools become relevant!!\pause

\begin{theorem}[Barban--Davenport--Harberstam Theorem]\pause
Let $\varphi$ be the Euler function. Then for $1 \leq Q \leq x$ and $\forall c>0$,\pause
$$\sum_{q \leq Q} \sum_{a \bmod q } \left|\sum_{\substack{p\leq x \\ p \equiv a \bmod q}} \log p - \frac{x}{\varphi(q)}\right|^2
\ll Q x \log x + \frac{x^2}{ \log^c x}$$
\end{theorem}\pause
\end{frame}

\begin{frame}
\frametitle{Sketch of proof. 4/4}

\begin{lemma}[Crucial analytic Lemma]\pause
$\forall c>0$,\pause
$$
\sum_{\begin{subarray}{l} f\leq 2x\\ (f,2r)=1\end{subarray}}\!\frac 1f\!\!\!\sum_{\begin{subarray}{c}p\leq x\\
4p\equiv r^2\bmod f^2
\end{subarray}}\!\!\!\!L(1,\chi_{d})\log p=k_rx+O\left(\frac{x}{\log^cx}\right)$$
where 
$$k_{r}=\frac23
\prod_{\ell>2}\frac{\ell-1-\left(\frac{-r^2}{\ell}\right)}{(\ell-1)(\ell-\left(\frac{-1}{\ell}\right))}$$
 \end{lemma}\pause
 
 The rest is classical analytic number theory...

 \end{frame}


\section{Lang Trotter Conjecture for Primitive points}

\begin{frame}
\frametitle{Lang Trotter Conjecture for Primitive points} \pause

\begin{definition}
   Let $E/\Q$ and let $P\in E(\Q)$ be of infinite order. 
   $P$ is called \emph{primitive} for a prime $p$ if the reduction $P\bmod p$ is a generator for $E(\F_p)$.\\ \pause
\centerline{\alert{$\langle P\bmod p\rangle =E(\F_p)$}}
\end{definition}\pause
Set \\
\centerline{$\pi_{E,P}(x)=\#\{p\le x: p\nmid\Delta_E\text{ and } P\text{ is primitive for } p\}$}\pause

\begin{conj}[Lang--Trotter for primitive points (1976)] The following asymptotic formula holds
$$\pi_{E,P}(x)\sim \delta_{E,P}\frac x{\log x}\qquad x\rightarrow\infty.$$
with\vspace*{-4pt}\pause
$$\delta_{E,P}=\sum_{n=1}^\infty\mu(n)\frac{\#\mathcal C_{P,n}}{\#\operatorname{Gal}(\Q(E[n],n^{-1}P)/\Q)}$$\pause
where $\Q(E[n],n^{-1}P)$ is the extension of $\Q(E[n])$ of the coordinates of the points
$Q\in E(\bar{\Q})$ such that $nQ=P$ and $\mathcal C_{P,n}$ is a union of conjugacy classes in 
$\operatorname{Gal}(\Q(E[n],n^{-1}P)/\Q)$.
\end{conj}
\end{frame}

\section{{Artin Conjecture for primitive roots}}
\begin{frame}
\frametitle{Statement of the Artin Conjecture}


\begin{conj}[\textcolor{red!70!black}{Artin Conjecture} (1927)]\pause Let $a\in\Q\setminus\{0,1,-1\}$ and set
$$P_a(x):=\{p\le x:\ a\text{ is a primitive root }\bmod p\}.$$
Then there exists $\delta_a\in\Q^{\ge0}$ such that
$$P_a(x)\sim\delta_a\prod_\ell\left(
1-\frac{1}{\ell(\ell-1)}\right)\times \pi(x)$$
\end{conj}\pause

\begin{theorem}[Hooley 1965] Let $a\in\Q\setminus\{-1,0,1\}$ and assume \emph{GRH} for all the Dedekind $\zeta$--functions
$\Q[e^{2\pi i/m},a^{1/m}], m\in\N$. Then the Artin Conjecture holds:\pause
$$P_a(x)=\delta_a\frac x{\log x}+O\left(\frac {x\log\log x}{\log^2 x}\right).$$
\end{theorem}

% \begin{Note}\pause
% \begin{itemize}[<+-|alert@+>]
%       \item the upper bound holds unconditionally
%       \item Vinogradov: GRH the be relaxed to a \emph{zero density assumption}
%       for the Dedekind $\zeta$--function of $\Q[a^{1/\ell}]$, $\ell$ prime
% \end{itemize}
% \end{Note}

\end{frame}

\section{Artin vs Lang Trotter}

\begin{frame}
\frametitle{Lang--Trotter Conjecture, Serre's Cyclicity \& Artin}
\framesubtitle{three ``sister'' conjectures}\pause

%Let $E/\Q$ be an elliptic curve without CM\pause

\begin{conj}[\textcolor{red!70!black}{Lang Trotter primitive points Conjecture}(1977)]\pause Let $P\in E(\Q)\setminus \operatorname{Tors}(E(\Q))$.
$\exists \alpha_{E,P}\in\Q^{\ge0}$ s.t.
$$\!\!\!\frac{\#\{p\le x: p\nmid\Delta_E, E(\F_p^*)=\langle P\bmod p\rangle\}}{\pi(x)}\sim
\alpha_{{E,P}}
\scriptstyle{\prod_\ell\left(
1-\frac{\ell^3-\ell-1}{\ell^2(\ell-1)^2(\ell+1)}\right)}\vspace*{-11pt}$$
\end{conj}\pause

% \begin{itemize}[<+-|alert@+>]
%  \item $B=\prod_\ell\left(
% 1-\frac{\ell^3-\ell-1}{\ell^2(\ell-1)^2(\ell+1)}\right)=0.440147366792057866\cdots$
% \item If $E[2]\subset E(\Q)$ then $\alpha_{E,P}=0$
% \item Similar conjecture also given for CM curves (more later)
% \item if $P=kQ$, $Q\in E(\Q)$ and $d=\gcd(k,\#\operatorname{Tor}(E(\Q)))>1$, then $\alpha_{E,P}=0$
% \ \hfill (since $\operatorname{ord}P\mid \frac{\#E(\F_p)}{d}$)
% \end{itemize}
% \end{frame}
% 
% 
% \begin{frame} 

\begin{conj}[\textcolor{red!70!black}{Serre's Cyclicity Conjecture} (1976)]\pause
$\exists \gamma_{E,P}\in\Q^{\ge0}$ s.t.%, as $x\rightarrow\infty$,
$$\frac{\#\{p\le x: p\nmid\Delta_E, E(\F_p^*)\text{ is cyclic}\}}{\pi(x)}\sim\gamma_{E,P}
\prod_\ell{\scriptstyle{\left(
1-\frac1{(\ell^2-1)(\ell^2-\ell)}\right)}}\vspace*{-11pt}$$
\end{conj}\pause

\begin{conj}[\textcolor{red!70!black}{Artin Conjecture} (1927)]\pause Let $a\in\Q\setminus\{0,1,-1\},
\exists \delta_a\in\Q^{\ge0}$ s. t.
$$\frac{\#\{p\le x:\ a\text{ primitive root }\bmod p\}}{\pi(x)}\sim\delta_a\prod_\ell\left(
1-\frac{1}{\ell(\ell-1)}\right)$$
\end{conj}

\end{frame}

\begin{frame}
\frametitle{Naive Densities}\pause

\begin{itemize}[<+-|alert@+>]
\item \textcolor{red!70!black}{The Artin Constant} (primitive roots naive density)
$$A=\prod_\ell\left(
1-\frac1{\ell(\ell-1)}\right)=0.37395581361920228\cdots$$
\item\textcolor{red!70!black}{The Lang Trotter first Constant} (LTC naive density)
$$\hspace*{-14pt} B=\prod_\ell\left(
1-\frac{\ell^3-\ell-1}{\ell^2(\ell-1)^2(\ell+1)}\right)=0.44014736679205786\cdots$$
\item \textcolor{red!70!black}{The Serre's Constant} (EC cyclicity naive density)
$C=\prod_\ell\left(
 1-\frac1{\ell(\ell-1)^2(\ell+1)}\right)= 0.81375190610681571\cdots$
\end{itemize}
% \pause
% \begin{itemize}[<+-|alert@+>]
%  \item $C=\prod_\ell\left(
% 1-\frac1{\ell(\ell-1)^2(\ell+1)}\right)= 0.81375190610681571\cdots$
% \item If $E[2]\subset E(\Q)$ then $\gamma_{E,P}=0$
% \item 1983: Ram eliminated GRH for the analogue on CM curves
% \item 1990: Rajiv \& Ram: if $E[2] \nsubseteq E(\Q)$, $\#\{p\le x: p\nmid\Delta_E, E(\F_p^*)\text{ is cyclic}\}\gg_E\frac{\pi(x)}{\log x}$ 
% \item 2004: Alina \& Ram: significant improvements on the error terms both unconditional and on GRH
% \end{itemize}
\end{frame}

\begin{frame}
\frametitle{Comparison between empirical data: \textsc{\textcolor{red!70!black}{AC}} vs \textsc{LTC} vs \textsc{SCC}}

\centerline{\begin{beamercolorbox}[shadow=true,center,rounded=true,wd=9cm]{postit}
Artin Conjecture\end{beamercolorbox}}\pause\smallskip

\begin{tiny}
\begin{center}
\begin{tabular}{|l|l|r|}
\hline
$q$& $P_q(2^{25})/\pi(2^{25})$ & $A-P_q(2^{25})/\pi(2^{25})$\\
\hline         
  2&$0.37395508\cdots$&   $0.0000007\cdots$\\
  3&$0.37388094\cdots$&   $0.0000748\cdots$\\
  7&$0.37409997\cdots$&  $-0.0001441\cdots$\\
 11&$0.37422450\cdots$&  $-0.0002686\cdots$\\
 19&$0.37400887\cdots$&  $-0.0000530\cdots$\\
 23&$0.37402147\cdots$&  $-0.0000656\cdots$\\
 31&$0.37422208\cdots$&  $-0.0002662\cdots$\\
\hline
\end{tabular}\end{center}
\end{tiny}\pause

\centerline{\begin{beamercolorbox}[shadow=true,center,rounded=true,wd=9.5cm]{postit}
Lang--Trotter Conjecture\qquad Serre Cyclicity Conjecture\end{beamercolorbox}}\pause\smallskip

\centerline{\begin{beamercolorbox}[shadow=true,center,rounded=true,wd=6.5cm]{formul}
$\displaystyle{\pi_{E,P}(x)=\#\{p\le x: \langle P\bmod p\rangle=E(\mathbb F_p^*)\}}$
\end{beamercolorbox}}\pause\smallskip

\centerline{\begin{beamercolorbox}[shadow=true,center,rounded=true,wd=6cm]{formul}
$\displaystyle{\pi_E^{\text{cycl}}(x)=\#\{p\le x: E(\mathbb F_p^*)\text{ is cyclic}\}}$
\end{beamercolorbox}}\pause\smallskip

\textsc{Serre's Curves of rank 1} (no torsion, Galois surjective $\forall\ell$)\pause

\begin{tiny}\begin{columns}[c]
\begin{column} {5cm}
\begin{tabular}{|l|l|r|}
\hline
 $E$ &\!\! $\frac{\pi_{E,P}(2^{25})}{\pi(2^{25})}$\hspace*{-2mm}\!\!&\hspace*{-1.8mm}\!\! $\alpha_{E,P}B-\frac{\pi_{E,P}(2^{25})}{\pi(2^{25})}$\hspace*{-2mm}\!\!\\
\hline
37.a1 &$0.44017485\cdots$\hspace*{-2mm}&   $-0.000027\cdots$\\
43.a1 &$0.44034784\cdots$\hspace*{-2mm}&   $-0.000200\cdots$\\
53.a1 &$0.44020198\cdots$\hspace*{-2mm}&   $-0.000054\cdots$\\
57.a1 &$0.44016176\cdots$\hspace*{-2mm}&   $-0.000014\cdots$\\
58.a1 &$0.44012203\cdots$\hspace*{-2mm}&    $0.000025\cdots$\\
61.a1 &$0.44034299\cdots$\hspace*{-2mm}&   $-0.000195\cdots$\\                       
77.a1 &$0.43964812\cdots$\hspace*{-2mm}&    $0.000499\cdots$\\
79.a1 &$0.44043021\cdots$\hspace*{-2mm}&  $ -0.000282\cdots$\\
\hline
\end{tabular}
\end{column}\pause
\begin{column}{5cm}
\begin{tabular}{|l|l|r|}
\hline
 $E$ &\!\! $\frac{\pi_E^{\text{cycl}}(2^{25})}{\pi(2^{25})}$\hspace*{-2mm}\!\!&\hspace*{-1.5mm}\!\! $\gamma_EC-\frac{\pi_E^{\text{cycl}}(2^{25})}{\pi(2^{25})}$\hspace*{-2mm}\!\!\\
\hline
37.a1 &$0.81383047\cdots$&   $-0.000078\cdots$\\
43.a1 &$0.81363907\cdots$&   $ 0.000112\cdots$\\
53.a1 &$0.81389250\cdots$&   $-0.000140\cdots$\\
57.a1 &$0.81387263\cdots$&   $-0.000120\cdots$\\
58.a1 &$0.81374131\cdots$&   $ 0.000010\cdots$\\
61.a1 &$0.81397584\cdots$&   $-0.000223\cdots$\\                       
77.a1 &$0.81380285\cdots$&   $-0.000050\cdots$\\
79.a1 &$0.81392157\cdots$&   $-0.000169\cdots$\\\hline
\end{tabular}\end{column}\end{columns}
\end{tiny}
\end{frame}


\section{Further reading}
\begin{frame}
\frametitle{Further Reading...}
\begin{scriptsize}
\begin{thebibliography}{99}
\bibitem{C}
\textsc{Cojocaru, Alina Carmen}, 
\textit{Cyclicity pf CM Elliptic Curves modulo $p$}
Trans. of the AMS \textbf{355}, 7, (2003) 2651--2662.

\bibitem{DP}  
\textsc{David, Chantal; Pappalardi, Francesco},
\textit{Average Frobenius Distribution of Elliptic Curves}, 
Internat. Math. Res. Notices \textbf{4} (1999) 165--183.

\bibitem{GM}
\textsc{Gupta, Rajiv; Murty M. Ram},
\textit{Primitive points on elliptic curves},
Compositio Mathematica \textbf{58}, n. 1 (1986), 13--44.

\bibitem{GM2}
\textsc{Gupta, Rajiv; Murty M. Ram},
\textit{Cyclicity and generation of points mod $p$ on elliptic curves},
Inventiones mathematicae \textbf{101} 1 (1990)  225--235

\bibitem{LT1}
\textsc{Lang, Serge; Trotter, Hale},
Frobenius distributions in $\operatorname{GL}_{2}$-extensions.
Lecture Notes in Mathematics, Vol. 504. \textit{Springer-Verlag}, Berlin--New York, 1976

\bibitem{LT2}
\textsc{Lang, Serge; Trotter, Hale},
\textit{Primitive points on elliptic curves}. Bull. Amer. Math. Soc. 
\textbf{83} (1977), no. 2, 289--292.

\bibitem{MMS}
\textsc{Murty, M. Ram; Murty, V. Kumar; Saradha, N.},
\textit{Modular Forms and the Chebotarev Density Theorem,}
American Journal of Mathematics, \textbf{110}, No. 2 (1988), 253--281

\bibitem{S1}
\textsc{Serre, Jean-Pierre}, 
Abelian $\ell$-adic representations and elliptic curves. 
With the collaboration of Willem Kuyk and John Labute. 
Second edition. Advanced Book Classics. Addison-Wesley Publishing Company, 
\textit{Advanced Book Program}, Redwood City, CA, 1989.

\bibitem{S2}
\textsc{Serre, Jean-Pierre},
\textit{Propri\'et\'es galoisiennes des points d'ordre fini des courbes elliptiques}. 
(French) Invent. Math. \textbf{15} (1972), no. 4, 259--331.

\bibitem{S3}
\textsc{Serre, Jean-Pierre},
\textit{Quelques applications du th\'eor\`eme de densit\'e de Chebotarev}. (French) 
Inst. Hautes \'Etudes Sci. Publ. Math. No. \textbf{54} (1981), 323--401. 

\end{thebibliography}
\end{scriptsize}
\end{frame}




\end{document}


