\documentclass[10pt,handout]{beamer} %,hyperref={pdfpagelabels=false},draft,handout,handout
\hfuzz=3pt
\usefonttheme{professionalfonts} % using non standard fonts for beamer
\usefonttheme{serif} % default family is serif

\usepackage[english]{babel}
\usepackage{lmodern}
\usepackage[latin1]{inputenc}
\usepackage{times}
\usepackage{amsthm}
\usepackage{hyperref}
%\usepackage[T1]{fontenc}
\usepackage{tikz}
\usepackage{colortbl}
\usepackage{yfonts}
\usepackage{pifont}
\usepackage{translator} % comment this, if not available
\mode<article>
{
 \usepackage{times}
 \usepackage{mathptmx}
 \usepackage[left=1.5cm,right=6cm,top=1.5cm,bottom=3cm]{geometry}
}

 \newcommand{\Q}{\mathbb Q}
 \newcommand{\Z}{\mathbb Z}
 \newcommand{\N}{\mathbb N}
 \newcommand{\F}{\mathbb F}
 \newcommand{\C}{\mathbb C}
 \newcommand{\R}{\mathbb R}
% Common theorem-like environments

\theoremstyle{definition}
\newtheorem{defi}[theorem]{D\'efinition}
\newtheorem{exercise}[theorem]{\translate{Exercise}}
\newtheorem{rem}[theorem]{\translate{Remarque}}
\newtheorem{conj}[theorem]{\translate{Conjecture}}
\newtheorem{proposition}[theorem]{\translate{Proposition}}
\newtheorem{notation}[theorem]{\translate{Notation}}
\newtheorem{Note}[theorem]{\translate{Remarque}}
\newtheorem{theoreme}[theorem]{{Th\'eor\`eme}}
\newtheorem{Theoreme}[theorem]{\translate{Th\'eor\`eme}}
\newtheorem{Block}[theorem]{\translate{}}

% New useful definitions:

%\newbox\mytempbox
%\newdimen\mytempdimen

%\newcommand\includegraphicscopyright[3][]{%
% \leavevmode\vbox{\vskip3pt\raggedright\setbox\mytempbox=\hbox{\includegraphics[#1]{#2}}%
% \mytempdimen=\wd\mytempbox\box\mytempbox\par\vskip1pt%
% \fontsize{3}{3.5}\selectfont{\color{black!25}{\vbox{\hsize=\mytempdimen#3}}}\vskip3pt%
%}}

%\newenvironment{colortabular}[1]{\medskip\rowcolors[]{1}{blue!20}{blue!10}\tabular{#1}\rowcolor{blue!40}}{\endtabular\medskip}

%\def\equad{\leavevmode\hbox{}\quad}

% \newenvironment{greencolortabular}[1]
% {\medskip\rowcolors[]{1}{green!50!black!20}{green!50!black!10}%
% \tabular{#1}\rowcolor{green!50!black!40}}%
% {\endtabular\medskip} 

\lecture[1]{Propri\'et\'es des r\'eductions de groupes de nombres rationnels\\ \ \ \\
\small{Introduction \`{a} la Conjecture d'Artin}}
{Conjecture d'Artin}
\date{}
\title[Dip. Mat. \& Fis.]{\insertlecture}
\subtitle{\ }
\author[Universit\`a Roma Tre]{Francesco Pappalardi}
\institute{Dipartimento di Matematica e Fisica\\
 Universit\`a Roma Tre}

% Beamer version theme settings

\useoutertheme[height=0pt,width=2cm,right]{sidebar}
\usecolortheme{rose,sidebartab}
\useinnertheme{circles}
\usefonttheme[only large]{structurebold}

\setbeamercolor{formul}{fg=black,bg=pink}
\setbeamercolor{sidebar right}{bg=black!15}
\setbeamercolor{structure}{fg=green!50!black}
\setbeamercolor{author}{parent=structure}
\setbeamercolor{postit}{fg=black,bg=yellow}
\setbeamercolor{greys}{fg=black,bg==black!25}
\setbeamerfont{title}{series=\normalfont,size=\LARGE}
\setbeamerfont{title in sidebar}{series=\bfseries}
\setbeamerfont{author in sidebar}{series=\bfseries}
\setbeamerfont*{item}{series=}
\setbeamerfont{frametitle}{size=}
\setbeamerfont{block title}{size=\small}
\setbeamerfont{subtitle}{size=\normalsize,series=\normalfont}
\setbeamertemplate{navigation symbols}{}
\setbeamertemplate{bibliography item}[book]
\setbeamertemplate{sidebar right}
{
 {\usebeamerfont{title in sidebar}%
 \vskip1.5em%
 \hskip3pt%
 \usebeamercolor[fg]{title in sidebar}%
 \insertshorttitle[width=2.1cm,respectlinebreaks]\par% left,
 \vskip1.25em%
 }%
 {%
 \hskip3pt%
 \usebeamercolor[fg]{author in sidebar}%
 \usebeamerfont{author in sidebar}%
 \insertshortauthor[width=2cm,center,respectlinebreaks]\par%
 \vskip1.25em%
 }%
 \hbox to2cm{\hss\insertlogo\hss}
 \vskip1.25em%
 \insertverticalnavigation{2cm}%
 \vfill
 \hbox to 2cm{\hfill\usebeamerfont{subsection in
 sidebar}\strut\usebeamercolor[fg]{subsection in
 sidebar}\insertframenumber\hskip5pt}%
 \vskip3pt%
}%

\setbeamertemplate{title page}
{
 \vbox{}
 \vskip1em
 %{\huge Lecture \insertshortlecture\par}
 {\usebeamercolor[fg]{title}\usebeamerfont{title}\inserttitle\par}%
 \ifx\insertsubtitle\@empty%
 \else%
 \vskip0.25em%
 {\usebeamerfont{subtitle}\usebeamercolor[fg]{subtitle}\insertsubtitle\par}%
 \fi%
 \vskip1em\par
 \textbf{\Large{S\'{e}minaire de Th\'{e}orie des Nombres}}\\
\textsl{Universit\'e de Lom\'e}\\ \emph{21 Juillet 2014}\par
 \vskip0pt plus1filll
 \leftskip=0pt plus1fill\insertauthor\par
 \insertinstitute\vskip1em
}

\logo{\includegraphics[width=1cm]{images/roma3.pdf}}



% Article version layout settings

\mode<article>

\makeatletter
\def\@listI{\leftmargin\leftmargini
 \parsep 0pt
 \topsep 5\p@ \@plus3\p@ \@minus5\p@
 \itemsep0pt}
\let\@listi=\@listI


\setbeamertemplate{frametitle}{\paragraph*{\insertframetitle\
 \ \small\insertframesubtitle}\ \par
}
\setbeamertemplate{frame end}{%
 \marginpar{\scriptsize\hbox to 1cm{\sffamily%
 \hfill\strut\insertframenumber}\hrule height .2pt}}
\setlength{\marginparwidth}{1cm}
\setlength{\marginparsep}{4.5cm}

\def\@maketitle{\makechapter}

% \def\makechapter{
% \newpage
% \null
% \vskip 2em%
% {%
% \parindent=0pt
% \raggedright
% \sffamily
% \vskip8pt
% {\fontsize{36pt}{36pt}\selectfont Kapitel \insertshortlecture \par\vskip2pt}
% {\fontsize{24pt}{28pt}\selectfont \color{blue!50!black} \insertlecture\par\vskip4pt}
% {\Large\selectfont \color{blue!50!black} \insertsubtitle\par}
% \vskip10pt
% % \normalsize\selectfont Druckfassung der
% % Vorlesung \emph{\lecturename} vom \@date\par\vskip1.5em
% %\hfill Till Tantau, Institut f\"ur Theoretische Informatik, Universit\"at zu L\"ubeck
% }
% \par
% \vskip 1.5em%
% }
% 
% \let\origstartsection=\@startsection
% \def\@startsection#1#2#3#4#5#6{%
% \origstartsection{#1}{#2}{#3}{#4}{#5}{#6\normalfont\sffamily\color{blue!50!black}\selectfont}}
% 
% \makeatother
% 
% \mode
% <all>
% 



% Typesetting Listings

\usepackage{listings}
\lstset{language=Java}

\alt<presentation>
{\lstset{%
 basicstyle=\footnotesize\ttfamily,
 commentstyle=\slshape\color{green!50!black},
 keywordstyle=\bfseries\color{blue!50!black},
 identifierstyle=\color{blue},
 stringstyle=\color{orange},
 escapechar=\#,
 emphstyle=\color{red}}
}
{
 \lstset{%
 basicstyle=\ttfamily,
 keywordstyle=\bfseries,
 commentstyle=\itshape,
 escapechar=\#,
 emphstyle=\bfseries\color{red}
 }
}


\begin{document}

\begin{frame}
\titlepage
\end{frame}

\section{Histoire}

\begin{frame}\frametitle{Histoire de la Conjecture d'Artin}
\framesubtitle{La question de Gauss sur les longueurs de p\'eriodes} \pause

\centerline{\begin{beamercolorbox}[shadow=true,center,rounded=true,wd=\textwidth]{formul}
Quels sont les nombres premiers $p$ tels que $1/p$ a une longueur $p-1$?
\end{beamercolorbox}}\bigskip\pause


\begin{columns}[c]
\begin{column}{3cm}<1->
 \includegraphics[width=3cm]{images/Gauss_1803.jpeg}\pause
\end{column}
\begin{column}{7cm}<2->
\begin{minipage}{7cm}
\textit{Par exemple:} \\
$\frac17=0.\overline{142857}$,\\
$\frac1{17}=0,\overline{0588235294117647}$,\\
$\frac1{19}=0.\overline{052631578947368421},$\\
$\vdots $\\
$\frac1{47}=$\tiny{$0.\overline{0212765957446808510638297872340425531914893617}\!\!\!\!\!$}\end{minipage}
\end{column}
\end{columns}\bigskip\pause


\begin{scriptsize}
Quelques nombres premiers avec cette propri\'et\'e: $7, 17, 19, 23, 29, 47, 59, 61, 97, 109, 113, 131, 149, 167, 179, 181, 193,\ldots$
\end{scriptsize}\pause\bigskip

\centerline{\alert{Soit $k_p:=$ la longueur de la p\'eriode de $1/p$}}\smallskip\medskip\pause

$k_3=1,\ k_{11}=2,\ k_{13}=6,$\hfill $k_2$ et $k_5$ sont pas d\'efinis
\end{frame}


\begin{frame}\frametitle{La question de Gauss sur les longueurs de p\'eriodes} \pause

La longueur de p\'eriode de la fraction $1/p$ est le plus petit $k$ tel que
$$\frac1p=0.\overline{a_1\cdots a_k}=0.a_1\cdots a_k\ a_1\cdots a_k\ \ldots$$\pause

En d'autres termes
\begin{eqnarray*}
\frac1p&=&\left(\frac{a_1}{10}+\cdots+\frac{a_k}{10^{k+1}}\right)\times\left(1+\frac1{10^k}+\frac1{10^{2k}}+\cdots\right)\\
 &=&\frac{M}{10^k-1}
\end{eqnarray*}\pause

D'o\`u $$M\times p=10^k-1$$\pause

{\alert{Donc $k_p$ est le plus petit entier $k$ tel que $10^k-1$ est divisible par $p$!}}
\end{frame}

	\section{Faits sur les longueurs de \texorpdfstring{p\'eriodes}{periode}}

\begin{frame}
 \frametitle{Propri\'et\'es alg\'ebrique des longueurs de p\'eriodes}
 
\begin{itemize}[<+-|alert@+>]
 \item La longueur de la p\'eriode $k_p$ de $1/p$ est le plus petit entier $k$ tel que $10^k-1$ est divisible par $p$

 \item Le petit Th\'eor\`eme de Fermat affirme que $10^{p-1}-1$ est divisible par $p$

 \item Donc $k_p\le p-1$ 

 \item En effet, il n'est pas difficile de montrer que $k_p$ est un diviseur de $p-1$

 \item Parfois, la longueur de la p\'eriode est petite:
 
\centerline{\scriptsize{ $1/1111111111111111111=0,\overline{0000000000000000009}$}}

\item la plupart du temps $k_p>\sqrt{p}$\qquad (pas \'evident!)

\item Gauss a, en particulier, demand\'e quelles sont les fr\'equences des p\'eriodes
\end{itemize}
\end{frame}

\begin{frame}
 \frametitle{Quelques statistiques sur la longueur de la p\'eriode:}
 
Soit $k_p$ la longueur de la p\'eriode de $1/p$. Le tableau suivant contient
$$\delta_m=\frac{\{p<2^{31}: k_p=\frac{p-1}{m}\}}{\#\{p\le 2^{31}\}}$$
pour $m=1,\ldots, 40$.\medskip\pause

\!\!{\scriptsize
\begin{tabular}{|r|r|r|r|r|r|r|r|}
\hline
$m$ & 1& 2& 3& 4& 5& 6& 7\\ 
$\delta_m$& 0.37393& 0.28047& 0.06649& 0.07133& 0.01890& 0.04986& 0.00893\\
\hline\hline
$m$ & 8& 9& 10& 11& 12& 13& 14\\ 
$\delta_m$& 0.01660& 0.00739& 0.01416& 0.00340& 0.01268& 0.00240& 0.00669\\
\hline\hline
$m$ & 15& 16& 17& 18& 18& 20& 21\\ 
$\delta_m$& 0.00335& 0.00415& 0.00136& 0.00553& 0.00109& 0.00235& 0.00158\\
\hline\hline
$m$ & 22& 23& 24& 25& 26& 27& 28\\ 
$\delta_m$& 0.00255& 0.00073& 0.00294& 0.00075& 0.00180& 0.00081& 0.00171\\
\hline\hline
$m$ & 29& 30& 31& 32& 33& 34& 35\\ 
$\delta_m$& 0.00046& 0.00251& 0.00039& 0.00103& 0.00060& 0.00103& 0.00044\\
\hline 
%& 0.00140410& 0.000281082& 0.000821037& 0.000423635& 0.00177109& 0.000227703& 0.00119420& 0.000208720& 0.000644002& 0.000372644& 0.000554704& 0.000172678& 0.000738304& 0.000184314& 0.0204208
\end{tabular}\!\!\!\!\!\!}\medskip\pause

\begin{Note}
 $2,94\%$ des nombres premiers $p\le2^{31}$ ont une longueur de p\'eriode $k_p=\frac{p-1}{m}$ avec $m>35$
\end{Note}\pause
\end{frame}

\begin{frame}
 \frametitle{Propri\'et\'es alg\'ebriques des longueurs de p\'eriodes}
\begin{itemize}[<+-|alert@+>]
\item Les p\'eriodes sont \'egalement d\'efinies par rapport \`a n'importe quelle base $a\in\N$
\item La longueur de la p\'eriode de $1/p$ en base $a$ est le plus petit $k_p(a)$ tel que $a^k-1$ est divisible par $p$ (c'est donc un diviseur de $p-1$)
\item Il n'est pas difficile de voir que:\\
\emph{ la longueur de la p\'eriode $k_p(a)$ v\'erifie $k_p(a)=p-1$ si et seulement si l'ensemble
$$\{a^j:\ j=1,\ldots,p-1\}$$
contient $p-1$ \textbf{\'el\'ements distincts modulo $p$}}
\item \emph{en d'autres termes, la longueur de la p\'eriode v\'erifie  $k_p(a)=p-1$ si et seulement si}\medskip

\centerline{$p$ est pas un diviseur de $a^s-a^r\quad\forall r,s:\ 1\le r<s\le p-1$}
\item nous exprimons cette condition par 
$$\langle a\bmod p\rangle=\F_p^*\quad\text{ou encore}\quad\#\langle a\bmod p\rangle=p-1$$
\item Si la longueur de la p\'eriode en base $a$ de $1/p$ est $p-1$ (c'est-\`a-dire $k_p(a)=p-1$), nous disons que \emph{$a$ est une racine primitive modulo $p$}

\end{itemize}
\end{frame}

\begin{frame}
 \frametitle{Propri\'et\'es alg\'ebriques des longueurs de p\'eriodes }
\framesubtitle{(De la longueur des p\'eriodes aux racines primitives)}

\begin{itemize}[<+-|alert@+>]
\item Donc \emph{$a$ est une racine primitive modulo $p$} si et seulement si $\langle a\bmod p\rangle=\F_p^*$\\ (c'est-\`a-dire
s'il y a $p-1$ puissances distinctes de $a$ modulo $p$)
\item 
Il n'est pas difficile de v\'erifier que, si
$p$ est un diviseur de $a$, alors l'\'ecriture de $1/p$ en base $a$ est finie.
\item par exemple $1/2=0.5$, \quad $1/5=0.2$ en base d\'ecimale, $1/10=0.1$ en base binaire
\item on peut \'etendre la d\'efinition de \emph{$a$ est une racine primitive modulo $p$}  au cas o\`u $a$ est un nombre rationnel
 et o\`u $p$ ne divise pas le num\'erateur, ni le d\'enominateur de $a$ (c'est-\`a-dire $v_p(a)=0$)
\item \emph{$a$ est une racine primitive modulo $p$} si et seulement si\medskip

%\textbf{$\!\!\!\!\!\!\!\!\forall$ 
Pour tout nombre premier $\ell$ qui divise $p-1$,
$p$ ne divise pas $a^{(p-1)/\ell}-1$
\item C'\'etait l'intuition d'Artin sur la\medskip

\centerline{ \emph{Conjecture des Racines Primitives}}
\end{itemize}
\end{frame}


\section{La Conjecture d'Artin}
\begin{frame}
\frametitle{La Conjecture d'Artin (1927)}
\begin{Note} Heuristiquement, etond donn\'e un nombre premier $\ell$, la probabilit\'e qu'un nombre premier $p$ v\'erifie les deux propri\'et\'es 
\begin{enumerate}
\item $\ell$ divise $p-1$,
\item $p$ divise $a^{(p-1)/\ell}-1$,
\end{enumerate}
 est $1/\ell(\ell-1)$.\pause

Par cons\'equent, la probabilit\'e pur $p$
que, pour tout premier $\ell$ qui divise $p-1$,  $a^{(p-1)/\ell}-1$ ne soit pas divisible par $p$, est
$$A=\prod_{\ell\ge2}\left(1-\frac1{\ell(\ell-1)}\right)=0,373955\ldots$$
\end{Note}\pause

\begin{defi} [$A$ est appel\'e la \emph{Constante d'Artin}]% (probably transcendental number)
\end{defi}\pause

\begin{conj}%[La conjecture d'Artin]
\centerline{$\lim_{x\rightarrow\infty}\frac{\#\{p\le x:\ p\ne2,5,\ \langle 10\bmod p\rangle=\F_p^*\}}{\#\{p\le x\}}= A$}
\end{conj}\pause

Qu'est-ce si au lieu de $10$, nous consid\'erons un $a\in\Z\setminus\{-1,0,1\}$?
\end{frame}

\begin{frame}
\frametitle{La conjecture d'Artin (1927)}

\centerline{\includegraphics[width=3cm]{images/EmilArtin.jpg}}

\centerline{Emil Artin (Mars 3, 1898 - D\'ecembre 20, 1962)}
\pause


\begin{conj}[La conjecture d'Artin -- (premi\`ere version)] Si $a\in\Q\setminus\left(\{-1,0,1\}\cup\{b^2: b\in\Q\}\right)$, alors
$$\#\{p\le x:\ v_p(a)=0,\ \langle a\bmod p\rangle=\F_p^*\}\sim A\pi(x)$$
\end{conj}\pause

ici $\pi(x)=\#\{p\le x\}$ et $A=\displaystyle{\prod_{\ell\ge2}}1-\frac1{\ell(\ell-1)}=0,37395\ldots$
\end{frame}

\begin{frame}
\frametitle{Certains tests num\'eriques pour la conjecture d'Artin}

Soit $$S_a=\{p\le 2^{29}:\ \langle a\bmod p\rangle=\F_p^*\},\quad d_a=\#S_a/\pi(2^{29})$$ 
Noter que $\pi(2^{29})=28192750$ et $A=0,373955\ldots$. \pause

\begin{center}
\begin{scriptsize}
\begin{tabular}{|c|l|l||r|l|l|}
\hline
$a$ & $S_a$ & $d_a$ & $a$& $S_a$ & $d_a$\\
\hline
-15& 10432805 &0.37005& 2& 10543421& 0.37397\\
-14& 10543340 &0.37397& 3& 10543631& 0.37398 \\
-13& 10542796 &0.37395& 5& 11098098& 0.39365 \\
-12& 12653339 &0.44881& 6& 10543607& 0.37398 \\
-11& 10639090 &0.37736& 7& 10544579& 0.37401 \\
-10& 10543135 &0.37396& 8& 6325893 & 0.22438 \\
-9 &10542743 &0.37395&10& 10542876& 0.37395 \\
-8 &6325704 &0.22437&11& 10542933& 0.37395 \\
-7 &10799148 &0.38304&12& 10545029& 0.37403\\
-6 &10543575 &0.37398&13& 10611720& 0.37639 \\
-5 &10542080 &0.37392&14& 10542946& 0.37395 \\
-4 &10543032 &0.37396&15& 10544134& 0.37400 \\
-3 &12651353 &0.44874&17& 10582932& 0.37537 \\
-2 &10542194 &0.37393&18& 10545385& 0.37404 \\\hline
\end{tabular}\end{scriptsize}
\end{center}
 \pause

Ces r\'esulats num\'eriques ne sont pas toujours totalement convaincants!\pause

\centerline{\alert{Notamment pour $a\in\{-15, -12, -11, -8, -7, -3, 5, 8, 13, 17\}$}}
\end{frame}

\section{Le factor d'enchev\^etrement de Lehmer}

\begin{frame}\frametitle{La conjecture d'Artin}
\framesubtitle{La correction de Lehmer}

\centerline{\includegraphics[width=2.5cm]{images/Lehmer.jpg}}

\centerline{Derrick Henry Lehmer (F\'evrier 1905 - Mai 1991)}
\pause

\begin{rem}[Lehmer] \'Etant donn\'es deux nombres premiers
$\ell_1$ et $\ell_2$, les probabilit\'es pour un nombre premier $p$
de v\'erifier 
\begin{enumerate}
\item $\ell_i$ divise $p-1$
\item $p$ divise $a^{(p-1)/\ell_i}-1$
\end{enumerate}
pour $i=1,2$ ne correspondent pas toujours \`a des \'ev\'enements  ind\'ependants!!
\end{rem}

Donc, il  faut un facteur de correction\\ \pause
(le \emph{facteur d'enchev\^etrement})
\end{frame}

\begin{frame}\frametitle{La conjecture d'Artin}
\framesubtitle{apr\`es la correction de Lehmer}

\begin{conj}[La conjecture d'Artin -- forme finale] Soit $a\in\Q^*\setminus\{1,-1\}$, alors $p-1=\#\langle a\mod p\rangle$
 pour une proportion de nombres premiers $\delta_a$ o\`u
 $$\delta_a=r_a\times t_a,$$
 o\`u si $h=\max\{j: a=b^j,b\in\Q\}$, $\partial(a)=\operatorname{disc}(\Q(\sqrt{a}))$,
 $$t_a=\prod_{\ell\ge2}\left(1-\frac{\gcd(h,\ell)}{\ell(\ell-1)}\right)$$
 et $r_a=1$ si $\partial(a)$ est pair , tandis que si   $\partial(a)$ est impair on a:\\
\centerline{ $r_a=1-\prod_{\ell\mid \partial(a)}\frac{-1}{\ell(\ell-1)/\gcd(\ell,h)-1}$}
\end{conj}

Noter que
\begin{itemize}[<+-|alert@+>]
\item $t_a$ est un multiple rationnel de la constante d'Artin $A$
\item $\delta_a=0$ si et seulement si $a$ est un carr\'e parfait
\item $\partial(a)$ est facile mais technique \`a d\'efinir
\end{itemize}
\end{frame}

\begin{frame}
\frametitle{La conjecture d'Artin}
\framesubtitle{Effet de l'enchev\^etrement Lehmer}

Nous n'avons pas \'et\'e convaincus par les valeurs correspondant \`a $a\in\{-15, -12, -11, -8, -7, -3, 5, 8, 13, 17\}$\pause

\begin{center}
\begin{tabular}{|c|l|l|}
\hline
$a$ & $\delta_a$ & $d_a$ \\
\hline
-15&0.37001 &0.37005\\
-12&0.44875 &0.44881 \\
-11&0.37709 &0.37736\\
-8 &0.22437 &0.22437\\
-7 &0.38308 &0.38304\\
-3 &0.44875 &0.44874\\
 5&0.39363 & 0.39365 \\
 8&0.22437 & 0.22438 \\
13 &0.37636& 0.37639 \\
17 &0.37533& 0.37537 \\
\hline
\end{tabular}\end{center}\pause

Pour toutes les autres valeurs de $a$ dans le tableau pr\'ec\'edent, $\delta_a=A$
\end{frame}

\section{Le resultat de Hooley}

\begin{frame}\frametitle{La conjecture d'Artin}
\framesubtitle{ce qui est connu sur la conjecture d'Artin}

\begin{theoreme}[C. Hooley (1965)] Si 
l'Hypoth\`ese de Riemann G\'en\'eralis\'ee (GRH) est valable pour les corps
 $\Q(a^{1/\ell})$ ($\ell$ nombre premier) 
alors la conjecture d'Artin (forme finale) est valable pour cette valeur de $a$
\end{theoreme}\pause

Qu'est-ce que GRH?\pause

\begin{itemize}[<+-|alert@+>]
\item C'est une conjecture compliqu\'ee en th\'eorie des nombres, si puissante
que souvent on suppose qu'elle est vraie
\item Elle est au del\`a du niveau de ce s\'eminaire
\item Il y a beaucoup de formulations diff\'erentes:
\item \emph{tous les z\'eros non triviaux de la fonction z\^eta de Dedekind sont 
%assis 
sur la droite $\Re s=1/2$}
\item \emph{Les nombres premiers peuvent \^etre compt\'es tr\`es pr\'ecis\'ement}
\end{itemize}
\end{frame}

\section{la quasi--r\'esolution}

\begin{frame}\frametitle{La conjecture d'Artin}
\framesubtitle{la quasi--r\'esolution}\pause


\centerline{
\includegraphics[width=2.5cm]{images/gupta.jpg}
\
\includegraphics[width=2.5cm]{images/murty.jpg}
\
\includegraphics[width=2.5cm]{images/heathbrown.jpg}
}\pause


\begin{theoreme}[R. Gupta, R. Murty \& R. Heath--Brown (1984/86)] $\exists g\in\{2,3,5\}$ t.q.
 $$\#\{p\le x:\ p>5, \langle g\bmod p\rangle=\F_p^*\}\gg\frac{\pi(x)}{\log x}$$
\end{theoreme}
\end{frame}

\section{Un nouveau r\'esultat}
 \begin{frame}
 \frametitle{La Conjecture d'Artin pour les racines quasi--primitives en  rang plus \'elev\'e}
\framesubtitle{travail conjoint avec Andrea Susa}

Notations:\pause

\begin{itemize}[<+-|alert@+>]
\item $\Gamma\subset\Q^*$ Sous-groupe de type fini
\item $r$ rang de $\Gamma$
\item $m \in \N^+$
\item $\sigma_\Gamma=\prod_{p: v_p(x)=0,\exists x \in \Gamma}p$
\item $\forall p\nmid\sigma_\Gamma$
$$\Gamma_p =\{g(\bmod{p}): g\in\Gamma\}\subset\F_p^*$$ 
et on d\'efinit
\item
$N_\Gamma(x,m) := \#\{p\leq x: p\nmid\sigma_\Gamma, |\Gamma_p| =\frac{p-1}m\}$
\item $\Gamma_p$ g\'en\'eralise la notion de $\langle a\bmod p\rangle$.
\item si $\Gamma=\langle a\rangle$ est de rang $1$, alors\\
$$N_\langle a\rangle(x,m) =\#\{p\leq x: \frac1p\text{ a une p\'eriode de longueur } \frac{p-1}m\}$$
\end{itemize}
\end{frame}

 \begin{frame}
 \frametitle{La Conjecture d'Artin pour les racines quasi--primitives en rang plus \'elev\'e}
\framesubtitle{travail conjoint avec Andrea Susa}

\begin{theoreme} Soit $\Gamma \subset \Q^*$ avec rang $r\ge2$. Soit $m\in\N$ 
 et supposons GRH   v\'erifi\'e pour $\Q(\zeta_{k},\Gamma^{1/k})$ ($k\in\N$). 
Alors, $\forall\epsilon>0$ et 
$m\le x^{\frac{r-1}{(r+1)(4r+2)}-\epsilon}$,\pause
$$N_{\Gamma}(x,m)=\left(\rho(\Gamma,m)+O\!\left(\frac{1}{\varphi(m^{r+1})\log^rx} %+x^{\frac{-9r+11}{20r}}\log^2 x
\right)\right) \pi(x),$$
o\`u\pause
$$\rho(\Gamma,m)= \sum_{ k \geq 1}
\frac{\mu(k)}{[\Q(\zeta_{mk},\Gamma^{1/mk}):\Q]}.
$$
\end{theoreme}\pause

Un analogue du r\'esultat ci-dessus est \'egalement valable, dans le cas o\`u $\Gamma\subset\Q^*$ est de rang infini.
\end{frame}
 
 \begin{frame}
 \frametitle{La Conjecture d'Artin pour les racines quasi--primitives en rang plus \'elev\'e}
\framesubtitle{travail en commun avec Andrea Susa}


\begin{theoreme}
 Soit $\Gamma \subset \Q^+=\{q\in\Q; q>0\}$ de rang $r\ge2$ et soit $m\in\N$. Soit $\Gamma(m):=\Gamma(\Q^*)^m/(\Q^*)^m$,\medskip

{\scriptsize{
 \centerline{$\displaystyle{A_{\Gamma,m}=\frac1{\varphi(m)|\Gamma(m)|}\times
\prod_{\substack{\ell>2\\ \ell\nmid
m}}\left(1-\frac1{(\ell-1)|\Gamma(\ell)|}\right)
\times\prod_{\substack{\ell>2\\ \ell\mid
m}}\left(1-\frac{|\Gamma(\ell^{v_\ell(m)})|}{\ell|\Gamma(\ell^{1+v_\ell(m)})|}\right)
}$}}}\medskip

et \medskip

{\scriptsize{
 \centerline{$\displaystyle{ B_{\Gamma,k} =\sum_{\substack{
\eta\mid\sigma_\Gamma\\
 \eta^{2^{v_2(k)-1}}\!\!\!\cdot{\Q^*}^{2^{v_2(k)}}\in\Gamma(2^{v_2(k)})\\ 
v_2(\partial(\eta))\le k}}\prod_{\substack{\ell\mid \partial(\eta)\\
\ell\nmid k}}\frac{-1}{(\ell-1)|\Gamma(\ell)|-1}.
}$}}}

Alors
$$\rho(\Gamma,m) = A_{\Gamma,m}\left( B_{\Gamma,m}
-\frac{|\Gamma(2^{v_{2}(m)})|}{(2,m)|\Gamma(2^{1+v_{2}(m)})|} B_{\Gamma,2m}\right).$$
\end{theoreme}
\end{frame}
 
\begin{frame}
\frametitle{La Conjecture d'Artin pour les racines quasi--primitives en rang plus \'elev\'e}
\framesubtitle{densit\'e nulle}

\begin{theoreme}\label{finite} Soit $\Gamma\subset\Q^+$ de type fini, $m\in\N$. Alors
\\
\centerline{$\rho(\Gamma,m)=0$}


si l'une des conditions suivantes est remplie::
\begin{enumerate}
 \item $2\nmid m$ et pour tous $g\in\Gamma, \partial(g)\mid m$;
 \item $2\mid m$, $3\nmid m$, $\Gamma(3)=\{1\}$ et $\exists \eta\mid\sigma_\Gamma,$
\hspace*{4cm} \begin{minipage}{5cm}\begin{itemize}
 \item $\eta^{2^{v_2(m/2)}}\!\!\!\!\cdot{\Q^*}^{2^{v_2(m)}}\in\Gamma(2^{v_2(m)})$
 \item $\partial(-3\eta)\mid m$ %$v_2(\partial(\eta))\le m$ t.q. $3$ is the only odd nombre premier that divise $\partial(\eta)$ et that doesn't divide $m$. 
 \end{itemize}\end{minipage}
 \end{enumerate}
(si $2\nmid m$, (1) est \'egalement n\'ecessaire pour $\rho(\Gamma,m)=0$).
Si $\Gamma\subset\Q^+$ et $m$ v\'erifie (1) ou (2) ci--dessus, alors\\
\centerline{$\{p: \text{ind}_p\Gamma=m\}\quad\text{est fini}.$}

Par cons\'equent, sous GRH, si $2\nmid m$,\\
\centerline{$\{p: \text{ind}_p\Gamma=m\}\quad\text{finie} \Longleftrightarrow \forall g\in\Gamma, \partial(g)\mid m.$}
\end{theoreme}
\end{frame}

\end{document}
