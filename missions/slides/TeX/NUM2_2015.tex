\documentclass[10pt,handout]{beamer} %,hyperref={pdfpagelabels=false},draft,handout,handout
\usepackage[orientation=landscape,size=custom,width=16,height=9,scale=0.30,debug]{beamerposter} 
\usepackage[english]{babel}
\usepackage{lmodern}% http://ctan.org/pkg/lm
\usepackage[latin1]{inputenc}
\usepackage{times,hyperref,tikz,colortbl,yfonts,translator,marvosym,pifont}
\usepackage[T1]{fontenc}
 \newcommand{\Q}{\mathbb Q}
 \newcommand{\Z}{\mathbb Z}
 \newcommand{\N}{\mathbb N}
 \newcommand{\F}{\mathbb F}
 \newcommand{\C}{\mathbb C}
 \newcommand{\R}{\mathbb R}
\useoutertheme[height=0pt,width=2cm,right]{sidebar}
\usecolortheme{rose,sidebartab}
\useinnertheme{circles}
\usefonttheme[only large]{structurebold}
\theoremstyle{definition}
\newtheorem{exercise}[theorem]{\translate{Exercise}}
\newtheorem{Note}[theorem]{\translate{Note}}
\lecture[4]{Elliptic curves over finite fields}{First Lecture}
\title[Elliptic curves over $\F_{q}$]{\insertlecture}
\setbeamercolor{formul}{fg=black,bg=pink}
\setbeamercolor{sidebar right}{bg=green!15}
\setbeamercolor{structure}{fg=black!120}
\setbeamercolor{postit}{fg=black,bg=yellow}
\setbeamercolor{greys}{fg=black,bg==black!25}
\setbeamerfont{title in sidebar}{series=\bfseries}
\setbeamerfont*{item}{series=}
\setbeamerfont{frametitle}{size=}
\setbeamerfont{block title}{size=\small}
\setbeamerfont{subtitle}{size=\normalsize,series=\normalfont}
\begin{document}

\begin{frame}
\includegraphics[width=1.6cm]{images/roma3.pdf}\hfill\includegraphics[width=1.9cm]{images/NUM2.jpeg}
\vfill

\begin{center}\begin{sc}
\begin{Large}

\textcolor{red}{Elliptic curves Cryptography}
\end{Large}\bigskip

\ {Francesco Pappalardi}\bigskip\bigskip

\begin{large}\begin{bf}\#2 - Second Lecture.
\end{bf}\end{large}\medskip

September $15^{\text{th}}$ 2015\medskip
\vfill
\end{sc}\end{center}

%\includegraphics[width=1.6cm]{images/cimpalogo.pdf}\hfill
\begin{minipage}[b]{9.3cm}
\textbf{National University of Mongolia}\\  %Монгол Улсын Их Сургууль
Ulan Baatar, Mongolia\\
September 15, 2015
\end{minipage}\hfill
%\includegraphics[width=1.9cm]{images/seams.png}
\end{frame}

\section{Reviews on PKC}
\begin{frame}
\frametitle{\textcolor{black}{\textbf{Private key}} versus \alert{Public Key}
}\pause\medskip

\centerline{\includegraphics[width=8cm]{images/figure2.pdf}}

\end{frame}

\begin{frame}
\frametitle{\textcolor{black}{\textbf{Private key}} versus \alert{Public Key}
}\pause\medskip

\centerline{\includegraphics[width=8cm]{images/figure3.pdf}}

\end{frame}


\begin{frame}
\frametitle{Classical General Examples of PKC}\bigskip

\begin{itemize}
  \item[\textcolor{blue}{\ding{182}}] {(1976) Diffie Hellmann Key exchange protocol}
   {\emph{IEEE Trans. Information Theory IT-22 (1976)}}
  \item[\textcolor{blue}{\ding{183}}]  {(1983) Massey Omura Cryptosystem}
{\emph{Proc. $4^{th}$ Benelux Symposium on Information Theory (1983)}}
  \item[\textcolor{blue}{\ding{184}}]  {(1984) ElGamal Cryptosystem}
  {\emph{IEEE Trans. Information Theory IT-31 (1985)}}
\end{itemize}\pause

\centerline{\includegraphics[width=6cm]{images/image033.jpg}}
\end{frame}


\begin{frame}
\frametitle{PKS}

\centerline{\includegraphics[width=9.4cm]{images/Satellite.jpg}}

\end{frame}

\subsection{DH}
\begin{frame}
\frametitle{Diffie--Hellmann key exchange}\pause

\begin{beamerboxesrounded}[upper=block title example,lower=block body alerted,shadow=true]{DHKEP}
\begin{itemize}
\item[\textcolor{blue}{\ding{182}}] \textbf{Alice} and
\textbf{Bob} agree on a cyclic group $G$ and on a \textit{\underline{generator}} $g$ in
$G$
 \item[\textcolor{blue}{\ding{183}}] {\textbf{Alice}
 picks a \textcolor{red}{secret} $a$,} {$0\leq a\leq |G|$}
 \item[\textcolor{blue}{\ding{184}}] {\textbf{Bob} picks
 a \textcolor{red}{secret} $b$, $0\leq b\leq |G|$}
 \item[\textcolor{blue}{\ding{185}}] {They compute and publish $g^a$
 (\textbf{Alice}) and $g^b$ (\textbf{Bob})}
 \item[\textcolor{blue}{\ding{186}}] {The common
 \textcolor{red}{secret} key is $g^{ab}$}
\end{itemize}
\end{beamerboxesrounded}

\centerline{\includegraphics[width=7cm]{images/dh.jpg}}
\end{frame}

\subsection{ElGamal}
\begin{frame}
\frametitle{ElGamal Cryptosystem }

\textbf{Alice} wants to sent a message $x\in G$ (cyclic group) to \textbf{Bob}

\begin{beamerboxesrounded}[upper=block title example,lower=block body alerted,shadow=true]{ElGamal SETUP:}
 \begin{itemize}
\item[\textcolor{blue}{\ding{182}}] \textbf{Alice} and \textbf{Bob} agree on a \textit{\underline{generator}} $g$ in $G$
\item[\textcolor{blue}{\ding{183}}] \textbf{Bob} picks a \textcolor{red}{secret} $b$, $0< b\leq |G|$, he computes $\beta=g^b\in G$ and publishes $\beta$
\end{itemize}
\end{beamerboxesrounded}\bigskip

\begin{beamerboxesrounded}[upper=block title example,lower=block body alerted,shadow=true]{ElGamal ENCRYPTION: \alert{(Alice)}}
\begin{itemize}
\item[\textcolor{red}{\ding{172}}] \textbf{Alice} picks a \textcolor{red}{secret} $k$, $0< k\leq |G|$
\item[\textcolor{red}{\ding{173}}] She computes $\alpha=g^k\in G$ and $\gamma=x\cdot \beta^k\in G$
\item[\textcolor{red}{\ding{174}}] The encrypted message is $\quad E(x)=(\alpha,\gamma)\in G\times G$
\end{itemize}
\end{beamerboxesrounded}\bigskip


\begin{beamerboxesrounded}[upper=block title example,lower=block body alerted,shadow=true]{ElGamal DECRYPTION: \textcolor{blue}{(Bob)}}
\begin{itemize}
\item[\textcolor{blue}{\ding{172}}] \textbf{Bob} computes $\quad D(\alpha,\gamma)=\gamma\cdot\alpha^{|G|-b}$
\item[\textcolor{blue}{\ding{173}}] It works since $\quad D(E(x))=D(\alpha,\gamma)=x\cdot g^{bk}\cdot g^{k(|G|-b)}=x$ since $g^{k|G|} =1$
\end{itemize} 
\end{beamerboxesrounded}

%\textbf{\textcolor{black}{Eve}} can decrypt the message if he can compute the discrete logarithm $X$, $\beta=g^X\bmod p$
\end{frame}

\subsection{Massey -- Omura}
\begin{frame}
\frametitle{Massey Omura on any finite Group $G$}\pause

\centerline{\textbf{Alice}\ \includegraphics[width=3.6cm]{images/MasseyOmura.jpg}\ \textbf{Bob}}\pause

\begin{beamerboxesrounded}[upper=block title example,lower=block body alerted,shadow=true]{SETUP:}
\begin{itemize}
\item[\textcolor{red}{\ding{172}}]
\textbf{Alice} and \textbf{Bob} each 
\begin{itemize}
\item pick a secret key $k_A,k_B\in U(\Z/|G|\Z)$
\item compute $\ell_A,\ell_B\in U(\Z/|G|\Z)$ such that
$k_A\ell_A\equiv 1(\bmod |G|)$ and $k_B\ell_B\equiv1(\bmod |G|)$
\end{itemize}
\item[\textcolor{red}{\ding{175}}]
{\textbf{Alice} key is $(k_A,\ell_A)$ ($k_A$ to lock and $\ell_A$ to unlock)}
\item[\textcolor{red}{\ding{176}}]
{\textbf{Bob} key is $(k_B,\ell_B)$ ($k_B$ to lock and $\ell_B$ to unlock)}
\end{itemize} 
\end{beamerboxesrounded}

%\centerline{\textbf{Alice} $(k_A,l_A)$\ \includegraphics[width=4cm]{images/MasseyOmura.jpg}\ \textbf{Bob} $(k_B,l_B)$}\pause

\begin{beamerboxesrounded}[upper=block title example,lower=block body alerted,shadow=true]{WORKING: To send the message $P$} 
\begin{itemize}
\item[\textcolor{blue}{\ding{172}}] {\textbf{Alice} computes and sends
$M=P^{k_A}\in G$}
\item[\textcolor{blue}{\ding{173}}] {\textbf{Bob} computes and sends back $N=M^{k_B}\in G$}
\item[\textcolor{blue}{\ding{174}}] {\textbf{Alice} computes $L=N^{\ell_A}\in G$ and sends
it back to \textbf{Bob}}
\item[\textcolor{blue}{\ding{175}}] {\textbf{Bob} decrypt the message computing $P=L^{\ell_B}\in G$}
\end{itemize}\pause
\end{beamerboxesrounded}

It works: $P=L^{\ell_B}=N^{\ell_A\ell_B}=M^{k_B\ell_Al_B}=P^{{k_Ak_B\ell_A\ell_B}}\in G$
\end{frame}




\section{Discrete Logarithms} 


\begin{frame}
\frametitle{The generic Discrete Logarithms problem}

\begin{itemize}
 \item  $G=\langle g\rangle$ cyclic group
 \item $g$ a generator
 \item $x\in G$
\end{itemize}

\begin{beamerboxesrounded}[upper=block title example,lower=block body alerted,shadow=true]{Discrete Logarithm Problem:}
\centerline{\begin{beamercolorbox}[rounded=true,shadow=true,wd=5cm,center]{postit}
Find $n\in\Z/|G|\Z$ such that $x=g^n$
                      \end{beamercolorbox}}
 \end{beamerboxesrounded}

\begin{itemize}[<+-| alert@+>]
 \item {Need to specify how to make the operations in $G$}
\item {If $G=\left(\Z/n\Z,+\right)$ then discrete logs are very easy.}
\item If $G=((\Z/n\Z)^*,\times)$ then $G$ is cyclic iff $n=2,4,p^\alpha,2\cdot p^\alpha$
where
$p$ is an odd prime: famous theorem of Gau\ss.
\item In $G=(\Z/p\Z)^*=:\F_p^*$ there is no efficient algorithm to compute DL.
\item \alert{We are interested in the case when $G=E(\F_q)$ where $E/\F_q$ is an elliptic curve}
\item Primordial public key cryptography is based on the difficulty of the Discrete Log problem
%\item Several algorithms to compute discrete logarithms are known. One for all is the \textbf{Shanks Baby Step Giant Step algorithm}:
\end{itemize}
\end{frame}

\subsection{DL Attacks}

\begin{frame}
\frametitle{Classical DL attacks}


 \begin{itemize}[<+-| alert@+>]
\item[\textcolor{red}{\ding{44}}] Shanks \alert{baby-step,
giant step} (BSGS) \textit{Proc. $2^{nd}$ Manitoba Conf. Numerical
Mathematics (Winnipeg, 1972).}
\item[\textcolor{red}{\ding{44}}] Pohlig--Hellmann
Algorithm \textit{IEEE Trans. Information Theory IT-24
(1978).}
\item[\textcolor{red}{\ding{44}}] Index computation
algorithm
\item[\textcolor{red}{\ding{44}}] Sieving algorithms
\emph{La Macchia \& Odlyzko, Designs Codes and Cryptography 1 (1991)}
\end{itemize}\pause\bigskip


\textbf{NOTE:} The last two are "very special" for $\F_q^*$

\end{frame}

\subsection{BSGS}
\begin{frame}
\frametitle{\textcolor{red}{\textsc{Discrete Logarithms:}} continues}
\framesubtitle{Shanks Baby Step Giant Step algorithm}



 \begin{center}\fbox{\textcolor{black}{
 \begin{minipage}[c]{11cm}
\texttt{\noindent 
\noindent 
\textcolor{red}{Input:}  A group $G=\langle g\rangle$ and $a\in G$\\
\textcolor{blue}{Output:}  $k\in\Z/|G|\Z$ such that $a=g^k$\\
1. $M:= \lceil \sqrt{|G|}\rceil$\\
2. For $j=0,1,2,\ldots,M$.\\
\hspace*{1cm} \qquad Compute $g^j$ and store the pair $(j, g^j)$ in a table\\
3. $A:=g^{-M}$, $B:=a$\\
5. For $i=0,1,2,\ldots,M-1$.\\
\hspace*{5mm} \qquad -1- Check if $B$ is the second component $(g^j)$ of any\\ \hspace*{1.3cm} \qquad pair in the table\\
\hspace*{5mm} \qquad -2- If so, return $iM + j$ and halt.\\  
\hspace*{5mm} \qquad -3- If not $B=B\cdot A$}
 \end{minipage}}}
 \end{center}
% Input: A cyclic group G of order n, having a generator α and an element β.
% 
% Output: A value x satisfying αx = β.
% 
%    1. m ← Ceiling(√n)
%    2. For all j where 0 ≤ j < m:
%          1. Compute αj and store the pair (j, αj) in a table. (See section "In practice")
%    3. Compute α−m.
%    4. γ ← β. (set γ = β)
%    5. For i = 0 to (m − 1):
%          1. Check to see if γ is the second component (αj) of any pair in the table.
%          2. If so, return im + j.
%          3. If not, γ ← γ • α−m.

\begin{itemize}[<+-| alert@+>]
  \item The BSGS algorithm is a generic algorithm. It works for every finite cyclic group.
 \item based on the fact that $\forall x\in\Z/n\Z,$  $x=j+ i m$ with $m=\lceil\sqrt{n}\rceil$, $0\le j<m$ and $0\le i<m$
 \item  Not necessary to know the order of the group $G$ in advance. The algorithm still works if an upper bound on the group order is known.
 \item Usually the BSGS algorithm is used for groups whose order is prime.
 \item The running time of the algorithm and the space complexity is $O(\sqrt{|G|})$, much better than the $O(|G|)$ 
 running time of the naive brute force
 \item The algorithm was originally developed by Daniel Shanks.
 \end{itemize}
\end{frame}

\subsection{Pohlih--Hellmann}
\begin{frame}
\frametitle{\textcolor{red}{\textsc{Discrete Logarithms:}} continues} 
\framesubtitle{The Pohlig--Hellman Algorithm}

 In some groups Discrete logs are easy. For example if $G$ is a cyclic group and $\#G=2^m$ then we know
that there are subgroups:\pause
$$\langle1\rangle=G_0\subset G_1\subset\cdots\subset G_m=G$$\pause
such that $G_i$ is cyclic and $\#G_i=2^i$. Furthermore
$$G_i=\left\{y\in G\text{ such that } y^{2^i}=1\right\}.$$\pause
If $G=\langle g\rangle$, for any $a\in G$, either $a^{2^{m-1}}=1$
or $a^{2^{m-1}}=g^{2^{m-1}}$. \pause
From this property we deduce the algorithm:
\begin{center}\fbox
{\textcolor{black}{
\begin{minipage}[c]{9cm}
\texttt{\noindent 
\textcolor{red}{Input:}  A group $G=\langle g\rangle$, $|G|=2^m$ and $a\in G$\\
\textcolor{blue}{Output:}  $k\in\Z/|G|\Z$ such that $a=g^k$\\
1. $A:=a$, $K=0$\\
2. For $j=1,2,\ldots, m$.\\
\hspace*{.7cm} \qquad If $A^{2^{m-j}}\neq 1$, $A:=g^{-2^{j-1}}\cdot A; K:=K+2^{j-1}$\\
3. Output $K$}
\end{minipage}}}
\end{center}
\end{frame}

\begin{frame}
\frametitle{\textcolor{red}{\textsc{Discrete Logarithms:}} continues} 
\framesubtitle{The Pohlig--Hellman Algorithm}

\begin{itemize}[<+-| alert@+>]
\item The above is a special case of the Pohlig-Hellman Algorithm which can be extended to the case when $|G|$ has only small prime divisors
\item To avoid this situation one crucial requirement for a DL-resistent group in cryptography is that $\#G$ has a large prime divisor
\item If $p=2^k+1$ is a Fermat prime, then DL in $(\F_p)^*$ are easy
\item Classical algorithm for factoring have often analogues for computing discrete logs. A very important one  is the \emph{Pollard $\rho$--method}
\item One of the strongest algorithms is the \alert{index calculus algorithm.} NOT generic. It works only in ${\mathbb F}_q^*$
\end{itemize}
\end{frame}
\subsection{DL records}
\begin{frame}
\frametitle{\textcolor{red}{\textsc{Discrete Logarithms:}} continues} 
\framesubtitle{Records}

\begin{beamerboxesrounded}[upper=block title example,lower=block body alerted,shadow=true]{Discrete Logarithm Records:}
 \begin{itemize}[<+-| alert@+>]
  \item $G=\F_p^*$: $p\approx 10^{180}$ (596-bit) \\ \hfill Cyril Bouvier, Pierrick Gaudry, Laurent Imbert, Hamza Jeljeli and Emmanuel Thom\'e (11 June 2014)
  \item $G=\F_{p^2}^*$: $p\approx 10^{80}$ \\ \hfill Razvan Barbulescu, Pierrick Gaudry, Aurore Guillevic, and Fran\c{c}ois Morain (25 June 2014)
  \item $G=\F_{2^\alpha}^*$: $\alpha=1279$ \\ \hfill Thorsten Kleinjung (17 October 2014)
  \item $G=E(\F_p)$: $p\approx 10^{35}$  \\ \hfill Joppe W. Bos, Marcelo E. Kaihara, T. Kleinjung, Arjen K. Lenstra and Peter L. Montgomery (July 2009)  \\
  \hfill $p=4451685225093714772084598273548427$
  \item $G=E(\F_{2^\alpha})$: $\alpha=113$ \hfill  Erich Wenger and Paul Wolfger (January 2015)
 \end{itemize}
\end{beamerboxesrounded}\bigskip\pause

\centerline{\begin{beamercolorbox}[rounded=true,shadow=true,wd=7cm,center]{postit}
with $ECC$ same security with $1/5$ of the size 
                         \end{beamercolorbox}}
\end{frame}

\section{Square roots}

\begin{frame}
\frametitle{The problem of ``Square Roots Modulo a prime''} 

\begin{beamerboxesrounded}[upper=block title example,lower=block body alerted,shadow=true]{Given an odd prime $p$ and a quadratic residue $a$}
\centerline{\begin{beamercolorbox}[rounded=true,shadow=true,wd=5cm,center]{postit}
Find $x$ such that $x^2\equiv a\bmod p$
                      \end{beamercolorbox}}
 \end{beamerboxesrounded}\pause

It can be solved efficiently if we are given a \alert{quadratic nonresidue} $g\in(\Z/p\Z)^*$\pause

\texttt{
\begin{enumerate}[<+-| alert@+>]
\item {Write $p-1=2^k\cdot q$ and we know that $(\Z/p\Z)^*$ has a (cyclic) subgroup $G$ with $2^k$ elements.}
\item { Note that $b=g^q$ is a generator of $G$
(in fact if it was $b^{2^j}\equiv1\bmod p$ for $j<k$, then $g^{(p-1)/2}\equiv1\bmod p$) and that $a^q\in G$}
\item {Use the last algorithm to compute $t$ such that $a^q=b^t$. Note that $t$ is even since
$a^{(p-1)/2}\equiv1\bmod p$.}
\item {Finally set $x=a^{(p-q)/2}b^{t/2}$ and observe that $\displaystyle{x^2=a^{(p-q)}b^{t}=a^p\equiv a\bmod p.}$}
\end{enumerate}\pause}

\begin{beamerboxesrounded}[upper=block title example,lower=block body alerted,shadow=true]{REMARKS:}
\begin{itemize}[<+-| alert@+>]
 \item The above is not deterministic. However Schoof in 1985 discovered a polynomial time algorithm which is
however not efficient.
\item To find a random point in an elliptic curve $E/\F_p$ one needs to compute square roots modulo $p$
\end{itemize}
\end{beamerboxesrounded}
\end{frame}


\begin{frame}
\frametitle{The problem of ``Modular Square Roots''} 

\begin{beamerboxesrounded}[upper=block title example,lower=block body alerted,shadow=true]{Given $n,a\in\N$}
\centerline{\begin{beamercolorbox}[rounded=true,shadow=true,wd=5cm,center]{postit}
Find $x$ (if it exists) such that $x^2\equiv a\bmod n$
                      \end{beamercolorbox}}
 \end{beamerboxesrounded}\pause

If the factorization of $n$ is known, then this problem (efficiently) can be solved in 3 steps:

{\begin{enumerate}[<+-| alert@+>]
 \item {For each prime divisor $p$ of $n$ find $x_p$ such that $x_p^2\equiv a \bmod p$}
\item {Use the Hensel's Lemma to lift $x_p$ to $y_p$ where $y_p^2\equiv a\bmod p^{v_p(n)}$}
\item {Use the Chinese remainder Theorem to find $x\in\Z/n\Z$ such that}\\ {$x\equiv y_p\bmod p^{v_p(n)} \ \forall p\mid n$.}
\item {Finally $x^2\equiv a\bmod n$.}
\end{enumerate}}\pause

%The last two tools (Hensel's Lemma and Chinese Remainder Theorem) will be covered later
 \end{frame}




\section{Reminder from Yesterday}

\begin{frame}
\frametitle{Reminder from Yesterday}

If $P,Q\in E(\F_q), r_{P,Q}:\begin{cases}
                     \text{line through $P$ and }Q &\text{if }P\neq Q\\
                     \text{tangent line to $E$ at }P &\text{if }P=Q,
                    \end{cases}$\\ \ \hfill $r_{P,\infty}:$ vertical line through $P$

\begin{center}
\includegraphics[width=4.9cm]{images/ad15.pdf}\includegraphics[width=4.9cm]{images/add7.pdf}\pause
\end{center}

{$r_{P,\infty}\cap E(\F_q)=\{P,\infty,P'\}$}\hfill$\rightsquigarrow$
{\begin{beamercolorbox}[shadow=true,center,rounded=true,wd=2cm]{formul}
             $-P:=P'$
            \end{beamercolorbox}}\medskip

{$r_{P,Q}\cap E(\F_q)=\{P,Q,R\}$}\hfill$\rightsquigarrow$
{\begin{beamercolorbox}[shadow=true,center,rounded=true,wd=2.9cm]{formul}
$P+_E Q:=-R$
            \end{beamercolorbox}}
\end{frame}

\begin{frame}
\frametitle{Formulas for Addition on $E$ (Summary)}
\centerline{\begin{beamercolorbox}[shadow=true,center,rounded=true,wd=6cm]{formul}
$E: y^2+a_1xy+a_3y=x^3+a_2x^2+a_4x+a_6$\end{beamercolorbox}}
$P_1 = (x_1, y_1), P_2 = (x_2, y_2)\in E(\F_q)\setminus\{\infty\}$,
\begin{beamerboxesrounded}[upper=block title example,lower=block body alerted,shadow=true]{Addition Laws for the sum of affine points}
\begin{itemize}
 \item If $P_1\neq P_2$
\begin{itemize}
 \item $x_1 = x_2\ \hfill\Rightarrow\hfil$\ \ \
\begin{beamercolorbox}[shadow=true,center,rounded=true,wd=2cm]{formul}$P_1 +_E P_2 = \infty$
\end{beamercolorbox}
 \item $x_1 \neq x_2$\\
\centerline{\begin{beamercolorbox}[shadow=true,center,wd=4cm]{postit}
             $\displaystyle\lambda=\frac{y_2-y_1}{x_2-x_1}\qquad \nu=\frac{y_1x_2-y_2x_1}{x_2-x_1}$
            \end{beamercolorbox}}
 \end{itemize}
\item If $P_1 = P_2$
\begin{itemize}
 \item $2y_1+a_1x+a_3 = 0\ \hfill\Rightarrow\hfil$\ \ \
\begin{beamercolorbox}[shadow=true,center,rounded=true,wd=3cm]{formul}$P_1 +_E P_2 = 2P_1 = \infty$\end{beamercolorbox}
\item $2y_1+a_1x+a_3\neq 0$\\
\centerline{\begin{beamercolorbox}[shadow=true,center,wd=7cm]{postit}
$\displaystyle\lambda=\frac{3x_1^2+2a_2x_1+a_4-a_1y_1}{2y_1+a_1x+a_3}, \nu=-\frac{a_3y_1+x_1^3-a_4x_1-2a_6}{2y_1+a_1x_1+a_3}$
            \end{beamercolorbox}}
\end{itemize}
\end{itemize}

Then\\
\centerline{\begin{beamercolorbox}[shadow=true,center,rounded=true,wd=11cm]{formul}
{\small $P_1 +_E P_2 = ({\color[cmyk]{0,1,1,0.5}\lambda^2-a_1\lambda-a_2-x_1-x_2},
{\color[cmyk]{1,0,1,0.5}-\lambda^3-a_1^2\lambda+(\lambda+a_1)(a_2+x_1+x_2)-a_3-\nu})$}
            \end{beamercolorbox}}
\end{beamerboxesrounded}
\end{frame}

\begin{frame}
\frametitle{Formulas for Addition on $E$ (Summary for special equation)}
\centerline{\begin{beamercolorbox}[shadow=true,center,rounded=true,wd=6cm]{formul}
$E: y^2=x^3+Ax+B$\end{beamercolorbox}}
$P_1 = (x_1, y_1), P_2 = (x_2, y_2)\in E(\F_q)\setminus\{\infty\}$,
\begin{beamerboxesrounded}[upper=block title example,lower=block body alerted,shadow=true]{Addition Laws for  the sum of affine points}
\begin{itemize}
 \item If $P_1\neq P_2$
\begin{itemize}
 \item $x_1 = x_2\ \hfill\Rightarrow\hfil$\ \ \
\begin{beamercolorbox}[shadow=true,center,rounded=true,wd=2cm]{formul}$P_1 +_E P_2 = \infty$
\end{beamercolorbox}
 \item $x_1 \neq x_2$\\
\centerline{\begin{beamercolorbox}[shadow=true,center,wd=4cm]{postit}
             $\displaystyle\lambda=\frac{y_2-y_1}{x_2-x_1}\qquad \nu=\frac{y_1x_2-y_2x_1}{x_2-x_1}$
            \end{beamercolorbox}}
 \end{itemize}
\item If $P_1 = P_2$
\begin{itemize}
 \item $y_1 = 0\ \hfill\Rightarrow\hfil$\ \ \
\begin{beamercolorbox}[shadow=true,center,rounded=true,wd=3cm]{formul}$P_1 +_E P_2 = 2P_1 = \infty$\end{beamercolorbox}
\item $y_1\neq 0$\\
\centerline{\begin{beamercolorbox}[shadow=true,center,wd=7cm]{postit}
$\displaystyle\lambda=\frac{3x_1^2+A}{2y_1}, \nu=-\frac{x_1^3-Ax_1-2B}{2y_1}$
            \end{beamercolorbox}}
\end{itemize}
\end{itemize}

Then\\
\centerline{\begin{beamercolorbox}[shadow=true,center,rounded=true,wd=11cm]{formul}
{\small $P_1 +_E P_2 = ({\color[cmyk]{0,1,1,0.5}\lambda^2-x_1-x_2},
{\color[cmyk]{1,0,1,0.5}-\lambda^3+\lambda(x_1+x_2)-\nu})$}
            \end{beamercolorbox}}
\end{beamerboxesrounded}

\end{frame}


\begin{frame}\frametitle{The division polynomials}

\begin{Definition}[Division Polynomials of $E:y^2=x^3+Ax+B$ ($p>3$)]\vspace*{-0.7cm}
\begin{align*}
        \psi_{0} =& 0,
        \psi_{1} = 1,
        \psi_{2} = 2y\\
        \psi_{3} =& 3x^{4} + 6Ax^{2} + 12Bx - A^{2}\\
        \psi_{4} =& 4y(x^{6} + 5Ax^{4} + 20Bx^{3} - 5A^{2}x^{2} - 4ABx - 8B^{2} - A^{3}) \\
        &\vdots\\
        \psi_{2m+1} =& \psi_{m+2}\psi_{m}^{3}-\psi_{m-1}\psi^{3}_{m+1} \qquad \text{ for } m \geq 2\\
        \psi_{2m}  =& \left(\frac{\psi_{m}}{2y}\right)\cdot(\psi_{m+2}\psi^{2}_{m-1}-\psi_{m-2}\psi^{2}_{m+1}) \quad \text{ for } m \geq 3
\end{align*}
The polynomial $\psi_m\in{\mathbb Z}[x,y]$ is the $m^{\text{th}}$ \emph{division polynomial}
\end{Definition}\pause
\begin{theorem}[$E: Y^2=X^3+AX+B$ elliptic curve, $P=(x,y)\in E$]
\centerline{\begin{beamercolorbox}[rounded=true,shadow=true,wd=9.2cm,center]{formul}
$\!\!\!mP=m(x,y)=\left ( \frac{\phi_{m}(x)}{\psi_{m}^{2}(x)}, \frac{\omega_{m}(x,y)}{\psi^{3}_{m}(x,y)} \right),$\\ where $\phi_{m}=x\psi_{m}^{2} - \psi_{m+1}\psi_{m-1},\omega_{m}=\frac{\psi_{m+2}\psi_{m-1}^{2}-\psi_{m-2}\psi_{m+1}^{2}}{4y}
\!\!\!$\end{beamercolorbox}}
\end{theorem}
\end{frame}

\subsection{Points of finite order}

\begin{frame}\frametitle{Points of order $m$}
\begin{definition}[$m$--torsion point] Let $E/K$ and let $\bar{K}$ an \emph{algebraic closure of $K$}.
\centerline{\begin{beamercolorbox}[rounded=true,shadow=true,wd=5cm,center]{postit}
$E[m]=\{P\in E(\bar{K}):\ mP=\infty\}$\end{beamercolorbox}}
\end{definition}\pause

\begin{theorem}[Structure of Torsion Points]
Let $E/K$  and $m\in\N$. 
$$E[m]\cong \begin{cases}
             C_m\oplus C_m & \text{if }p=\operatorname{char}(K)\nmid m\\
             C_m\oplus C_{m'}\quad\text{or}\quad E[m] \cong C_{m'}\oplus C_{m'} &\text{if }m=p^rm', p\nmid m'
            \end{cases}
$$\end{theorem}\pause

% \begin{beamerboxesrounded}[upper=block title example,lower=block body alerted,shadow=true]{Idea of the proof:}
% Let $[m]:E\rightarrow E, P\mapsto mP$. Then
% \alert{$$\#E[m]=\#\operatorname{Ker[m]}\le\partial\phi_m=m^2$$}
%  \hfil equality holds iff \alert{$p\nmid m$}.
% \end{beamerboxesrounded}

\begin{beamerboxesrounded}[upper=block title example,lower=block body alerted,shadow=true]{FACTS:}
\begin{itemize}[<+-| alert@+>]
\item $E[2m+1]\setminus \{\infty\}= \{(x,y)\in E(\bar{K}):\  \psi_{2m+1}(x)=0\}$
\item $E[2m]\setminus E[2]= \{(x,y)\in E(\bar{K}):\  y^{-1}\psi_{2m}(x)=0\}$
\item \alert{Corollary of the Theorem of Structure for torsion} $\exists n,k\in\mathbb N$ such that
$E(\F_q)\cong C_n\oplus C_{nk}$
\item \alert{Property of Weil pairing} $n\mid q-1$.
\end{itemize}
\end{beamerboxesrounded}



%\vspace*{-2mm}
% \end{frame}
% 
% \begin{frame}
% 
% \pause
% 
% \begin{example}%\vspace*{-.7cm}
%  \begin{small}
%  \begin{align*}
% \psi_4(x)=&2y(x^6
%  + 5 A x^4
%  + 20 B x^3
%  - 5 A^2 x^2
%  - 4 B A x
%  -A^3
%  - 8 B^2)\\
%  \\
%  \psi_5(x)=&5 x^{12}
%  + 62 A x^{10}
%  + 380 B x^9
%  - 105 A^2 x^8
%  + 240 B A x^7
%  + \left(-300 A^3
%  - 240 B^2\right)  x^6
%  - 696 B A^2 x^5
%  + \left(-125 A^4
%  - 1920 B^2 A\right)  x^4\\&
%  + \left(-80 B A^3
%  - 1600 B^3\right)  x^3
%  + \left(-50 A^5
%  - 240 B^2 A^2\right)  x^2
%  + \left(-100 B A^4
%  - 640 B^3 A\right)  x
%  + \left(A^6
%  - 32 B^2 A^3
%  - 256 B^4\right)\\
% \\
%  \psi_6(x)=&2y(
%  6 x^{16}
%  + 144 A x^{14}
%  + 1344 B x^{13}
%  - 728 A^2 x^{12}
%  + \left(-2576 A^3
%  - 5376 B^2\right)  x^{10}
%  - 9152 B A^2 x^9
%  + \left(-1884 A^4
%  - 39744 B^2 A\right)  x^8\\&
%  + \left(1536 B A^3
%  - 44544 B^3\right)  x^7
%  + \left(-2576 A^5
%  - 5376 B^2 A^2\right)  x^6
%  + \left(-6720 B A^4
%  - 32256 B^3 A\right)  x^5\\&
%  + \left(-728 A^6
%  - 8064 B^2 A^3
%  - 10752 B^4\right)  x^4
%  + \left(-3584 B A^5
%  - 25088 B^3 A^2\right)  x^3
%  + \left(144 A^7
%  - 3072 B^2 A^4
%  - 27648 B^4 A\right)  x^2\\&
%  + \left(192 B A^6
%  - 512 B^3 A^3
%  - 12288 B^5\right)  x
%  + \left(6 A^8
%  + 192 B^2 A^5
%  + 1024 B^4 A^2\right))
%   \end{align*}
%  \end{small}%\vspace*{-7mm}
% \end{example}
% 
% \end{frame}
% 
% \subsection{The group structure}
% \begin{frame}\frametitle{Group Structure of $E(\F_q)$}
% 
% \begin{exercise} Use division polynomials in Sage to write a list of all curves $E$ over $\F_{103}$
% such that $E(\F_{103})\supset E[6]$. Do the same for curves over $\F_{5^4}$.
% \end{exercise}\pause


\end{frame}

\section{Important Results}
\subsection{Hasse's Theorem}
\begin{frame}
\begin{theorem}[Hasse]
Let $E$ be an elliptic curve over the finite field $\F_q$. Then the order of $E(\F_q)$
satisfies
$$\left|q+1-\#E(\F_q)\right|\le 2\sqrt q.$$
\end{theorem}\pause

So \alert{$\#E(\F_q)\in [(\sqrt q -1)^2, (\sqrt q+1)^2]$} the \emph{Hasse interval} ${\mathcal I}_q$


 \begin{example}[Hasse Intervals]
\begin{scriptsize}
 \centerline{\begin{tabular}{|l|l|}
\hline
 $q$ & ${\mathcal I}_q$\\
\hline
$2$ & $\{1, 2, 3, 4, 5\}$\\
$3$ & $\{1, 2, 3, 4, 5, 6, 7\}$\\
$4$ & $\{1, 2, 3, 4, 5, 6, 7, 8, 9 \}$\\
$5$ & $\{2, 3, 4, 5, 6, 7, 8, 9, 10\}$\\
$7$ & $\{3, 4, 5, 6, 7, 8, 9, 10, 11, 12, 13\}$\\
$8$ & $\{4, 5, 6, 7, 8, 9, 10, 11, 12, 13, 14\}$\\
$9$ & $\{4, 5, 6, 7, 8, 9, 10, 11, 12, 13, 14, 15, 16\}$\\
$11$ & $\{6, 7, 8, 9, 10, 11, 12, 13, 14, 15, 16, 17, 18\}$\\
$13$ & $\{7, 8, 9, 10, 11, 12, 13, 14, 15, 16, 17, 18, 19, 20, 21\}$\\
$16$ & $\{9, 10, 11, 12, 13, 14, 15, 16, 17, 18, 19, 20, 21, 22, 23, 25\}$\\
$17$ & $\{10, 11, 12, 13, 14, 15, 16, 17, 18, 19, 20, 21, 22, 23, 24, 25, 26\}$\\
$19$ & $\{12, 13, 14, 15, 16, 17, 18, 19, 20, 21, 22, 23, 24, 25, 26, 27, 28\}$\\
$23$ & $\{15, 16, 17, 18, 19, 20, 21, 22, 23, 24, 25, 26, 27, 28, 29, 30, 31, 32,
 33\}$\\
$25$ & $\{16, 17, 18, 19, 20, 21, 22, 23, 24, 25, 26, 27, 28, 29, 30, 31, 32, 33,
 34, 35, 36\}$\\
$27$ & $\{18, 19, 20, 21, 22, 23, 24, 25, 26, 27, 28, 29, 30, 31, 32, 33, 34, 35,
 36, 37, 38\}$\\
$29$ & $\{20, 21, 22, 23, 24, 25, 26, 27, 28, 29, 30, 31, 32, 33, 34, 35, 36, 37,
 38, 39, 40\}$\\
$31$ & $\{21, 22, 23, 24, 25, 26, 27, 28, 29, 30, 31, 32, 33, 34, 35, 36, 37, 38,
 39, 40, 41, 42, 43 \}$\\
$32$ & $\{22, 23, 24, 25, 26, 27, 28, 29, 30, 31, 32, 33, 34, 35, 36, 37, 38, 39,
 40, 41, 42, 43, 44\}$\\  \hline
\end{tabular}}
\end{scriptsize}
\end{example}



\end{frame}

\subsection{Waterhouse's Theorem}
\begin{frame}[label=current]
\begin{theorem}[Waterhouse]\pause
\label{lem:Water}
 Let $q=p^n$ and let $N = q + 1-a$.\\
\centerline{$\exists E/\F_q\text{ s.t.}\#E(\F_q) = N\Leftrightarrow|a|\le 2\sqrt q\text{ and}$}
 one of the following is satisfied:\pause
\begin{itemize}[<+-| alert@+>]
\item[(i)] $\gcd(a, p) = 1$;
\item[(ii)] $n$ even and one of the following is satisfied:
\begin{enumerate}
  \item $a=\pm 2\sqrt q$;
  \item $p\not\equiv 1 \pmod 3$, and $a = \pm\sqrt q$;
  \item $p\not\equiv 1 \pmod 4$, and $a = 0$;
\end{enumerate}
\item[(iii)] $n$ is odd, and one of the following is satisfied:
 \begin{enumerate}
   \item $p = 2$ or $3$, and $a = \pm p^{(n+1)/2}$;
   \item $a = 0$.
 \end{enumerate}
 \end{itemize}
\end{theorem}

%\setbeamercovered{transparent}

\begin{example}[$q$ prime $\forall N\in I_q,\exists E/\F_q, \#E(\F_q)=N.$ $q$ not prime:]
\begin{small}
\centerline{\begin{tabular}{|l|l|}
\hline
 $q$ & $a\in$\\
\hline%\vspace*{-3.12pt}
\!\!$4=2^2$\!\! &\!\!\!\! $\{{\color<5->{green}-4},{\color<3->{green}-3},{\color<6->{green}-2},{\color<3->{green}-1},{\color<7->{green}0},{\color<3->{green}1},{\color<6->{green}2}, {\color<3->{green}3}, {\color<5->{green}4}\}$\\
\!\!$8=2^3$\!\! &\!\!\!\! $\{{\color<3->{green}-5},{\color<9->{green}-4},{\color<3->{green}-3},-2,{\color<3->{green}-1},{\color<10->{green}0},{\color<3->{green}1},2,{\color<3->{green}3}, {\color<9->{green}4},{\color<3->{green}5}\}$\\
\!\!$9=3^2$\!\! &\!\!\!\! $\{{\color<5->{green}-6},{\color<3->{green}-5},{\color<3->{green}-4},{\color<6->{green}-3},{\color<3->{green}-2},{\color<3->{green}-1},{\color<7->{green}0},{\color<3->{green}1},{\color<3->{green}2}, {\color<6->{green}3},{\color<3->{green}4},{\color<3->{green}5},{\color<5->{green}6}\}$\\
\!\!$16=2^4$\!\! &\!\!\!\! $\{{\color<5->{green}-8},{\color<3->{green}-7},-6,{\color<3->{green}-5},{\color<6->{green}-4},{\color<3->{green}-3},-2,{\color<3->{green}-1},{\color<7->{green}0},{\color<3->{green}1},2,{\color<3->{green}3}, {\color<6->{green}4},{\color<3->{green}5}, 6,{\color<3->{green}7},{\color<5->{green}8}\}$\\
\!\!$25=5^2$\!\! &\!\!\!\! $\{{\color<5->{green}-10},{\color<3->{green}-9},{\color<3->{green}-8},{\color<3->{green}-7},{\color<3->{green}-6},{\color<6->{green}-5},{\color<3->{green}-4},{\color<3->{green}-3},{\color<3->{green}-2},{\color<3->{green}-1},0,{\color<3->{green}1},{\color<3->{green}2}, {\color<3->{green}3}, {\color<3->{green}4},{\color<6->{green}5},{\color<3->{green}6},{\color<3->{green}7}, {\color<3->{green}8},{\color<3->{green}9}, {\color<3->{green}10}\}$\\
\!\!$27=3^3$\!\! &\!\!\!\! $\{{\color<3->{green}-10},{\color<9->{green}-9},{\color<3->{green}-8},{\color<3->{green}-7},-6,{\color<3->{green}-5},{\color<3->{green}-4},-3,{\color<3->{green}-2},{\color<3->{green}-1},{\color<10->{green}0},{\color<3->{green}1},{\color<3->{green}2}, 3, {\color<3->{green}4},{\color<3->{green}5},6,{\color<3->{green}7},{\color<3->{green}8},{\color<9->{green}9},{\color<3->{green} 10}\}$\!\!\!\!\\
\!\!$32=2^5$\!\!&\!\!\!\! $\{{\color<3->{green}-11},-10,{\color<3->{green}-9},{\color<9->{green}-8},{\color<3->{green}-7},-6,{\color<3->{green}-5},-4,{\color<3->{green}-3},-2,{\color<3->{green}-1},{\color<10->{green}0},{\color<3->{green}1},2, {\color<3->{green}3}, 4,{\color<3->{green}5}, 6, {\color<3->{green}7}, {\color<9->{green}8}, {\color<3->{green}9},10,{\color<3->{green}11}\}$\!\!\!\!\\  \hline
\end{tabular}}\end{small}
\end{example}
\end{frame}

\subsection{R\"uck's Theorem}
\begin{frame}
\begin{theorem}[R\"uck]
Suppose $N$ is a possible order of an elliptic curve $/\F_q$,  $q=p^n$.  Write

\centerline{
$N = p^e n_1 n_2,\quad p\nmid n_1 n_2\quad\text{and}\quad n_1\mid n_2\ (\text{possibly }n_1 = 1).$}

There exists $E/\F_q$ s.t.
$$E(\F_q)\cong C_{n_1}\oplus C_{n_2p^e}$$
if and only if
\begin{enumerate}[<+-| alert@+>]
\item $n_1 = n_2$ in the case~(ii).1 of Waterhouse's Theorem;
\item $n_1 |q - 1$ in all other cases of  Waterhouse's Theorem.
\end{enumerate}
\end{theorem}\pause

\begin{example}
\begin{itemize}[<+-| alert@+>]
\item If $q=p^{2n}$ and $\#E(\F_q)=q+1\pm2\sqrt{q}=(p^n\pm1)^2$, then

\alert{\centerline{$E(\F_q)\cong C_{p^n\pm1}\oplus C_{p^n\pm1}.$}}
\item Let $N=100$ and $q=101\ \Rightarrow\ \exists E_1, E_2, E_3, E_4/\F_{101}$ s.t.

\alert{\centerline{$E_1(\F_{101})\cong C_{10}\oplus C_{10}\qquad E_2(\F_{101})\cong C_{2}\oplus C_{50}$}}

\alert{\centerline{$E_3(\F_{101})\cong C_{5}\oplus C_{20}\qquad E_4(\F_{101})\cong C_{100}$}}

\end{itemize}
\end{example}
\end{frame}



% \section{Endomorphisms}
% 
% \begin{frame}
% \frametitle{Endomorphisms}
% 
% \begin{definition} A map \alert{$\alpha: E(\bar{K})\rightarrow E(\bar{K})$} is called
% an \alert{endomorphism} if\pause
% \begin{itemize}[<+-| alert@+>]
%   \item $\alpha(P+_EQ)=\alpha(P)+_E\alpha(Q)$ ($\alpha$ is a group homomorphism)
%   \item $\exists R_1,R_2\in \bar{K}(x,y)$ s.t. $\alpha(x,y)=(R_1(x,y),R_2(x,y))\qquad\forall (x,y)\not\in\operatorname{Ker}(\alpha) $
% \end{itemize}\pause
% ($\bar{K}(x,y)$ is the field of \emph{rational functions}, \pause  $\alpha(\infty)=\infty$
% )\vspace*{-1pt}
% \end{definition}\vspace*{-3.5pt}\pause
% 
% \begin{exercise}[Show that we can always assume]
% \centerline{\begin{beamercolorbox}[shadow=true,left,rounded=true,wd=7cm]{postit}
% $\alpha(x,y)=(r_1(x),yr_2(x)),\qquad \exists r_1,r_2\in\bar{K}(x)$
% \end{beamercolorbox}}
% 
% \hfil\ \textbf{Hint:} use $y^2=x^3+Ax+B$ and $\alpha(-(x,y))=-\alpha(x,y)$,
% \end{exercise}\vspace*{-1pt}\pause
% 
% \begin{beamerboxesrounded}[upper=block title example,lower=block body alerted,shadow=true]{Remarks/Examples:}
% \begin{itemize}[<+-| alert@+>]
% \item if $r_1(x)=p(x)/q(x)$ with $\gcd(p,q)=1$ and $(x_0,y_0)\in E(\bar{K})$ with $q(x_0)=0$ $\Rightarrow$ $\alpha(x_0,y_0)=\infty$
% \item $[m](x,y)=\left(\frac{\phi_m}{\psi_m^2},\frac{\omega_m}{\psi_m^3}\right)$ is an
% endomorphism $\forall m\in\Z$
% \item $\Phi_q:E(\bar{\F}_q))\rightarrow E(\bar{\F}_q)), (x,y)\mapsto(x^q,y^q)$ is called
% \emph{Frobenius Endomorphism}
% \end{itemize}
% \end{beamerboxesrounded}
% \end{frame}
% 
% \begin{frame}
% \frametitle{Endomorphisms (continues)}
% 
% \begin{theorem} If $\alpha\neq[0]$ is an endomorphism, then it is surjective.
% \end{theorem}\pause
% 
% \begin{proof}[Sketch of the proof] Assume \alert{$p>3$}, \alert{$\alpha(x,y)=(p(x)/q(x),yr_2(x)$} and \alert{$(a,b)\in E(\bar{K})$}.\medskip
% 
% \begin{itemize}
%   \item If \alert{$p(x)-aq(x)$} is not constant, let $x_0$ be one of its roots.
% Choose $y_0$ a square root of $x_0^2+AX_0+B$.\medskip
% 
% Then either \alert{$\alpha(x_0,y_0)=(a,b)$} or \alert{$\alpha(x_0,-y_0)=(a,b)$}.\medskip
% 
%   \item If \alert{$p(x)-aq(x)$} is  constant,\\
% \hfill   this happens only for one value of $a$!
% \begin{itemize}
% \item[] Let \alert{$(a_1,b_1)\in E(\bar{K})$}:\\
% \alert{$(a_1,b_1)\neq (a,\pm b)$} and \alert{$(a_1,b_1)+_E(a, b)\neq (a,\pm b).$}\medskip
% 
% \item[] Then \alert{$(a_1,b_1)=\alpha(P_1)$} and \alert{$(a_1,b_1)+_E(a,b)=\alpha(P_2)$}\medskip
% 
% \item[] Finally \alert{$(a,b)=\alpha(P_2-P_1)$}
% \end{itemize}
% \end{itemize}
% \end{proof}
% \end{frame}
% 
% 
% \begin{frame}
% \frametitle{Endomorphisms (continues)}
% 
% \begin{definition} Suppose $\alpha: E\rightarrow E, (x,y)=(r_1(x),yr_2(x))$ endomorphism. Write
% $r_1(x)=p(x)/q(x)$ with $\gcd(p(x),q(x))=1$.
% \begin{itemize}[<+-| alert@+>]
%   \item The \textbf{degree} of $\alpha$ is $\deg\alpha:=\max\{\deg p,\deg q\}$
%   \item $\alpha$ is said \textbf{separable} if $(p'(x),q'(x))\neq(0,0)$ \hfill (identically)
% \end{itemize}
% \end{definition}\pause
% 
% \begin{lemma}
% \begin{itemize}[<+-|alert@+>]
% \item $\Phi_q(x,y)=(x^q,y^q)$ is a non separable endomorphism of degree $q$
% \item $[m](x,y)=\left(\frac{\phi_m}{\psi_m^2},\frac{\omega_m}{\psi^3_m}\right)$ has degree $m^2$
% \item $[m]$ separable iff $p\nmid m$.
% \end{itemize}
% \end{lemma}\pause
% 
% \begin{proof}
% \alert{\emph{First:}} Use the fact that $x\mapsto x^q$ is the identity on $\F_q$ hence it
% fixes the coefficients of the Weierstra\ss\ equation.\pause \alert{\emph{Second:}} already done.
% \pause \alert{\emph{Third}} See \cite[Proposition 2.28]{washington}
% \end{proof}
% \end{frame}
% 
% \begin{frame}
% \frametitle{Endomorphisms (continues)}
% 
% \begin{theorem}
% Let $\alpha\neq0$ be an endomorphism. Then
% \centerline{\begin{beamercolorbox}[shadow=true,left,rounded=true,wd=7cm]{postit}
% $\#\operatorname{Ker}(\alpha)\begin{cases}=\deg\alpha&\text{if }\alpha\text{ is separable}\\
%                                         <\deg\alpha&\text{otherwise}\end{cases}$
%                                         \end{beamercolorbox}}
% \end{theorem}\pause
% 
% \begin{proof}
% It is same proof as $\#E[m]=\#\operatorname{Ker}[m]\le\partial\phi_m= m^2$\pause \\ \ \hfill (equality for $p\nmid m$)
% \end{proof}\pause
% 
% \begin{Definition} Let $E/K$. The \emph{ring of endomorphisms}
% \alert{$$\operatorname{End}(E):=\{\alpha: E\rightarrow E, \alpha\text{ is an endomorphism}\}.$$}
% where for all $\alpha_1,\alpha_2\in\operatorname{End}(E)$,\pause
% \begin{itemize}[<+-|alert@+>]
%   \item $(\alpha_1+\alpha_2)P:=\alpha_1(P)+_E\alpha_2(P)$
%   \item $(\alpha_1\alpha_2)P=\alpha_1(\alpha_2(P))$
% \end{itemize}
% \end{Definition}
% \end{frame}
% 
% \subsection{Separability}
% \begin{frame}
% \frametitle{Endomorphisms (continues)}
% 
% \begin{block}{Properties of $\operatorname{End}(E)$:}
% \begin{itemize}[<+-|alert@+>]
%   \item \alert{$[0]:P\mapsto\infty$} is the zero element
%   \item \alert{$[1]:P\mapsto P$} is the identity element
%   \item \alert{$\Z\hookrightarrow\operatorname{End}(E)$}, $m\mapsto [m]$
%   \item \alert{$\operatorname{End}(E)$} is not necessarily commutative
%   \item if $K=\F_q$, \alert{$\Phi_q\in\operatorname{End}(E)$}. So \alert{$\Z[\Phi_q]\subset\operatorname{End}(E)$}
% \end{itemize}\pause
% \end{block}
% 
% Recall that $\alert{\alpha\in \operatorname{End}(E)}$ is said \textbf{separable}
%  if $(p'(x),q'(x))\neq(0,0)$ where $\alpha(x,y)=(p(x)/q(x),yr(x))$.\pause
% 
% \begin{lemma} Let \alert{$\Phi_q: (x,y)\mapsto(x^q,y^q)$} be the Frobenius endomorphism and
% let $r,s\in\Z$. Then
% \centerline{\begin{beamercolorbox}[shadow=true,left,rounded=true,center,wd=8cm]{postit}
% $r\Phi_q+s\in \operatorname{End}(E)$ is separable $\ \Leftrightarrow\ p\nmid s$
% \end{beamercolorbox}}
% \end{lemma}\pause
% 
% \begin{proof}
% See \cite[Proposition 2.29]{washington}
% \end{proof}
% 
% \end{frame}
% 
% \subsection{the degree of endomorphism}
% \begin{frame}
% 
% Recall that the \textbf{degree} if $\alpha$ is \alert{$\deg\alpha:=\max\{\deg p,\deg q\}$} where \alert{$\alpha(x,y)=(p(x)/q(x),yr(x))$}.\pause
% 
% \begin{lemma} $\forall r,s\in\Z$ and $\forall \alpha,\beta\in\operatorname{End}(E)$,\\
% \alert{\centerline{\small{$\!\!\deg(r\alpha+s\beta)=r^2\deg\alpha+s^2\deg\beta+rs(\deg(\alpha+\beta)-\deg\alpha-\deg\beta)$}}}
% \end{lemma}\pause
% 
% \begin{proof} Let $m\in\N$ with $p\nmid m$ and fix a basis $P, Q$ of $E[m]\cong C_m\oplus C_m$.\pause
% 
% Then $\alpha(P)=aP+bQ$ and $\alpha(Q)=cP+dQ$ with \\
% \centerline{\begin{beamercolorbox}[shadow=true,left,rounded=true,center,wd=\textwidth]{postit}
% $\alpha_m=\begin{pmatrix}a&b\\c&d\end{pmatrix}$ with entries in $\Z/m\Z$.
% \end{beamercolorbox}}\pause
% 
% We claim that \alert{$\deg(\alpha)\equiv\det\alpha_m\bmod m$}. In fact if $\zeta=e_m(P,Q)$
% is the Weil pairing (primitive root).\pause\\
% \centerline{\alert{$\zeta^{\deg(\alpha)}=e_m(\alpha(P),\alpha(Q))=
% e_m(aP+bQ,cP+dQ)=\zeta^{ad-bc}$}}\pause
% So
% \begin{beamercolorbox}[shadow=true,left,rounded=true,center,wd=6cm]{postit}
% $\deg(\alpha)\equiv ad-bc=\det\alpha_m(\bmod m)$.\end{beamercolorbox}\pause A calculation shows\\
% \centerline{\begin{scriptsize}\alert{
% $\det(r\alpha_m+s\beta_m)= r^2\det\alpha_m+s^2\det\beta_m+rs\det(\alpha_m+\beta_m)-\det\alpha_m-\det\beta_m)$}
% \end{scriptsize}}\pause
% \centerline{\begin{scriptsize}
% So\hfill \alert{$\deg(r\alpha+s\beta)\equiv r^2\deg\alpha+s^2\deg\beta+rs\deg(\alpha+\beta)-\deg\alpha-\deg\beta\bmod m$}
% \end{scriptsize}}\pause
% Since it holds for $\infty$--many $m$'s the above is an equality.\vspace{-1.5pt}
% \end{proof}
% \end{frame}
% 
% 
% \section{Hasse's Theorem}
% \begin{frame}
% \begin{theorem}[Hasse]
% Let $E$ be an elliptic curve over the finite field $\F_q$. Then the order of $E(\F_q)$
% satisfies
% $$\left|q+1-\#E(\F_q)\right|\le 2\sqrt q.$$
% \end{theorem}\pause
% 
% So \alert{$\#E(\F_q)\in [(\sqrt q -1)^2, (\sqrt q+1)^2]$} the \emph{Hasse interval} ${\mathcal I}_q$
% 
% 
%  \begin{example}[Hasse Intervals]
% \begin{small}\centerline{\begin{tabular}{|l|l|}
% \hline
%  $q$ & ${\mathcal I}_q$\\
% \hline
% $2$ & $\{1, 2, 3, 4, 5\}$\\
% $3$ & $\{1, 2, 3, 4, 5, 6, 7\}$\\
% $4$ & $\{1, 2, 3, 4, 5, 6, 7, 8, 9 \}$\\
% $5$ & $\{2, 3, 4, 5, 6, 7, 8, 9, 10\}$\\
% $7$ & $\{3, 4, 5, 6, 7, 8, 9, 10, 11, 12, 13\}$\\
% $8$ & $\{4, 5, 6, \alert{7}, 8, 9, 10, \alert{11}, 12, 13, 14\}$\\
% $9$ & $\{4, 5, 6, 7, 8, 9, 10, 11, 12, 13, 14, 15, 16\}$\\
% $11$ & $\{6, 7, 8, 9, 10, 11, 12, 13, 14, 15, 16, 17, 18\}$\\
% $13$ & $\{7, 8, 9, 10, 11, 12, 13, 14, 15, 16, 17, 18, 19, 20, 21\}$\\
% $16$ & $\{9, 10, \alert{11}, 12, 13, 14, \alert{15}, 16, 17, 18, \alert{19}, 20, 21, \alert{22}, 23, 25\}$\\
% $17$ & $\{10, 11, 12, 13, 14, 15, 16, 17, 18, 19, 20, 21, 22, 23, 24, 25, 26\}$\\
% $19$ & $\{12, 13, 14, 15, 16, 17, 18, 19, 20, 21, 22, 23, 24, 25, 26, 27, 28\}$\\
% $23$ & $\{15, 16, 17, 18, 19, 20, 21, 22, 23, 24, 25, 26, 27, 28, 29, 30, 31, 32,
%  33\}$\\
% $25$ & $\{16, 17, 18, 19, 20, 21, 22, 23, 24, 25, \alert{26}, 27, 28, 29, 30, 31, 32, 33,
%  34, 35, 36\}$\\
% $27$ & $\{18, 19, 20, 21, \alert{22}, 23, 24, \alert{25}, 26, 27, 28, 29, 30, \alert{31}, 32, 33, \alert{34}, 35,
%  36, 37, 38\}$\\
% $29$ & $\{20, 21, 22, 23, 24, 25, 26, 27, 28, 29, 30, 31, 32, 33, 34, 35, 36, 37,
%  38, 39, 40\}$\\
% $31$ & $\{21, 22, 23, 24, 25, 26, 27, 28, 29, 30, 31, 32, 33, 34, 35, 36, 37, 38,
%  39, 40, 41, 42, 43 \}$\\
% $32$ & $\{22, \alert{23}, 24, 25, 26, \alert{27}, 28, \alert{29}, 30, \alert{31}, 32, 33, 34, \alert{35}, 36, \alert{37}, 38, \alert{39},
%  40, 41, 42, \alert{43}, 44\}$\\  \hline
% \end{tabular}}\end{small}
% \end{example}
% \end{frame}
% 
% \subsection{Frobenius endomorphism}
% \begin{frame}
% \frametitle{The Frobenius endomorphism $\Phi_q$}\pause
% 
% \centerline{\begin{beamercolorbox}[shadow=true,left,rounded=true,center,wd=8cm]{postit}
% $\Phi_q:\bar{\F}_q\rightarrow\bar{\F}_q, x\mapsto x^q$ is a field automorphism
% \end{beamercolorbox}}\pause
% 
% Given $\alpha\in\bar{\F}_q$,
% \centerline{\begin{beamercolorbox}[shadow=true,left,rounded=true,center,wd=6cm]{formul}
% $\alpha\in\F_{q^n}\ \Leftrightarrow\ \Phi_q^n(\alpha)=\alpha^{q^n}=\alpha$\pause
% \end{beamercolorbox}}\pause
% 
% Fixed points of powers of $\Phi_q$ are exactly elements of $\F_{q^n}$\pause
% 
% \centerline{\begin{beamercolorbox}[shadow=true,left,rounded=true,center,wd=8cm]{postit}
% $\Phi_q:E(\bar{\F}_q)\rightarrow E(\bar{\F}_q), (x,y)\mapsto(x^q,y^q),\infty\mapsto\infty$\end{beamercolorbox}}\pause
% 
% \begin{block}{Properties of $\Phi_q$}
% \begin{itemize}[<+-|alert@+>]
% \item $\Phi_q\in \operatorname{End}(E)$, it is not separable and has degree $q$
% \item $\Phi_q(x,y)=(x,y)\ \Longleftrightarrow\ (x,y)\in E(\F_q)$
% \item $\operatorname{Ker}(\Phi_q-1)=E(\F_q)$
% \item $\#\operatorname{Ker}(\Phi_q-1)=\deg(\Phi_q-1)$ (since $\Phi_q-1$ is separable)
% \item if we can compute $\deg(\Phi_q-1)$, we can compute $\#E(\F_q)$
% \item $\Phi_{q}^n(x,y)=(x^{q^n},y^{q^n})$ so  \alert{$\Phi_{q}^n(x,y)=(x,y)\Leftrightarrow(x,y)\in\F_{q^n}$}
% \item $\operatorname{Ker}(\Phi_q^n-1)=E(\F_{q^n})$
% \end{itemize}
% \end{block}
% \end{frame}
% 
% \subsection{proof}
% \begin{frame}
% \frametitle{Proof of Hasse's Theorem}
% 
% \begin{lemma} Let $E/\F_q$ and write
% $a=q+1-\#E(\F_q)=q+1-\deg(\Phi_q-1).$ Then $\forall r,s\in\Z$, $\gcd(q,s)=1$,\pause
% 
% \centerline{\begin{beamercolorbox}[rounded=true,shadow=true,wd=6cm,center]{formul}
% $\deg(r\phi+s)=r^2q+s^2-rsa$
% \end{beamercolorbox}}
% \end{lemma}\pause
% 
% \begin{proof}{Proof of the Lemma} From a previous proposition, we know that
% \alert{\scriptsize $\deg(r\Phi_q+s)=r^2\deg(\Phi_q)+s^2\deg([-1])-rs(\deg(\Phi_q-1)-\deg(\Phi_q)-\deg([-1]))$}\pause\\
% But\\
% \centerline{$\deg(\Phi_q)=q$, $\deg([-1])=1$ and $\deg(\Phi_q-1)-q-1=-a$}
% \end{proof}\pause
% 
% 
% \begin{proof}[Proof of Hasse's Theorem]
% \alert{\centerline{$q\left(\frac rs\right)^2-a\left(\frac rs\right)+1=\frac{\deg(r\Phi_q+s)}{s^2}\ge0$}}
% 
% on a dense set of rational numbers.\\\pause
% This implies $\forall X\in\R$, $X^2-aX+q\ge0$.\pause So\\
% \alert{\centerline{$a^2-4q\le0\ \Leftrightarrow\ |a|\le2\sqrt{q}!!$}}\end{proof}
% 
% \end{frame}
% 
% \begin{frame}
% \frametitle{Proof of Hasse's Theorem (continues)}
% 
% \begin{beamerboxesrounded}[upper=block title example,lower=block body alerted,shadow=true]
% {Ingredients for the proof:}\pause
% \begin{enumerate}[<+-|alert@+>]
%              \item $E(\F_q)=\operatorname{Ker}(\Phi_q-1)$
%              \item $\Phi_q-1$ is separable
%              \item $\#\operatorname{Ker}(\Phi_q-1)=\deg(\Phi_q-1)$
%            \end{enumerate}
% \end{beamerboxesrounded}\pause
% 
% \begin{corollary} Let $a=q+1-\#E(\F_q)$. Then
% \begin{enumerate}[<+-|alert@+>]
%              \item\
% {\begin{beamercolorbox}[vmode,rounded=true,shadow=true,wd=3.2cm,center]{formul}
% $\Phi_q^2-a\Phi_q+q=0$\end{beamercolorbox}}\\
% \hfill is an identity of endomorphisms.
% \item
% 
% $a\in\Z$ is the unique integer $k$ such that $\Phi_q^2-k\Phi_q+q=0$
% \item\
% {\begin{beamercolorbox}[rounded=true,shadow=true,wd=7cm,center]{formul}
% $a\equiv\operatorname{Tr}((\Phi_q)_m)\bmod m\ \forall m\text{ s.t. }\gcd(m,q)=1$\end{beamercolorbox}}
% \end{enumerate}
% \end{corollary}
% \end{frame}
% 
% \begin{frame}
% \begin{proof}[Sketch of the Proof of Corollary]
% Let $m\in\N$ s.t. $\gcd(m,q)=1$. Choose a basis for $E[m]$ and write
% \alert{$$(\Phi_q)_m=\begin{pmatrix}s&t\\u&v\end{pmatrix}$$}\pause
% $\Phi_q-1$ separable implies
% \alert{
% \begin{align*}
% \#\operatorname{Ker}(\Phi_q-1)&=\deg(\Phi_q-1)\equiv\det((\Phi_q)_m-I))\\
%                                &=\det((\Phi_q)_m)-\operatorname{Tr}((\Phi_q)_m)+1 (\bmod m).
% \end{align*}
% }\pause
% Hence
% \alert{$$\operatorname{Tr}((\Phi_q)_m)\equiv a(\bmod m)$$}\pause
% By Cayley--Hamilton
% \alert{$$(\Phi_q)_m^2-a(\Phi_q)_m+qI\equiv0(\bmod m)$$}\pause
% Since this happens for infinitely many $m$'s,
% \alert{$$\Phi_q^2-a\Phi_q+q=0$$}as endomorphism.\pause \end{proof}
% \end{frame}

\begin{frame}\frametitle{Subfield curves}

\begin{definition}
Let $E/\F_q$ and write $E(\F_q)=q+1-a$, ($|a|\le2\sqrt{q}$). The \emph{characteristic}
polynomial of $E$ is
$$P_E(T)=T^2-aT+q\in\Z[T].$$
and its roots:
$$\alpha=\frac12\left(a+\sqrt{a^2-4q}\right)\qquad\beta=\frac12\left(a-\sqrt{a^2-4q}\right)$$
are called \emph{characteristic roots of Frobenius} ($P_E(\Phi_q)=0$).
\end{definition}

\begin{theorem} $\forall n\in\N$
\centerline{$\#E(\F_{q^n})=q^n+1-(\alpha^n+\beta^n).$}
\end{theorem}
\end{frame}

% \begin{frame}\frametitle{Subfield curves (continues)}
% \begin{theorem} $\forall n\in\N$
% $\#E(\F_{q^n})=q^n+1-(\alpha^n+\beta^n).$
% \end{theorem}
% 
% \begin{proof} Note that\pause
% \begin{enumerate}[<+-|alert@+>]
%   \item Result is true for $n=1$, {$\alpha+\beta=a$}
%   \item {$\alpha^n+\beta^n\in\Z$, $(\alpha\beta)^n=q^n$}
%   \item $f(X)=(X^n-\alpha^n)(X^n-\beta^n)=X^{2n}-(\alpha^n+\beta^n)X^n+q^n\in\Z[X]$
% 
%   \item $f(X)$ is divisible by $X^2-aX+q=(X-\alpha)(X-\beta)$
% 
%   \item $(\Phi_q)^n|_{\bar{\F}_{q^n}}=\Phi_{q^n}:(x,y)\mapsto(x^{q^n},y^{q^n})$
% 
%   \item $(\Phi_q^n)^2-(\alpha^n+\beta^n)\Phi_q^n+q^n=Q(\Phi_q))(\Phi_q^2-a\Phi_q+q)=0$
%   where $f(X)=Q(X)(X^2-aX+q)$
% \end{enumerate}\pause
% Hence $\Phi_q^n$ satisfies
% \alert{\centerline{$X^2-((\alpha^n+\beta^n))X+q.$}}\pause\\
% So\\
% \alert{\centerline{$\alpha^n+\beta^n=q^n+1-\#E(\F_{q^n}).$}}\pause
% Characteristic polynomial of $\Phi_{q^n}$: \pause
% \alert{ $X^2-(\alpha^n+\beta^n)X+q^n$}
% \end{proof}
%
%\end{frame}

\begin{frame}\frametitle{Subfield curves (continues)}

\alert{\centerline{$E(\F_{q})=q+1-a\ \Rightarrow\ E(\F_{q^n})=q^n+1-(\alpha^n+\beta^n)$}}

\hfill where $P_E(T)=T^2-aT+q=(T-\alpha)(T-\beta)\in\Z[T]$\pause
\begin{block}{Curves $/\F_2$}
\begin{center}
\begin{tabular}{|l|c|l|l|}
\hline
 $E$  & $a$ & $P_E(T)$ &$(\alpha,\beta)$\\
\hline
&&&\\
 $y^2+xy=x^3+x^2+1$ & $1$ & $T^2-T+2$& $\frac12(1\pm\sqrt{-7})$\\
&&&\\
$y^2+xy=x^3+1$  & $-1$ & $T^2+T+2$&$\frac12(-1\pm\sqrt{-7})$\\
&&&\\
$y^2+y=x^3+x$ &$-2$ & $T^2+2T+2$&$-1\pm i$\\
&&&\\
 $y^2+y=x^3+x+1$& $2$ &  $T^2-2T+2$&$1\pm i$\\
&&&\\
$y^2+y=x^3$  & $0$ & $T^2+2$ &$\pm\sqrt{-2}$\\
&&&\\\hline
\end{tabular}
\end{center}
\end{block}\pause


$$E:y^2+xy=x^3+x^2+1\ \Rightarrow\quad
E(\F_{2^{100}})=2^{100}+1-\left(\frac{1+\sqrt{-7}}2\right)^{100}-
\left(\frac{1-\sqrt{-7}}2\right)^{100} =
1267650600228229382588845215376$$
\end{frame}

\begin{frame}\frametitle{Subfield curves}
\alert{\centerline{$E(\F_{q})=q+1-a\ \Rightarrow\ E(\F_{q^n})=q^n+1-(\alpha^n+\beta^n)$}}

\hfill where $P_E(T)=T^2-aT+q=(T-\alpha)(T-\beta)\in\Z[T]$\pause
\begin{block}{Curves $/\F_3$}
\begin{center}
\begin{tabular}{|l|r|c|c|c|}
\hline
$i$ & $E_i$ & $a$ & $P_{E_i}(T)$ &$(\alpha,\beta)$\\
\hline
$1$& $y^2=x^3+x$ & $0$ & $T^2+3$ & $\pm\sqrt{-3}$\\
\hline
$2$&$y^2=x^3 - x$ & $0$ & $T^2+3$ & $\pm\sqrt{-3}$\\
\hline
$3$&$y^2=x^3 - x +1$& $-3$ & $T^2+3T+3$ & $\frac12(-3\pm\sqrt{-3})$\\
\hline
$4$&$y^2=x^3 - x -1$  &$3$ & $T^2-3T+3$ & $\frac12(3\pm\sqrt{-3})$\\
\hline
$5$&$y^2=x^3 + x^2 - 1$ & $1$ & $T^2-T+3$ & $\frac12(1\pm\sqrt{-11})$\\
\hline
$6$&$y^2=x^3 - x^2 + 1$ & $-1$ &$T^2+T+3$ & $\frac12(-1\pm\sqrt{-11})$\\
\hline
$7$&$y^2=x^3 + x^2 + 1$ & $-2$ & $T^2+2T+3$ & $-1\pm\sqrt{-2}$\\
\hline
$8$&$y^2=x^3 - x^2 - 1$ & $2$ &  $T^2-2T+3$ & $1\pm\sqrt{-2}$\\
\hline
\end{tabular}
\end{center}
\end{block}\pause


\begin{lemma} Let $s_n=\alpha^n+\beta^n$ where $\alpha\beta=q$ and $\alpha+\beta=a$. Then
$$s_0=2,\quad,s_1=a\quad\text{and}\quad s_{n+1}=as_n-qs_{n-1}$$
\end{lemma}
\end{frame}

\section{Legendre Symbols}
\begin{frame}
\frametitle{Legendre Symbols}

Recall the \emph{Finite field Legendre symbols}: let $x\in\F_q$,\pause

\centerline{\begin{beamercolorbox}[rounded=true,shadow=true,wd=8cm,center]{postit}
\alert{$\left(\frac{x}{\F_q}\right)=\begin{cases}
+1 &\text{ if }t^2=x\text{ has a solution }t\in\F_q^*\\
-1 &\text{ if }t^2=x\text{ has no solution }t\in\F_q\\
0 &\text{ if }x=0
\end{cases}$}\end{beamercolorbox}}\pause

\begin{theorem} Let $E:y^2=x^3+Ax+B$ over $\F_q$. Then
\centerline{\begin{beamercolorbox}[rounded=true,shadow=true,wd=6cm,center]{formul}
$\#E(\F_q)=q+1+\sum_{x\in\F_q}\left(\frac{x^3+Ax+B}{\F_q}\right)$\end{beamercolorbox}}
\end{theorem}\pause

\begin{proof} Note that
\alert{\centerline{$1+\left(\frac{x_0^3+Ax_0+B}{\F_q}\right)=\begin{cases}
2 &\text{if }\exists y_0\in\F_q^*\text{ s.t. }(x_0,\pm y_0)\in E(\F_q)\\
1 &\text{if }(x_0,0)\in E(\F_q)\\
0 &\text{otherwise}
\end{cases}$}}\pause

Hence
\centerline{{$\#E(\F_q)=1+\sum_{x\in\F_q}\left(1+\left(\frac{x^3+Ax+B}{\F_q}\right)\right)$}}\vspace*{-2.7pt}
\end{proof}
\end{frame}

\begin{frame}\frametitle{Last Slide}
\begin{corollary} Let
\alert{$E: y^2=x^3+Ax+B$} over $\F_q$ and
\alert{$E_\mu: y^2=x^3+\mu^2 Ax+\mu^3B$},
$\mu\in\F_q^*\setminus(\F_q^*)^2$ its \emph{twist}. Then
\centerline{\begin{beamercolorbox}[rounded=true,shadow=true,wd=8.5cm,center]{formul}
$\#E(\F_q)=q+1-a\ \Leftrightarrow\ \#E_\mu(\F_q)=q+1+a$\end{beamercolorbox}}
and
\centerline{\begin{beamercolorbox}[rounded=true,shadow=true,wd=5cm,center]{formul}
$\#E(\F_{q^2})=\#E_\mu(\F_{q^2}).$\end{beamercolorbox}}
\end{corollary}\pause

\begin{proof}
\alert{\begin{align*}
\#E_\mu(\F_q)&=q+1+\sum_{x\in\F_q}\left(\frac{x^3+\mu^2 Ax+\mu^3B}{\F_q}\right)\\ &=
q+1+\left(\frac{\mu}{\F_q}\right)\sum_{x\in\F_q}\left(\frac{x^3+Ax+B}{\F_q}\right)\end{align*}}
and $\left(\frac{\mu}{\F_q}\right)=-1$\vspace*{-2pt}
\end{proof}
\end{frame}

\section{Further reading}
\begin{frame}
\frametitle{Further Reading...}
\begin{scriptsize}
\begin{thebibliography}{99}
\bibitem{BSS} \textsc{Ian~F.~Blake,~Gadiel~Seroussi,~and~Nigel~P.~Smart},
Advances in elliptic curve cryptography, London Mathematical Society Lecture Note Series, vol. 317, Cambridge University Press, Cambridge, 2005.
 \bibitem{C} \textsc{J.~W.~S.~Cassels},
Lectures on elliptic curves, London Mathematical Society Student Texts, vol. 24, Cambridge University Press, Cambridge, 1991.
 \bibitem{CR} \textsc{John~E.~Cremona},
Algorithms for modular elliptic curves, 2nd ed., Cambridge University Press, Cambridge, 1997.
 \bibitem{Kn} \textsc{Anthony~W.~Knapp},
Elliptic curves, Mathematical Notes, vol. 40, Princeton University Press, Princeton, NJ, 1992.
 \bibitem{Ko} \textsc{Neal~Koblitz},
Introduction to elliptic curves and modular forms, Graduate Texts in Mathematics, vol. 97, Springer-Verlag, New York, 1984.
 %\bibitem{Po} \textsc{Poonen B} Elliptic curves (introduction)(19s) notes
 \bibitem{Sil} \textsc{Joseph~H.~Silverman},
The arithmetic of elliptic curves, Graduate Texts in Mathematics, vol. 106, Springer-Verlag, New York, 1986.
\bibitem{ST} \textsc{Joseph~H.~Silverman~and~John~Tate},
Rational points on elliptic curves, Undergraduate Texts in Mathematics, Springer-Verlag, New York, 1992.
\bibitem{washington} \textsc{Lawrence~C.~Washington},
Elliptic curves: Number theory and cryptography, 2nd ED. Discrete Mathematics and Its Applications, Chapman \& Hall/CRC, 2008.
\bibitem{Zimm} \textsc{Horst~G.~Zimmer},
Computational aspects of the theory of elliptic curves, Number theory and applications
(Banff, AB, 1988) NATO Adv. Sci. Inst. Ser. C Math. Phys. Sci., vol. 265, Kluwer Acad. Publ., Dordrecht, 1989, pp. 279--324.
\end{thebibliography}
\end{scriptsize}
\end{frame}

\end{document}


