\documentclass[10pt,handout]{beamer}% ,draft,final,,handout
\usepackage[english]{babel}
\usepackage{lmodern}
\usepackage[latin1]{inputenc}
\usepackage{times}
\usepackage{listings}
\usepackage{hyperref}
\usepackage[T1]{fontenc}
\usepackage{colortbl}
\let\Tiny=\tiny

 \newcommand{\Q}{\mathbb Q}
 \newcommand{\Z}{\mathbb Z}
 \newcommand{\N}{\mathbb N}
 \newcommand{\F}{\mathbb F}
 \newcommand{\C}{\mathbb C}
 \newcommand{\R}{\mathbb R}


\author[F.~Pappalardi]{Francesco Pappalardi}
\institute{Dipartimento di Matematica e Fisica\\
  Universit\`a Roma Tre}
\setbeamertemplate{title page}
{
  \vbox{}
  \vskip1em
  {\huge Lecture \insertshortlecture\par}
  {\usebeamercolor[fg]{title}\usebeamerfont{title}\inserttitle\par}%
  \ifx\insertsubtitle\@empty%
  \else%
    \vskip0.25em%
    {\usebeamerfont{subtitle}\usebeamercolor[fg]{subtitle}\insertsubtitle\par}%
  \fi%
  \vskip1em\par
   \textbf{College of Sciences\\ Department of Mathematics}\\ University of Salahaddin, \\
   \emph{Erbil, Kurdistan}   
   \insertdate\par
  \vskip0pt plus1filll
  \leftskip=0pt plus1fill\insertauthor\par
  \insertinstitute\vskip1em
}

\logo{\includegraphics[width=1cm]{images/roma3.pdf}}
\lecture[4]{Elliptic curves over finite fields}{First Steps}
\date{December 7$^{\textrm{th}}$, 2014}
\title[Elliptic curves over $\F_{q}$]{\insertlecture}
\subtitle{First steps}

% Beamer version theme settings

\useoutertheme[height=0pt,width=2cm,right]{sidebar}
\usecolortheme{rose,sidebartab}
\useinnertheme{circles}
\usefonttheme[only large]{structurebold}
\setbeamercolor{formul}{fg=black,bg=pink}
\setbeamercolor{sidebar right}{bg=black!15}
\setbeamercolor{structure}{fg=blue}
\setbeamercolor{author}{parent=structure}
 \setbeamercolor{postit}{fg=black,bg=yellow}
 \setbeamercolor{greys}{fg=black,bg==black!25}

\setbeamerfont{title}{series=\normalfont,size=\LARGE}
\setbeamerfont{title in sidebar}{series=\bfseries}
\setbeamerfont{author in sidebar}{series=\bfseries}
 \setbeamerfont*{item}{series=}
\setbeamerfont{frametitle}{size=}
\setbeamerfont{block title}{size=\small}
\setbeamerfont{subtitle}{size=\normalsize,series=\normalfont}

\setbeamertemplate{navigation symbols}{}
\setbeamertemplate{bibliography item}[book]
\setbeamertemplate{sidebar right}
{
  {\usebeamerfont{title in sidebar}%
    \vskip1.5em%
    \hskip3pt%
    \usebeamercolor[fg]{title in sidebar}%
    \insertshorttitle[width=2.1cm,respectlinebreaks]\par%   left,
    \vskip1.25em%
  }%
  {%
    \hskip3pt%
    \usebeamercolor[fg]{author in sidebar}%
    \usebeamerfont{author in sidebar}%
    \insertshortauthor[width=2cm,center,respectlinebreaks]\par%
    \vskip1.25em%
  }%
  \hbox to2cm{\hss\insertlogo\hss}
  \vskip1.25em%
  \insertverticalnavigation{2cm}%
  \vfill
  \hbox to 2cm{\hfill\usebeamerfont{subsection in
      sidebar}\strut\usebeamercolor[fg]{subsection in
      sidebar}\insertshortlecture.\insertframenumber\hskip5pt}%
  \vskip3pt%
}%



% Article version layout settings

\mode<article>
\makeatletter
\def\@listI{\leftmargin\leftmargini
  \parsep 0pt
  \topsep 5\p@   \@plus3\p@ \@minus5\p@
  \itemsep0pt}
\let\@listi=\@listI


\setbeamertemplate{frametitle}{\paragraph*{\insertframetitle\
    \ \small\insertframesubtitle}\ \par
}
\setbeamertemplate{frame end}{%
  \marginpar{\scriptsize\hbox to 1cm{\sffamily%
      \hfill\strut\insertshortlecture.\insertframenumber}\hrule height .2pt}}
\setlength{\marginparwidth}{1cm}
\setlength{\marginparsep}{4.5cm}

\def\@maketitle{\makechapter}


\let\origstartsection=\@startsection
\def\@startsection#1#2#3#4#5#6{%
  \origstartsection{#1}{#2}{#3}{#4}{#5}{#6\normalfont\sffamily\color{blue!50!black}\selectfont}}

\makeatother

\mode
<all>

% Typesetting Listings

\lstset{language=Java}

\theoremstyle{definition}
\newtheorem{exercise}[theorem]{\translate{Fact:}}
\newtheorem{Note}[theorem]{\translate{Note}}

\begin{document}

\begin{frame}
\titlepage
\end{frame}

\section{Reminder from Thursday}

\begin{frame}\frametitle{Elliptic curves over $\F_q$}

\begin{definition}[Elliptic curve] An elliptic curve over a field $K$ is the data of a non
singular Weierstra\ss\ equation
$E: y^2+a_1xy+a_3y=x^3+a_2x^2+a_4x+a_6, a_i\in K$
\end{definition}

If $p=\operatorname{char}K>3$,
\centerline{\begin{beamercolorbox}[shadow=true,center,rounded=true,wd=\textwidth]{formul}
\begin{align*}
\Delta_E&:=\frac{1}{2^4}\left(-a_1^5 a_3 a_4 - 8 a_1^3 a_2 a_3 a_4 - 16 a_1 a_2^2 a_3 a_4 + 36 a_1^2 a_3^2 a_4 \right. \\
  &-a_1^4 a_4^2 - 8 a_1^2 a_2 a_4^2 - 16 a_2^2 a_4^2 + 96 a_1 a_3 a_4^2 +64 a_4^3 + \\
  & a_1^6 a_6 + 12 a_1^4 a_2 a_6 + 48 a_1^2 a_2^2 a_6 + 64 a_2^3 a_6 -36 a_1^3 a_3 a_6\\
  &\left. - 144 a_1 a_2 a_3 a_6 - 72 a_1^2 a_4 a_6 - 288 a_2 a_4 a_6 +
  432 a_6^2  \right)\ne0
 \end{align*}
\end{beamercolorbox}}
\end{frame}


\begin{frame}\frametitle{Elliptic curves over $K$}


After applying a suitable affine transformation we can always assume that $E/K$
has a Weierstra\ss\ equation of the following form

\begin{small}
 \begin{example}[Classification ($p=\operatorname{char}K$)]
\centerline{\begin{tabular}{|l|c|l|}
\hline
 $E$ & $p$ & $\Delta_E$\\
\hline
&&\\
 $y^2=x^3+Ax+B$ & $\ge5$ & $4A^3+27B^2$\\
&&\\
$y^2+xy=x^3+a_2x^2+a_6$ & $2$ & $a_6^2$\\
&&\\
 $y^2+a_3y=x^3+a_4x+a_6$  & $2$ & $a_3^4$\\
&&\\
 $y^2=x^3+Ax^2+Bx+C$ & $3$ & $\!\begin{array}{l}
                               4A^3C-A^2B^2-18ABC\\+4B^3+27C^2
                              \end{array}$\\
&&\\\hline
\end{tabular}}
\end{example}
\end{small}\pause

\centerline{\begin{beamercolorbox}[shadow=true,left,rounded=true,wd=\textwidth]{formul}
Let $E/\F_q$ elliptic curve, set $\infty:=[0,1,0]$. Set\\
\centerline{$E(\F_q)=\{(x,y)\in \F_q^2:\ y^2=x^3+Ax+B\}\cup\{\infty\}$}
\end{beamercolorbox}}
\end{frame}

% \begin{frame}
% \frametitle{Projective lines}
% \framesubtitle{tangent lines to projective curves}
% 
% \begin{Definition}
% If $P=[x_1,y_1,z_1], Q=[x_2,y_2,z_2]\in\mathbb P_2(\F_q)$, \emph{the projective
% line} through $P$, $Q$ is\\
% \centerline{$r_{P,Q}: \det\left|\begin{matrix}
%                       X & Y & Z \\ x_1&y_1&z_1\\ x_2&y_2&z_2
%                      \end{matrix}\right|=0$}
% \end{Definition}\pause
% 
% \begin{Definition}
% The \emph{tangent line} to a projective curve $F(X,Y,Z)=0$ at a non singular point $P=[X_0,Y_0,Z_0]$
% ($F(X_0,Y_0,Z_0)=0$) is
% \scriptsize{{\color[cmyk]{1,0,1,0.5}$\frac{\partial F}{\partial X}(X_0,Y_0,Z_0)X+\frac{\partial F}{\partial Y}(X_0,Y_0,Z_0)Y+
% \frac{\partial F}{\partial Z}(X_0,Y_0,Z_0)Z=0$}}
% \end{Definition}\pause
% 
% \begin{exercise}[Properties:]
% \begin{enumerate}[<+-| alert@+>]
%  \item $P$ belongs to its (projective) tangent line
%  \item $P$ affine $\Rightarrow$ its tangent line is the homogenized of the affine tangent line
%  \item the tangent line to $E/\F_q$ at $\infty=[0,1,0]$ is $Z=0$ (line at infinity)\vspace*{-5.2pt}
% \end{enumerate}
% \end{exercise}
% \end{frame}


\section{The sum of points}
\begin{frame}
\frametitle{The definition of $E(\F_q)$}
\centerline{\begin{beamercolorbox}[shadow=true,left,rounded=true,wd=\textwidth]{formul}
Let $E/\F_q$ elliptic curve. Set
$$\!\!\!E(\F_q)=\{(x,y)\in \F_q^2: y^2+a_1xy+a_3y\!=\!x^3+a_2x^2+a_4x+a_6\}\cup\{\infty\}\!$$
\end{beamercolorbox}}\pause

\ \hfill \begin{beamercolorbox}[shadow=true,left,rounded=true,wd=9cm]{postit}
 Hence\pause
$$E(\F_q)\subset\F_q^2\cup\{\infty\}$$\pause
\ \hfill$\infty$ might be though as the ``vertical direction''
\end{beamercolorbox}\pause

\begin{Definition}[line through points $P,Q\in E(\F_q)$]
$r_{P,Q}:\begin{cases}
                     \text{line through $P$ and }Q &\text{if }P\neq Q\\
                     \text{tangent line to $E$ at }P &\text{if }P=Q
                    \end{cases}$
\end{Definition}\pause

\begin{itemize}[<+-| alert@+>]
\item if $\#(r_{P,Q}\cap E(\F_q))\ge2\ \Rightarrow\ \#(r_{P,Q}\cap E(\F_q))=3$\\
\hfill\scriptsize{\alert{if tangent line, contact point is counted with multiplicity}}  
\item $r_{\infty,\infty}\cap E(\F_q)=\{\infty,\infty,\infty\}$%\vspace*{-4.4pt}
 % $\#(r_{P_1,P_1}\cap E(\F_q))=2$
 \item $r_{P,Q}: aX+b=0$ (vertical) $\Rightarrow \infty=\in r_{P,Q}$
\end{itemize}

\end{frame}

\begin{frame}
\frametitle{History (from \textsc{Wikipedia})}

\begin{columns}[c]
\begin{column}{4.5cm}
\begin{small}
\textbf{Carl Gustav Jacob Jacobi} (10/12/1804 -- 18/02/1851) was a German mathematician,
who made fundamental contributions to elliptic functions, dynamics, differential equations,
and number theory.
\end{small}\\
\centerline{\includegraphics[width=1.8cm]{images/Jacobi.jpg}}
%\centerline{\scriptsize{Carl Gustav Jacob Jacobi}}\\
\begin{scriptsize}\begin{block}{Some of His Achievements:}
\begin{itemize}
 \item Theta and elliptic function
 \item Hamilton Jacobi Theory
 \item Inventor of determinants
 \item Jacobi Identity\\
 \tiny{ $[A,[B,C]]+[B,[C,A]]+[C,[A,B]]=0$}
\end{itemize}
\end{block}\end{scriptsize}
\end{column}\pause
\begin{column}{5.5cm}\vspace*{-16.3pt}
\begin{center}
\includegraphics[width=5.5cm]{images/add1.pdf}\pause
\llap{\includegraphics[width=5.5cm]{images/add2.pdf}}\pause
\llap{\includegraphics[width=5.5cm]{images/add3.pdf}}\pause
\llap{\includegraphics[width=5.5cm]{images/add5.pdf}}\pause
\llap{\includegraphics[width=5.5cm]{images/add6.pdf}}\pause
\llap{\includegraphics[width=5.5cm]{images/add7.pdf}}\pause
\llap{\includegraphics[width=5.5cm]{images/add1.pdf}}\pause
\llap{\includegraphics[width=5.5cm]{images/add8.pdf}}\pause
\llap{\includegraphics[width=5.5cm]{images/add9.pdf}}\pause
\llap{\includegraphics[width=5.5cm]{images/ad10.pdf}}\pause
\llap{\includegraphics[width=5.5cm]{images/ad11.pdf}}\pause
\llap{\includegraphics[width=5.5cm]{images/ad12.pdf}}\pause
\llap{\includegraphics[width=5.5cm]{images/add1.pdf}}\pause
\llap{\includegraphics[width=5.5cm]{images/ad13.pdf}}\pause
\llap{\includegraphics[width=5.5cm]{images/ad14.pdf}}\pause
\llap{\includegraphics[width=5.5cm]{images/ad15.pdf}}\pause
\llap{\includegraphics[width=5.5cm]{images/add7.pdf}}\pause
\end{center}
\small{
$r_{P,Q}\cap E(\F_q)=\{P,Q,R\}$\\
$r_{R,\infty}\cap E(\F_q)=\{\infty,R,R'\}$}
\centerline{\begin{beamercolorbox}[shadow=true,center,rounded=true,wd=2cm]{formul}
$P+_E Q:=R'$\pause
            \end{beamercolorbox}}\smallskip

 \small{$r_{P,\infty}\cap E(\F_q)=\{P,\infty,P'\}$}\\
 \centerline{\begin{beamercolorbox}[shadow=true,center,rounded=true,wd=2cm]{formul}
             $-P:=P'$
            \end{beamercolorbox}}

\end{column}
\end{columns}
\end{frame}

\begin{frame}
\frametitle{Properties of the operation ``$+_E$''}

\begin{Theorem}
 The addition law on $E(\F_q)$ has the following
properties:
\begin{enumerate}[<+-| alert@+>][(a)]
 \item $P+_EQ\in E(\F_q)\hfill\forall P,Q\in E(\F_q)$
 \item  $P+_E\infty=\infty+_E P=P\hfill\forall P\in E(\F_q)$
 \item  $P+_E(-P)=\infty\hfill\forall P\in E(\F_q)$
 \item  $P+_E(Q +_E R)=(P+_E Q)+_E R\hfill\forall P,Q,R\in E(\F_q)$
 \item  $P+_E Q=Q +_E P\hfill\forall P,Q\in E(\F_q)$
\end{enumerate}
 \end{Theorem}\pause

\begin{itemize}[<+-| alert@+>]
%  \item By ``a point of $E/\F_q$ ($P\in E$)'' we mean $P\in E(\bar{\F}_q)$
%  in analogy for $E/\Q$ where ``a point of $E$'' means  $P\in E(\C)$
 \item $\left(E(\F_q),+_E\right)$  \alert{commutative group}
 \item All group properties are easy except \alert{associative law (d)}
 \item Geometric proof of associativity uses \emph{Pappo's Theorem}
 %\item We shall comment on how to do it by explicit computation
 \item can substitute $\F_q$ with any field $K$; Theorem holds for $\left(E(K),+_E\right)$
\item In particular, if $E/\F_q$, can consider the groups $E(\overline{\F}_q)$ or $E(\F_{q^n})$
\end{itemize}
\end{frame}

\begin{frame}
\frametitle{Computing the inverse $-P$}
\centerline{\begin{beamercolorbox}[shadow=true,center,rounded=true,wd=7cm]{formul}
$E: y^2+a_1xy+a_3y=x^3+a_2x^2+a_4x+a_6$\end{beamercolorbox}}\pause

If $P=(x_1,y_1)\in E(\F_q)$

\begin{beamercolorbox}[shadow=true,left,rounded=true,wd=\textwidth]{formul}
           \textbf{\color[rgb]{1,0.3,1}Definition:}  $-P:=P'$ where $r_{P,\infty}\cap E(\F_q)=\{P,\infty,P'\}$\hfill\
            \end{beamercolorbox}\hfill\pause

Write $P'=(x_1',y_1')$. Since $r_{P,\infty}: x=x_1\ \Rightarrow x_1'=x_1$ and $y_1$ satisfies
\centerline{\begin{beamercolorbox}[shadow=true,center,rounded=true,wd=8.8cm]{postit}
$y^2+a_1x_1y+a_3y-(x_1^3+a_2x_1^2+a_4x_1+a_6)=(y-y_1)(y-y_1')$
\end{beamercolorbox}}\pause

So $y_1+y_1'=-a_1x_1-a_3$ (\alert{both coefficients of $y$}) and
\centerline{\begin{beamercolorbox}[shadow=true,center,rounded=true,wd=6cm]{postit}
$-P=-(x_1,y_1)=(x_1,-a_1x_1-a_3-y_1)$
 \end{beamercolorbox}}\pause


So, if $P_1=(x_1,y_1), P_2=(x_2,y_2)\in E(\F_q)$,
\begin{beamercolorbox}[shadow=true,center,rounded=true,wd=\textwidth]{formul}
           \textbf{\color[rgb]{1,0.3,1}Definition:} $P_1+_EP_2=-P_3$ where $r_{P_1,P_2}\cap E(\F_q)=\{P_1,P_2,P_3\}\!$
            \end{beamercolorbox}\hfill\pause

Finally, if $P_3=(x_3,y_3)$, then
\centerline{\begin{beamercolorbox}[shadow=true,center,rounded=true,wd=6cm]{postit}
$P_1+_EP_2=-P_3=(x_3,-a_1x_3-a_3-y_3)$
\end{beamercolorbox}}
\end{frame}

\begin{frame}
\frametitle{Lines through points of $E$}
\centerline{\begin{beamercolorbox}[shadow=true,center,rounded=true,wd=\textwidth]{formul}
$E: y^2+a_1xy+a_3y=x^3+a_2x^2+a_4x+a_6$\end{beamercolorbox}}
where $a_1, a_3, a_2, a_4 ,a_6\in\F_q,$ \pause

\begin{beamerboxesrounded}[upper=block title example,lower=block body alerted,shadow=true]
{$P_1=(x_1,y_1), P_2=(x_2,y_2)\in E(\F_q)$}
\begin{enumerate}[<+-| alert@+>]
 \item $P_1\neq P_2$ and $x_1\neq x_2\hfil \Longrightarrow\hfil r_{P_1,P_2}: y=\lambda x+\nu$
\begin{beamercolorbox}[shadow=true,center,rounded=true,wd=6cm]{postit}
$$\lambda= \frac{y_2-y_1}{x_2-x_1},\qquad \nu=\frac{y_1x_2-x_1y_2}{x_2-x_1}$$
\end{beamercolorbox}
 \item $P_1\neq P_2$ and $x_1=x_2\hfil \Longrightarrow\hfil r_{P_1,P_2}: x=x_1$
 \item  $P_1=P_2$ and $2y_1+a_1x_1+a_3\neq0\ \Longrightarrow\ r_{P_1,P_2}: y=\lambda x+\nu$
\hspace*{-0.5cm}\begin{beamercolorbox}[shadow=true,center,rounded=true,wd=9.3cm]{postit}
$$\lambda=\frac{3x_1^2+2a_2x_1+a_4-a_1y_1}{2y_1+a_1x_1+a_3},
 \nu=-\frac{a_3y_1+x_1^3-a_4x_1-2a_6}{2y_1+a_1x_1+a_3}$$
\end{beamercolorbox}
\item $P_1=P_2$ and $2y_1+a_1x_1+a_3=0\hfil \Longrightarrow\hfil r_{P_1,P_2}: x=x_1$
\item $r_{P_1,\infty}: x=x_1\hfill r_{\infty,\infty}: Z=0$
\end{enumerate}
\end{beamerboxesrounded}
\end{frame}

\begin{frame}
\frametitle{Intersection between a line and $E$}
We want to compute $P_3= (x_3,y_3)$ where $r_{P_1,P_2}: y=\lambda x +\nu$,
\centerline{\begin{beamercolorbox}[shadow=true,center,rounded=true,wd=5cm]{postit}
$r_{P_1,P_2}\cap E(\F_q)=\{P_1,P_2,P_3\}$
\end{beamercolorbox}}
\pause

We find the intersection:
\centerline{\begin{beamercolorbox}[shadow=true,center,rounded=true,wd=\textwidth]{formul}
$r_{P_1,P_2}\cap E(\F_q)=$ \scriptsize{\ $\begin{cases}
 E:y^2+a_1xy+a_3y=x^3+a_2x^2+a_4x+a_6\\ r_{P_1,P_2}: y=\lambda x +\nu
 \end{cases}$}
\end{beamercolorbox}}
 \pause
Substituting\\
\centerline{\begin{beamercolorbox}[center,wd=\textwidth]{postit}
$(\lambda x +\nu)^2+a_1x(\lambda x +\nu)+a_3(\lambda x +\nu)=x^3+a_2x^2+a_4x+a_6$
            \end{beamercolorbox}}\pause
Since $x_1$ and $x_2$ are solutions, we can
find $x_3$ by comparing
\begin{scriptsize}

\centerline{\begin{beamercolorbox}[center,wd=9cm]{postit}
    \begin{align*}
&x^3+a_2x^2+a_4x+a_6-((\lambda x +\nu)^2+a_1x(\lambda x +\nu)+a_3(\lambda x +\nu))&=\\
\uncover<5->{&x^3+(\alert{a_2-\lambda^2-a_1\lambda})x^2+\cdots&=\\ }
\uncover<6->{&(x - x_1)(x - x_2)(x - x_3) = x^3 - ({\color[rgb]{0,0,1}x_1 + x_2 + x_3})x^2 + \cdots&\\ } %(x_1x_2 + x_1x_3 + x_2x_3)x - x_1x_2x_3
\notag
\end{align*}\vskip-1.5em
\end{beamercolorbox}}
\end{scriptsize}\pause\pause\pause

Equating coeffcients of $x^2$,
\centerline{\begin{beamercolorbox}[center,wd=8cm]{postit}
$x_3 = \lambda^2-a_1\lambda-a_2- x_1-x_2,\qquad y_3 = \lambda x_3 + \nu$
            \end{beamercolorbox}}
\pause
Finally\\
\centerline{\begin{beamercolorbox}[shadow=true,center,rounded=true,wd=\textwidth]{formul}
\small{$P_3 =({\color[cmyk]{0,1,1,0.5}\lambda^2-a_1\lambda-a_2-x_1-x_2},{\color[cmyk]{1,0,1,0.5}\lambda^3-a_1\lambda^2-\lambda(a_2+x_1+x_2)+\nu})$}
            \end{beamercolorbox}}
\end{frame}

\begin{frame}
\frametitle{Formulas for Addition on $E$ (Summary)}
\centerline{\begin{beamercolorbox}[shadow=true,center,rounded=true,wd=\textwidth]{formul}
$E: y^2+a_1xy+a_3y=x^3+a_2x^2+a_4x+a_6$\end{beamercolorbox}}\pause
$P_1 = (x_1, y_1), P_2 = (x_2, y_2)\in E(\F_q)\setminus\{\infty\}$,
\begin{beamerboxesrounded}[upper=block title example,lower=block body alerted,shadow=true]{Addition Laws for the sum of affine points}
\begin{itemize}[<+-| alert@+>]
 \item If $P_1\neq P_2$
\begin{itemize}
 \item $x_1 = x_2\ \hfill\Rightarrow\hfil$\ \ \
\begin{beamercolorbox}[shadow=true,center,rounded=true,wd=2cm]{formul}$P_1 +_E P_2 = \infty$
\end{beamercolorbox}
 \item $x_1 \neq x_2$\\
\centerline{\begin{beamercolorbox}[shadow=true,center,wd=4cm]{postit}
             $\lambda=\frac{y_2-y_1}{x_2-x_1}\qquad \nu=\frac{y_1x_2-y_2x_1}{x_2-x_1}$
            \end{beamercolorbox}}
 \end{itemize}
\item If $P_1 = P_2$
\begin{itemize}
 \item $2y_1+a_1x+a_3 = 0\ \hfill\Rightarrow\hfil$\ \ \
\begin{beamercolorbox}[shadow=true,center,rounded=true,wd=3cm]{formul}$P_1 +_E P_2 = 2P_1 = \infty$\end{beamercolorbox}
\item $2y_1+a_1x+a_3\neq 0$\\
\centerline{\begin{beamercolorbox}[shadow=true,center,wd=7cm]{postit}
$\lambda=\frac{3x_1^2+2a_2x_1+a_4-a_1y_1}{2y_1+a_1x+a_3}, \nu=-\frac{a_3y_1+x_1^3-a_4x_1-2a_6}{2y_1+a_1x_1+a_3}$
            \end{beamercolorbox}}
\end{itemize}
\end{itemize}\pause

Then\\
\centerline{\begin{beamercolorbox}[shadow=true,center,rounded=true,wd=10cm]{formul}
\scriptsize{$P_1 +_E P_2 = ({\color[cmyk]{0,1,1,0.5}\lambda^2-a_1\lambda-a_2-x_1-x_2},
{\color[cmyk]{1,0,1,0.5}-\lambda^3-a_1^2\lambda+(\lambda+a_1)(a_2+x_1+x_2)-a_3-\nu})$}
            \end{beamercolorbox}}
\end{beamerboxesrounded}
\end{frame}

\begin{frame}
\frametitle{Formulas for Addition on $E$ (Summary for special equation)}
\centerline{\begin{beamercolorbox}[shadow=true,center,rounded=true,wd=\textwidth]{formul}
$E: y^2=x^3+Ax+B$\end{beamercolorbox}}
$P_1 = (x_1, y_1), P_2 = (x_2, y_2)\in E(\F_q)\setminus\{\infty\}$,
\begin{beamerboxesrounded}[upper=block title example,lower=block body alerted,shadow=true]{Addition Laws for  the sum of affine points}
\begin{itemize}
 \item If $P_1\neq P_2$
\begin{itemize}
 \item $x_1 = x_2\ \hfill\Rightarrow\hfil$\ \ \
\begin{beamercolorbox}[shadow=true,center,rounded=true,wd=2cm]{formul}$P_1 +_E P_2 = \infty$
\end{beamercolorbox}
 \item $x_1 \neq x_2$\\
\centerline{\begin{beamercolorbox}[shadow=true,center,wd=4cm]{postit}
             $\lambda=\frac{y_2-y_1}{x_2-x_1}\qquad \nu=\frac{y_1x_2-y_2x_1}{x_2-x_1}$
            \end{beamercolorbox}}
 \end{itemize}
\item If $P_1 = P_2$
\begin{itemize}
 \item $y_1 = 0\ \hfill\Rightarrow\hfil$\ \ \
\begin{beamercolorbox}[shadow=true,center,rounded=true,wd=3cm]{formul}$P_1 +_E P_2 = 2P_1 = \infty$\end{beamercolorbox}
\item $y_1\neq 0$\\
\centerline{\begin{beamercolorbox}[shadow=true,center,wd=7cm]{postit}
$\lambda=\frac{3x_1^2+A}{2y_1}, \nu=-\frac{x_1^3-Ax_1-2B}{2y_1}$
            \end{beamercolorbox}}
\end{itemize}
\end{itemize}

Then\\
\centerline{\begin{beamercolorbox}[shadow=true,center,rounded=true,wd=7cm]{formul}
\small{$P_1 +_E P_2 = ({\color[cmyk]{0,1,1,0.5}\lambda^2-x_1-x_2},
{\color[cmyk]{1,0,1,0.5}-\lambda^3+\lambda(x_1+x_2)-\nu})$}
            \end{beamercolorbox}}
\end{beamerboxesrounded}

\end{frame}

\begin{frame}
 \begin{Theorem}
 The addition law on $E/K$ ($K$ field) has the following
properties:
\begin{enumerate}[(a)]
 \item  $P+_EQ\in E \hfill\forall P,Q\in E$
 \item  $P+_E\infty=\infty+_E P=P\hfill\forall P\in E$
 \item  $P+_E(-P)=\infty\hfill\forall P\in E$
 \item  $P+_E(Q +_E R)=(P+_E Q)+_E R\hfill\forall P,Q,R\in E$
 \item  $P+_E Q=Q +_E P\hfill\forall P,Q\in E$
\end{enumerate}
So $(E(\bar{K}),+_E)$ is an abelian group.
 \end{Theorem}\pause

 \begin{beamerboxesrounded}[upper=block title example,lower=block body alerted,shadow=true]{Remark:}
If $E/K \ \Rightarrow\ \forall L, K\subseteq L\subseteq\bar{K}, E(L)$ is an abelian group.
\end{beamerboxesrounded}\medskip\pause


\centerline{\begin{beamercolorbox}[shadow=true,center,rounded=true,wd=6cm]{postit}
$$-P=-(x_1,y_1)=(x_1,-a_1x_1-a_3-y_1)$$
\end{beamercolorbox}}
\end{frame}


\begin{frame}
 \frametitle{A Finite Field Example}

Over $\F_p$ geometric pictures don't make sense.

 \begin{example}
Let
$E: y^2 = x^3 - 5x + 8 /\F_{37}$,\pause\hfill $P = (6, 3) , Q = (9, 10)\in E(\F_{37})$\pause

\centerline{\begin{beamercolorbox}[shadow=true,center,rounded=true,wd=7.3cm]{formul}
$r_{P,Q}: y=27x+26\quad r_{P,P}: y=11x+11 $
            \end{beamercolorbox}}\pause

\centerline{\begin{beamercolorbox}[shadow=true,center,rounded=true,wd=\textwidth]{postit}
$r_{P,Q}\cap E(\F_{37})=\begin{cases}
                          y^2 = x^3 - 5x + 8 \\ y = 27 x + 26
                         \end{cases}\!\!=\{(6,3), (9,10), (11,27)\}$
                                     \end{beamercolorbox}}\pause

\centerline{\begin{beamercolorbox}[shadow=true,center,rounded=true,wd=\textwidth]{postit}
$r_{P,P}\cap E(\F_{37})=\begin{cases}
                          y^2 = x^3 - 5x + 8 \\ y =11 x + 11
                         \end{cases}\!\!=\{(6,3), (6,3), (35,26)\}$\end{beamercolorbox}}\pause


\centerline{\begin{beamercolorbox}[shadow=true,center,rounded=true,wd=7.3cm]{formul}
$P+_EQ=(11,10)\qquad 2P=(35,11)$
            \end{beamercolorbox}}\pause

\ \hfill\scriptsize{$3P=(34,25), 4P=(8,6), 5P=(16,19),\ldots 3P+4Q=(31,28),\ldots$}
 \end{example}

%  \begin{beamerboxesrounded}[upper=block title example,lower=block body alerted,shadow=true]{Exercise}
%  Compute the order and the {\color[rgb]{0.1,0.3,1}{Group Structure}} of $E(\F_{37})$
%   \end{beamerboxesrounded}
\end{frame}

\begin{frame}%[label=current]
 \frametitle{Group Structure}

\begin{theorem}[Classification of finite abelian groups]
 If $G$ is {\color[rgb]{0.9,0.3,0.2}{abelian and finite}},  $\exists n_1,\ldots,n_k\in\N^{>1}$ such that
 \begin{enumerate}[<+-| alert@+>]
\item $n_1\mid n_2\mid\cdots\mid n_k$
\item $G\cong C_{n_1}\oplus\cdots\oplus C_{n_k}$
\end{enumerate}
Furthermore $n_1,\ldots,n_k$ ({\color[rgb]{0.9,0.3,0.2} Group Structure}) are unique
 \end{theorem}\pause

\begin{example}[One can verify that:]
 $$C_{2400}\oplus C_{72} \oplus C_{1440}\cong C_{12}\oplus C_{60}\oplus C_{15200}$$
\end{example}\pause

Shall show that
\centerline{\begin{beamercolorbox}[shadow=true,center,rounded=true,wd=6cm]{formul}
$$E(\F_q)\cong C_n\oplus C_{nk}\qquad\exists n,k\in\N^{>0}$$
            \end{beamercolorbox}}\pause

            (i.e. $E(\F_q)$ is either cyclic ($n=1$) or the product of $2$ cyclic groups)
\end{frame}

% \begin{frame}%[label=current2]
%  \frametitle{Proof of the associativity}
%  \centerline{\begin{beamercolorbox}[shadow=true,center,rounded=true,wd=7.3cm]{formul}
%              $P+_E(Q+_ER)=(P+_EQ)+_ER\quad\forall P,Q,R\in E$
%             \end{beamercolorbox}}\pause
%  We should verify the above in many different cases according if $Q=R$, $P=Q$, $P=Q+_ER,\ldots$\pause
% 
% Here we deal with the \emph{generic case}. i.e. All the points
% \alert{$\pm P, \pm R,\pm Q,\pm(Q+_ER),\pm(P+_EQ),\infty$} all different
% 
% {\begin{beamercolorbox}[shadow=true,left,rounded=true,wd=9.6cm]{postit}
%  \scriptsize{{\color[rgb]{1,0.1,0.1}\texttt{Mathematica code}}\\
% \texttt{L[x\_,y\_,r\_,s\_]:=(s-y)/(r-x);\\
% M[x\_,y\_,r\_,s\_]:=(yr-sx)/(r-x);\\
% A[\{x\_,y\_\},\{r\_,s\_\}]:=\{(L[x,y,r,s])$^2$-(x+r),\\
% \ \hfill -(L[x,y,r,s])$^3$+L[x,y,r,s](x+r)-M[x,y,r,s]\}\\
% Together[A[A[\{x,y\},\{u,v\}],\{h,k\}]-A[\{x,y\},A[\{u,v\},\{h,k\}]]]\\
% det = Det[(\{\{1,x$_1$,x$_1^3$-y$_1^2$\},\{1,x$_2$,x$_2^3$-y$_2^2$\},\{1,x$_3$,x$_3^3$-y$_3^2$\}\})]\\
% PolynomialQ[Together[Numerator[Factor[res[[1]]]]/det],\\
% \ \hfill\{x$_1$,x$_2$,x$_3$,y$_1$,y$_2$,y$_3$\}]
% PolynomialQ[Together[Numerator[Factor[res[[2]]]]/det],\\ \ \hfill\{x$_1$,x$_2$,x$_3$,y$_1$,y$_2$,y$_3$\}]}}
%              \end{beamercolorbox}}\pause
% 
% 
% % {\begin{beamercolorbox}[shadow=true,left,rounded=true,wd=9.6cm]{postit}
% %  \scriptsize{{\color[rgb]{1,0.1,0.1}\texttt{Pari code}}\\
% % \texttt{L(a,b,c,d)=(d-b)/(c-a);\\
% % M(a,b,c,d)=(b*c-a*d)/(c-a);\\
% % AX(a,b,c,d)=L(a,b,c,d)\^{}2-(a+c);\\
% % AY(a,b,c,d)=-L(a,b,c,d)\^{}3+L(a,b,c,d)*(a+c)-M(a,b,c,d);\\
% % simplify(AX(x1,y1,AX(x2,y2,x3,y3),AY(x2,y2,x3,y3))-\\ \ \hfill AX(AX(x1,y1,x2,y2),AY(x1,y1,x2,y2),x3,y3));\\
% % simplify(AY(x1,y1,AX(x2,y2,x3,y3),AY(x2,y2,x3,y3))-\\ \ \hfill AY(AX(x1,y1,x2,y2),AY(x1,y1,x2,y2),x3,y3))}}
% % \end{beamercolorbox}}\pause
% 
% \begin{scriptsize}
% \begin{itemize}[<+-| alert@+>]
%  \item runs in 2 seconds on a PC
%  %\item Complete proof requires several cases (e.g. $(P+_EP)+_ER=P+_E(P+_ER)$)
%  \item For an elementary proof:
%  ``\text{An Elementary Proof of the Group Law for Elliptic Curves.}''
% Department of Mathematics: Rice
% University. Web. 20 Nov. 2009.\\ \ \hfill\texttt{http://math.rice.edu/\~{}friedl/papers/AAELLIPTIC.PDF}
% \item More cases to check. e.g  \alert{$P+_E2Q=(P+_EQ)+_EQ$}
% \end{itemize}
% \end{scriptsize}
% \end{frame}



\section{Examples}
\subsection{Structure of \texorpdfstring{$E(\F_2)$}{E(F2)}}
\begin{frame}
\frametitle{EXAMPLE: Elliptic curves over $\F_2$}

From our previous list:
\begin{block}{Groups of points}
\begin{tabular}{|l|c|l|}
\hline
 $E$ & $E(\F_2)$ & $|E(\F_2)|$\\
\hline
&&\\
 $y^2+xy=x^3+x^2+1$ & $\{\infty,(0,1)\}$& $2$\\
&&\\
$y^2+xy=x^3+1$ & $\{\infty,(0,1),(1,0),(1,1)\}$ & $4$\\
&&\\
$y^2+y=x^3+x$&$\{\infty,(0,0),(0,1),$ &\\ &$(1,0),(1,1)\}$&$5$\\
&&\\
 $y^2+y=x^3+x+1$ &$\{\infty\}$&$1$\\
&&\\
$y^2+y=x^3$ & $\{\infty,(0,0), (0,1)\}$ & $3$ \\
&&\\\hline
\end{tabular}
\end{block}
\pause
So for each curve $E(\F_2)$ is cyclic except possibly for the second for which we need to distinguish between
$C_4$ and $C_2\oplus C_2$.\pause

\ \hfill \begin{beamercolorbox}[center,wd=9cm]{postit}
Note: each $C_i, i=1,\ldots,5$ is represented by a curve $/\F_2$
            \end{beamercolorbox}
\end{frame}


\subsection{Structure of \texorpdfstring{$E(\F_3)$}{E(F3)}}
\begin{frame}
\frametitle{EXAMPLE: Elliptic curves over $\F_3$}
From our previous list:

\begin{block}{Groups of points}%\begin{center}
\begin{tabular}{|l|r|c|c|}
\hline
$i$ & $E_i$ & $E_i(\F_3)$ &$|E_i(\F_3)|\!$\\
\hline
$1$& $y^2=x^3+x$ & \scriptsize{$\{\infty,(0,0),(2,1),(2,2)\}$}& $4$\\
\hline
$2$&$y^2=x^3 - x$ & \scriptsize{$\{\infty,(1,0),(2,0),(0,0)\}$} & $4$\\
\hline
$3$&$y^2=x^3 - x +1$&\tiny{$\{\infty,(0,1),(0,2),(1,1),(1,2),(2,1),(2,2)\}$} & $7$\\
\hline
$4$&$y^2=x^3 - x -1$  &\scriptsize{$\{\infty\}$}&$1$\\
\hline
$5$&$y^2=x^3 + x^2 - 1$ & \scriptsize{$\{\infty,(1,1), (1,2)\}$} & $3$ \\
\hline
$6$&$y^2=x^3 + x^2 + 1$ & \Tiny{$\{\infty,(0,1), (0,2), (1,0),(2,1), (2,2)\}$} & $6$ \\
\hline
$7$&$y^2=x^3 - x^2 + 1$ & \scriptsize{$\{\infty,(0,1), (0,2), (1,1), (1,2),\}$} & $5$ \\
\hline
$8$&$y^2=x^3 - x^2 - 1$ & \scriptsize{$\{\infty,(2,0))\}$} & $2$ \\
\hline
\end{tabular}
%\end{center}
\end{block}
\pause
Each $E_i(\F_3)$ is cyclic except possibly for $E_1(\F_3)$ and $E_2(\F_3)$ that could be either
$C_4$ or $C_2\oplus C_2$. We shall see that:\pause

\centerline{\begin{beamercolorbox}[shadow=true,center,rounded=true,wd=7cm]{formul}
$E_1(\F_3)\cong C_4\qquad\text{and}\qquad E_2(\F_3)\cong C_2\oplus C_2$
\end{beamercolorbox}}\pause

\ \hfill \begin{beamercolorbox}[center,wd=9cm]{postit}
Note: each $C_i, i=1,\ldots,7$ is represented by a curve $/\F_3$
            \end{beamercolorbox}


\end{frame}


\subsection{Further Examples}
\begin{frame}
\frametitle{EXAMPLE: Elliptic curves over $\F_5$}

\begin{example}[Elliptic curves over $\F_5$]
\begin{itemize}[<+-| alert@+>]
 \item $\forall E/\F_5$ (12 elliptic curves)
 \item $\#E(\F_5)\in \{2,3,4,5,6,7,8,9,10\}.$ 
 \item $\forall n, 2\le n\le10, \exists!\ E/\F_5: \#E(\F_5)=n$
%each number corresponds to a unique curve
\item[] with three exceptions:

\item \alert{$E_1: y^2=x^3+1$} and \alert{$E_2: y^2=x^3+2$}\hfill both order $6$
$$E_1(\F_5)\cong E_2(\F_5)\cong C_6$$
\item \alert{$E_3: y^2=x^3+x$} and \alert{$E_4: y^2=x^3+x+2$}
\hfill both order $4$
$$E_3(\F_5)\cong C_2\oplus C_2\qquad E_4(\F_5)\cong C_4$$
\item \alert{$E_5: y^2=x^3+4x$} and \alert{$E_6: y^2=x^3+4x+1$}
\hfill both order $8$
$$E_5(\F_5)\cong C_2\oplus C_4\qquad E_6(\F_5)\cong C_8$$
\item \alert{$E_7: y^2=x^3+x+1$}\hfill  order $9$ and $E_7(\F_5)\cong C_9$
\end{itemize}
\end{example}
\end{frame}


\section{Points of finite order}

\subsection{Points of order 2}
\begin{frame}\frametitle{Determining points of order $2$}
Let $P=(x_1,y_1)\in E(\F_q)\setminus\{\infty\},$\\ \pause
\centerline{
 \begin{beamercolorbox}[rounded=true,shadow=true,wd=9cm,center]{formul}
$P$ has order $2\ \Longleftrightarrow\ 2P=\infty\ \Longleftrightarrow\ P=-P$
\end{beamercolorbox}}\pause
So
\centerline{\small{
 \begin{beamercolorbox}[rounded=true,shadow=true,wd=10cm,center]{formul}
$-P=(x_1,-a_1x_1-a_3-y_1)=(x_1,y_1)=P\ \pause \Longrightarrow\ 2y_1=-a_1x_1-a_3$\end{beamercolorbox}}}\pause\medskip

If $p\neq2$, can assume $E: y^2=x^3+Ax^2+Bx+C$\pause

\centerline{\small{
 \begin{beamercolorbox}[rounded=true,shadow=true,wd=10cm,center]{formul}
$-P=(x_1,-y_1)=(x_1,y_1)=P\ \pause \Longrightarrow\ y_1=0,
x_1^3+Ax_1^2+Bx_1+C=0$\hfill
\end{beamercolorbox}}}\pause\medskip

\begin{Note}
\begin{itemize}[<+-| alert@+>]
 \item the number of points of order $2$ in $E(\F_q)$ equals the number of roots of $X^3+Ax^2+Bx+C$ in $\F_q$
 \item roots are distinct since discriminant $\Delta_E\neq0$
 \item $E(\F_{q^6})$ has always $3$ points of order $2$ if $E/\F_q$
 \item $E[2]:=\{P\in E(\bar{\F}_q): 2P=\infty\}\cong C_2\oplus C_2$
\end{itemize}
\end{Note}

\end{frame}

\begin{frame}\frametitle{Determining points of order $2$ (continues)}

\begin{itemize}[<+-| alert@+>]
\item If $p=2$ and $E: y^2+a_3y=x^3+a_2x^2+a_6$\pause

 \begin{beamercolorbox}[rounded=true,shadow=true,wd=10cm,center]{formul}
$-P=(x_1,a_3+y_1)=(x_1,y_1)=P\ \pause \Longrightarrow\ a_3=0$\end{beamercolorbox}\pause\medskip

Absurd ($a_3=0$) and there are no points of order $2$.
%$\begin{cases}
 %  x=a_3/a_1\\
  % y^2+a_1xy+a_3y+x^3+a_2x^2+a_4x+a_6=0
  %\end{cases}\longrightarrow$,
\item If $p=2$ and $E: y^2+xy=x^3+a_4x+a_6$\pause

 \begin{beamercolorbox}[rounded=true,shadow=true,wd=10cm,center]{formul}
$-P=(x_1,x_1+y_1)=(x_1,y_1)=P\ \pause \Longrightarrow\ x_1=0,y_1^2=a_6$\end{beamercolorbox}\pause\medskip

So there is exactly one point of order $2$ namely $(0,\sqrt{a_6})$
\end{itemize}\pause

\begin{Definition}{$2$--torsion points}
$$E[2]=\{P\in E: 2P=\infty\}.$$
\end{Definition}
In conclusion
$$E[2]\cong \begin{cases}
C_2\oplus C_2 &\text{if }p>2\\
C_2           &\text{if }p=2, E: y^2+xy=x^3+a_4x+a_6\\
\{\infty\}    &\text{if }p=2, E: y^2+a_3y=x^3+a_2x^2+a_6
\end{cases}
$$

\end{frame}

\begin{frame}
\frametitle{Elliptic curves over $\F_2, \F_3$ and $\F_5$}
\begin{small}
\begin{block}{Each curve $/\F_2$ has cyclic $E(\F_2)$.}
\begin{tabular}{|l|c|l|}
\hline
 $E$ & $E(\F_2)$ & $|E(\F_2)|$\\
\hline
 $y^2+xy=x^3+x^2+1$ & $\{\infty,(0,1)\}$& $2$\\
\hline
$y^2+xy=x^3+1$ & $\{\infty,(0,1),(1,0),(1,1)\}$ & $4$\\
\hline
$y^2+y=x^3+x$&$\{\infty,(0,0),(0,1),(1,0),(1,1)\}$&$5$\\
\hline
$y^2+y=x^3+x+1$ &$\{\infty\}$&$1$\\
\hline
$y^2+y=x^3$ & $\{\infty,(0,0), (0,1)\}$ & $3$ \\
\hline
\end{tabular}
\end{block}\end{small}
\pause
\begin{itemize}
 \item $E_1: y^2=x^3+x\qquad\qquad E_2:  y^2=x^3-x$\\
\centerline{\begin{beamercolorbox}[shadow=true,center,rounded=true,wd=7.5cm]{formul}
$E_1(\F_3)\cong C_4\qquad\text{and}\qquad E_2(\F_3)\cong C_2\oplus C_2$
\end{beamercolorbox}}
\item $E_3: y^2=x^3+x\qquad\qquad E_4: y^2=x^3+x+2$\\
\centerline{\begin{beamercolorbox}[shadow=true,center,rounded=true,wd=7.5cm]{formul}
$E_3(\F_5)\cong C_2\oplus C_2\qquad\text{and}\qquad E_4(\F_5)\cong C_4$
\end{beamercolorbox}}
\item $E_5: y^2=x^3+4x\qquad\qquad E_6: y^2=x^3+4x+1$\\
\centerline{\begin{beamercolorbox}[shadow=true,center,rounded=true,wd=7.5cm]{formul}
$E_5(\F_5)\cong C_2\oplus C_4\qquad\text{and}\qquad E_6(\F_5)\cong C_8$
\end{beamercolorbox}}
\end{itemize}
\end{frame}

\end{document}


