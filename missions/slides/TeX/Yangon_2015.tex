\documentclass[12pt,handout]{beamer} %,hyperref={pdfpagelabels=false},draft,handout,heout
\vfuzz=4pt
\hfuzz=4pt
\usepackage[orientation=lescape,size=custom,width=16,height=9,scale=0.30,debug]{beamerposter} 
\usepackage[english]{babel}
\usepackage{lmodern}
\usepackage[latin1]{inputenc}
\usepackage{times}
\usepackage{hyperref}
\usepackage[T1]{fontenc}
\usepackage{tikz}
\usepackage{colortbl}
\usepackage{yfonts}
\usepackage{translator} % comment this, se not available
\mode<article>
{
  \usepackage{times}
  \usepackage{mathptmx}
  \usepackage[left=1.5cm,right=6cm,top=1.5cm,bottom=3cm]{geometry}
}

\newcommand{\Q}{\mathbb Q}
\newcommand{\Z}{\mathbb Z}
\newcommand{\N}{\mathbb N}
\newcommand{\F}{\mathbb F}
\newcommand{\C}{\mathbb C}
\newcommand{\R}{\mathbb R}
% Common theorem-like environments

\theoremstyle{definition}
\newtheorem{exercise}[theorem]{\translate{Exercise}}
\newtheorem{rem}[theorem]{\translate{Remark}}
\newtheorem{conj}[theorem]{\translate{Conjecture}}
\newtheorem{teo}{Theorem}
\newtheorem{Note}[theorem]{\translate{Note}}
\newtheorem{Defi}[theorem]{\translate{Definition}}
\newtheorem{lem}{Lemma}

% New useful definitions:

\newbox\mytempbox
\newdimen\mytempdimen

\newcommand\includegraphicscopyright[3][]{%
  \leavevmode\vbox{\vskip3pt\raggedright\setbox\mytempbox=\hbox{\includegraphics[#1]{#2}}%
    \mytempdimen=\wd\mytempbox\box\mytempbox\par\vskip1pt%
    \fontsize{3}{3.5}\selectfont{\color{black!25}{\vbox{\hsize=\mytempdimen#3}}}\vskip3pt%
}}

\newenvironment{colortabular}[1]{\medskip\rowcolors[]{1}{blue!20}{blue!10}\tabular{#1}\rowcolor{blue!40}}{\endtabular\medskip}

\def\equad{\leavevmode\hbox{}\quad}

\newenvironment{greencolortabular}[1]
{\medskip\rowcolors[]{1}{green!50!black!20}{green!50!black!10}%
  \tabular{#1}\rowcolor{green!50!black!40}}%
{\endtabular\medskip} 

\lecture[1]{On Never Primitive points for Elliptic curves}
\date{}
\title[Dip. Matem. \& Fisica]{\insertlecture}
\subtitle{\ }
\author[Universit\`a Roma Tre]{Francesco Pappalardi}
\institute{Dipartimento di Matematica e Fisica\\
  Universit\`a Roma Tre\\ 
  \& \emph{Roman Number Theory Association}}

% Beamer version theme poniamotings

\useoutertheme[height=0pt,width=2cm,right]{sidebar}
\usecolortheme{rose,sidebartab}
\useinnertheme{circles}
\usefonttheme[solo large]{structurebold}

\setbeamercolor{formul}{fg=black,bg=pink}
\setbeamercolor{sidebar right}{bg=black!15}
\setbeamercolor{structure}{fg=blue}
\setbeamercolor{author}{parent=structure}
 \setbeamercolor{postit}{fg=black,bg=yellow}
 \setbeamercolor{greys}{fg=black,bg==black!25}

\setbeamerfont{title}{series=\normalfont,size=\LARGE}
\setbeamerfont{title in sidebar}{series=\bfseries}
\setbeamerfont{author in sidebar}{series=\bfseries}
\setbeamerfont*{item}{series=}
\setbeamerfont{frametitle}{size=}
\setbeamerfont{block title}{size=\small}
\setbeamerfont{subtitle}{size=\normalsize,series=\normalfont}

\setbeamertemplate{navigation symbols}{}
\setbeamertemplate{bibliography item}[book]
\setbeamertemplate{sidebar right}
{
  {\usebeamerfont{title in sidebar}%
    \vskip1.5em%
    \hskip3pt%
    \usebeamercolor[fg]{title in sidebar}%
    \insertshorttitle[width=2.1cm,respectlinebreaks]\par%   left,
    \vskip1.25em%
  }%
  {%
    \hskip3pt%
    \usebeamercolor[fg]{author in sidebar}%
    \usebeamerfont{author in sidebar}%
    \insertshortauthor[width=2cm,center,respectlinebreaks]\par%
    \vskip1.25em%
  }%
  \hbox to2cm{\hss\insertlogo\hss}
  \vskip1.25em%
  \insertverticalnavigation{2cm}%
  \vfill
  \hbox to 2cm{\hfill\usebeamerfont{subsection in
      sidebar}\strut\usebeamercolor[fg]{subsection in
      sidebar}\insertframenumber\hskip5pt}%
  \vskip3pt%
}%

\setbeamertemplate{title page}
{
  \vbox{}
  \vskip1em
  %{\huge Lecture \insertshortlecture\par}
  {\usebeamercolor[fg]{title}\usebeamerfont{title}\inserttitle\par}%
  \ifx\insertsubtitle\@empty%
  \else%
    \vskip0.25em%
    {\usebeamerfont{subtitle}\usebeamercolor[fg]{subtitle}\insertsubtitle\par}%
  \fi%
  \vskip1em\par
   \textbf{\Large{The $8^{\textrm{th}}$ International Conference on Science and}}\\
   \ \\
   \textbf{\Large{Mathematical Education
   in Developing Countries}}\\
\ \\
   \textbf{University of Yangon}\\
\ \\   \textbf{Myanmar}\\ \emph{ $4^{\textrm{th}}$-$6^{\textrm{th}}$ December 2015},\par
  \vskip0pt plus1filll
  \leftskip=0pt plus1fill\insertauthor\par
  \insertinstitute\vskip1em
}

\logo{\includegraphics[width=1cm]{roma3.pdf}}



% Article version layout poniamotings

\mode<article>

\makeatletter
\def\@listI{\leftmargin\leftmargini
  \parsep 0pt
  \topsep 5\p@   \@plus3\p@ \@minus5\p@
  \itemsep0pt}
\let\@listi=\@listI


\setbeamertemplate{frametitle}{\paragraph*{\insertframetitle\
    \ \small\insertframesubtitle}\ \par
}
\setbeamertemplate{frame end}{%
  \marginpar{\scriptsize\hbox to 1cm{\sffamily%
      \hfill\strut\insertframenumber}\hrule height .2pt}}
\setlength{\marginparwidth}{1cm}
\setlength{\marginparsep}{4.5cm}

\def\@maketitle{\makechapter}

\def\makechapter{
  \newpage
  \null
  \vskip 2em%
  {%
    \parindent=0pt
    \raggedright
    \sffamily
    \vskip8pt
    {\fontsize{36pt}{36pt}\selectfont Kapitel \insertshortlecture \par\vskip2pt}
    {\fontsize{24pt}{28pt}\selectfont \color{blue!50!black} \insertlecture\par\vskip4pt}
    {\Large\selectfont \color{blue!50!black} \insertsubtitle\par}
    \vskip10pt
%    \normalsize\selectfont Druckfassung der
   % Vorlesung \emph{\lecturename} vom \@date\par\vskip1.5em
    %\hfill Till Tantau, Institut f\"ur Theoretische Informatik, Universit\"at zu L\"ubeck
  }
  \par
  \vskip 1.5em%
}

\let\origstartsection=\@startsection
\def\@startsection#1#2#3#4#5#6{%
  \origstartsection{#1}{#2}{#3}{#4}{#5}{#6\normalfont\sffamily\color{blue!50!black}\selectfont}}

\makeatother

\mode
<all>




% Typesetting Listings

\usepackage{listings}
\lstset{language=Java}

\alt<presentation>
{\lstset{%
  basicstyle=\footnotesize\ttfamily,
  commentstyle=\slshape\color{green!50!black},
  keywordstyle=\bfseries\color{blue!50!black},
  identifierstyle=\color{blue},
  stringstyle=\color{orange},
  escapechar=\#,
  emphstyle=\color{red}}
}
{
  \lstset{%
    basicstyle=\ttfamily,
    keywordstyle=\bfseries,
    commentstyle=\itshape,
    escapechar=\#,
    emphstyle=\bfseries\color{red}
  }
}



\begin{document}

\begin{frame}
\titlepage
\end{frame}



%\section{Introduction}

\begin{frame}
 \frametitle{Notations}

\begin{alertblock}{Fields of characteristics 0}
 \begin{enumerate}
 \item $\Z$ is the ring of integers
 \item $\Q$ is the field of rational numbers
\item $\R$ is the field of real numbers
\item $\C$ is the fields of complex numbers
 \item For every prime $p$, $\F_p=\{0,1,\ldots,p-1\}$ is the prime field;
\end{enumerate}
\end{alertblock}

\centerline{\begin{beamercolorbox}[shadow=true,center,rounded=true,wd=5cm]{formul}
{\Huge{$\Z\subsetneq\Q\subsetneq\R\subsetneq\C$}}\end{beamercolorbox}}\bigskip

\centerline{\begin{beamercolorbox}[shadow=true,center,rounded=true,wd=7cm]{postit}
{\Large{$\Z\twoheadrightarrow\F_p, n\longmapsto n(\bmod p)\text{ surjective map}$}}\end{beamercolorbox}}


\end{frame}

\begin{frame}{The Weierstra\ss\ Equation}

A Weierstra\ss\ equation $E$ over a $K$ (field) is an equation
\centerline{\begin{beamercolorbox}[shadow=true,center,rounded=true,wd=6cm]{formul}
$E: y^2=x^3+Ax^2+Bx+C$\end{beamercolorbox}}
where $A,B,C\in K$ \pause

\begin{center}
% \includegraphics[width=60mm]{images/elliptic1.pdf}\pause
%\llap{\includegraphics[width=60mm]{images/elliptic2.pdf}}\pause
%\llap{\includegraphics[width=60mm]{images/elliptic3.pdf}}\pause
%\llap{\includegraphics[width=60mm]{images/elliptic3b.pdf}}\pause
%\llap{\includegraphics[width=60mm]{images/elliptic4.pdf}}\pause
%\llap{\includegraphics[width=60mm]{images/elliptic5.pdf}}\pause
%\llap{\includegraphics[width=60mm]{images/elliptic6.pdf}}\pause
{\includegraphics[width=60mm]{images/elliptic7.pdf}}\pause
\llap{\includegraphics[width=60mm]{images/elliptic8.pdf}}\pause
\llap{\includegraphics[width=60mm]{images/elliptic9.pdf}}\pause
\llap{\includegraphics[width=60mm]{images/elliptic9b.pdf}}\pause
\llap{\includegraphics[width=60mm]{images/elliptic10.pdf}}\pause
\llap{\includegraphics[width=60mm]{images/elliptic10b.pdf}}\pause
\llap{\includegraphics[width=60mm]{images/elliptic6.pdf}}\pause
\end{center}\pause

\begin{center}
 
 \begin{beamercolorbox}[sep=1em,wd=9.5cm]{postit}
A Weierstra\ss\ equation  is called \textbf{elliptic curve} if it is \emph{non singular}!\pause

(i.e. $4A^3C-A^2B^2-18ABC+4B^3+27C^2\ne0$)
 \end{beamercolorbox}
\end{center}\pause

We consider (most of times) simplified Weierstra\ss\ equation $y^2=x^3+ax+b$ that are elliptic curves when
$4a^3+27b^2\neq0$
 \end{frame}
%\section{The sum of points}
\begin{frame}
\frametitle{The definition of $E(K)$}
%\centerline{\begin{beamercolorbox}[shadow=true,left,rounded=true,wd=9cm]{formul}
Let $E/K$ elliptic curve and consider $\infty$ to be an extra point. Set\pause

\centerline{\begin{beamercolorbox}[shadow=true,center,rounded=true,wd=9cm]{postit}
$$E(K)=\{(x,y)\in K^2:\ y^2+=x^3+ax+b\}\cup\{\infty\}\subseteq K^2\cup\{\infty\}$$\end{beamercolorbox}}

 
%\end{beamercolorbox}}
\pause

\ \hfill$\infty$ might be though as the ``vertical direction''

\begin{Definition}[line through points $P,Q\in E(K)$]
$r_{P,Q}:\begin{cases}
                     \text{line through $P$ and }Q &\text{if }P\neq Q\\
                     \text{tangent line to $E$ at }P &\text{if }P=Q
                    \end{cases}$\hfill projective or affine
\end{Definition}\pause

\begin{itemize}[<+-| alert@+>]
\item if $\#(r_{P,Q}\cap E(K))\ge2\ \Rightarrow\ \#(r_{P,Q}\cap E(K))=3$\\
\hfill\scriptsize{\alert{if tangent line, contact point is counted with multiplicity}}  
 \item $r_{P,Q}: aX+b=0$ (vertical) $\Rightarrow \infty\in r_{P,Q}$
\item $r_{\infty,\infty}\cap E(K)=\{\infty,\infty,\infty\}$%\vspace*{-4.4pt}
 % $\#(r_{P_1,P_1}\cap E(K))=2$
% \vspace*{-4.4pt}
\end{itemize}

\end{frame}

\begin{frame}
\frametitle{History (from \textsc{Wikipedia})}

\begin{columns}[c]
\begin{column}{4.5cm}
\begin{small}
\textbf{Carl Gustav Jacob Jacobi} (10/12/1804 -- 18/02/1851) was a German mathematician,
who made fundamental contributions to elliptic functions, dynamics, differential equations,
and number theory.
\end{small}\\
\centerline{\includegraphics[width=1.8cm]{images/Jacobi.jpg}}
%\centerline{\scriptsize{Carl Gustav Jacob Jacobi}}\\
\begin{scriptsize}\begin{block}{Some of His Achievements:}
\begin{itemize}
 \item Theta and elliptic function
 \item Hamilton Jacobi Theory
 \item Inventor of determinants
 \item Jacobi Identity\\
 \tiny{ $[A,[B,C]]+[B,[C,A]]+[C,[A,B]]=0$}
\end{itemize}
\end{block}\end{scriptsize}
\end{column}\pause
\begin{column}{5.5cm}\vspace*{-16.3pt}
\begin{center}
\includegraphics[width=5.5cm]{images/add1.pdf}\pause
\llap{\includegraphics[width=5.5cm]{images/add2.pdf}}\pause
\llap{\includegraphics[width=5.5cm]{images/add3.pdf}}\pause
\llap{\includegraphics[width=5.5cm]{images/add5.pdf}}\pause
\llap{\includegraphics[width=5.5cm]{images/add6.pdf}}\pause
\llap{\includegraphics[width=5.5cm]{images/add7.pdf}}\pause
\llap{\includegraphics[width=5.5cm]{images/add1.pdf}}\pause
\llap{\includegraphics[width=5.5cm]{images/add8.pdf}}\pause
\llap{\includegraphics[width=5.5cm]{images/add9.pdf}}\pause
\llap{\includegraphics[width=5.5cm]{images/ad10.pdf}}\pause
\llap{\includegraphics[width=5.5cm]{images/ad11.pdf}}\pause
\llap{\includegraphics[width=5.5cm]{images/ad12.pdf}}\pause
\llap{\includegraphics[width=5.5cm]{images/add1.pdf}}\pause
\llap{\includegraphics[width=5.5cm]{images/ad13.pdf}}\pause
\llap{\includegraphics[width=5.5cm]{images/ad14.pdf}}\pause
\llap{\includegraphics[width=5.5cm]{images/ad15.pdf}}\pause
\llap{\includegraphics[width=5.5cm]{images/add7.pdf}}\pause
\end{center}
\small{
$r_{P,Q}\cap E(K)=\{P,Q,R\}$\\
$r_{R,\infty}\cap E(K)=\{\infty,R,R'\}$}
\centerline{\begin{beamercolorbox}[shadow=true,center,rounded=true,wd=2cm]{formul}
$P+_E Q:=R'$\pause
            \end{beamercolorbox}}%\smallskip

 \small{$r_{P,\infty}\cap E(K)=\{P,\infty,P'\}$}\\
 \centerline{\begin{beamercolorbox}[shadow=true,center,rounded=true,wd=2cm]{formul}
             $-P:=P'$
            \end{beamercolorbox}}

\end{column}
\end{columns}
\end{frame}

\begin{frame}
\frametitle{Properties of the operation ``$+_E$''}

\begin{Theorem}
 The addition law on $E(K)$ has the following
properties:
\begin{enumerate}[<+-| alert@+>][(a)]
 \item $P+_EQ\in E(K)\hfill\forall P,Q\in E(K)$
 \item  $P+_E\infty=\infty+_E P=P\hfill\forall P\in E(K)$
 \item  $P+_E(-P)=\infty\hfill\forall P\in E(K)$
 \item  $P+_E(Q +_E R)=(P+_E Q)+_E R\hfill\forall P,Q,R\in E(K)$
 \item  $P+_E Q=Q +_E P\hfill\forall P,Q\in E(K)$
\end{enumerate}
 \end{Theorem}\pause

\begin{itemize}[<+-| alert@+>]
%  \item By ``a point of $E/K$ ($P\in E$)'' we mean $P\in E(\bar{\F}_q)$
%  in analogy for $E/\Q$ where ``a point of $E$'' means  $P\in E(\C)$
 \item $\left(E(K),+_E\right)$  \alert{commutative group}
 \item All group properties are easy except \alert{associative law (d)}
 \item Geometric proof of associativity uses \emph{Pappo's Theorem}
%  \item We shall comment on how to do it by explicit computation
%  \item can substitute $K$ with any field $K$; Theorem holds for $\left(E(K),+_E\right)$
% \item In particular, if $E/K$, can consider the groups $E(\overline{\F}_q)$ or $E(\F_{q^n})$
\end{itemize}
\end{frame}

\begin{frame}
\frametitle{Formulas for Addition on $E$}
\centerline{\begin{beamercolorbox}[shadow=true,center,rounded=true,wd=6cm]{formul}
$E: y^2=x^3+Ax+B$\end{beamercolorbox}}\bigskip\pause

$P_1 = (x_1, y_1), P_2 = (x_2, y_2)\in E(K)\setminus\{\infty\}$,\pause

\begin{beamerboxesrounded}[upper=block title example,lower=block body alerted,shadow=true]{Addition Law}
\begin{itemize}[<+-| alert@+>]
 \item If $P_1\neq P_2$
\begin{itemize}[<+-| alert@+>]
 \item $x_1 \neq x_2$\\
\centerline{\begin{beamercolorbox}[shadow=true,center,wd=4cm]{postit}
             $\displaystyle{\lambda=\frac{y_2-y_1}{x_2-x_1}\qquad \nu=\frac{y_1x_2-y_2x_1}{x_2-x_1}}$
            \end{beamercolorbox}}
 \item $x_1 = x_2\ \Rightarrow\ P_1 +_E P_2 = \infty$
            \end{itemize}
\item If $P_1 = P_2$
\begin{itemize}[<+-| alert@+>]
\item $y_1\neq 0$\\
\centerline{\begin{beamercolorbox}[shadow=true,center,wd=5cm]{postit}
$\displaystyle{\lambda=\frac{3x_1^2+A}{2y_1}, \nu=-\frac{x_1^3-Ax_1-2B}{2y_1}}$
            \end{beamercolorbox}}

 \item $y_1 = 0\ \Rightarrow P_1 +_E P_2 = 2P_1 = \infty$
\end{itemize}
 \end{itemize}\pause

Then

\centerline{\begin{beamercolorbox}[shadow=true,center,rounded=true,wd=7cm]{formul}
{ $P_1 +_E P_2 = ({\color[cmyk]{0,1,1,0.5}\lambda^2-x_1-x_2},
{\color[cmyk]{1,0,1,0.5}-\lambda^3+\lambda(x_1+x_2)-\nu})$}
            \end{beamercolorbox}}
\end{beamerboxesrounded}

\end{frame}



\begin{frame}
 \frametitle{Elliptic curves over $\C$ and over $\R$}

  \centerline{\begin{beamercolorbox}[shadow=true,center,rounded=true,wd=6cm]{formul}
$$E(\C)\cong \R/\Z\oplus\R/\Z$$
It is a compact Rieman surface of genus $1$
            \end{beamercolorbox}}\pause

\centerline{\includegraphics[width=2.8cm]{images/torus.jpg}}\pause

  \centerline{\begin{beamercolorbox}[shadow=true,center,rounded=true,wd=6cm]{formul}
$$E(\R)\cong\begin{cases}
             \R/\Z\\ \R/\Z\oplus\{\pm1\} 
            \end{cases}
 $$
It is a circle or two circles
            \end{beamercolorbox}}\pause

 
\end{frame}

\begin{frame}
 \frametitle{Elliptic curves over $\Q$}
 
 \begin{teo}[Mordell Theorem]
If $E/\Q$ is an elliptic curve, then $\exists r\in\N$ and $G$ and finite abelian group $G$ 
such that
$$E(\Q)\cong\Z^r\oplus G.$$
In other words, $E(\Q)$ is finitely generated.
\end{teo}\bigskip\pause

 \begin{teo}[Mazur Torsion Theorem]
 If $\Z/n\Z$ denotes the cyclic group of order $n$, then the possible torsion subgroups 
 $$G=\operatorname{Tor}(E(\Q))\cong\begin{cases}
 \Z/n\Z\quad \text{with}\ 1 \le n \le 10\\        
 \Z/12\Z\\
 \Z/2\Z\oplus\Z/2n\Z\quad \text{with}\ 1 \le n \le 4.
         \end{cases}$$
 \end{teo}\pause
 
 It is not known if $r$ (the rank of $E$) is bounded. 

 \end{frame}

\begin{frame}
 \frametitle{Elliptic curves over $\F_p$}

\begin{teo}
 \centerline{\begin{beamercolorbox}[shadow=true,center,rounded=true,wd=6cm]{formul}
$$E(\F_p)\cong \Z/n\Z\oplus\Z/nk\Z\qquad\exists n,k\in\N^{>0}$$
            \end{beamercolorbox}}\pause

            (i.e. $E(\F_p)$ is either cyclic ($n=1$) or the product of $2$ cyclic groups) 
\end{teo}\pause

\begin{teo}[Weil]
 $$n\mid p-1$$
\end{teo}\pause
            
\begin{teo}[Hasse]
Let $E$ be an elliptic curve over the finite field $\F_p$. Then the order of $E(\F_p)$
satisfies
$$\left|p+1-\#E(\F_p)\right|\le 2\sqrt p.$$
\end{teo} 
 \end{frame}

\begin{frame}
 \frametitle{From Elliptic curves over $\Q$ to Elliptic curves over $\F_p$}
 
If $E/\Q$ then $\exists a,b\in\Z$ s.t.: 
 \centerline{\begin{beamercolorbox}[shadow=true,center,rounded=true,wd=6cm]{formul}
$$E: y^2=x^3+ax+b$$
            \end{beamercolorbox}}\pause

For all primes $p\nmid 4a^3+27b^2$, we can consider \emph{the reduces curve} $\bar{E}/\F_p$:
$$\bar{E}: y^2=x^3+\bar{a}x+\bar{b}.$$
where $\bar{a}=a\bmod p$ and $\bar{b}=b\bmod p$.\pause

Given a certain property $\mathbb P$ ``defined on finite groups'', we consider

 \centerline{\begin{beamercolorbox}[shadow=true,center,rounded=true,wd=6cm]{formul}
$$\pi_E(x,\mathbb P)=\#\{p\le x: \bar{E}(\F_p)\text{ satisfies }\mathbb P\}.$$
            \end{beamercolorbox}}\pause

We are interested in studying the behaviour of $\pi_E(x,\mathbb P)$ and $x\rightarrow\infty$ for 
various properties $\mathbb P$.
\end{frame}

\begin{frame}
\frametitle{Serre's Cyclicity Conjecture} 

\begin{Theorem}[Serre's Cyclicity Conjecture under the Riemann Hypothesis (1976)]
Let $E/\Q$ be an elliptic curve and assume GRH
%for the division fields
%$\Q(E[m])$. 
Then $\exists \gamma_{E,P}\in\R^{\ge0}$ s.t.%, as $x\rightarrow\infty$,
$$\#\{p\le x:  \bar{E}(\F_p)\text{ is cyclic}\}\sim\gamma_{E,P}\frac{x}{\log x}\quad\text{ as }x\rightarrow\infty$$
\end{Theorem}\pause

%\begin{itemize}[<+-|alert@+>]
% \item If $E[2]\subset E(\Q)$ then $\gamma_{E,P}=0$
% \item 1983: Ram Murty eliminated GRH for the analogue on CM curves
% \item 1990: Rajiv Gupta \& Ram Murty: if $E[2] \nsubseteq E(\Q)$, $\#\{p\le x: p\nmid\Delta_E, E(\F_p)\text{ is cyclic}\}\gg_E\frac{\pi(x)}{\log x}$ 
% \item 2004: Alina Cojocaru \& Ram Murty: significant improvements on the error terms both unconditional and on GRH
% \end{itemize}

\frametitle{Lang Trotter Conjecture for \textit{primitive points}}
 
\begin{conj}[Lang--Trotter primitive points Conjecture (1977)] Let $E/\Q$, $P\in E(\Q)$ with infinite order.
$\exists \alpha_{E,P}\in\R^{\ge0}$ s.t.
$$\#\{p\le x:  \bar{E}(\F_p)=\langle P\bmod p\rangle\}\sim\alpha_{E,P}\frac{x}{\log x}\quad\text{ as }x\rightarrow\infty$$
\end{conj}\pause

\begin{itemize}[<+-|alert@+>]
\item[] For most of the $E$'s:
\item If $C=\prod_\ell\left(1-\frac1{\ell(\ell-1)^2(\ell+1)}\right)= 0.81375190610681571\cdots$, then $\gamma_{E,P}=q\cdot C$ with $q\in\Q^{\ge0}$ 
\item If $B=\prod_\ell\left(
1-\frac{\ell^3-\ell-1}{\ell^2(\ell-1)^2(\ell+1)}\right)=0.440147366792057866\cdots$, then  $\alpha_{E,P}=q'\cdot B$ with $q'\in\Q^{\ge0}$ 
\item It is possible that $\alpha_{E,P}=0$ or that $\gamma_{E,P}=0$
\item $\gamma_{E,P}=0\ \Longleftrightarrow\ \Z/2\Z\oplus\Z/2\Z\subseteq E(\Q)$
\item if $P=kQ$, $Q\in E(\Q)$ and $d=\gcd(k,\#\operatorname{Tor}(E(\Q))>1$, then $\alpha_{E,P}=0$
%\ \hfill (since $\operatorname{ord}P\mid \frac{\#\bar{E}(\F_p)}{d}$)
\end{itemize}
\end{frame}

\begin{frame}
\frametitle{Comparison between empirical data in Serre's Conjecture and Lang--Trotter Conjecture}
\framesubtitle{Tests on Curves of rank 1, no torsion, Galois surjective $\forall\ell$}
% \centerline{\begin{beamercolorbox}[shadow=true,center,rounded=true,wd=9cm]{postit}
% $\displaystyle{A=0.373955813619202288054728054346416415111\cdots}$
% \end{beamercolorbox}}\pause\smallskip
% 
% \centerline{\begin{beamercolorbox}[shadow=true,center,rounded=true,wd=6cm]{formul}
% $\displaystyle{\pi_q(x)=\#\{p\le x: \langle q\bmod p\rangle=\mathbb F_p^*\}}$
% \end{beamercolorbox}}\pause\smallskip
% 
% \begin{tiny}
% \begin{center}
% \begin{tabular}{|l|l|r|}
% \hline
% $q$& $\pi_q(2^{25})/\pi(2^{25})$ & $A-\pi_q(2^{25})/\pi(2^{25})$\\
% \hline         
%   2&$0.37395508\cdots$&   $0.0000007\cdots$\\
%   3&$0.37388094\cdots$&   $0.0000748\cdots$\\
%   7&$0.37409997\cdots$&  $-0.0001441\cdots$\\
%  11&$0.37422450\cdots$&  $-0.0002686\cdots$\\
%  19&$0.37400887\cdots$&  $-0.0000530\cdots$\\
%  23&$0.37402147\cdots$&  $-0.0000656\cdots$\\
%  31&$0.37422208\cdots$&  $-0.0002662\cdots$\\
% \hline
% \end{tabular}\end{center}
% \end{tiny}\pause
%
%\centerline{\begin{beamercolorbox}[shadow=true,center,rounded=true,wd=9cm]{postit}
%$\displaystyle{B=0.4401473667920578662600197830604687798836\cdots}$
%\end{beamercolorbox}}\pause\smallskip

\centerline{\begin{beamercolorbox}[shadow=true,center,rounded=true,wd=11.5cm]{formul}
$\displaystyle{\pi_P(x)=\#\{p\le x: \langle P\bmod p\rangle=\bar{E}(\mathbb F_p^*)\}
\qquad \pi_{\text{cycl}}(x)=\#\{p\le x: \bar{E}(\mathbb F_p^*)\ \text{is cyclic}\}}$
\end{beamercolorbox}}\pause\smallskip

\begin{center}
\begin{tabular}{|l|l|r|}
\hline
\hspace*{-1mm} label \hspace*{-3mm}&\!\! $\frac{\pi_P(2^{25})}{\pi(2^{25})}$\!\!&\!\! $B-\frac{\pi_P(2^{25})}{\pi(2^{25})}$\!\!\\
\hline
37.a1 &$0.44017485\cdots$&   $-0.000027\cdots$\\
43.a1 &$0.44034784\cdots$&   $-0.000200\cdots$\\
53.a1 &$0.44020198\cdots$&   $-0.000054\cdots$\\
57.a1 &$0.44016176\cdots$&   $-0.000014\cdots$\\
58.a1 &$0.44012203\cdots$&    $0.000025\cdots$\\
61.a1 &$0.44034299\cdots$&   $-0.000195\cdots$\\                       
77.a1 &$0.43964812\cdots$&    $0.000499\cdots$\\
79.a1 &$0.44043021\cdots$&  $ -0.000282\cdots$\\
\hline
\end{tabular}\pause

\begin{tabular}{|l|l|r|}
\hline
\hspace*{-1mm} label \hspace*{-3mm}&\!\! $\frac{\pi_{cycl}(2^{25})}{\pi(2^{25})}$\!\!&\!\! $C-\frac{\pi_{cycl}(2^{25})}{\pi(2^{25})}$\!\!\\
\hline
37.a1 &$0.81383047\cdots$&   $-0.000078\cdots$\\
43.a1 &$0.81363907\cdots$&   $ 0.000112\cdots$\\
53.a1 &$0.81389250\cdots$&   $-0.000140\cdots$\\
57.a1 &$0.81387263\cdots$&   $-0.000120\cdots$\\
58.a1 &$0.81374131\cdots$&   $ 0.000010\cdots$\\
61.a1 &$0.81397584\cdots$&   $-0.000223\cdots$\\                       
77.a1 &$0.81380285\cdots$&   $-0.000050\cdots$\\
79.a1 &$0.81392157\cdots$&   $-0.000169\cdots$\\\hline
\end{tabular}
\end{center}
\end{frame}

\begin{frame}
 \frametitle{The notion of never primitive point}
 
 \begin{Defi} Let $E/\Q$ be an elliptic curve such that  $\Z/2\Z\oplus\Z/2\Z\nsubseteq E(\Q)$. A point $P\in E(\Q)$ is called a 
\textbf{never primitive} if
\begin{itemize}
 \item $P$ has infinite order 
 \item for all $\ell\mid\#\operatorname{Tor}(E(\Q))$, $P$ is not the $\ell$--th power of a rational point $Q\in E(\Q)$
 \item $\langle P\bmod p\rangle \neq \bar{E}(\F_p)$ for all $p$ large enough
\end{itemize}
\end{Defi}\pause

\begin{itemize}[<+-|alert@+>]
 \item Hence, given $p$, a \emph{primitive point} $P$ modulo $p$ satisfies
$\langle P\bmod p\rangle = \bar{E}(\F_p).$
\item A \textbf{never primitive} point never satisfies the above
\item if $\Z/2\Z\oplus\Z/2\Z\subseteq E(\Q)$, no point is ever primitive since $\bar{E}(\F_p)$ is never cyclic
\item[] we avoid such obvious cases
\item we are interested in examples of curves with \textbf{never primitive points}
\end{itemize}


\end{frame}


\begin{frame}
 \frametitle{Twists with a Never Primitive point}

\begin{Defi} Given an elliptic curve $E/\Q$ with Weierstra\ss\ equation 
$$E: y^2=x^3+Ax^2+Bx+C$$
and $D\in\Q^*$, the \textbf{twisted curve $E_D$} of $E$ by $D$ is
$$E_D: y^2=x^3+ADx^2+BD^2x+CD^3.$$
\end{Defi}
 
 
 \begin{teo}
Let $E/\Q$ be an elliptic curve such that $E(\Q)$ contains a point of order $2$. \\
There $\exists \infty D\in\Z$ s.t. 
the twisted curve $E_D$ is such that $E_D(\Q)$ contains a never primitive point.
\end{teo}

% \begin{proof}
Every elliptic curve with a point of order $2$ can be written in the form:
$$E: y^2=x^3+ax^2+bx\quad\text{with }a^2-4b\neq0$$ 
Set $D=s (a s+2)\left(1-bs^2\right)$. Then, $\forall s\in\Q$ except possibly when $D$ is a perfect square,
$$P_D\left(\left(1-bs^2\right)^2,\left(a s+1+bs^2\right)\left(b-s^2\right)^2\right)\in E_D(\Q)\quad\textbf{is never primitive}.$$

%be an ellptic curve over $\Q$. We have that $(0,0)\in\operatorname{Tor}(E(\Q))$. We shall also
%assume that $$ is not a perfect square 
% so that 
% $$E[2]=\left\{\infty,(0,0),\left(\frac{-a+\sqrt{\delta}}{2},0\right),\left(\frac{-a-\sqrt{\delta}}{2},0\right)\right\}\nsubseteq E(\Q).$$ 
% 
% Let us consider the twists $E_D: y^2=x^3+aDx^2+bD^2$ for which there $t\in\Q$ with 
% $P_D(1,t)\in E_D(\Q)$. Then $t^2=1+aD+bD^2$ and if we make the substitutions:
% $D\to \frac{X-ab}{2 b}, t\to \frac{Y}{2}$ and we obtain
% $$X^2-b Y^2=\delta$$
% It is immediate to check that the above has as solution 
% $(a,2)\in\Q^2$ and that all the rational solutions can be parameterized by
% $$
% \begin{cases}
% X=&\displaystyle a\times\frac{1+bs^2}{1-bs^2}+2\times\frac{2 s}{1-bs^2} \\
% \\
% Y=&\displaystyle a\times\frac{2 s}{1-bs^2}+2\times\frac{1+bs^2}{1-bs^2}.
% \end{cases}
% $$
% Hence
% $$
% D=\displaystyle\frac{s (a s+2)}{\left(1-bs^2\right)}\qquad t=\displaystyle\frac{\left(a s+1+bs^2\right)}{1-bs^2}.
% $$
% Changing coordinates ($(x,y)\longleftarrow(x/\left(1-bs^2\right)^2,y/\left(1-bs^2\right)^3)$, we obtain
% The setminus
% \begin{eqnarray*}
% \frac12P_D&=&\left\{\left(-s(1-bs^2)(a + 2 b s\pm \sqrt{\delta(1-bs^2)},\right)\right.\\
% & &\left.\left((as+1)(1-bs^2)(-1\pm\sqrt{1-bs^2}),\pm(as+1)^2\sqrt{(1-bs^2)^3}(1\pm\sqrt{1-bs^2})\right)\right\}.
% \end{eqnarray*}
% We can apply Theorem\ref{criterion} with $\ell=2$, $\alpha=\delta$ and $\beta= (1-bs^2)$.
% \end{proof}

 \end{frame}

 \begin{frame}
\frametitle{Other parametric families of curves with a never primitive point}

\begin{teo}[1 - Jones, Pappalardi]
Let $s\in\Q\setminus\{\pm1\}$ and let
$$E_s: y^2=x^3-27(s^2-1)^2.$$\pause
Then 
\begin{itemize}
 \item $P_s(s^2+3,s(s^2-9))\in E(\Q)\setminus\operatorname{Tor}(E(\Q))$
 \item $\operatorname{Tors}(E_s(\Q))$ is trivial %for all but finitely many $s$
 \item $P_s$ is a \emph{never--primitive} point %for all but finitely many $s$
\end{itemize}
\end{teo}\pause


\begin{teo}[2 - Jones, Pappalardi] Let $s\in\Q\setminus\{0,\pm3,\pm\frac13\}$, and let 
$$E_s: y^2=x^3-3s^2(s^2-8)x-2s^2(s^4-12s^2+24).$$\pause
Then 
\begin{itemize}
 \item $P_s(2s^2+1,9s^2-1)\in E(\Q)\setminus\operatorname{Tor}(E(\Q))$
 \item $\operatorname{Tors}(E_s(\Q))$ is trivial %for all but finitely many $s$
 \item $P_s$ is a \emph{never--primitive} point %for all but finitely many $s$
\end{itemize}
\end{teo}

% ha discriminante and $j$--invariante rispettivamente:
% $
% \Delta_{E_s}= -2^8 3^3 s^4 (s^2-9)\qquad j_{E_s}=3^3s^2 \left(\frac{s^2-8}{s^2-9}\right)^3.
% $
\end{frame}

\begin{frame}
 \frametitle{Galois Action on the root sets}
 
 The construction and its proof is based on the study of the Galois Action on the root-sets of $P$:\pause

 \begin{Defi} Given $E/Q$, $P\in E(\Q)$ and $n\in\N$. 
 $$E[n]:=\{Q\in\C: nQ=\infty\}$$
 and
 $$\frac1nP:=\{Q\in\C: nQ=P\}$$
  \end{Defi}\pause

  \begin{rem} Note that
  \begin{itemize}[<+-|alert@+>]
   \item $E[n]$ is an abelian group
   \item $E[n]\cong \Z/n\Z\oplus \Z/n\Z$
   \item if $R\in E[n]$ and $S\in \frac1nP$, then
   $R+S\in\frac1nP$
   \item $\frac1nP$ is a $\Z/n\Z$--affine space.
   \item $\operatorname{Gal}(\Q(E[n])/\Q)\subset\operatorname{Aut}(E[n])\cong\operatorname{GL}_2(\Z/n\Z)$
   \item $\operatorname{Gal}(\Q(\frac{1}{n}P)/\Q)\subset\operatorname{Aff}(\frac{1}n{P})\cong\operatorname{GL}_2(\Z/n\Z)\ltimes \Z/n\Z$
   \item To verify the Theorems one needs to compute the above Galois Groups for each elements of the family under consideration
   \end{itemize}
  \end{rem}
 \end{frame}




\begin{frame}
 \frametitle{A ``never primitivity'' criterion}
  
 \begin{lem}[1]
 Let $E/\Q$ be an elliptic curve, $P\in E(\Q)\setminus\operatorname{Tor}(E(\Q))$ and $\ell\ge3$ be a prime such that
 \begin{itemize}
 \item $P$ is not an $\ell$-th power of a point in $E(\Q)$\medskip
  \item $\Q(E[\ell])=\Q(\zeta_\ell,\alpha^{1/\ell}),\qquad \exists \alpha\in\Q^*$\medskip
  \item $\Q(\frac1\ell P)\cap\R=\{Q_1,\ldots,Q_\ell\}$\medskip
  \item $\Q(Q_i)=\Q((\alpha^i\beta)^{1/\ell}), i=1,\ldots,\ell,\qquad \exists \beta\in\Q^*$.\medskip
 \end{itemize}
Then $\Q(\frac1\ell P)=\Q(\zeta_\ell,\alpha^{1/\ell},\beta^{1/\ell})$ and $P$ is \textbf{never primitive}.
 \end{lem}\pause
 
The proofs of both Theorems use the previous Lemma with $\ell=3$
 
 \frametitle{Idea of the proof of Theorem 2}

\begin{lem}[2] Let $s\in\Z\setminus\{0,\pm1,\pm3,\pm13\}$ and consider $E_s$, the elliptic curve in Theorem 2. Let $\alpha=\sqrt[3]{s(s^2-9)}$ and set
$\displaystyle T\left(\frac{1}{3}(s^2+4s\alpha+\alpha^2),\frac{4}{3}(\alpha^3+s\alpha^2+s^2\alpha)\right)\in E_s(\C).$\pause
 Then\\
\centerline{$\displaystyle{E_s[3]:=\left\{\infty,(-s^2,\pm4\sqrt{-3}s)\right\}
\cup\left\{\pm T,\pm T^\sigma, \pm T^\sigma\right\}}$}\pause

where $\sigma\in\operatorname{Gal}(\overline{\Q}/\Q(\sqrt{-3})$) is such that $\sigma(\sqrt[3]d)=e^{2\pi i/3}\sqrt[3]d$
 $\forall d\in\Q$. Hence\\
\centerline{$\displaystyle\Q(E_s[3])=\Q(e^{2\pi i/3},\sqrt[3]{s(s^2-9)}).$}
\end{lem}

% \begin{Note}
%  Se $\ell$ \`e un primo tale che\\
% \centerline{$S_\ell=G_\ell$}
% allora per ogni primo $p\nmid\Delta'_E$,\\
% \centerline{$\ell\mid[E(\F_p):\langle P\bmod p\rangle].$} 
% \end{Note}\pause

\end{frame}


\begin{frame}
\frametitle{Idea of the proof of Theorem 2}

\begin{lem}[3] Let $s\in\Z\setminus\{0,\pm1,\pm3,\pm13\}$ and consider $E_s$, the elliptic curve in Theorem 2. Set %$\alpha=\sqrt[3]{s(s^2-9)}$,
$$
\beta=\sqrt[3]{s^2(s+3)},\qquad 
\gamma=\sqrt[3]{s^2(s-3)}=\frac{\alpha\beta^2}{s(s+3)},\qquad  
\delta=\sqrt[3]{(s-3)^2(s+3)}=\frac{\alpha^2\beta^2}{s^2(s+3)} $$\pause
and $P_\gamma(x_\gamma,y_\gamma), P_\beta(x_\beta,y_\beta), P_\delta(x_\delta,y_\delta)$ dove 
\begin{eqnarray*}
x_\beta=s(3s-8)+4(s-1)\beta+4\beta^2,& &y_\beta=4(s(3-s)(1-3s)-s(7-3s)\beta-(4-3s)\beta^2)\\
x_\gamma= s(3s+8)+4(s+1)\gamma+4\gamma^2,& &y_\gamma= 4(s(3+s)(1+3s)+s(7+3s)\gamma+(4 + 3 s)\gamma^2)\\
x_\delta=3+(s+1)\delta+\frac{s-1}{s-3}\delta^2,& &y_\delta=s^2-9+(s-3)\delta+\frac{s+3}{s-3}\delta^2.
\end{eqnarray*}\pause
Then $\Q(P_\gamma)=\Q(\gamma)$, $\Q(P_\beta)=\Q(\beta)$, $\Q(P_\delta)=\Q(\beta)$ and\\
\centerline{$\frac13P=\left\{
P_\beta,P_\beta^\sigma,P_\beta^{\sigma^2},
P_\gamma,P_\gamma^\sigma,P_\gamma^{\sigma^2},
P_\delta,P_\delta^\sigma,P_\delta^{\sigma^2}\right\}.$}\pause
Hence
$$\Q(E_s[3],\frac13P)=\Q(e^{2\pi i/3},\sqrt[3]{s(s^2-9)},\sqrt[3]{s^2(s-3)}).$$
\end{lem}\pause

The result follows from the previous lemmas
\end{frame}
\end{document}

