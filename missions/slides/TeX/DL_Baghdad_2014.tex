\documentclass[landscape,display]{powersem}%
\usepackage{fancybox,marvosym,graphicx,amsmath,amssymb,pifont}
\usepackage[bookmarksopen,colorlinks,urlcolor=red,pdfpagemode=FullScreen]{hyperref}
\usepackage{fixseminar}
\usepackage{color}
\usepackage[latin1]{inputenc}
\usepackage[coloremph,colormath,colorhighlight,lightbackground]{texpower}

\hfuzz=30pt \vfuzz=30pt \setlength{\slidewidth}{25cm}
\setlength{\slideheight}{17.5cm} \slideframe{}
\def\slideitemsep{.5ex plus .3ex minus .2ex}
\renewcommand{\slidetopmargin}{10mm}
\renewcommand{\slidebottommargin}{15mm}
\renewcommand{\slideleftmargin}{5mm}
\renewcommand{\sliderightmargin}{5mm}
\newcommand{\heading}[1]{%
 \begin{center}
  \large\bf
  \shadowbox{{\textcolor{conceptcolor}{#1}}}%
 \end{center}
 \vspace{1ex minus 1ex}}
\backgroundstyle[startcolor=white,
                   endcolor=grey,%firstgradprogression=3,
            rightpanelwidth=-7\semcm,,rightpanelcolor=pagecolor]{hgradient}%
%%%%%%%%%%%%% DATI DEL SEMINARIO IN QUESTIONE %%%%%%%%%%%%

\newcommand{\Ccal}{{\mathcal C}}
\newcommand{\N}{{\mathbb N}}
\newcommand{\F}{{\mathbb F}}
\newcommand{\Z}{{\mathbb Z}}
\definecolor{verdescu}{rgb}{0,0.6,0.6}
\definecolor{rossoscu}{rgb}{1,0,0.2}

\newcommand{\manorossa}{\textcolor{conceptcolor}{\ding{43}}}
\newcommand{\matitablu}{\textcolor{altemcolor}{\ding{46}}}
\newcommand{\verde}{\textcolor{black}}
\newpagestyle{327}%
 {\textcolor{codecolor}{\textit{Public Key Cryptography}} \hspace{\fill}\rightmark
\hspace{1cm}\thepage}
 {\includegraphics[width=4mm]{images/dipmat.pdf}\hspace{\fill}\textcolor{codecolor}{\sc Universit\`a Roma Tre}
 \hspace{\fill}\includegraphics[width=5mm]{images/roma3.pdf}}%%
\pagestyle{327} \markright{\textcolor{conceptcolor}{College for Women, Baghdad University}}

%\title{\textcolor{red}{Finite fields, Permutation Polynomials. Computational aspects with
%applications to public key cryptography}}
%\author{\textcolor{green}{Francesco Pappalardi}}
%\date{ Dhahran, Saudi Arabia\\
%Workshop on Industrial Mathematics\\
%\textcolor{red}{Dhahran February 29, 2004}}


\begin{document}
\begin{slide}\pagestyle{empty}
\addtocounter{slide}{-1}
\includegraphics[width=1.3cm]{images/crypto.jpg}\ \hfill \includegraphics[width=1.3cm]{images/crypto.jpg}
\vfil

\begin{sc}\begin{center}
\small{
\textbf{Lecture in Advanced Number Theory}\\ \emph{College of Science for Women}\\ Baghdad University}

April 1, 2014
\vspace*{1cm}

\begin{Large}
\textcolor{underlcolor}{Finite fields, Permutation Polynomials.
Computational aspects with applications to public key
cryptography}
\end{Large}
\vfill
Francesco Pappalardi
\end{center}
\end{sc}

\vfill

\includegraphics[width=1.3cm]{images/crypto.jpg}\ \hfill \includegraphics[width=1.3cm]{images/crypto.jpg}
\end{slide}

\begin{slide}\pageTransitionWipe{30}

\heading{\verde{\textbf{Private key}} versus \textbf{Public Key}
}\pause\medskip

\centerline{\includegraphics[width=8cm]{images/figure2.pdf}}

\end{slide}

\begin{slide}\pageTransitionWipe{30}\addtocounter{slide}{-1}

\heading{\textbf{Private key} versus \verde{\textbf{Public
Key}}}\medskip

\centerline{\includegraphics[width=8cm]{images/figure3.pdf}}

\end{slide}


\begin{slide}\pageTransitionWipe{30}

\heading{Classical General Examples of PKC}\bigskip

\parstepwise{\begin{itemize}
  \item[\textcolor{blue}{\ding{182}}] \step{(1976) Diffie Hellmann Key exchange protocol}
   \step{\emph{IEEE Trans. Information Theory IT-22 (1976)}}
  \item[\textcolor{blue}{\ding{183}}]  \step{(1983) Massey Omura Cryptosystem}\\
\step{\emph{Proc. $4^{th}$ Benelux Symposium on Information Theory (1983)}}
  \item[\textcolor{blue}{\ding{184}}]  \step{(1984) ElGamal Cryptosystem}
  \step{\emph{IEEE Trans. Information Theory IT-31 (1985)}}
\end{itemize}}\pause

\centerline{\includegraphics[width=6cm]{images/image033.jpg}}
\end{slide}


\begin{slide}\pageTransitionWipe{30}
\heading{Diffie--Hellmann key exchange 1/5}

\centerline{\includegraphics[width=9.4cm]{images/Satellite.jpg}}

\end{slide}

\begin{slide}\pageTransitionWipe{30}
\heading{Diffie--Hellmann key exchange 2/5}\pause
\parstepwise{
\begin{itemize}
\item[\textcolor{blue}{\ding{182}}] \step{\textbf{Alice} and
\textbf{Bob} agree on a prime $p$ and a \textit{\underline{generator}} $g$ in
$\Z/p\Z$}
\item[\textcolor{blue}{\ding{183}}] \step{\textbf{Alice}
picks a \textcolor{red}{secret} $a$,} \step{$0\leq a\leq p-1$}
\item[\textcolor{blue}{\ding{184}}] \step{\textbf{Bob} picks
a \textcolor{red}{secret} $b$, $0\leq b\leq p-1$}
\item[\textcolor{blue}{\ding{185}}] \step{They compute and publish $g^a\bmod p$
(\textbf{Alice}) and $g^b\bmod p$ (\textbf{Bob})}
\item[\textcolor{blue}{\ding{186}}] \step{The common
\textcolor{red}{secret} key is $g^{ab}\bmod p$}
\end{itemize}}\pause

{\parstepwise{\includegraphics[width=7cm]{images/dh.jpg} \step{\emph{what
is a \underline{generator} of $\Z/p\Z$?}}}}

\end{slide}

\begin{slide}\pageTransitionWipe{30}
\heading{Diffie--Hellmann key exchange 3/5}

\parstepwise{
\begin{itemize}
\item[\textcolor{red}{\ding{43}}] \step{A generator (or primitive
root) $g$ of a prime number $p$ is a number} \step{whose powers
mod $p$, generate $1,\ldots, p - 1$}
\item[\textcolor{red}{\ding{43}}] \step{So $g\bmod p,\ g^2\bmod p,
\ldots,\ g^{p-1}\bmod p$ are all distinct,} \step{i.e., a
permutation of $1$ through $p - 1$}
\item[\textcolor{red}{\ding{43}}] \step{In other words: for all
$b\in\Z/p\Z, b\neq0$,} \step{there exists an exponent
$i\in\{0,1,\ldots,p-1\}$ such that $b=g^i\bmod p$}
\item[\textcolor{red}{\ding{43}}] \step{Given $b\in\Z$, exponent
$i$ above is} \step{the \emph{discrete logarithm} of $b$ for base
$g$ $\bmod p$} \item[\textcolor{red}{\ding{43}}] \step{Computing
discrete logs appears infeasible in general}
\end{itemize}}
\end{slide}


\begin{slide}\pageTransitionWipe{30}
\heading{Diffie--Hellmann key exchange 4/5}

\centerline{\includegraphics[width=7cm]{images/dh.jpg}}\pause

\parstepwise{
\begin{itemize}
\item[\textcolor{blue}{\ding{43}}]
\step{\textbf{\textcolor{black}{Eve}}
knows $g^a$, $g^b$ but would like to compute $g^{ab}$};
\item[\textcolor{blue}{\ding{43}}]
\step{\textbf{\textcolor{black}{Eve}}
could compute a discrete logarithm to find $a$ and then $(g^b)^a$}
\item[\textcolor{blue}{\ding{43}}] \step{for given $\alpha,g,p$,
\textbf{\textcolor{black}{Eve}} should solve:}\end{itemize}}\pause
$$g^X\equiv\alpha\bmod p$$
\end{slide}


\begin{slide}\pageTransitionWipe{30}
\heading{Diffie--Hellmann key exchange 5/5}\pause

A ``criptographically meaningful size'' example:\pause

\begin{tiny}
$p=370273307460967425842481081357528298315386585184169353328410050632472746552261503118421027658$
  $\ \ \ \ \ 721711241508544733578984012456938357678209461867245573821426204444288523552318347549870943602$
  $\ \ \ \ \ 1902398769259658537444365842890327$
\pause $g=5$\pause

$\textcolor{red}{a=230884090203989538822791747965302672267956566803890984719811170401834881423535241039556153839}$
$\textcolor{red}{\ \ \ \ \
50300790706016512170324186640960442741350790022942149093292104570603304669117473786798985}$
$\textcolor{red}{\ \ \ \ \
00024210343154844771162635809902530822}$\pause

$\textcolor{red}{b=202628627712040976052737350793757540205242681192017941068774728007392912193775762330719406560}$
$\textcolor{red}{\ \ \ \ \
04093331116419046740605076855604279856790686813698840332610088778267557488150882421959663}$
$\textcolor{red}{\ \ \ \ \
70518057438047030854128879946541952289}$\pause

$\textcolor{blue}{5^a=249451424107893262892484442575689156622349940771024747733612460962310329209496530481469732410}$
$\textcolor{blue}{\ \ \ \ \ \ \
95957576012477323952872295620523253758143768040422343030840568653423985771858393578141665}$
$\textcolor{blue}{\ \ \ \ \ \ \
18479146351026737882783508710913577680}$\pause

$\textcolor{blue}{5^b=287293760357523957032946092556813694596882586743260552838382768832192594422702357607546631218}$
$\textcolor{blue}{\ \ \ \ \ \ \
64001485395789301444617793223201594706097398360331195161213836214741498824201098331045762}$
$\textcolor{blue}{\ \ \ \ \ \ \
16804562648795943563091024975401008295}$\pause\vspace{-2mm}

$\textcolor{black}{5^{ab}=36674172125349300306071275329964633749875664216293811088694156172838197865927916343627669411}$\vspace{-2.5mm}
$\textcolor{black}{\ \ \ \ \ \ \ \ \ 4396823489217444401038685650925971812733853762885262933444987558589066268362684366645128712}$\vspace{-2.5mm}
$\textcolor{black}{\ \ \ \ \ \ \ \ \ 2395082920958736911545732951584464496}$
\end{tiny}

\end{slide}

\begin{slide}\pageTransitionWipe{30}
\heading{Discrete Logarithms Computation}\pause

Some classical algorithms:\pause

\parstepwise{
\begin{itemize}
\item[\textcolor{red}{\ding{44}}] \step{Shanks \emph{baby-step,
giant step}} \step{\textit{Proc. $2^{nd}$ Manitoba Conf. Numerical
Mathematics (Winnipeg, 1972).}}

\item[\textcolor{red}{\ding{44}}] \step{Pohlig--Hellmann
Algorithm} \step{\textit{IEEE Trans. Information Theory IT-24
(1978).}}

\item[\textcolor{red}{\ding{44}}] \step{Index computation
algorithm}


\item[\textcolor{red}{\ding{44}}] \step{Sieving algorithms}
\step{\emph{La Macchia \& Odlyzko, Designs Codes and Cryptography 1 (1991)}}
\end{itemize}}\pause\bigskip


\textbf{NOTE:} The last two are "very special" for $\Z/p\Z$
\end{slide}


\begin{slide}\pageTransitionWipe{30}


\heading{Discrete Logarithms computation Records 1/2}

%Computing a discrete logarithm in $\Z/p\Z$, where $p$ a 90 digits prime ,

A. Joux et R. Lercier, 1998.\pause

\begin{tiny}
$\begin{array}{rcl}  p &=& \lfloor 10^{89} \pi \rfloor+ 156137\\
   &=& 314159265358979323846264338327950288419716939937510582097494459230781640628620899862959619,\\
   & &g = 2,  \end{array}$\pause

$\begin{array}{rcl}  y &=& \lfloor 10^{89} e \rfloor\\
   &=& 271828182845904523536028747135266249775724709369995957496696762772407663035354759457138217
   \end{array}$\pause
$$2^X\equiv y\bmod p$$\pause

$  y =
g^{1767138072114216962732048234071620272302057952449914157493844716677918658538374188101093},$\pause
$  y+1 =
g^{31160419870582697488207880919786823820449120001421617617058468654271221802926927230033421},$\pause
$  y+2 =
g^{308988329335044525333827764914501407237168034577534227927033783999866774252739278678837301},$\pause
$  y+3 =
g^{65806888002788380103712986883663253187183505405451188935055113209887949364255134815297846},$\pause
$  y+4 =
g^{40696010882128699199753165934604918894868490454360617887844587935353795462185105078977093}$\pause
\end{tiny}\medskip

\centerline{\emph{It took $4.5$ months... on a Pentium PRO 180 MHz }}
\end{slide}

\begin{slide}\pageTransitionWipe{30}


\heading{Discrete Logarithms computation Records 2/2}\pause

\textbf{A. Joux et R. Lercier} (CNRS / Ecole Polytechnique)\pause

\parstepwise{\begin{itemize}
\item[\textcolor{black}{\ding{172}}] \step{1999 $p\cong 10^{100}$\\}
\step{\texttt{500MHz
quadri-processors Dec Alpha Server}}\step{ -- 8.5 months;}\medskip

\item[\textcolor{black}{\ding{173}}] \step{2001 $p\cong 10^{110}$\\ }
\step{\texttt{525MHz quadri-processors Digital
Alpha Server 8400}}\step{ -- 20 days;}\medskip

\item[\textcolor{black}{\ding{174}}] \step{2001 $p\cong 10^{120}$;}
\step{\textbf{(Current Record!)}} \step{\texttt{2.5 months}}
\end{itemize}}\pause

$ p = \lfloor 10^{119} \pi \rfloor+ 207819$, $g=2$\pause

$y = \lfloor 10^{119}\rfloor$\pause

\begin{tiny}

$y=g^{{262112280685811387636008622038191827370390768520656974243035\atop
       380382193478767436018681449804940840373741641452864730765082}},$
          \end{tiny}

\end{slide}

\begin{slide}\pageTransitionWipe{30}
\heading{ElGamal Cryptosystem 1/2}

\textbf{Alice} wants to sent a message $x\in\Z/p\Z$ to
\textbf{Bob}\bigskip\pause

\centerline{\textcolor{black}{\textsc{SETUP:}}}\pause
\parstepwise{
\begin{itemize}
\item[\textcolor{blue}{\ding{182}}] \step{\textbf{Alice} and
\textbf{Bob} agree on a prime $p$ and a \textit{\underline{generator}} $g$ in
$\Z/p\Z$}\medskip
\item[\textcolor{blue}{\ding{183}}] \step{\textbf{Bob} picks
a \textcolor{red}{secret} $b$, $0< b\leq p-1$, }
\step{he computes $\beta=g^b\bmod p$ and publishes $\beta$}
\end{itemize}}\pause\bigskip

\centerline{\textcolor{black}{\textsc{ENCRYPTION:}} (Alice)}\pause
\parstepwise{
\begin{itemize}
\item[\textcolor{red}{\ding{172}}] \step{\textbf{Alice}
picks a \textcolor{red}{secret} $k$,} \step{$0< k\leq p-1$}\medskip
\item[\textcolor{red}{\ding{173}}] \step{She computes $\alpha=g^k\bmod p$
and $\gamma=x\cdot \beta^a\bmod p$}\medskip
\item[\textcolor{red}{\ding{174}}] \step{The encrypted message is}\\
\step{{\hspace{4cm}$E(x)=(\alpha,\gamma)\in \Z/p\Z\times\Z/p\Z$}}
\end{itemize}}
\end{slide}

\begin{slide}\pageTransitionWipe{30}
\heading{ElGamal Cryptosystem 2/2}\pause\medskip

\centerline{\textcolor{black}{\textsc{DECRYPTION:}} (Bob)}\pause
\parstepwise{
\begin{itemize}
\item[\textcolor{blue}{\ding{172}}] \step{\textbf{Bob} computes}\\
\step{\hspace{4cm}$D(\alpha,\gamma)=\gamma\cdot\alpha^{p-1-b}\bmod p$}
\item[\textcolor{blue}{\ding{173}}] \step{It works because}\\
\step{\hspace{4cm}$D(E(x))=D(\alpha,\gamma)=x\cdot g^{bk}\cdot g^{k(p-1-b)}=x$}\\
\step{since $g^{k(p-1)}\bmod p =1$ by}
\end{itemize}}\pause\smallskip

\begin{center}
\begin{tabular}{|c|}
\hline \textbf{\textcolor{red}{Fermat Little Theorem}} If $p$ is prime, $p\nmid a\in{\mathbb N}$\\
$a^{p-1}\equiv1\bmod p$
\\\hline\end{tabular}
\end{center}\pause

\textbf{\textcolor{black}{Eve}} can decrypt the message if he can
compute the discrete logarithm $X$,\pause
$$\beta=g^X\bmod p$$
\end{slide}


\begin{slide}\pageTransitionWipe{30}
\heading{Massey Omura 1/2}\pause

\centerline{\textbf{Alice}\
\includegraphics[width=5cm]{images/MasseyOmura.jpg}\ \textbf{Bob}}\pause

\parstepwise{
\begin{itemize}
\item[\textcolor{red}{\ding{172}}]
\step{\textbf{Alice} and \textbf{Bob} each picks a secret key $k_A,k_B\in\{1,\ldots,p-1\}$}
\item[\textcolor{red}{\ding{173}}]
\step{They compute $l_A,l_B\in\{1,\ldots,p-1\}$ such that}
\item[\textcolor{red}{\ding{174}}]
\step{$k_Al_A=1(\bmod p-1)$ and $k_Bl_B=1(\bmod p-1)$}
\item[\textcolor{red}{\ding{175}}]
\step{\textbf{Alice} key is $(k_A,l_A)$ ($k_A$ to lock and $l_A$ to unlock)}
\item[\textcolor{red}{\ding{176}}]
\step{\textbf{Bob} key is $(k_B,l_B)$ ($k_B$ to lock and $l_B$ to unlock)}
\end{itemize}}


\end{slide}

\begin{slide}\pageTransitionWipe{30}
\heading{Massey Omura 2/2}

\centerline{\textbf{Alice} $(k_A,l_A)$\
\includegraphics[width=4cm]{images/MasseyOmura.jpg}\ \textbf{Bob}
$(k_B,l_B)$}\pause

\parstepwise{
\begin{itemize}
\item[\textcolor{blue}{\ding{172}}] \step{To send the message $P$, \textbf{Alice} computes and sends
$M=P^{k_A}\bmod p$}
\item[\textcolor{blue}{\ding{173}}] \step{\textbf{Bob} computes and sends back $N=M^{k_B}\bmod p$}
\item[\textcolor{blue}{\ding{174}}] \step{\textbf{Alice} computes $L=N^{l_A}\pmod p$ and sends
it back to \textbf{Bob}}
\item[\textcolor{blue}{\ding{175}}] \step{\textbf{Bob} decrypt the message computing $P=L^{l_B}\pmod p$}
\end{itemize}}\pause

It works: $P=L^{l_B}=N^{l_Al_B}=M^{k_Bl_Al_B}=P^{{k_Ak_Bl_Al_B}}$ by Fermat Little Theorem
\end{slide}


\begin{slide}\pageTransitionWipe{30}
\heading{From $\Z/p\Z$ to cyclic groups}\pause

We can substitute $\Z/p\Z$ with a set $G$ where it is possible to
compute powers $P^a$ and there is a generator (there is $g\in G$
such that for each $\alpha\in G$, $\alpha=g^i$ for a suitable $i$);
\emph{cyclic groups}\pause

\centerline{\textbf{\textcolor{red}{Examples of cyclic
groups}}}\pause

\parstepwise{
\begin{itemize}
\item[\textcolor{black}{\ding{172}}] \step{Elliptic curves modulo $p$}
\item[\textcolor{black}{\ding{173}}] \step{Multiplicative groups of Finite Fields}
\item[\textcolor{black}{\ding{174}}] \step{Dickson Polynomials over finite fields}
\end{itemize}}\pause


\end{slide}


\begin{slide}\pageTransitionWipe{30}

\heading{Finite Fields}\pause

\parstepwise{\begin{itemize}
\item[\textcolor{red}{\ding{43}}]\step{ Let
$\F_p=\Z/p\Z=\{0,1,\ldots,p-1\}$\hspace*{3cm} (\textcolor{red}{field} if
$p$ prime)}
\item[\textcolor{red}{\ding{43}}] \step{Given $f\in \F_p[x]$
\textcolor{red}{irreducible} ($m=\partial(f)$)}\\
\step{{\hspace*{1cm}\fbox{$\F_p[x]/(f)=\{ a_0+a_1t+\cdots +
a_{m-1}t^{m-1}\ |\ a_i\in\F_p\}$}}}
\item[\textcolor{red}{\ding{43}}] \step{$\F_p[x]/(f)$ is a field}\\ %under componentwise addition and
\step{\hspace*{1cm}($g_1\star g_2\in\F_p[x]/(f)$ is $g_1g_2\bmod f$)}
\item[\textcolor{red}{\ding{43}}] \step{$\F_p[x]/(f)$ does not depend on $f$}\\
\step{(i.e. if $h\in \F_p[x]$ irreducible, $\partial f=\partial
h\ \ =\!\!\!=\!\!\!>\ \ \F_p[x]/(f)\cong \F_p[x]/(h)$ )}
\item[\textcolor{red}{\ding{43}}] \step{{\hspace*{4cm}\fbox{$\F_{p^m}=\F_p[x]/(f)$}}}\\
\step{\emph{any choice of $f$ with $m=\partial f$ is the same}}
\item[\textcolor{red}{\ding{43}}] \step{$|\F_{p^m}|=p^m$}
\item[\textcolor{red}{\ding{43}}] \step{$\F_{p^m}^*=\F_{p^m}\setminus\{0\}$ is a cyclic group under multiplication}
\end{itemize}}

\end{slide}

\begin{slide}\pageTransitionWipe{30}
\heading{Producing $\F_q$}\pause

Set $q=p^m$\pause
\parstepwise{\begin{itemize}
\item[\textcolor{red}{\ding{43}}] \step{Produce $\F_q$ $<\!\!\!=\!\!\!=\!\!\!>$ find $f\in I_m(q)$}\\
\step{\hspace*{1cm}{($I_m(q)=\{f\in\F_p[x], f \textrm{ irreducible},
\partial f=m\}$)}}
\item[\textcolor{red}{\ding{43}}]
\step{$\displaystyle\sum_{d|m}d|I_d(q)|=q^m$}
\item[\textcolor{red}{\ding{43}}] \step{$|I_m(q)|=\frac{q^m-q}{m}$\hspace{1cm}
(if $m$ is prime)\hspace{1cm}$|I_m(q)|\sim \frac{q^m}{m}$}
%\item[\textcolor{red}{\ding{43}}] \step{If $m\nmid p-1$ \& $m$ is prime
%$=\!\!\!=\!\!\!>\ \ \frac{x^m-1}{x-1}\in I_{m-1}(q)$}
\item[\textcolor{red}{\ding{43}}] \step{Some fields of cryptographic size:}\\
\step{\begin{small}$\F_{2^{503}}=\F_2[x]/(x^{503} + x^3 +
1),\F_{5323^{20}}=\F_{5323}[x]/(f)$\end{small}}\\
\begin{tiny}\step{$f=x^{20} +
{1451}x^{18} + {5202}x^{17} + {752}x^{16} + {3778}x^{15} +
{4598}x^{14 } + {2563}x^{13} + {5275}x^{12} + {4260}x^{11} +\vspace{-3mm}$}\\
\step{${862}x^{10} + {4659}x^{9} + {34 84}x^{8} + {1510}x^{7} +
{4556}x^{6} + {2317}x^{5} + {2171}x^{4} + {3100}x^{3} +
{4100}x^{2} + {682}x + {5110}$}\end{tiny}
\item[\textcolor{red}{\ding{43}}] \step{Good to find $f$
\textcolor{blue}{sparse}}
\end{itemize}}

\end{slide}

\begin{slide}\pageTransitionWipe{30}
\heading{Interpolation on $\F_q$}\pause Given $h: \F_q\rightarrow
\F_q$ a function.\pause $h$ \textcolor{blue}{ can always be
interpolated with a polynomial in }$\F_q[x]$ !\pause
\textcolor{red}{\ding{43}} \textsc{Lagrange interpolation}\pause
$$f_h(x)=\sum_{c\in \F_q}h(c)\prod_{\substack{d\in \F_q\\ d\neq c}}\frac{x-d}{c-d}\in\F_q[x]$$\pause
\textcolor{red}{\ding{43}} \textsc{Finite fields
interpolation}\pause
$$f_h(x)=\sum_{c\in \F_q}h(c)\left(1-(x-c)^{q-1}\right)\in\F_q[x]$$\pause
\vspace*{-4mm}\hspace*{5cm}$d^{q-1}=\begin{cases}1 & d\neq 0\\ 0 &
d=0\end{cases}$
\end{slide}


\begin{slide}\pageTransitionWipe{30}
\heading{More on interpolation in $\F_q$}\pause\medskip

\textcolor{red}{\ding{43}} If $f_1,f_2\in\F_q[x]$  with
$f_1(c)=f_2(c) \forall c\in\F_q$,\pause
\centerline{\fbox{$=\!\!\!=\!\!\!>\ \  x^{q}-x \mid
f_1(x)-f_2(x)$}}\pause\medskip

\textcolor{red}{\ding{43}} The interpolant polynomial is unique mod
$x^q-x$\pause \centerline{\fbox{$=\!\!\!=\!\!\!>\ \ $ unique with
degree $\leq q-1$}}\pause\medskip

\textcolor{red}{\ding{43}} If $c_h=\#\{c\in\F_q\ |\ h(c)\neq c
\}$,\pause \centerline{\fbox{$q-c_h\leq \partial f_h\leq
q-2$}}\pause\medskip

\textcolor{red}{\ding{43}} \textbf{Problem.} \emph{Find functions
with sparse interpolation polynomial}\pause\medskip\medskip


\centerline{\textcolor{blue}{Better if they are
$\rightsquigarrow$}\textcolor{rossoscu}{\emph{Permutation
polynomials}}}
\end{slide}


\begin{slide}\pageTransitionWipe{30}
\heading{Permutation polynomials}\medskip\pause

$${\mathcal S}(\F_q)=\{\sigma: \F_q\rightarrow\F_q\ |\ \sigma
(1:1)\}$$\pause \ \hfill permutations of $\F_q$\pause

\textcolor{red}{\ding{43}} $f\in\F_q[x]$ is called
\textcolor{blue}{permutation polynomial} (\textcolor{blue}{PP})
if\pause \centerline{``\emph{$f$ (as a funtion) is a
permutation}''}\pause \centerline{(i.e. $\exists \sigma\in {\mathcal
S}(\F_q), \sigma(c)=f(c)\ \forall c\in\F_q$)}\pause

\textcolor{red}{\ding{43}} If $f_\sigma(x)=\sum_{c\in
\F_q}\sigma(c)\left(1-(x-c)^{q-1}\right)\in\F_q[x]$
$=\!\!\!=\!\!\!>\ \ $\pause \centerline{\fbox{$f\in\F_q[x]$ is PP
$<\!\!\!=\!\!\!=\!\!\!>\ \ \exists \sigma \in{\mathcal S}(\F_q),
f\equiv f_\sigma \bmod x^q-x$}}\pause

\textcolor{red}{\ding{43}} \textbf{Examples:}\pause
\parstepwise{\begin{itemize}
\item[\textcolor{blue}{\ding{46}}] \step{$ax+b,\ \ \ \ \ \  a,b\in\F_q,
a\neq 0$}
\item[\textcolor{blue}{\ding{46}}] \step{$x^k,\hspace{1.5cm}
(k,q-1)=1$}
\end{itemize}}
\end{slide}

\begin{slide}\pageTransitionWipe{30}
\heading{More examples of PP}\pause

\textcolor{blue}{\ding{46}} \textcolor{blue}{\textsc{Composition.}}
$f\circ g$ is PP \hfill if $ f,g$ are  PP\pause
\textcolor{blue}{\ding{46}} $x^{(q+m-1)/m}+ax$ is a PP \hfill if
$m|q-1$\pause \textcolor{blue}{\ding{46}}
\textcolor{blue}{\textsc{Linearized Polynomials}} Let $q=p^m$,\pause
\centerline{{$L(x)=\displaystyle\sum_{s=0}^{r-1}\alpha_sx^{q^s}$}\
\ \ \ \ ($\alpha_s\in\F_{p^m}$)}\pause
\parstepwise{\begin{itemize}
\item[\textcolor{green}{\ding{243}}]\step{$L(c_1+c_2)=L(c_1)+L(c_2)$}
\item[\textcolor{green}{\ding{243}}]\step{$L\in \operatorname{GL}_m(\F_p) \subset {\mathcal S}(\F_{p^m})$\ \ \
$<\!\!\!=\!\!\!=\!\!\!>$\ \ \ $\det(\alpha_{i-j}^{q^j})\neq0$}\\
\step{\hspace*{3.66cm} \ \
$<\!\!\!=\!\!\!=\!\!\!>$\ \ \ $L(x)=0$ has 1 solution}
\end{itemize}}
\end{slide}



\begin{slide}\pageTransitionWipe{30}
\heading{One more example of PP}\pause

\textcolor{blue}{\ding{46}} \textcolor{blue}{\textsc{Dickson
Polynomials.}} If $a\in\F_q$, $k\in\N$\medskip\pause

\centerline{{$\displaystyle
D_k(x,a)=\sum_{j=0}^{[k/2]}\frac{k}{k-j}\binom{k-j}{j}(-a)^jx^{k-2j}$}}
\bigskip\pause

\parstepwise{\begin{itemize}
\item[\textcolor{green}{\ding{243}}] \step{if $a\neq0$, $D_k(x,a)$ is a PP $<\!\!\!=\!\!\!=\!\!\!>$\ \
$(k,q^2-1)=1$}
\item[\textcolor{green}{\ding{243}}] \step{$D_k(x,0)=x^k$ is a PP $<\!\!\!=\!\!\!=\!\!\!>$\ \ $(k,q-1)=1$}
\item[\textcolor{green}{\ding{243}}] \step{\textcolor{yellow}{\textbf{Note:}} if $(mn,q^2-1)=1$,
\medskip}\\
\step{\hspace*{2cm}\fbox{$D_m(D_n(x,\pm 1),\pm1)=D_{mn}(x,\pm1)$}}
\end{itemize}}
\end{slide}

\begin{slide}\pageTransitionWipe{30}
\heading{\emph{Dickson} analogue of DH Key--exchange}\pause

\parstepwise{\begin{itemize}
\item[\textcolor{green}{\ding{172}}] \step{\textbf{Alice} and \textbf{Bob} agree on a finite field $\F_q$, and a generator
$\gamma\in\F_q$}
\item[\textcolor{green}{\ding{173}}] \step{\textbf{Alice} picks a \textcolor{red}{secret} $a\in[0,q^2-1]$, \textbf{Bob} picks a
\textcolor{red}{secret}
$b\in[0,q^2-1]$}
\item[\textcolor{green}{\ding{174}}] \step{They compute and publish $D_a(\gamma,1)$
(\textbf{Alice}) and $D_b(\gamma,1)$ (\textbf{Bob})}
\item[\textcolor{green}{\ding{175}}] \step{The common \textcolor{red}{secret} key is}
\end{itemize}}\pause
\centerline{{$D_{ab}(\gamma,1)=D_a(D_b(\gamma,1,1))=D_b(D_a(\gamma,1,1))$}}
\bigskip\pause

\begin{small}\textcolor{blue}{\textbf{NOTE.}} There is a fast algorithm to compute the value of a Dickson polynomial at
an element of $\F_q$\bigskip\pause

\textcolor{blue}{\textbf{Problem.}} Find new classes of PP\end{small}

\end{slide}


\begin{slide}\pageTransitionWipe{30}
\heading{Enumeration of PP by degree}\pause

\centerline{\fbox{\textcolor{blue}{$N_d(q)=\{\sigma\in{\mathcal S
}(\F_q)\ |\ \partial(f_\sigma)=d\}$}}}\pause

\textbf{Problem.} \emph{Compute $N_d(q)$}\pause

\parstepwise{\begin{itemize}
\item[\textcolor{red}{\ding{43}}] \step{$\displaystyle\sum_{d\leq q-2}N_{d}(q)=q!$\hspace{3cm} $(\partial f_\sigma\leq q-2)$}
\item[\textcolor{red}{\ding{43}}] \step{$N_1(q)=q(q-1)$}
\item[\textcolor{red}{\ding{43}}] \step{$N_d(q)=0$ if $d|q-1$\hspace{3cm} (Hermite criterion)}
\item[\textcolor{red}{\ding{43}}] \step{$N_d(q)$ is known for $d< 6$}
\item[\textcolor{red}{\ding{43}}] \step{\emph{Almost all permutation polynomials have degree $q-2$}}
\end{itemize}}\pause
\begin{center}
\begin{tabular}{|c|}\hline
(S. Konyagin, FP -- 2002) $M_q=\{\sigma\in{\mathcal S}(\F_q)\ |\ \partial
f_\sigma<q-2\}$\\ $|\#M_q-(q-1)!|\leq \sqrt{2e/\pi}q^{q/2}$\\
\hline\end{tabular}\end{center}
\end{slide}

\begin{slide}\pageTransitionWipe{30}
\heading{A recent result}\vspace{-1mm}\pause

\centerline{\present{\fbox{${\mathcal
N}_d=\#\left\{\sigma\in\mathcal S(\F_q)\ |\
\partial(f_\sigma)< q-d-1\right\}$}}}\pause\bigskip

\replacecolor{highlightcolor}{inactivecolor}
\centerline{\shadowbox{\highlightboxed{\begin{minipage}{10cm}
\verde{\textbf{Theorem}}
{\it S. Konyagin, F\!P -- 2003}\\
\textcolor{black}{\textit{Let $\alpha=(e-2)/3e=0.08808\cdots$ and
$d< \alpha q$. Then
$$\left|{\mathcal N}_d-\frac{q!}{q^d}\right|\leq
2^ddq^{2+q-d}\binom{q}{d}\left(\frac{2d}{q-d}\right)^{(q-d)/2}.
$$
It follows that
\\
\centerline{$\displaystyle{\mathcal N_d\sim \frac{q!}{q^d}}$}
\\
if $d\leq\alpha q$ and $\alpha<0.03983$}}
\end{minipage}}}}

\end{slide}


\begin{slide}\pageTransitionWipe{30}
\heading{Other ways of counting} {If $\sigma\in{\mathcal
S}(\F_q)$,\hspace{3cm}\fbox{\textcolor{blue}{$c_\sigma=\#\{a\in\F_q\
|\ \sigma(a)\neq a\}$}}}
$$\sigma\neq id=\!\!\!=\!\!\!>\ \  q-c_\sigma\leq\partial f_\sigma \leq q-2$$
{\scriptsize{(since $f_\sigma(x)-x$ has at least $q-c_\sigma$ roots)}}

\textbf{Consequences.}\begin{itemize}
\item[\textcolor{green}{\ding{43}}]  $2$--cycles have degree $q-2$
\item[\textcolor{green}{\ding{43}}]  $3$--cycles have degree $q-2$ or $q-3$
\item[\textcolor{green}{\ding{43}}]  $k$--cycles have degree in $[q-k,q-2]$
\end{itemize}
(\textcolor{blue}{\emph{Wells}}) \fbox{$\displaystyle{\#\{\sigma\in 3\textrm{--cyle},\ \partial(f_\sigma)=q-3\}=
\begin{cases}\frac{2}{3}q(q-1) & q\equiv1\bmod 3 \\ 0 & q\equiv 0 \bmod 3 \\ \frac13q(q-1)&q\equiv0\bmod3
\end{cases}}$}
\end{slide}


\begin{slide}\pageTransitionWipe{30}
\heading{More enumeration functions}

\begin{itemize}
\item[\textcolor{red}{\ding{43}}]  $\sigma_1$, $\sigma_2$ conjugated $=\!\!\!=\!\!\!>\ \ $\ \ $c_{\sigma_1}=c_{\sigma_2}$

\item[\textcolor{red}{\ding{43}}] $\Ccal$ \textcolor{green}{\emph{conjugation class of permutations}}

\item[\textcolor{red}{\ding{43}}] $c_\Ccal = \#\{$ elements $\in\F_q$ moved by any $\sigma\in\Ccal\}$\\
{\small (i.e. $c_\Ccal=c_\sigma$ for any $\sigma\in\Ccal$\ \ \ \ \ $q-c_\Ccal\leq f_\sigma$)}

\item[\textcolor{red}{\ding{43}}] $\Ccal=[k]$=$k$--cycles\ \ \ $=\!\!\!=\!\!\!>\ \ \ \ \ c_{[k]}=k$


\item[\textcolor{red}{\ding{43}}] \textcolor{blue}{Natural enumeration functions}:
\begin{itemize}

\item[\textcolor{red}{\ding{55}}] $m_{\Ccal}(q)=\#\{\sigma\in\Ccal,
\partial f_{\sigma}=q-c_\Ccal\}$\hfill (\emph{minimal degree})


\item[\textcolor{red}{\ding{55}}] $M_{\Ccal}(q)=\#\{\sigma\in\Ccal ,
\partial f_{\sigma}<q-2\}$\hfill (\emph{non-maximal degree})

\end{itemize}
\end{itemize}

\end{slide}


\begin{slide}\pageTransitionWipe{30}
\heading{Permutation  Classes with non maximal degree}

Let $\Ccal=(m_1,\ldots,m_t)$ be the  class of permutations with
$m_1$ 1-cycles, $\ldots$, $m_t$ $t$-cycles. The
number $c_\Ccal$ of elements in $\F_q$ moved by any element of $\Ccal$ is
$$c_\Ccal =2m_2+3m_3+\cdots+tm_t$$\medskip

\centerline{\fbox{\textcolor{blue}{$M_{\Ccal}(q)=\#\{\sigma\ \in\Ccal,
\partial f_{\sigma}<q-2\}$}}}

\begin{center}\begin{tabular}{|l|}
\hline
\textcolor{green}{\textsc{Theorem 1}} (C. Malvenuto, FP - 2002). $\exists
N=N_{\Ccal}\in\N$, $f_1,\cdots, f_N\in\Z[x]$,\\ $f_i$ monic,
$\partial f_i=c_\Ccal-3$ such that if $q\equiv a\bmod N$, then\\
\\
\hspace{3cm}\textcolor{blue}{$\displaystyle M_{\Ccal}(q)=\frac{q(q-1)}{m_2!2^{m_2}\cdots m_t!t^{m_t}}f_a(q)$}\\
\hline\end{tabular}\end{center}
\end{slide}

\begin{slide}\pageTransitionWipe{30}
\heading{$k$--cycles with minimal degree}

\centerline{\fbox{\textcolor{blue}{$m_{[k]}(q)=\#\{\sigma\ k{\textrm{--cycle},
\partial f_{\sigma}=q-k}\}$}}}

\textcolor{green}{\textsc{Theorem 2}} (C. Malvenuto, FP - 2003).
\begin{itemize}
\item[\textrm{\textcolor{blue}{\ding{42}}}]  If $q\equiv1\bmod k$\ \ \ $=\!\!\!=\!\!\!>\ \ $
$$m_{[k]}(q)\geq \frac{\varphi(k)}{k}q(q-1).$$
\item[\textrm{\textcolor{blue}{\ding{42}}}]  If $q=p^f$, $p\geq
2\cdot3^{[k/3]-1}\ \ \ =\!\!\!=\!\!\!>\ \ $
$$m_{[k]}(q)\leq\frac{(k-1)!}{k}q(q-1).$$
\end{itemize}
\end{slide}


\begin{slide}\pageTransitionWipe{30}
\heading{Consequences of Theorem 1}
\begin{itemize}
\item[\textrm{\textcolor{blue}{\ding{41}}}]
$\displaystyle{\frac{M_{\Ccal}(q)}{\#\Ccal}=\frac{1}{q}+O\left(\frac{1}{q^2}\right)}$

\item[\textrm{\textcolor{blue}{\ding{41}}}] If $\Ccal$ is fixed,
$$\operatorname{Prob}(\partial f_\sigma<q-2\ |\
\sigma\in\Ccal)\sim\frac{1}{q}$$

\item[\textrm{\textcolor{blue}{\ding{41}}}] If $q=2^r$, $\Ccal_r$ is the
conjugation class of $r$ transposition,
$$M_{\Ccal_r}(q)=\frac{q!}{r!2^r(q-2r+1)!}-\frac{q-2(r-1)(2r-1)}{2r}M_{\Ccal_{r-1}}(q)$$

\item[\textrm{\textcolor{blue}{\ding{41}}}] One can compute
$M_{\Ccal}(q)$ for $c_\Ccal\leq 6$
\end{itemize}
\end{slide}

\begin{slide}\pageTransitionWipe{30}
\heading{Table 1. $\#c_\Ccal\leq 6$, ($q$ odd)}

$$\begin{array}{rrl}
\textrm{\textcolor{blue}{\ding{34}}}& M_{[4]}(q)=
&\frac{1}{4}q(q-1)\left(q-5-2\eta(-1)-4\eta(-3)\right)\\\\
\textrm{\textcolor{blue}{\ding{34}}}& M_{[2\ 2]}(q)=
&\frac{1}{8}q(q-1)(q-4)\left\{1+\eta(-1)\right\}\\\\
\textrm{\textcolor{blue}{\ding{34}}}& M_{[5]}(q)  = &
\frac{1}{5}q(q-1)\left(q^2-\left(9-\eta(5)-5\eta(-1)+5\eta(-9)\right)q+
\right.\\
&&\left.\hspace{1cm} +26+ 5\eta(-7) +15\eta(-3)
+15\eta(-1)\right)\\\\
\textrm{\textcolor{blue}{\ding{34}}} &  M_{[2\ 3]}(q)  = & \frac{1}{6}q(q-1)  \left(q^2-(9+\eta(-3)+3\eta(-1))q+ \right.\\
&&\left.\hspace{1cm}+(24+6\eta(-3)+
18\eta(-1)+6\eta(-7))\right)+ \\
&&\hspace{1cm}\eta(-1)(1-\eta(9))q(q-5) .\end{array}$$

\end{slide}

\begin{slide}\pageTransitionWipe{30}
\heading{Table 2. $\#c_\Ccal\leq 6$, ($q$ even)\ }

$$\begin{array}{rrl}
\textrm{\textcolor{blue}{\ding{34}}} &  M_{[4]}(2^n)=& \frac{1}{4}2^n(2^n-1)(2^{n}-4)(1+(-1)^n)\\
\\
\textrm{\textcolor{blue}{\ding{34}}} &  M_{[2\ 2]}(2^n)=& \frac{1}{8}2^{n}(2^n-1)(2^{n}-2)\\
\\
\textrm{\textcolor{blue}{\ding{34}}} &  M_{[5]}(2^n)= & \frac{1}{5}2^n(2^n-1)(2^n-3-(-1)^n)(2^n-6-3(-1)^n)\\
\\
\textrm{\textcolor{blue}{\ding{34}}} &  M_{[2\ 3]}(2^n)=&
\frac{1}{6}2^n(2^n-1)(2^n-3-(-1)^n)(2^n-6).\end{array}$$
\end{slide}

\begin{slide}\pageTransitionWipe{30}
\heading{Table 3. $\#c_\Ccal= 6$, ($q$ odd, $3\nmid q$)}
\begin{small}
$$ \begin{array}{|rcl|}
\hline
 M_{[6]}(q)&=&\frac{q(q-1)}{6}\{{q}^{3}-14\,{q}^{2}+[68-6\,\eta(5)-6\,
\eta(50)]q-\\
&& [154+66\,\eta(-3)+93\,\eta(-1)+12\,
\eta(-2)+54\,\eta(-7)]\} \\
 M_{[4\ 2]}(q)& =&\frac{q(q-1)}{8}(
{q}^{3}-[14-\eta(2)]{q}^{2}+\\
&& [71+12\,\eta(-1)+\eta(-2)+4\,\eta(-3)-8\,\eta
(50)]q\\
& &-[148+100\,\eta(-1)+24\,\eta(-2)+44\,\eta(-3)+40\,\eta(-7)])\\
 M_{[3\ 3]}(q)& =&\frac{q(q-1)}{18}( q^3-13\,{q}^{2}+[62+9\,\eta(-1)+4\,\eta(-3)]q\\
&& -[150+99\,\eta(-1)+42\,\eta(-3)+72\,\eta(-7)])\\
 M_{[2\ 2\
2]}(q)&=&\frac{q(q-1)}{48}(q^3-[14+3\,\eta(-1)]{q}^{2}+[70+36\,\eta(-1)+6\,\eta(-2)]q\\
&&-[136+120\,\eta(-1)+48\,\eta(-2)+8\,\eta(-3)])\\
 \hline \end{array}
$$\end{small}

\end{slide}

\begin{slide}\pageTransitionWipe{30}
\heading{Table 4. $\#c_\Ccal= 6$}
\begin{small}
$$ \begin{array}{|rcl|}\hline
M_{[6]}(3^n)&=&\frac{3^n(3^n-1)}{6}\{3^{3n}-[14+2(-1)^n]3^{2n}+[71+39(-1)^n]3^n-\\
&& [162+147(-1)^n]\}
\\
&&\\
M_{[4\ 2]}(3^n)&=& \frac{3^n(3^n-1)}{8}\{{3}^{3n}-[14+3\left (-1\right
)^{n}]{3}^{2n}+[72+40\left (-1 \right )^{n}]{3}^{n}-\\
&&[164+140\left(-1\right )^{n}]\}
\\
&& \\
M_{[3\ 3]}(3^n)&=&\frac{3^n(3^n-1)}{18}\{\left (1+\left (-1\right )^{n}\right
){3}^{3\,n}-[14+15\,\left (-1 \right
)^{n}]{3}^{2\,n}+\\
&&[71+81\,\left (-1\right )^{n}]{3}^{n}-[150+171\,\left (-1\right )^{n}]\}
\\
&& \\
M_{[2\ 2\ 2]}(3^n)&=&\frac{3^n(3^n-1)}{48}\{3^{3n}-[14+3(-1)^n]{3}^{2n}+[76+36(-1)^n]3^n
-\\
&&+[168+120(-1)^n]\}\\
\hline \end{array}$$
\end{small}
\end{slide}

\begin{slide}\pageTransitionWipe{30}
\heading{Table 5. $\#c_\Ccal= 6$}
\begin{small}
$$ \begin{array}{|rcl|}
\hline &&\\
M_{[6]}(2^n)& = \frac{2^n(2^n-1)}{6}&\{(2^n-3-(-1)^n)(2^{2n}-(11-(-1)^n)2^n+\\ && (41+7(-1)^n))\}\\
&&\\
M_{[4\ 2]}(2^n)& =\frac{2^n(2^n-1)}{8}&\{(2^n-3-(-1)^n)(2^{2n}-11\cdot2^n+37+(-1)^n)\}\\
&& \\
M_{[3\ 3]}(2^n)& =\frac{2^n(2^n-1)}{18}&\{(2^n-3-(-1)^n)(2^{2n}-\\ && (10-(-1)^n)2^n+45-3(-1)^n))\}\\
&&\\
M_{[2\ 2\ 2]}(2^n)&= \frac{2^n(2^n-1)}{48}&\{(2^n-2)(2^n-4)(2^n-8)\}\\
&&\\
\hline
 \end{array}
$$
    \end{small}
\end{slide}

\begin{slide}\pageTransitionWipe{30}

\heading{Sketch of the Proof of Theorem 2. (1/3)}

\textcolor{blue}{\textsc{Step 1.}} Translate the problem into one on counting points of an algebraic varieties
$$m_{k}(q)=\frac{q(q-1)}{k}n_k(q)$$
where $n_k(q)=\{\sigma\in[k]\ |\ \partial f_\sigma=q-k, \sigma(0)=1\}$.\\
Need to show $|n_k(q)|\leq (k-1)!.$
Now
$$f_\sigma(x)=\sum_{c\in\F_q}\sigma(c)\left(1-(x-c)^{q-1}\right)=A_1x^{q-2}+A_2x^{q-3}+\cdots+A_{q-1}.$$
with
$\displaystyle {A_j=\sum_{c\in\F_q}\sigma(c)c^j=\sum_{c\in\F_q}\sigma(c)\left(c^{j}-c^{j-1}\right)=\sum_{\substack{c\in \F_q\\\sigma(c)
\neq c}}(\sigma(c)-c)c^j.}$
\end{slide}


\begin{slide}\pageTransitionWipe{30}
\heading{Sketch of the Proof of Theorem 2. (2/3)}
If $\sigma=(0,\ 1,\ x_1,\ x_2,\ \ldots,\ x_{k-2})\in{\mathcal S}(\F_q)$,
$$A_{j}(\sigma)=(1-x_1)+(x_1-x_2)x_1^j+\cdots (x_{k-2}-x_{k-2})x^j_{k-3}+x_{k-2}^{j+1}.$$
\noindent\textbf{Def. (Affine $k$--th Silvia set)}
\begin{scriptsize}
${\mathcal A}_k:
\left\{\begin{array}{rcl}
(1-x_1)+x_1(x_1-x_2)+\cdots +x_{k-3}(x_{k-3}-x_{k-2})+x_{k-2}^{2}& = &0\\
(1-x_1)+x_1^2(x_1-x_2)+\cdots +x_{k-3}^2(x_{k-3}-x_{k-2})+x_{k-2}^{3}& = &0\\
& \vdots & \\
(1-x_1)+x_1^{k-2}(x_1-x_2)+\cdots+
x_{k-3}^{k-2}(x_{k-3}-x_{k-2})+x_{k-2}^{k-1}& = &0
\end{array}\right.$
\end{scriptsize}
$$n_{k}(q)=\#\{\underline{x}=(x_1,\ldots,x_{k-2})\in\F_q^{k-2}\ |\ \underline{x}\in{\mathcal A}_k(\F_q), x_i\neq x_j\}
\leq \#{\mathcal A}_k(\F_q)$$
\centerline{\fbox{\textcolor{blue}{
$\dim_{\overline{\F}_q}{\mathcal A}_k=0\ \ \ \substack{\textrm{Bezout Thm.}\\
=\!\!\!=\!\!\!>\ \ }\ \ \  \#{\mathcal A}(\F_q)\leq (k-1)!$}}}

%\noindent\textbf{Def. (Projective $k$--th Silvia set)}
%
%$${\mathcal V}_k:
%\hspace*{-0.2cm}\!\left\{\!\!\begin{array}{rl}
%X_0(X_0-X_1)+X_1(X_1-X_2)+\cdots +X_{k-3}(X_{k-3}-X_{k-2})+X_{k-2}^{2}&=0\\
%X_0^2(X_0-X_1)+X_1^2(X_1-X_2)+\cdots +X_{k-3}^2(X_{k-3}-X_{k-2})+X_{k-2}^{3}&=0\\
%& \vdots \\
%X_0^{k-2}(X_0-X_1)+X_1^2(X_1-X_2)+\cdots+
%X_{k-3}^{k-2}(X_{k-3}-X_{k-2})+X_{k-2}^{k-1}&=0
%\end{array}\right.$$
\end{slide}

\begin{slide}\pageTransitionWipe{30}
\heading{Sketch of the Proof of Theorem 2. (3/3)}

\textcolor{blue}{\textsc{Step 2.}}
\begin{center}\begin{tabular}{|l|}
\hline
\textcolor{green}{\textbf{Theorem.}} If $\textbf K$ is an algebrically closed field,\hspace{3cm}\ \\
\multicolumn{1}{|c|}{
$\displaystyle{\operatorname{char}({\textbf{K}})=\begin{cases}0 & \textrm{or} \\ > 2\cdot 3^{[k/3]-1}.\end{cases}}$}\\
\\
Then\\
\multicolumn{1}{|c|}{\fbox{\textcolor{blue}{$\displaystyle\dim_{{\textbf{K}}}{\mathcal A}_k=0.$}}}\\
\\\hline\end{tabular}\end{center}

\textbf{NOTE.} \begin{itemize}
\item[\textcolor{green}{\ding{45}}] Proof is based on finding projective hyperplanes disjoint from ${\mathcal A}_k$
\item[\textcolor{green}{\ding{45}}] There are examples of small values of $q$ with $\dim_{{\textbf{K}}}{\mathcal A}_k>0$
\end{itemize}
\end{slide}

\end{document}


\begin{slide}\pageTransitionWipe{30}
\heading{Numerical Examples (4--cycles)}

\begin{tabular}{|c|}
\hline
\\
$m_{[4]}(\F_q)=\frac{1}{4}q(q-1)\cdot\left\{
\begin{array}{rrll}
6 & \textrm{if } q\equiv &1 &\pmod{20}\\
4 & \textrm{if } q\equiv &11 &\pmod{20}\\
2 & \textrm{if } q\equiv &9,13,17&\pmod{20}\\
0 & \textrm{if } q\equiv & 3,7,19&\pmod{20},\\
\end{array}\right.$\\
$m_{[4]}(\F_{5^n})=\frac{1}{2}{5^n}({5^n-1}),$
\ \ $m_{[4]}(\F_{2^n})=\left\{\begin{array}{rl}
{2^n}({2^n-1}) & \text{if $4|n$}\\
0 & \text{otherwise.}\end{array}\right.
$
\\
\\ \hline\end{tabular}

\end{slide}

\begin{slide}\pageTransitionWipe{30}
\heading{Numerical Examples (5--cycles)}


$$\text{If } q\not\in\{2, 13, 61, 3719, 3100067\} \Rightarrow
m_{[5]}(\F_q)=\frac{q(q-1)}{5}(r_q+t_q+u_q),$$

\begin{scriptsize}
$$t_q=\left\{
\begin{array}{rl}
4 & \text{if $q\equiv1\pmod 5$}\\
1& \text{if $q\equiv0\pmod 5$}\\
0 & \text{otherwise,}\end{array}
\right.
 \ u_q=\left\{
\begin{array}{rl}
-1 & \text{if $p=11,41$}\\
0 & \text{otherwise,}\end{array}
\right.\  r_q=\#\{\genfrac{}{}{0pt}{}{\F_q-\text{roots}}{\text{of } g_2}\}
$$

$$\begin{array}{rl}
g_2(x)=&
2\,{x}^{20}-29\,{x}^{19}+229\,{x}^{18}-1249\,{x}^{17}+5187\,{x}^{16}-17222\,{x}^{15}+\\
&47040\,{x}^{14}-107505\,{x}^{13}+207622\,{x}^{12}-340496\,{x}^{11}+474638\,{x}^{10}-\\
&560999\,{x}^{9}+559052\,{x}^{8}-465487\,{x}^{7}+319628\,{x}^{6}-177653\,{x}^{5}+\\
&77807\,{x}^{4}-25797\,{x}^{3}+6074\,{x}^{2}-904\,x+64.
\end{array}$$

\end{scriptsize}

\end{slide}

\begin{slide}\pageTransitionWipe{30}

\begin{center}
\begin{tabular}{|c|}
\hline
$g_2(\alpha)=0, \sigma_\alpha=(0,1,\alpha,y(\alpha),z(\alpha))
\ \Rightarrow\ \partial f_{\sigma_\alpha}=q-5\text{ (minimal) }$
 \\
\hline\end{tabular}
\bigskip\end{center}

\begin{tiny}
$$\begin{array}{l}\mbox{\normalsize $y(x)=$}
\frac{1}{(2)^{3}(13)(61)(3719)(3100067)}\left(6245340990732510-74275247020348477\,x\right.\\
+425897367479627411\,x^{2}-1556772755104088477\,x^{3}+4068122356423765520\,x^{4}\\
-8092377944341897339\,x^{5}+12739155747072503154\,x^{6}-16281608694400072277\,x^{7}+\\
17191467892889878476\,x^{8}-15176855331347725064\,x^{9}+11289210111615920188\,x^{10}\\
-7103742513094855073\,x^{11}+3782081407301444460\,x^{12}-1696979431552752820\,x^{13}\\
+635807089991226023\,x^{14}-195705738631474759\,x^{15}+48121368022605621\,x^{16}\\
\left.-9009616966592957\,x^{17}+1165803130533438\,x^{18}-82558295396232\,x^{19}\right)
\end{array}$$
$$
\begin{array}{l}\mbox{\normalsize $z(x)=$}
\frac{1}{(2)^{3}(13)(61)(3719)(3100067)}\left(-292290150269490\,{x}^{19}+3950333490943181\,{x}^{18}\right.\\
-29484664428617801\,{x}^{17}+152268243151302965\,{x}^{16}-599002775464475543\,{x}^{15}\\
+1880438345917167218\,{x}^{14}-4841135989461751552\,{x}^{13}+10378374551469856881\,{x}^{12}\\
-18679878403151115130\,{x}^{11}+28303942873286020848\,{x}^{10}-36041151267474587782\,{x}^{9}\\
+38336702176933085823\,{x}^{8}-33711958096174593304\,{x}^{7}+24129466512539278343\,{x}^{6}\\
-13742359416000756136\,{x}^{5}+6020424561116746133\,{x}^{4}-1925677501494324283\,{x}^{3}\\
+413273185040891961\,{x}^{2}-51203861193252214\,x+2593061963570136)
\end{array}
$$\end{tiny}

\end{slide}

\begin{slide}\pageTransitionWipe{30}
\heading{Numerical Examples (6--cycles)}

$$\text{If } p\gg1\ \ \Rightarrow \ \
m_{[6]}(\F_p)=\frac{p(p-1)}{6}(s_1+s_2+s_3+s_4)\ \ \ \text{where}$$
$$s_i=\#\left\{\genfrac{}{}{0pt}{}{\F_q-\text{roots}}{\text{of } f_i
}\right\},$$

$$
\begin{array}{l}
f_1(x)={x}^{2}-3\,x+3\\
f_2(x)={x}^{4}-3\,{x}^{3}+9\,{x}^{2}-9\,x +3\\
f_3(x)={x}^{6}-4\,{x}^{5}+12\,{x}^{4}-22\,{x}^{3}+25\,{x}^{2}-14\,x+3\\
f_4(x)= \ \textbf{Devil's Hat}.
\end{array}
$$
\end{slide}

\begin{slide}\pageTransitionWipe{30}
\heading{Galois Structure of the Silvia set}

\begin{center}
\begin{tabular}{rcl}
$\operatorname{Gal}(\Q(f_1)/\Q))$ & $\cong$ & $\Z/2\Z $
(cyclotomic permutations)\\
\\
$\operatorname{Gal}(\Q(f_2)/\Q))$ & $\cong$ & $D_4$ \\
\\
$\operatorname{Gal}(\Q(f_3)/\Q))$ & $\cong$ & $(\Z/3\Z)^2
\rtimes S_2$\\
\\
$\operatorname{Gal}(\Q(\textit{Devil's Hat}))$ & $\cong$ & ???
\\
\end{tabular}\end{center}

 (exponent probably $=(2)^{5}(3)^{3}(5)(7)(11)(13)(17)$ )


Later discovered that

$$\operatorname{Gal}(\Q(\textit{Devil's Hat}))\leq
({\Z}/6{\Z})^{18}\rtimes S_{18}$$

\end{slide}
\end{document}
