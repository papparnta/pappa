\documentclass[landscape]{powersem} %,display
\usepackage{fancybox,marvosym,graphicx,amsmath,amssymb,pifont,textcomp}
\usepackage[bookmarksopen,colorlinks,urlcolor=red,pdfpagemode=FullScreen]{hyperref}
\usepackage{fixseminar}
\usepackage[usenames,dvipsnames]{color}
\usepackage[latin1]{inputenc}
\usepackage{eurosans}
\usepackage[%coloremph
,colormath%,colorhighlight
,whitebackground]{texpower}
\hfuzz=30pt
\vfuzz=30pt
\setlength{\slidewidth}{25cm} \setlength{\slideheight}{17cm}
\slideframe{}%shadow
\def\slideitemsep{.5ex plus .3ex minus .2ex}
\renewcommand{\slidetopmargin}{10mm}
\renewcommand{\slidebottommargin}{15mm}
\renewcommand{\slideleftmargin}{5mm}
\renewcommand{\sliderightmargin}{5mm}
\newcommand{\psd}{\pause}%\addtocounter{slide}{-1}}
\newcommand{\Ccal}{{\mathcal{C}}}
\newcommand{\Exp}{\operatorname{Exp}}
\newcommand{\F}{{\mathbb{F}}}
\newcommand{\C}{{\mathbb C}}
\newcommand{\Q}{{{\mathbb Q}}}
\newcommand{\Z}{{\mathbb Z}}
\newcommand{\N}{{\mathbb N}}
\newcommand{\manorossa}{\textcolor{conceptcolor}{\ding{43}}}
\newcommand{\matitablu}{\textcolor{altemcolor}{\ding{46}}}
\newcommand{\verde}{\textcolor{black}}
\newcommand{\heading}[1]{%
 \begin{center}
  %\large\bf
  \Ovalbox{{#1}}%\textcolor{conceptcolor}{
 \end{center}
 \vspace{1ex minus 1ex}}
\definecolor{verdescu}{rgb}{0,0.6,0.6}
\definecolor{rossoscu}{rgb}{1,0,0.2}

%\backgroundstyle[startcolor=white,
 %                  endcolor=inactivecolor,%firstgradprogression=3,
  %          rightpanelwidth=-7\semcm,,rightpanelcolor=pagecolor]{hgradient}%
%%%%%%%%%%%%% DATI DEL SEMINARIO IN QUESTIONE %%%%%%%%%%%%

\newpagestyle{327}%
 {\textcolor{codecolor}{\textit{Basic Algorithms in Number Theory}} \hspace{\fill}\rightmark
\hspace{0.5cm}\thepage}
 {}%{\includegraphics[width=4mm]{images/dipmat.pdf}\hspace{\fill}\textcolor{codecolor}{\sc Universit\`a Roma Tre}
 %\hspace{\fill}\includegraphics[width=5mm]{images/roma3.pdf}}%%
\pagestyle{327} \markright{\textcolor{conceptcolor}{Algorithmic Complexity ...}}

\begin{document}

%\begin{slide}\pageTransitionWipe{30}
%\maketitle
%\end{slide}

\begin{slide}\ \vfill

\begin{sc}\begin{center}
\begin{Large}

\textcolor{underlcolor}{Basic Algorithms in Number Theory}
\end{Large}\bigskip

\ {Francesco Pappalardi}\bigskip\bigskip

\begin{large}\begin{bf}Polynomials, Hensel's Lemma, Chinese Remainder Theorem and more.
\end{bf}\end{large}\medskip

July $22^{\textrm{th}}$ 2010\medskip
\vfil

\end{center}
\end{sc}
\vfill
\end{slide}

\begin{slide}
\heading{Let's play with $2^{2067}+131$}

Let $p=2^{2067}+131$. Is it prime? \pause

Do we believe Mathematica?\pause

\textcolor{red}{No we do not believe her!!!\\
So let us check it with Solovay Strassen (from yesterday Lab)}\pause

\textcolor{black}{\textbf{Exercise:} Check that she is right with Miller-Rabin Test.}\pause

Can we prove that certainly $p$ is prime maybe by factoring $p-1$?\pause

Answer: NOWAY!!\pause

\textbf{We want to compute the square root of $5\bmod p$}\pause
Can we do it? We ask Mathematica.\pause
Yes, so let us have a look at the slide about it on Lecture 2. 
\end{slide}

\begin{slide}

\heading{\textcolor{red}{\textbf{PROBLEM 9.}} \textsc{Square Roots Modulo a prime:}} 

\fbox{Given an odd prime $p$ and a quadratic residue $a$, find $x$
s. t. $x^2\equiv a\bmod p$}

It can be solved efficiently if we are given a quadratic nonresidue $g\in(\Z/p\Z)^*$

\begin{enumerate}
\item We write $p-1=2^k\cdot q$ and we know that $(\Z/p\Z)^*$ has a (cyclic) subgroup $G$ 
with $2^k$ elements
\item Note that $b=g^q$ is a generator of $G$ and that $a^q\in G$
\item Use the Pohlig-Hellmann Algorithm to compute $t$ such that $a^q=b^t$. 
\item Finally set $x=a^{(p-q)/2}b^{t/2}$ and observe that\\
$\hspace*{3cm}\displaystyle{x^2=a^{(p-q)}b^{t}=a^p\equiv a\bmod p.}$
\end{enumerate}


\end{slide}


\begin{slide}
\heading{Solution of $X^2\equiv5(\bmod 2^{2067}+131)$}

The first thing we need is a quadratic residue modulo $p$ and
we ask Mathematica.\pause

\textbf{Exercise:} Find the least quadratic non residue.\pause

Now we observe that $p-1=2\times q$ with $q$ odd so that $q=(p-1)/2$.\pause

Hence Part 2. is easy since $b=g^{(p-1)/2}\equiv p-1\bmod p$ and what about
$5^{(p-1)/2}$?\pause

We do NOT ask Mathematica since we know that it is one!\pause 

Therefore $t=0$ (even as expected) and

$x=5^{(p-q)/2}(-1)^{t/2}\bmod p$ DONE!\pause

\textbf{Exercise (To do in Mathematica).} Compute the roots of $X^2\equiv 6(\bmod 2^{2067}+2949)$ 
and of $X^2\equiv 10(\bmod 2^{2067}+2949)$

\end{slide}

\begin{slide}
\heading{Polynomials in $(\Z/n\Z)[X]$}

A polynomial $f\in(\Z/n\Z)[X]$ is 
$$f(X)=a_0+a_1X+\cdots+a_kX^k\quad\textbf{where}\quad a_0,\ldots,a_k\in\Z/n\Z$$\pause
The degree of $f$ is $\deg f=k$ when $a_k\neq0$.\pause
\noindent\textbf{Example:} If $f(X)=5+10X+21X^3\in\Z[x]$, then we can ``reduce'' it modulo
$n$. So\vspace*{-3mm}
$$f(X)\equiv X^3\bmod 5\quad\text{ which is the same as saying:} f(X)=X^3\in\Z/5\Z[X].$$ 
$$f(X)\equiv 2+X\bmod 3\quad\text{ which is the same as saying:} f(X)=2+X\in\Z/3\Z[X].$$ 
$$f(X)\equiv 5+3X\bmod 7\quad\text{ which is the same as saying:} f(X)=5+3X\in\Z/7\Z[X].$$\pause
For the time being we restrict ourselves to the case of $f\in\Z/p\Z[X]$. The fact that
$\Z/p\Z$ is a field is important. (Notation $\F_p=\Z/p\Z$ to remind us this)\pause
We can add, subtract and multiply polynomials in $\F_p[X]$.
\end{slide}

\begin{slide}
\heading{Polynomials in $\F_p[X]$}

We can also divide them!! for $f, g\in\F_p[X]$ there exists $q$, $r\in\F_p[X]$ such that
$$f=qg+r\quad\text{and}\quad \deg r<\deg g.$$
\pause

\textbf{Example:} Let $f=X^3+X+1, g=X^2+1\in\F_3[X]$. Then
$$X^3+X+1=(X^2+X+2)(X+1)+2\quad\text{ so that }q=X^2+X+2, r=2$$
\pause

In Mathematica:\\
 \begin{tt}PolynomialQuotientRemainder[x\^\ 3 + x + 1, x + 1, x, Modulus -> 3]\end{tt} finds
$p$ and $r$.

\end{slide}

\begin{slide}
\heading{Polynomials in $\F_p[X]$}

The complexity for summing or subtracting $f,g\in\F_p[X]$ with $\max\{\deg f,\deg g\}<n$,
is $O(\log p^n)$. Why?\pause

The complexity of multiplying or dividing $f,g\in\F_p[X]$ with $\max\{\deg f,\deg g\}<n$,
can be shown to be $O(\log^2(p^n))$. \pause

\textcolor{red}{Important difference:} Polynomials in $\F_p[X]$ are not invertible except when they
are constant but not zero. So $\F_p[X]$ looks much more like $\Z$ than like $\Z/m\Z$.

But if $f,g\in\F_p[X]$, the $\gcd(f,g)$ exists and it is fast to calculate!!! why?\pause

YES! The EEA also applies to $\F_p[X]$ (Indeed it applies when there is a true division)

\end{slide}

\begin{slide}
\heading{Polynomials in $\F_p[X]$}

\textbf{Example} Let $f=X^3+X^2+X+1$, $g=X^3+X+1\in\F_2[X]$, Then
\begin{itemize}
	\item $f=1(g)+ X^2$;
        \item $g=X(X^2)+X+1$;
        \item $X^2=(X+1)(X+1)+1$;
        \item $X+1=(X+1)1+0$.
\end{itemize}

So the sequence of quotients are $1, X, X+1, X+1\in\F_2[X]$ and we can apply the recursions to compute
the Bezout Identity.

However in Mathematica:\\
 \begin{tt}PolynomialGCD[(x+1)\^~3,x\^~3+x, Modulus -> 2] \\
PolynomialExtendedGCD[1+X+X\^~2+X\^~3,1+X+X\^~3, Modulus -> 2]\end{tt}
\end{slide}

\begin{slide}
\heading{Polynomials in $\F_p[X]$}

As in $\Z$ every $f\in\F_p[X]$ can be written as the product of irreducible polinomials.\pause

Mathematica Knows how to do it:\\
\begin{tt}Factor[x\^~3-3x\^~2-2x+6,Modulus -> 3]\end{tt}\pause

The polynomial $X^p-X\in\F_p[X]$ is very special. What is its factorization?\pause
 
$$X^p-X=\prod_{a\in\F_p}(X-a)\in\F_p[X].$$

Why is it true?\pause

FLT says that $a^p=a, \forall a\in\F_p$. Let's Look at one example.
\end{slide}



\begin{slide}
\heading{\textcolor{red}{\textbf{PROBLEM 12.}} \textsc{Irreducibility Test for Polynomials in $\F_p$:}} 
Given $f\in\F_p[X]$, determine if $f$ is irreducible:\pause
\textbf{Theorem.} \textit{Let $X^{p^n}-X\in\F_p[X]$. Then
$$X^{p^n}-X =\prod_{\substack{f\in\F_p[X]\\ f\text{irreducible}\\
f\text{ monic} \\ \deg f\text{ divides }n}}f$$}\pause\vspace*{-3mm}
We cannot prove it here but we deduce an algorithm:\vspace*{-3mm}
\begin{center}
\fbox{
\textcolor{black}{
\begin{minipage}[c]{11cm}
\texttt{\noindent 
\textcolor{red}{Input:} $f\in \F_p[X]$ monic\\
\textcolor{blue}{Output:} ``\textcolor{Brown}{Irreducible}'' or ``\textcolor{Brown}{Composite}''\\
1. $n:=\deg f$\\
2. For $j = 1,\ldots, \lceil n/2\rceil$\\
\hspace*{2cm} \qquad if $\gcd(X^{p^j}-X,f)\neq1$ then\\
\hspace*{2cm} \qquad\qquad Output ``\textcolor{Brown}{Composite}'' and halt.\\
3. Output ``\textcolor{Brown}{Irreducible}''.} 
\end{minipage}}}
\end{center}\vspace*{-5mm}
\end{slide}



\begin{slide}
\heading{Polynomial equations modulo prime and prime powers}

Often one considers the problem of finding roots of polynomial $f\in\Z/n\Z[X]$.\pause
When $n=p$ is prime then one can exploit the extra properties coming from the identity
$$X^p-X=\prod_{a\in\F_p}(X-a)\in\F_p[X].$$\pause
From this identity it follows that $\gcd(f,X^p-X)$ is the product of liner factor $(X-a)$ where
$a$ is a root of $f$.\pause
Similarly we have that
$$X^{(p-1)/2}-1=\prod_{\substack{a\in\F_p\\ \left(\frac ap\right)=1}}(X-a)\in\F_p[X].$$\pause

This identity suggests the Cantor Zassenhaus Algorithm 
\end{slide}


\begin{slide}
\heading{Cantor--Zassenhaus Algorithm}
\begin{center}\fbox{\textcolor{black}{
\begin{minipage}[c]{11cm}
\texttt{\noindent 
\textcolor{black}{CZ$(p)$}\\
\textcolor{red}{Input:} a prime $p$ and a polynomial $f\in\F_p[X]$\\
\textcolor{blue}{Output:} a list of the roots of $f$\\
1. $f:=\gcd(f(X),X^p-X)\in\F_p[X]$\\
2. If $\deg(f)=0$ Output ``NO ROOTS''\\
3. If $\deg(f)=1$,\\
\hspace*{5mm} \quad Output the root of $f$ and halt\\
4. Choose $b$ at random in $\F_p$\\
\hspace*{5mm} \quad $g:=\gcd(f(X),(X+b)^{(p-1)/2})$\\
\hspace*{5mm} \quad If $0<\deg(g)<\deg(f)$\\
\hspace*{5mm} \quad Output $CZ(g)\cap CZ(f/g)$\\
\hspace*{5mm} \quad Else goto step 3}
\end{minipage}}}
\end{center}\pause\vspace*{-3mm}
The algorithm is correct since $f$ in (\begin{tt}Step 4\end{tt}) is the product of $(X-a)$ ($a$ root of $f$). So $g$ is
the product of $X-a$ with $a+b$ quadratic residue. \\
CZ$(p)$ has polynomial (probabilistic) complexity in $\log p^n$. %($f$ has size $O(\log p^n)$).
\end{slide}

\begin{slide}
\heading{Polynomial equations modulo prime powers}

There is an explicit contruction due to Kurt Hensel that allows to ``lift'' a solution of
$f(X)\equiv 0\bmod p^n$ to a solution of $f(X)\equiv0\bmod p^{2n}$.\pause

\textcolor{blue}{Example: (Square Roots modulo Odd Prime Powers.} Suppose $x\in\F_p$ is a square root of $a\in\F_p$ .
\pause
Let $y= (x^2+a)/2x\bmod p^2$ ($y$ is well defined since $\gcd(2x,p^2)=1$). Then
$$y^2 -a= \frac{(x^2-a)^2}{4x^2}\equiv0\bmod p^2$$
since $p^2$ divides $(x^2-a)^2$.\pause

The general story if the famous Hensel's Lemma.
\end{slide}

\begin{slide}
\heading{Polynomial equations modulo prime powers}

\noindent\textbf{Theorem} (\textsc{Hensel's Lemma}). \textit{Let $p$ be a prime, $f(X)\in\Z[X]$ and $a\in\Z$ such that
$$f(a)\equiv0\bmod p^k,\qquad f'(a)\not\equiv0\bmod p.$$
Then $b:= a-f(a)/f'(a)\bmod p^{2k}$ is the unique integer modulo $p^{2k}$ that satisfies
$$f(b)\equiv0\bmod p^{2k},\qquad b\equiv a\bmod p^k.$$}\pause
\textsc{Proof.} Replacing $f(x)$ by $f(x+a)$ we can restric to $a=0$. Then
$$f(X)=f(0)+f'(0)X+ h(X)X^2\quad \text{where }h(X)\in\Z[X].$$
Hence if $b\equiv 0\bmod p^k$, then $f(b)\equiv f(0)+bf'(0)\bmod p^{2k}$. Finally $b=-f(0)/f'(0)$ is the
unique lift of $0$ modulo $p^{2k}$ that satisfies $f(b)\equiv0\bmod p^{2k}.\square$

\end{slide}

\begin{slide}
\heading{Chinese Remainder Theorem}

 \textsc{Chinese Remainder Theorem.} \textit{Let $m_1,\ldots,m_s\in\mathbb N$
pairwise coprime and let $a_1,\ldots,a_s\in\mathbb Z$. Set $M=m_1\cdots m_s$. There exists 
a unique $x\in\mathbb Z/M\mathbb Z$ such that
$$
\begin{cases}
x\equiv a_1\bmod m_1\\
x\equiv a_2\bmod m_2\\
\ \ \ \vdots\\
x\equiv a_s\bmod m_s.
\end{cases}
$$
Furthermore if $a_1,\ldots,a_s\in\mathbb Z/M\mathbb Z$, then $x$ can be computed in time $O(s\log^2 M).$}
\end{slide}

\begin{slide}
\heading{Chinese Remainder Theorem continues}

\noindent\textsc{Proof.} Let us first assume that $s=2$. Then from EEA we can write
$1=m_1x+m_2y$ for  appropriate $x, y\in\Z$. Consider the integer
$$c=a_1m_2y+a_2m_1x.$$
Then $c\equiv a_1\bmod m_1$ and $a\equiv a_2\bmod m_2$. Furthermore if $c'$ has the same 
property, then $d=c-c'$ is divisible by $m_1$ and $m_2$. Since $\gcd(m_1,m_2)=1$ we have that
$m_1m_2$ divides $d$ so that $c\equiv c'\bmod m_1m_2.$\\
If $s>2$ then we can iterate the same process and consider the system:
$$
\begin{cases}
x\equiv c\bmod m_1m_2\\
x\equiv a_3\bmod m_3\\
\ \ \ \vdots\\
x\equiv a_s\bmod m_s.
\end{cases}.\quad\square
$$
In Mathematica,
\texttt{ChineseRemainder[$\{3, 4\}, \{4, 5\}$]} coincides with
$
\begin{cases}
x\equiv 3\bmod 4\\
x\equiv 4\bmod 5
\end{cases}
$

\end{slide}

\begin{slide}
\heading{Chinese Remainder Theorem (applications)}

It can be used to prove the multiplicativity of the Euler $\varphi$ function. More precisely,
it implies that, if $\gcd(m,n)=1$, then the map:
$$(\Z/mn\Z)^*\rightarrow (\Z/m\Z)^*\times(\Z/n\Z)^*, a\mapsto (a\bmod m,a\bmod n)$$
is surjective. 
\pause
It can be used to glue solutions of congruence equations.\pause
Let $f\in\Z[X]$ and suppose that $a, b\in\Z$ are such that
$$f(a)\equiv(\bmod n),\quad f(b)\equiv(\bmod m).$$
If $\gcd(m,n)=1$, then a solution $c$ of 
$$
\begin{cases}
x\equiv a\bmod n\\
x\equiv b\bmod m\\
\end{cases}
$$
has the property that $f(c)\equiv0(\bmod nm)$.
\end{slide}

\begin{slide}
\heading{Algorithms to be implemented in Mathematica (Lectures 1)}

\begin{tt}
\begin{enumerate}
	\item Right-to-Left Exponentiation in $\Z/m\Z$
	\item Left-to-Right Exponentiation in $\Z/m\Z$
	\item Test of Primality using the factorization of $n-1$ 
	\item Computation of Legendre/Jacobi Symbols (via recursive algorithm)
	\item Solovay Strassen probabilistic Primality Test
	\item Probabilistic Search of Quadratic Nonresidues
        \item Deterministic Search of Quadratic Nonresidues
        \item Power test via the newton Method
	\item Miller Rabin probabilistic primality test
	\item Implementation of RSA
	\item Pollard $\rho$ method and $n-1$ method
\end{enumerate}
\end{tt}

\end{slide}

\begin{slide}
\heading{Algorithms to be implemented in Mathematica (Lectures 2/3)}

\begin{tt}
\begin{enumerate}
	\item Search for primitive root in $n = 2; 4; p^\alpha; 2 p^\alpha$ (with resident commands)
	\item Shank's BSGS for Discrete Logs
	\item Pohlig-Hellman Algorithm for groups with $|G|=2^\alpha$.
	\item Algorithm to compute square root modulo a prime
	\item Binary Euclidean Algorithms
	\item Extended Euclidean Algorithm (EEA) for Bezout identity
	\item Cantor--Zassenhaus Algorithm
        \item Lifting roots modulo powers of primes
	\item Chinese Remainder Theorem
	\item \textit{Finite fields on Mathematica}
	\item \textit{Elliptic curves in Mathematica}
	\item \textit{The Riemann Zeta function in Mathematica}
\end{enumerate}
\end{tt}

\end{slide}
\end{document}
