\documentclass[landscape,handout]{powersem} %,display

\usepackage{fancybox,marvosym,graphicx,amsmath,amssymb,pifont}
\usepackage[bookmarksopen,colorlinks,urlcolor=red]{hyperref} %,pdfpagemode=FullScreen
\usepackage{fixseminar}
\usepackage{color}
\usepackage[latin1]{inputenc}
\usepackage[coloremph,colormath,colorhighlight,lightbackground]{texpower}
\hfuzz=30pt \vfuzz=30pt \setlength{\slidewidth}{25cm}
\setlength{\slideheight}{17.5cm} \slideframe{}
\def\slideitemsep{.5ex plus .3ex minus .2ex}
\renewcommand{\slidetopmargin}{10mm}
\renewcommand{\slidebottommargin}{15mm}
\renewcommand{\slideleftmargin}{5mm}
\renewcommand{\sliderightmargin}{5mm}
\newcommand{\heading}[1]{%
 \begin{center}
  \large\bf
  \shadowbox{{\textcolor{conceptcolor}{#1}}}%
 \end{center}
 \vspace{1ex minus 1ex}}
\backgroundstyle[startcolor=white,
                   endcolor=grey,%firstgradprogression=3,
            rightpanelwidth=-7\semcm,,rightpanelcolor=pagecolor]{hgradient}%
%%%%%%%%%%%%% DATI DEL SEMINARIO IN QUESTIONE %%%%%%%%%%%%

\newpagestyle{327}%
 {\textcolor{codecolor}{\tiny{\textit{Factoring integers,..., RSA}}} \hspace{\fill}\rightmark
Erbil, Kurdistan\hspace{0.3cm}\thepage}
 {\includegraphics[width=4mm]{images/dipmat.pdf}\hspace{\fill}\tiny{\textcolor{codecolor}{\sc Universit\`a Roma Tre}}
 \hspace{\fill}\includegraphics[width=5mm]{images/roma3.pdf}}%%
\pagestyle{327}

\begin{document}


\begin{slide}\pageTransitionWipe{30}
\addtocounter{slide}{-1}
\includegraphics[width=1.3cm]{images/crypto.jpg}\ \hfill \includegraphics[width=1.3cm]{images/crypto.jpg}
\vfil

\begin{sc}\begin{center}
\vfil
\small{
\textbf{Lecture in Number Theory}\\ \emph{
College of Sciences\\ Department of Mathematics\\ University of Salahaddin}}

Debember 1, 2014
\vspace*{1cm}

\begin{Large}
\textcolor{underlcolor}{Factoring integers, Producing primes and the RSA cryptosystem}
\end{Large}
\vfill
Francesco Pappalardi
\end{center}
\end{sc}

\vfill

\includegraphics[width=1.3cm]{images/crypto.jpg}\ \hfill \includegraphics[width=1.3cm]{images/crypto.jpg}
\end{slide}


\begin{slide}%\pageTransitionWipe{30}
\heading{How large are large numbers?}\pause

\parstepwise{\begin{itemize}
  \item[\textcolor{blue}{\ding{43}}]  \step{\textsc{number of cells in a human body:}\hspace*{2.4cm}$10^{15}$}
\bigskip\bigskip
  \item[\textcolor{blue}{\ding{43}}]  \step{\textsc{number of atoms in the universe:}\hspace*{3.685cm}$10^{80}$}
\bigskip\bigskip
  \item[\textcolor{blue}{\ding{43}}]  \step{\textsc{number of subatomic particles in the universe:}\hspace*{2.04cm}$10^{120}$}
\bigskip\bigskip
  \item[\textcolor{blue}{\ding{43}}]  \step{\textsc{number of atoms in a Human Brain:}\hspace*{2.5cm}$10^{27}$}
\bigskip\bigskip
  \item[\textcolor{blue}{\ding{43}}]  \step{\textsc{number of atoms in a cat:}\hspace*{3.95cm}$10^{26}$}
\end{itemize}}
\end{slide}


\begin{slide}\pageTransitionWipe{30}

\begin{center}
\begin{small}
\ $RSA_{2048}$ = 25195908475657893494027183240048398571429282126204
032027777137836043662020707595556264018525880784406918290641249
515082189298559149176184502808489120072844992687392807287776735
971418347270261896375014971824691165077613379859095700097330459
748808428401797429100642458691817195118746121515172654632282216
869987549182422433637259085141865462043576798423387184774447920
739934236584823824281198163815010674810451660377306056201619676
256133844143603833904414952634432190114657544454178424020924616
515723350778707749817125772467962926386356373289912154831438167
899885040445364023527381951378636564391212010397122822120720357
\end{small}\end{center}\pause
\bigskip

\centerline{$RSA_{2048}$ is a 617 (decimal) digit number}\pause

\bigskip

\heading{\begin{small}\texttt{http://www.rsa.com/rsalabs/challenges/factoring/numbers.html/}\end{small}}
\end{slide}

\begin{slide}\pageTransitionWipe{30}

\centerline{$RSA_{2048}$=$p\cdot q$,\ \ \ \   $p,q\approx
10^{308}$}\pause


\heading{{\bf PROBLEM:} {\it Compute $p$ and $q$}}\pause

\centerline{\textcolor{black}{\textsc{Price offered on MArch 18, 1991:}}
 200.000 US\$ ($\sim
232.700.000$ Iraq Dinars)!!}
\bigskip\pause

\begin{center}
\begin{tabular}{|c|}
\hline \textbf{\textcolor{red}{Theorem.}} If $a\in{\mathbb N}$ \ \ $ \exists!\  p_1<p_2<\cdots<p_k$\ \textit{primes}\\
$ \ \ \textrm{s.t.} \ \ a=p_1^{\alpha_1}\cdots
p_k^{\alpha_k}$\\\hline\end{tabular}
\end{center}\pause
\bigskip

\textbf{Regrettably:} RSAlabs believes that factoring in one year
requires: \center{\begin{tabular}{|c|c|c|}\hline
number & computers & memory \\
$RSA_{1620}$ & $1.6\times10^{15}$&  $120$ Tb\\
$RSA_{1024}$ & $342,000,000$ &  $170$ Gb\\
$RSA_{760}$  & 215,000  & $4$Gb.\\ \hline
\end{tabular}}

\end{slide}

\begin{slide}\pageTransitionWipe{30}


\heading{\small{http://www.rsa.com/rsalabs/challenges/factoring/numbers.html}}\pause


\begin{center}
\begin{tabular}{|c|c|}\hline
 \textcolor{blue}{Challenge Number} & \textcolor{blue}{Prize (\$US)}  \\
$RSA_{576}$ &  \$10,000   \\
$RSA_{640}$ &  \$20,000    \\
$RSA_{704}$ &     \$30,000 \\
$RSA_{768}$ &     \$50,000 \\
$RSA_{896}$ &     \$75,000 \\
$RSA_{1024}$ &     \$100,000 \\
$RSA_{1536}$ &  \$150,000 \\
$RSA_{2048}$ &     \$200,000   \\
\hline
\end{tabular}
\end{center}\vfill
\end{slide}

\begin{slide}\pageTransitionWipe{30}
\addtocounter{slide}{-1}

\heading{\small{http://www.rsa.com/rsalabs/challenges/factoring/numbers.html}}


\begin{center}
\begin{tabular}{|c|c|c|}\hline
 \textcolor{blue}{Numero} & \textcolor{blue}{Premio (\$US)} & \textcolor{blue}{Status} \\
$RSA_{576}$ &  \$10,000  & Factored December 2003\\
$RSA_{640}$ &  \$20,000   & Factored November 2005\\
$RSA_{704}$ &     \$30,000&  Factored July, 2 2012\\
$RSA_{768}$ &     \$50,000&   Factored  December, 12 2009\\
$RSA_{896}$ &     \$75,000& Not factored\\
$RSA_{1024}$ &     \$100,000&  Not factored\\
$RSA_{1536}$ &  \$150,000 & Not factored\\
$RSA_{2048}$ &     \$200,000 &  Not factored\\
\hline
\end{tabular}
\end{center}\pause\bigskip

\scriptsize{
The RSA challenges ended in 2007.  RSA Laboratories stated:\\
{\small ``\textit{Now that the industry has a considerably more advanced understanding of the 
cryptanalytic strength of common symmetric-key and public-key algorithms, these challenges are no longer active.}''}}
\end{slide}

\begin{slide}\pageTransitionWipe{30}
\heading{Famous citation!!!}\bigskip

\centerline{\includegraphics[width=4cm]{images/borel1.jpg}}

\textit{A phenomenon whose probability is  $10^{-50}$ never happens, and it will never observed.}\bigskip

\textsc{- \'Emil Borel (Les probabilit\'es et sa vie)}
\end{slide}



\begin{slide}\pageTransitionWipe{30}
\heading{History of the ``\emph{Art of Factoring}''}\pause

\begin{itemize}
\item[\textcolor{black}{\ding{243}}] 220 BC Greeks (Eratosthenes of Cyrene )\pause
\item[\textcolor{black}{\ding{243}}] 1730 Euler $2^{2^5}+1=641\cdot 6700417 $\pause
\item[\textcolor{black}{\ding{243}}] 1750--1800 Fermat, Gauss (Sieves - Tables)\pause
\item[\textcolor{black}{\ding{243}}] 1880 Landry \& Le Lasseur:
$$2^{2^6}+1= 274177 \times 67280421310721$$\pause
\item[\textcolor{black}{\ding{243}}] 1919 Pierre and Eug\`ene Carissan (Factoring Machine)\pause
\item[\textcolor{black}{\ding{243}}] 1970 Morrison \& Brillhart
$$2^{2^7}+1=
59649589127497217 \times 5704689200685129054721 $$\pause
\item[\textcolor{black}{\ding{243}}] 1982 Quadratic Sieve \textbf{QS}
(Pomerance)\hfil $\rightsquigarrow$\hfil Number Fields Sieve \textbf{NFS}\pause
\item[\textcolor{black}{\ding{243}}] 1987 Elliptic curves factoring \textbf{ECF} (Lenstra)
\end{itemize}
\end{slide}

\begin{slide}\pageTransitionWipe{30}
\heading{History of the ``\emph{Art of Factoring}''}
\bigskip\bigskip

\begin{center}
\includegraphics{images/eratostene.jpg}

220 BC Greeks (Eratosthenes of Cyrene)
\end{center}
\end{slide}

\begin{slide}\pageTransitionWipe{30}
\heading{History of the ``\emph{Art of Factoring}''}
\bigskip\bigskip

\begin{center}
\includegraphics[width=5cm]{images/Euler_9.jpeg}

1730 Euler $2^{2^5}+1=641\cdot 6700417 $
\end{center}
\end{slide}

 \begin{slide}
\heading {How did Euler factor $2^{2^5}+1$?}\pause 

\noindent\textsc{Proposition} \textit{Suppose $p$ is a prime factor of $b^n+1$. Then
\begin{enumerate}
 \item $p$ is a divisor of $b^{d}+1$ for some proper divisor $d$ of $n$ such that $n/d$ is odd or 
 \item $p-1$ is divisible by $2n$.
\end{enumerate}}\medskip
\medskip\pause

\noindent\textcolor{red}{\textit{Application:}} Let $b=2$ and $n=2^5=64$. Then $2^{2^5}+1$ is prime or it is divisible by
a prime $p$ such that $p-1$ is divisible by $128$. \pause

Note that \\
$1+1\times128=3\times43$, $1+2\times128=257$ is prime,\\ 
$1+3\times128=5\times7\times11$, $1+4\times 128=3^3\times19$ and $1+5\cdot 128=641$
is prime.

Finally
$$\frac{2^{2^5}+1}{641}=\frac{4294967297}{641}=6700417$$

\end{slide}


\begin{slide}\pageTransitionWipe{30}
\heading{History of the ``\emph{Art of Factoring}''}
\bigskip\bigskip

\begin{center}
\includegraphics[width=7cm]{images/euler1.jpg}

1730 Euler $2^{2^5}+1=641\cdot 6700417 $
\end{center} 
\end{slide}

\begin{slide}\pageTransitionWipe{30}
\heading{History of the ``\emph{Art of Factoring}''}

\begin{center}
\includegraphics[width=4.5cm]{images/fermat.jpg}
\includegraphics[width=4.5cm]{images/Gauss_1803.jpeg}

1750--1800 Fermat, Gauss (Sieves - Tables)
\end{center} 
\end{slide}

\begin{slide}\pageTransitionWipe{30}
\heading{History of the ``\emph{Art of Factoring}''}

\begin{center}
\includegraphics[width=5cm]{images/fermatstamp.jpg}
\includegraphics[width=5cm]{images/Gauss_banknote.jpeg}

1750--1800 Fermat, Gauss (Sieves - Tables)\pause

Factoring with sieves
$N=x^2-y^2=(x-y)(x+y)$
\end{center} 
\end{slide}


\begin{slide}\pageTransitionWipe{30}
\heading{Carissan's ancient Factoring Machine}\pause


\begin{figure}
  \centering
\includegraphics[width=5cm]{images/cari.jpg}
  \caption{Conservatoire Nationale des Arts et M\'etiers in Paris}
\end{figure}\pause

\heading{\begin{small}\texttt{http://www.math.uwaterloo.ca/~shallit/Papers/carissan.html
}\end{small}}

\end{slide}

\begin{slide}\pageTransitionWipe{30}
\begin{figure}
  \centering \includegraphics[width=3cm]{images/caris2.jpg}
  \caption{Lieutenant Eug\`ene Carissan}\end{figure}\pause
\centerline{\begin{tabular}{rcl}
$225058681=229\times982789$ & &{2 minutes}\\
$3450315521=1409\times 2418769$ & & {3 minutes}\\
$3570537526921=841249\times4244329$ & & {18 minutes}\\
\end{tabular}}

\end{slide}

\begin{slide}\pageTransitionWipe{30}
\heading{State of the ``\emph{Art of Factoring}''}
\bigskip

\begin{center}
\includegraphics[width=5cm]{images/brillhart.JPG}

1970 - John Brillhart \&  Michael A. Morrison  $2^{2^7}+1=
59649589127497217 \times 5704689200685129054721 $
\end{center}
\end{slide}

\begin{slide}\pageTransitionWipe{30}
\heading{State of the ``\emph{Art of Factoring}''}
\bigskip

$$   F_{n} = 2^{(2^n)} + 1$$
 is called the $n$--th Fermat number 

\begin{center}\hspace*{-.7cm}
\includegraphics[width=13cm]{images/f11.png}
\end{center}\pause

Up to today only from $F_0$ to $F_{11}$ are factores.\\ It is not known the factorization of  
$$F_{12}=2^{2^{12}}+1$$\pause





\end{slide}



\begin{slide}\pageTransitionWipe{30}
\heading{State of the ``\emph{Art of Factoring}''}
\bigskip\bigskip

\begin{center}
\includegraphics[width=3cm]{images/smallcarl1.jpg}

1982 - Carl Pomerance - Quadratic Sieve
\end{center}
\end{slide}

\begin{slide}\pageTransitionWipe{30}
\heading{State of the ``\emph{Art of Factoring}''}
\bigskip\bigskip

\begin{center}
\includegraphics[width=4cm]{images/Lenstra.jpg}

1987 - Hendrik Lenstra - Elliptic curves factoring
\end{center}
\end{slide}



\begin{slide}\pageTransitionWipe{30}
\heading{Contemporary Factoring}\vspace{-3mm}\pause

\begin{itemize}
  \item[\textcolor{blue}{\ding{182}}] 1994, Quadratic Sieve (QS): (8 months, 600 volunteers, 20 nations)\\
  D.Atkins, M. Graff, A. Lenstra, P. Leyland
\begin{tiny}\begin{tabular}{l}
$  RSA_{129} = 114381625757888867669235779976146612010218296721242362562561842935706$\\
\hspace*{5mm}$935245733897830597123563958705058989075147599290026879543541=$\\
$        = 3490529510847650949147849619903898133417764638493387843990820577 \times$\\
$
32769132993266709549961988190834461413177642967992942539798288533
$\end{tabular}\end{tiny}\pause

  \item[\textcolor{blue}{\ding{183}}] (February 2 1999), Number Field Sieve (NFS): (160 Sun, 4 months)
 \begin{tiny}\begin{tabular}{c}
\hspace*{-1cm}$ RSA_{155} = 109417386415705274218097073220403576120037329454492059909138421314763499842$\\
$88934784717997257891267332497625752899781833797076537244027146743531593354333897=$\\
$=102639592829741105772054196573991675900716567808038066803341933521790711307779
\times$\\
$106603488380168454820927220360012878679207958575989291522270608237193062808643
$\end{tabular}\end{tiny}\pause

  \item[\textcolor{blue}{\ding{184}}] (December 3, 2003) (NFS): J. Franke et al. (174 decimal digits)
 \begin{tiny}\begin{tabular}{c}
\hspace*{-1cm}$ RSA_{576} = 1881988129206079638386972394616504398071635633794173827007633564229888597152346$\\
$65485319060606504743045317388011303396716199692321205734031879550656996221305168759307650257059=$\\
$=398075086424064937397125500550386491199064362342526708406385189575946388957261768583317\times$\\
$472772146107435302536223071973048224632914695302097116459852171130520711256363590397527
$\end{tabular}\end{tiny}\pause

\item[\textcolor{blue}{\ding{185}}]
  Elliptic curves factoring: introduced by  H. Lenstra.
  suitable to detect small factors (50 digits)\pause

\end{itemize}

\centerline{\textcolor{red}{all have "\emph{sub--exponential complexity}"}}
\end{slide}

\begin{slide}\pageTransitionWipe{30}
\heading{The factorization of $RSA_{200}$}

\begin{tiny}

$RSA_{200}=2799783391122132787082946763872260162107044678695542853756000992932612840010$
          $7609345671052955360856061822351910951365788637105954482006576775098580557613$
          $579098734950144178863178946295187237869221823983$\pause


Date: Mon, 9 May 2005 18:05:10 +0200 (CEST) 
From: "Thorsten Kleinjung"
Subject: rsa200 

We have factored RSA200 by GNFS. The factors are

35324619344027701212726049781984643686711974001976\
25023649303468776121253679423200058547956528088349

and

79258699544783330333470858414800596877379758573642\
19960734330341455767872818152135381409304740185467


We did lattice sieving for most special q between 3e8 and 11e8
using mainly factor base bounds of 3e8 on the algebraic side and 18e7 
on
the rational side. The bounds for large primes were $2^{35}$. This produced
26e8 relations. Together with 5e7 relations from line sieving the total
yield was 27e8 relations. After removing duplicates 226e7 relations
remained. A filter job produced a matrix with 64e6 rows and columns,
having 11e9 non-zero entries. This was solved by Block-Wiedemann.

Sieving has been done on a variety of machines. We estimate that
lattice sieving would have taken 55 years on a single 2.2 GHz Opteron 
CPU.
Note that this number could have been improved if instead of the PIII-
binary which we used for sieving, we had used a version of the
lattice-siever optimized for Opteron CPU's which we developed in the 
meantime.
The matrix step was performed on a cluster of 80 2.2 GHz Opterons 
connected via a Gigabit network and took about 3 months.

We started sieving shortly before Christmas 2003 and continued until
October 2004. The matrix step began in December 2004.
Line sieving was done by P. Montgomery and H. te Riele at the CWI, by
F. Bahr and his family.

More details will be given later.

F. Bahr, M. Boehm, J. Franke, T. Kleinjung
\end{tiny}
\end{slide}

\begin{slide}\pageTransitionWipe{30}
\heading{Factorization of $RSA_{768}$}

\centerline{\includegraphics[width=12cm]{images/RSA-768.png}}

\end{slide}


\begin{slide}\pageTransitionWipe{30}

\centerline{\Large{\textcolor{blue}{RSA}}}

\centerline{\includegraphics[width=9cm]{images/rsa-photo.jpg}}

\centerline{Adi Shamir, Ron L. Rivest, Leonard Adleman (1978)}
\end{slide}

\begin{slide}

\centerline{\Large{\textcolor{blue}{RSA}}}

\centerline{\includegraphics[width=9cm]{images/RSA-2003.jpg}}

\centerline{Ron L. Rivest, Adi Shamir, Leonard Adleman (2003)}
\end{slide}

\begin{slide}\pageTransitionWipe{30}
\heading{The RSA cryptosystem}\pause

1978 R. L. Rivest, A. Shamir, L. Adleman (Patent expired in
1998)\pause

\textbf{\textcolor{black}{Problem:}} \emph{Alice wants to send the
message $\mathcal P$ to Bob so that Charles cannot read it}\pause

\begin{center}\begin{large}
\begin{tabular}{|lcr|}\hline
  $A$ (\textsl{Alice}) & $-\!\!\!-\!\!\!-\!\!\!-\!\!\!-\!\!\!-\!\!\!\rightarrow$ & $B$ (\textsl{Bob})\\
   & $\uparrow$ &  \\
   & $C$ (\textsl{Charles}) &  \\ \hline\end{tabular}
\end{large}\end{center}\pause

\parstepwise{\begin{itemize}
  \item[\textcolor{blue}{\ding{182}}] \step{\textsc{Key generation} \hspace{3cm} Bob has to do it}
  \item[\textcolor{blue}{\ding{183}}]  \step{\textsc{Encryption}\ \ \  \hspace{3cm} Alice has to do it}
  \item[\textcolor{blue}{\ding{184}}]  \step{\textsc{Decryption}\ \ \  \hspace{3cm} Bob has to do it}
  \item[\textcolor{blue}{\ding{185}}]  \step{\textsc{Attack}\ \ \ \ \ \ \hspace{3cm}
  Charles would like to do it}
\end{itemize}}
\end{slide}

\begin{slide}\pageTransitionWipe{30}
\heading{\textsc{Bob: Key generation}}\pause

\parstepwise{\begin{itemize}
  \item[\textcolor{blue}{\ding{45}}] \step{\underline{He chooses} \emph{\textcolor{red}{randomly}}  $p$ and $q$ primes \ \ \ \ \ \ \ ($p,q\approx 10^{100}$)}
  \item[\textcolor{blue}{\ding{45}}] \step{\underline{He computes} \ \ $M=p\times q$, $\varphi(M)=(p-1)\times(q-1)$}
  \item[\textcolor{blue}{\ding{45}}] \step{\underline{He chooses} an integer $e$  s.t.}
\item[\ ]\step{\ \ \ \ \ \ $0\leq e\leq \varphi(M)\ \
\textrm{and}\ \  \gcd(e,\varphi(M))=1$} \item[\ ]\step{\ \ \ \ \ \
\scriptsize\textsc{Note.} One could take  $e=3$ and $p\equiv
q\equiv 2\bmod 3$} \item[\ ]\step{\ \ \ \ \ \ \hspace{4cm}\ \ \
Experts recommend $e=2^{16}+1$}
  \item[\textcolor{blue}{\ding{45}}] \step{\underline{He computes} arithmetic inverse $d$ of  $e$ modulo $\varphi(M)$}
\item[\ ] \step{\ \ \ \ \ \ {(i.e. $d\in\mathbb N$ (unique $\leq \varphi(M)$) s.t. $e\times d\equiv 1 \pmod{\varphi(M)}$)}}
  \item[\textcolor{blue}{\ding{45}}] \step{\underline{Publishes} $(M,e)$ \textbf{\textcolor{blue}{public key}} and hides
  \textbf{\textcolor{red}{secret key}} $d$}
\end{itemize}}\pause\bigskip
\center{\textbf{\textcolor{black}{Problem:}} \emph{How does Bob do
all this?}- We will go came back to it!}

\end{slide}



%%%%%%%%%%%%%%%
\begin{slide}\pageTransitionWipe{30}
\heading{Alice: Encryption}\pause

Represent the message ${\mathcal P}$ as an element of  ${\mathbb Z}/M{\mathbb Z}$\pause

(\textcolor{blue}{for example})\fbox{ $A\leftrightarrow 1\ \ \
B\leftrightarrow2\ \ \ C\leftrightarrow3\ \ \ldots\ \
Z\leftrightarrow26\ \ \ AA\leftrightarrow 27 \ldots$}\pause
\begin{scriptsize}
$$\texttt{\textcolor{red}{Sukumar}}\leftrightarrow
19\cdot26^6+21\cdot26^5+11\cdot26^4+21\cdot26^3+12\cdot26^2+1\cdot26+18=
                          6124312628$$ Note. Better if texts are not too short. Otherwise one performs
some  \textsl{padding}
\end{scriptsize}\pause

\begin{Large}
\begin{center}\begin{tabular}{|c|}\hline
${\mathcal C}=E({\mathcal P})={\mathcal P}^e\pmod M$\\\hline\end{tabular}\end{center}
\end{Large}\pause

\scriptsize{ \textcolor{blue}{Example}: $p=9049465727$,
$q=8789181607$, $M=79537397720925283289 $, $e=2^{16}+1=65537$,
${\mathcal P}=\texttt{\textcolor{red}{Sukumar}}$:\pause
$$E(\texttt{\textcolor{red}{Sukumar}})=6124312628^{65537}\,(\bmod 79537397720925283289)$$
$$=25439695120356558116={\mathcal C}=\texttt{\textcolor{black}{JGEBNBAUYTCOFJ}}\hfill$$}
\end{slide}

\begin{slide}\pageTransitionWipe{30}
\heading{Bob: Decryption}\pause

\begin{Large}
\begin{center}\begin{tabular}{|c|}\hline
${\mathcal P}=D({\mathcal C})={\mathcal C}^d\pmod M$\\\hline\end{tabular}\end{center}
\end{Large}\pause

\textbf{\textcolor{blue}{Note.}} Bob decrypts because he is the
only one that knows $d$.\pause

\centerline{\begin{tabular}{|c|}\hline
\textbf{\textcolor{red}{Theorem. (Euler)}} If $a,m\in {\mathbb N}$, $\gcd(a,m)=1$, \\
$a^{\varphi(m)}\equiv 1\pmod m.$\\
If $n_1\equiv n_2\bmod\varphi(m)$ then $a^{n_1}\equiv a^{n_2}\bmod m$.\\
\hline
\end{tabular}}\pause

Therefore ($ed\equiv1\bmod\varphi(M)$)\medskip

\centerline{
\begin{tabular}{|c|}\hline
{\Large $D(E({\mathcal P}))={\mathcal P}^{ed}\equiv {\mathcal P}\bmod M$}\\
\hline
\end{tabular}}\pause

\scriptsize{\textcolor{blue}{Example}(cont.):$d=65537^{-1}\bmod \varphi(9049465727\cdot8789181607)=57173914060643780153$\\
\hspace*{2mm}$D(\texttt{\textcolor{black}{JGEBNBAUYTCOFJ}})=$\\
\hspace*{2mm}$25439695120356558116
^{57173914060643780153}(\bmod79537397720925283289)=\texttt{\textcolor{red}{Sukumar}}$}

\end{slide}

\begin{slide}\pageTransitionWipe{30}

\centerline{\Large{\textcolor{blue}{RSA at work}}}\bigskip\bigskip

\centerline{\includegraphics[width=7cm]{images/rsa.jpg}}

\end{slide}

\begin{slide}\pageTransitionWipe{30}%1/Mod(65537,9049465726*8789181606)
\heading{Repeated squaring algorithm}\pause

\textbf{\textcolor{black}{Problem:}} How does one compute $a^b\bmod
c$?\pause \centerline{$25439695120356558116^{57173914060643780153
}(\bmod 79537397720925283289)$}\pause

\parstepwise{\begin{itemize}
\item[\textcolor{blue}{\ding{45}}]
\step{\underline{Compute the binary expansion}  $b=\displaystyle{\sum_{j=0}^{[\log_2b]}\epsilon_j2^j}$}
\item[\ ]
\step{\scriptsize{$57173914060643780153$}=\tiny{$1100011001011100101000101111101010111100110110001001
00011000111001$}}
\item[\textcolor{blue}{\ding{45}}] \step{\underline{Compute recursively}  $a^{2^j}\bmod c, j=1,\ldots,[\log_2b]$:}
\item[\ ]\step{\ \ \ \ \ $\displaystyle{a^{2^j}\bmod c=\left(a^{2^{j-1}}\bmod c\right)^2\bmod c}$}
\item[\textcolor{blue}{\ding{45}}] \step{\underline{Multiply} the $a^{2^j}\bmod c$ with $\epsilon_j=1$}%\vspace{-2mm}
\item[\ ]\step{\ \ \ \ \ ${a^b\bmod c=\left(\prod_{j=0,\epsilon_j=1}^{[\log_2b]}a^{2^j}\bmod c\right)\bmod c}$}
\end{itemize}}

\end{slide}

\begin{slide}\pageTransitionWipe{30}
\heading{$\#\{\textrm{oper. in\ }  {\mathbb Z}/c{\mathbb Z}\
\textrm{to compute } a^b\bmod c\} \leq 2\log_2b$}\pause

\texttt{\textcolor{black}{JGEBNBAUYTCOFJ}} is decrypted with $131$
operations in {\small$${\mathbb Z}/79537397720925283289\mathbb Z$$}
\medskip\pause

\textsc{\textcolor{blue}{Pseudo code:}} $e_c(a,b)=a^b\bmod c$\pause

\begin{center}
\texttt{\begin{tabular}{|lclcll|}\hline
$e_c(a,b)$ &= &  if   & $b=1$  & then   & $a \bmod c$\\
              &  &  if   &   $2|b$  & then       & $e_c(a,\frac{b}{2})^2 \bmod c$\\
              &  &   else   &        &            & $a*e_c(a,\frac{b-1}{2})^2 \bmod c$\\
              \hline
\end{tabular}}
\end{center}\pause\bigskip

To encrypt with $e=2^{16}+1$, only $17$ operations  in ${\mathbb Z}/M{\mathbb Z}$ are enough
\end{slide}

\begin{slide}\pageTransitionWipe{30}
\heading{Key generation}\pause\bigskip

\textcolor{black}{\textbf{Problem.}} Produce a random prime
$p\approx 10^{100}$

\begin{center}\texttt{
\begin{tabular}{|cl|}\hline
&\textbf{\textcolor{blue}{Probabilistic algorithm (type Las Vegas)}}\\
\hline
  1.& Let $p=\textsc{Random}(10^{100})$\\
  2.& If \textsc{isprime}($p$)=1 then \textsc{Output}=$p$ else goto 1\\
\hline
\end{tabular}
}\end{center}\pause\medskip

\textcolor{black}{\textbf{subproblems:}}\pause

\textsl{\textcolor{red}{A.}} How many iterations are necessary?
\centerline{(i.e. how are primes distributes?)}\pause

\noindent\textsl{\textcolor{red}{B.}} How does one check if $p$ is
prime? \centerline{(i.e. how does one compute
${{\texttt{\textsc{isprime}}(p)}}$?) $\rightsquigarrow$ Primality test}\pause
\bigskip

\centerline{\fbox{\scriptsize{\textit{\textcolor{red}{False Metropolitan Legend:}
 \textcolor{blue}{Check primality is equivalent to factoring}}}}}

\end{slide}


\begin{slide}\pageTransitionWipe{30}
\heading{\textsl{A.} Distribution of prime numbers}\pause

$$\pi(x)=\#\{p\leq x\ \textrm{t. c. } p \textrm{ is prime}\}$$\pause

\begin{center}
\begin{tabular}{|c|}\hline
\textbf{\textcolor{red}{Theorem.}} (Hadamard - de la vallee Pussen - 1897)\\
 $\displaystyle{\pi(x)\sim \frac{x}{\log x}}$\\
\hline\end{tabular}
\end{center}\pause

Quantitative version:

\small{
\begin{center}
\begin{tabular}{|c|}\hline
\textbf{\textcolor{red}{Theorem.}} (Rosser - Schoenfeld) if $x\geq 67$\\
 $\displaystyle{\frac{x}{\log x-1/2}< \pi(x)< \frac{x}{\log x-3/2}}$\\
\hline\end{tabular}
\end{center}}\pause

Therefore

$$ 0.0043523959267
 < Prob\left((\texttt{\textsc{Random}}(10^{100})=\texttt{prime}\right)< 0.004371422086
$$\end{slide}


\begin{slide}\pageTransitionWipe{30}
If $P_{k}$ is the probability that among $k$ random numbers$\leq
10^{100}$ there is a prime one, then\pause
$$P_k=1-\left(1-\frac{\pi(10^{100})}{10^{100}}\right)^k$$\pause

Therefore
$$0.663942
<P_{250}<
 0.66554440
$$
\bigskip\pause

\textcolor{red}{To speed up the process}: One can
consider only odd random numbers not divisible by
$3$ nor by $5$.\pause

Let
$$\Psi(x,30)=\#\left\{n\leq x\ \textrm{s.t.} \gcd(n,30)=1\right\}$$

%1-(1-1/(100*log(10)-3/2))^50
\end{slide}

\begin{slide}\pageTransitionWipe{30}
\textcolor{red}{To speed up the process}: One can
consider only odd random numbers not divisible by
$3$ nor by $5$.\pause

Let
$$\Psi(x,30)=\#\left\{n\leq x\ \textrm{s.t.} \gcd(n,30)=1\right\}$$
then\pause
$$\frac{4}{15}x-4<\Psi(x,30)<\frac{4}{15}x+4$$\pause

Hence, if $P_k'$ is the probability that among $k$ random numbers $\leq 10^{100}$
coprime with 30, there is a prime one, then\pause
$$P'_k=1-\left(1-\frac{\pi(10^{100})}{\Psi(10^{100},30)}\right)^k$$\pause
and
$$0.98365832
<P'_{250}<
 0.98395199$$
\end{slide}

%1-(1-10^100/(100*log(10)-3/2)/(4*10^100/15+4))^250

\begin{slide}\pageTransitionWipe{30}
\heading{\textsl{B.} Primality test}\pause

%\hspace*{2.5cm}$\rightsquigarrow$ \textsc{\textcolor{blue}{Pseudo
%Primi and Pseudo Primi Forti}}

\begin{center}
\begin{tabular}{|c|}
\hline \textbf{\textcolor{red}{Fermat Little Theorem.}} If $p$ is prime, $p\nmid a\in{\mathbb N}$\\
$a^{p-1}\equiv1\bmod p$
\\\hline\end{tabular}
\end{center}\pause
\bigskip

\textbf{\textcolor{blue}{NON-primality test}}
$$M\in{\mathbb Z},\ \ 2^{M-1}\not\equiv 1\bmod M =\!\!\!=\!\!\!>\ \  M \textrm{composite!}$$\pause

\textsc{\textcolor{blue}{Example}:} $2^{RSA_{2048}-1}\not\equiv1\bmod RSA_{2048}$\\
\centerline{Therefore $RSA_{2048}$ is composite!}\pause

Fermat little Theorem does not invert. Infact\pause

$$2^{93960}\equiv 1\pmod{93961}\ \ \ \ \textrm{but}\ \ \ \ 93961
=7\times31\times433$$
 \end{slide}


\begin{slide}\pageTransitionWipe{30}
\heading{Strong pseudo primes}\pause

From now on $m\equiv3\bmod4$ (just to simplify the notation)\pause

{\textbf{\textcolor{red}{Definition.}}} $m\in\mathbb N$,
$m\equiv3\bmod4$, composite is said \emph{\textcolor{black}{strong pseudo prime} (SPSP)} in base $a$ if
$$a^{(m-1)/2}\equiv \pm1\pmod{m}.$$\pause

\textbf{\textcolor{blue}{Note.}} If $p>2$ prime $=\!\!\!=\!\!\!>\ \ a^{(p-1)/2}\equiv \pm1\pmod{p}$\\
%\ \ \ ($x=a^{(p-1)/2}$ is root of the polynomial $x^2-1$ $=\!\!\!=\!\!\!>\ \ x=\pm1$).
Let \fbox{${\mathcal S}=\{a\in {\mathbb Z}/m{\mathbb Z}\ \textrm{s.t.}\ \gcd(m,a)=1, a^{(m-1)/2}\equiv \pm1\pmod{m}\}$}%\\
\pause %$=\!\!\!=\!\!\!>$
\parstepwise{\begin{itemize}
  \item[\textcolor{blue}{\ding{192}}] \step{${\mathcal S}\subseteq ({\mathbb Z}/m{\mathbb Z})^*$ subgroup}
  \item[\textcolor{blue}{\ding{193}}] \step{If $m$ is composite $=\!\!\!=\!\!\!>$ proper subgroup}
  \item[\textcolor{blue}{\ding{194}}] \step{If $m$ is composite $=\!\!\!=\!\!\!>\ \ \ \#{\mathcal S}\leq \frac{\varphi(m)}{4}$}
  \item[\textcolor{blue}{\ding{195}}] \step{If $m$ is composite $=\!\!\!=\!\!\!>\ \ \ Prob(m\ \textrm{PSPF in base }a)\leq 0,25$}
\end{itemize}}
\end{slide}

\begin{slide}\pageTransitionWipe{30}
\heading{Miller--Rabin primality test}\pause

Let $m\equiv 3\bmod 4$\pause

\center{\texttt{\begin{tabular}{|l|}\hline
  % after \\: \hline or \cline{col1-col2} \cline{col3-col4} ...
  \textsc{\textcolor{blue}{Miller Rabin algorithm with $k$ iterations}} \\
\hline
$N=(m-1)/2$\\
for $j=0$ to $k$ do\ \
  $a=$Random$(m)$ \\
if $a^N\not\equiv\pm1\bmod m$ then OUPUT=($m$ composite): END \\
endfor OUTPUT=($m$ prime) \\\hline
\end{tabular}}}\pause

\emph{\textcolor{red}{Monte Carlo primality test}}\pause

$Prob($Miller Rabin says \texttt{$m$ prime} and $m$ is composite)
$\lessapprox\frac{1}{4^k}$\pause

In the real world, software uses Miller Rabin with $k=10$
\end{slide}


\begin{slide}\pageTransitionWipe{30}
\heading{Deterministic primality tests}\pause

\textbf{\textcolor{blue}{Theorem. (Miller, Bach)}} If $m$ is
composite, then \centerline{\textbf{{GRH}} $=\!\!\!=\!\!\!>\
\exists a\leq 2\log^2 m$ s.t.\ \ $a^{(m-1)/2}\not\equiv
\pm1\pmod{m}$.}\\ \ \hfill\textcolor{red}{(i.e. $m \textit{
is not SPSP in base } a.$)}\pause

\small{\textbf{\textcolor{blue}{Consequence:}} ``Miller--Rabin
\emph{de--randomizes on GRH}'' ($m\equiv3\bmod4$)}\pause

\center{\texttt{\begin{tabular}{|lll|}\hline
  % after \\: \hline or \cline{col1-col2} \cline{col3-col4} ...
  %\textsc{\textcolor{blue}{Miller Rabin algorithm with $k$ iterations}} \\
%\hline
for & $a=2$ to $2\log^2 m$ & do\ \ \\
    & if $a^{(m-1)/2}\not\equiv\pm1\bmod m$ & then \\ & & OUPUT=($m$ composite): END \\
endfor & & OUTPUT=($m$ prime) \\\hline
\end{tabular}}}\pause

\emph{\textcolor{red}{Deterministic Polynomial time algorithm}}\pause

\textcolor{blue}{It runs in $O(\log^5m)$ operations in ${\mathbb
Z}/m\mathbb Z$.}

%\textcolor{rossoscu}{(i.e. $m\equiv3\bmod4$ is prime if and
%only if: \fbox{$a^{(m-1)/2}\equiv \pm1 \bmod m\ \ \ \forall a\leq
%2\log^2 m$})}}

\end{slide}


\begin{slide}\pageTransitionWipe{30}
\heading{Certified prime records}\pause



\begin{itemize}
\item[\textcolor{red}{\ding{46}}] {\ $2^{57885161}-1$,\hspace{2cm}  $17425170$ digits (discovered in  01/2014 )}
\item[\textcolor{red}{\ding{46}}] {\ $2^{43112609}-1$,\hspace{2cm}  $12978189$ digits (discovered in  2008)}
\item[\textcolor{red}{\ding{46}}] {\ $2^{42643801}-1$,\hspace{2cm}  $12837064$ digits (discovered in  2009)}
\item[\textcolor{red}{\ding{46}}] {\ $2^{37156667}-1$,\hspace{2cm}  $11185272$ digits (discovered in  2008)}
\item[\textcolor{red}{\ding{46}}] {\ $2^{32582657}-1$,\hspace{2cm}  $9808358$ digits (discovered in  2006)}
\item[\textcolor{red}{\ding{46}}] {\ $2^{30402457}-1$,\hspace{2cm}  $9152052$ digits (discovered in  2005)}
\item[\textcolor{red}{\ding{46}}] {\ $2^{25964951}-1$,\hspace{2cm}  $7816230$ digits (discovered in  2005)}
\item[\textcolor{red}{\ding{46}}] {\ $2^{24036583}-1$,\hspace{2cm}  $6320430$ digits (discovered in  2004)}
 \item[\textcolor{red}{\ding{46}}] {\ $2^{20996011}-1$,\hspace{2cm}  $6320430$ digits (discovered in  2003)}
 \item[\textcolor{red}{\ding{46}}] {\ $2^{13466917}-1$,\hspace{2cm}    $4053946$ digits  (discovered in 2001)}
 \item[\textcolor{red}{\ding{46}}] {\ $2^{6972593}-1$,\hspace{2cm} $2098960$ digits (discovered in  1999)}
 \item[\textcolor{red}{\ding{46}}] {\ $5359\times2^{5054502}+1$,\hspace{2cm} $1521561$ digits (discovered in  2003)}
% \item[\textcolor{red}{\ding{46}}] {\ $2^{3021377}-1$,\hspace{2cm} $909526$ digits (discovered in  1998)}
% \item[\textcolor{red}{\ding{46}}] \step{\ $2^{2976221}-1$,\hspace{2cm} $895932$ digits (discovered in  1997)}
% \item[\textcolor{red}{\ding{46}}] \step{\ $1372930^{131072}+1$,\hspace{2cm} $804474$ digits (discovered in  2003)}
% \item[\textcolor{red}{\ding{46}}] \step{\ $1176694^{131072}+1$,\hspace{2cm} $795695$ digits (discovered in  2003)}
%572186131072+1 754652 g0 2004 Generalized Fermat
%3.22478785+1 746190 g245 2003 Divides Fermat F(2478782), GF(2478782,3), GF(2478776,6), GF(2478782,12)
\end{itemize}

\end{slide}


\begin{slide}\pageTransitionWipe{30}
\heading{Great Internet Mersenne Prime Search (GIMPS)}\pause

\includegraphics[width=3cm]{images/GIMPS_logo.png}
The \emph{Great Internet Mersenne Prime Search (GIMPS)} 
is a collaborative project of volunteers who use freely available software to search for Mersenne prime numbers
(i.e. prime numbers of the form $2^p-1$ ($p$ prime)). \pause

The project was founded by George Woltman in January 1996.


\end{slide}

\begin{slide}\pageTransitionWipe{30}
\heading{The AKS deterministic primality test}\pause

\centerline{Department of Computer Science \& Engineering,}
\centerline{I.I.T. Kanpur, Agost 8, 2002.}\pause

\centerline{\includegraphics[width=4cm]{images/primality-group.jpg}}

\centerline{Nitin Saxena, Neeraj Kayal and Manindra Agarwal}\pause

\centerline{\textcolor{blue}{\emph{New deterministic,
polynomial--time, primality test.}}}\pause

Solves $\# 1$ open question in computational number theory\pause


\heading{http://www.cse.iitk.ac.in/news/primality.html}
\end{slide}

\begin{slide}\pageTransitionWipe{30}
\heading{How does the AKS work?}\pause

\noindent\textcolor{blue}{\textbf{Theorem. (AKS)}} \emph{Let
$n\in{\mathbb N}$. Assume $q, r$ primes, $S\subseteq {\mathbb N}$
finite:
\begin{itemize}
    \item $q|r-1$;
    \item $n^{(r-1)/q}\bmod r\not\in\{0,1\}$;
    \item $\gcd(n,b-b')=1,\ \ \forall b,b'\in S$ (distinct);
    \item $\binom{q+\#S-1}{\#S}\geq n^{2\lfloor\sqrt{r}\rfloor}$;
    \item $(x+b)^n=x^n+b$ in ${\mathbb Z}/n{\mathbb Z}[x]/(x^r-1),\ \ \forall b\in
    S$;
\end{itemize}
Then $n$ is a power of a prime} \ \hfill \textcolor{black}{{\tiny
Bernstein formulation}}\pause

 \textcolor{red}{Fouvry Theorem (1985)} $=\!\!\!=\!\!\!>\ \ \ \exists
r\approx\log^6n, s\approx\log^4n$\pause

\hspace*{3.65cm} $=\!\!\!=\!\!\!>\ \ \ $ \textcolor{blue}{AKS runs
in $O(\log^{15}n)$\\  \hspace*{4.7cm}operations in ${\mathbb
Z}/n{\mathbb Z}$.}\pause

Many simplifications and improvements: \textcolor{red}{Bernstein,
Lenstra, Pomerance.....}
\end{slide}

\begin{slide}\pageTransitionWipe{30}
\heading{Why is RSA safe?}\pause

\parstepwise{\begin{itemize}

\item[\textcolor{black}{\ding{43}}] \step{It is clear that if
Charles can factor $M$,} \step{then he can also compute
$\varphi(M)$ and then also $d$ so to decrypt messages}

\item[\textcolor{black}{\ding{43}}] \step{Computing $\varphi(M)$ is
equivalent to completely factor $M$. In fact}
\item[\ ] \step{\ \ \ \ \ \ $\displaystyle{p,q=\frac{M-\varphi(M)+1\pm\sqrt{(M-\varphi(M)+1)^2-4M}}{2}}$}
\item[\textcolor{black}{\ding{43}}] \step{\textbf{\textcolor{blue}{RSA
Hypothesis.}} The only way to compute efficiently}
\item[\ ] \step{\ \ \ \ \ \ $\displaystyle{x^{1/e}\bmod M,\ \ \ \forall x\in{\mathbb Z}/M\mathbb Z}$}
\item[\ ] \step{\ \ \ \ \ \ (i.e. decrypt messages) is to factor $M$}
\item[\ ] \step{\ \ \ \ \ \ In other words}
\item[\ ] \step{\ \ \ \ \ \ \ \emph{\textcolor{black}{The two problems are
polynomially equivalent}}}
\end{itemize}}
\end{slide}

\begin{slide}\pageTransitionWipe{30}

\heading{Two kinds of Cryptography}\pause

\begin{itemize}
  \item[\textcolor{blue}{\ding{43}}] {\bf Private key (or symmetric)}
\begin{itemize}
  \item[\textcolor{red}{\ding{46}}] Lucifer
  \item[\textcolor{red}{\ding{46}}] DES
  \item[\textcolor{red}{\ding{46}}] AES
\end{itemize}\pause

  \item[\textcolor{blue}{\ding{43}}] {\bf Public key}
\begin{itemize}
  \item[\textcolor{red}{\ding{46}}] RSA
  \item[\textcolor{red}{\ding{46}}] Diffie--Hellmann
  \item[\textcolor{red}{\ding{46}}] Knapsack
  \item[\textcolor{red}{\ding{46}}] NTRU
\end{itemize}
\end{itemize}

\end{slide}


\begin{slide}\pageTransitionWipe{30}
\heading{Another quotation!!!}\bigskip\bigskip\bigskip

\textit{
Have you ever noticed that there's no attempt being made to find really
large numbers that aren't prime. I mean, wouldn't you like to see a 
news report that says ``Today the Department of Computer Sciences at the
University of Washington annouced that $2^{58,111,625,031}+8$ is even''.
This is the largest non-prime yet reported.}\bigskip

\textsc{- University of Washington (Bathroom Graffiti)
}
\end{slide}

\end{document}
