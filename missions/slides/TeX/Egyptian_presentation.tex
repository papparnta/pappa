\documentclass[presentation]{beamer} %handout
\usepackage{beamerthemeshadow}
\usepackage{amsfonts,amsthm}
\usepackage{graphicx}
%\usepackage{sansmathaccent}
%\pdfmapfile{+sansmathaccent.map}
\usepackage{lmodern}
%  \setbeamercolor{block body}{#2}
\setbeamercolor{block title}{bg=green,fg=black}
\setbeamercolor{frametitle}{bg=red}
\setbeamercolor{frametitle right}{bg=gray!60!white}
\vfuzz=6pt

\title[One Day Symposium on Algebra \& Number Theory]
{Egyptian fractions: from Rhind Mathematical Papyrus to Erd\H os and Tao}
\institute{COMSATS University Islamabad, Lahore Campus}
\author[Lahore, October 13, 2018]{Francesco Pappalardi}
\date{October 10, 2018}

\begin{document}
\frame{\titlepage}

\frame{\frametitle{The Rhind Mathematical Papyrus}
    \begin{center}
     \includegraphics[width=11cm]{images/Rhind_Mathematical_Papyrus.jpg}

     British Museum
     \end{center}}

 \frame{\frametitle{Fractions in Egypt}
    \begin{center}
     \includegraphics[width=11cm]{images/e2.jpg}
     \end{center}}
     
 \frame{\frametitle{Fractions in Egypt}
    \begin{center}
     \includegraphics[width=11cm]{images/e3.jpg}
     \end{center}}

 \frame{\frametitle{Fractions in Egypt}
    \begin{center}
     \includegraphics[width=11cm]{images/e4.jpg}
     \end{center}}
\frame{\frametitle{Fractions in Egypt}
\framesubtitle{powers of two}
    \begin{center}
     \includegraphics[width=5cm]{images/e5.jpg}

     \end{center}}
     
     \frame{\frametitle{Egyptian Fraction Expansion (EFE)}
\begin{block}{EFE}
     Given $a/b\in\mathbb Q^>$, an \emph{Egyptian Fraction Expansion} of $a/b$ with length $k$ 
is the expression
$$\frac ab=\frac1{x_1}+\frac1{x_2}+\cdots+\frac1{x_k}$$

where $x_1,\ldots,x_k\in\mathbb N$\bigskip
\end{block}\pause

\textbf{Every $a/b\in\mathbb Q^>$ has an EFE with distinct $x_1,\ldots,x_n$!!}
     }

          \frame{\frametitle{The Greedy Algorithm}

     \begin{block}{Fibonacci (1200's)}
     \begin{columns}
\begin{column}{0.4\textwidth}  %%<--- here
\hspace*{1cm}     \includegraphics[width=0.8\textwidth]{images/Leonardo_Fibonacci.jpg}
\end{column}
\begin{column}{0.7\textwidth}
     \begin{itemize}[<+->]
      \item Given $0<a/b<1$, the identity:
      $$\frac ab=\frac1{b_1}+\frac{a_1}{bb_1}$$
      \item[]  can be found with \begin{enumerate}[<+->]
       \item $b_1,a_1\in\mathbb N$
      \item $1\le a_1<a$ 
      \item  $b_1>1$,                   
                  \end{enumerate}
      \item Hence we can iterate the process to obtain EFE for $a/b$
      \item $\frac ab=\frac1{b_1}+\frac1{b_2}+\frac{a_2}{bb_1b_2}=$\\ $=\frac1{b_1}+\frac1{b_2}+\frac1{b_3}+\frac{a_3}{bb_1b_2b_3}=\cdots$
      \item it takes at most $a$ steps
     \end{itemize}
\end{column}
\end{columns}
     \end{block}
}

          \frame{\frametitle{The Greedy Algorithm}
          \framesubtitle{Euclidean Division to find $a_1$ and $b_1$}
     \begin{block}{Euclid ($\approx300$ BC )}
     \begin{columns}
\begin{column}{0.4\textwidth}  %%<--- here
\hspace*{1cm}     \includegraphics[width=0.8\textwidth]{images/Euclid_3.jpeg}
\end{column}
\begin{column}{0.7\textwidth}
     \begin{itemize}[<+->]
      \item Given $a, b\in\mathbb N$,  $\exists q, r\in \mathbb N$ s.t.
$$ b=aq+r,\qquad 0\le r< a$$
%       \item indeed $q =\lfloor\frac{b}{a}\rfloor $ and $r:=b\bmod a$
% \item       where $\lceil\alpha \rceil=\min\{n\in\mathbb Z: n\ge\alpha\}$
      \item a quick computation shows 
      $$\frac{a}{b}=\frac{1}{q+1}+\frac{a-r}{b(q+1)}$$
      \item Hence
      \begin{enumerate}
       \item $b_1=q+1> 1$;
       \item $0<a_1=a-r<a$ \\ since $\gcd(a,b)=1$
      \end{enumerate}
      \end{itemize}
\end{column}
\end{columns}
     \end{block}
          
          }


     \frame{\frametitle{The Greedy Algorithm}
     
\begin{block}{\textbf{Example: The Greedy Algorithm at work}}
 \begin{align*}
      \frac5{121}&=\frac{1}{25}+\frac{4}{3025}\\
    &=\frac{1}{25}+\frac1{757}+\frac3{2289925}\\
       &=\cdots\\
       &=\frac{1}{25}+\frac{1}{757}+\frac{1}{763309}+\frac{1}{873960180913}+\\ & \hspace*{4cm} +\frac{1}{1527612795642093418846225}
\end{align*}
\end{block}\pause

     However, $$\frac{5}{121}=\frac1{33}+\frac{1}{121}+\frac{1}{363}$$
}

     \frame{\frametitle{The Takenouchi Algorithm (1921)}
\begin{block}{how Takenouchi Algorithm works}
      \begin{enumerate}[<+->]
 \item based on the identity:
     $$\frac{1}{b}+\frac{1}{b}=\begin{cases}
                                \frac1{b/2} & \text{if } 2\mid b\\
                                \frac1{\frac{b+1}2}+\frac1{\frac{b(b+1)}2} & \text{otherwise}
                               \end{cases}$$
\item Write
$\frac ab=\overbrace{\frac1b+\cdots+\frac1b}^{a-\text{times}}$
\item Apply the above identity $[a/2]$ times
$$\frac ab=\overbrace{\frac1{\frac{b+1}2}+\cdots+\frac1{\frac{b+1}2}}^{a/2-\text{times}}+\overbrace{\frac1{\frac{b(b+1)}2}+\cdots+\frac1{\frac{b(b+1)}2}}^{a/2-\text{times}}$$
\item reiterate using the first identity
                               \end{enumerate}
                               
\end{block}
}

\frame{\frametitle{The Takenouchi Algorithm (1921)}
 \begin{block}{\textbf{Example:}}
\begin{align*}
\frac{5}{121}&=\frac{1}{121}+\frac{1}{121}+\frac{1}{121}+\frac{1}{121}+\frac{1}{121}\\
&=\frac{1}{121}+\frac{1}{61}+\frac{1}{61}+\frac{1}{61\times121}+\frac{1}{61\times121}\\
&=\frac{1}{121}+\frac{1}{31}+\frac1{1891}+\frac{1}{3691}+\frac{1}{27243271}
\\
\end{align*}
\end{block}\pause

     However it is still worse than, $$\frac{5}{121}=\frac1{33}+\frac{1}{121}+\frac{1}{363}$$

 }
 \frame{\frametitle{Minimizing length \& Denominators' sizes}

 \centerline{
 \includegraphics[width=0.4\textwidth]{images/Gerald_Tenenbaum.jpg}
 \qquad
\includegraphics[width=0.28\textwidth]{images/Hisashi_Yokota}}

\begin{theorem}[\textbf{Tenenbaum -- Yokota (1990)}]
 Given $a/b\in\mathbb Q\cap (0,1)$, $\exists$ EFE s.t.
 \begin{itemize}[<+->]
  \item it has length $O(\sqrt{\log b})$;
  \item each denominator is $O\left(b\log b(\log\log b)^4(\log\log\log b)^2\right)$
 \end{itemize}
 \end{theorem}
}

 \frame{\frametitle{thinking at ESE-expansion as a Waring problem with negative exponent...}

 \begin{theorem}[\textbf{Graham (1964)}]
\begin{columns}
\begin{column}{0.55\textwidth}
 Given $a/b\in\mathbb Q^>$,
 $$\frac ab=\frac{1}{y_1^2}+\cdots+\frac{1}{y_k^2}$$
 admits a solution in distinct integers $y_1,\ldots,y_k$
$$\Longleftrightarrow\quad a/b\in(0,\pi^2/6-1)\cup[1,\pi^2/6)$$
 \end{column}
\begin{column}{0.3\textwidth}  %%<--- here
%\hspace*{1cm}     
\includegraphics[width=0.8\textwidth]{images/Ronald_Graham.jpg}
\end{column}
\end{columns}

 \end{theorem}\pause
 

\begin{block}{\textbf{Note: Graham result is quite general ... for example}}\pause  \emph{$\frac ab=\frac{1}{y_1^2}+\cdots+\frac{1}{y_k^2}$ with $y_j^2\equiv4\bmod5$ distinct $\Leftrightarrow$   $5\nmid b$  and\\ $\qquad\qquad a/b\in(0,\alpha-\frac{13}{36})\cap[\frac1{9},\alpha-\frac14)\cap[\frac14,\alpha-\frac19)\cap[\alpha,\frac{13}{36})$\\ where $\alpha=  2(5 -\sqrt5)\pi^2/125$}
 \end{block}
} 

\frame{\frametitle{The Erd\H{o}s-Strau\ss\ Conjecture}

\begin{block}{\textbf{Erd\H{o}s-Strau\ss\ Conjecture (ESC) (1950):}}
\begin{columns}
 \begin{column}{0.4\textwidth}
$\forall n>2$,
$$\frac4n=\frac1x+\frac1y+\frac1z$$
admits a solution in positive distinct integers $x, y, z$  
 \end{column}
\begin{column}{0.5\textwidth}
\includegraphics[width=0.45\textwidth]{images/Paul_Erdos.jpg} 
\includegraphics[width=0.45\textwidth]{images/Hernst_Strauss.jpeg}
\end{column}
\end{columns}
\end{block}\pause

\begin{block}{\textbf{Note:}}\begin{itemize}[<+->]
                        \item enough to consider (for prime $p\ge3$),
$\frac4p=\frac1x+\frac1y+\frac1z$
\item many computations. Record (2012) 
(Bello--Hern\'andez, Benito and Fern\'andez): ESC holds for $n\le 2\times 10^{14}$
%\item \textbf{Schinzel Conjecture:} given $a\in\mathbb N$, $\exists N_a$ s.t. if $n>N_a$,
% $\frac an=\frac1x+\frac1y+\frac1z$ admits a solution in distinct integers $x, y, z$
\end{itemize}
 \end{block}

}


\frame{\frametitle{The  Schinzel\ Conjecture}

\begin{block}{\textbf{Schinzel Conjecture:}}
\begin{columns}
 \begin{column}{0.7\textwidth}
given $a\in\mathbb N$, $\exists N_a$ s.t. if $n>N_a$,
 $$\frac an=\frac1x+\frac1y+\frac1z$$ admits a solution in 
 distinct integers $x, y, z$ \end{column}
\begin{column}{0.25\textwidth}
\includegraphics[width=0.9\textwidth]{images/Andrzej_Schinzel.jpg} 
\end{column}
\end{columns}
\end{block}\pause

\begin{theorem}[\textbf{Vaughan (1970):}]
\begin{columns}
 \begin{column}{0.7\textwidth}
$$\#\left\{n\le T: \genfrac{}{}{0pt}{0}{\frac an=\frac1x+\frac1y+\frac1z}
 {\text{ has no solution}}\right\}\ll \frac T{e^{c\log^{2/3}T}}$$
\end{column}
\begin{column}{0.25\textwidth}
\includegraphics[width=0.9\textwidth]{images/Robert_Charles_Vaughan.jpg} 
\end{column}
\end{columns}
\end{theorem}\pause

\begin{block}{\textbf{Elsholtz -- Tao (2013):} new results about ESC ... later}
\end{block}

 
}

\frame{\frametitle{Fixing the denominator}

\begin{definition}[Enumerating functions for fixed denominator]
Fix $n\in\mathbb N$ and set
\begin{enumerate}[<+->]
 \item $\mathcal A_k(n)=\left\{a\in\mathbb N:
\frac an=\frac1{x_1}+\cdots+\frac1{x_k}, \exists x_1,\ldots,x_k\in\mathbb N\right\}
$
\item $\mathcal A_k^*(n)=\left\{a\in\mathcal A_k(n): \gcd(a,n)=1\right\}$
\item 
$A_k(n)=\#\mathcal A_k(n)$
\item $A_k^*(n)=\#\mathcal A_k^*(n)$
\end{enumerate}


\end{definition}

Note that:
$$A_k(n)=\sum_{d\mid n}A_k^*(d)$$


}

\frame{\frametitle{Fixing the denominator}

\begin{block}{Numerics:}
{\tiny
\begin{tabular}{|c|l|l||c|l|l||c|l|l||c|l|l|}
\hline
$n$ & $A_2(n)$ & $A_3(n)$  & $n$ & $A_2(n)$ & $A_3(n)$ & $n$ & $A_2(n)$ & $A_3(n)$ & $n$ & $A_2(n)$ & $A_3(n)$ \\
\hline
2&4&6&27&18&41&52&27&68&77&25&75\\
3&5&8&28&23&49&53&10&36&78&39&101\\
4&7&11&29&10&26&54&35&82&79&12&45\\
5&6&11&30&29&58&55&24&65&80&49&118\\
6&10&16&31&8&27&56&36&85&81&28&81\\
7&6&13&32&23&51&57&21&62&82&18&59\\
8&11&19&33&18&44&58&18&53&83&14&50\\
9&10&19&34&17&42&59&14&41&84&60&139\\
10&12&22&35&20&49&60&51&109&85&22&78\\
11&8&16&36&34&69&61&6&28&86&19&62\\
12&17&29&37&6&27&62&18&56&87&25&77\\
13&6&18&38&17&45&63&33&86&88&39&105\\
14&13&26&39&20&51&64&32&81&89&14&48\\
15&14&29&40&33&71&65&22&69&90&58&138\\
16&16&31&41&10&29&66&36&89&91&20&79\\
17&8&21&42&34&74&67&8&39&92&29&86\\
18&20&38&43&8&30&68&30&79&93&21&75\\
19&8&22&44&25&61&69&25&70&94&21&69\\
20&21&41&45&28&69&70&39&98&95&24&82\\
21&17&37&46&17&47&71&14&42&96&59&143\\
22&14&32&47&12&36&72&54&121&97&8&47\\
23&10&25&48&41&87&73&6&36&98&32&94\\
24&27&51&49&14&46&74&17&57&99&36&107\\
25&12&33&50&27&67&75&33&91&100&48&126\\
\hline
\end{tabular} }
\end{block}

}

\frame{\frametitle{Fixing the denominator - the binary case}

\begin{block}{\textbf{Croot, Dobbs, Friedlander, Hetzel, F\!\!P (2000)}:}
\begin{columns}
 \begin{column}{0.6\textwidth}
\begin{enumerate}[<+->]
 \item $\forall\varepsilon>0,$ $$A_2(n)\ll n^{\epsilon}$$
 \item $\displaystyle T\log^3 T\ll \sum_{n\le T}A_2(n)\ll T\log^3T$
\end{enumerate}
\end{column}
\begin{column}{0.35\textwidth}
\includegraphics[width=0.4\textwidth]{images/Ernest_S_Croot.jpg} 
\includegraphics[width=0.4\textwidth]{images/David_Dobbs.jpg} 

\includegraphics[width=0.4\textwidth]{images/Andrew_Hetzel.JPG} 
\includegraphics[width=0.4\textwidth]{images/John_Friedlander.jpg} 
\end{column}
\end{columns}
\end{block}}

\frame{\frametitle{Fixing the denominator - the binary case}
\begin{lemma}[Rav Criterion (1966)] 
\begin{columns}
 \begin{column}{0.75\textwidth}
Let $a$, $n\in\mathbb N$ 
s.t. $(a,n)=1$. 
$$\frac an=  \frac{1}{x}+ \frac{1}{y}$$ has solution $x, y\in\mathbb N\quad\Leftrightarrow\quad\exists (u_1,u_2)\in{\bf N}^2$
with\\ \hfill $(u_1,u_2)=1$,\\ \hfill $u_1u_2|n$ and $a\left| {u_1+u_2} \right.$
\end{column}
\begin{column}{0.2\textwidth}
\includegraphics[width=0.8\textwidth]{images/Yehuda_Rav.jpg} 
\end{column}
\end{columns}
\end{lemma}\pause

\begin{block}{\textbf{Consequence:} let $\tau(n)$ be number of divisors of $n$ and $[m,n]$ be the lowest common multiple of $n$ and $m$}
$$A_2^*(p^k)=\tau([p^k+1,p^{k-1}+1,\ldots,p+1])$$
\end{block}
}

\frame{\frametitle{Fixing the denominator - the general case}

\begin{theorem}[\textbf{Croot, Dobbs, Friedlander, Hetzel, F\!\!P (2000)}]
\begin{columns}
 \begin{column}{0.6\textwidth}
\begin{itemize}[<+->]
 \item $\forall\varepsilon>0$,
$\qquad A_3(n)\ll_\epsilon n^{1/2+\epsilon}$
 \item by an induction argument, $\forall\varepsilon>0$,
 $$A_k(n)\ll_\epsilon n^{\alpha_k+\epsilon}$$
where $\alpha_k=1-2/(3^{k-2}+1)$
\end{itemize}
\end{column}
\begin{column}{0.35\textwidth}
\includegraphics[width=0.3\textwidth]{images/Ernest_S_Croot.jpg} 
\includegraphics[width=0.3\textwidth]{images/David_Dobbs.jpg} 

\includegraphics[width=0.3\textwidth]{images/Andrew_Hetzel.JPG} 
\includegraphics[width=0.3\textwidth]{images/John_Friedlander.jpg} 
\end{column}
\end{columns}
\end{theorem}\pause

\begin{theorem}[\textbf{Banderier, Luca, F\!\!P (2018)}]
\begin{columns}
 \begin{column}{0.6\textwidth}
  \begin{itemize}[<+->]
 \item $\forall\varepsilon>0$,
 $\qquad A_3(n)\ll_\epsilon n^{1/3+\epsilon}$
 \item by an induction argument, $\forall\varepsilon>0$,
 $$A_k(n)\ll_\epsilon n^{\beta_k+\epsilon}$$
where $\beta_k=1-2/(2\cdot3^{k-3}+1)$
\end{itemize}
\end{column}
\begin{column}{0.35\textwidth}
\includegraphics[width=0.45\textwidth]{images/Cyril_Banderier.jpeg} 
\includegraphics[width=0.45\textwidth]{images/Florian_Luca.jpeg} 
\end{column}
\end{columns}
\end{theorem}

}

\frame{\frametitle{Fixing the denominator - the general case}
\framesubtitle{generalizing Rav criterion}
\begin{lemma}
Let $a/n\in\mathbb Q^>$. 
$a/n=1/x+1/y+1/z$ for some $x,y,z\in\mathbb N$ $\quad \Leftrightarrow \exists$ six positive integers $D_1,D_2,D_3,v_1,v_2,v_3$ with 
\begin{itemize}[<+->]
\item[(i)] $[D_1,D_2,D_3]\mid n$;
\item[(ii)]   $v_1v_2v_3\mid D_1v_1+D_2v_2+D_3v_3$; 
\item[(iii)] $a\mid (D_1v_1+D_2v_2+D_3v_3)/(v_1v_2v_3)$
\end{itemize}\medskip \pause
Conversely, if there are such integers, then  by putting $E=[{D_1,D_2,D_3}]$, 
$f_1:=n/E$, $f_2=(D_1v_1+D_2v_2+D_3v_3)/(av_1v_2v_3)$ and $f=f_1f_2$, a representation is 
$$
\frac{a}{n}=\frac{1}{(E/D_1)v_2v_3f}+\frac{1}{(E/D_2)v_1v_3f}+\frac{1}{(E/D_3)v_1v_2f}
$$
\end{lemma}

}

\frame{\frametitle{back to Erd\H{o}s-Strau\ss\ Conjecture}
\framesubtitle{the polynomial families of solution}

\begin{block}{Polynomial families of solutions}
\begin{itemize}[<+->]
 \item $\displaystyle\frac {4}{n}=
\frac {1}{n}+\frac {1}{(n+1)/3}+\frac {1}{n(n+1)/3}$
\item[] $\qquad\qquad\Longrightarrow$ if $n\equiv2\bmod 3$, ESC holds for $n$
\item $\displaystyle\frac {4}{n}=\frac{1}{n/3}+\frac{1}{4n/3}+\frac{1}{4n}$
\item[] $\qquad\qquad\Longrightarrow$ if $n\equiv0\bmod 3$, ESC holds for $n$
\item Need to solve ESC for $n\equiv 1\bmod3$
\item \textbf{idea can be pushed:} $4/n$ requires four terms with the \emph{greedy algorithm} if and only if $n\equiv 1$ or $17 (\bmod 24)$
\item \textbf{example} if $n=5+24t$
$$\frac{4}{n}=
\frac{1}{6t+1}+\frac{1}{(2+8t)(6t+1)}+\frac{1}{(5+24t)(6t+1)(2+8t)}$$
\end{itemize}
\end{block}
}

\frame{
\frametitle{back to Erd\H{o}s-Strau\ss\ Conjecture $4/n=1/x+1/y+1/z$}
\begin{block}{(Another) example ($n\equiv 7\bmod 24$)}
$$\frac{4}{7+24t}=
\frac{1}{6t+2}+\frac{1}{(8+24t)(6t+2)}+\frac{1}{(7+24t)(8+24t)(6t+2)}$$
\end{block}\pause

\begin{definition}[solvable congruences] We say that $r(\bmod q)\in\mathbb Z/q\mathbb Z^*$ is \emph{solvable by polynomials}
if $\exists P_1, P_2, P_3\in\mathbb Q[x]$ which take 
positive integer values for sufficiently large integer argument and such that for all $n\equiv r(\bmod q)$:
$$\frac4n=\frac1{P_1(n)}+\frac1{P_2(n)}+\frac1{P_3(n)}$$ 
\end{definition}\pause

}

\frame{
\frametitle{back to Erd\H{o}s-Strau\ss\ Conjecture $4/n=1/x+1/y+1/z$}

\begin{theorem} [\textbf{Elsholtz--Tao (2013)}] 
\begin{columns}
 \begin{column}{0.4\textwidth}
There is a classification of solvable conguences by polynomials 
\end{column}
\begin{column}{0.55\textwidth}
\includegraphics[width=0.4\textwidth]{images/Christian_Elsholtz.jpg}
\includegraphics[width=0.4\textwidth]{images/Terrence_Tao.jpeg} 
\end{column}
\end{columns}
\end{theorem}\pause

\begin{theorem}[\textbf{Mordell (1969)}]
\begin{columns}
 \begin{column}{0.7\textwidth}
 All (primitive) congruence classes $r(\bmod 840)$ are solvable by polynomials unless $r$ is a perfect square\medskip
 
 (i.e. $r=1^2,11^2,13^2,17^2,19^2,23^2$)
\end{column}
\begin{column}{0.25\textwidth}
\includegraphics[width=0.9\textwidth]{images/Louis_Mordell.jpeg} 
\end{column}
\end{columns}
\end{theorem}
}

\frame{\frametitle{back to Erd\H{o}s-Strau\ss\ Conjecture $4/n=1/x+1/y+1/z$}
\begin{block}{\textbf{Remarks \& Definitions:}}
\begin{itemize}[<+->]
  \item Up to reordering, solutions of $\frac4p=\frac1x+\frac1y+\frac1z$ are of two types:
 \begin{itemize}[<+->]
  \item[I.] $p\mid x$ \& $p\nmid yz$
  \item[II.] $p\mid \gcd(x,y)$ \& $p\nmid z$
 \end{itemize}
\item in analogy, we say that, up to reordering, a solutions of $\frac4n=\frac1x+\frac1y+\frac1z$ is of type:
 \begin{itemize}[<+->]
  \item[I.] if $n\mid x$ \& $\gcd(n, yz)=1$
  \item[II.] $n\mid \gcd(x,y)$ \& $\gcd(n,z)=1$
\end{itemize}
\item $f(n)$ be the number of solutions of $4/n=1/x+1/y+1/z$
\item Set $f_I(n)$ (resp $f_{II}(n)$) be the number of solutions of type I (resp II) of $4/n=1/x+1/y+1/z$
\item $f(p)=3f_I(p)+3f_{II}(p)$
\item $f(n)\geq 3f_I(n)+3f_{II}(n)$
\end{itemize}
\end{block}


}

\frame{\frametitle{back to Erd\H{o}s-Strau\ss\ Conjecture $4/n=1/x+1/y+1/z$}
\framesubtitle{Elsholtz -- Tao paper}

\begin{theorem}[some of Elsholtz -- Tao's results]
\begin{columns}
 \begin{column}{0.7\textwidth}
\begin{itemize}[<+->]
 \item $f_I(n)\ll n^{3/5+\varepsilon}$,\quad $f_{II}(n)\ll n^{2/5+\varepsilon}$
 \item $N\log^3 N\ll \displaystyle\sum_{n\le N}f_I(n)\ll N\log^3 N$
 \item $N\log^3 N\ll \displaystyle\sum_{n\le N}f_{II}(n)\ll N\log^3 N$
 \item $N\log^2 N\ll\!\! \displaystyle\sum_{p\le N}f_I(p)\ll\!\! \scriptsize{N\log^2 N\log\log N}$
 \item $N\log^2 N\ll \displaystyle\sum_{p\le N}f_{II}(p)\ll N\log^2 N$
 \item $f(n)\gg e^{\left((\log3+o(1))\frac{\log n}{\log\log n}\right)}$ for $\infty$ $n$
 \item $f(n)\gg(\log n)^{0.54}$ for almost all $n$
 \item $f(p)\gg(\log p)^{0.54}$ for almost all $p$
\end{itemize}
\end{column}
\begin{column}{0.25\textwidth}
\includegraphics[width=0.8\textwidth]{images/Christian_Elsholtz.jpg}
\bigskip

\includegraphics[width=0.8\textwidth]{images/Terrence_Tao.jpeg} 
\end{column}
\end{columns}
 \end{theorem}

}

\frame{\frametitle{back to Erd\H{o}s-Strau\ss\ Conjecture $4/n=1/x+1/y+1/z$}
\framesubtitle{A key idea on the Elsholtz -- Tao paper}
Let
$S_{m,n}=\{(x,y,z)\in\mathbb C^3: mxyz=nyz+nxy+nxz\}\subset\mathbb C^3.$

$A_3(n)$ equals the number of $m\in\mathbb N$ s.t. $S_{m,n}\cap\mathbb N^3\ne\emptyset$. Set

$$\Sigma^\mathrm{I}_{m,n}=\left\{(a,b,c,d,e,f)\in\mathbb C^6:\begin{array}{l}
                                mabd=ne+1, ce=a+b
                                                     \\ mabcd=n(a+b)+c\\
                                macde=ne+ma^2d+1\\
                                                     mbcde=ne+mb^2d+1\\
                                     macd=n+f,     ef=ma^2d+1\\
                                                     bf=na+c\\
                                                     n^2+mc^2d=f(mbcd-n)
                                                    \end{array}
 \right\}$$
 
 which is a 3-dimensional algebraic variety. The map
 $$\pi^\mathrm{I}_{m,n}: \Sigma^\mathrm{I}_{m,n}
 \longrightarrow S_{m,n}, (a,b,c,d,e)\mapsto (abdn,acd,bcd)$$
 is well defined after quotienting by the dilation symmetry
 $(a,b,c,d,e,f)\mapsto(\lambda a,\lambda b,\lambda c,\lambda^{-2} d,e,f) $ this map is bijective
}

\frame{\frametitle{back to $A_3(p)$}
\framesubtitle{Adapting Elsholtz -- Tao construction}

\begin{theorem}[\textbf{Banderier, Luca, F\!\!P (2018)}]
\begin{columns}
 \begin{column}{0.6\textwidth}
$$\sum_{p\le N}A_{II,3}(p)\ll N\log^2N\log\log N$$
where $A_{II,3}(p)$ is the number of $a\in\mathbb N$ s.t. 
$$\frac ap=\frac1{px}+\frac1{py}+\frac1z$$
admits a solution $x, y, z\in \mathbb N$
\end{column}
\begin{column}{0.35\textwidth}
\includegraphics[width=0.45\textwidth]{images/Cyril_Banderier.jpeg} 
\includegraphics[width=0.45\textwidth]{images/Florian_Luca.jpeg} 
\end{column}
\end{columns}
\end{theorem}
}

\frame{\frametitle{back to $A_3(p)$}
\framesubtitle{what goes into the proof...}

these are classical \emph{elementary analytic number theory}
proof:\bigskip\pause

\begin{columns}
 \begin{column}{0.65\textwidth}
\begin{itemize}[<+->]
 \item Dirichlet average divisor in special sparse sequences
 \item Prime in arithmetic progression
 \item Brun Titchmarsh estimates
 \item Bombieri--Vinogradov Theorem
\end{itemize}
 \end{column}
\begin{column}{0.33\textwidth}
\includegraphics[width=0.4\textwidth]{images/Johann_Peter_Gustav_Lejeune_Dirichlet.png} 
\quad
\includegraphics[width=0.4\textwidth]{images/Titchmarsh.jpeg} 
\bigskip

\includegraphics[width=0.4\textwidth]{images/Enrico_Bombieri.jpeg} 
\quad
\includegraphics[width=0.4\textwidth]{images/Ivan_Vinogradov.jpg} 
 \end{column}
\end{columns}
}


\end{document}
