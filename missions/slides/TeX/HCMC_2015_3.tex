\documentclass[10pt,handout]{beamer} %,hyperref={pdfpagelabels=false},draft,handout,handout
\usepackage[orientation=landscape,size=custom,width=16,height=9,scale=0.30,debug]{beamerposter} 
\usepackage[english]{babel}
\usepackage{lmodern}% http://ctan.org/pkg/lm
\usepackage[latin1]{inputenc}
\usepackage{times,hyperref,tikz,colortbl,yfonts,translator}
\usepackage[T1]{fontenc}
 \newcommand{\Q}{\mathbb Q}
 \newcommand{\Z}{\mathbb Z}
 \newcommand{\N}{\mathbb N}
 \newcommand{\F}{\mathbb F}
 \newcommand{\C}{\mathbb C}
 \newcommand{\R}{\mathbb R}
\useoutertheme[height=0pt,width=2cm,right]{sidebar}
\usecolortheme{rose,sidebartab}
\useinnertheme{circles}
\usefonttheme[only large]{structurebold}
\theoremstyle{definition}
\newtheorem{exercise}[theorem]{\translate{Exercise}}
\newtheorem{Note}[theorem]{\translate{Note}}
\lecture[4]{Elliptic curves over finite fields}{First Steps}
\title[Elliptic curves over $\F_{q}$]{\insertlecture}
\setbeamercolor{formul}{fg=black,bg=pink}
\setbeamercolor{sidebar right}{bg=green!15}
\setbeamercolor{structure}{fg=black!120}
\setbeamercolor{postit}{fg=black,bg=yellow}
\setbeamercolor{greys}{fg=black,bg==black!25}
\setbeamerfont{title in sidebar}{series=\bfseries}
\setbeamerfont*{item}{series=}
\setbeamerfont{frametitle}{size=}
\setbeamerfont{block title}{size=\small}
\setbeamerfont{subtitle}{size=\normalsize,series=\normalfont}
\begin{document}

\begin{frame}
\includegraphics[width=1.6cm]{images/roma3.pdf}\hfill\includegraphics[width=1.9cm]{images/HCMCUS.jpeg}
\vfill

\begin{center}\begin{sc}
\begin{Large}

\textcolor{red}{Elliptic curves over finite fields}
\end{Large}\bigskip

\ {Francesco Pappalardi}\bigskip\bigskip

\begin{large}\begin{bf}\#3 - First Steps.
\end{bf}\end{large}\medskip

September $4^{\text{th}}$ 2015\medskip
\vfill
\end{sc}\end{center}

\includegraphics[width=1.6cm]{images/cimpalogo.pdf}\hfill
\begin{minipage}[b]{9.3cm}
\textbf{SEAMS School 2015}\\
\textit{Number Theory and Applications in Cryptography and Coding Theory}\\
University of Science, Ho Chi Minh, Vietnam\\
August 31 - September 08, 2015
\end{minipage}\hfill
\includegraphics[width=1.9cm]{images/seams.png}
\end{frame}

\section{Introduction}
\subsection{History}

\begin{frame}\frametitle{Proto--History (from \textsc{Wikipedia})}

\begin{columns}[c]
\begin{column}{6cm}%\begin{small}
Giulio Carlo, Count Fagnano, and Marquis de Toschi (December 6, 1682 -- September 26, 1766)
was an Italian mathematician. He was probably the first to direct attention to the theory of
\emph{elliptic integrals}. Fagnano was born in Senigallia.\medskip

He made his higher studies at the \emph{Collegio Clementino} in Rome and there won great distinction,
except in mathematics, to which his aversion was extreme. Only after his college course he took up the study of
mathematics.\medskip

Later, without help from any teacher, he mastered mathematics from its foundations.%\end{small}
\begin{block}{Some of His Achievements:}
\begin{itemize}
 \item $\pi=2i\log\frac{1-i}{1+1}$
 \item Length of \emph{Lemniscate}
\end{itemize}
\end{block}
\end{column}
\begin{column}{4cm}
\includegraphics[width=2.5cm]{images/fagnano.jpg}\\
{Carlo Fagnano}
\smallskip

\includegraphics[width=2.5cm]{images/ColegioClementino.jpg}\\
{Collegio Clementino}
\smallskip

\includegraphics[width=2.5cm]{images/lemniscate.pdf}\\
{\scriptsize Lemniscate $(x^2+y^2)^2=2a^2(x^2-y^2)$}\\
{$\ell=4\int_0^a\frac{a^2dr}{\sqrt{a^4-r^4}}=\frac{a \sqrt\pi\Gamma(\frac54)}{\Gamma(\frac34)}$}
\end{column}
\end{columns}
\end{frame}

\subsection{length of ellipses}
\begin{frame}\frametitle{Length of Ellipses}
\begin{columns}[c]
\begin{column}{5.5cm}<1->\begin{small}
\centerline{
\begin{beamercolorbox}[shadow=true,center,rounded=true,wd=2cm]{formul}
$\mathcal{E}: \frac{x^2}4+\frac{y^2}{16}=1$
\end{beamercolorbox}}
\centerline{\includegraphics[width=2.3cm]{images/ellipse.pdf}}\end{small}\pause
\begin{scriptsize}
\begin{block}
{\scriptsize{The length of the arc of a plane curve $y=f(x)$, $f:[a,b]\rightarrow\R$
is:}}
$$\ell=\int_a^b\sqrt{1+(f'(t))^2}dt$$
\end{block}
\pause
\end{scriptsize}
\end{column}
\begin{column}{8cm}<2->
\begin{small}
Applying this formula to $\mathcal{E}$:
\centerline{\begin{beamercolorbox}[shadow=true,center,rounded=true,wd=5.5cm]{formul}
 \begin{align*}
\ell(\mathcal{E})&=4\int_0^4\sqrt{1+\left(\frac{d\sqrt{16(1-t^2/4)}}{dt}\right)^2}dt\\
&=4\int_0^1\sqrt{\frac{1+3x^2}{1-x^2}}dx\qquad x=t/2
                    \end{align*}
\end{beamercolorbox}}
\pause

If $y$ is the integrand, then we have the identity:

\centerline{\begin{beamercolorbox}[shadow=true,center,rounded=true,wd=3cm]{formul}
$y^2(1-x^2)=1+3x^2$
\end{beamercolorbox}}\pause

Apply the invertible change of variables:


 \centerline{\begin{beamercolorbox}[shadow=true,center,rounded=true,wd=2cm]{formul}
 $\begin{cases}
   x = 1-2/t\\ y=\frac u{t-1}
  \end{cases}
$\end{beamercolorbox}}

Arrive to

\centerline{\begin{beamercolorbox}[shadow=true,center,rounded=true,wd=3cm]{formul}
$u^2 = t^3 - 4 t^2 + 6 t -3$
\end{beamercolorbox}}
\end{small}
\end{column}
\end{columns}
\end{frame}

\subsection{why Elliptic curves?}
\begin{frame}\frametitle{What are Elliptic Curves?}
\framesubtitle{Reasons to study them}\pause

Elliptic Curves
\begin{enumerate}[<+-| alert@+>]
\item are curves and finite groups at the same time
\item are non singular projective curves of \emph{genus} 1
\item have important applications in Algorithmic Number Theory and Cryptography
\item are the topic of the \alert{Birch and Swinnerton-Dyer conjecture} (one of the seven Millennium Prize Problems)
\item have a group law that is a consequence of the fact that they intersect every line in
exactly three points (in the projective plane over $\C$ and counted with multiplicity)
\item represent a mathematical world in itself ... Each of them does!!
\end{enumerate}
\end{frame}

\subsection{Fields}

\begin{frame}
 \frametitle{Notations}

\begin{alertblock}{Fields of characteristics 0}
 \begin{enumerate}[<+-| alert@+>]
 \item $\Q$ is the field of rational numbers
\item $\R$ and $\C$ are the fields of real and complex numbers
\item $K\subset\C$, $\dim_\Q K<\infty$ is a \emph{number field}
\begin{itemize}
\item $\Q[\sqrt{d}]$, $d\in\Q$
\item $\Q[\alpha]$, $f(\alpha)=0$, $f\in\Q[X]$
irreducible
\end{itemize}
\end{enumerate}
\end{alertblock}

\begin{exampleblock}{Finite fields}
 \begin{enumerate}[<+-| alert@+>]
 \item $\F_p=\{0,1,\ldots,p-1\}$ is the prime field;
 \item $\F_q$ is a finite field with $q=p^n$ elements
 \item $\F_q=\F_p[\xi]$, $f(\xi)=0$, $f\in\F_p[X]$
irreducible, $\partial f=n$
\item $\F_4=\F_2[\xi]$, $\xi^2=1+\xi$
\item $\F_8=\F_2[\alpha]$, $\alpha^3=\alpha+1$ but also $\F_8=\F_2[\beta]$, $\beta^3=\beta^2+1$, ($\beta=\alpha^2+1$)
\item $\F_{101^{101}}=\F_{101}[\omega], \omega^{101}=\omega+1$
\end{enumerate}
\end{exampleblock}

\end{frame}

\begin{frame}[label=current]
\frametitle{Notations}
%
 \begin{block}{Algebraic Closure of $\F_q$}\pause
 \begin{itemize}[<+-| alert@+>] % [<ballot@+-| visible@1-,+(1)>]
  \item $\C\supset\Q$ satisfies that \emph{Fundamental Theorem of Algebra}! (i.e. $\forall f\in\Q[x], \partial f>1, \exists\alpha\in\C,
 f(\alpha)=0)$
  \item We need a field that plays the role, for $\F_q$, that $\C$ plays for $\Q$. It will be $\overline{\F}_q$, called
\emph{algebraic closure of $\F_q$}
 \item[] \ \hfil 
 \begin{beamercolorbox}[rounded=true,shadow=true,wd=6.5cm]{postit}
         \begin{enumerate}
          \item $\forall n\in\N$, we fix an $\F_{q^n}$
          \item We also require that $\F_{q^n}\subseteq\F_{q^m}$ if $n\mid m$
          \item We let $\overline{\F}_q=\displaystyle\bigcup_{n\in\N}\F_{q^n}$
         \end{enumerate}
\end{beamercolorbox}
 \item
  \textbf{Fact:} $\overline{\F}_q$ is \emph{algebraically closed}\\ (i.e. $\forall f\in\F_q[x], \partial f>1, \exists\alpha\in\overline{\F}_q,
 f(\alpha)=0)$
 \end{itemize}
 \end{block}

%\begin{scriptsize}
If $F(x,y)\in\Q[x,y]$ a \emph{point of the curve $F=0$}, means $(x_0,y_0)\in\C^2$ s.t.
$F(x_0,y_0)=0$. \pause

If $F(x,y)\in\F_q[x,y]$ a \emph{point of the curve $F=0$}, means $(x_0,y_0)\in\overline{\F}_q^2$ s.t.
$F(x_0,y_0)=0$.
%\end{scriptsize}
\end{frame}


\section{Weierstra\ss\ Equations}

\begin{frame}{The (general) Weierstra\ss\ Equation}

An elliptic curve $E$ over a $\F_q$ (finite field) is given by an equation
\centerline{\begin{beamercolorbox}[shadow=true,center,rounded=true,wd=6cm]{formul}
$E: y^2+a_1xy+a_3y=x^3+a_2x^2+a_4x+a_6$\end{beamercolorbox}}
where $a_1, a_3, a_2, a_4 ,a_6\in\F_q$ \pause

\begin{center}
 \includegraphics[width=60mm]{images/elliptic1.pdf}\pause
\llap{\includegraphics[width=60mm]{images/elliptic2.pdf}}\pause
\llap{\includegraphics[width=60mm]{images/elliptic3.pdf}}\pause
\llap{\includegraphics[width=60mm]{images/elliptic3b.pdf}}\pause
\llap{\includegraphics[width=60mm]{images/elliptic4.pdf}}\pause
\llap{\includegraphics[width=60mm]{images/elliptic5.pdf}}\pause
\llap{\includegraphics[width=60mm]{images/elliptic6.pdf}}\pause
\llap{\includegraphics[width=60mm]{images/elliptic7.pdf}}\pause
\llap{\includegraphics[width=60mm]{images/elliptic8.pdf}}\pause
\llap{\includegraphics[width=60mm]{images/elliptic9.pdf}}\pause
\llap{\includegraphics[width=60mm]{images/elliptic9b.pdf}}\pause
\llap{\includegraphics[width=60mm]{images/elliptic10.pdf}}\pause
\llap{\includegraphics[width=60mm]{images/elliptic10b.pdf}}\pause
\llap{\includegraphics[width=60mm]{images/elliptic6.pdf}}\pause
\end{center}

 \begin{beamercolorbox}[sep=1em,wd=5.5cm]{postit}
 The equation should not be \emph{singular}
 \end{beamercolorbox}
\end{frame}

\subsection{Singular points}

\begin{frame}{Tangent line to a plane curve}

Given $f(x,y)\in\F_q[x,y]$ and a point $(x_0,y_0)$ such that $f(x_0,y_0)=0$, the \emph{tangent
line} is:
\centerline{\begin{beamercolorbox}[shadow=true,center,rounded=true,wd=6cm]{formul}
             $\frac{\partial f}{\partial x}(x_0,y_0)(x-x_0)+\frac{\partial f}{\partial y}(x_0,y_0)(y-y_0)=0$
            \end{beamercolorbox}}\pause

If

\centerline{\begin{beamercolorbox}[shadow=true,center,rounded=true,wd=4cm]{formul}
             $\frac{\partial f}{\partial x}(x_0,y_0)=\frac{\partial f}{\partial y}(x_0,y_0)=0,$
            \end{beamercolorbox}}\pause

such a tangent line cannot be computed and we say that $(x_0,y_0)$ is \emph{singular}\pause

\begin{Definition} A non singular curve is a curve without any singular point
\end{Definition}\pause


\begin{example}
The tangent line to $x^2+y^2=1$ over $\F_7$ at $(2,2)$ is $$x+y=4$$
\end{example}

\end{frame}

\begin{frame}
\frametitle{Singular points}
\framesubtitle{The classical definition}\pause

\begin{Definition}
A \emph{singular} point $(x_0,y_0)$ on a curve $f(x,y)=0$ is a point % $(x_0,y_0)$ on the curve (i.e. $f(x_0,y_0)=0$)
such that
$$\begin{cases}
\frac{\partial f}{\partial x}(x_0,y_0)=0\\
 \frac{\partial f}{\partial y}(x_0,y_0)=0
  \end{cases}$$\end{Definition}\pause

So, at a singular point there is no (unique) tangent line!!
In the special case of Weierstra\ss\ equations:
\centerline{\begin{beamercolorbox}[shadow=true,center,rounded=true,wd=6cm]{formul}
$E: y^2+a_1xy+a_3y=x^3+a_2x^2+a_4x+a_6$\end{beamercolorbox}}
we have\pause

\centerline{\begin{beamercolorbox}[shadow=true,center,rounded=true,wd=6cm]{postit}
$\begin{cases}
\partial_x=0\\
  \partial_y=0
  \end{cases} \longrightarrow \begin{cases}
a_1y=3x^2+2a_2x+a_4\\
  2y+a_1x+a_3=0
  \end{cases}$
            \end{beamercolorbox}}\pause

We can express this condition in terms of the coefficients $a_1, a_2, a_3, a_4, a_5$.
  \end{frame}

\subsection{The Discriminant}

\begin{frame}
\frametitle{The Discriminant of an Equation}
\framesubtitle{The condition of absence of singular points in terms of $a_1, a_2, a_3, a_4, a_6$}
\pause

With a bit of \texttt{Mathematica}
%\begin{scriptsize}
\centerline{\begin{beamercolorbox}[shadow=true,left,rounded=true,wd=11cm]{postit}
\texttt{Ell:=-a\_6-a\_4x-a\_2x\^{ }2-x\^{ }3+a\_3y+a\_1xy+y\^{ }2;\\
SS := Solve[\{D[Ell,x]==0,D[Ell,y]==0\},\{y,x\}];\\
Simplify[ReplaceAll[{Ell},SS[[1]]]*ReplaceAll[{Ell},SS[[2]]]]}\end{beamercolorbox}}
%\end{scriptsize}
we obtain\pause

\centerline{\begin{beamercolorbox}[shadow=true,center,rounded=true,wd=8.5cm]{formul}
\begin{align*}
\Delta'_E&:=\frac{1}{2^43^3}\left(-a_1^5 a_3 a_4 - 8 a_1^3 a_2 a_3 a_4 - 16 a_1 a_2^2 a_3 a_4 + 36 a_1^2 a_3^2 a_4 \right. \\
  &-a_1^4 a_4^2 - 8 a_1^2 a_2 a_4^2 - 16 a_2^2 a_4^2 + 96 a_1 a_3 a_4^2 +64 a_4^3 + \\
  & a_1^6 a_6 + 12 a_1^4 a_2 a_6 + 48 a_1^2 a_2^2 a_6 + 64 a_2^3 a_6 -36 a_1^3 a_3 a_6\\
  &\left. - 144 a_1 a_2 a_3 a_6 - 72 a_1^2 a_4 a_6 - 288 a_2 a_4 a_6 +
  432 a_6^2  \right)
 \end{align*}
\end{beamercolorbox}}\pause

\begin{Definition} The \emph{discriminant} of a Weierstra\ss\ equation over $\F_q$,  $q=p^n$, $p\ge5$ is
$$\Delta_E:=3^3\Delta'_E$$
\end{Definition}
\end{frame}

%\subsection{Discriminant \texorpdfstring{$/\F_{2^a}$}{F2a}}

\begin{frame}
\frametitle{The discriminant of $E/\F_{2^\alpha}$}
\centerline{
\begin{beamercolorbox}[shadow=true,center,rounded=true,wd=7.3cm]{formul}
$E: y^2+a_1xy+a_3y=x^3+a_2x^2+a_4x+a_6, a_i\in\F_{2^\alpha}$
\end{beamercolorbox}
}\pause

If $p=2$, the singularity condition becomes:
$$\begin{cases}
\partial_x=0\\
  \partial_y=0
  \end{cases}\longrightarrow\begin{cases}
a_1y=x^2+a_4\\
  a_1x+a_3=0
  \end{cases}
$$\pause
%\begin{small}
\begin{exampleblock}{Classification of Weierstra\ss\ equations over $\F_{2^\alpha}$}\pause
\begin{itemize}[<+-| alert@+>]
 \item Case $a_1\neq0$:
\hfill
{\begin{beamercolorbox}[shadow=true,left,rounded=true,wd=8.7cm]{postit}
{\small\texttt{El:=a6+a4x+a2x\^{}2+x\^{}3+a3y+a1xy+y\^{}2;\\
 Simplify[ReplaceAll[El,$\{$x$\to$a3/a1,y$\to$((a3/a1)\^{}2+a4)/a1$\}$]]}}
\end{beamercolorbox}}\pause

we obtain
\hfill \begin{beamercolorbox}[shadow=true,center,rounded=true,wd=8.5cm]{formul}
$\Delta_E:=
(a_1^6 a_6+a_1^5 a_3 a_4+a_1^4 a_2 a_3^2+a_1^4 a_4^2+a_1^3 a_3^3+a_3^4)/{a_1^6}$
\end{beamercolorbox}
 \item Case $a_1=0$ and $a_3\neq0$:
 curve non singular ($\Delta_E:=a_3$)
 \item Case $a_1=0$ and $a_3=0$: \alert{\emph{curve singular}}\hfill
$(x_0,y_0)$, ($x_0^2=a_4, y_0^2=a_2a_4+a_6$) singular point!
\end{itemize}
\end{exampleblock}
%\end{small}
\end{frame}

%\subsection{Discriminant \texorpdfstring{$/\F_{3^a}$}{F3a}}

% \begin{frame}
% \frametitle{The discriminant of $E/\F_{3^\alpha}$}
% \centerline{\begin{beamercolorbox}[shadow=true,center,rounded=true,wd=6cm]{formul}
% $E:y^2+a_1xy+a_3y=x^3+a_2x^2+a_4x+a_6\quad a_i\in\F_{3^\alpha}$\end{beamercolorbox}}\pause
% If $p=3$, the singularity condition becomes:\\
% \small{$
% \begin{cases}
% \partial_x=0\\
%   \partial_y=0
%   \end{cases}\pause\rightarrow\begin{cases}
% a_1y=-a_2x+a_4\\
%   -y+a_1x+a_3=0
%   \end{cases}\pause\rightarrow\begin{cases}
% (a_1^2+a_2)x=a_1a_3+a_4\\
%   y=a_1x+a_3
%   \end{cases}
% $}
% \end{frame}


\begin{frame}
\frametitle{Special Weierstra\ss\ equation of $E/\F_{p^\alpha}, p\neq2$}
\centerline{\begin{beamercolorbox}[shadow=true,center,rounded=true,wd=8cm]{formul}
$E: y^2+a_1xy+a_3y=x^3+a_2x^2+a_4x+a_6\quad a_i\in\F_{p^\alpha}$\end{beamercolorbox}}\pause

If we ``complete the squares`` by applying the transformation:\begin{scriptsize}
\centerline{\begin{beamercolorbox}[shadow=true,center,rounded=true,wd=2.7cm]{postit}
        $\begin{cases}
  x\leftarrow x \\ y\leftarrow y -\frac{a_1x+a_3}2
 \end{cases}$            \end{beamercolorbox}}\end{scriptsize}\pause

 the Weierstra\ss\ equation becomes:
\centerline{\begin{beamercolorbox}[shadow=true,center,rounded=true,wd=5.1cm]{formul}
$E': y^2=x^3+a'_2x^2+a'_4x+a'_6$
            \end{beamercolorbox}}
where $a'_2=a_2+\frac{a_1^2}4, a'_4= a_4+\frac{a_1a_3}2, a'_6= a_6+\frac{a_3^2}4$\pause

If $p\ge5$, we can also apply the transformation\\ \pause
\centerline{\begin{scriptsize}
\begin{beamercolorbox}[shadow=true,center,rounded=true,wd=2cm]{postit}
$\begin{cases}
  x\leftarrow x-\frac{a'_2}{3} \\ y\leftarrow y
 \end{cases}$\end{beamercolorbox}
\end{scriptsize}} obtaining the equations:\pause

\centerline{\begin{beamercolorbox}[shadow=true,center,rounded=true,wd=4.2cm]{formul}
$E'': y^2=x^3+a''_4x+a''_6$
            \end{beamercolorbox}}
\hfil\ \hspace*{-1.2cm} where $a''_4=a'_4-\frac{{a'_2}^2}3, a''_6= a'_6+\frac{2{a'_2}^3}{27}-\frac{a'_2a'_4}3$
\end{frame}

\begin{frame}
\frametitle{Special Weierstra\ss\ equation for $E/\F_{2^\alpha}$}
\framesubtitle{Case $a_1\neq0$}
\centerline{\begin{beamercolorbox}[shadow=true,left,rounded=true,wd=9cm]{formul}
$E: y^2+a_1xy+a_3y=x^3+a_2x^2+a_4x+a_6\qquad a_i\in\F_{2^\alpha}$\hfill\\
\ \hfill $\Delta_E:=\frac{a_1^6 a_6+a_1^5 a_3 a_4+a_1^4 a_2 a_3^2+a_1^4 a_4^2+a_1^3 a_3^3+a_3^4}{a_1^6}$
\end{beamercolorbox}}\pause

If we apply the affine transformation:\begin{scriptsize}
\centerline{\begin{beamercolorbox}[shadow=true,left,rounded=true,wd=3.9cm]{postit}
        $\begin{cases}
x\longleftarrow a_1^2x+a_3/a_1\\\
y\longleftarrow a_1^3y+(a_1^2a_4+a_3^2)/a_1^2
  \end{cases}$\end{beamercolorbox}}\end{scriptsize}\pause

we obtain

\centerline{\begin{beamercolorbox}[shadow=true,center,rounded=true,wd=7cm]{formul}
$E': y^2+xy=x^3+\left(\frac{a_2}{a_1^2}+\frac{a_3}{a_1^3}\right)x^2+\frac{\Delta_E}{a_1^6}$\hfill \\
\ \hfill Surprisingly $\Delta_{E'}=\Delta_E/a_1^6$
\end{beamercolorbox}}\pause\bigskip

With \texttt{Mathematica}

\begin{scriptsize}
\ \hfill{\begin{beamercolorbox}[shadow=true,left,rounded=true,wd=8cm]{postit}
      \texttt{El:=a6+a4x+a2x\^{ }2+x\^{ }3+a3y+a1xy+y\^{ }2;\\
Simplify[PolynomialMod[ReplaceAll[El, \\
\ \hfill \{x->a1\^{ }2 x+a3/a1, y->a1\^{ }3y+(a1\^{ }2a4+a3\^{ }2)/a1\^{ }3\}],2]]}
\end{beamercolorbox}}\end{scriptsize}
\end{frame}


\begin{frame}
\frametitle{Special Weierstra\ss\ equation for $E/\F_{2^\alpha}$}
\framesubtitle{Case $a_1=0$ and $\Delta_E:=a_3\neq0$}
\centerline{\begin{beamercolorbox}[shadow=true,center,rounded=true,wd=8cm]{formul}
$E: y^2+a_1xy+a_3y=x^3+a_2x^2+a_4x+a_6\qquad a_i\in\F_{2^\alpha}$\\
\end{beamercolorbox}}\pause

If we apply the affine transformation:\begin{scriptsize}
\centerline{\begin{beamercolorbox}[shadow=true,center,rounded=true,wd=2.7cm]{postit}
        $\begin{cases}
x\longleftarrow x+a_2\\
y\longleftarrow y
  \end{cases}$\end{beamercolorbox}}\end{scriptsize}\pause

we obtain

\centerline{\begin{beamercolorbox}[shadow=true,center,rounded=true,wd=7cm]{formul}
$E: y^2+a_3y=x^3+(a_4+a_2^2)x+(a_6+a_2a_4)$
\end{beamercolorbox}}\pause\medskip

With \texttt{Mathematica}

\begin{scriptsize}
\ \hfill{\begin{beamercolorbox}[shadow=true,left,rounded=true,wd=8.6cm]{postit}
      \texttt{El:=a6+a4x+a2x\^{ }2+x\^{ }3+a3y+y\^{ }2;
 Simplify[PolynomialMod[ReplaceAll[El,\{x->x+a2,y->y\}],2]]}
\end{beamercolorbox}}\end{scriptsize}\pause

\begin{small}
\begin{Definition}
 Two Weierstra\ss\ equations over $\F_q$ are said (affinely) equivalent if there exists a (affine) change of variables that takes one
into the other
\end{Definition}
\end{small}\pause

\begin{block}{Exercise}
\begin{columns}
\begin{column}[t]{5cm}
Prove that necessarily the change of variables has form
\end{column}
\begin{column}[t]{7cm}
 $\begin{cases}
x\longleftarrow u^2 x+r\\
y\longleftarrow u^3 y+ u^2s x + t
  \end{cases} r,s,t,u\in\F_q$
\end{column}
\end{columns}
\end{block}
\end{frame}

\begin{frame}
\frametitle{The Weierstra\ss\ equation}
\framesubtitle{Classification of simplified forms}

After applying a suitable affine transformation we can always assume that $E/\F_q (q=p^n)$
has a Weierstra\ss\ equation of the following form\pause

\begin{scriptsize}
 \begin{example}[Classification]
\centerline{\begin{tabular}{|l|c|l|}
\hline
 $E$ & $p$ & $\Delta_E$\\
\hline
&&\\
 $y^2=x^3+Ax+B$ & $\ge5$ & $4A^3+27B^2$\\
&&\\
$y^2+xy=x^3+a_2x^2+a_6$ & $2$ & $a_6^2$\\
&&\\
 $y^2+a_3y=x^3+a_4x+a_6$  & $2$ & $a_3^4$\\
&&\\
 $y^2=x^3+Ax^2+Bx+C$ & $3$ & $
                               4A^3C-A^2B^2-18ABC+4B^3+27C^2$
                              \\
&&\\\hline
\end{tabular}}
\end{example}
\end{scriptsize}\pause

\begin{definition}[Elliptic curve] An elliptic curve is the data of a non
singular Weierstra\ss\ equation (i.e. $\Delta_E\neq0$)
\end{definition}\pause

\centerline{\alert{\textbf{Note:} If $p\ge3, \Delta_E\neq0\Leftrightarrow x^3+Ax^2+Bx+C$ has {no} double root}}
\end{frame}

\subsection{Elliptic curves \texorpdfstring{$/\F_2$}{F2}}
\begin{frame}
\frametitle{Elliptic curves over $\F_2$}

All possible Weierstra\ss\ equations over $\F_2$ are:\pause

\begin{beamerboxesrounded}[upper=block title example,lower=block body alerted,shadow=true]{Weierstra\ss\ equations over $\F_2$}
\begin{enumerate}
 \item $y^2+xy=x^3+x^2+1$
 \item$y^2+xy=x^3+1$
 \item$y^2+y=x^3+x$
 \item$y^2+y=x^3+x+1$
 \item$y^2+y=x^3$
 \item$y^2+y=x^3+1$
 \end{enumerate}
\end{beamerboxesrounded}
\pause

However the change of variables
$\begin{cases} x\leftarrow x+1\\ y\leftarrow y+x\end{cases}$ takes the sixth curve
into the fifth. Hence we can remove the sixth from the list.
\pause\bigskip

\begin{beamerboxesrounded}[upper=postit,lower=block body,shadow=true]{Fact:}
There are $5$ affinely inequivalent elliptic curves over $\F_2$
\end{beamerboxesrounded}
\end{frame}

\subsection{Elliptic curves \texorpdfstring{$/\F_3$}{F3}}
\begin{frame}
\frametitle{Elliptic curves in characteristic $3$}

Via a suitable transformation ($x\rightarrow u^2x+r, y\rightarrow u^3y+u^2sx+t$) over $\F_3$,  $8$ inequivalent
elliptic curves over $\F_3$ are found:\pause

\begin{beamerboxesrounded}[upper=block title example,lower=block body alerted,shadow=true]{Weierstra\ss\ equations over $\F_3$}
\begin{enumerate}
 \item $y^2=x^3+x$
 \item$y^2=x^3 - x$
 \item$y^2=x^3 - x +1$
 \item$y^2=x^3 - x -1$
 \item$y^2=x^3 + x^2 + 1$
 \item$y^2=x^3 + x^2 - 1$
 \item$y^2=x^3 - x^2 + 1$
 \item$y^2=x^3 - x^2 - 1$
 \end{enumerate}
\end{beamerboxesrounded}\pause

\begin{block}{Exercise: Prove that}
\begin{enumerate}[<+-| alert@+>]
\item Over  $\F_5$ there are 12 elliptic curves
          \item Compute all of them
           \item How many are there over $\F_4$, over $\F_7$ and over $\F_8$?
         \end{enumerate}
\end{block}

% 1 x^3 + 2*x         2-torsion 2  pts= 2
% 2 x^3 -   x + 2     2-torsion 1  pts= 3
% 3 x^3 + x + 2       2-torsion 2  pts= 4
% 4 x^3 + x           2-torsion 3  pts= 4
% 5 x^3 - 2*x + 2     2-torsion 1  pts= 5
% 6 x^3 + 1           2-torsion 2  pts= 6
% 7 x^3 + 2           2-torsion 2  pts= 6
% 8 x^3 +2*x + 1      2-torsion 1  pts= 7
% 9 x^3 -  x          2-torsion 3  pts= 8
% 10 x^3 - x + 1      2-torsion 2  pts= 8
% 11 x^3 + x + 1      2-torsion 1  pts= 9
% 12 x^3 - 2*x        2-torsion 2  pts= 10

% p=5;S=0;for(a=0,p-1,for(b=0,p-1,if((4*a^3+27*b^2)%p>0,print1(S++" "x^3+a*x+b" 2-torsion "matsize(factormod(x^3+a*x+b,p))[1]);T=1;for(x=0,p-1,for(y=0,p-1,if((y^2-x^3-a*x-b)%p==0,T++)));print("  pts= "T))))

\end{frame}

\section{The sum of points}
\begin{frame}
\frametitle{The definition of $E(\F_q)$}
\centerline{\begin{beamercolorbox}[shadow=true,left,rounded=true,wd=9cm]{formul}
Let $E/\F_q$ elliptic curve, $\infty:=[0,1,0]$. Set\\
\ \\
\scriptsize{$E(\F_q)=\{[X,Y,Z]\in\mathbb P_2(\F_q):\ Y^2Z+a_1XYZ+a_3YZ^2=X^3+a_2X^2Z+a_4XZ^2+a_6Z^3\}$}\\
\ \\
or equivalently\\
\ \\
\qquad $E(\F_q)=\{(x,y)\in \F_q^2:\ y^2+a_1xy+a_3y=x^3+a_2x^2+a_4x+a_6\}\cup\{\infty\}$
\end{beamercolorbox}}\pause

\ \hfill \begin{beamercolorbox}[shadow=true,left,rounded=true,wd=9cm]{postit}
 We can think either\pause
\begin{itemize}
 \item<1-> $E(\F_q)\subset\mathbb P_2(\F_q)$   \pause       \hfil$\dashrightarrow$ geometric advantages
 \item<2-> $E(\F_q)\subset\F_q^2\cup\{\infty\}$\pause \hfil$\dashrightarrow$ algebraic advantages
\end{itemize}\pause
\ \hfill$\infty$ might be though as the ``vertical direction''
\end{beamercolorbox}\pause

\begin{Definition}[line through points $P,Q\in E(\F_q)$]
$r_{P,Q}:\begin{cases}
                     \text{line through $P$ and }Q &\text{if }P\neq Q\\
                     \text{tangent line to $E$ at }P &\text{if }P=Q
                    \end{cases}$\hfill projective or affine
\end{Definition}\pause

\begin{itemize}[<+-| alert@+>]
\item if $\#(r_{P,Q}\cap E(\F_q))\ge2\ \Rightarrow\ \#(r_{P,Q}\cap E(\F_q))=3$\\
\hfill\scriptsize{\alert{if tangent line, contact point is counted with multiplicity}}  \item $r_{\infty,\infty}\cap E(\F_q)=\{\infty,\infty,\infty\}$%\vspace*{-4.4pt}
 % $\#(r_{P_1,P_1}\cap E(\F_q))=2$
 \item $r_{P,Q}: aX+bZ=0$ (vertical) $\Rightarrow \infty=[0,1,0]\in r_{P,Q}$
                    \vspace*{-4.4pt}
\end{itemize}

\end{frame}

\begin{frame}
\frametitle{History (from \textsc{Wikipedia})}

\begin{columns}[c]
\begin{column}{4.5cm}
\begin{small}
\textbf{Carl Gustav Jacob Jacobi} (10/12/1804 -- 18/02/1851) was a German mathematician,
who made fundamental contributions to elliptic functions, dynamics, differential equations,
and number theory.
\end{small}\\
\centerline{\includegraphics[width=1.8cm]{images/Jacobi.jpg}}
%\centerline{\scriptsize{Carl Gustav Jacob Jacobi}}\\
\begin{scriptsize}\begin{block}{Some of His Achievements:}
\begin{itemize}
 \item Theta and elliptic function
 \item Hamilton Jacobi Theory
 \item Inventor of determinants
 \item Jacobi Identity\\
 \tiny{ $[A,[B,C]]+[B,[C,A]]+[C,[A,B]]=0$}
\end{itemize}
\end{block}\end{scriptsize}
\end{column}\pause
\begin{column}{5.5cm}\vspace*{-16.3pt}
\begin{center}
\includegraphics[width=5.5cm]{images/add1.pdf}\pause
\llap{\includegraphics[width=5.5cm]{images/add2.pdf}}\pause
\llap{\includegraphics[width=5.5cm]{images/add3.pdf}}\pause
\llap{\includegraphics[width=5.5cm]{images/add5.pdf}}\pause
\llap{\includegraphics[width=5.5cm]{images/add6.pdf}}\pause
\llap{\includegraphics[width=5.5cm]{images/add7.pdf}}\pause
\llap{\includegraphics[width=5.5cm]{images/add1.pdf}}\pause
\llap{\includegraphics[width=5.5cm]{images/add8.pdf}}\pause
\llap{\includegraphics[width=5.5cm]{images/add9.pdf}}\pause
\llap{\includegraphics[width=5.5cm]{images/ad10.pdf}}\pause
\llap{\includegraphics[width=5.5cm]{images/ad11.pdf}}\pause
\llap{\includegraphics[width=5.5cm]{images/ad12.pdf}}\pause
\llap{\includegraphics[width=5.5cm]{images/add1.pdf}}\pause
\llap{\includegraphics[width=5.5cm]{images/ad13.pdf}}\pause
\llap{\includegraphics[width=5.5cm]{images/ad14.pdf}}\pause
\llap{\includegraphics[width=5.5cm]{images/ad15.pdf}}\pause
\llap{\includegraphics[width=5.5cm]{images/add7.pdf}}\pause
\end{center}
\small{
$r_{P,Q}\cap E(\F_q)=\{P,Q,R\}$\\
$r_{R,\infty}\cap E(\F_q)=\{\infty,R,R'\}$}
\centerline{\begin{beamercolorbox}[shadow=true,center,rounded=true,wd=2cm]{formul}
$P+_E Q:=R'$\pause
            \end{beamercolorbox}}%\smallskip

 \small{$r_{P,\infty}\cap E(\F_q)=\{P,\infty,P'\}$}\\
 \centerline{\begin{beamercolorbox}[shadow=true,center,rounded=true,wd=2cm]{formul}
             $-P:=P'$
            \end{beamercolorbox}}

\end{column}
\end{columns}
\end{frame}

\begin{frame}
\frametitle{Properties of the operation ``$+_E$''}

\begin{Theorem}
 The addition law on $E(\F_q)$ has the following
properties:
\begin{enumerate}[<+-| alert@+>][(a)]
 \item $P+_EQ\in E(\F_q)\hfill\forall P,Q\in E(\F_q)$
 \item  $P+_E\infty=\infty+_E P=P\hfill\forall P\in E(\F_q)$
 \item  $P+_E(-P)=\infty\hfill\forall P\in E(\F_q)$
 \item  $P+_E(Q +_E R)=(P+_E Q)+_E R\hfill\forall P,Q,R\in E(\F_q)$
 \item  $P+_E Q=Q +_E P\hfill\forall P,Q\in E(\F_q)$
\end{enumerate}
 \end{Theorem}\pause

\begin{itemize}[<+-| alert@+>]
%  \item By ``a point of $E/\F_q$ ($P\in E$)'' we mean $P\in E(\bar{\F}_q)$
%  in analogy for $E/\Q$ where ``a point of $E$'' means  $P\in E(\C)$
 \item $\left(E(\F_q),+_E\right)$  \alert{commutative group}
 \item All group properties are easy except \alert{associative law (d)}
 \item Geometric proof of associativity uses \emph{Pappo's Theorem}
 \item We shall comment on how to do it by explicit computation
 \item can substitute $\F_q$ with any field $K$; Theorem holds for $\left(E(K),+_E\right)$
\item In particular, if $E/\F_q$, can consider the groups $E(\overline{\F}_q)$ or $E(\F_{q^n})$
\end{itemize}
\end{frame}

\begin{frame}
\frametitle{Computing the inverse $-P$}
\centerline{\begin{beamercolorbox}[shadow=true,center,rounded=true,wd=7cm]{formul}
$E: y^2+a_1xy+a_3y=x^3+a_2x^2+a_4x+a_6$\end{beamercolorbox}}\pause

If $P=(x_1,y_1)\in E(\F_q)$

\ \hfill\begin{beamercolorbox}[shadow=true,left,rounded=true,wd=8cm]{formul}
           \textbf{\color[rgb]{1,0.3,1}Definition:}  $-P:=P'$ where $r_{P,\infty}\cap E(\F_q)=\{P,\infty,P'\}$\hfill\
            \end{beamercolorbox}\hfill\pause

            Write $P'=(x_1',y_1')$. Since $r_{P,\infty}: x=x_1\ \Rightarrow x_1'=x_1$ and $y_1$ satisfies
\centerline{\begin{beamercolorbox}[shadow=true,center,rounded=true,wd=8.8cm]{postit}
$y^2+a_1x_1y+a_3y-(x_1^3+a_2x_1^2+a_4x_1+a_6)=(y-y_1)(y-y_1')$
\end{beamercolorbox}}\bigskip\pause

So $y_1+y_1'=-a_1x_1-a_3$ (\alert{both coefficients of $y$}) and
\centerline{\begin{beamercolorbox}[shadow=true,center,rounded=true,wd=6cm]{postit}
$-P=-(x_1,y_1)=(x_1,-a_1x_1-a_3-y_1)$
 \end{beamercolorbox}}\bigskip\pause


So, if $P_1=(x_1,y_1), P_2=(x_2,y_2)\in E(\F_q)$,

\ \hfill\begin{beamercolorbox}[shadow=true,center,rounded=true,wd=9cm]{formul}
           \textbf{\color[rgb]{1,0.3,1}Definition:} $P_1+_EP_2=-P_3$ where $r_{P_1,P_2}\cap E(\F_q)=\{P_1,P_2,P_3\}\!$
            \end{beamercolorbox}\bigskip\pause

Finally, if $P_3=(x_3,y_3)$, then
\centerline{\begin{beamercolorbox}[shadow=true,center,rounded=true,wd=6cm]{postit}
$P_1+_EP_2=-P_3=(x_3,-a_1x_3-a_3-y_3)$
\end{beamercolorbox}}
\end{frame}

\begin{frame}
\frametitle{Lines through points of $E$}
\centerline{\begin{beamercolorbox}[shadow=true,center,rounded=true,wd=6cm]{formul}
$E: y^2+a_1xy+a_3y=x^3+a_2x^2+a_4x+a_6$\end{beamercolorbox}}
where $a_1, a_3, a_2, a_4 ,a_6\in\F_q,$ \pause

\begin{beamerboxesrounded}[upper=block title example,lower=block body alerted,shadow=true]
{$P_1=(x_1,y_1), P_2=(x_2,y_2)\in E(\F_q)$}
\begin{enumerate}[<+-| alert@+>]
 \item $P_1\neq P_2$ and $x_1\neq x_2\hfil \Longrightarrow\hfil r_{P_1,P_2}: y=\lambda x+\nu$
\begin{beamercolorbox}[shadow=true,center,rounded=true,wd=6cm]{postit}
$$\lambda= \frac{y_2-y_1}{x_2-x_1},\qquad \nu=\frac{y_1x_2-x_1y_2}{x_2-x_1}$$
\end{beamercolorbox}
 \item $P_1\neq P_2$ and $x_1=x_2\hfil \Longrightarrow\hfil r_{P_1,P_2}: x=x_1$
 \item  $P_1=P_2$ and $2y_1+a_1x_1+a_3\neq0\ \Longrightarrow\ r_{P_1,P_2}: y=\lambda x+\nu$
\hspace*{-0.5cm}\begin{beamercolorbox}[shadow=true,center,rounded=true,wd=9.3cm]{postit}
$$\lambda=\frac{3x_1^2+2a_2x_1+a_4-a_1y_1}{2y_1+a_1x_1+a_3},
 \nu=-\frac{a_3y_1+x_1^3-a_4x_1-2a_6}{2y_1+a_1x_1+a_3}$$
\end{beamercolorbox}
\item $P_1=P_2$ and $2y_1+a_1x_1+a_3=0\hfil \Longrightarrow\hfil r_{P_1,P_2}: x=x_1$
\item $r_{P_1,\infty}: x=x_1\hfill r_{\infty,\infty}: Z=0$
\end{enumerate}
\end{beamerboxesrounded}
\end{frame}

\begin{frame}
\frametitle{Intersection between a line and $E$}
We want to compute $P_3= (x_3,y_3)$ where $r_{P_1,P_2}: y=\lambda x +\nu$,
\centerline{\begin{beamercolorbox}[shadow=true,center,rounded=true,wd=5cm]{postit}
$r_{P_1,P_2}\cap E(\F_q)=\{P_1,P_2,P_3\}$
\end{beamercolorbox}}
\pause

We find the intersection:
\centerline{\begin{beamercolorbox}[shadow=true,center,rounded=true,wd=7cm]{formul}
$r_{P_1,P_2}\cap E(\F_q)=$ \scriptsize{\ $\begin{cases}
 E:y^2+a_1xy+a_3y=x^3+a_2x^2+a_4x+a_6\\ r_{P_1,P_2}: y=\lambda x +\nu
 \end{cases}$}
\end{beamercolorbox}}
 \pause
Substituting\\
\centerline{\begin{beamercolorbox}[center,wd=8cm]{postit}
$(\lambda x +\nu)^2+a_1x(\lambda x +\nu)+a_3(\lambda x +\nu)=x^3+a_2x^2+a_4x+a_6$
            \end{beamercolorbox}}\medskip\pause

            Since $x_1$ and $x_2$ are solutions, we can
find $x_3$ by comparing
\begin{scriptsize}

\centerline{\begin{beamercolorbox}[center,wd=9cm]{postit}
    \begin{align*}
&x^3+a_2x^2+a_4x+a_6-((\lambda x +\nu)^2+a_1x(\lambda x +\nu)+a_3(\lambda x +\nu))&=\\
\uncover<5->{&x^3+(\alert{a_2-\lambda^2-a_1\lambda})x^2+\cdots&=\\ }
\uncover<6->{&(x - x_1)(x - x_2)(x - x_3) = x^3 - ({\color[rgb]{0,0,1}x_1 + x_2 + x_3})x^2 + \cdots&\\ } %(x_1x_2 + x_1x_3 + x_2x_3)x - x_1x_2x_3
\notag
\end{align*}\vskip-1.5em
\end{beamercolorbox}}
\end{scriptsize}\pause

Equating coeffcients of $x^2$,
\centerline{\begin{beamercolorbox}[center,wd=8cm]{postit}
$x_3 = \lambda^2-a_1\lambda-a_2- x_1-x_2,\qquad y_3 = \lambda x_3 + \nu$
            \end{beamercolorbox}}
\pause
Finally\\
\centerline{\begin{beamercolorbox}[shadow=true,center,rounded=true,wd=9cm]{formul}
\small{$P_3 =({\color[cmyk]{0,1,1,0.5}\lambda^2-a_1\lambda-a_2-x_1-x_2},{\color[cmyk]{1,0,1,0.5}\lambda^3-a_1\lambda^2-\lambda(a_2+x_1+x_2)+\nu})$}
            \end{beamercolorbox}}
\end{frame}

\begin{frame}
\frametitle{Formulas for Addition on $E$ (Summary)}
\centerline{\begin{beamercolorbox}[shadow=true,center,rounded=true,wd=6cm]{formul}
$E: y^2+a_1xy+a_3y=x^3+a_2x^2+a_4x+a_6$\end{beamercolorbox}}\pause
$P_1 = (x_1, y_1), P_2 = (x_2, y_2)\in E(\F_q)\setminus\{\infty\}$,
\begin{beamerboxesrounded}[upper=block title example,lower=block body alerted,shadow=true]{Addition Laws for the sum of affine points}
\begin{itemize}[<+-| alert@+>]
 \item If $P_1\neq P_2$
\begin{itemize}
 \item $x_1 = x_2\ \hfill\Rightarrow\hfil$\ \ \
\begin{beamercolorbox}[shadow=true,center,rounded=true,wd=2cm]{formul}$P_1 +_E P_2 = \infty$
\end{beamercolorbox}
 \item $x_1 \neq x_2$\\
\centerline{\begin{beamercolorbox}[shadow=true,center,wd=4cm]{postit}
             $\lambda=\frac{y_2-y_1}{x_2-x_1}\qquad \nu=\frac{y_1x_2-y_2x_1}{x_2-x_1}$
            \end{beamercolorbox}}
 \end{itemize}
\item If $P_1 = P_2$
\begin{itemize}
 \item $2y_1+a_1x+a_3 = 0\ \hfill\Rightarrow\hfil$\ \ \
\begin{beamercolorbox}[shadow=true,center,rounded=true,wd=3cm]{formul}$P_1 +_E P_2 = 2P_1 = \infty$\end{beamercolorbox}
\item $2y_1+a_1x+a_3\neq 0$\\
\centerline{\begin{beamercolorbox}[shadow=true,center,wd=7cm]{postit}
$\lambda=\frac{3x_1^2+2a_2x_1+a_4-a_1y_1}{2y_1+a_1x+a_3}, \nu=-\frac{a_3y_1+x_1^3-a_4x_1-2a_6}{2y_1+a_1x_1+a_3}$
            \end{beamercolorbox}}
\end{itemize}
\end{itemize}\pause

Then\\
\centerline{\begin{beamercolorbox}[shadow=true,center,rounded=true,wd=11cm]{formul}
{\small $P_1 +_E P_2 = ({\color[cmyk]{0,1,1,0.5}\lambda^2-a_1\lambda-a_2-x_1-x_2},
{\color[cmyk]{1,0,1,0.5}-\lambda^3-a_1^2\lambda+(\lambda+a_1)(a_2+x_1+x_2)-a_3-\nu})$}
            \end{beamercolorbox}}
\end{beamerboxesrounded}
\end{frame}

\begin{frame}
\frametitle{Formulas for Addition on $E$ (Summary for special equation)}
\centerline{\begin{beamercolorbox}[shadow=true,center,rounded=true,wd=6cm]{formul}
$E: y^2=x^3+Ax+B$\end{beamercolorbox}}
$P_1 = (x_1, y_1), P_2 = (x_2, y_2)\in E(\F_q)\setminus\{\infty\}$,
\begin{beamerboxesrounded}[upper=block title example,lower=block body alerted,shadow=true]{Addition Laws for  the sum of affine points}
\begin{itemize}
 \item If $P_1\neq P_2$
\begin{itemize}
 \item $x_1 = x_2\ \hfill\Rightarrow\hfil$\ \ \
\begin{beamercolorbox}[shadow=true,center,rounded=true,wd=2cm]{formul}$P_1 +_E P_2 = \infty$
\end{beamercolorbox}
 \item $x_1 \neq x_2$\\
\centerline{\begin{beamercolorbox}[shadow=true,center,wd=4cm]{postit}
             $\lambda=\frac{y_2-y_1}{x_2-x_1}\qquad \nu=\frac{y_1x_2-y_2x_1}{x_2-x_1}$
            \end{beamercolorbox}}
 \end{itemize}
\item If $P_1 = P_2$
\begin{itemize}
 \item $y_1 = 0\ \hfill\Rightarrow\hfil$\ \ \
\begin{beamercolorbox}[shadow=true,center,rounded=true,wd=3cm]{formul}$P_1 +_E P_2 = 2P_1 = \infty$\end{beamercolorbox}
\item $y_1\neq 0$\\
\centerline{\begin{beamercolorbox}[shadow=true,center,wd=7cm]{postit}
$\lambda=\frac{3x_1^2+A}{2y_1}, \nu=-\frac{x_1^3-Ax_1-2B}{2y_1}$
            \end{beamercolorbox}}
\end{itemize}
\end{itemize}

Then\\
\centerline{\begin{beamercolorbox}[shadow=true,center,rounded=true,wd=11cm]{formul}
{\small $P_1 +_E P_2 = ({\color[cmyk]{0,1,1,0.5}\lambda^2-x_1-x_2},
{\color[cmyk]{1,0,1,0.5}-\lambda^3+\lambda(x_1+x_2)-\nu})$}
            \end{beamercolorbox}}
\end{beamerboxesrounded}

\end{frame}


\begin{frame}
 \frametitle{A Finite Field Example}

Over $\F_p$ geometric pictures don't make sense.\pause
 \begin{example}
Let
$E: y^2 = x^3 - 5x + 8 /\F_{37}$,\pause\hfill $P = (6, 3) , Q = (9, 10)\in E(\F_{37})$\pause

\centerline{\begin{beamercolorbox}[shadow=true,center,rounded=true,wd=6cm]{formul}
$r_{P,Q}: y=27x+26\quad r_{P,P}: y=11x+11 $
            \end{beamercolorbox}}\pause

\centerline{\begin{beamercolorbox}[shadow=true,center,rounded=true,wd=9cm]{postit}
$r_{P,Q}\cap E(\F_{37})=\begin{cases}
                          y^2 = x^3 - 5x + 8 \\ y = 27 x + 26
                         \end{cases}\!\!=\{(6,3), (9,10), (11,27)\}$
                                     \end{beamercolorbox}}\pause

\centerline{\begin{beamercolorbox}[shadow=true,center,rounded=true,wd=9cm]{postit}
$r_{P,P}\cap E(\F_{37})=\begin{cases}
                          y^2 = x^3 - 5x + 8 \\ y =11 x + 11
                         \end{cases}\!\!=\{(6,3), (6,3), (35,26)\}$\end{beamercolorbox}}\pause


\centerline{\begin{beamercolorbox}[shadow=true,center,rounded=true,wd=7.3cm]{formul}
$P+_EQ=(11,10)\qquad 2P=(35,11)$
            \end{beamercolorbox}}\pause

\ \hfill\scriptsize{$3P=(34,25), 4P=(8,6), 5P=(16,19),\ldots 3P+4Q=(31,28),\ldots$}
 \end{example}\pause

 \begin{beamerboxesrounded}[upper=block title example,lower=block body alerted,shadow=true]{Exercise}
 $\bullet$ Compute the order and the {\color[rgb]{0.1,0.3,1}{Group Structure}} of $E(\F_{37})$\\
 $\bullet$ Show that if $E_1/\F_q$ is equivalent to $E_2/\F_q$, then $E_1(\F_{q^n})\cong E_2(\F_{q^n})\forall n\in\N$.
  \end{beamerboxesrounded}
\end{frame}

\begin{frame}%[label=current]
 \frametitle{Group Structure}

\begin{theorem}[Classification of finite abelian groups]
 If $G$ is {\color[rgb]{0.9,0.3,0.2}{abelian and finite}},  $\exists n_1,\ldots,n_k\in\N^{>1}$ such that
 \begin{enumerate}[<+-| alert@+>]
\item $n_1\mid n_2\mid\cdots\mid n_k$
\item $G\cong C_{n_1}\oplus\cdots\oplus C_{n_k}$
\end{enumerate}
\ \hfill Furthermore $n_1,\ldots,n_k$ ({\color[rgb]{0.9,0.3,0.2} Group Structure}) are unique
 \end{theorem}\pause

\begin{example}[One can verify that:]
 $$C_{2400}\oplus C_{72} \oplus C_{1440}\cong C_{12}\oplus C_{60}\oplus C_{15200}$$
\end{example}\pause

Shall show that
\centerline{\begin{beamercolorbox}[shadow=true,center,rounded=true,wd=6cm]{formul}
$$E(\F_q)\cong C_n\oplus C_{nk}\qquad\exists n,k\in\N^{>0}$$
            \end{beamercolorbox}}\pause

            (i.e. $E(\F_q)$ is either cyclic ($n=1$) or the product of $2$ cyclic groups)
\end{frame}

\begin{frame}%[label=current2]
 \frametitle{Proof of the associativity}
 \centerline{\begin{beamercolorbox}[shadow=true,center,rounded=true,wd=7.3cm]{formul}
             $P+_E(Q+_ER)=(P+_EQ)+_ER\quad\forall P,Q,R\in E$
            \end{beamercolorbox}}\pause
 We should verify the above in many different cases according if $Q=R$, $P=Q$, $P=Q+_ER,\ldots$\pause

Here we deal with the \emph{generic case}. i.e. All the points
\alert{$\pm P, \pm R,\pm Q,\pm(Q+_ER),\pm(P+_EQ),\infty$} all different

\ \hfill {\begin{beamercolorbox}[shadow=true,left,rounded=true,wd=9.6cm]{postit}
 {\small{\color[rgb]{1,0.1,0.1}\texttt{Mathematica code}}\\
\texttt{L[x\_,y\_,r\_,s\_]:=(s-y)/(r-x);\\
M[x\_,y\_,r\_,s\_]:=(yr-sx)/(r-x);\\
A[\{x\_,y\_\},\{r\_,s\_\}]:=\{(L[x,y,r,s])$^2$-(x+r),\\
\ \hfill -(L[x,y,r,s])$^3$+L[x,y,r,s](x+r)-M[x,y,r,s]\}\\
Together[A[A[\{x,y\},\{u,v\}],\{h,k\}]-A[\{x,y\},A[\{u,v\},\{h,k\}]]]\\
det = Det[(\{\{1,x$_1$,x$_1^3$-y$_1^2$\},\{1,x$_2$,x$_2^3$-y$_2^2$\},\{1,x$_3$,x$_3^3$-y$_3^2$\}\})]\\
PolynomialQ[Together[Numerator[Factor[res[[1]]]]/det],\\
\ \hfill\{x$_1$,x$_2$,x$_3$,y$_1$,y$_2$,y$_3$\}]
PolynomialQ[Together[Numerator[Factor[res[[2]]]]/det],\\ \ \hfill\{x$_1$,x$_2$,x$_3$,y$_1$,y$_2$,y$_3$\}]}}
             \end{beamercolorbox}}\pause


% {\begin{beamercolorbox}[shadow=true,left,rounded=true,wd=9.6cm]{postit}
%  \scriptsize{{\color[rgb]{1,0.1,0.1}\texttt{Pari code}}\\
% \texttt{L(a,b,c,d)=(d-b)/(c-a);\\
% M(a,b,c,d)=(b*c-a*d)/(c-a);\\
% AX(a,b,c,d)=L(a,b,c,d)\^{}2-(a+c);\\
% AY(a,b,c,d)=-L(a,b,c,d)\^{}3+L(a,b,c,d)*(a+c)-M(a,b,c,d);\\
% simplify(AX(x1,y1,AX(x2,y2,x3,y3),AY(x2,y2,x3,y3))-\\ \ \hfill AX(AX(x1,y1,x2,y2),AY(x1,y1,x2,y2),x3,y3));\\
% simplify(AY(x1,y1,AX(x2,y2,x3,y3),AY(x2,y2,x3,y3))-\\ \ \hfill AY(AX(x1,y1,x2,y2),AY(x1,y1,x2,y2),x3,y3))}}
% \end{beamercolorbox}}\pause

\begin{small}
\begin{itemize}[<+-| alert@+>]
 \item runs in 2 seconds on a PC
 %\item Complete proof requires several cases (e.g. $(P+_EP)+_ER=P+_E(P+_ER)$)
 \item For an elementary proof:
 ``\text{An Elementary Proof of the Group Law for Elliptic Curves.}''
Department of Mathematics: Rice
University. Web. 20 Nov. 2009.\\ \ \hfill\texttt{http://math.rice.edu/\~{}friedl/papers/AAELLIPTIC.PDF}
\item More cases to check. e.g  \alert{$P+_E2Q=(P+_EQ)+_EQ$}
\end{itemize}
\end{small}
\end{frame}



\section{Examples}
\subsection{Structure of \texorpdfstring{$E(\F_2)$}{E(F2)}}
\begin{frame}
\frametitle{EXAMPLE: Elliptic curves over $\F_2$}

From our previous list:
\begin{block}{Groups of points}

\centerline{
\begin{tabular}{|l|c|l|}
\hline
 $E$ & $E(\F_2)$ & $|E(\F_2)|$\\
\hline
&&\\
 $y^2+xy=x^3+x^2+1$ & $\{\infty,(0,1)\}$& $2$\\
&&\\
$y^2+xy=x^3+1$ & $\{\infty,(0,1),(1,0),(1,1)\}$ & $4$\\
&&\\
$y^2+y=x^3+x$&$\{\infty,(0,0),(0,1), (1,0),(1,1)\}$&$5$\\
&&\\
 $y^2+y=x^3+x+1$ &$\{\infty\}$&$1$\\
&&\\
$y^2+y=x^3$ & $\{\infty,(0,0), (0,1)\}$ & $3$ \\
&&\\\hline
\end{tabular}}
\end{block}
\pause
So for each curve $E(\F_2)$ is cyclic except possibly for the second for which we need to distinguish between
$C_4$ and $C_2\oplus C_2$.\pause

\ \hfill \begin{beamercolorbox}[center,wd=9cm]{postit}
Note: each $C_i, i=1,\ldots,5$ is represented by a curve $/\F_2$
            \end{beamercolorbox}
\end{frame}


\subsection{Structure of \texorpdfstring{$E(\F_3)$}{E(F3)}}
\begin{frame}
\frametitle{EXAMPLE: Elliptic curves over $\F_3$}
From our previous list:

\begin{block}{Groups of points}\centerline{
\begin{tabular}{|l|r|c|c|}
\hline
$i$ & $E_i$ & $E_i(\F_3)$ &$E_i(\F_3)$\\
\hline
$1$& $y^2=x^3+x$ & {$\{\infty,(0,0),(2,1),(2,2)\}$}& $C_4$\\
\hline
$2$&$y^2=x^3 - x$ & {$\{\infty,(1,0),(2,0),(0,0)\}$} & $C_2\oplus C_2$\\
\hline
$3$&$y^2=x^3 - x +1$&{$\{\infty,(0,1),(0,2),(1,1),(1,2),(2,1),(2,2)\}$} & $C_7$\\
\hline
$4$&$y^2=x^3 - x -1$  &{$\{\infty\}$}&$\{1\}$\\
\hline
$5$&$y^2=x^3 + x^2 - 1$ &{$\{\infty,(1,1), (1,2)\}$} & $C_3$ \\
\hline
$6$&$y^2=x^3 + x^2 + 1$ & {$\{\infty,(0,1), (0,2), (1,0),(2,1), (2,2)\}$} & $C_6$ \\
\hline
$7$&$y^2=x^3 - x^2 + 1$ & {$\{\infty,(0,1), (0,2), (1,1), (1,2),\}$} & $C_5$ \\
\hline
$8$&$y^2=x^3 - x^2 - 1$ & {$\{\infty,(2,0))\}$} & $C_2$ \\
\hline
\end{tabular}}
\end{block}
\pause

\ \hfill \begin{beamercolorbox}[center,wd=9cm]{postit}
Note: each $C_i, i=1,\ldots,7$ is represented by a curve $/\F_3$
            \end{beamercolorbox}\pause

            \begin{beamerboxesrounded}[upper=block title example,lower=block body alerted,shadow=true]{Exercise:
            let $\left(\frac{a}{q}\right)$ be the kronecker symbol. 
            Show that the number of non--isomorphic (i.e. inequivalent) classes of elliptic curves over $\F_q$ is }
$$2q+3+\left(\frac{-4}{q}\right)+2\left(\frac{-3}{q}\right)$$
  \end{beamerboxesrounded}
\end{frame}


\subsection{Further Examples}
\begin{frame}
\frametitle{EXAMPLE: Elliptic curves over $\F_5$ and $\F_4$}

$\forall E/\F_5$ (12 elliptic curves), $\#E(\F_5)\in \{2,3,4,5,6,7,8,9,10\}.$ $\forall n, 2\le n\le10 \exists! E/\F_5: \#E(\F_5)=n$
%each number corresponds to a unique curve
with the exceptions:

\begin{example}[Elliptic curves over $\F_5$]
\begin{itemize}[<+-| alert@+>]
 \item \alert{$E_1: y^2=x^3+1$} and \alert{$E_2: y^2=x^3+2$}\hfill both order $6$\\
 \begin{columns}
\begin{column}{4cm}
\begin{beamercolorbox}[shadow=true,center,rounded=true,wd=2.5cm]{postit}
        $\begin{cases}
x\longleftarrow 2x\\
y\longleftarrow \sqrt{3}y
  \end{cases}$\end{beamercolorbox}
 \end{column}
 \begin{column}{5cm}
$E_1$ and $E_2$ affinely equivalent over $\F_5[\sqrt{3}]=\F_{25}$ (\emph{twists})
 \end{column}
 \end{columns}
\item \alert{$E_3: y^2=x^3+x$} and \alert{$E_4: y^2=x^3+x+2$}
\hfill order $4$
$$E_3(\F_5)\cong C_2\oplus C_2\qquad E_4(\F_5)\cong C_4$$
\item \alert{$E_5: y^2=x^3+4x$} and \alert{$E_6: y^2=x^3+4x+1$}
\hfill both order $8$
$$E_5(\F_5)\cong C_2\times\oplus C_4\qquad E_6(\F_5)\cong C_8$$
\item \alert{$E_7: y^2=x^3+x+1$}\hfill  order $9$ and $E_7(\F_5)\cong C_9$
\end{itemize}
\end{example}\pause\vspace*{-3pt}
\begin{beamerboxesrounded}[upper=block title example,lower=block body alerted,shadow=true]{\textbf{Exercise:} Classify all elliptic curves over $\F_4=\F_2[\xi], \xi^2=\xi+1$}
 \end{beamerboxesrounded}
\end{frame}

%\subsection{Further reading.}
\begin{frame}
\frametitle{Further Reading...}
\begin{scriptsize}
\begin{thebibliography}{99}
\bibitem{BSS} \textsc{Ian~F.~Blake,~Gadiel~Seroussi,~and~Nigel~P.~Smart},
Advances in elliptic curve cryptography, London Mathematical Society Lecture Note Series, vol. 317, Cambridge University Press, Cambridge, 2005.
 \bibitem{C} \textsc{J.~W.~S.~Cassels},
Lectures on elliptic curves, London Mathematical Society Student Texts, vol. 24, Cambridge University Press, Cambridge, 1991.
 \bibitem{CR} \textsc{John~E.~Cremona},
Algorithms for modular elliptic curves, 2nd ed., Cambridge University Press, Cambridge, 1997.
 \bibitem{Kn} \textsc{Anthony~W.~Knapp},
Elliptic curves, Mathematical Notes, vol. 40, Princeton University Press, Princeton, NJ, 1992.
 \bibitem{Ko} \textsc{Neal~Koblitz},
Introduction to elliptic curves and modular forms, Graduate Texts in Mathematics, vol. 97, Springer-Verlag, New York, 1984.
 %\bibitem{Po} \textsc{Poonen B} Elliptic curves (introduction)(19s) notes
 \bibitem{Sil} \textsc{Joseph~H.~Silverman},
The arithmetic of elliptic curves, Graduate Texts in Mathematics, vol. 106, Springer-Verlag, New York, 1986.
\bibitem{ST} \textsc{Joseph~H.~Silverman~and~John~Tate},
Rational points on elliptic curves, Undergraduate Texts in Mathematics, Springer-Verlag, New York, 1992.
\bibitem{washington} \textsc{Lawrence~C.~Washington},
Elliptic curves: Number theory and cryptography, 2nd ED. Discrete Mathematics and Its Applications, Chapman \& Hall/CRC, 2008.
\bibitem{Zimm} \textsc{Horst~G.~Zimmer},
Computational aspects of the theory of elliptic curves, Number theory and applications
(Banff, AB, 1988) NATO Adv. Sci. Inst. Ser. C Math. Phys. Sci., vol. 265, Kluwer Acad. Publ., Dordrecht, 1989, pp. 279--324.
\end{thebibliography}
\end{scriptsize}
\end{frame}

\end{document}


