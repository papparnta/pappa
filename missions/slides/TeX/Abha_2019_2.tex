\documentclass[handout]{beamer}%,[handout] %,hyperref={pdfpagelabels=false},draft,handout,handout
\usepackage[orientation=landscape,size=custom,width=16,
height=9,scale=0.42,debug]{beamerposter} 
\usepackage[english]{babel}
\usepackage{lmodern}% http://ctan.org/pkg/lm
\usepackage[latin1]{inputenc}
\usepackage{times,hyperref,tikz,colortbl,yfonts,translator}
\usepackage[T1]{fontenc}
 \newcommand{\Q}{\mathbb Q}
 \newcommand{\Z}{\mathbb Z}
 \newcommand{\N}{\mathbb N}
 \newcommand{\F}{\mathbb F}
 \newcommand{\C}{\mathbb C}
 \newcommand{\R}{\mathbb R}
%\useoutertheme[height=0pt,width=2cm,right]{sidebar}
\usecolortheme{rose,sidebartab}
\useinnertheme{circles}
\usefonttheme[only large]{structurebold}
\theoremstyle{definition}
\newtheorem{exercise}[theorem]{\translate{Exercise}}
\newtheorem{Note}[theorem]{\translate{Note}}
\lecture[4]{Elliptic curves over finite fields}{First Lecture}
\title[Elliptic curves over $\F_{q}$]{\insertlecture}
\setbeamercolor{formul}{fg=black,bg=pink}
\setbeamercolor{sidebar right}{bg=green!15}
\setbeamercolor{structure}{fg=black!120}
\setbeamercolor{postit}{fg=black,bg=yellow}
\setbeamercolor{greys}{fg=black,bg==black!25}
\setbeamerfont{title in sidebar}{series=\bfseries}
\setbeamerfont*{item}{series=}
\setbeamerfont{frametitle}{size=}
\setbeamerfont{block title}{size=\small}
\setbeamerfont{subtitle}{size=\normalsize,series=\normalfont}
\begin{document}

\begin{frame}
\includegraphics[width=1.6cm]{images/roma3.pdf}\hfill\includegraphics[width=1.9cm]{images/kku.jpeg}
\vfill

\begin{center}\begin{sc}
\begin{Large}

\textcolor{red}{Elliptic curves Cryptography}
\end{Large}\bigskip

\ {Francesco Pappalardi}\bigskip\bigskip

\begin{large}\begin{bf}\#2 - Second Lecture.
\end{bf}\end{large}\medskip

June $17^{\text{th}}$ 2019\medskip
\vfill
\end{sc}\end{center}

%\includegraphics[width=1.6cm]{images/cimpalogo.pdf}\hfill
\begin{minipage}[b]{9.3cm}
\textsc{WAMS School:\\o
Introductory topics in Number Theory\\ and Differential Geometry}\\
\textbf{King Khalid University}\\
Abha, Saudi Arabia
\end{minipage}\hfill
%\includegraphics[width=1.9cm]{images/seams.png}
\end{frame}

\begin{frame}
\begin{beamercolorbox}[shadow=true,left,rounded=true,wd=12cm]{formul}
$E/\F_q$ elliptic curve ($D_E=D_E(a_1,a_2,a_3,a_4,a6)\neq0$)\\
$E(\F_q)=\{(x,y)\in \F_q^2:\ y^2+a_1xy+a_3y=x^3+a_2x^2+a_4x+a_6\}\cup\{\infty\}$
\end{beamercolorbox}

\includegraphics[width=7cm]{images/add7.pdf}
\end{frame}


\begin{frame}
\frametitle{Properties of the operation ``$+_E$''}

\begin{Theorem}
 The addition law on $E(\F_q)$ has the following
properties:
\begin{enumerate}[<+-| alert@+>][(a)]
 \item $P+_EQ\in E(\F_q)\hfill\forall P,Q\in E(\F_q)$
 \item  $P+_E\infty=\infty+_E P=P\hfill\forall P\in E(\F_q)$
 \item  $P+_E(-P)=\infty\hfill\forall P\in E(\F_q)$
 \item  $P+_E(Q +_E R)=(P+_E Q)+_E R\hfill\forall P,Q,R\in E(\F_q)$
 \item  $P+_E Q=Q +_E P\hfill\forall P,Q\in E(\F_q)$
\end{enumerate}
 \end{Theorem}\pause

\begin{itemize}[<+-| alert@+>]
%  \item By ``a point of $E/\F_q$ ($P\in E$)'' we mean $P\in E(\overline{\F_q})$
%  in analogy for $E/\Q$ where ``a point of $E$'' means  $P\in E(\C)$
 \item $\left(E(\F_q),+_E\right)$  \alert{commutative group}
 \item All group properties are easy except \alert{associative law (d)}
 \item Geometric proof of associativity uses \emph{Pappo's Theorem}
% \item We shall comment on how to do it by explicit computation
 \item can substitute $\F_q$ with any field $K$; Theorem holds for $\left(E(K),+_E\right)$
\item $-P=-(x_1,y_1)=(x_1,-a_1x_1-a_3-y_1)$
%In particular, if $E/\F_q$, can consider the groups $E(\overline{\F_q})$ or $E(\F_{q^n})$
\end{itemize}
\end{frame}

% \begin{frame}
% \frametitle{Computing the inverse $-P$}
% \centerline{\begin{beamercolorbox}[shadow=true,center,rounded=true,wd=7cm]{formul}
% $E: y^2+a_1xy+a_3y=x^3+a_2x^2+a_4x+a_6$\end{beamercolorbox}}\pause
% 
% If $P=(x_1,y_1)\in E(\F_q)$
% 
% \ \hfill\begin{beamercolorbox}[shadow=true,left,rounded=true,wd=8cm]{formul}
%            \textbf{\color[rgb]{1,0.3,1}Definition:}  $-P:=P'$ where $r_{P,\infty}\cap E(\F_q)=\{P,\infty,P'\}$\hfill\
%             \end{beamercolorbox}\hfill\pause
% 
%             Write $P'=(x_1',y_1')$. Since $r_{P,\infty}: x=x_1\ \Rightarrow x_1'=x_1$ and $y_1$ satisfies
% \centerline{\begin{beamercolorbox}[shadow=true,center,rounded=true,wd=8.8cm]{postit}
% $y^2+a_1x_1y+a_3y-(x_1^3+a_2x_1^2+a_4x_1+a_6)=(y-y_1)(y-y_1')$
% \end{beamercolorbox}}\bigskip\pause
% 
% So $y_1+y_1'=-a_1x_1-a_3$ (\alert{both coefficients of $y$}) and
% \centerline{\begin{beamercolorbox}[shadow=true,center,rounded=true,wd=6cm]{postit}
% $-P=-(x_1,y_1)=(x_1,-a_1x_1-a_3-y_1)$
%  \end{beamercolorbox}}\bigskip\pause
% 
% 
% So, if $P_1=(x_1,y_1), P_2=(x_2,y_2)\in E(\F_q)$,
% 
% \ \hfill\begin{beamercolorbox}[shadow=true,center,rounded=true,wd=9cm]{formul}
%            \textbf{\color[rgb]{1,0.3,1}Definition:} $P_1+_EP_2=-P_3$ where $r_{P_1,P_2}\cap E(\F_q)=\{P_1,P_2,P_3\}\!$
%             \end{beamercolorbox}\bigskip\pause
% 
% Finally, if $P_3=(x_3,y_3)$, then
% \centerline{\begin{beamercolorbox}[shadow=true,center,rounded=true,wd=6cm]{postit}
% $P_1+_EP_2=-P_3=(x_3,-a_1x_3-a_3-y_3)$
% \end{beamercolorbox}}
% \end{frame}

% \begin{frame}
% \frametitle{Lines through points of $E$}
% \centerline{\begin{beamercolorbox}[shadow=true,center,rounded=true,wd=6cm]{formul}
% $E: y^2+a_1xy+a_3y=x^3+a_2x^2+a_4x+a_6$\end{beamercolorbox}}
% where $a_1, a_3, a_2, a_4 ,a_6\in\F_q,$ \pause
% 
% \begin{beamerboxesrounded}[upper=block title example,lower=block body alerted,shadow=true]
% {$P_1=(x_1,y_1), P_2=(x_2,y_2)\in E(\F_q)$}
% \begin{enumerate}[<+-| alert@+>]
%  \item $P_1\neq P_2$ and $x_1\neq x_2\hfil \Longrightarrow\hfil r_{P_1,P_2}: y=\lambda x+\nu$
% \begin{beamercolorbox}[shadow=true,center,rounded=true,wd=6cm]{postit}
% $$\lambda= \frac{y_2-y_1}{x_2-x_1},\qquad \nu=\frac{y_1x_2-x_1y_2}{x_2-x_1}$$
% \end{beamercolorbox}
%  \item $P_1\neq P_2$ and $x_1=x_2\hfil \Longrightarrow\hfil r_{P_1,P_2}: x=x_1$
%  \item  $P_1=P_2$ and $2y_1+a_1x_1+a_3\neq0\ \Longrightarrow\ r_{P_1,P_2}: y=\lambda x+\nu$
% \hspace*{-0.5cm}\begin{beamercolorbox}[shadow=true,center,rounded=true,wd=9.3cm]{postit}
% $$\lambda=\frac{3x_1^2+2a_2x_1+a_4-a_1y_1}{2y_1+a_1x_1+a_3},
%  \nu=-\frac{a_3y_1+x_1^3-a_4x_1-2a_6}{2y_1+a_1x_1+a_3}$$
% \end{beamercolorbox}
% \item $P_1=P_2$ and $2y_1+a_1x_1+a_3=0\hfil \Longrightarrow\hfil r_{P_1,P_2}: x=x_1$
% \item $r_{P_1,\infty}: x=x_1\hfill r_{\infty,\infty}: Z=0$
% \end{enumerate}
% \end{beamerboxesrounded}
% \end{frame}
% 
% \begin{frame}
% \frametitle{Intersection between a line and $E$}
% We want to compute $P_3= (x_3,y_3)$ where $r_{P_1,P_2}: y=\lambda x +\nu$,
% \centerline{\begin{beamercolorbox}[shadow=true,center,rounded=true,wd=5cm]{postit}
% $r_{P_1,P_2}\cap E(\F_q)=\{P_1,P_2,P_3\}$
% \end{beamercolorbox}}
% \pause
% 
% We find the intersection:
% \centerline{\begin{beamercolorbox}[shadow=true,center,rounded=true,wd=7cm]{formul}
% $r_{P_1,P_2}\cap E(\F_q)=$ \scriptsize{\ $\begin{cases}
%  E:y^2+a_1xy+a_3y=x^3+a_2x^2+a_4x+a_6\\ r_{P_1,P_2}: y=\lambda x +\nu
%  \end{cases}$}
% \end{beamercolorbox}}
%  \pause
% Substituting\\
% \centerline{\begin{beamercolorbox}[center,wd=8cm]{postit}
% $(\lambda x +\nu)^2+a_1x(\lambda x +\nu)+a_3(\lambda x +\nu)=x^3+a_2x^2+a_4x+a_6$
%             \end{beamercolorbox}}\medskip\pause
% 
%             Since $x_1$ and $x_2$ are solutions, we can
% find $x_3$ by comparing
% \begin{scriptsize}
% 
% \centerline{\begin{beamercolorbox}[center,wd=9cm]{postit}
%     \begin{align*}
% &x^3+a_2x^2+a_4x+a_6-((\lambda x +\nu)^2+a_1x(\lambda x +\nu)+a_3(\lambda x +\nu))&=\\
% \uncover<5->{&x^3+(\alert{a_2-\lambda^2-a_1\lambda})x^2+\cdots&=\\ }
% \uncover<6->{&(x - x_1)(x - x_2)(x - x_3) = x^3 - ({\color[rgb]{0,0,1}x_1 + x_2 + x_3})x^2 + \cdots&\\ } %(x_1x_2 + x_1x_3 + x_2x_3)x - x_1x_2x_3
% \notag
% \end{align*}\vskip-1.5em
% \end{beamercolorbox}}
% \end{scriptsize}\pause
% 
% Equating coeffcients of $x^2$,
% \centerline{\begin{beamercolorbox}[center,wd=8cm]{postit}
% $x_3 = \lambda^2-a_1\lambda-a_2- x_1-x_2,\qquad y_3 = \lambda x_3 + \nu$
%             \end{beamercolorbox}}
% \pause
% Finally\\
% \centerline{\begin{beamercolorbox}[shadow=true,center,rounded=true,wd=9cm]{formul}
% \small{$P_3 =({\color[cmyk]{0,1,1,0.5}\lambda^2-a_1\lambda-a_2-x_1-x_2},{\color[cmyk]{1,0,1,0.5}\lambda^3-a_1\lambda^2-\lambda(a_2+x_1+x_2)+\nu})$}
%             \end{beamercolorbox}}
% \end{frame}

\begin{frame}
\frametitle{Formulas for Addition on $E$ (Summary)}
\centerline{\begin{beamercolorbox}[shadow=true,center,rounded=true,wd=8cm]{formul}
$E: y^2+a_1xy+a_3y=x^3+a_2x^2+a_4x+a_6$\end{beamercolorbox}}
$P_1 = (x_1, y_1), P_2 = (x_2, y_2)\in E(\F_q)\setminus\{\infty\}$,
\begin{beamerboxesrounded}[upper=block title example,lower=block body alerted,shadow=true]{Addition Laws for the sum of affine points}
\begin{itemize}[<+-| alert@+>]
 \item If $P_1\neq P_2$
\begin{itemize}
 \item $x_1 = x_2\ \hfill\Rightarrow\hfil$\ \ \
\begin{beamercolorbox}[shadow=true,center,rounded=true,wd=2cm]{formul}$P_1 +_E P_2 = \infty$
\end{beamercolorbox}
 \item $x_1 \neq x_2$\\
\centerline{\begin{beamercolorbox}[shadow=true,center,wd=7cm]{postit}
             $\displaystyle\lambda=\frac{y_2-y_1}{x_2-x_1}\qquad \nu=\frac{y_1x_2-y_2x_1}{x_2-x_1}$
            \end{beamercolorbox}}
 \end{itemize}
\item If $P_1 = P_2$
\begin{itemize}
 \item $2y_1+a_1x+a_3 = 0\ \hfill\Rightarrow\hfil$\ \ \
\begin{beamercolorbox}[shadow=true,center,rounded=true,wd=3cm]{formul}$P_1 +_E P_2 = 2P_1 = \infty$\end{beamercolorbox}
\item $2y_1+a_1x+a_3\neq 0$\\
\centerline{\begin{beamercolorbox}[shadow=true,center,wd=10cm]{postit}
$\displaystyle\lambda=\frac{3x_1^2+2a_2x_1+a_4-a_1y_1}{2y_1+a_1x+a_3}, \nu=-\frac{a_3y_1+x_1^3-a_4x_1-2a_6}{2y_1+a_1x_1+a_3}$
            \end{beamercolorbox}}
\end{itemize}
\end{itemize}\pause

Then\\
\centerline{\begin{beamercolorbox}[shadow=true,center,rounded=true,wd=12cm]{formul}
{\scriptsize $P_1 +_E P_2 = ({\color[cmyk]{0,1,1,0.5}\lambda^2-a_1\lambda-a_2-x_1-x_2},
{\color[cmyk]{1,0,1,0.5}-\lambda^3-a_1^2\lambda+(\lambda+a_1)(a_2+x_1+x_2)-a_3-\nu})$}
            \end{beamercolorbox}}
\end{beamerboxesrounded}
\end{frame}

\begin{frame}
\frametitle{Formulas for Addition on $E$ (Summary for special equation)}
\centerline{\begin{beamercolorbox}[shadow=true,center,rounded=true,wd=6cm]{formul}
$E: y^2=x^3+Ax+B$\end{beamercolorbox}}
$P_1 = (x_1, y_1), P_2 = (x_2, y_2)\in E(\F_q)\setminus\{\infty\}$,
\begin{beamerboxesrounded}[upper=block title example,lower=block body alerted,shadow=true]{Addition Laws for  the sum of affine points}
\begin{itemize}
 \item If $P_1\neq P_2$
\begin{itemize}
 \item $x_1 = x_2\ \hfill\Rightarrow\hfil$\ \ \
\begin{beamercolorbox}[shadow=true,center,rounded=true,wd=2cm]{formul}$P_1 +_E P_2 = \infty$
\end{beamercolorbox}
 \item $x_1 \neq x_2$\\
\centerline{\begin{beamercolorbox}[shadow=true,center,wd=7cm]{postit}
             $\displaystyle\lambda=\frac{y_2-y_1}{x_2-x_1}\qquad \nu=\frac{y_1x_2-y_2x_1}{x_2-x_1}$
            \end{beamercolorbox}}
 \end{itemize}
\item If $P_1 = P_2$
\begin{itemize}
 \item $y_1 = 0\ \hfill\Rightarrow\hfil$\ \ \
\begin{beamercolorbox}[shadow=true,center,rounded=true,wd=3cm]{formul}$P_1 +_E P_2 = 2P_1 = \infty$\end{beamercolorbox}
\item $y_1\neq 0$\\
\centerline{\begin{beamercolorbox}[shadow=true,center,wd=7cm]{postit}
$\displaystyle\lambda=\frac{3x_1^2+A}{2y_1}, \nu=-\frac{x_1^3-Ax_1-2B}{2y_1}$
            \end{beamercolorbox}}
\end{itemize}
\end{itemize}

Then\\
\centerline{\begin{beamercolorbox}[shadow=true,center,rounded=true,wd=11cm]{formul}
{\small $P_1 +_E P_2 = ({\color[cmyk]{0,1,1,0.5}\lambda^2-x_1-x_2},
{\color[cmyk]{1,0,1,0.5}-\lambda^3+\lambda(x_1+x_2)-\nu})$}
            \end{beamercolorbox}}
\end{beamerboxesrounded}

\end{frame}


% \begin{frame}
%  \frametitle{A Finite Field Example}
% 
% Over $\F_p$ geometric pictures don't make sense.\pause
%  \begin{example}
% Let
% $E: y^2 = x^3 - 5x + 8 /\F_{37}$,\pause\hfill $P = (6, 3) , Q = (9, 10)\in E(\F_{37})$\pause
% 
% \centerline{\begin{beamercolorbox}[shadow=true,center,rounded=true,wd=6cm]{formul}
% $r_{P,Q}: y=27x+26\quad r_{P,P}: y=11x+11 $
%             \end{beamercolorbox}}\pause
% 
% \centerline{\begin{beamercolorbox}[shadow=true,center,rounded=true,wd=9cm]{postit}
% $r_{P,Q}\cap E(\F_{37})=\begin{cases}
%                           y^2 = x^3 - 5x + 8 \\ y = 27 x + 26
%                          \end{cases}\!\!=\{(6,3), (9,10), (11,27)\}$
%                                      \end{beamercolorbox}}\pause
% 
% \centerline{\begin{beamercolorbox}[shadow=true,center,rounded=true,wd=9cm]{postit}
% $r_{P,P}\cap E(\F_{37})=\begin{cases}
%                           y^2 = x^3 - 5x + 8 \\ y =11 x + 11
%                          \end{cases}\!\!=\{(6,3), (6,3), (35,26)\}$\end{beamercolorbox}}\pause
% 
% 
% \centerline{\begin{beamercolorbox}[shadow=true,center,rounded=true,wd=7.3cm]{formul}
% $P+_EQ=(11,10)\qquad 2P=(35,11)$
%             \end{beamercolorbox}}\pause
% 
% \ \hfill\scriptsize{$3P=(34,25), 4P=(8,6), 5P=(16,19),\ldots 3P+4Q=(31,28),\ldots$}
%  \end{example}\pause
% 
%  \begin{beamerboxesrounded}[upper=block title example,lower=block body alerted,shadow=true]{Exercise}
%  $\bullet$ Compute the order and the {\color[rgb]{0.1,0.3,1}{Group Structure}} of $E(\F_{37})$\\
%  $\bullet$ Show that if $E_1/\F_q$ is equivalent to $E_2/\F_q$, then $E_1(\F_{q^n})\cong E_2(\F_{q^n})\forall n\in\N$.
%   \end{beamerboxesrounded}
% \end{frame}

\begin{frame}%[label=current]
 \frametitle{Group Structure}

\begin{theorem}[Classification of finite abelian groups]
 If $G$ is {\color[rgb]{0.9,0.3,0.2}{abelian and finite}},  $\exists n_1,\ldots,n_k\in\N^{>1}$ such that
 \begin{enumerate}[<+-| alert@+>]
\item $n_1\mid n_2\mid\cdots\mid n_k$
\item $G\cong C_{n_1}\oplus\cdots\oplus C_{n_k}$
\end{enumerate}
\ \hfill Furthermore $n_1,\ldots,n_k$ ({\color[rgb]{0.9,0.3,0.2} Group Structure}) are unique
 \end{theorem}\pause

% \begin{example}[One can verify that:]
%  $$C_{2400}\oplus C_{72} \oplus C_{1440}\cong C_{12}\oplus C_{60}\oplus C_{15200}$$
% \end{example}\pause

\begin{theorem}[Structure Theorem for Elliptic curves over a finite field] Let $E/\F_q$ be 
an elliptic curve, then
$$E(\F_q)\cong C_n\oplus C_{nk}\qquad\exists n,k\in\N^{>0}.$$
(i.e. $E(\F_q)$ is either cyclic ($n=1$) or the product of $2$ cyclic groups)
\pause
\end{theorem}            
\end{frame}

% \begin{frame}%[label=current2]
%  \frametitle{Proof of the associativity}
%  \centerline{\begin{beamercolorbox}[shadow=true,center,rounded=true,wd=7.3cm]{formul}
%              $P+_E(Q+_ER)=(P+_EQ)+_ER\quad\forall P,Q,R\in E$
%             \end{beamercolorbox}}\pause
%  We should verify the above in many different cases according if $Q=R$, $P=Q$, $P=Q+_ER,\ldots$\pause
% 
% Here we deal with the \emph{generic case}. i.e. All the points
% \alert{$\pm P, \pm R,\pm Q,\pm(Q+_ER),\pm(P+_EQ),\infty$} all different
% 
% \ \hfill {\begin{beamercolorbox}[shadow=true,left,rounded=true,wd=9.6cm]{postit}
%  {\small{\color[rgb]{1,0.1,0.1}\texttt{Mathematica code}}\\
% \texttt{L[x\_,y\_,r\_,s\_]:=(s-y)/(r-x);\\
% M[x\_,y\_,r\_,s\_]:=(yr-sx)/(r-x);\\
% A[\{x\_,y\_\},\{r\_,s\_\}]:=\{(L[x,y,r,s])$^2$-(x+r),\\
% \ \hfill -(L[x,y,r,s])$^3$+L[x,y,r,s](x+r)-M[x,y,r,s]\}\\
% Together[A[A[\{x,y\},\{u,v\}],\{h,k\}]-A[\{x,y\},A[\{u,v\},\{h,k\}]]]\\
% det = Det[(\{\{1,x$_1$,x$_1^3$-y$_1^2$\},\{1,x$_2$,x$_2^3$-y$_2^2$\},\{1,x$_3$,x$_3^3$-y$_3^2$\}\})]\\
% PolynomialQ[Together[Numerator[Factor[res[[1]]]]/det],\\
% \ \hfill\{x$_1$,x$_2$,x$_3$,y$_1$,y$_2$,y$_3$\}]
% PolynomialQ[Together[Numerator[Factor[res[[2]]]]/det],\\ \ \hfill\{x$_1$,x$_2$,x$_3$,y$_1$,y$_2$,y$_3$\}]}}
%              \end{beamercolorbox}}\pause
% 
% 
% % {\begin{beamercolorbox}[shadow=true,left,rounded=true,wd=9.6cm]{postit}
% %  \scriptsize{{\color[rgb]{1,0.1,0.1}\texttt{Pari code}}\\
% % \texttt{L(a,b,c,d)=(d-b)/(c-a);\\
% % M(a,b,c,d)=(b*c-a*d)/(c-a);\\
% % AX(a,b,c,d)=L(a,b,c,d)\^{}2-(a+c);\\
% % AY(a,b,c,d)=-L(a,b,c,d)\^{}3+L(a,b,c,d)*(a+c)-M(a,b,c,d);\\
% % simplify(AX(x1,y1,AX(x2,y2,x3,y3),AY(x2,y2,x3,y3))-\\ \ \hfill AX(AX(x1,y1,x2,y2),AY(x1,y1,x2,y2),x3,y3));\\
% % simplify(AY(x1,y1,AX(x2,y2,x3,y3),AY(x2,y2,x3,y3))-\\ \ \hfill AY(AX(x1,y1,x2,y2),AY(x1,y1,x2,y2),x3,y3))}}
% % \end{beamercolorbox}}\pause
% 
% \begin{small}
% \begin{itemize}[<+-| alert@+>]
%  \item runs in 2 seconds on a PC
%  %\item Complete proof requires several cases (e.g. $(P+_EP)+_ER=P+_E(P+_ER)$)
%  \item For an elementary proof:
%  ``\text{An Elementary Proof of the Group Law for Elliptic Curves.}''
% Department of Mathematics: Rice
% University. Web. 20 Nov. 2009.\\ \ \hfill\texttt{http://math.rice.edu/\~{}friedl/papers/AAELLIPTIC.PDF}
% \item More cases to check. e.g  \alert{$P+_E2Q=(P+_EQ)+_EQ$}
% \end{itemize}
% \end{small}
% \end{frame}



\section{Examples}
\subsection{Structure of \texorpdfstring{$E(\F_2)$ and $E(\F_3)$}{E(F2) and E(F3)}}
\begin{frame}
\frametitle{EXAMPLE: Elliptic curves over $\F_2$}

From our previous list:
\begin{block}{Groups of points of curves over $\F_2$}

\centerline{
\begin{tabular}{|l|c|l|}
\hline
 $E$ & $E(\F_2)$ & $E(\F_2)$\\
\hline
 $y^2+xy=x^3+x^2+1$ & $\{\infty,(0,1)\}$& $C_2$\\
$y^2+xy=x^3+1$ & $\{\infty,(0,1),(1,0),(1,1)\}$ & $C_4$\\
$y^2+y=x^3+x$&$\{\infty,(0,0),(0,1), (1,0),(1,1)\}$&$C_5$\\
 $y^2+y=x^3+x+1$ &$\{\infty\}$&$1$\\
$y^2+y=x^3$ & $\{\infty,(0,0), (0,1)\}$ & $C_3$ \\
\hline
\end{tabular}}
\end{block}
\ \hfill \begin{beamercolorbox}[center,wd=10cm]{postit}
Note: each $C_i, i=1,\ldots,5$ is represented by a curve $/\F_2$
            \end{beamercolorbox}\pause
\end{frame}

\begin{frame}
\frametitle{EXAMPLE: Elliptic curves over $\F_3$}

From our previous list:

\begin{block}{Groups of points of curves over $\F_3$}\centerline{
\begin{tabular}{|l|r|c|c|}
\hline
$i$ & $E_i$ & $E_i(\F_3)$ &$E_i(\F_3)$\\
\hline
$1$& $y^2=x^3+x$ & {\scriptsize $\{\infty,(0,0),(2,1),(2,2)\}$}& $C_4$\\
\hline
$2$&$y^2=x^3 - x$ & {\scriptsize $\{\infty,(1,0),(2,0),(0,0)\}$} & $C_2\oplus C_2$\\
\hline
$3$&$y^2=x^3 - x +1$&{\scriptsize $\{\infty,(0,1),(0,2),(1,1),(1,2),(2,1),(2,2)\}$} & $C_7$\\
\hline
$4$&$y^2=x^3 - x -1$  &{\scriptsize $\{\infty\}$}&$\{1\}$\\
\hline
$5$&$y^2=x^3 + x^2 - 1$ &{\scriptsize $\{\infty,(1,1), (1,2)\}$} & $C_3$ \\
\hline
$6$&$y^2=x^3 + x^2 + 1$ & {\scriptsize $\{\infty,(0,1), (0,2), (1,0),(2,1), (2,2)\}$} & $C_6$ \\
\hline
$7$&$y^2=x^3 - x^2 + 1$ & {\scriptsize $\{\infty,(0,1), (0,2), (1,1), (1,2),\}$} & $C_5$ \\
\hline
$8$&$y^2=x^3 - x^2 - 1$ & {\scriptsize $\{\infty,(2,0))\}$} & $C_2$ \\
\hline
\end{tabular}}
\end{block}
\pause

\ \hfill \begin{beamercolorbox}[center,wd=10cm]{postit}
Note: each $C_i, i=1,\ldots,7$ is represented by a curve $/\F_3$
            \end{beamercolorbox}\pause

\end{frame}


% \subsection{Structure of \texorpdfstring{$E(\F_3)$}{E(F3)}}
% \begin{frame}
% \frametitle{EXAMPLE: Elliptic curves over $\F_3$}
% From our previous list:
% 
% \begin{block}{Groups of points}\centerline{
% \begin{tabular}{|l|r|c|c|}
% \hline
% $i$ & $E_i$ & $E_i(\F_3)$ &$E_i(\F_3)$\\
% \hline
% $1$& $y^2=x^3+x$ & {$\{\infty,(0,0),(2,1),(2,2)\}$}& $C_4$\\
% \hline
% $2$&$y^2=x^3 - x$ & {$\{\infty,(1,0),(2,0),(0,0)\}$} & $C_2\oplus C_2$\\
% \hline
% $3$&$y^2=x^3 - x +1$&{$\{\infty,(0,1),(0,2),(1,1),(1,2),(2,1),(2,2)\}$} & $C_7$\\
% \hline
% $4$&$y^2=x^3 - x -1$  &{$\{\infty\}$}&$\{1\}$\\
% \hline
% $5$&$y^2=x^3 + x^2 - 1$ &{$\{\infty,(1,1), (1,2)\}$} & $C_3$ \\
% \hline
% $6$&$y^2=x^3 + x^2 + 1$ & {$\{\infty,(0,1), (0,2), (1,0),(2,1), (2,2)\}$} & $C_6$ \\
% \hline
% $7$&$y^2=x^3 - x^2 + 1$ & {$\{\infty,(0,1), (0,2), (1,1), (1,2),\}$} & $C_5$ \\
% \hline
% $8$&$y^2=x^3 - x^2 - 1$ & {$\{\infty,(2,0))\}$} & $C_2$ \\
% \hline
% \end{tabular}}
% \end{block}
% \pause
% 
% \ \hfill \begin{beamercolorbox}[center,wd=9cm]{postit}
% Note: each $C_i, i=1,\ldots,7$ is represented by a curve $/\F_3$
%             \end{beamercolorbox}\pause
% 
%             \begin{beamerboxesrounded}[upper=block title example,lower=block body alerted,shadow=true]{Exercise:
%             let $\left(\frac{a}{q}\right)$ be the Kronecker symbol. 
%             Show that the number of non--isomorphic (i.e. inequivalent) classes of elliptic curves over $\F_q$ is }
% $$2q+3+\left(\frac{-4}{q}\right)+2\left(\frac{-3}{q}\right)$$
%   \end{beamerboxesrounded}
% \end{frame}


% \subsection{Further Examples}
% \begin{frame}
% \frametitle{EXAMPLE: Elliptic curves over $\F_5$ and $\F_4$}
% 
% $\forall E/\F_5$ (12 elliptic curves), $\#E(\F_5)\in \{2,3,4,5,6,7,8,9,10\}.$ $\forall n, 2\le n\le10 \exists! E/\F_5: \#E(\F_5)=n$
% %each number corresponds to a unique curve
% with the exceptions:
% 
% \begin{example}[Elliptic curves over $\F_5$]
% \begin{itemize}[<+-| alert@+>]
%  \item \alert{$E_1: y^2=x^3+1$} and \alert{$E_2: y^2=x^3+2$}\hfill both order $6$\\
%  \begin{columns}
% \begin{column}{4cm}
% \begin{beamercolorbox}[shadow=true,center,rounded=true,wd=2.5cm]{postit}
%         $\begin{cases}
% x\longleftarrow 2x\\
% y\longleftarrow \sqrt{3}y
%   \end{cases}$\end{beamercolorbox}
%  \end{column}
%  \begin{column}{5cm}
% $E_1$ and $E_2$ affinely equivalent over $\F_5[\sqrt{3}]=\F_{25}$ (\emph{twists})
%  \end{column}
%  \end{columns}
% \item \alert{$E_3: y^2=x^3+x$} and \alert{$E_4: y^2=x^3+x+2$}
% \hfill order $4$
% $$E_3(\F_5)\cong C_2\oplus C_2\qquad E_4(\F_5)\cong C_4$$
% \item \alert{$E_5: y^2=x^3+4x$} and \alert{$E_6: y^2=x^3+4x+1$}
% \hfill both order $8$
% $$E_5(\F_5)\cong C_2\times\oplus C_4\qquad E_6(\F_5)\cong C_8$$
% \item \alert{$E_7: y^2=x^3+x+1$}\hfill  order $9$ and $E_7(\F_5)\cong C_9$
% \end{itemize}
% \end{example}\pause\vspace*{-3pt}
% \begin{beamerboxesrounded}[upper=block title example,lower=block body alerted,shadow=true]{\textbf{Exercise:} Classify all elliptic curves over $\F_4=\F_2[\xi], \xi^2=\xi+1$}
%  \end{beamerboxesrounded}
% \end{frame}

%\subsection{Further reading.}

\section{Points of finite order}

\subsection{Points of order 2}
\begin{frame}\frametitle{Determining points of order $2$}
Let $P=(x_1,y_1)\in E(\F_q)\setminus\{\infty\},$\\ \pause
\centerline{
 \begin{beamercolorbox}[rounded=true,shadow=true,wd=8cm,center]{postit}
$P$ has order $2\ \Longleftrightarrow\ 2P=\infty\ \Longleftrightarrow\ P=-P$
\end{beamercolorbox}}\pause
So
\centerline{\small{
 \begin{beamercolorbox}[rounded=true,shadow=true,wd=10cm,center]{formul}
$-P=(x_1,-a_1x_1-a_3-y_1)=(x_1,y_1)=P\ \pause \Longrightarrow\ 2y_1=-a_1x_1-a_3$\end{beamercolorbox}}}\pause\medskip

If $p\neq2$, can assume $E: y^2=x^3+Ax^2+Bx+C$\pause

\centerline{\small{
 \begin{beamercolorbox}[rounded=true,shadow=true,wd=10cm,center]{formul}
$-P=(x_1,-y_1)=(x_1,y_1)=P\ \pause \Longrightarrow\ y_1=0,
x_1^3+Ax_1^2+Bx_1+C=0$\hfill
\end{beamercolorbox}}}\pause\medskip

\begin{Note}
\begin{itemize}[<+-| alert@+>]
 \item the number of points of order $2$ in $E(\F_q)$ equals the number of roots of $X^3+Ax^2+Bx+C$ in $\F_q$
 \item roots are distinct since discriminant ${D}_E\neq0$
% \item $E(\F_{q^6})$ has always $3$ points of order $2$ if $E/\F_q$
% \item $E[2]:=\{P\in E(\bar{\F}_q): 2P=\infty\}\cong C_2\oplus C_2$
\end{itemize}
\end{Note}

% $\begin{cases}
%    2y=-a_1x-a_3\\
%    y^2+a_1xy+a_3y=x^3+a_2x^2+a_4x+a_6
%   \end{cases}\longrightarrow$, $\begin{cases}
%    2y=-a_1x-a_3\\
%    x^3+(a_2+a_1^2/4)x^2+(a_4+a_1a_3/2)x+a_6+a_3^2/4=0
%   \end{cases}$

\end{frame}

\begin{frame}\frametitle{Determining points of order $2$ (continues)}

% \begin{itemize}[<+-| alert@+>]
% \item If $p=2$ and $E: y^2+a_3y=x^3+a_2x^2+a_6$\pause
% 
%  \begin{beamercolorbox}[rounded=true,shadow=true,wd=10cm,center]{formul}
% $-P=(x_1,a_3+y_1)=(x_1,y_1)=P\ \pause \Longrightarrow\ a_3=0$\end{beamercolorbox}\pause\medskip
% 
% Absurd ($a_3=0$) and there are no points of order $2$.
% %$\begin{cases}
%  %  x=a_3/a_1\\
%   % y^2+a_1xy+a_3y+x^3+a_2x^2+a_4x+a_6=0
%   %\end{cases}\longrightarrow$,
% \item If $p=2$ and $E: y^2+xy=x^3+a_4x+a_6$\pause
% 
%  \begin{beamercolorbox}[rounded=true,shadow=true,wd=10cm,center]{formul}
% $-P=(x_1,x_1+y_1)=(x_1,y_1)=P\ \pause \Longrightarrow\ x_1=0,y_1^2=a_6$\end{beamercolorbox}\pause\medskip
% 
% So there is exactly one point of order $2$ namely $(0,\sqrt{a_6})$
% \end{itemize}\pause

\begin{Definition}{$2$--torsion points}\quad
$E[2]=\{P\in E(\overline{\F_q}): 2P=\infty\}.$
\end{Definition}\pause

\begin{beamerboxesrounded}[upper=block title example,lower=block body alerted,shadow=true]{FACTS:}
$$E[2]\cong \begin{cases}
C_2\oplus C_2 &\text{if }p>2\\
C_2           &\text{if }p=2, E: y^2+xy=x^3+a_4x+a_6\\
\{\infty\}    &\text{if }p=2, E: y^2+a_3y=x^3+a_2x^2+a_6
\end{cases}
$$
\end{beamerboxesrounded}\pause



\begin{block}{Each curve $/\F_2$ has cyclic $E(\F_2)$.}
\centerline{\begin{tabular}{|l|c|l|}
\hline
 $E$ & $E(\F_2)$ & $|E(\F_2)|$\\
\hline
 $y^2+xy=x^3+x^2+1$ & $\{\infty,(0,1)\}$& $2$\\
\hline
$y^2+xy=x^3+1$ & $\{\infty,(0,1),(1,0),(1,1)\}$ & $4$\\
\hline
$y^2+y=x^3+x$&$\{\infty,(0,0),(0,1),(1,0),(1,1)\}$&$5$\\
\hline
$y^2+y=x^3+x+1$ &$\{\infty\}$&$1$\\
\hline
$y^2+y=x^3$ & $\{\infty,(0,0), (0,1)\}$ & $3$ \\
\hline
\end{tabular}}
\end{block}

\end{frame}

% \begin{frame}
% \frametitle{Elliptic curves over $\F_2, \F_3$ and $\F_5$}
% \begin{block}{Each curve $/\F_2$ has cyclic $E(\F_2)$.}
% \centerline{\begin{tabular}{|l|c|l|}
% \hline
%  $E$ & $E(\F_2)$ & $|E(\F_2)|$\\
% \hline
%  $y^2+xy=x^3+x^2+1$ & $\{\infty,(0,1)\}$& $2$\\
% \hline
% $y^2+xy=x^3+1$ & $\{\infty,(0,1),(1,0),(1,1)\}$ & $4$\\
% \hline
% $y^2+y=x^3+x$&$\{\infty,(0,0),(0,1),(1,0),(1,1)\}$&$5$\\
% \hline
% $y^2+y=x^3+x+1$ &$\{\infty\}$&$1$\\
% \hline
% $y^2+y=x^3$ & $\{\infty,(0,0), (0,1)\}$ & $3$ \\
% \hline
% \end{tabular}}
% \end{block}
% \pause
% \begin{itemize}
%  \item $E_1: y^2=x^3+x\qquad\qquad E_2:  y^2=x^3-x$\\
% \centerline{\begin{beamercolorbox}[shadow=true,center,rounded=true,wd=7.5cm]{formul}
% $E_1(\F_3)\cong C_4\qquad\text{and}\qquad E_2(\F_3)\cong C_2\oplus C_2$
% \end{beamercolorbox}}
% \item $E_3: y^2=x^3+x\qquad\qquad E_4: y^2=x^3+x+2$\\
% \centerline{\begin{beamercolorbox}[shadow=true,center,rounded=true,wd=7.5cm]{formul}
% $E_3(\F_5)\cong C_2\oplus C_2\qquad\text{and}\qquad E_4(\F_5)\cong C_4$
% \end{beamercolorbox}}
% \item $E_5: y^2=x^3+4x\qquad\qquad E_6: y^2=x^3+4x+1$\\
% \centerline{\begin{beamercolorbox}[shadow=true,center,rounded=true,wd=7.5cm]{formul}
% $E_5(\F_5)\cong C_2\oplus C_4\qquad\text{and}\qquad E_6(\F_5)\cong C_8$
% \end{beamercolorbox}}
% \end{itemize}
% \end{frame}

\subsection{Points of order 3}
\begin{frame}\frametitle{Determining points of order $3$}
Let  $P=(x_1,y_1)\in E(\F_q)$

\centerline{
 \begin{beamercolorbox}[rounded=true,shadow=true,wd=8cm,center]{postit}
$P$ has order $3\ \Longleftrightarrow\ 3P=\infty\ \Longleftrightarrow\ 2P=-P$
\end{beamercolorbox}}\pause\smallskip

So, if $p>3$ and $E: y^2=x^2+Ax+B$

 \begin{beamercolorbox}[rounded=true,shadow=true,wd=14.4cm,center]{formul}
\hspace*{-3mm}$2P=(x_{2P},y_{2P})=2(x_1,y_1)=({\color[cmyk]{0,1,1,0.5}\lambda^2-2x_1},
{\color[cmyk]{1,0,1,0.5}-\lambda^3+2\lambda x_1-\nu})$ where
$\lambda=\frac{3x_1^2+A}{2y_1}, \nu=-\frac{x_1^3-Ax_1-2B}{2y_1}$
\end{beamercolorbox}\pause\smallskip


\centerline{
 \begin{beamercolorbox}[rounded=true,shadow=true,wd=8cm,center]{postit}
$P$ has order $3\ \Longleftrightarrow\ x_{2P}=\lambda^2-2x_1=x_1$
\end{beamercolorbox}}\pause%\smallskip

Substituting $\lambda$,\pause\
\centerline{
 \begin{beamercolorbox}[rounded=true,shadow=true,wd=6cm,center]{formul}
 $x_{2P}-x_1=\frac{-3x_1^4-6Ax_1^2-12Bx_1+A^2}{4(x_1^3+Ax_1+4B)}=0$
\end{beamercolorbox}}\end{frame}

\begin{frame}\frametitle{Determining points of order $3$}
\begin{Note}[Conclusions]
\begin{itemize}[<+-| alert@+>]
 \item $\psi_3(x):= 3x^4+6Ax^2+12Bx-A^2$ called the $3^{\text{rd}}$ \emph{division} polynomial
 \item $(x_1,y_1)\in E(\F_q)$ has order $3\quad \Rightarrow \psi_3(x_1)=0$
 \item $E(\F_q)$ has at most $8$ points of order $3$
 \item If $p\neq 3$, $E[3]:=\{P\in E(\overline{\F_q}): 3P=\infty\}\cong C_3\oplus C_3$
 \item If $p=3$, $E: y^2=x^3+Ax^2+Bx+C$ and $P=(x_1,y_1)$
has order $3$, then
\begin{enumerate}[<+-| alert@+>]
 \item $Ax_1^3+AC-B^2=0$
 \item $E[3]\cong C_3$ if $A\neq0$ and $E[3]=\{\infty\}$ otherwise
\end{enumerate}
 \end{itemize}
 \end{Note}
\end{frame}

% \begin{frame}\frametitle{Determining points of order $3$ (continues)}
% 
% % \begin{Note} Let $E: y^2=x^3+Ax^2+Bx+C, A,B,C\in\F_{3^n}$ and let $P=(x_1,y_1)\in E(\F_{3^n})$
% % has order $3$, then
% % \begin{enumerate}[<+-| alert@+>]
% %  \item $Ax_1^3+AC-B^2=0$
% %  \item $E[3]\cong C_3$ if $A\neq0$ and $E[3]=\{\infty\}$ otherwise
% % \end{enumerate}
% % \end{Note}\pause
% 
% \begin{example}
% If $E: y^2=x^3+x+1$, then $\#E(\F_5)=9$.\pause
% $$\psi_3(x)=(x + 3)(x + 4)(x^2 + 3x + 4)$$
% Hence
% \centerline{$E[3]=\left\{
% \infty,(2,\pm1),(1,\pm\sqrt{3}),(1\alert{\pm}2\sqrt{3},\pm(1\alert{\pm}\sqrt{3}))\right\}$}\pause
% \begin{enumerate}[<+-| alert@+>]
%  \item $E(\F_5)=\{\infty,(2,\pm1),(0,\pm1),(3,\pm1),(4,\pm2)\}\cong C_9$
%  \item Since $\F_{25}=\F_5[\sqrt{3}]\quad\Rightarrow\quad  E[3]\subset E(\F_{25})$
%  \item $\#E(\F_{25})=27\quad\Rightarrow\quad E(\F_{25})\cong C_3\oplus C_9$
% \end{enumerate}
% 
% 
% \end{example}
% \end{frame}

\begin{frame}\frametitle{Determining points of order $3$ (continues)}

\begin{beamerboxesrounded}[upper=block title example,lower=block body alerted,shadow=true]{FACTS:}
$$E[3]\cong \begin{cases}
C_3\oplus C_3 &\text{if }p\ne3\\
C_3           &\text{if }p=3, E: y^2=x^3+Ax^2+Bx+C, A\neq 0\\
\{\infty\}    &\text{if }p=3, E: y^2=x^3+Bx+C
\end{cases}
$$
\end{beamerboxesrounded}\pause


\begin{block}{Example: inequivalent curves $/\F_7$ with $\#E(\F_7)=9$.}
\begin{tabular}{|l|c|c|c|}
\hline
 $E$ & $\psi_3(x)$ & $E[3]\cap E(\F_7)$ & $\!\!\!E(\F_7)\cong\!\!\!$\\
\hline
 $\!\!y^2=x^3+2\!\!$ & $x(x + 1)(x + 2)(x + 4)$ &{\small$\!\!\!\left\{
\infty,(0,\pm3),(-1,\pm1), (5,\pm1),(3,\pm1)\right\}\!\!$}
& $\!\!\!C_3\oplus C_3\!\!\!$\\
\hline
$\!\!y^2=x^3+3x+2\!\!$ & $\!\!(x + 2)(x^3 + 5x^2 + 3x + 2)\!\!$ & $\{\infty,(5,\pm3)\}$ & $C_9$ \\
\hline
$\!\!y^2=x^3+5x+2\!\!$ & $\!\!(x + 4)(x^3 + 3x^2 + 5x + 2)\!\!$ & $\{\infty,(3,\pm3)\}$ & $C_9$ \\
\hline
$\!\!y^2=x^3+6x+2\!\!$ & $\!\!(x + 1)(x^3 + 6x^2 + 6x + 2)\!\!$ & $\{\infty,(6,\pm3)\}$ & $C_9$ \\
\hline
\end{tabular}
\end{block}%\end{small}
\end{frame}


\end{document}


