\documentclass[landscape]{powersem} %,display
\usepackage{fancybox,marvosym,graphicx,amsmath,amssymb,pifont,textcomp}
\usepackage[bookmarksopen,colorlinks,urlcolor=red]{hyperref} %,pdfpagemode=FullScreen
\usepackage{fixseminar}
\usepackage{color}
\usepackage[latin1]{inputenc}
\usepackage{eurosans}
\usepackage[coloremph,colormath,colorhighlight,lightbackground]{texpower}
\hfuzz=30pt
\vfuzz=30pt
\setlength{\slidewidth}{25cm} \setlength{\slideheight}{17cm}
\slideframe{}%shadow
\def\slideitemsep{.5ex plus .3ex minus .2ex}
\renewcommand{\slidetopmargin}{10mm}
\renewcommand{\slidebottommargin}{15mm}
\renewcommand{\slideleftmargin}{5mm}
\renewcommand{\sliderightmargin}{5mm}
\newcommand{\psd}{\pause}%\addtocounter{slide}{-1}}
\newcommand{\Ccal}{{\mathcal{C}}}
\newcommand{\F}{{\mathbb{F}}}
\newcommand{\C}{{\mathbb C}}
\newcommand{\Q}{{{\mathbb Q}}}
\newcommand{\Z}{{\mathbb Z}}
\newcommand{\N}{{\mathbb N}}
\newcommand{\manorossa}{\textcolor{conceptcolor}{\ding{43}}}
\newcommand{\matitablu}{\textcolor{altemcolor}{\ding{46}}}
\newcommand{\verde}{\textcolor{black}}
\newcommand{\heading}[1]{%
 \begin{center}
  \large\bf
  \Ovalbox{{\textcolor{conceptcolor}{#1}}}%
 \end{center}
 \vspace{1ex minus 1ex}}
\definecolor{verdescu}{rgb}{0,0.6,0.6}
\definecolor{rossoscu}{rgb}{1,0,0.2}

\backgroundstyle[startcolor=white,
                   endcolor=white %inactivecolor,%firstgradprogression=3,
            rightpanelwidth=-7\semcm,,rightpanelcolor=pagecolor]{hgradient}%
%%%%%%%%%%%%% DATI DEL SEMINARIO IN QUESTIONE %%%%%%%%%%%%

\newpagestyle{327}%
 {\textcolor{codecolor}{\textit{Factorisation d'entiers}} \hspace{\fill}\rightmark
\hspace{2mm}\thepage}
 {\includegraphics[width=4mm]{images/dipmat.pdf}\hspace{\fill}\textcolor{codecolor}{\sc Universit\`a Roma Tre}
 \hspace{\fill}\includegraphics[width=5mm]{images/roma3.pdf}}%%
\pagestyle{327} \markright{\textcolor{conceptcolor}{\'ECOLE DE THEORIE DES NOMBRES}}

\begin{document}

%\begin{slide}\pageTransitionWipe{30}
%\maketitle
%\end{slide}

\begin{slide}\pageTransitionWipe{30}
\addtocounter{slide}{-1}
\includegraphics[width=1.3cm]{images/crypto.jpg}\ \hfill \includegraphics[width=1.3cm]{images/crypto.jpg}
\vfil

\begin{sc}\begin{center}
\begin{Large}

\textcolor{underlcolor}{Factorisation d'entiers}
\end{Large}\bigskip

\ {Francesco Pappalardi}\bigskip\bigskip

\begin{large}\begin{bf}Th\'eorie des nombres et algorithmique
\end{bf}\end{large}\medskip

15-26 novembre, Bamako (Mali)\medskip

%\present{\doublebox{\includegraphics[width=4cm]{institute.jpg} \
%%\begin{minipage}[c]{7cm} \vspace*{-1cm}\textsc{National Institute of Advanced Studies}\\
%%\hspace*{3cm}\textbf{\emph{NIAS}}\end{minipage}\
%%\includegraphics[width=1.5cm]{logo.pdf}
%}}\\

\end{center}
\end{sc}
\vfill

\includegraphics[width=1.3cm]{images/crypto.jpg}\ \hfill \includegraphics[width=1.3cm]{images/crypto.jpg}
\end{slide}

\begin{slide}\pageTransitionWipe{30}
\heading{Quelle est la taille des ``grands nombres''}\pause

\parstepwise{\begin{itemize}
 \item[\textcolor{blue}{\ding{43}}]  \step{\textsc{nombre de combinaisons \`a la loterie}:\hspace*{1.4cm}$622.614.630$}
\bigskip\medskip
  \item[\textcolor{blue}{\ding{43}}]  \step{\textsc{nombre de cellules dans un corps humain}:\hspace*{2.4cm}$10^{15}$}
\bigskip\medskip
  \item[\textcolor{blue}{\ding{43}}]  \step{\textsc{nombre d'atomes dans l'univers:}\hspace*{3.685cm}$10^{80}$}
\bigskip\medskip
  \item[\textcolor{blue}{\ding{43}}]  \step{\textsc{nombre de particules subatomiques:}\hspace*{3.04cm}$10^{120}$}
\bigskip\medskip
  \item[\textcolor{blue}{\ding{43}}]  \step{\textsc{nombre d'atomes dans le cerveau humain:}\hspace*{2.5cm}$10^{27}$}
\bigskip\medskip
  \item[\textcolor{blue}{\ding{43}}]  \step{\textsc{nombre d'atomes dans un chat:}\hspace*{3.95cm}$10^{26}$}
\end{itemize}}
\end{slide}

\begin{slide}\pageTransitionWipe{30}

\begin{center}
\begin{small}
\ $RSA_{2048}$ = 25195908475657893494027183240048398571429282126204
032027777137836043662020707595556264018525880784406918290641249
515082189298559149176184502808489120072844992687392807287776735
971418347270261896375014971824691165077613379859095700097330459
748808428401797429100642458691817195118746121515172654632282216
869987549182422433637259085141865462043576798423387184774447920
739934236584823824281198163815010674810451660377306056201619676
256133844143603833904414952634432190114657544454178424020924616
515723350778707749817125772467962926386356373289912154831438167
899885040445364023527381951378636564391212010397122822120720357
\end{small}\end{center}\psd
\bigskip

\centerline{$RSA_{2048}$ est un nombre avec 617 chiffres (d\'ecimaux)}\psd

\bigskip

\heading{\begin{small}\texttt{http://www.rsa.com/rsalabs/challenges/factoring/challengenumbers.txt}\end{small}}
\end{slide}



% ----------------------------------------------------------------

\begin{slide}\pageTransitionWipe{30}

\centerline{$RSA_{2048}$=$p\cdot q$,\ \ \ \   $p,q\approx
10^{308}$}\psd


\heading{{\bf PROBLEME:} {\ it Calculer $p$ et $q$}}\psd

\centerline{\textcolor{black}{\textsc{Prix:}}
 200.000 US\$ ($\sim$ 94.580.000 XOF)!!}
\bigskip\psd

\begin{center}
\begin{tabular}{|c|}
\hline \textbf{\textcolor{red}{Th\'eor\`eme.}} Si $a \in{\mathbb N}$, il ya $p_1<p_2<\cdots<p_k$\ \textit{premier} unique\\
$ \ \ \textrm{telle que} \ \ a=p_1^{\alpha_1}\cdots
p_k^{\alpha_k}$\\\hline\end{tabular}
\end{center}\psd
\bigskip

\textbf{Malheureusement:} RSAlabs estime que l'affacturage en un an nous avons besoin: \center{\begin{tabular}{|c|c|c|}\hline
nombre & ordinateurs & m\'emoire\\
$RSA_{1620}$ & $1.6\times10^{15}$&  $120$ Tb\\
$RSA_{1024}$ & $342,000,000$ &  $170$ Gb\\
$RSA_{760}$  & 215,000  & $4$Gb.\\ \hline
\end{tabular}}
\end{slide}



\begin{slide}\pageTransitionWipe{30}


\heading{\begin{small}\texttt{http://www.rsa.com/rsalabs/challenges/factoring/challengenumbers.txt}\end{small}}\psd


\begin{center}
\begin{tabular}{|c|c|}\hline
 \textcolor{blue}{Nombre} & \textcolor{blue}{Prix (\$US)}  \\
$RSA_{576}$ &  \$10,000   \\
$RSA_{640}$ &  \$20,000    \\
$RSA_{704}$ &     \$30,000 \\
$RSA_{768}$ &     \$50,000 \\
$RSA_{896}$ &     \$75,000 \\
$RSA_{1024}$ &     \$100,000 \\
$RSA_{1536}$ &  \$150,000 \\
$RSA_{2048}$ &     \$200,000   \\
\hline
\end{tabular}
\end{center}
\end{slide}

\begin{slide}\pageTransitionWipe{30}
\addtocounter{slide}{-1}

\heading{\begin{small}\texttt{http://www.rsa.com/rsalabs/challenges/factoring/challengenumbers.txt}\end{small}}


\begin{center}
\begin{tabular}{|c|c|c|}\hline
 \textcolor{blue}{Nombre} & \textcolor{blue}{Prix (\$US)} & \textcolor{blue}{Etat} \\
$RSA_{576}$ &  \$10,000  & Factoriz\'e D\'ecembre 2003\\
$RSA_{640}$ &  \$20,000   & Factoriz\'e Novembre 2005\\
$RSA_{704}$ &     \$30,000&  pas factoriz\'e\\
$RSA_{768}$ &     \$50,000&   pas factoriz\'e\\
$RSA_{896}$ &     \$75,000& pas factoriz\'e\\
$RSA_{1024}$ &     \$100,000&  pas factoriz\'e\\
$RSA_{1536}$ &  \$150,000 & pas factoriz\'e\\
$RSA_{2048}$ &     \$200,000 &  pas factoriz\'e\\
\hline
\end{tabular}
\end{center}
\end{slide}

\begin{slide}\pageTransitionWipe{30}
\heading{C\'el\`ebre citation!!!}\bigskip

\centerline{\includegraphics[width=4cm]{images/borel1.jpg}}

\textit{Un ph\'enom\`ene dont la probabilit\'e est $10^{-50}$ ne se produira jamais, et
moins sera jamais observ\'e.}\bigskip

\textsc{- \'Emil Borel (La probabilit\'es et sa vie)}
\end{slide}

\begin{slide}
\centerline{\includegraphics[width=11cm]{images/School_of_Athens.jpeg}}\vspace*{-8cm}
\parstepwise{
\step{
      \hspace*{-1mm}\colorbox{white}{\shadowbox{L'\'Ecole d'Ath\`enes (Raffaello Sanzio)}}
      }
\bigskip\bigskip\bigskip\bigskip\\ %\vspace*{1cm}\hspace*{2cm}
\step{\hspace*{3cm}
      \begin{minipage}{5cm}
                      \shadowbox{\includegraphics[width=3.5cm]{images/Euclid_7.jpeg}}\\
                      \colorbox{white}{\begin{minipage}[c]{4cm}
                               \ \ \ Euclide d'Alexandrie\vspace*{-2mm}\\
                                 {\tiny Naissance: 325 avant JC (Approx.)}\vspace*{-2.5mm}\\
                                 {\tiny D\'ec\`es: 265 avant JC (Approx.)}
                                \end{minipage}}\\
       \end{minipage}
       }\\ \bigskip\medskip% \bigskip\\
\step{\hspace*{1.5cm}
      \colorbox{white}{\shadowbox{Il ya une infinit\'e de nombres premier!!}}
      }
            }
\end{slide}

\begin{slide}\pageTransitionWipe{30}
\heading{Etat de ``\emph{l'art de la factorisation}''}
\bigskip\bigskip

\begin{center}
\includegraphics{images/eratostene.jpg}

220AC  (\'Eratosth\`ene de Cyr\`ene)
\end{center}
\end{slide}

\begin{slide}\pageTransitionWipe{30}
\heading{Etat de ``\emph{l'art de la factorisation}''}
\bigskip\bigskip

\begin{center}
\includegraphics[width=5cm]{images/Euler_9.jpeg}

1730 Euler $2^{2^5}+1=641\cdot 6700417 $
\end{center}
\end{slide}

\begin{slide}
\heading {Comment avez Euler factoris\'e $2^{2^5}+1$?}\pause 

\noindent\textsc{Proposition} \textit{Supposons quw $p\mid b^n+1$. Il s'ensuit que
\begin{enumerate}
 \item $p\mid b^{d}+1$ pour certains diviseur propre $d$ de $n$ tel que $n/d$ est impair, ou bien
 \item $p\equiv 1\bmod 2n$.
\end{enumerate}}\medskip
\medskip

\noindent\textcolor{red}{\textit{Application.}} Soit $b=2$ et $n=2^5=64$. Alors $2^{2^5}+1$ est soit an nombre premier ou bien
est divisible par un nombre premier $p\equiv1\bmod 128$. 

Notez que \\
$1+1\times128=3\times43$, $1+2\times128=257$ est premier,\\ 
$1+3\times128=5\times7\times11$, $1+4\times 128=3^3\times19$ et $1+5\cdot 128=641$
est premier.

Enfin  
$$\frac{2^{2^5}+1}{641}=\frac{6700417}{641}=6700417$$.

\end{slide}
 

\begin{slide}\pageTransitionWipe{30}
\heading{Etat de ``\emph{l'art de la factorisation}''}
\bigskip\bigskip

\begin{center}
\includegraphics[width=7cm]{images/euler1.jpg}

1730 Euler $2^{2^5}+1=641\cdot 6700417 $
\end{center} 
\end{slide}

\begin{slide}\pageTransitionWipe{30}
\heading{Etat de ``\emph{l'art de la factorisation}''}

\begin{center}
\includegraphics[width=4.5cm]{images/fermat.jpg}
\includegraphics[width=4.5cm]{images/Gauss_1803.jpeg}

1750--1800 Fermat, Gauss (Cribles - Tableaux)
\end{center} 
\end{slide}

\begin{slide}\pageTransitionWipe{30}
\heading{Etat de ``\emph{l'art de la factorisation}''}

\begin{center}
\includegraphics[width=5cm]{images/fermatstamp.jpg}
\includegraphics[width=5cm]{images/Gauss_banknote.jpeg}

1750--1800 Fermat, Gauss (Cribles - Tableaux)\psd

Premier algorithme de factorisation par crible
$N=x^2-y^2=(x-y)(x+y)$
\end{center} 
\end{slide}

\begin{slide}\pageTransitionWipe{30}
\heading{Etat de ``\emph{l'art de la factorisation}''}
\begin{itemize}
\item[\textcolor{black}{\ding{243}}] 220AC  (\'Eratosth\`ene de Cyr\`ene)
\item[\textcolor{black}{\ding{243}}] 1730 Euler $2^{2^5}+1=641\cdot 6700417 $
\item[\textcolor{black}{\ding{243}}] 1750--1800 Fermat, Gauss (Cribles - Tableaux)\psd
\item[\textcolor{black}{\ding{243}}] 1880 Landry \& Le Lasseur:
$$2^{2^6}+1= 274177 \times 67280421310721$$\psd
\vspace{-3mm}\item[\textcolor{black}{\ding{243}}] 1919 Pierre et Eug\`ene Carissan (Machine pour Factoriser)
\end{itemize}
\end{slide}

\begin{slide}\pageTransitionWipe{30}
\heading{Ancien Machine pour factoriser dei Carissan}\psd


\begin{figure}
  \centering
\includegraphics[width=5cm]{images/cari.jpg}
  \caption{Conservatoire Nationale des Arts et M\'etiers in Paris}
\end{figure}\psd

\heading{\begin{small}\texttt{http://www.cs.uwaterloo.ca/\~{}shallit/Papers/carissan.html}\end{small}}

\end{slide}

\begin{slide}\pageTransitionWipe{30}
\begin{figure}
  \centering \includegraphics[width=3cm]{images/caris2.jpg}
  \caption{Lieutenant Eug\`ene Carissan}\end{figure}\psd
\centerline{\begin{tabular}{rcl}
$225058681=229\times982789$ & &{2 minutes}\\
$3450315521=1409\times 2418769$ & & {3 minutes}\\
$3570537526921=841249\times4244329$ & & {18 minutes}\\
\end{tabular}}

\end{slide}

\begin{slide}\pageTransitionWipe{30}
\heading{Etat de ``\emph{l'art de la factorisation}''}
\begin{itemize}
\item[\textcolor{black}{\ding{243}}] 220AC  (\'Eratosth\`ene de Cyr\`ene)
\item[\textcolor{black}{\ding{243}}] 1730 Euler $2^{2^5}+1=641\cdot 6700417 $
\item[\textcolor{black}{\ding{243}}] 1750--1800 Fermat, Gauss (Cribles - Tebleaux)
\item[\textcolor{black}{\ding{243}}] 1880 Landry \& Le Lasseur:
$$2^{2^6}+1= 274177 \times 67280421310721$$
\vspace{-3mm}\item[\textcolor{black}{\ding{243}}] 1919 Pierre et Eug\`ene Carissan (Machine pour Factoriser)\psd
\item[\textcolor{black}{\ding{243}}] 1970 Morrison \& Brillhart
$$2^{2^7}+1=
59649589127497217 \times 5704689200685129054721 $$
\end{itemize}
\end{slide}


\begin{slide}\pageTransitionWipe{30}
\heading{Etat de ``\emph{l'art de la factorisation}''}
\bigskip

\begin{center}
\includegraphics[width=5cm]{images/brillhart.JPG}

1970 - John Brillhart \&  Michael A. Morrison  $2^{2^7}+1=
59649589127497217 \times 5704689200685129054721 $
\end{center}
\end{slide}

\begin{slide}\pageTransitionWipe{30}
\heading{Etat de ``\emph{l'art de la factorisation}''}
\bigskip\bigskip

\begin{center}
\includegraphics[width=3cm]{images/smallcarl1.jpg}

1982 - Carl Pomerance - Le Crible Quadratique
\end{center}
\end{slide}


\begin{slide}\pageTransitionWipe{30}
\heading{Etat de ``\emph{l'art de la factorisation}''}
\begin{itemize}
\item[\textcolor{black}{\ding{243}}] 220AC  (\'Eratosth\`ene de Cyr\`ene)
\item[\textcolor{black}{\ding{243}}] 1730 Euler $2^{2^5}+1=641\cdot 6700417 $
\item[\textcolor{black}{\ding{243}}] 1750--1800 Fermat, Gauss (Cribles - Tebleaux)
\item[\textcolor{black}{\ding{243}}] 1880 Landry \& Le Lasseur:
\centerline{$2^{2^6}+1= 274177 \times 67280421310721$}
\item[\textcolor{black}{\ding{243}}] 1919 Pierre et Eug\`ene Carissan (Machine pour Factoriser)
\item[\textcolor{black}{\ding{243}}] 1970 Morrison \& Brillhart
\centerline{$2^{2^7}+1=
59649589127497217 \times 5704689200685129054721 $}
\item[\textcolor{black}{\ding{243}}] 1982 Crible Quadratique \textbf{QS}
(Pomerance)\hfil $\rightsquigarrow$\hfil Crible del sur corps num\`erique \textbf{NFS}\psd
\item[\textcolor{black}{\ding{243}}] 1987 Factorisation avec  Courbes Elliptiques \textbf{ECF} (Lenstra)
\end{itemize}
\end{slide}

\begin{slide}\pageTransitionWipe{30}
\heading{Etat de ``\emph{l'art de la factorisation}''}
\bigskip\bigskip

\begin{center}
\includegraphics[width=4cm]{images/Lenstra.jpg}

1987 - Hendrik Lenstra - Factorisation avec  courbes elliptiques
\end{center}
\end{slide}



\begin{slide}\pageTransitionWipe{30}
\heading{Factorisation Contemporanea}\vspace{-3mm}\psd

\begin{itemize}
  \item[\textcolor{blue}{\ding{182}}] 1994, Crible Quadratique (QS): (8 mois, 600 volontaires, 20 Nations)\\
  D.Atkins, M. Graff, A. Lenstra, P. Leyland
\begin{tiny}\begin{tabular}{l}
$  RSA_{129} = 114381625757888867669235779976146612010218296721242362562561842935706$\\
\hspace*{5mm}$935245733897830597123563958705058989075147599290026879543541=$\\
$        = 3490529510847650949147849619903898133417764638493387843990820577 \times$\\
$
32769132993266709549961988190834461413177642967992942539798288533
$\end{tabular}\end{tiny}\psd

  \item[\textcolor{blue}{\ding{183}}] (2 F\'evrier 1999), Crible sur corps num\`erique (NFS): (160 Sun, 4 mois)
 \begin{tiny}\begin{tabular}{c}
\hspace*{-1cm}$ RSA_{155} = 109417386415705274218097073220403576120037329454492059909138421314763499842$\\
$88934784717997257891267332497625752899781833797076537244027146743531593354333897=$\\
$=102639592829741105772054196573991675900716567808038066803341933521790711307779
\times$\\
$106603488380168454820927220360012878679207958575989291522270608237193062808643
$\end{tabular}\end{tiny}\psd

  \item[\textcolor{blue}{\ding{184}}] (3 D\'ecembre, 2003) (NFS): J. Franke et al. (174 chiffres d\'ecimal)
 \begin{tiny}\begin{tabular}{c}
\hspace*{-1cm}$ RSA_{576} = 1881988129206079638386972394616504398071635633794173827007633564229888597152346$\\
$65485319060606504743045317388011303396716199692321205734031879550656996221305168759307650257059=$\\
$=398075086424064937397125500550386491199064362342526708406385189575946388957261768583317\times$\\
$472772146107435302536223071973048224632914695302097116459852171130520711256363590397527
$\end{tabular}\end{tiny}\psd

\item[\textcolor{blue}{\ding{185}}]
  Factorisation avec  courbes elliptiques: mis en place par  H. Lenstra.
  convenient pour trouver des petits factors (50 chiffres)\psd

\end{itemize}

\centerline{\textcolor{red}{Tous: "\emph{complexit\'e sous--exponentielle}"}}
\end{slide}

\begin{slide}\pageTransitionWipe{30}
\heading{La factorisation de $RSA_{200}$}

\begin{tiny}

$RSA_{200}=2799783391122132787082946763872260162107044678695542853756000992932612840010$
          $7609345671052955360856061822351910951365788637105954482006576775098580557613$
          $579098734950144178863178946295187237869221823983$\psd


Date: Mon, 9 May 2005 18:05:10 +0200 (CEST) 
From: "Thorsten Kleinjung"
Subject: rsa200 

We have factored RSA200 by GNFS. The factors are

35324619344027701212726049781984643686711974001976\
25023649303468776121253679423200058547956528088349

and

79258699544783330333470858414800596877379758573642\
19960734330341455767872818152135381409304740185467


We did lattice sieving for most special q between 3e8 and 11e8
using mainly factor base bounds of 3e8 on the algebraic side and 18e7 
on
the rational side. The bounds for large primes were $2^{35}$. This produced
26e8 relations. Together with 5e7 relations from line sieving the total
yield was 27e8 relations. After removing duplicates 226e7 relations
remained. A filter job produced a matrix with 64e6 rows and columns,
having 11e9 non-zero entries. This was solved by Block-Wiedemann.

Sieving has been done on a variety of machines. We estimate that
lattice sieving would have taken 55 years on a single 2.2 GHz Opteron 
CPU.
Note that this number could have been improved if instead of the PIII-
binary which we used for sieving, we had used a version of the
lattice-siever optimized for Opteron CPU's which we developed in the 
meantime.
The matrix step was performed on a cluster of 80 2.2 GHz Opterons 
connected via a Gigabit network and took about 3 months.

We started sieving shortly before Christmas 2003 and continued until
October 2004. The matrix step began in December 2004.
Line sieving was done by P. Montgomery and H. te Riele at the CWI, by
F. Bahr and his family.

More details will be given later.

F. Bahr, M. Boehm, J. Franke, T. Kleinjung
\end{tiny}
\end{slide}

\begin{slide}\pageTransitionWipe{30}

\centerline{\Large{\textcolor{blue}{RSA}}}

\centerline{\includegraphics[width=9cm]{images/rsa-photo.jpg}}

\centerline{Adi Shamir, Ron L. Rivest, Leonard Adleman (1978)}
\end{slide}

\begin{slide}

\centerline{\Large{\textcolor{blue}{RSA}}}

\centerline{\includegraphics[width=9cm]{images/RSA-2003.jpg}}

\centerline{Ron L. Rivest, Adi Shamir, Leonard Adleman (2003)}
\end{slide}

\begin{slide}

\heading{\textcolor{red}{\textsc{Probl\`eme:} \'Etant donn\'e $n\in\N$, trouver un diviseur propre de $n$}}\pause
\parstepwise{\begin{itemize}
 \item \step{Un probl\`eme tr\`es ancien et tres difficile;}
\item \step{Trial division requires $O(\sqrt n)$ division which is an exponential time}\\ \step{(i.e. impractical)}
\item \step{Plusieurs algorithmes diff\'erents}
\item \step{nous passons en revue la m\'ethode \'el\'egante de Pollard (m\'etode $\rho$).}
\end{itemize}}
{Suppose $n$ is not a power and consider the function:}\pause
\centerline{$f:\Z/n\Z\longrightarrow\Z/n\Z,\quad x\mapsto f(x)=x^2+1.$}

The $k$-th iterate of $f$ is $f^k(x)=f^{k-1}(f(x))$ with $f^1(x)=f(x)$.

 If $x_0\in\Z/n\Z$ is chosen ``sufficiently
randomly'',  the sequence $\{f^{k}(x_0)\}$ behaves as a random sequence of elements of $\Z/n\Z$ and we exploit
this fact.

\end{slide}


\begin{slide}
\heading{Pollard $\rho$ factoring method}
\begin{center}\fbox{\textcolor{black}{
\begin{minipage}[c]{11cm}
\texttt{\noindent 
\noindent 
\textcolor{red}{Input:}  $n\in\N$ odd and not a perfect power (to be factored)\\
\textcolor{blue}{Output:}  a non trivial factor of $n$\\
1. Choose at random $x\in\Z/n\Z=\{0,1,\ldots,n-1\}$\\
2. For $i=1,2\ldots$.\\
\hspace*{1cm} \qquad $g:=\gcd(f^i(x)-f^{2i}(x),n)$\\
\hspace*{1cm} \qquad If $g=1$, goto next $i$\\  
\hspace*{1cm} \qquad If $1<g<n$ then output $g$ and halt\\
\hspace*{1cm} \qquad If $g=n$ then go to Step 1 and choose another $x$.}
\end{minipage}}}
\end{center}
\pause\vspace*{-3mm}
What is going on here?\pause\vspace*{-3mm}
Is is obviously a probabilistic algorithm but it is not even clear that it will ever terminate.

But in fact it terminates with complexity $O(\sqrt[4]n)$ which is attained in the worst case (i.e.
when $n$ is an RSA module (for RSA see course in Cryptography by K. Chakraborty).
\end{slide}

\begin{slide}
\heading{\textsc{The birthday paradox}}

\noindent\textbf{Elementary Probability Question:} \textit{what is the chance that in a sequence of $k$ elements (where
repetitions are allowed) from a set of $n$ elements, there is a repetition?} \pause

\noindent\textit{Answer:}  The chance is $\displaystyle{1-\frac{n!}{n^k(n-k)!}\approx 1-e^{-k(k-1)/2n}}$

\fbox{In a party of $23$ friends there $50.04\%$ chances that $2$ have the same birthday!!}

Relevance to the $\rho$-Factoring method:
\begin{center}
\fbox{\begin{minipage}[l]{12cm}If $d$ is a divisor of $n$, then in $O(\sqrt{d})=O(\sqrt[4]{n})$ steps there is a high chance that in the sequence 
$\{f^{k}(x_0)\bmod d\}$ there is a repetition modulo $d$.\end{minipage}}
\end{center}

\noindent\textsc{Remark (WHY $\rho$).} If $y_1,\ldots,y_m,y_{m+1},\ldots,y_{m+k}=y_m,y_{m+k+1}=y_{m+1},\ldots$. and $i$ is the smallest
multiple of $k$ with $i\ge m$, then $y_i=y_{2i}$ (the Floyd's cycle trick). 

\end{slide}




\begin{slide}
\heading{\textcolor{red}{R\'ef\'erences pour ce cours}}
\vspace*{-2mm}\small{\begin{itemize}
 \item[\mbox{[1]}] J. Buhler \& S. Wagon \textit{Basic algorithms in number theory} Algorithmic Number Theory,
MSRI Publications
Volume \textbf{44}, 2008
{\small \url{http://www.msri.org/communications/books/Book44/files/02buhler.pdf}
\href{http://www.msri.org/communications/books/Book44/files/02buhler.pdf}{}}
\item[\mbox{[2]}] C. Pomerance \textit{Smooth numbers and the quadratic sieve}
Algorithmic Number Theory,
MSRI Publications
Volume \textbf{44}, 2008
{\small \url{http://www.msri.org/communications/books/Book44/files/03carl.pdf}
\href{http://www.msri.org/communications/books/Book44/files/03carl.pdf}{}}
\item[\mbox{[3]}] R. Crandall and C. Pomerance, \textit{Prime numbers}, 2nd
ed., Springer-Verlag, New York, 2005.
\item[\mbox{[4]}] E. Bach and J. Shallit, \textit{Algorithmic number theory, I: Efficient
algorithms}, MIT Press, Cambridge, MA, 1996.
\item[\mbox{[5]}] J. von zur Gathen and J. Gerhard, \textit{Modern computer
algebra}, 2nd ed., Cambridge University Press, Cambridge, 2003.
\item[\mbox{[6]}] V. Shoup, \textit{A computational introduction to number theory and algebra,}
Cambridge University Press, Cambridge, 2005.
\item[\mbox{[7]}] 
These notes \vspace*{-2mm} 
{\small \url{http://www.mat.uniroma3.it/users/pappa/bamako2010_A.pdf}
\href{http://www.mat.uniroma3.it/users/pappa/bamako2010_A.pdf}{}}
\end{itemize}}
\end{slide}


\end{document}
