\documentclass[10pt,handout]{beamer} %,hyperref={pdfpagelabels=false},draft,handout,handout
\usepackage[orientation=landscape,size=custom,width=16,height=9,scale=0.30,debug]{beamerposter} 
\usepackage[english]{babel}
\usepackage{lmodern}% http://ctan.org/pkg/lm
\usepackage[latin1]{inputenc}
\usepackage{times,hyperref,tikz,colortbl,yfonts,translator}
\usepackage[T1]{fontenc}
 \newcommand{\Q}{\mathbb Q}
 \newcommand{\Z}{\mathbb Z}
 \newcommand{\N}{\mathbb N}
 \newcommand{\F}{\mathbb F}
 \newcommand{\C}{\mathbb C}
 \newcommand{\R}{\mathbb R}
\useoutertheme[height=0pt,width=2cm,right]{sidebar}
\usecolortheme{rose,sidebartab}
\useinnertheme{circles}
\usefonttheme[only large]{structurebold}
\theoremstyle{definition}
\newtheorem{exercise}[theorem]{\translate{Exercise}}
\newtheorem{Note}[theorem]{\translate{Note}}
\lecture[4]{Elliptic curves over finite fields}{First Lecture}
\title[Elliptic curves over $\F_{q}$]{\insertlecture}
\setbeamercolor{formul}{fg=black,bg=pink}
\setbeamercolor{sidebar right}{bg=green!15}
\setbeamercolor{structure}{fg=black!120}
\setbeamercolor{postit}{fg=black,bg=yellow}
\setbeamercolor{greys}{fg=black,bg==black!25}
\setbeamerfont{title in sidebar}{series=\bfseries}
\setbeamerfont*{item}{series=}
\setbeamerfont{frametitle}{size=}
\setbeamerfont{block title}{size=\small}
\setbeamerfont{subtitle}{size=\normalsize,series=\normalfont}
\begin{document}

\begin{frame}
\includegraphics[width=1.6cm]{images/roma3.pdf}\hfill\includegraphics[width=1.9cm]{images/NUM2.jpeg}
\vfill

\begin{center}\begin{sc}
\begin{Large}

\textcolor{red}{Elliptic curves Cryptography}
\end{Large}\bigskip

\ {Francesco Pappalardi}\bigskip\bigskip

\begin{large}\begin{bf}\#1 - First Lecture.
\end{bf}\end{large}\medskip

September $14^{\text{th}}$ 2015\medskip
\vfill
\end{sc}\end{center}

%\includegraphics[width=1.6cm]{images/cimpalogo.pdf}\hfill
\begin{minipage}[b]{9.3cm}
\textbf{National University of Mongolia}\\  %Монгол Улсын Их Сургууль
Ulan Baatar, Mongolia\\
September 14, 2015
\end{minipage}\hfill
%\includegraphics[width=1.9cm]{images/seams.png}
\end{frame}

\section{Introduction}

\begin{frame}
 \frametitle{Three Lectures on Elliptic Curves Cryptography}

 \begin{Note}[Program of the Lectures]
  \begin{enumerate}[<+-| alert@+>]
   \item Generalities on Elliptic Curves over finite Fields
   \item Basic facts on Discrete Logarithms on finite groups, generic attacks (Pohlig--Hellmann, BSGS, Index Calculus)
   \item Elliptic curves Cryptography: pairing based Cryptography, MOV attacks, anomalous curves
  \end{enumerate}
 \end{Note} 
\end{frame}


\subsection{Fields}

\begin{frame}
 \frametitle{Notations}

\begin{alertblock}{Fields of characteristics 0}
 \begin{enumerate}[<+-| alert@+>]
 \item $\Q$ is the field of rational numbers
\item $\R$ and $\C$ are the fields of real and complex numbers
\item $K\subset\C$, $\dim_\Q K<\infty$ is a \emph{number field}
\begin{itemize}
\item $\Q[\sqrt{d}]$, $d\in\Q$
\item $\Q[\alpha]$, $f(\alpha)=0$, $f\in\Q[X]$
irreducible
\end{itemize}
\end{enumerate}
\end{alertblock}

\begin{exampleblock}{Finite fields}
 \begin{enumerate}[<+-| alert@+>]
 \item $\F_p=\{0,1,\ldots,p-1\}$ is the prime field;
 \item $\F_q$ is a finite field with $q=p^n$ elements
 \item $\F_q=\F_p[\xi]$, $f(\xi)=0$, $f\in\F_p[X]$
irreducible, $\partial f=n$
\item $\F_4=\F_2[\xi]$, $\xi^2=1+\xi$
\item $\F_8=\F_2[\alpha]$, $\alpha^3=\alpha+1$ but also $\F_8=\F_2[\beta]$, $\beta^3=\beta^2+1$, ($\beta=\alpha^2+1$)
\item $\F_{101^{101}}=\F_{101}[\omega], \omega^{101}=\omega+1$
\end{enumerate}
\end{exampleblock}

\end{frame}

\begin{frame}[label=current]
\frametitle{Notations}
%
 \begin{block}{Algebraic Closure of $\F_q$}\pause
 \begin{itemize}[<+-| alert@+>] % [<ballot@+-| visible@1-,+(1)>]
  \item $\C\supset\Q$ satisfies that \emph{Fundamental Theorem of Algebra}! (i.e. $\forall f\in\Q[x], \partial f>1, \exists\alpha\in\C,
 f(\alpha)=0)$
  \item We need a field that plays the role, for $\F_q$, that $\C$ plays for $\Q$. It will be $\overline{\F_q}$, called
\emph{algebraic closure of $\F_q$}
 \item[] \ \hfil 
 \begin{beamercolorbox}[rounded=true,shadow=true,wd=6.5cm]{postit}
         \begin{enumerate}
          \item $\forall n\in\N$, we fix an $\F_{q^n}$
          \item We also require that $\F_{q^n}\subseteq\F_{q^m}$ if $n\mid m$
          \item We let $\overline{\F_q}=\displaystyle\bigcup_{n\in\N}\F_{q^n}$
         \end{enumerate}
\end{beamercolorbox}
 \item
  \textbf{Fact:} $\overline{\F_q}$ is \emph{algebraically closed}\\ (i.e. $\forall f\in\F_q[x], \partial f>1, \exists\alpha\in\overline{\F_q},
 f(\alpha)=0)$
 \end{itemize}
 \end{block}

%\begin{scriptsize}
If $F(x,y)\in\Q[x,y]$ a \emph{point of the curve $F=0$}, means $(x_0,y_0)\in\C^2$ s.t.
$F(x_0,y_0)=0$. \pause

If $F(x,y)\in\F_q[x,y]$ a \emph{point of the curve $F=0$}, means $(x_0,y_0)\in\overline{\F_q}^2$ s.t.
$F(x_0,y_0)=0$.
%\end{scriptsize}
\end{frame}


\section{Weierstra\ss\ Equations}

\begin{frame}{The (general) Weierstra\ss\ Equation}

An elliptic curve $E$ over a $\F_q$ (finite field) is given by an equation
\centerline{\begin{beamercolorbox}[shadow=true,center,rounded=true,wd=6cm]{formul}
$E: y^2+a_1xy+a_3y=x^3+a_2x^2+a_4x+a_6$\end{beamercolorbox}}
where $a_1, a_3, a_2, a_4 ,a_6\in\F_q$ \pause

\begin{center}
 \includegraphics[width=60mm]{images/elliptic1.pdf}\pause
\llap{\includegraphics[width=60mm]{images/elliptic2.pdf}}\pause
\llap{\includegraphics[width=60mm]{images/elliptic3.pdf}}\pause
\llap{\includegraphics[width=60mm]{images/elliptic3b.pdf}}\pause
\llap{\includegraphics[width=60mm]{images/elliptic4.pdf}}\pause
\llap{\includegraphics[width=60mm]{images/elliptic5.pdf}}\pause
\llap{\includegraphics[width=60mm]{images/elliptic6.pdf}}\pause
\llap{\includegraphics[width=60mm]{images/elliptic7.pdf}}\pause
\llap{\includegraphics[width=60mm]{images/elliptic8.pdf}}\pause
\llap{\includegraphics[width=60mm]{images/elliptic9.pdf}}\pause
\llap{\includegraphics[width=60mm]{images/elliptic9b.pdf}}\pause
\llap{\includegraphics[width=60mm]{images/elliptic10.pdf}}\pause
\llap{\includegraphics[width=60mm]{images/elliptic10b.pdf}}\pause
\llap{\includegraphics[width=60mm]{images/elliptic6.pdf}}\pause
\end{center}

 \begin{beamercolorbox}[sep=1em,wd=5.5cm]{postit}
 The equation should not be \emph{singular}
 \end{beamercolorbox}
\end{frame}

\subsection{The Discriminant}

\begin{frame}
\frametitle{The Discriminant of an Equation}
\framesubtitle{The condition of absence of singular points in terms of $a_1, a_2, a_3, a_4, a_6$}
\pause

\begin{Definition} The \emph{discriminant} of a Weierstra\ss\ equation over $\F_q$,  $q=p^n$, $p\ge3$ is
\centerline{\begin{beamercolorbox}[shadow=true,center,rounded=true,wd=8.5cm]{formul}
\begin{align*}
\Delta_E&:=\frac{1}{2^4}\left(-a_1^5 a_3 a_4 - 8 a_1^3 a_2 a_3 a_4 - 16 a_1 a_2^2 a_3 a_4 + 36 a_1^2 a_3^2 a_4 \right. \\
  &-a_1^4 a_4^2 - 8 a_1^2 a_2 a_4^2 - 16 a_2^2 a_4^2 + 96 a_1 a_3 a_4^2 +64 a_4^3 + \\
  & a_1^6 a_6 + 12 a_1^4 a_2 a_6 + 48 a_1^2 a_2^2 a_6 + 64 a_2^3 a_6 -36 a_1^3 a_3 a_6\\
  &\left. - 144 a_1 a_2 a_3 a_6 - 72 a_1^2 a_4 a_6 - 288 a_2 a_4 a_6 +
  432 a_6^2  \right)
 \end{align*}
\end{beamercolorbox}}\pause
\end{Definition}

\begin{Note}
 $E$ is \emph{non singular} if and only if $\Delta_E\ne0$
\end{Note}

\end{frame}




\begin{frame}
\frametitle{Special Weierstra\ss\ equation of $E/\F_{p^\alpha}, p\neq2$}
\centerline{\begin{beamercolorbox}[shadow=true,center,rounded=true,wd=8cm]{formul}
$E: y^2+a_1xy+a_3y=x^3+a_2x^2+a_4x+a_6\quad a_i\in\F_{p^\alpha}$\end{beamercolorbox}}\pause

If we ``complete the squares`` by applying the transformation:%\begin{scriptsize}
\centerline{\begin{beamercolorbox}[shadow=true,center,rounded=true,wd=3.7cm]{postit}
        $\begin{cases}
  x\leftarrow x \\ y\leftarrow y -\frac{a_1x+a_3}2
 \end{cases}$            \end{beamercolorbox}}%\end{scriptsize}
 \pause

 the Weierstra\ss\ equation becomes:
\centerline{\begin{beamercolorbox}[shadow=true,center,rounded=true,wd=5.1cm]{formul}
$E': y^2=x^3+a'_2x^2+a'_4x+a'_6$
            \end{beamercolorbox}}
where $a'_2=a_2+\frac{a_1^2}4, a'_4= a_4+\frac{a_1a_3}2, a'_6= a_6+\frac{a_3^2}4$\pause

If $p\ge5$, we can also apply the transformation\\ \pause
\centerline{%\begin{scriptsize}
\begin{beamercolorbox}[shadow=true,center,rounded=true,wd=2cm]{postit}
$\begin{cases}
  x\leftarrow x-\frac{a'_2}{3} \\ y\leftarrow y
 \end{cases}$\end{beamercolorbox}
%\end{scriptsize}
} obtaining the equations:\pause

\centerline{\begin{beamercolorbox}[shadow=true,center,rounded=true,wd=4.2cm]{formul}
$E'': y^2=x^3+a''_4x+a''_6$
            \end{beamercolorbox}}
\hfil\ \hspace*{-1.2cm} where $a''_4=a'_4-\frac{{a'_2}^2}3, a''_6= a'_6+\frac{2{a'_2}^3}{27}-\frac{a'_2a'_4}3$
\end{frame}

% \begin{frame}
% \frametitle{Special Weierstra\ss\ equation for $E/\F_{2^\alpha}$}
% \framesubtitle{Case $a_1\neq0$}
% \centerline{\begin{beamercolorbox}[shadow=true,left,rounded=true,wd=9cm]{formul}
% $E: y^2+a_1xy+a_3y=x^3+a_2x^2+a_4x+a_6\qquad a_i\in\F_{2^\alpha}$\hfill\\
% \ \hfill $\Delta_E:=\frac{a_1^6 a_6+a_1^5 a_3 a_4+a_1^4 a_2 a_3^2+a_1^4 a_4^2+a_1^3 a_3^3+a_3^4}{a_1^6}$
% \end{beamercolorbox}}\pause
% 
% If we apply the affine transformation:\begin{scriptsize}
% \centerline{\begin{beamercolorbox}[shadow=true,left,rounded=true,wd=3.9cm]{postit}
%         $\begin{cases}
% x\longleftarrow a_1^2x+a_3/a_1\\\
% y\longleftarrow a_1^3y+(a_1^2a_4+a_3^2)/a_1^2
%   \end{cases}$\end{beamercolorbox}}\end{scriptsize}\pause
% 
% we obtain
% 
% \centerline{\begin{beamercolorbox}[shadow=true,center,rounded=true,wd=7cm]{formul}
% $E': y^2+xy=x^3+\left(\frac{a_2}{a_1^2}+\frac{a_3}{a_1^3}\right)x^2+\frac{\Delta_E}{a_1^6}$\hfill \\
% \ \hfill Surprisingly $\Delta_{E'}=\Delta_E/a_1^6$
% \end{beamercolorbox}}\pause\bigskip
% 
% With \texttt{Mathematica}
% 
% \begin{scriptsize}
% \ \hfill{\begin{beamercolorbox}[shadow=true,left,rounded=true,wd=8cm]{postit}
%       \texttt{El:=a6+a4x+a2x\^{ }2+x\^{ }3+a3y+a1xy+y\^{ }2;\\
% Simplify[PolynomialMod[ReplaceAll[El, \\
% \ \hfill \{x->a1\^{ }2 x+a3/a1, y->a1\^{ }3y+(a1\^{ }2a4+a3\^{ }2)/a1\^{ }3\}],2]]}
% \end{beamercolorbox}}\end{scriptsize}
% \end{frame}
% 
% 
\begin{frame}
% \frametitle{Special Weierstra\ss\ equation for $E/\F_{2^\alpha}$}
% \framesubtitle{Case $a_1=0$ and $\Delta_E:=a_3\neq0$}
% \centerline{\begin{beamercolorbox}[shadow=true,center,rounded=true,wd=8cm]{formul}
% $E: y^2+a_1xy+a_3y=x^3+a_2x^2+a_4x+a_6\qquad a_i\in\F_{2^\alpha}$\\
% \end{beamercolorbox}}\pause
% 
% If we apply the affine transformation:\begin{scriptsize}
% \centerline{\begin{beamercolorbox}[shadow=true,center,rounded=true,wd=2.7cm]{postit}
%         $\begin{cases}
% x\longleftarrow x+a_2\\
% y\longleftarrow y
%   \end{cases}$\end{beamercolorbox}}\end{scriptsize}\pause
% 
% we obtain
% 
% \centerline{\begin{beamercolorbox}[shadow=true,center,rounded=true,wd=7cm]{formul}
% $E: y^2+a_3y=x^3+(a_4+a_2^2)x+(a_6+a_2a_4)$
% \end{beamercolorbox}}\pause\medskip
% 
% With \texttt{Mathematica}
% 
% \begin{scriptsize}
% \ \hfill{\begin{beamercolorbox}[shadow=true,left,rounded=true,wd=8.6cm]{postit}
%       \texttt{El:=a6+a4x+a2x\^{ }2+x\^{ }3+a3y+y\^{ }2;
%  Simplify[PolynomialMod[ReplaceAll[El,\{x->x+a2,y->y\}],2]]}
% \end{beamercolorbox}}\end{scriptsize}\pause
% 
%\begin{small}
\begin{Definition}
 Two Weierstra\ss\ equations over $\F_q$ are said (affinely) equivalent if there exists a (affine) change of variables that takes one
into the other
\end{Definition}
%\end{small}
\pause


\begin{Note}{The only affine transformation that take a Weierstrass equations in another Weierstrass equation have the form}
$$\begin{cases}
x\longleftarrow u^2 x+r\\
y\longleftarrow u^3 y+ u^2s x + t
  \end{cases} r,s,t,u\in\F_q$$
\end{Note}
\end{frame}

\begin{frame}
\frametitle{The Weierstra\ss\ equation}
\framesubtitle{Classification of simplified forms}

After applying a suitable affine transformation we can always assume that $E/\F_q (q=p^n)$
has a Weierstra\ss\ equation of the following form\pause

\begin{scriptsize}
 \begin{example}[Classification]
\centerline{\begin{tabular}{|l|c|l|}
\hline
 $E$ & $p$ & $\Delta_E$\\
\hline
&&\\
 $y^2=x^3+Ax+B$ & $\ge5$ & $4A^3+27B^2$\\
&&\\
$y^2+xy=x^3+a_2x^2+a_6$ & $2$ & $a_6^2$\\
&&\\
 $y^2+a_3y=x^3+a_4x+a_6$  & $2$ & $a_3^4$\\
&&\\
 $y^2=x^3+Ax^2+Bx+C$ & $3$ & $
                               4A^3C-A^2B^2-18ABC+4B^3+27C^2$
                              \\
&&\\\hline
\end{tabular}}
\end{example}
\end{scriptsize}\pause

\begin{definition}[Elliptic curve] An elliptic curve is the data of a non
singular Weierstra\ss\ equation (i.e. $\Delta_E\neq0$)
\end{definition}\pause

\centerline{\alert{\textbf{Note:} If $p\ge3, \Delta_E\neq0\Leftrightarrow x^3+Ax^2+Bx+C$ has {no} double root}}
\end{frame}

\subsection{Elliptic curves \texorpdfstring{$/\F_2$}{F2}}
\begin{frame}
\frametitle{Elliptic curves over $\F_2$}

All possible Weierstra\ss\ equations over $\F_2$ are:\pause

\begin{beamerboxesrounded}[upper=block title example,lower=block body alerted,shadow=true]{Weierstra\ss\ equations over $\F_2$}
\begin{enumerate}
 \item $y^2+xy=x^3+x^2+1$
 \item$y^2+xy=x^3+1$
 \item$y^2+y=x^3+x$
 \item$y^2+y=x^3+x+1$
 \item$y^2+y=x^3$
 \item$y^2+y=x^3+1$
 \end{enumerate}
\end{beamerboxesrounded}
\pause

However the change of variables
$\begin{cases} x\leftarrow x+1\\ y\leftarrow y+x\end{cases}$ takes the sixth curve
into the fifth. Hence we can remove the sixth from the list.
\pause\bigskip

\begin{beamerboxesrounded}[upper=postit,lower=block body,shadow=true]{Fact:}
There are $5$ affinely inequivalent elliptic curves over $\F_2$
\end{beamerboxesrounded}
\end{frame}

\subsection{Elliptic curves \texorpdfstring{$/\F_3$}{F3}}
\begin{frame}
\frametitle{Elliptic curves in characteristic $3$}

Via a suitable transformation ($x\rightarrow u^2x+r, y\rightarrow u^3y+u^2sx+t$) over $\F_3$,  $8$ inequivalent
elliptic curves over $\F_3$ are found:\pause

\begin{beamerboxesrounded}[upper=block title example,lower=block body alerted,shadow=true]{Weierstra\ss\ equations over $\F_3$}
\begin{enumerate}
 \item $y^2=x^3+x$
 \item$y^2=x^3 - x$
 \item$y^2=x^3 - x +1$
 \item$y^2=x^3 - x -1$
 \item$y^2=x^3 + x^2 + 1$
 \item$y^2=x^3 + x^2 - 1$
 \item$y^2=x^3 - x^2 + 1$
 \item$y^2=x^3 - x^2 - 1$
 \end{enumerate}
\end{beamerboxesrounded}\pause

\begin{beamerboxesrounded}[upper=postit,lower=block body,shadow=true]{Fact:}
             let $\left(\frac{a}{q}\right)$ be the Kronecker symbol. 
            Then the number of non--isomorphic (i.e. inequivalent) classes of elliptic curves over $\F_q$ is 
$$2q+3+\left(\frac{-4}{q}\right)+2\left(\frac{-3}{q}\right)$$
\end{beamerboxesrounded}



% p=5;S=0;for(a=0,p-1,for(b=0,p-1,if((4*a^3+27*b^2)%p>0,print1(S++" "x^3+a*x+b" 2-torsion "matsize(factormod(x^3+a*x+b,p))[1]);T=1;for(x=0,p-1,for(y=0,p-1,if((y^2-x^3-a*x-b)%p==0,T++)));print("  pts= "T))))

\end{frame}

\section{The sum of points}
\begin{frame}
\frametitle{The definition of $E(\F_q)$}
\begin{beamercolorbox}[shadow=true,left,rounded=true,wd=12cm]{formul}
Let $E/\F_q$ elliptic curve and consider a ``symbol'' $\infty$ (point at infinity). Set
$$E(\F_q)=\{(x,y)\in \F_q^2:\ y^2+a_1xy+a_3y=x^3+a_2x^2+a_4x+a_6\}\cup\{\infty\}$$
\end{beamercolorbox}\pause

\ \hfill \begin{beamercolorbox}[shadow=true,left,rounded=true,wd=9cm]{postit}
Hence\pause
\begin{itemize}
 \item<1-> $E(\F_q)\subset\F_q^2\cup\{\infty\}$
 \item<2-> If $\F_q\subset\F_{q^n}$, then $E(\F_q)\subset E(\F_{q^n})$
 \item<3-> We may think that $\infty$ sits on the top of the $y$--axis (``vertical direction'') 
\end{itemize}
\end{beamercolorbox}\pause

\begin{Definition}[line through points $P,Q\in E(\F_q)$]
$r_{P,Q}:\begin{cases}
                     \text{line through $P$ and }Q &\text{if }P\neq Q\\
                     \text{tangent line to $E$ at }P &\text{if }P=Q
                    \end{cases}$\hfill projective or affine
\end{Definition}\pause

\begin{itemize}[<+-| alert@+>]
\item if $\#(r_{P,Q}\cap E(\F_q))\ge2\ \Rightarrow\ \#(r_{P,Q}\cap E(\F_q))=3$\\
\hfill\small{\alert{if tangent line, contact point is counted with multiplicity}}  
\item $r_{\infty,\infty}\cap E(\F_q)=\{\infty,\infty,\infty\}$%\vspace*{-4.4pt}
\end{itemize}

\end{frame}

\begin{frame}
\frametitle{History (from \textsc{Wikipedia})}

\begin{columns}[c]
\begin{column}{4.5cm}
\begin{small}
\textbf{Carl Gustav Jacob Jacobi} (10/12/1804 -- 18/02/1851) was a German mathematician,
who made fundamental contributions to elliptic functions, dynamics, differential equations,
and number theory.
\end{small}\\
\centerline{\includegraphics[width=1.8cm]{images/Jacobi.jpg}}
%\centerline{\scriptsize{Carl Gustav Jacob Jacobi}}\\
\begin{scriptsize}\begin{block}{Some of His Achievements:}
\begin{itemize}
 \item Theta and elliptic function
 \item Hamilton Jacobi Theory
 \item Inventor of determinants
 \item Jacobi Identity\\
 \tiny{ $[A,[B,C]]+[B,[C,A]]+[C,[A,B]]=0$}
\end{itemize}
\end{block}\end{scriptsize}
\end{column}\pause
\begin{column}{5.5cm}\vspace*{-16.3pt}
\begin{center}
\includegraphics[width=5.5cm]{images/add1.pdf}\pause
\llap{\includegraphics[width=5.5cm]{images/add2.pdf}}\pause
\llap{\includegraphics[width=5.5cm]{images/add3.pdf}}\pause
\llap{\includegraphics[width=5.5cm]{images/add5.pdf}}\pause
\llap{\includegraphics[width=5.5cm]{images/add6.pdf}}\pause
\llap{\includegraphics[width=5.5cm]{images/add7.pdf}}\pause
\llap{\includegraphics[width=5.5cm]{images/add1.pdf}}\pause
\llap{\includegraphics[width=5.5cm]{images/add8.pdf}}\pause
\llap{\includegraphics[width=5.5cm]{images/add9.pdf}}\pause
\llap{\includegraphics[width=5.5cm]{images/ad10.pdf}}\pause
\llap{\includegraphics[width=5.5cm]{images/ad11.pdf}}\pause
\llap{\includegraphics[width=5.5cm]{images/ad12.pdf}}\pause
\llap{\includegraphics[width=5.5cm]{images/add1.pdf}}\pause
\llap{\includegraphics[width=5.5cm]{images/ad13.pdf}}\pause
\llap{\includegraphics[width=5.5cm]{images/ad14.pdf}}\pause
\llap{\includegraphics[width=5.5cm]{images/ad15.pdf}}\pause
\llap{\includegraphics[width=5.5cm]{images/add7.pdf}}\pause
\end{center}
\small{
$r_{P,Q}\cap E(\F_q)=\{P,Q,R\}$\\
$r_{R,\infty}\cap E(\F_q)=\{\infty,R,R'\}$}
\centerline{\begin{beamercolorbox}[shadow=true,center,rounded=true,wd=2cm]{formul}
$P+_E Q:=R'$\pause
            \end{beamercolorbox}}%\smallskip

 \small{$r_{P,\infty}\cap E(\F_q)=\{P,\infty,P'\}$}\\
 \centerline{\begin{beamercolorbox}[shadow=true,center,rounded=true,wd=2cm]{formul}
             $-P:=P'$
            \end{beamercolorbox}}

\end{column}
\end{columns}
\end{frame}

\begin{frame}
\frametitle{Properties of the operation ``$+_E$''}

\begin{Theorem}
 The addition law on $E(\F_q)$ has the following
properties:
\begin{enumerate}[<+-| alert@+>][(a)]
 \item $P+_EQ\in E(\F_q)\hfill\forall P,Q\in E(\F_q)$
 \item  $P+_E\infty=\infty+_E P=P\hfill\forall P\in E(\F_q)$
 \item  $P+_E(-P)=\infty\hfill\forall P\in E(\F_q)$
 \item  $P+_E(Q +_E R)=(P+_E Q)+_E R\hfill\forall P,Q,R\in E(\F_q)$
 \item  $P+_E Q=Q +_E P\hfill\forall P,Q\in E(\F_q)$
\end{enumerate}
 \end{Theorem}\pause

\begin{itemize}[<+-| alert@+>]
%  \item By ``a point of $E/\F_q$ ($P\in E$)'' we mean $P\in E(\overline{\F_q})$
%  in analogy for $E/\Q$ where ``a point of $E$'' means  $P\in E(\C)$
 \item $\left(E(\F_q),+_E\right)$  \alert{commutative group}
 \item All group properties are easy except \alert{associative law (d)}
 \item Geometric proof of associativity uses \emph{Pappo's Theorem}
% \item We shall comment on how to do it by explicit computation
 \item can substitute $\F_q$ with any field $K$; Theorem holds for $\left(E(K),+_E\right)$
\item $-P=-(x_1,y_1)=(x_1,-a_1x_1-a_3-y_1)$
%In particular, if $E/\F_q$, can consider the groups $E(\overline{\F_q})$ or $E(\F_{q^n})$
\end{itemize}
\end{frame}

% \begin{frame}
% \frametitle{Computing the inverse $-P$}
% \centerline{\begin{beamercolorbox}[shadow=true,center,rounded=true,wd=7cm]{formul}
% $E: y^2+a_1xy+a_3y=x^3+a_2x^2+a_4x+a_6$\end{beamercolorbox}}\pause
% 
% If $P=(x_1,y_1)\in E(\F_q)$
% 
% \ \hfill\begin{beamercolorbox}[shadow=true,left,rounded=true,wd=8cm]{formul}
%            \textbf{\color[rgb]{1,0.3,1}Definition:}  $-P:=P'$ where $r_{P,\infty}\cap E(\F_q)=\{P,\infty,P'\}$\hfill\
%             \end{beamercolorbox}\hfill\pause
% 
%             Write $P'=(x_1',y_1')$. Since $r_{P,\infty}: x=x_1\ \Rightarrow x_1'=x_1$ and $y_1$ satisfies
% \centerline{\begin{beamercolorbox}[shadow=true,center,rounded=true,wd=8.8cm]{postit}
% $y^2+a_1x_1y+a_3y-(x_1^3+a_2x_1^2+a_4x_1+a_6)=(y-y_1)(y-y_1')$
% \end{beamercolorbox}}\bigskip\pause
% 
% So $y_1+y_1'=-a_1x_1-a_3$ (\alert{both coefficients of $y$}) and
% \centerline{\begin{beamercolorbox}[shadow=true,center,rounded=true,wd=6cm]{postit}
% $-P=-(x_1,y_1)=(x_1,-a_1x_1-a_3-y_1)$
%  \end{beamercolorbox}}\bigskip\pause
% 
% 
% So, if $P_1=(x_1,y_1), P_2=(x_2,y_2)\in E(\F_q)$,
% 
% \ \hfill\begin{beamercolorbox}[shadow=true,center,rounded=true,wd=9cm]{formul}
%            \textbf{\color[rgb]{1,0.3,1}Definition:} $P_1+_EP_2=-P_3$ where $r_{P_1,P_2}\cap E(\F_q)=\{P_1,P_2,P_3\}\!$
%             \end{beamercolorbox}\bigskip\pause
% 
% Finally, if $P_3=(x_3,y_3)$, then
% \centerline{\begin{beamercolorbox}[shadow=true,center,rounded=true,wd=6cm]{postit}
% $P_1+_EP_2=-P_3=(x_3,-a_1x_3-a_3-y_3)$
% \end{beamercolorbox}}
% \end{frame}

% \begin{frame}
% \frametitle{Lines through points of $E$}
% \centerline{\begin{beamercolorbox}[shadow=true,center,rounded=true,wd=6cm]{formul}
% $E: y^2+a_1xy+a_3y=x^3+a_2x^2+a_4x+a_6$\end{beamercolorbox}}
% where $a_1, a_3, a_2, a_4 ,a_6\in\F_q,$ \pause
% 
% \begin{beamerboxesrounded}[upper=block title example,lower=block body alerted,shadow=true]
% {$P_1=(x_1,y_1), P_2=(x_2,y_2)\in E(\F_q)$}
% \begin{enumerate}[<+-| alert@+>]
%  \item $P_1\neq P_2$ and $x_1\neq x_2\hfil \Longrightarrow\hfil r_{P_1,P_2}: y=\lambda x+\nu$
% \begin{beamercolorbox}[shadow=true,center,rounded=true,wd=6cm]{postit}
% $$\lambda= \frac{y_2-y_1}{x_2-x_1},\qquad \nu=\frac{y_1x_2-x_1y_2}{x_2-x_1}$$
% \end{beamercolorbox}
%  \item $P_1\neq P_2$ and $x_1=x_2\hfil \Longrightarrow\hfil r_{P_1,P_2}: x=x_1$
%  \item  $P_1=P_2$ and $2y_1+a_1x_1+a_3\neq0\ \Longrightarrow\ r_{P_1,P_2}: y=\lambda x+\nu$
% \hspace*{-0.5cm}\begin{beamercolorbox}[shadow=true,center,rounded=true,wd=9.3cm]{postit}
% $$\lambda=\frac{3x_1^2+2a_2x_1+a_4-a_1y_1}{2y_1+a_1x_1+a_3},
%  \nu=-\frac{a_3y_1+x_1^3-a_4x_1-2a_6}{2y_1+a_1x_1+a_3}$$
% \end{beamercolorbox}
% \item $P_1=P_2$ and $2y_1+a_1x_1+a_3=0\hfil \Longrightarrow\hfil r_{P_1,P_2}: x=x_1$
% \item $r_{P_1,\infty}: x=x_1\hfill r_{\infty,\infty}: Z=0$
% \end{enumerate}
% \end{beamerboxesrounded}
% \end{frame}
% 
% \begin{frame}
% \frametitle{Intersection between a line and $E$}
% We want to compute $P_3= (x_3,y_3)$ where $r_{P_1,P_2}: y=\lambda x +\nu$,
% \centerline{\begin{beamercolorbox}[shadow=true,center,rounded=true,wd=5cm]{postit}
% $r_{P_1,P_2}\cap E(\F_q)=\{P_1,P_2,P_3\}$
% \end{beamercolorbox}}
% \pause
% 
% We find the intersection:
% \centerline{\begin{beamercolorbox}[shadow=true,center,rounded=true,wd=7cm]{formul}
% $r_{P_1,P_2}\cap E(\F_q)=$ \scriptsize{\ $\begin{cases}
%  E:y^2+a_1xy+a_3y=x^3+a_2x^2+a_4x+a_6\\ r_{P_1,P_2}: y=\lambda x +\nu
%  \end{cases}$}
% \end{beamercolorbox}}
%  \pause
% Substituting\\
% \centerline{\begin{beamercolorbox}[center,wd=8cm]{postit}
% $(\lambda x +\nu)^2+a_1x(\lambda x +\nu)+a_3(\lambda x +\nu)=x^3+a_2x^2+a_4x+a_6$
%             \end{beamercolorbox}}\medskip\pause
% 
%             Since $x_1$ and $x_2$ are solutions, we can
% find $x_3$ by comparing
% \begin{scriptsize}
% 
% \centerline{\begin{beamercolorbox}[center,wd=9cm]{postit}
%     \begin{align*}
% &x^3+a_2x^2+a_4x+a_6-((\lambda x +\nu)^2+a_1x(\lambda x +\nu)+a_3(\lambda x +\nu))&=\\
% \uncover<5->{&x^3+(\alert{a_2-\lambda^2-a_1\lambda})x^2+\cdots&=\\ }
% \uncover<6->{&(x - x_1)(x - x_2)(x - x_3) = x^3 - ({\color[rgb]{0,0,1}x_1 + x_2 + x_3})x^2 + \cdots&\\ } %(x_1x_2 + x_1x_3 + x_2x_3)x - x_1x_2x_3
% \notag
% \end{align*}\vskip-1.5em
% \end{beamercolorbox}}
% \end{scriptsize}\pause
% 
% Equating coeffcients of $x^2$,
% \centerline{\begin{beamercolorbox}[center,wd=8cm]{postit}
% $x_3 = \lambda^2-a_1\lambda-a_2- x_1-x_2,\qquad y_3 = \lambda x_3 + \nu$
%             \end{beamercolorbox}}
% \pause
% Finally\\
% \centerline{\begin{beamercolorbox}[shadow=true,center,rounded=true,wd=9cm]{formul}
% \small{$P_3 =({\color[cmyk]{0,1,1,0.5}\lambda^2-a_1\lambda-a_2-x_1-x_2},{\color[cmyk]{1,0,1,0.5}\lambda^3-a_1\lambda^2-\lambda(a_2+x_1+x_2)+\nu})$}
%             \end{beamercolorbox}}
% \end{frame}

\begin{frame}
\frametitle{Formulas for Addition on $E$ (Summary)}
\centerline{\begin{beamercolorbox}[shadow=true,center,rounded=true,wd=6cm]{formul}
$E: y^2+a_1xy+a_3y=x^3+a_2x^2+a_4x+a_6$\end{beamercolorbox}}
$P_1 = (x_1, y_1), P_2 = (x_2, y_2)\in E(\F_q)\setminus\{\infty\}$,
\begin{beamerboxesrounded}[upper=block title example,lower=block body alerted,shadow=true]{Addition Laws for the sum of affine points}
\begin{itemize}[<+-| alert@+>]
 \item If $P_1\neq P_2$
\begin{itemize}
 \item $x_1 = x_2\ \hfill\Rightarrow\hfil$\ \ \
\begin{beamercolorbox}[shadow=true,center,rounded=true,wd=2cm]{formul}$P_1 +_E P_2 = \infty$
\end{beamercolorbox}
 \item $x_1 \neq x_2$\\
\centerline{\begin{beamercolorbox}[shadow=true,center,wd=4cm]{postit}
             $\displaystyle\lambda=\frac{y_2-y_1}{x_2-x_1}\qquad \nu=\frac{y_1x_2-y_2x_1}{x_2-x_1}$
            \end{beamercolorbox}}
 \end{itemize}
\item If $P_1 = P_2$
\begin{itemize}
 \item $2y_1+a_1x+a_3 = 0\ \hfill\Rightarrow\hfil$\ \ \
\begin{beamercolorbox}[shadow=true,center,rounded=true,wd=3cm]{formul}$P_1 +_E P_2 = 2P_1 = \infty$\end{beamercolorbox}
\item $2y_1+a_1x+a_3\neq 0$\\
\centerline{\begin{beamercolorbox}[shadow=true,center,wd=7cm]{postit}
$\displaystyle\lambda=\frac{3x_1^2+2a_2x_1+a_4-a_1y_1}{2y_1+a_1x+a_3}, \nu=-\frac{a_3y_1+x_1^3-a_4x_1-2a_6}{2y_1+a_1x_1+a_3}$
            \end{beamercolorbox}}
\end{itemize}
\end{itemize}\pause

Then\\
\centerline{\begin{beamercolorbox}[shadow=true,center,rounded=true,wd=11cm]{formul}
{\small $P_1 +_E P_2 = ({\color[cmyk]{0,1,1,0.5}\lambda^2-a_1\lambda-a_2-x_1-x_2},
{\color[cmyk]{1,0,1,0.5}-\lambda^3-a_1^2\lambda+(\lambda+a_1)(a_2+x_1+x_2)-a_3-\nu})$}
            \end{beamercolorbox}}
\end{beamerboxesrounded}
\end{frame}

\begin{frame}
\frametitle{Formulas for Addition on $E$ (Summary for special equation)}
\centerline{\begin{beamercolorbox}[shadow=true,center,rounded=true,wd=6cm]{formul}
$E: y^2=x^3+Ax+B$\end{beamercolorbox}}
$P_1 = (x_1, y_1), P_2 = (x_2, y_2)\in E(\F_q)\setminus\{\infty\}$,
\begin{beamerboxesrounded}[upper=block title example,lower=block body alerted,shadow=true]{Addition Laws for  the sum of affine points}
\begin{itemize}
 \item If $P_1\neq P_2$
\begin{itemize}
 \item $x_1 = x_2\ \hfill\Rightarrow\hfil$\ \ \
\begin{beamercolorbox}[shadow=true,center,rounded=true,wd=2cm]{formul}$P_1 +_E P_2 = \infty$
\end{beamercolorbox}
 \item $x_1 \neq x_2$\\
\centerline{\begin{beamercolorbox}[shadow=true,center,wd=4cm]{postit}
             $\displaystyle\lambda=\frac{y_2-y_1}{x_2-x_1}\qquad \nu=\frac{y_1x_2-y_2x_1}{x_2-x_1}$
            \end{beamercolorbox}}
 \end{itemize}
\item If $P_1 = P_2$
\begin{itemize}
 \item $y_1 = 0\ \hfill\Rightarrow\hfil$\ \ \
\begin{beamercolorbox}[shadow=true,center,rounded=true,wd=3cm]{formul}$P_1 +_E P_2 = 2P_1 = \infty$\end{beamercolorbox}
\item $y_1\neq 0$\\
\centerline{\begin{beamercolorbox}[shadow=true,center,wd=7cm]{postit}
$\displaystyle\lambda=\frac{3x_1^2+A}{2y_1}, \nu=-\frac{x_1^3-Ax_1-2B}{2y_1}$
            \end{beamercolorbox}}
\end{itemize}
\end{itemize}

Then\\
\centerline{\begin{beamercolorbox}[shadow=true,center,rounded=true,wd=11cm]{formul}
{\small $P_1 +_E P_2 = ({\color[cmyk]{0,1,1,0.5}\lambda^2-x_1-x_2},
{\color[cmyk]{1,0,1,0.5}-\lambda^3+\lambda(x_1+x_2)-\nu})$}
            \end{beamercolorbox}}
\end{beamerboxesrounded}

\end{frame}


% \begin{frame}
%  \frametitle{A Finite Field Example}
% 
% Over $\F_p$ geometric pictures don't make sense.\pause
%  \begin{example}
% Let
% $E: y^2 = x^3 - 5x + 8 /\F_{37}$,\pause\hfill $P = (6, 3) , Q = (9, 10)\in E(\F_{37})$\pause
% 
% \centerline{\begin{beamercolorbox}[shadow=true,center,rounded=true,wd=6cm]{formul}
% $r_{P,Q}: y=27x+26\quad r_{P,P}: y=11x+11 $
%             \end{beamercolorbox}}\pause
% 
% \centerline{\begin{beamercolorbox}[shadow=true,center,rounded=true,wd=9cm]{postit}
% $r_{P,Q}\cap E(\F_{37})=\begin{cases}
%                           y^2 = x^3 - 5x + 8 \\ y = 27 x + 26
%                          \end{cases}\!\!=\{(6,3), (9,10), (11,27)\}$
%                                      \end{beamercolorbox}}\pause
% 
% \centerline{\begin{beamercolorbox}[shadow=true,center,rounded=true,wd=9cm]{postit}
% $r_{P,P}\cap E(\F_{37})=\begin{cases}
%                           y^2 = x^3 - 5x + 8 \\ y =11 x + 11
%                          \end{cases}\!\!=\{(6,3), (6,3), (35,26)\}$\end{beamercolorbox}}\pause
% 
% 
% \centerline{\begin{beamercolorbox}[shadow=true,center,rounded=true,wd=7.3cm]{formul}
% $P+_EQ=(11,10)\qquad 2P=(35,11)$
%             \end{beamercolorbox}}\pause
% 
% \ \hfill\scriptsize{$3P=(34,25), 4P=(8,6), 5P=(16,19),\ldots 3P+4Q=(31,28),\ldots$}
%  \end{example}\pause
% 
%  \begin{beamerboxesrounded}[upper=block title example,lower=block body alerted,shadow=true]{Exercise}
%  $\bullet$ Compute the order and the {\color[rgb]{0.1,0.3,1}{Group Structure}} of $E(\F_{37})$\\
%  $\bullet$ Show that if $E_1/\F_q$ is equivalent to $E_2/\F_q$, then $E_1(\F_{q^n})\cong E_2(\F_{q^n})\forall n\in\N$.
%   \end{beamerboxesrounded}
% \end{frame}

\begin{frame}%[label=current]
 \frametitle{Group Structure}

\begin{theorem}[Classification of finite abelian groups]
 If $G$ is {\color[rgb]{0.9,0.3,0.2}{abelian and finite}},  $\exists n_1,\ldots,n_k\in\N^{>1}$ such that
 \begin{enumerate}[<+-| alert@+>]
\item $n_1\mid n_2\mid\cdots\mid n_k$
\item $G\cong C_{n_1}\oplus\cdots\oplus C_{n_k}$
\end{enumerate}
\ \hfill Furthermore $n_1,\ldots,n_k$ ({\color[rgb]{0.9,0.3,0.2} Group Structure}) are unique
 \end{theorem}\pause

% \begin{example}[One can verify that:]
%  $$C_{2400}\oplus C_{72} \oplus C_{1440}\cong C_{12}\oplus C_{60}\oplus C_{15200}$$
% \end{example}\pause

\begin{theorem}[Structure Theorem for Elliptic curves over a finite field] Let $E/\F_q$ be 
an elliptic curve, then
$$E(\F_q)\cong C_n\oplus C_{nk}\qquad\exists n,k\in\N^{>0}.$$
(i.e. $E(\F_q)$ is either cyclic ($n=1$) or the product of $2$ cyclic groups)
\pause
\end{theorem}            
\end{frame}

% \begin{frame}%[label=current2]
%  \frametitle{Proof of the associativity}
%  \centerline{\begin{beamercolorbox}[shadow=true,center,rounded=true,wd=7.3cm]{formul}
%              $P+_E(Q+_ER)=(P+_EQ)+_ER\quad\forall P,Q,R\in E$
%             \end{beamercolorbox}}\pause
%  We should verify the above in many different cases according if $Q=R$, $P=Q$, $P=Q+_ER,\ldots$\pause
% 
% Here we deal with the \emph{generic case}. i.e. All the points
% \alert{$\pm P, \pm R,\pm Q,\pm(Q+_ER),\pm(P+_EQ),\infty$} all different
% 
% \ \hfill {\begin{beamercolorbox}[shadow=true,left,rounded=true,wd=9.6cm]{postit}
%  {\small{\color[rgb]{1,0.1,0.1}\texttt{Mathematica code}}\\
% \texttt{L[x\_,y\_,r\_,s\_]:=(s-y)/(r-x);\\
% M[x\_,y\_,r\_,s\_]:=(yr-sx)/(r-x);\\
% A[\{x\_,y\_\},\{r\_,s\_\}]:=\{(L[x,y,r,s])$^2$-(x+r),\\
% \ \hfill -(L[x,y,r,s])$^3$+L[x,y,r,s](x+r)-M[x,y,r,s]\}\\
% Together[A[A[\{x,y\},\{u,v\}],\{h,k\}]-A[\{x,y\},A[\{u,v\},\{h,k\}]]]\\
% det = Det[(\{\{1,x$_1$,x$_1^3$-y$_1^2$\},\{1,x$_2$,x$_2^3$-y$_2^2$\},\{1,x$_3$,x$_3^3$-y$_3^2$\}\})]\\
% PolynomialQ[Together[Numerator[Factor[res[[1]]]]/det],\\
% \ \hfill\{x$_1$,x$_2$,x$_3$,y$_1$,y$_2$,y$_3$\}]
% PolynomialQ[Together[Numerator[Factor[res[[2]]]]/det],\\ \ \hfill\{x$_1$,x$_2$,x$_3$,y$_1$,y$_2$,y$_3$\}]}}
%              \end{beamercolorbox}}\pause
% 
% 
% % {\begin{beamercolorbox}[shadow=true,left,rounded=true,wd=9.6cm]{postit}
% %  \scriptsize{{\color[rgb]{1,0.1,0.1}\texttt{Pari code}}\\
% % \texttt{L(a,b,c,d)=(d-b)/(c-a);\\
% % M(a,b,c,d)=(b*c-a*d)/(c-a);\\
% % AX(a,b,c,d)=L(a,b,c,d)\^{}2-(a+c);\\
% % AY(a,b,c,d)=-L(a,b,c,d)\^{}3+L(a,b,c,d)*(a+c)-M(a,b,c,d);\\
% % simplify(AX(x1,y1,AX(x2,y2,x3,y3),AY(x2,y2,x3,y3))-\\ \ \hfill AX(AX(x1,y1,x2,y2),AY(x1,y1,x2,y2),x3,y3));\\
% % simplify(AY(x1,y1,AX(x2,y2,x3,y3),AY(x2,y2,x3,y3))-\\ \ \hfill AY(AX(x1,y1,x2,y2),AY(x1,y1,x2,y2),x3,y3))}}
% % \end{beamercolorbox}}\pause
% 
% \begin{small}
% \begin{itemize}[<+-| alert@+>]
%  \item runs in 2 seconds on a PC
%  %\item Complete proof requires several cases (e.g. $(P+_EP)+_ER=P+_E(P+_ER)$)
%  \item For an elementary proof:
%  ``\text{An Elementary Proof of the Group Law for Elliptic Curves.}''
% Department of Mathematics: Rice
% University. Web. 20 Nov. 2009.\\ \ \hfill\texttt{http://math.rice.edu/\~{}friedl/papers/AAELLIPTIC.PDF}
% \item More cases to check. e.g  \alert{$P+_E2Q=(P+_EQ)+_EQ$}
% \end{itemize}
% \end{small}
% \end{frame}



\section{Examples}
\subsection{Structure of \texorpdfstring{$E(\F_2)$ and $E(\F_3)$}{E(F2) and E(F3)}}
\begin{frame}
\frametitle{EXAMPLE: Elliptic curves over $\F_2$ and over $\F_3$}

From our previous list:
\begin{block}{Groups of points of curves over $\F_2$}

\centerline{
\begin{tabular}{|l|c|l|}
\hline
 $E$ & $E(\F_2)$ & $E(\F_2)$\\
\hline
 $y^2+xy=x^3+x^2+1$ & $\{\infty,(0,1)\}$& $C_2$\\
$y^2+xy=x^3+1$ & $\{\infty,(0,1),(1,0),(1,1)\}$ & $C_4$\\
$y^2+y=x^3+x$&$\{\infty,(0,0),(0,1), (1,0),(1,1)\}$&$C_5$\\
 $y^2+y=x^3+x+1$ &$\{\infty\}$&$1$\\
$y^2+y=x^3$ & $\{\infty,(0,0), (0,1)\}$ & $C_3$ \\
\hline
\end{tabular}}
\end{block}
\ \hfill \begin{beamercolorbox}[center,wd=7cm]{postit}
Note: each $C_i, i=1,\ldots,5$ is represented by a curve $/\F_2$
            \end{beamercolorbox}\pause

\pause

\begin{block}{Groups of points of curves over $\F_3$}\centerline{
\begin{tabular}{|l|r|c|c|}
\hline
$i$ & $E_i$ & $E_i(\F_3)$ &$E_i(\F_3)$\\
\hline
$1$& $y^2=x^3+x$ & {$\{\infty,(0,0),(2,1),(2,2)\}$}& $C_4$\\
\hline
$2$&$y^2=x^3 - x$ & {$\{\infty,(1,0),(2,0),(0,0)\}$} & $C_2\oplus C_2$\\
\hline
$3$&$y^2=x^3 - x +1$&{$\{\infty,(0,1),(0,2),(1,1),(1,2),(2,1),(2,2)\}$} & $C_7$\\
\hline
$4$&$y^2=x^3 - x -1$  &{$\{\infty\}$}&$\{1\}$\\
\hline
$5$&$y^2=x^3 + x^2 - 1$ &{$\{\infty,(1,1), (1,2)\}$} & $C_3$ \\
\hline
$6$&$y^2=x^3 + x^2 + 1$ & {$\{\infty,(0,1), (0,2), (1,0),(2,1), (2,2)\}$} & $C_6$ \\
\hline
$7$&$y^2=x^3 - x^2 + 1$ & {$\{\infty,(0,1), (0,2), (1,1), (1,2),\}$} & $C_5$ \\
\hline
$8$&$y^2=x^3 - x^2 - 1$ & {$\{\infty,(2,0))\}$} & $C_2$ \\
\hline
\end{tabular}}
\end{block}
\pause

\ \hfill \begin{beamercolorbox}[center,wd=7cm]{postit}
Note: each $C_i, i=1,\ldots,7$ is represented by a curve $/\F_3$
            \end{beamercolorbox}\pause

\end{frame}


% \subsection{Structure of \texorpdfstring{$E(\F_3)$}{E(F3)}}
% \begin{frame}
% \frametitle{EXAMPLE: Elliptic curves over $\F_3$}
% From our previous list:
% 
% \begin{block}{Groups of points}\centerline{
% \begin{tabular}{|l|r|c|c|}
% \hline
% $i$ & $E_i$ & $E_i(\F_3)$ &$E_i(\F_3)$\\
% \hline
% $1$& $y^2=x^3+x$ & {$\{\infty,(0,0),(2,1),(2,2)\}$}& $C_4$\\
% \hline
% $2$&$y^2=x^3 - x$ & {$\{\infty,(1,0),(2,0),(0,0)\}$} & $C_2\oplus C_2$\\
% \hline
% $3$&$y^2=x^3 - x +1$&{$\{\infty,(0,1),(0,2),(1,1),(1,2),(2,1),(2,2)\}$} & $C_7$\\
% \hline
% $4$&$y^2=x^3 - x -1$  &{$\{\infty\}$}&$\{1\}$\\
% \hline
% $5$&$y^2=x^3 + x^2 - 1$ &{$\{\infty,(1,1), (1,2)\}$} & $C_3$ \\
% \hline
% $6$&$y^2=x^3 + x^2 + 1$ & {$\{\infty,(0,1), (0,2), (1,0),(2,1), (2,2)\}$} & $C_6$ \\
% \hline
% $7$&$y^2=x^3 - x^2 + 1$ & {$\{\infty,(0,1), (0,2), (1,1), (1,2),\}$} & $C_5$ \\
% \hline
% $8$&$y^2=x^3 - x^2 - 1$ & {$\{\infty,(2,0))\}$} & $C_2$ \\
% \hline
% \end{tabular}}
% \end{block}
% \pause
% 
% \ \hfill \begin{beamercolorbox}[center,wd=9cm]{postit}
% Note: each $C_i, i=1,\ldots,7$ is represented by a curve $/\F_3$
%             \end{beamercolorbox}\pause
% 
%             \begin{beamerboxesrounded}[upper=block title example,lower=block body alerted,shadow=true]{Exercise:
%             let $\left(\frac{a}{q}\right)$ be the Kronecker symbol. 
%             Show that the number of non--isomorphic (i.e. inequivalent) classes of elliptic curves over $\F_q$ is }
% $$2q+3+\left(\frac{-4}{q}\right)+2\left(\frac{-3}{q}\right)$$
%   \end{beamerboxesrounded}
% \end{frame}


% \subsection{Further Examples}
% \begin{frame}
% \frametitle{EXAMPLE: Elliptic curves over $\F_5$ and $\F_4$}
% 
% $\forall E/\F_5$ (12 elliptic curves), $\#E(\F_5)\in \{2,3,4,5,6,7,8,9,10\}.$ $\forall n, 2\le n\le10 \exists! E/\F_5: \#E(\F_5)=n$
% %each number corresponds to a unique curve
% with the exceptions:
% 
% \begin{example}[Elliptic curves over $\F_5$]
% \begin{itemize}[<+-| alert@+>]
%  \item \alert{$E_1: y^2=x^3+1$} and \alert{$E_2: y^2=x^3+2$}\hfill both order $6$\\
%  \begin{columns}
% \begin{column}{4cm}
% \begin{beamercolorbox}[shadow=true,center,rounded=true,wd=2.5cm]{postit}
%         $\begin{cases}
% x\longleftarrow 2x\\
% y\longleftarrow \sqrt{3}y
%   \end{cases}$\end{beamercolorbox}
%  \end{column}
%  \begin{column}{5cm}
% $E_1$ and $E_2$ affinely equivalent over $\F_5[\sqrt{3}]=\F_{25}$ (\emph{twists})
%  \end{column}
%  \end{columns}
% \item \alert{$E_3: y^2=x^3+x$} and \alert{$E_4: y^2=x^3+x+2$}
% \hfill order $4$
% $$E_3(\F_5)\cong C_2\oplus C_2\qquad E_4(\F_5)\cong C_4$$
% \item \alert{$E_5: y^2=x^3+4x$} and \alert{$E_6: y^2=x^3+4x+1$}
% \hfill both order $8$
% $$E_5(\F_5)\cong C_2\times\oplus C_4\qquad E_6(\F_5)\cong C_8$$
% \item \alert{$E_7: y^2=x^3+x+1$}\hfill  order $9$ and $E_7(\F_5)\cong C_9$
% \end{itemize}
% \end{example}\pause\vspace*{-3pt}
% \begin{beamerboxesrounded}[upper=block title example,lower=block body alerted,shadow=true]{\textbf{Exercise:} Classify all elliptic curves over $\F_4=\F_2[\xi], \xi^2=\xi+1$}
%  \end{beamerboxesrounded}
% \end{frame}

%\subsection{Further reading.}

\section{the \texorpdfstring{$j$}{j}-invariant}
\begin{frame}
\frametitle{The $j$-invariant}
Let  $E/K: y^2=x^3+Ax+B$, $p\ge5$ and $\Delta_E:=4A^3+27B^2$. \pause

\begin{definition} The $j$--invariant of $E$ is
$j=j(E)=1728\frac{4A^3}{4A^3+27B^2}$
\end{definition}

% \centerline{
% \begin{beamercolorbox}[shadow=true,center,rounded=true,wd=8.2cm]{postit} $\begin{cases}
% x\longleftarrow u^{-2} x\\
% y\longleftarrow u^{-3} y
%   \end{cases} u\in K^*\ \rightsquigarrow\ E\longrightarrow E_u: y^2=x^3+u^4Ax+u^6B$
%   \end{beamercolorbox}}\pause

  \begin{definition} Let $u\in K^*$. The elliptic curve $E_u:y^2=x^3+u^2Ax+u^3B$ is called the \alert{twist} of $E$ by $u$
\end{definition}


\begin{beamerboxesrounded}[upper=block title example,lower=block body alerted,shadow=true]{Properties of
$j$--invariants}
\begin{enumerate}[<+-| alert@+>]
  \item $j(E)=j(E_u), \forall u\in K^*$
  \item $j(E'/K)=j(E''/K)\ \Rightarrow\ \exists u\in\overline{K}^*$ s.t. $E''=E'_u$
  %\ \hfill if $K=\F_q$ can take $u\in\F_{q^{12}}$
  \item $j\ne 0,1728\Rightarrow E: y^2=x^3+\frac{3j}{1728-j}x+\frac{2j}{1728-j},$ $j(E)=j$
  \item $j=0\ \Rightarrow\ E: y^2=x^3+B,\quad j=1728\ \Rightarrow\ E: y^2=x^3+Ax$
  %\item The above have more isomorphism that usual ($6$ and $4$ respectively)
  \item $j: K\leftarrow\!\rightarrow\{\bar{K}$--affinely
  equivalent classes of $E/K\}$.
  \item $p=2, 3$ different definition
 \item $E$ and $E_\mu$ are $\F_q[\sqrt{\mu}]$--affinely equivalent
  \item $\#E(\F_{q^2})=\#E_\mu(\F_{q^2})$
  \item usually $\#E(\F_{q})\neq\#E_\mu(\F_{q})$
  \end{enumerate}
\end{beamerboxesrounded}
\end{frame}

% \begin{frame}
% \frametitle{Examples of $j$ invariants}
% 
% From Friday \alert{$E_1: y^2=x^3+1$} and \alert{$E_2: y^2=x^3+2$}\pause
% 
% \begin{beamercolorbox}[shadow=true,center,rounded=true,wd=9cm]{formul}
% $\#E_1(\F_5)=\#E_2(\F_5)=6\qquad$ and $\qquad j(E_1)=j(E_2)=0$
% \end{beamercolorbox}
% 
% \begin{columns}
% \begin{column}{4cm}
% \begin{beamercolorbox}[shadow=true,center,rounded=true,wd=2.5cm]{postit}
%         $\begin{cases}
% x\longleftarrow 2x\\
% y\longleftarrow \sqrt{3}y
%   \end{cases}$\end{beamercolorbox}
%  \end{column}
%  \begin{column}{5cm}
% $E_1$ and $E_2$ affinely equivalent over $\F_5[\sqrt{3}]=\F_{25}$ (\emph{twists})
%  \end{column}
%  \end{columns}\pause
% 
% \begin{definition}[twisted curve] Let $E/\F_q: y^2=x^3+Ax+B, \mu\in\F_q^*\setminus(\F_q^*)^2$.
% $$E_\mu: y^2=x^3+\mu^2Ax+\mu^3B$$ is called \alert{twisted curve.}
% \end{definition}\pause
% 
% \begin{beamerboxesrounded}[upper=block title example,lower=block body alerted,shadow=true]{Exercise: prove that}
% \begin{itemize}[<+-| alert@+>]
%   \item $j(E)=j(E_\mu)$
%   \item $E$ and $E_\mu$ are $\F_q[\sqrt{\mu}]$--affinely equivalent
%   \item $\#E(\F_{q^2})=\#E_\mu(\F_{q^2})$
%   \item usually $\#E(\F_{q})\neq\#E_\mu(\F_{q})$
% \end{itemize}
% \end{beamerboxesrounded}
% \end{frame}


\section{Points of finite order}

\subsection{Points of order 2}
\begin{frame}\frametitle{Determining points of order $2$}
Let $P=(x_1,y_1)\in E(\F_q)\setminus\{\infty\},$\\ \pause
\centerline{
 \begin{beamercolorbox}[rounded=true,shadow=true,wd=6cm,center]{postit}
$P$ has order $2\ \Longleftrightarrow\ 2P=\infty\ \Longleftrightarrow\ P=-P$
\end{beamercolorbox}}\pause
So
\centerline{\small{
 \begin{beamercolorbox}[rounded=true,shadow=true,wd=10cm,center]{formul}
$-P=(x_1,-a_1x_1-a_3-y_1)=(x_1,y_1)=P\ \pause \Longrightarrow\ 2y_1=-a_1x_1-a_3$\end{beamercolorbox}}}\pause\medskip

If $p\neq2$, can assume $E: y^2=x^3+Ax^2+Bx+C$\pause

\centerline{\small{
 \begin{beamercolorbox}[rounded=true,shadow=true,wd=10cm,center]{formul}
$-P=(x_1,-y_1)=(x_1,y_1)=P\ \pause \Longrightarrow\ y_1=0,
x_1^3+Ax_1^2+Bx_1+C=0$\hfill
\end{beamercolorbox}}}\pause\medskip

\begin{Note}
\begin{itemize}[<+-| alert@+>]
 \item the number of points of order $2$ in $E(\F_q)$ equals the number of roots of $X^3+Ax^2+Bx+C$ in $\F_q$
 \item roots are distinct since discriminant $\Delta_E\neq0$
% \item $E(\F_{q^6})$ has always $3$ points of order $2$ if $E/\F_q$
% \item $E[2]:=\{P\in E(\bar{\F}_q): 2P=\infty\}\cong C_2\oplus C_2$
\end{itemize}
\end{Note}

% $\begin{cases}
%    2y=-a_1x-a_3\\
%    y^2+a_1xy+a_3y=x^3+a_2x^2+a_4x+a_6
%   \end{cases}\longrightarrow$, $\begin{cases}
%    2y=-a_1x-a_3\\
%    x^3+(a_2+a_1^2/4)x^2+(a_4+a_1a_3/2)x+a_6+a_3^2/4=0
%   \end{cases}$

\end{frame}

\begin{frame}\frametitle{Determining points of order $2$ (continues)}

% \begin{itemize}[<+-| alert@+>]
% \item If $p=2$ and $E: y^2+a_3y=x^3+a_2x^2+a_6$\pause
% 
%  \begin{beamercolorbox}[rounded=true,shadow=true,wd=10cm,center]{formul}
% $-P=(x_1,a_3+y_1)=(x_1,y_1)=P\ \pause \Longrightarrow\ a_3=0$\end{beamercolorbox}\pause\medskip
% 
% Absurd ($a_3=0$) and there are no points of order $2$.
% %$\begin{cases}
%  %  x=a_3/a_1\\
%   % y^2+a_1xy+a_3y+x^3+a_2x^2+a_4x+a_6=0
%   %\end{cases}\longrightarrow$,
% \item If $p=2$ and $E: y^2+xy=x^3+a_4x+a_6$\pause
% 
%  \begin{beamercolorbox}[rounded=true,shadow=true,wd=10cm,center]{formul}
% $-P=(x_1,x_1+y_1)=(x_1,y_1)=P\ \pause \Longrightarrow\ x_1=0,y_1^2=a_6$\end{beamercolorbox}\pause\medskip
% 
% So there is exactly one point of order $2$ namely $(0,\sqrt{a_6})$
% \end{itemize}\pause

\begin{Definition}{$2$--torsion points}
$$E[2]=\{P\in E(\overline{\F_q}): 2P=\infty\}.$$
\end{Definition}\pause

\begin{beamerboxesrounded}[upper=block title example,lower=block body alerted,shadow=true]{FACTS:}
$$E[2]\cong \begin{cases}
C_2\oplus C_2 &\text{if }p>2\\
C_2           &\text{if }p=2, E: y^2+xy=x^3+a_4x+a_6\\
\{\infty\}    &\text{if }p=2, E: y^2+a_3y=x^3+a_2x^2+a_6
\end{cases}
$$
\end{beamerboxesrounded}\pause



\begin{block}{Each curve $/\F_2$ has cyclic $E(\F_2)$.}
\centerline{\begin{tabular}{|l|c|l|}
\hline
 $E$ & $E(\F_2)$ & $|E(\F_2)|$\\
\hline
 $y^2+xy=x^3+x^2+1$ & $\{\infty,(0,1)\}$& $2$\\
\hline
$y^2+xy=x^3+1$ & $\{\infty,(0,1),(1,0),(1,1)\}$ & $4$\\
\hline
$y^2+y=x^3+x$&$\{\infty,(0,0),(0,1),(1,0),(1,1)\}$&$5$\\
\hline
$y^2+y=x^3+x+1$ &$\{\infty\}$&$1$\\
\hline
$y^2+y=x^3$ & $\{\infty,(0,0), (0,1)\}$ & $3$ \\
\hline
\end{tabular}}
\end{block}

\end{frame}

% \begin{frame}
% \frametitle{Elliptic curves over $\F_2, \F_3$ and $\F_5$}
% \begin{block}{Each curve $/\F_2$ has cyclic $E(\F_2)$.}
% \centerline{\begin{tabular}{|l|c|l|}
% \hline
%  $E$ & $E(\F_2)$ & $|E(\F_2)|$\\
% \hline
%  $y^2+xy=x^3+x^2+1$ & $\{\infty,(0,1)\}$& $2$\\
% \hline
% $y^2+xy=x^3+1$ & $\{\infty,(0,1),(1,0),(1,1)\}$ & $4$\\
% \hline
% $y^2+y=x^3+x$&$\{\infty,(0,0),(0,1),(1,0),(1,1)\}$&$5$\\
% \hline
% $y^2+y=x^3+x+1$ &$\{\infty\}$&$1$\\
% \hline
% $y^2+y=x^3$ & $\{\infty,(0,0), (0,1)\}$ & $3$ \\
% \hline
% \end{tabular}}
% \end{block}
% \pause
% \begin{itemize}
%  \item $E_1: y^2=x^3+x\qquad\qquad E_2:  y^2=x^3-x$\\
% \centerline{\begin{beamercolorbox}[shadow=true,center,rounded=true,wd=7.5cm]{formul}
% $E_1(\F_3)\cong C_4\qquad\text{and}\qquad E_2(\F_3)\cong C_2\oplus C_2$
% \end{beamercolorbox}}
% \item $E_3: y^2=x^3+x\qquad\qquad E_4: y^2=x^3+x+2$\\
% \centerline{\begin{beamercolorbox}[shadow=true,center,rounded=true,wd=7.5cm]{formul}
% $E_3(\F_5)\cong C_2\oplus C_2\qquad\text{and}\qquad E_4(\F_5)\cong C_4$
% \end{beamercolorbox}}
% \item $E_5: y^2=x^3+4x\qquad\qquad E_6: y^2=x^3+4x+1$\\
% \centerline{\begin{beamercolorbox}[shadow=true,center,rounded=true,wd=7.5cm]{formul}
% $E_5(\F_5)\cong C_2\oplus C_4\qquad\text{and}\qquad E_6(\F_5)\cong C_8$
% \end{beamercolorbox}}
% \end{itemize}
% \end{frame}

\subsection{Points of order 3}
\begin{frame}\frametitle{Determining points of order $3$}
Let  $P=(x_1,y_1)\in E(\F_q)$
\centerline{
 \begin{beamercolorbox}[rounded=true,shadow=true,wd=6cm,center]{postit}
$P$ has order $3\ \Longleftrightarrow\ 3P=\infty\ \Longleftrightarrow\ 2P=-P$
\end{beamercolorbox}}\pause\smallskip

So, if $p>3$ and $E: y^2=x^2+Ax+B$

 \begin{beamercolorbox}[rounded=true,shadow=true,wd=12cm,center]{formul}
$2P=(x_{2P},y_{2P})=2(x_1,y_1)=({\color[cmyk]{0,1,1,0.5}\lambda^2-2x_1},
{\color[cmyk]{1,0,1,0.5}-\lambda^3+2\lambda x_1-\nu})$\hfill where
$\lambda=\frac{3x_1^2+A}{2y_1}, \nu=-\frac{x_1^3-Ax_1-2B}{2y_1}$.
\end{beamercolorbox}\pause\smallskip


\centerline{
 \begin{beamercolorbox}[rounded=true,shadow=true,wd=6cm,center]{postit}
$P$ has order $3\ \Longleftrightarrow\ x_{2P}=\lambda^2-2x_1=x_1$
\end{beamercolorbox}}\pause%\smallskip

Substituting $\lambda$,\pause\
\centerline{
 \begin{beamercolorbox}[rounded=true,shadow=true,wd=6cm,center]{formul}
 $x_{2P}-x_1=\frac{-3x_1^4-6Ax_1^2-12Bx_1+A^2}{4(x_1^3+Ax_1+4B)}=0$
\end{beamercolorbox}}\pause

\begin{Note}[Conclusions]
\begin{itemize}[<+-| alert@+>]
 \item $\psi_3(x):= 3x^4+6Ax^2+12Bx-A^2$ called the $3^{\text{rd}}$ \emph{division} polynomial
 \item $(x_1,y_1)\in E(\F_q)$ has order $3\quad \Rightarrow \psi_3(x_1)=0$
 \item $E(\F_q)$ has at most $8$ points of order $3$
 \item If $p\neq 3$, $E[3]:=\{P\in E(\overline{\F_q}): 3P=\infty\}\cong C_3\oplus C_3$
 \item If $p=3$, $E: y^2=x^3+Ax^2+Bx+C$ and $P=(x_1,y_1)$
has order $3$, then
\begin{enumerate}[<+-| alert@+>]
 \item $Ax_1^3+AC-B^2=0$
 \item $E[3]\cong C_3$ if $A\neq0$ and $E[3]=\{\infty\}$ otherwise
\end{enumerate}
 \end{itemize}
 \end{Note}
\end{frame}

% \begin{frame}\frametitle{Determining points of order $3$ (continues)}
% 
% % \begin{Note} Let $E: y^2=x^3+Ax^2+Bx+C, A,B,C\in\F_{3^n}$ and let $P=(x_1,y_1)\in E(\F_{3^n})$
% % has order $3$, then
% % \begin{enumerate}[<+-| alert@+>]
% %  \item $Ax_1^3+AC-B^2=0$
% %  \item $E[3]\cong C_3$ if $A\neq0$ and $E[3]=\{\infty\}$ otherwise
% % \end{enumerate}
% % \end{Note}\pause
% 
% \begin{example}
% If $E: y^2=x^3+x+1$, then $\#E(\F_5)=9$.\pause
% $$\psi_3(x)=(x + 3)(x + 4)(x^2 + 3x + 4)$$
% Hence
% \centerline{$E[3]=\left\{
% \infty,(2,\pm1),(1,\pm\sqrt{3}),(1\alert{\pm}2\sqrt{3},\pm(1\alert{\pm}\sqrt{3}))\right\}$}\pause
% \begin{enumerate}[<+-| alert@+>]
%  \item $E(\F_5)=\{\infty,(2,\pm1),(0,\pm1),(3,\pm1),(4,\pm2)\}\cong C_9$
%  \item Since $\F_{25}=\F_5[\sqrt{3}]\quad\Rightarrow\quad  E[3]\subset E(\F_{25})$
%  \item $\#E(\F_{25})=27\quad\Rightarrow\quad E(\F_{25})\cong C_3\oplus C_9$
% \end{enumerate}
% 
% 
% \end{example}
% \end{frame}

\begin{frame}\frametitle{Determining points of order $3$ (continues)}

\begin{beamerboxesrounded}[upper=block title example,lower=block body alerted,shadow=true]{FACTS:}
$$E[3]\cong \begin{cases}
C_3\oplus C_3 &\text{if }p\ne3\\
C_3           &\text{if }p=3, E: y^2=x^3+Ax^2+Bx+C, A\neq 0\\
\{\infty\}    &\text{if }p=3, E: y^2=x^3+Bx+C
\end{cases}
$$
\end{beamerboxesrounded}\pause


\begin{block}{Example: inequivalent curves $/\F_7$ with $\#E(\F_7)=9$.}
\begin{tabular}{|l|c|c|c|}
\hline
 $E$ & $\psi_3(x)$ & $E[3]\cap E(\F_7)$ & $\!\!\!E(\F_7)\cong\!\!\!$\\
\hline
 $\!\!y^2=x^3+2\!\!$ & $x(x + 1)(x + 2)(x + 4)$ &$\!\!\!\left\{
\infty,(0,\pm3),(-1,\pm1), (5,\pm1),(3,\pm1)\right\}\!\!$
& $\!\!\!C_3\oplus C_3\!\!\!$\\
\hline
$\!\!y^2=x^3+3x+2\!\!$ & $\!\!(x + 2)(x^3 + 5x^2 + 3x + 2)\!\!$ & $\{\infty,(5,\pm3)\}$ & $C_9$ \\
\hline
$\!\!y^2=x^3+5x+2\!\!$ & $\!\!(x + 4)(x^3 + 3x^2 + 5x + 2)\!\!$ & $\{\infty,(3,\pm3)\}$ & $C_9$ \\
\hline
$\!\!y^2=x^3+6x+2\!\!$ & $\!\!(x + 1)(x^3 + 6x^2 + 6x + 2)\!\!$ & $\{\infty,(6,\pm3)\}$ & $C_9$ \\
\hline
\end{tabular}
\end{block}%\end{small}
\pause

\begin{block}
{One count the number of inequivalent $E/\F_q$ with $\#E(\F_q)=r$}
%\pause \ \hfill \alert{Answer:} \pause Next Time!!
\end{block}

\begin{example}[A curve over $\F_4=\F_2(\xi), \xi^2=\xi+1;\qquad E: y^2+y=x^3$]\pause
 We know $E(\F_2)=\{\infty, (0,0), (0,1)\}\subset E(\F_4).$\pause\\
 \begin{scriptsize}$E(\F_4)=\{\infty,(0,0),(0,1),(1,\xi),(1,\xi+1),(\xi,\xi),(\xi,\xi+1),
 (\xi+1,\xi),(\xi+1,\xi+1)\}$\end{scriptsize} \pause

\begin{small}\centerline{
\begin{beamercolorbox}[rounded=true,shadow=true,wd=7cm,center]{postit}
$\psi_3(x)=x^4+x=x(x+1)(x+\xi)(x+\xi+1)\Rightarrow E(\F_4)\cong C_3\oplus C_3$
\end{beamercolorbox}}
\end{small}
\end{example}

% \begin{Note}[Suppose $(x_0,y_0)\in E/\F_{2^n}$ has order $3$. Show that]
% \begin{enumerate}[<+-| alert@+>]
%   \item $E: y^2+a_3y=x^3+a_4x+a_6\ \Rightarrow\ x_0^4+a_3^2x_0+(a_4a_3)^2=0$
%   \item $E: y^2+xy=x^3+a_2x^2+a_6\ \Rightarrow\ x_0^4+x_0^3+a_6=0$
% \end{enumerate}
%\end{Note}

%\begin{scriptsize}
%\begin{block}{Inequivalent curves $/\F_8$ with $\#E(\F_8)=9$.}
%\begin{tabular}{|l|c|c|c|}
%\hline
% $E$ & $\psi_3(x)$ & $E[3]\cap E(\F_8)$ & $\!\!\!E(\F_8)\cong\!\!\!$\\
%\hline
% $\!\!y^2=x^3+2\!\!$ & $x(x + 1)(x + 2)(x + 4)$ &$\!\!\!\left\{\!\!\!\begin{array}{l}
%\infty,(0,\pm3),(-1,\pm1),\!\!\! \\ (5,\pm1),(3,\pm1)\end{array}\!\!\!\!\right\}\!\!$
%& $\!\!\!C_3\oplus C_3\!\!\!$\\
%\hline
%\end{tabular}
%\end{block}\end{scriptsize}

\end{frame}

\subsection{Points of finite order}

\begin{frame}\frametitle{Determining points of order (dividing) $m$}\pause
\begin{definition}[$m$--torsion point] Let $E/K$ and let $\overline{K}$ an \emph{algebraic closure of $K$}.

\centerline{\begin{beamercolorbox}[rounded=true,shadow=true,wd=5cm,center]{postit}
$E[m]=\{P\in E(\overline{K}):\ mP=\infty\}$\end{beamercolorbox}}
\end{definition}\pause

\begin{theorem}[Structure of Torsion Points]
Let $E/K$  and $m\in\N$. If $p=\operatorname{char}(K)\nmid m$,\pause

\centerline{\begin{beamercolorbox}[rounded=true,shadow=true,wd=3.5cm,center]{formul}
$E[m]\cong C_m\oplus C_m$\end{beamercolorbox}}

If $m=p^rm', p\nmid m'$,

\centerline{\begin{beamercolorbox}[rounded=true,shadow=true,wd=8cm,center]{formul}
$E[m]\cong C_m\oplus C_{m'}\qquad\text{or}\qquad E[m] \cong C_{m'}\oplus C_{m'}$\end{beamercolorbox}}
\end{theorem}\pause

\begin{block}\ \hfill
$E/\F_p$ is called $\begin{cases} \text{\emph{ordinary}} &\text{ if }E[p]\cong C_p\\
                \text{\emph{supersingular}} &\text{ if }E[p]=\{\infty\}
                    \end{cases}$\end{block}
\end{frame}

\subsection{The group structure}
\begin{frame}\frametitle{Group Structure of $E(\F_q)$}

\begin{corollary} Let $E/\F_q$. $\exists n,k\in\mathbb N$ are such that
\centerline{\begin{beamercolorbox}[rounded=true,shadow=true,wd=6cm,center]{formul}
$$E(\F_q)\cong C_n\oplus C_{nk}$$\end{beamercolorbox}}
\end{corollary}\pause

\begin{proof}
From classification Theorem of finite abelian group\\
\centerline{$E(\F_q)\cong  C_{n_1}\oplus C_{n_2}\oplus\cdots\oplus C_{n_r}$}
with $n_i|n_{i+1}$ for $i\ge1$.\pause

Hence $E(\F_q)$ contains $n_1^r$ points of order dividing $n_1$. From
\emph{Structure of Torsion Theorem}, $\#E[n_1]\le n_1^2$.
So $r\le2$\end{proof}\pause

\begin{theorem}[Corollary of Weil Pairing]  Let $E/\F_q$ and $n,k\in\mathbb N$ s.t.
$E(\F_q)\cong C_n\oplus C_{nk}.$
Then $n\mid q-1$.
\end{theorem}\pause

\ \hfil \alert{We shall discuss Weil Pairing Wednesday}
\end{frame}


\section{Division polynomials}

\begin{frame}\frametitle{The division polynomials}\pause

\begin{Definition}[Division Polynomials of $E:y^2=x^3+Ax+B$ ($p>3$)]\vspace*{-0.7cm}
\begin{align*}
        \psi_{0} =& 0\\
        \psi_{1} =& 1\\
        \psi_{2} =& 2y\\
        \psi_{3} =& 3x^{4} + 6Ax^{2} + 12Bx - A^{2}\\
        \psi_{4} =& 4y(x^{6} + 5Ax^{4} + 20Bx^{3} - 5A^{2}x^{2} - 4ABx - 8B^{2} - A^{3}) \\
        &\vdots\\
        \psi_{2m+1} =& \psi_{m+2}\psi_{m}^{3}-\psi_{m-1}\psi^{3}_{m+1} \qquad \text{ for } m \geq 2\\
        \psi_{2m}  =& \left(\frac{\psi_{m}}{2y}\right)\cdot(\psi_{m+2}\psi^{2}_{m-1}-\psi_{m-2}\psi^{2}_{m+1}) \quad \text{ for } m \geq 3
\end{align*}
The polynomial $\psi_m\in{\mathbb Z}[x,y]$ is called the $m^{\text{th}}$ \emph{division polynomial}
\end{Definition}

\begin{beamerboxesrounded}[upper=block title example,lower=block body alerted,shadow=true]{FACTS:}
\begin{itemize}
\item $\psi_{2m+1}\in\mathbb{Z}[x]\qquad\text{and}\qquad\psi_{2m}\in 2y\mathbb{Z}[x]$
\item $\psi_m=\begin{cases} y(mx^{(m^2-4)/2}+\cdots) &\text{if $m$ is even}\\
mx^{(m^2-1)/2}+\cdots &\text{if $m$ is odd.}\end{cases}$
\item $\psi_m^2=m^2x^{m^2-1}+\cdots$
\end{itemize}
\end{beamerboxesrounded}

\end{frame}

\begin{frame}
\begin{block}{Remark.}
\begin{itemize}
\item $E[2m+1]\setminus \{\infty\}= \{(x,y)\in E(\bar{K}):\  \psi_{2m+1}(x)=0\}$
\item $E[2m]\setminus E[2]= \{(x,y)\in E(\bar{K}):\  y^{-1}\psi_{2m}(x)=0\}$
\end{itemize}
\end{block}%\vspace*{-2mm}
\pause

\begin{example}%\vspace*{-.7cm}
 \begin{small}
 \begin{align*}
\psi_4(x)=&2y(x^6
 + 5 A x^4
 + 20 B x^3
 - 5 A^2 x^2
 - 4 B A x
 -A^3
 - 8 B^2)\\
 \\
 \psi_5(x)=&5 x^{12}
 + 62 A x^{10}
 + 380 B x^9
 - 105 A^2 x^8
 + 240 B A x^7
 + \left(-300 A^3
 - 240 B^2\right)  x^6
 - 696 B A^2 x^5
 + \left(-125 A^4
 - 1920 B^2 A\right)  x^4\\&
 + \left(-80 B A^3
 - 1600 B^3\right)  x^3
 + \left(-50 A^5
 - 240 B^2 A^2\right)  x^2
 + \left(-100 B A^4
 - 640 B^3 A\right)  x
 + \left(A^6
 - 32 B^2 A^3
 - 256 B^4\right)\\
\\
 \psi_6(x)=&2y(
 6 x^{16}
 + 144 A x^{14}
 + 1344 B x^{13}
 - 728 A^2 x^{12}
 + \left(-2576 A^3
 - 5376 B^2\right)  x^{10}
 - 9152 B A^2 x^9
 + \left(-1884 A^4
 - 39744 B^2 A\right)  x^8\\&
 + \left(1536 B A^3
 - 44544 B^3\right)  x^7
 + \left(-2576 A^5
 - 5376 B^2 A^2\right)  x^6
 + \left(-6720 B A^4
 - 32256 B^3 A\right)  x^5\\&
 + \left(-728 A^6
 - 8064 B^2 A^3
 - 10752 B^4\right)  x^4
 + \left(-3584 B A^5
 - 25088 B^3 A^2\right)  x^3
 + \left(144 A^7
 - 3072 B^2 A^4
 - 27648 B^4 A\right)  x^2\\&
 + \left(192 B A^6
 - 512 B^3 A^3
 - 12288 B^5\right)  x
 + \left(6 A^8
 + 192 B^2 A^5
 + 1024 B^4 A^2\right))
  \end{align*}
 \end{small}%\vspace*{-7mm}
\end{example}

\end{frame}


% \begin{frame}\frametitle{The division polynomials}
% 
% \begin{lemma} Let $E: y^2=x^3+Ax+B$, ($p>3$) and let $\psi_m\in{\mathbb Z}[x,y]$ the $m^{\text{th}}$ \emph{division polynomial}. Then
% \centerline{\begin{beamercolorbox}[rounded=true,shadow=true,wd=7cm,center]{formul}
% $\psi_{2m+1}\in\mathbb{Z}[x]\qquad\text{and}\qquad\psi_{2m}\in 2y\mathbb{Z}[x]$\end{beamercolorbox}}
% \end{lemma}\pause
% 
% 
% \begin{lemma}
% \centerline{\begin{beamercolorbox}[rounded=true,shadow=true,wd=8cm,center]{formul}
% $\psi_m=\begin{cases} y(mx^{(m^2-4)/2}+\cdots) &\text{if $m$ is even}\\
% mx^{(m^2-1)/2}+\cdots &\text{if $m$ is odd.}\end{cases}$
% \end{beamercolorbox}}
% Hence $\psi_m^2=m^2x^{m^2-1}+\cdots$
% \end{lemma}\pause
% \end{frame}

\begin{frame}
\begin{theorem}[$E: Y^2=X^3+AX+B$ elliptic curve, $P=(x,y)\in E$]\pause
\centerline{\begin{beamercolorbox}[rounded=true,shadow=true,wd=10.2cm,center]{formul}
$$\!\!\!m(x,y)=\left(x - \frac {\psi_{m-1} \psi_{m+1}}{\psi_{m}^{2}(x)}, \frac{\psi_{2 m}(x,y)}{2\psi_{m}^{4}(x)} \right)=\left ( \frac{\phi_{m}(x)}{\psi_{m}^{2}(x)}, \frac{\omega_{m}(x,y)}{\psi^{3}_{m}(x,y)} \right)
\!\!\!$$\end{beamercolorbox}}\pause
where
\centerline{\begin{beamercolorbox}[rounded=true,shadow=true,wd=8.5cm,center]{formul}
 $\phi_{m}=x\psi_{m}^{2} - \psi_{m+1}\psi_{m-1},\omega_{m}=\frac{\psi_{m+2}\psi_{m-1}^{2}-\psi_{m-2}\psi_{m+1}^{2}}{4y}$
 \end{beamercolorbox}}
\end{theorem}\pause

\begin{beamerboxesrounded}[upper=block title example,lower=block body alerted,shadow=true]{FACTS:}
\begin{itemize}
\item $\phi_m(x)=x^{m^2}+\cdots$\qquad $\psi_m(x)^2=m^2x^{m^2-1}+\cdots\in\Z[x]$
\item $\omega_{2m+1}\in y\Z[x]$, $\omega_{2m}\in\Z[x]$
  \item $\frac{\omega_{m}(x,y)}{\psi^{3}_{m}(x,y)} \in y\Z(x)$
  \item $\gcd(\psi_m^2(x),\phi_m(x))=1$
\item $E[2m+1]\setminus \{\infty\}= \{(x,y)\in E(\overline{K}):\  \psi_{2m+1}(x)=0\}$
\item $E[2m]\setminus E[2]= \{(x,y)\in E(\overline{K}):\  y^{-1}\psi_{2m}(x)=0\}$
\end{itemize}
\end{beamerboxesrounded}

\end{frame}

% \begin{frame}
% \begin{lemma}
% \centerline{\begin{beamercolorbox}[rounded=true,shadow=true,wd=8cm,center]{formul}
% $\#E[m]=\#\{P\in E(\overline{K}): mP=\infty\}\begin{cases}=m^2&\text{if }p\nmid m\\
% <m^2&\text{if }p\mid m\end{cases}$
%  \end{beamercolorbox}}
% \end{lemma}
% 
% \begin{proof} Consider the homomorphism:\\
% \centerline{\alert{$[m]:E(\overline{K})\rightarrow E(\overline{K}), P\mapsto mP$}}\pause
% 
% If $p\nmid m$, need to show that \\
% \centerline{\alert{$\#\operatorname{Ker}[m]=\#E[m]=m^2$}}\pause
% \medskip
% 
% We shall prove that
% $\exists P_0=(a,b)\in [m](E(\overline{K}))\setminus\{\infty\}$ s.t.\\
% \centerline{\alert{$\#\{P\in E(\overline{K}):\ mP=P_0\}=m^2$}}\pause
% \medskip
% 
% Since $E(\overline{K})$ infinite, we can choose $(a,b)\in [m](E(\overline{K}))$ s.t. %\vspace*{-7mm}
% \begin{enumerate}[<+->]
%   \item \alert{$ab\neq0$}
%   \item \alert{$\forall x_0\in\overline{K}:
% (\phi_m'\psi_m-2\phi_m\psi_m')(x_0)\psi_m(x_0)=0\Rightarrow a\ne \frac{\phi_{m}(x_0)}{\psi_{m}^{2}(x_0)}$}\\
% \qquad if $p\nmid m$, conditions imply that \alert{$\phi_m(x)-a\psi_m^2(x)$}\\ \qquad has
% $m^2=\partial(\phi_m(x)-a\psi_m^2(x))$ distinct roots\\
% \qquad in fact $\partial \phi_m(x)=m^2$ and $\partial\psi_m^2(x)=m^2-1$\vspace*{-5mm}
% \end{enumerate}
% \end{proof}
% 
% 
% \end{frame}

% \begin{frame}
% \begin{proof}[Proof continues]
% Write
% 
% \centerline{\alert{$mP=m(x,y)=\left(\frac{\phi_m(x)}{\psi_m^2(x)},\frac{\omega_m(x,y)}{\psi_m(x)^3}\right)=
% \left(\frac{\phi_m(x)}{\psi_m^2(x)},{y r(x)}\right)$}}\medskip\pause
% 
% The map
% 
% \alert{\centerline{$\{\alpha\in\overline{K}: \phi_m(\alpha)-a\psi_m(\alpha)^2=0\}\leftrightarrow\{P\in E(\overline{K}): mP=(a,b)\}$}}
% \alert{\centerline{$\alpha_0\mapsto(\alpha_0,br(\alpha_0)^{-1})$}}\medskip
% 
% is a well defined bijection.\medskip\pause
% 
% Hence there are $m^2$ points $P\in E(\overline{K})$ with $mP=(a,b)$\medskip\pause
% 
% So there are $m^2$ elements in $\operatorname{Ker}[m].$\medskip\medskip\pause
% 
% If $p\mid m$, the proof is the same except that $\phi_m(x)-a\psi_m(x)^2$
% has multiple roots!!
% 
% In fact $\phi_m'(x)-a\psi_m'(x)^2=0$
% \end{proof}
% \end{frame}
% 
% 
% \begin{frame}\frametitle{From Lemma, Theorem follows:}
% If $p\nmid m$, apply classification Theorem of finite Groups:
% \alert{$$E[m]\cong C_{n_1}\oplus C_{n_2}\oplus\cdots C_{n_k},$$}
% $n_i\mid n_{i+1}$. Let $\ell\mid n_1$, then $E[\ell]\subset E[m]$. Hence
% $\ell^k=\ell^2\ \Rightarrow\ k=2$. So
% \alert{$$E[m]\cong C_{n_1}\oplus C_{n_2}$$}\pause
% 
% Finally $n_2\mid m$ and $n_1n_2=m^2$ so $m=n_1=n_2$.\medskip \pause
% 
% If $p\mid m$, write $m=p^jm'$, $p\nmid m'$ and
% \alert{$$E[m]\cong E[m']\oplus E[p^j]\cong C_{m'}\oplus C_{m'}\oplus E[p^j]$$}
% 
% The statement follows from:
% 
% \centerline{\alert{$E[p^j]\cong \begin{cases}
%                   \{\infty\} \\
%                   C_{p^j}
%                 \end{cases}$}\qquad and \qquad \alert{$ C_{m'}\oplus C_{p^j}\cong C_{m'p^j}$}}\pause
% 
% which is done by induction.
% %                Proof: see \cite[page 86]{washington}
% \end{frame}
% 
% \begin{frame}\frametitle{From Lemma, Theorem follows (continues)}
% 
% Induction base:
% \centerline{\alert{$E[p]\cong \begin{cases}
%                   \{\infty\} \\
%                   C_{p}
%                 \end{cases}$}\pause\qquad \alert{if follows from } $\#E[p]<p^2$}\pause
% 
% \begin{itemize}[<+->]
% \item If \alert{$E[p]=\{\infty\}\ \Rightarrow\ E[p^j]=\{\infty\}\ \forall j\ge2$}:
% 
% In fact if $E[p^j]\neq\{\infty\}$ then it would contain some element of order $p$\pause (contradiction).
% 
% \item If \alert{$E[p]\cong C_p$, then $E[p^j]\cong C_{p^j}\ \forall j\ge2$}:
% 
% In fact $E[p^j]$ is cyclic (otherwise $E[p]$ would not be cyclic!)\pause
% 
% \centerline{\begin{beamercolorbox}[shadow=true,center,rounded=true,wd=\textwidth]{postit}
% \textbf{Fact:} \alert{$[p]: E(\overline{K})\ \rightarrow\ E(\overline{K}$)} is surjective
% (to be proven tomorrow) \end{beamercolorbox}}\pause
% 
% If $P\in E$ and $\operatorname{ord}P=p^{j-1}\ \Rightarrow\ \exists Q\in E$ s.t. $pQ=P$ and
% $\operatorname{Q}=p^{j}$.
% 
% Hence $E[p^j]\cong C_{p^j}$ since it contains an element of order $p^j$.
% \end{itemize}
% %We know that $P\in E[m]$ if and only if $(m-1)P=-P$. By the Theorem we need
% %to count the $P\in E$ such that $\psi_{m-2} \psi_{m}=0$.
% %If $m=2n+1$,

% \begin{beamerboxesrounded}[upper=block title example,lower=block body alerted,shadow=true]{Remark:}
% \begin{itemize}
% \item $E[2m+1]\setminus \{\infty\}= \{(x,y)\in E(\overline{K}):\  \psi_{2m+1}(x)=0\}$
% \item $E[2m]\setminus E[2]= \{(x,y)\in E(\overline{K}):\  y^{-1}\psi_{2m}(x)=0\}$
% \end{itemize}
% \end{beamerboxesrounded}


% \end{frame}

%Statement is correct for $n=1, 2, 3, 4$.
%$$nP=(n-1)P+_EP=(\lambda^2-x,).$$
%
%
%
% The $\ell$-torsion group of $E/\overline{\F_q}$ is isomorphic to
%$E[\ell]\cong \begin{cases}
%               C_\ell\oplus C_\ell &  \text{if }\ell\neq p\\
%C_\ell\text{ or } \{\infty\} &\text{ if }\ell=p.
%              \end{cases}$
%Hence the degree of $\psi_\ell$ is equal to either $\frac{1}{2}(\ell^2-1)$, $\frac{1}{2}(\ell-1)$, or $0$.

% \section{Important Results}
% \subsection{Hasse's Theorem}
% \begin{frame}
% \begin{theorem}[Hasse]
% Let $E$ be an elliptic curve over the finite field $\F_q$. Then the order of $E(\F_q)$
% satisfies
% $$\left|q+1-\#E(\F_q)\right|\le 2\sqrt q.$$
% \end{theorem}\pause
% 
% So \alert{$\#E(\F_q)\in [(\sqrt q -1)^2, (\sqrt q+1)^2]$} the \emph{Hasse interval} ${\mathcal I}_q$
% 
% 
%  \begin{example}[Hasse Intervals]
% \begin{scriptsize}
%  \centerline{\begin{tabular}{|l|l|}
% \hline
%  $q$ & ${\mathcal I}_q$\\
% \hline
% $2$ & $\{1, 2, 3, 4, 5\}$\\
% $3$ & $\{1, 2, 3, 4, 5, 6, 7\}$\\
% $4$ & $\{1, 2, 3, 4, 5, 6, 7, 8, 9 \}$\\
% $5$ & $\{2, 3, 4, 5, 6, 7, 8, 9, 10\}$\\
% $7$ & $\{3, 4, 5, 6, 7, 8, 9, 10, 11, 12, 13\}$\\
% $8$ & $\{4, 5, 6, 7, 8, 9, 10, 11, 12, 13, 14\}$\\
% $9$ & $\{4, 5, 6, 7, 8, 9, 10, 11, 12, 13, 14, 15, 16\}$\\
% $11$ & $\{6, 7, 8, 9, 10, 11, 12, 13, 14, 15, 16, 17, 18\}$\\
% $13$ & $\{7, 8, 9, 10, 11, 12, 13, 14, 15, 16, 17, 18, 19, 20, 21\}$\\
% $16$ & $\{9, 10, 11, 12, 13, 14, 15, 16, 17, 18, 19, 20, 21, 22, 23, 25\}$\\
% $17$ & $\{10, 11, 12, 13, 14, 15, 16, 17, 18, 19, 20, 21, 22, 23, 24, 25, 26\}$\\
% $19$ & $\{12, 13, 14, 15, 16, 17, 18, 19, 20, 21, 22, 23, 24, 25, 26, 27, 28\}$\\
% $23$ & $\{15, 16, 17, 18, 19, 20, 21, 22, 23, 24, 25, 26, 27, 28, 29, 30, 31, 32,
%  33\}$\\
% $25$ & $\{16, 17, 18, 19, 20, 21, 22, 23, 24, 25, 26, 27, 28, 29, 30, 31, 32, 33,
%  34, 35, 36\}$\\
% $27$ & $\{18, 19, 20, 21, 22, 23, 24, 25, 26, 27, 28, 29, 30, 31, 32, 33, 34, 35,
%  36, 37, 38\}$\\
% $29$ & $\{20, 21, 22, 23, 24, 25, 26, 27, 28, 29, 30, 31, 32, 33, 34, 35, 36, 37,
%  38, 39, 40\}$\\
% $31$ & $\{21, 22, 23, 24, 25, 26, 27, 28, 29, 30, 31, 32, 33, 34, 35, 36, 37, 38,
%  39, 40, 41, 42, 43 \}$\\
% $32$ & $\{22, 23, 24, 25, 26, 27, 28, 29, 30, 31, 32, 33, 34, 35, 36, 37, 38, 39,
%  40, 41, 42, 43, 44\}$\\  \hline
% \end{tabular}}
% \end{scriptsize}
% \end{example}
% 
% 
% 
% \end{frame}
% 
% \subsection{Waterhouse's Theorem}
% \begin{frame}[label=current]
% \begin{theorem}[Waterhouse]\pause
% \label{lem:Water}
%  Let $q=p^n$ and let $N = q + 1-a$.\\
% \centerline{$\exists E/\F_q\text{ s.t.}\#E(\F_q) = N\Leftrightarrow|a|\le 2\sqrt q\text{ and}$}
%  one of the following is satisfied:\pause
% \begin{itemize}[<+-| alert@+>]
% \item[(i)] $\gcd(a, p) = 1$;
% \item[(ii)] $n$ even and one of the following is satisfied:
% \begin{enumerate}
%   \item $a=\pm 2\sqrt q$;
%   \item $p\not\equiv 1 \pmod 3$, and $a = \pm\sqrt q$;
%   \item $p\not\equiv 1 \pmod 4$, and $a = 0$;
% \end{enumerate}
% \item[(iii)] $n$ is odd, and one of the following is satisfied:
%  \begin{enumerate}
%    \item $p = 2$ or $3$, and $a = \pm p^{(n+1)/2}$;
%    \item $a = 0$.
%  \end{enumerate}
%  \end{itemize}
% \end{theorem}
% 
% %\setbeamercovered{transparent}
% 
% \begin{example}[$q$ prime $\forall N\in I_q,\exists E/\F_q, \#E(\F_q)=N.$ $q$ not prime:]
% \begin{small}
% \centerline{\begin{tabular}{|l|l|}
% \hline
%  $q$ & $a\in$\\
% \hline%\vspace*{-3.12pt}
% \!\!$4=2^2$\!\! &\!\!\!\! $\{{\color<5->{green}-4},{\color<3->{green}-3},{\color<6->{green}-2},{\color<3->{green}-1},{\color<7->{green}0},{\color<3->{green}1},{\color<6->{green}2}, {\color<3->{green}3}, {\color<5->{green}4}\}$\\
% \!\!$8=2^3$\!\! &\!\!\!\! $\{{\color<3->{green}-5},{\color<9->{green}-4},{\color<3->{green}-3},-2,{\color<3->{green}-1},{\color<10->{green}0},{\color<3->{green}1},2,{\color<3->{green}3}, {\color<9->{green}4},{\color<3->{green}5}\}$\\
% \!\!$9=3^2$\!\! &\!\!\!\! $\{{\color<5->{green}-6},{\color<3->{green}-5},{\color<3->{green}-4},{\color<6->{green}-3},{\color<3->{green}-2},{\color<3->{green}-1},{\color<7->{green}0},{\color<3->{green}1},{\color<3->{green}2}, {\color<6->{green}3},{\color<3->{green}4},{\color<3->{green}5},{\color<5->{green}6}\}$\\
% \!\!$16=2^4$\!\! &\!\!\!\! $\{{\color<5->{green}-8},{\color<3->{green}-7},-6,{\color<3->{green}-5},{\color<6->{green}-4},{\color<3->{green}-3},-2,{\color<3->{green}-1},{\color<7->{green}0},{\color<3->{green}1},2,{\color<3->{green}3}, {\color<6->{green}4},{\color<3->{green}5}, 6,{\color<3->{green}7},{\color<5->{green}8}\}$\\
% \!\!$25=5^2$\!\! &\!\!\!\! $\{{\color<5->{green}-10},{\color<3->{green}-9},{\color<3->{green}-8},{\color<3->{green}-7},{\color<3->{green}-6},{\color<6->{green}-5},{\color<3->{green}-4},{\color<3->{green}-3},{\color<3->{green}-2},{\color<3->{green}-1},0,{\color<3->{green}1},{\color<3->{green}2}, {\color<3->{green}3}, {\color<3->{green}4},{\color<6->{green}5},{\color<3->{green}6},{\color<3->{green}7}, {\color<3->{green}8},{\color<3->{green}9}, {\color<3->{green}10}\}$\\
% \!\!$27=3^3$\!\! &\!\!\!\! $\{{\color<3->{green}-10},{\color<9->{green}-9},{\color<3->{green}-8},{\color<3->{green}-7},-6,{\color<3->{green}-5},{\color<3->{green}-4},-3,{\color<3->{green}-2},{\color<3->{green}-1},{\color<10->{green}0},{\color<3->{green}1},{\color<3->{green}2}, 3, {\color<3->{green}4},{\color<3->{green}5},6,{\color<3->{green}7},{\color<3->{green}8},{\color<9->{green}9},{\color<3->{green} 10}\}$\!\!\!\!\\
% \!\!$32=2^5$\!\!&\!\!\!\! $\{{\color<3->{green}-11},-10,{\color<3->{green}-9},{\color<9->{green}-8},{\color<3->{green}-7},-6,{\color<3->{green}-5},-4,{\color<3->{green}-3},-2,{\color<3->{green}-1},{\color<10->{green}0},{\color<3->{green}1},2, {\color<3->{green}3}, 4,{\color<3->{green}5}, 6, {\color<3->{green}7}, {\color<9->{green}8}, {\color<3->{green}9},10,{\color<3->{green}11}\}$\!\!\!\!\\  \hline
% \end{tabular}}\end{small}
% \end{example}
% \end{frame}
% 
% \subsection{R\"uck's Theorem}
% \begin{frame}
% \begin{theorem}[R\"uck]
% Suppose $N$ is a possible order of an elliptic curve $/\F_q$,  $q=p^n$.  Write
% 
% \centerline{
% $N = p^e n_1 n_2,\quad p\nmid n_1 n_2\quad\text{and}\quad n_1\mid n_2\ (\text{possibly }n_1 = 1).$}
% 
% There exists $E/\F_q$ s.t.
% $$E(\F_q)\cong C_{n_1}\oplus C_{n_2p^e}$$
% if and only if
% \begin{enumerate}[<+-| alert@+>]
% \item $n_1 = n_2$ in the case~(ii).1 of Waterhouse's Theorem;
% \item $n_1 |q - 1$ in all other cases of  Waterhouse's Theorem.
% \end{enumerate}
% \end{theorem}\pause
% 
% \begin{example}
% \begin{itemize}[<+->]
% \item If $q=p^{2n}$ and $\#E(\F_q)=q+1\pm2\sqrt{q}=(p^n\pm1)^2$, then
% 
% \alert{\centerline{$E(\F_q)\cong C_{p^n\pm1}\oplus C_{p^n\pm1}.$}}
% \item Let $N=100$ and $q=101\ \Rightarrow\ \exists E_1, E_2, E_3, E_4/\F_{101}$ s.t.
% 
% \alert{\centerline{$E_1(\F_{101})\cong C_{10}\oplus C_{10}\qquad E_2(\F_{101})\cong C_{2}\oplus C_{50}$}}
% 
% \alert{\centerline{$E_3(\F_{101})\cong C_{5}\oplus C_{20}\qquad E_4(\F_{101})\cong C_{100}$}}
% 
% \end{itemize}
% \end{example}
% \end{frame}


\begin{frame}
\frametitle{Further Reading...}
\begin{scriptsize}
\begin{thebibliography}{99}
\bibitem{BSS} \textsc{Ian~F.~Blake,~Gadiel~Seroussi,~and~Nigel~P.~Smart},
Advances in elliptic curve cryptography, London Mathematical Society Lecture Note Series, vol. 317, Cambridge University Press, Cambridge, 2005.
 \bibitem{C} \textsc{J.~W.~S.~Cassels},
Lectures on elliptic curves, London Mathematical Society Student Texts, vol. 24, Cambridge University Press, Cambridge, 1991.
 \bibitem{CR} \textsc{John~E.~Cremona},
Algorithms for modular elliptic curves, 2nd ed., Cambridge University Press, Cambridge, 1997.
 \bibitem{Kn} \textsc{Anthony~W.~Knapp},
Elliptic curves, Mathematical Notes, vol. 40, Princeton University Press, Princeton, NJ, 1992.
 \bibitem{Ko} \textsc{Neal~Koblitz},
Introduction to elliptic curves and modular forms, Graduate Texts in Mathematics, vol. 97, Springer-Verlag, New York, 1984.
 %\bibitem{Po} \textsc{Poonen B} Elliptic curves (introduction)(19s) notes
 \bibitem{Sil} \textsc{Joseph~H.~Silverman},
The arithmetic of elliptic curves, Graduate Texts in Mathematics, vol. 106, Springer-Verlag, New York, 1986.
\bibitem{ST} \textsc{Joseph~H.~Silverman~and~John~Tate},
Rational points on elliptic curves, Undergraduate Texts in Mathematics, Springer-Verlag, New York, 1992.
\bibitem{washington} \textsc{Lawrence~C.~Washington},
Elliptic curves: Number theory and cryptography, 2nd ED. Discrete Mathematics and Its Applications, Chapman \& Hall/CRC, 2008.
\bibitem{Zimm} \textsc{Horst~G.~Zimmer},
Computational aspects of the theory of elliptic curves, Number theory and applications
(Banff, AB, 1988) NATO Adv. Sci. Inst. Ser. C Math. Phys. Sci., vol. 265, Kluwer Acad. Publ., Dordrecht, 1989, pp. 279--324.
\end{thebibliography}
\end{scriptsize}
\end{frame}

\end{document}


