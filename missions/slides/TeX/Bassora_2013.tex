\documentclass[10pt,handout]{beamer} %,hyperref={pdfpagelabels=false},draft,handout,final
\input{bassora.sty}
%\let\Tiny=\tiny

%\includeonlyframes{current,current1,current2}

\lecture[1]{Properties of Reductions of Groups of Rational Numbers}{Artin Conjecture}
\date{October 23-24, 2013}

\def\lecturename{$2^{\text{nd}}$ International Conference of
Mathematics and its Applications- ICMA}

\title[Artin Conjecture]{\insertlecture}
\subtitle{On Artin--Gau\ss\ Conjeture}

\begin{document}

\begin{frame}
\titlepage
\end{frame}

\section{History}

\begin{frame}\frametitle{History of Artin Conjecture}
\framesubtitle{Gau\ss\ question on lengths of periods} \pause


\centerline{\begin{beamercolorbox}[shadow=true,center,rounded=true,wd=\textwidth]{formul}
What are the primes $p$ s.t. $1/p$ has length $p-1$?
\end{beamercolorbox}}\bigskip\pause


\begin{columns}[c]
\begin{column}{3cm}<1->
 \includegraphics[width=3cm]{images/Gauss_1803.jpeg}\pause
\end{column}
\begin{column}{7cm}<2->
\begin{minipage}{7cm}
\textit{For example:} \\
$\frac17=0.\overline{142857}$,\\
$\frac1{17}=0,\overline{0588235294117647}$,\\
$\frac1{19}=0.\overline{052631578947368421},$\\
$\vdots $\\
$\frac1{47}=$\scriptsize{$0.\overline{0212765957446808510638297872340425531914893617}\!\!\!\!\!$}\end{minipage}
\end{column}
\end{columns}\bigskip\pause

\begin{scriptsize}
First few primes with this property: $7, 17, 19, 23, 29, 47, 59, 61, 97, 109, 113, 131, 149, 167, 179, 181, 193,\ldots$
\end{scriptsize}\pause\bigskip

\centerline{\alert{$k_p:=$ length of the period of $1/p$}}\smallskip\medskip\pause

$k_3=1,\ k_{11}=2,\ k_{13}=6,$\hfill $k_2$ and $k_5$ are not defined 
\end{frame}


\begin{frame}\frametitle{Gau\ss\ question on lengths of periods} \pause

The period--length of the fraction $1/p$ is the least $k$ s.t. 
$$\frac1p=0.\overline{a_1\cdots a_k}=0.a_1\cdots a_k\ a_1\cdots a_k\ \ldots$$\pause

In other words
\begin{eqnarray*}
\frac1p&=&\left(\frac{a_1}{10}+\cdots+\frac{a_k}{10^{k+1}}\right)\times\left(1+\frac1{10^k}+\frac1{10^{2k}}+\cdots\right)\\
  &=&\frac{M}{10^k-1}
\end{eqnarray*}\pause

Hence $$M\times p=10^k-1$$\pause

{\alert{So $k_p$ is the least integer such that $10^k-1$ is divisible by $p$!}}
\end{frame}

	\section{facts abount period lengths}

\begin{frame}
 \frametitle{Algebraic properties of period lengths}
\begin{itemize}[<+-|alert@+>]
 \item The period length $k_p$ of $1/p$  is the least integer such that $10^k-1$ is divisible by $p$
\item Fermat Little Theorem says that $10^{p-1}-1$ is divisible by $p$
\item So $k_p\le p-1$ 
\item Indeed it is not hard to show $k_p$  is a divisor of $p-1$
\item Sometimes the period is small:
\centerline{\scriptsize{ $1/1111111111111111111=0,\overline{0000000000000000009}$}}
\item  most of the times $k_p>\sqrt{p}$\qquad not obvious!
\item Gau\ss\ in particular asked what are the frequencies of periods
\end{itemize}
\end{frame}

\begin{frame}
 \frametitle{Some statistics on period lengths:}
 
Let $k_p$ be the period length of $1/p$. The following table contains
$$\delta_m=\frac{\{p<2^{31}: k_p=\frac{p-1}{m}\}}{\#\{p\le 2^{31}\}}$$
for $m=1,\ldots, 40$.\medskip\pause

\!\!{\scriptsize
\begin{tabular}{|r|r|r|r|r|r|r|r|}
\hline
$m$       &       1&       2&       3&       4&       5&       6&       7\\ 
$\delta_m$& 0.37393& 0.28047& 0.06649& 0.07133& 0.01890& 0.04986& 0.00893\\
\hline\hline
$m$       &       8&       9&      10&      11&      12&      13&      14\\ 
$\delta_m$& 0.01660& 0.00739& 0.01416& 0.00340& 0.01268& 0.00240& 0.00669\\
\hline\hline
$m$       &      15&      16&      17&      18&      18&      20&      21\\ 
$\delta_m$& 0.00335& 0.00415& 0.00136& 0.00553& 0.00109& 0.00235& 0.00158\\
\hline\hline
$m$       &      22&      23&      24&      25&      26&      27&      28\\ 
$\delta_m$& 0.00255& 0.00073& 0.00294& 0.00075& 0.00180& 0.00081& 0.00171\\
\hline\hline
$m$       &      29&      30&      31&      32&      33&      34&      35\\ 
$\delta_m$& 0.00046& 0.00251& 0.00039& 0.00103& 0.00060& 0.00103& 0.00044\\
\hline 
%& 0.00140410& 0.000281082& 0.000821037& 0.000423635& 0.00177109& 0.000227703& 0.00119420& 0.000208720& 0.000644002& 0.000372644& 0.000554704& 0.000172678& 0.000738304& 0.000184314& 0.0204208
\end{tabular}\!\!\!\!\!\!}\medskip\pause

\begin{Note}
  $2,94\%$ of primes $p\le2^{31}$ have period $k_p=\frac{p-1}{m}$ with $m>35$
\end{Note}\pause
\end{frame}

\begin{frame}
 \frametitle{More algebraic properties of period lengths}
\begin{itemize}[<+-|alert@+>]
\item Period are also defined with respect to any base $a\in\N$
\item The period length of $1/p$ in base $a$ is the least $k_p(a)$ such that $a^k-1$ is divisile by $p$ (a divisor of $p-1$)
\item It is not difficult to see that:\\
\emph{ the period length $k_p(a)=p-1$ if and only if the set
$$\{a^j:\ j=1,\ldots,p-1\}$$
contains $p-1$ \textbf{distinct elements modulo $p$}}
\item \emph{in other words the period length $k_p(a)=p-1$ if and only if}\medskip

\centerline{$p$ is not a divisor of $a^s-a^r\quad\forall r,s:\ 1\le r<s\le p-1$}
\item we express that condition writing 
$$\langle a\bmod p\rangle=\F_p^*\quad\text{or also}\quad\#\langle a\bmod p\rangle=p-1$$
\item If the period length in base $a$ of $1/p$ is $p-1$ (i.e. $k_p(a)=p-1$), we say that \emph{$a$ is a primitive root modulo $p$}

\end{itemize}
\end{frame}


\begin{frame}
  \frametitle{Algebraic properties of period lengths}
\framesubtitle{from period lengths to primitive roots}

\begin{itemize}[<+-|alert@+>]
\item So \emph{$a$ is a primitive root modulo $p$} if and only if $\langle a\bmod p\rangle=\F_p^*$\\  (i.e.
if there are $p-1$ distinct powers of $a$ modulo $p$)
\item It is not hard to check that if $p$ is a divisor of $a$, then $1/p$ is a finite expansion in base $a$.
\item for example $1/2=0.5$\quad $1/5=0.2$ in decimal base and $1/10=0.1$ in binary base
\item the condition \emph{$a$ is a primitive root modulo $p$} makes sense also when $a$ is a rational
number and $p$ does not divide numerator and denominator of $a$ (i.e. $v_p(a)=0$)
\item \emph{$a$ is a primitive root modulo $p$} iff \medskip

\textbf{$\!\!\!\!\!\!\!\!\forall$ primes $\ell$ that divide $p-1$,
$p$ does not divide $a^{(p-1)/\ell}-1$} 
\item This is the base for Artin intuition on the\medskip

\centerline{ \emph{Primitive Roots Conjecture}}
\end{itemize}
\end{frame}


\section{Artin Conjecture}
\begin{frame}
\frametitle{Artin Conjecture (1927)}
\begin{Note} Heuristically, the probability that a prime $\ell$ is such that  both 
\begin{enumerate}
\item $\ell$ divides $p-1$
\item $p$ divides $a^{(p-1)/\ell}-1$
\end{enumerate}
are satisfied is $1/\ell(\ell-1)$.\pause

Hence the probability that $a^{(p-1)/\ell}-1$ is not divisible by $p$ for all primes $\ell$ dividing $p-1$ is
$$A=\prod_{\ell\le2}\left(1-\frac1{\ell(\ell-1)}\right)=0,373955\ldots$$
\end{Note}\pause

\begin{definition} [$A$ is called the \emph{Artin constant}]% (probably  transcendental number)
\end{definition}\pause

\begin{conj}%[Artin Conjecture]
\centerline{$lim_{x\rightarrow\infty}\frac{\#\{p\le x:\ p\ne2,5,\  \langle 10\bmod p\rangle=\F_p^*\}}{\#\{p\le x\}}= A$}
\end{conj}\pause

What if instead of $10$ we consider $a\in\Z\setminus\{-1,0,1\}$?
\end{frame}

\begin{frame}
\frametitle{Artin Conjecture (1927)}

\centerline{\includegraphics[width=3cm]{images/EmilArtin.jpg}}

\centerline{Emil Artin (March 3, 1898 - December 20, 1962)}
\pause


\begin{conj}[Artin Conjecture -- first version] If $a\in\Q\setminus\left(\{-1,0,1\}\cup\{b^2: b\in\Q\}\right)$, then
$$\#\{p\le x:\ v_p(a)=0,\  \langle a\bmod p\rangle=\F_p^*\}\sim A\pi(x)$$
\end{conj}\pause

here $\pi(x)=\#\{p\le x\}$ and $A=\displaystyle{\prod_{\ell\le2}}1-\frac1{\ell(\ell-1)}=0,37395\ldots$
\end{frame}

\begin{frame}
\frametitle{Some numerical tests for Artin Conjecture}

Let $$S_a=\{p\le 2^{29}:\ \langle a\bmod p\rangle=\F_p^*\},\quad d_a=\#S_a/\pi(2^{29})$$ 
Note that $\pi(2^{29})=28192750$ and $A=0,373955\ldots$.  \pause

\begin{center}
\begin{scriptsize}
\begin{tabular}{|c|l|l||r|l|l|}
\hline
$a$ & $S_a$ & $d_a$ & $a$& $S_a$ & $d_a$\\
\hline
-15& 10432805 &0.37005& 2& 10543421& 0.37397\\
-14& 10543340 &0.37397& 3& 10543631& 0.37398  \\
-13& 10542796 &0.37395& 5& 11098098& 0.39365    \\
-12& 12653339 &0.44881& 6& 10543607& 0.37398      \\
-11& 10639090 &0.37736& 7& 10544579& 0.37401        \\
-10& 10543135 &0.37396& 8& 6325893 & 0.22438          \\
-9 &10542743  &0.37395&10& 10542876& 0.37395            \\
-8 &6325704   &0.22437&11& 10542933& 0.37395              \\
-7 &10799148  &0.38304&12& 10545029& 0.37403\\
-6 &10543575  &0.37398&13& 10611720& 0.37639  \\
-5 &10542080  &0.37392&14& 10542946& 0.37395    \\
-4 &10543032  &0.37396&15& 10544134& 0.37400      \\
-3 &12651353  &0.44874&17& 10582932& 0.37537 \\
-2 &10542194  &0.37393&18& 10545385& 0.37404 \\\hline
\end{tabular}\end{scriptsize}
\end{center}
 \pause

Not always so totally convincing evidence!\pause

\centerline{\alert{Not convincing for $a\in\{-15, -12, -11, -8, -7, -3, 5, 8, 13, 17\}$}}
\end{frame}

\section{Lehmer's entanglement factor}

\begin{frame}\frametitle{Artin Conjecture}
\framesubtitle{Lehmer's correction}

\centerline{\includegraphics[width=2.5cm]{images/lehmer.jpg}}

\centerline{Derrick Henry Lehmer (Feb 1905 -  May 1991)}
\pause

\begin{rem}[Lehmer's Remark] The probabilities that, given two primes $\ell_1$ and $\ell_2$, a prime $p$
 is such that 
\begin{enumerate}
\item $\ell_i$ divides $p-1$
\item $p$ divides $a^{(p-1)/\ell_i}-1$
\end{enumerate}
for $i=1,2$ are not always independent!!
\end{rem}

So there is the need for a correction factor\\ \pause
 (the \emph{entanglement factor})
\end{frame}

\begin{frame}\frametitle{Artin Conjecture}
\framesubtitle{after Lehmer's correction}

\begin{conj}[Artin Conjecture -- final form] Let $a\in\Q^*\setminus\{1,-1\}$, then $p-1=\#\langle a\mod p\rangle$
 for a proportion of primes $\delta_a$ where
 $$\delta_a=r_a\times t_a,$$
 where if $h=\max\{j: a=b^j,b\in\Q\}$, $\partial(a)=\operatorname{disc}(\Q(\sqrt{a}))$,
 $$t_a=\prod_{\ell\ge2}\left(1-\frac{\gcd(h,\ell)}{\ell(\ell-1)}\right)$$
 and $r_a=1$ unless if $\partial(a)$ is odd in which case:\\
\centerline{ $r_a=1-\prod_{\ell\mid \partial(a)}\frac{-1}{\ell(\ell-1)/\gcd(\ell,h)-1}$}
\end{conj}

Note that
\begin{itemize}[<+-|alert@+>]
\item $t_a$ is a rational  multiple of the Artin Constant $A$
\item  $\delta_a=0$ iff $a$ is a perfect square
\item  $\partial(a)$ is easy but technical to define
\end{itemize}
\end{frame}

\begin{frame}
\frametitle{Artin Conjecture}
\framesubtitle{Effect of the Lehmer entanglement}

We were not convinced for  $a\in\{-15, -12, -11, -8, -7, -3, 5, 8, 13, 17\}$\pause


\begin{center}
\begin{tabular}{|c|l|l|}
\hline
$a$ & $\delta_a$ & $d_a$ \\
\hline
-15&0.37001 &0.37005\\
-12&0.44875  &0.44881     \\
-11&0.37709 &0.37736\\
-8 &0.22437  &0.22437\\
-7 &0.38308  &0.38304\\
-3 &0.44875  &0.44874\\
 5&0.39363 & 0.39365    \\
 8&0.22437 & 0.22438          \\
13 &0.37636& 0.37639  \\
17 &0.37533& 0.37537 \\
\hline
\end{tabular}\end{center}\pause

For all other values of $a$ in the previous table, $\delta_a=A$

\end{frame}

\section{Hooley's result}

\begin{frame}\frametitle{Artin Conjecture}
\framesubtitle{what it is known on Artin Conjecture}

\begin{theorem}[C. Hooley (1965)] If the Generalized Riemann Hypothesis (GRH)  holds for the fields $\Q(a^{1/\ell})$ ($\ell$ prime) then the modified Artin Conjecture holds for $a$
\end{theorem}\pause

What is the GRH?\pause

\begin{itemize}[<+-|alert@+>]
\item It is a complicated conjecture in Number Theory, so important that it often assumed as an Hypothesis
\item Stating it is behind the scope of this seminar
\item It has many different formulations:
\item \emph{all the non trivial zeroes of the Dedekind zeta functions sit on the line $\Re s=1/2$}
\item \emph{primes can be counted very precisely}
\end{itemize}

\end{frame}

\section{the Quasi Resolution}

\begin{frame}\frametitle{Artin Conjecture}
\framesubtitle{The quasi resolution}\pause


\centerline{
\includegraphics[width=2.5cm]{images/gupta.jpg}
\
\includegraphics[width=2.5cm]{images/murty.jpg}
\
\includegraphics[width=2.5cm]{images/heathbrown.jpg}
}\pause


\begin{theorem}[R. Gupta, R.  Murty \& R. Heath--Brown (1984/86)] $\exists g\in\{2,3,5\}$ s.t.
 $$\#\{p\le x:\ p>5, \langle g\bmod p\rangle=\F_p^*\}\gg\frac{\pi(x)}{\log x}$$
\end{theorem}
\end{frame}

\section{A new result}
 \begin{frame}
 \frametitle{The higher rank Artin Quasi--primitive root Conjecture}
\framesubtitle{joint work with Andrea Susa}

Notations:\pause

\begin{itemize}[<+-|alert@+>]
\item  $\Gamma\subset\Q^*$ finitely  generated  subgroup
\item $r$ rank of $\Gamma$
\item   $m \in \N^+$
\item  $\sigma_\Gamma=\prod_{p: v_p(x)=0,\exists x \in \Gamma}p$
\item   $\forall p\nmid\sigma_\Gamma$
$$\Gamma_p =\{g(\bmod{p}): g\in\Gamma\}\subset\F_p^*$$ 
is well defined
\item
$N_\Gamma(x,m) := \#\{p\leq x: p\nmid\sigma_\Gamma, |\Gamma_p| =\frac{p-1}m\}$
\item $\Gamma_p$ generalizes the notion of $\langle a\bmod p\rangle$.
\item if $\Gamma=\langle a\rangle$ has rank $1$ then\\
$$N_\langle a\rangle(x,m) =\#\{p\leq x: \frac1p\text{ has period of length } \frac{p-1}m\}$$
\end{itemize}
\end{frame}

 \begin{frame}
 \frametitle{The higher rank Artin Quasi--primitive root Conjecture}
\framesubtitle{joint work with Andrea Susa}

\begin{Theorem} Let $\Gamma \subset \Q^*$ has rank $r\ge2$,  let $m\in\N$ and  
assume GRH holds for $\Q(\zeta_{k},\Gamma^{1/k})$ ($k\in\N$). 
Then, $\forall\epsilon>0$ and 
$m\le x^{\frac{r-1}{(r+1)(4r+2)}-\epsilon}$,\pause
$$N_{\Gamma}(x,m)=\left(\rho(\Gamma,m)+O\!\left(\frac{1}{\varphi(m^{r+1})\log^rx} %+x^{\frac{-9r+11}{20r}}\log^2 x
\right)\right) \pi(x),$$
where\pause
$$\rho(\Gamma,m)= \sum_{ k \geq 1}
\frac{\mu(k)}{[\Q(\zeta_{mk},\Gamma^{1/mk}):\Q]}.
$$
\end{Theorem}\pause

An analogue of the above result holds also in the case when $\Gamma\subset\Q^*$ has 
infinite rank.
\end{frame}
 
 \begin{frame}
 \frametitle{The $r$--rank Artin Quasi--primitive root Conjecture}
\framesubtitle{joint work with Andrea Susa}

\begin{Theorem}
 Let $\Gamma \subset \Q^+=\{q\in\Q; q>0\}$  with rank $r\ge2$ and $m\in\N$. Let $\Gamma(m):=\Gamma(\Q^*)^m/(\Q^*)^m$,\medskip

{\scriptsize{
 \centerline{$\displaystyle{A_{\Gamma,m}=\frac1{\varphi(m)|\Gamma(m)|}\times
\prod_{\substack{\ell>2\\ \ell\nmid
m}}\left(1-\frac1{(\ell-1)|\Gamma(\ell)|}\right)
\times\prod_{\substack{\ell>2\\ \ell\mid
m}}\left(1-\frac{|\Gamma(\ell^{v_\ell(m)})|}{\ell|\Gamma(\ell^{1+v_\ell(m)})|}\right)
}$}}}\medskip

and \medskip

{\scriptsize{
 \centerline{$\displaystyle{ B_{\Gamma,k} =\sum_{\substack{
\eta\mid\sigma_\Gamma\\
 \eta^{2^{v_2(k)-1}}\!\!\!\cdot{\Q^*}^{2^{v_2(k)}}\in\Gamma(2^{v_2(k)})\\ 
v_2(\partial(\eta))\le k}}\prod_{\substack{\ell\mid \partial(\eta)\\
\ell\nmid k}}\frac{-1}{(\ell-1)|\Gamma(\ell)|-1}.
}$}}}

Then
$$\rho(\Gamma,m) = A_{\Gamma,m}\left(  B_{\Gamma,m}
-\frac{|\Gamma(2^{v_{2}(m)})|}{(2,m)|\Gamma(2^{1+v_{2}(m)})|} B_{\Gamma,2m}\right).$$
\end{Theorem}
\end{frame}
 
 
\begin{frame}
\frametitle{The Artin Quasi--primitive root Conjecture}
\framesubtitle{vanishing of the density}

\begin{Theorem}\label{finite} Let $\Gamma\subset\Q^+$ fin. gen.,  $m\in\N$. Then
\\
\centerline{$\rho(\Gamma,m)=0$}

if one of the following holds:
\begin{enumerate}
 \item $2\nmid m$ and for all $g\in\Gamma, \partial(g)\mid m$;
 \item $2\mid m$, $3\nmid m$, $\Gamma(3)=\{1\}$ and $\exists \eta\mid\sigma_\Gamma,$
\hspace*{4cm} \begin{minipage}{5cm}\begin{itemize}
 \item $\eta^{2^{v_2(m/2)}}\!\!\!\!\cdot{\Q^*}^{2^{v_2(m)}}\in\Gamma(2^{v_2(m)})$
 \item $\partial(-3\eta)\mid m$ %$v_2(\partial(\eta))\le m$ s.t. $3$ is the only odd prime that divides $\partial(\eta)$ and that doesn't divide $m$. 
 \end{itemize}\end{minipage}
 \end{enumerate}
(if $2\nmid m$, (1) is also necessary for $\rho(\Gamma,m)=0$).
If $\Gamma\subset\Q^+$ and $m$ satisfy one of (1) or (2) above, then\\
\centerline{$\{p: \text{ind}_p\Gamma=m\}\quad\text{finite}.$}

Hence, on GRH, if $2\nmid m$,\\
\centerline{$\{p: \text{ind}_p\Gamma=m\}\quad\text{finite} \Longleftrightarrow \forall g\in\Gamma, \partial(g)\mid m.$}
\end{Theorem}
\end{frame}

\end{document}
