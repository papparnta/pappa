\documentclass[landscape,handout]{powersem} %
\usepackage{fancybox,marvosym,graphicx,amsmath,amssymb,pifont,textcomp}
\usepackage[bookmarksopen,colorlinks,urlcolor=red,pdfpagemode=FullScreen]{hyperref}
\usepackage{fixseminar}
\usepackage[usenames,dvipsnames]{color}
\usepackage[latin1]{inputenc}
%\usepackage{eurosans}
\usepackage[%coloremph
,colormath%,colorhighlight
,whitebackground]{texpower}
\hfuzz=30pt
\vfuzz=30pt
\setlength{\slidewidth}{25cm} \setlength{\slideheight}{17cm}
\slideframe{}%shadow
\def\slideitemsep{.5ex plus .3ex minus .2ex}
\renewcommand{\slidetopmargin}{10mm}
\renewcommand{\slidebottommargin}{15mm}
\renewcommand{\slideleftmargin}{5mm}
\renewcommand{\sliderightmargin}{5mm}
\newcommand{\psd}{\pause}%\addtocounter{slide}{-1}}
\newcommand{\Ccal}{{\mathcal{C}}}
\newcommand{\Exp}{\operatorname{Exp}}
\newcommand{\F}{{\mathbb{F}}}
\newcommand{\C}{{\mathbb C}}
\newcommand{\Q}{{{\mathbb Q}}}
\newcommand{\Z}{{\mathbb Z}}
\newcommand{\N}{{\mathbb N}}
\newcommand{\manorossa}{\textcolor{conceptcolor}{\ding{43}}}
\newcommand{\matitablu}{\textcolor{altemcolor}{\ding{46}}}
\newcommand{\verde}{\textcolor{black}}
\newcommand{\heading}[1]{%
 \begin{center}
  %\large\bf
  \Ovalbox{{#1}}%\textcolor{conceptcolor}{
 \end{center}
 \vspace{1ex minus 1ex}}
\definecolor{verdescu}{rgb}{0,0.6,0.6}
\definecolor{rossoscu}{rgb}{1,0,0.2}

%\backgroundstyle[startcolor=white,
 %                  endcolor=inactivecolor,%firstgradprogression=3,
  %          rightpanelwidth=-7\semcm,,rightpanelcolor=pagecolor]{hgradient}%
%%%%%%%%%%%%% DATI DEL SEMINARIO IN QUESTIONE %%%%%%%%%%%%

\newpagestyle{327}%
 {\textcolor{codecolor}{\textit{Basic Algorithms in Number Theory}} \hspace{\fill}\rightmark
\hspace{0.5cm}\thepage}
 {}%{\includegraphics[width=4mm]{images/dipmat.pdf}\hspace{\fill}\textcolor{codecolor}{\sc Universit\`a Roma Tre}
 %\hspace{\fill}\includegraphics[width=5mm]{images/roma3.pdf}}%%
\pagestyle{327} \markright{\textcolor{conceptcolor}{Algorithmic Complexity ...}}

\begin{document}

%\begin{slide}\pageTransitionWipe{30}
%\maketitle
%\end{slide}

\begin{slide}
\includegraphics[width=1.6cm]{images/roma3.pdf}\hfill\includegraphics[width=1.9cm]{images/photo_erbil.jpg}
\vfill

\begin{center}\begin{sc}
\begin{Large}

\textcolor{underlcolor}{Basic Algorithms in Number Theory}
\end{Large}\bigskip

\ {Francesco Pappalardi}\bigskip\bigskip

\begin{large}\begin{bf}\#2 - Discrete Logs, Modular Square Roots \& Euclidean Algorithm.
\end{bf}\end{large}\medskip

December $3^{\textrm{rd}}$ 2014\medskip
\vfil\end{sc}\end{center}
\begin{it}
\ \hfill Mathematics Department

\ \hfill College of Sciences

\ \hfill University of Salahaddin, Erbil
\end{it}

\vfill
\end{slide}

\begin{slide}
\heading{\textsc{Yesterday's Problems}}
\begin{enumerate}
	\item \textsc{Multiplication:} for $x,y\in\Z$, find $x\cdot y$
	\item \textsc{Exponentiation:} for $x\in G$ (group) and $n\in\N$, find $x^n$ (Complexity of operations in $\Z/m\Z$)
	\item \textsc{GCD:} Given $a,b\in\N$ find $\gcd(a,b)$
	\item  \textsc{Primality:} Given $n\in\N$ odd, determine if it is prime (Legendre/Jacobi Symbols - Probabilistic Algorithms with probability of error)
	\item \textsc{Quadratic Nonresidues:} given an odd prime $p$, find
a quadratic non residue mod $p$.
	\item \textsc{Power Test:} Given $n\in\N$ determine if $n=b^k (\exists k>1)$
\end{enumerate}
\end{slide}

\begin{slide}

\heading{\textcolor{red}{\textbf{PROBLEM 7.}} \textsc{Factoring:} Given $n\in\N$, find a proper divisor of $n$}\pause
\parstepwise{\begin{itemize}
 \item \step{A very old problem and a difficult one;}
\item \step{Trial division requires $O(\sqrt n)$ division which is an exponential time}\\ \step{(i.e. impractical)}
\item \step{Several different algorithms}
\item \step{A very important one uses \emph{elliptic curves}$\ldots$}%\\
%\step{\textit{see next week course by J. Jimenez Urroz}.}
\item \step{we review the elegant Pollard $\rho$ method.}
\end{itemize}}
{Suppose $n$ is not a power and consider the function:}\pause
\centerline{$f:\Z/n\Z\longrightarrow\Z/n\Z,\quad x\mapsto f(x)=x^2+1.$}

The $k$-th iterate of $f$ is $f^k(x)=f^{k-1}(f(x))$ with $f^1(x)=f(x)$.

 If $x_0\in\Z/n\Z$ is chosen ``sufficiently
randomly'',  the sequence $\{f^{k}(x_0)\}$ behaves as a random sequence of elements of $\Z/n\Z$ and we exploit
this fact.

\end{slide}


\begin{slide}
\heading{Pollard $\rho$ factoring method}
\begin{center}\fbox{\textcolor{black}{
\begin{minipage}[c]{11cm}
\texttt{\noindent 
\noindent 
\textcolor{red}{Input:}  $n\in\N$ odd and not a perfect power (to be factored)\\
\textcolor{blue}{Output:}  a non trivial factor of $n$\\
1. Choose at random $x\in\Z/n\Z=\{0,1,\ldots,n-1\}$\\
2. For $i=1,2\ldots$.\\
\hspace*{1cm} \qquad $g:=\gcd(f^i(x)-f^{2i}(x),n)$\\
\hspace*{1cm} \qquad If $g=1$, goto next $i$\\  
\hspace*{1cm} \qquad If $1<g<n$ then output $g$ and halt\\
\hspace*{1cm} \qquad If $g=n$ then go to Step 1 and choose another $x$.}
\end{minipage}}}
\end{center}
\pause%\vspace*{-3mm}
What is going on here?\pause%\vspace*{-3mm}
Is is obviously a probabilistic algorithm but it is not even clear that it will ever terminate.

But in fact it terminates with complexity $O(\sqrt[4]n)$ which is attained with high probability, in the worst case (i.e.
when $n$ is an RSA module) %(for RSA see course in Cryptography by K. Chakraborty).
\end{slide}

\begin{slide}
\heading{\textsc{The birthday paradox}}

\noindent\textbf{Elementary Probability Question:} \textit{what is the chance that in a sequence of $k$ elements (where
repetitions are allowed) from a set of $n$ elements, there is a repetition?} \pause

\noindent\textit{Answer:}  The chance is $\displaystyle{1-\frac{n!}{n^k(n-k)!}\approx 1-e^{-k(k-1)/2n}}$

\fbox{In a party of $23$ friends there $50.04\%$ chances that $2$ have the same birthday!!}

Relevance to the $\rho$-Factoring method:
\begin{center}
\fbox{\begin{minipage}[l]{12cm}If $d$ is a divisor of $n$, then in $O(\sqrt{d})=O(\sqrt[4]{n})$ steps there is a high chance that in the sequence 
$\{f^{k}(x_0)\bmod d\}$ there is a repetition modulo $d$.\end{minipage}}
\end{center}

\noindent\textsc{Remark (WHY $\rho$).} If $y_1,\ldots,y_m,y_{m+1},\ldots,y_{m+k}=y_m,y_{m+k+1}=y_{m+1},\ldots$. and $i$ is the smallest
multiple of $k$ with $i\ge m$, then $y_i=y_{2i}$ (the Floyd's cycle trick). 

\end{slide}

\begin{slide}
\heading{\textsc{Contemporary Factoring}}

Contemporary records in factoring are obtained by the \emph{Number Field Sieve} (NFS) which is an evolution of the \emph{Quadratic Sieve} (QS). 
These (together with the ECM-factoring) have sub-exponential heuristic complexity.\pause%\vspace*{-3mm}

More precisely let:
$$L_n[a;c]=\exp\left(((c+o(1)(\log n)^a(\log\log n)^{1-a})\right).$$
which is a quantity that oscillates between exponential $(a=1)$ and polynomial $(a=0)$ as
a function of $\log n$. Then the complexities are respectively\pause

\begin{itemize}
 \item[\textbf{ECM}] algorithm with heuristic complexity $L_n[1/2,1]$ \hfill (Lenstra 1987) 
\item[\textbf{NFS}] algorithm with heuristic complexity $L_n[1/3;4/3^{3/2}]$ \hfill (Pollard) 
\item[\textbf{QS}] algorithm with heuristic complexity $L_n[1/2,1]$ \hfill (Dickson, Pomerance) 
\end{itemize}
 

\end{slide}

\begin{slide}

\heading{\textcolor{red}{\textbf{PROBLEM 8.}} \textsc{Discrete Logarithms:}} 
 Given $x$ in a cyclic group $G=\langle g\rangle$, find $n$ such that $x=g^n$.

\parstepwise{\begin{itemize}
 \item \step{to make sense one has to specify how to make the operations in $G$}
\item \step{If $G=\left(\Z/n\Z,+\right)$ then discrete logs are very easy.}
\item \step{If $G=((\Z/n\Z)^*,\times)$ then we know that $G$ is cyclic iff $n=2,4,p^\alpha,2\cdot p^\alpha$}\\
\step{ where
$p$ is an odd prime. This is a famous theorem of Gau\ss.}
\item \step{Already in $(\Z/p\Z)^*$ there is no efficient algorithm to compute DL.}
\item \step{It is already an interesting problem, given $p$, to compute a primitive root}\\ \step{ $g$ modulo $p$ (i.e. to determine $g\in(\Z/p\Z)^*$
such that $\langle g\rangle=(\Z/p\Z)^*$)}
\item \step{The famous \emph{Artin Conjecture for primitive roots} stated that any $g$}\\\step{ (except $0,\pm1$ and perfect squares)
is a primitive root for a positive}\\\step{ proportion of primes}
\item \step{Known to be true assuming the GRH. It is also known that one out of}\\ \step{$2, 3$  and $5$ is a primitive root for infinitely many primes.} 
\end{itemize}}
\end{slide}

\begin{slide}
\heading{\textcolor{red}{\textsc{Discrete Logarithms:}} continues}\vspace*{-2mm} 
\parstepwise{\begin{itemize}
 \item \step{Primordial public key cryptography is based on the difficulty of the }\\ \step{Discrete Log problem}
\item \step{Several algorithms to compute discrete logarithms are known.}\\ \step{One for all is the \textbf{Shanks Baby Step Giant Step algorithm}.}
\end{itemize}}\vspace*{-1mm}
\begin{center}\fbox{\textcolor{black}{
\begin{minipage}[c]{11cm}
\texttt{\noindent 
\noindent 
\textcolor{red}{Input:}  A group $G=\langle g\rangle$ and $a\in G$\\
\textcolor{blue}{Output:}  $k\in\Z/|G|\Z$ such that $a=g^k$\\
1. $M:= \lceil \sqrt{|G|}\rceil$\\
2. For $j=0,1,2,\ldots,M$.\\
\hspace*{1cm} \qquad Compute $g^j$ and store the pair $(j, g^j)$ in a table\\
3. $A:=g^{-M}$, $B:=a$\\
5. For $i=0,1,2,\ldots,M-1$.\\
\hspace*{5mm} \qquad -1- Check if $B$ is the second component $(g^j)$ of any\\ \hspace*{1.3cm} \qquad pair in the table\\
\hspace*{5mm} \qquad -2- If so, return $iM + j$ and halt.\\  
\hspace*{5mm} \qquad -3- If not $B=B\cdot A$}
\end{minipage}}}
\end{center}
% Input: A cyclic group G of order n, having a generator α and an element β.
% 
% Output: A value x satisfying αx = β.
% 
%    1. m ← Ceiling(√n)
%    2. For all j where 0 ≤ j < m:
%          1. Compute αj and store the pair (j, αj) in a table. (See section "In practice")
%    3. Compute α−m.
%    4. γ ← β. (set γ = β)
%    5. For i = 0 to (m − 1):
%          1. Check to see if γ is the second component (αj) of any pair in the table.
%          2. If so, return im + j.
%          3. If not, γ ← γ • α−m.
\end{slide}

\begin{slide}
\heading{\textcolor{red}{\textsc{Discrete Logarithms:}} continues} 

\parstepwise{\begin{itemize}
 \item \step{The BSGS algorithm is a generic algorithm.}\\ \step{It works for every finite cyclic group.}
\item \step{It is based on the fact that any $x\in\Z/n\Z$ can be written as $x=j+ i m$}\\
\step{with $m=\lceil\sqrt{n}\rceil$, $0\le j<m$ and $0\le i<m-1$}
 \item  \step{It is not necessary to know the order of the group $G$ in advance.}\\ 
\step{The algorithm still works if an upper bound on the group order is known.}
\item \step{Usually the BSGS algorithm is used for groups whose order is prime.}
\item \step{The running time of the algorithm and the space complexity is $O(\sqrt{|G|})$,}\\ \step{much better than the $O(|G|)$ 
running time of the naive brute force}
\item \step{The algorithm was originally developed by Daniel Shanks.}
\end{itemize}}
\end{slide}

\begin{slide}
\heading{\textcolor{red}{\textsc{Discrete Logarithms:}} continues} 

 In some groups Discrete logs are easy. For example if $G$ is a cyclic group and $\#G=2^m$ then we know
that there are subgroups:\pause\vspace*{-7mm}
$$\langle1\rangle=G_0\subset G_1\subset\cdots\subset G_m=G$$\pause\vspace*{-3mm}
such that $G_i$ is cyclic and $\#G_i=2^i$. Furthermore\vspace*{-2mm}
$$G_i=\left\{y\in G\text{ such that } y^{2^i}=1\right\}.$$\pause\vspace*{-3mm}
Hence if $G=\langle g\rangle$, for any $a\in G$, either $a^{2^{m-1}}=1$
or $(ga)^{2^{m-1}}=1$\pause
From this property we deduce the algorithm:%\vspace*{-2mm}
\begin{center}\fbox
{\textcolor{black}{
\begin{minipage}[c]{9cm}
\texttt{\noindent 
\textcolor{red}{Input:}  A group $G=\langle g\rangle$, $|G|=2^m$ and $a\in G$\\
\textcolor{blue}{Output:}  $k\in\Z/|G|\Z$ such that $a=g^k$\\
1. $A:=a$, $K=2^m$\\
2. For $j=1,2,\ldots, m$.\\
\hspace*{1cm} \qquad If $A^{2^{m-j}}\neq 1$, $A:=g^{2^{j-1}}\cdot A; K:=K-2^{j-1}$\\
3 Output $K$}
\end{minipage}}}
\end{center}
\end{slide}

\begin{slide}
\heading{\textcolor{red}{\textsc{Discrete Logarithms:}} continues} 
\parstepwise{
\begin{itemize}
\item \step{The above is a special case of the Pohlig-Hellman Algorithm which works}\\ \step{when $|G|$ has only small prime divisors}
\item \step{To avoid this situation one crucial requirement for a DL-resistent group}\\ \step{in cryptography is that $\#G$ has a large prime divisor.}
\item \step{If $p=2^k+1$ is a Fermat prime, then DL in $(\Z/p\Z)^*$ are easy.}
\item \step{Classical algorithm for factoring have often analogues for computing}\\ \step{discrete logs. A very important one  is the \emph{index calculus algorithm}.}
\end{itemize}}
\end{slide}

\begin{slide}

\heading{\textcolor{red}{\textbf{PROBLEM 9.}} \textsc{Square Roots Modulo a prime:}} 

\fbox{Given an odd prime $p$ and a quadratic residue $a$, find $x$
s. t. $x^2\equiv a\bmod p$}\pause

It can be solved efficiently if we are given a quadratic nonresidue $g\in(\Z/p\Z)^*$\pause

\parstepwise{\begin{enumerate}
\item \step{We write $p-1=2^k\cdot q$ and we know that $(\Z/p\Z)^*$ has a (cyclic)}\\ \step{subgroup $G$ with $2^k$ elements.}
\item \step{ Note that $b=g^q$ is a generator of $G$
(in fact if it was $b^{2^j}\equiv1\bmod p$}\\ \step{for $j<k$, then $g^{(p-1)/2}\equiv1\bmod p$) and that $a^q\in G$}
\item \step{Use the last algorithm to compute $t$ such that $a^q=b^t$. Note that $t$ is}\\ \step{even since
$a^{(p-1)/2}\equiv1\bmod p$.}
\item \step{Finally set $x=a^{(p-q)/2}b^{t/2}$ and observe that}\\
\step{$\hspace*{3cm}\displaystyle{x^2=a^{(p-q)}b^{t}=a^p\equiv a\bmod p.}$}
\end{enumerate}}\pause

The above is not deterministic. However Schoof in 1985 discovered a polynomial time algorithm which is
however not efficient.
\end{slide}


\begin{slide}

\heading{\textcolor{red}{\textbf{PROBLEM 10.}} \textsc{Modular Square Roots:}}

\centerline{\fbox{Given $n,a\in\N$, find $x$ such that $x^2\equiv a\bmod n$}}\pause

If the factorization of $n$ is known, then this problem (efficiently) can be solved in 3 steps:
\parstepwise{\begin{enumerate}
 \item \step{For each prime divisor $p$ of $n$ find $x_p$ such that $x_p^2\equiv a \bmod p$}
\item \step{Use the Hensel's Lemma to lift $x_p$ to $y_p$ where $y_p^2\equiv a\bmod p^{v_p(n)}$}
\item \step{Use the Chinese remainder Theorem to find $x\in\Z/n\Z$ such that}\\ \step{$x\equiv y_p\bmod p^{v_p(n)} \ \forall p\mid n$.}
\item \step{Finally $x^2\equiv a\bmod n$.}
\end{enumerate}}\pause

The last two tools (Hensel's Lemma and Chinese Remainder Theorem) will be covered in Lecture 3.
 \end{slide}


\begin{slide}

\heading{ \textsc{Modular Square Roots:}\quad (continues)}  

On the opposite direction, suppose that for each $a\in\Z/n\Z$ we can solve $X^2\equiv a\bmod n$.
We want to use this hypothetical algorithm to find a factor of $n$.\pause

Choose $y$ at random in $\Z/n\Z$ and find $x$ such that $x^2\equiv y^2\bmod n$.\pause

Any common divisor of $x$ and $y$ also divides $n$. So we can assume that $x$ and $y$ are coprime.\pause

If $p>1$ is a prime factor of $n$, then $p$ divides $(x+y)(x-y)$. In addition $p$ divides exactly one
of the factors $(x+y)$ or $(x-y)$.\pause

If $y$ is random, then any of the primes that divides $x^2-y^2$ has $50\%$ chances of $x+y$ of $x-y$.\pause

Finally $\gcd(x-y,n)$ is a proper divisor of $n$. \pause

If the above fails, then try again choosing a different random $y$. After $k$ choices, the probability
that $n$ is not factored is $O(2^{-k})$.
 \end{slide}


\begin{slide}

\heading{ \textsc{Modular Square Roots:}\quad (continues)}  

The \textsc{Factoring} and \textsc{Modular square roots} are in practice equivalent in difficulty.\pause\bigskip\bigskip

The difficulty of solving the analogue problem for $e$--th roots modulo $n$ 
$$\textbf{i.e. Given $e, C, n$, find $x\in\Z/n\Z$ such that }x^e\equiv C\bmod n$$ \pause

is the base of the security of RSA

\end{slide}

\begin{slide}
\heading{\textcolor{red}{\textbf{PROBLEM 11.}} \textsc{Diophantine Equations:}}
\pause
  
\textbf{PROBLEM 11.} \textsc{Diophantine Equations:} \textit{Given $f(X_1,\ldots,X_n)\in\Z[X_1,\ldots,X_n]$, find $x=(x_1,\ldots,x_n)\in\Z^n$
such that $f(x)=0$.}

For a general $f$ this is an \text{undecidable problem} (Matijasevic, Robinson, Davis, Putnam).
\pause

Although the problem might be easy for some specific $f$, there is no algorithm (efficient or otherwise) that
takes $f$ as input and always determines whether $f(x)=0$ has a solution in integers.\pause

Hilbert's tenth problem is the tenth on the list of Hilbert's problems of 1900.

\begin{quote}\textit{Given a Diophantine equation with any number of unknown quantities and with rational integral numerical coefficients: 
To devise a process according to which it can be determined in a finite number of operations whether the equation is solvable in rational integers.}\end{quote}

\end{slide}



\begin{slide}
\centerline{\includegraphics[width=11cm]{images/School_of_Athens.jpeg}}\vspace*{-8cm}
\parstepwise{
\step{
      \hspace*{-1mm}\colorbox{white}{\shadowbox{La Scuola di Atene (Raffaello Sanzio)}}
      }
\bigskip\bigskip\bigskip\bigskip\\ %\vspace*{1cm}\hspace*{2cm}
\step{\hspace*{3cm}
      \begin{minipage}{5cm}
                      \shadowbox{\includegraphics[width=3.5cm]{images/Euclid_7.jpeg}}\\
                      \colorbox{white}{\begin{minipage}[c]{4cm}
                               \ \ \ Euclide di Alessandria\vspace*{-2mm}\\
                                 {\tiny Birth: 325 A.C. (approximately)}\vspace*{-2.5mm}\\
                                 {\tiny Death: 265 A.C. (approximately)}
                                \end{minipage}}\\
       \end{minipage}
       }\\ \bigskip\medskip% \bigskip\\
\step{\hspace*{1.5cm}
      \colorbox{white}{\shadowbox{The Euclidean Algorithm}}
      }
            }
\end{slide}

\begin{slide}
\heading{Extended Euclidean Algorithm}

Let $a,b\in\N$ (not both zero), we will also assume that $a\geq b$. The $\gcd(a,b)$ is greatest common divisor of $a$ and $b$.\pause
Clearly $\gcd(a,0)=a$. If the factorization of $a$ and $b$ is known the it is easy to compute $\gcd(a,b)$. In fact
$$\gcd(a,b)=\prod_{p\ \text{prime}}p^{\min\{v_p(a),v_p(b)\}}.$$\pause
The $p$--adic valuation $v_p(n)$ of an integer $n$ is
$$v_p(n)=\max\{\alpha\ge0\text{ such that $p^\alpha$ divides $n$}\}$$
so that the product above is indeed finite.\pause

Furthermore
$$\gcd(a,b)=\min\{|xa+yb|>0\text{ such that }x,y\in\Z\}.$$
\end{slide}

\begin{slide}
\heading{Extended Euclidean Algorithm}\pause

From the above identity it follows immediately that $\gcd(a,b)$ exists and that $\gcd(a,b)=xa+by$ for appropriate
$x,y\in\Z$. In many applications it is crucial to compute $x,y$ that realize the above identity and they
are called the \emph{Bezout coefficients}. \pause

\noindent \textbf{Theorem.} \textit{Given $a,b\in\N$, $0<b\le a$, then there exists $x,y,z$ such that $z=\gcd(a,b)$ and $z=ax+by$.
Furthermore they can be computed with an algorithm (EEA) with bit complexity $O(\log^2a)$.}
\end{slide}



\begin{slide}
\heading{Extended Euclidean Algorithm}

It is based on successive divisions:
$$
\begin{array}{rclcr}
       a &    =     & b       \cdot     q_0 & +        & r_1 \\
       b &    =     & r_1     \cdot     q_1 & +        & r_2 \\
     r_1 &    =     & r_2     \cdot     q_2 & +        & r_3 \\
     r_2 &    =     & r_3     \cdot     q_3 & +        & r_4 \\
%     r_3 &    =     & r_4     \cdot     q_4 & +        & r_5 \\
         &  \vdots  &                  &      \vdots    &     \\
 r_{k-2} &    =     & r_{k-1} \cdot q_{k-1} & +        & r_k \\
 r_{k-1} &    =     & r_k     \cdot   q_{k} &          &     
\end{array}
$$\pause
Note that
$$
\begin{array}{rl}
    a=bq_0+r_1\geq bq_0\geq (r_1q_1+r_2)q_0 &\geq r_1q_1q_0\geq\cdots \\
   \cdots & \geq r_kq_kq_{k-1}\cdots q_0
\geq q_kq_{k-1}\cdots q_0,
  \end{array}
$$
\end{slide}

\begin{slide}
\heading{Extended Euclidean Algorithm}

The $j+1$--th division requires time $O(\log r_{j}\log q_j)$ and using the fact that $\log r_i\leq \log b$,
we obtain that the total time for running the EEA is\vspace*{-3mm}
$$O(\log b\sum_{j=0}^{k}\log q_k)=O(\log b\log (q_0\cdots q_k))=O(\log b\log a).$$\vspace*{-3mm}\pause
A variation of the EEC with the same complexity but other advantages is
\begin{center}\textsc{Binary gcd-algorithm (J. Stein -- 1967)}
\begin{tt}
\fbox{
\begin{tabular}{|rcclcr|}
\hline
$(a,b)$ & $=$ & if & $a<b$ & then & $(b,a)$\\
        &     & if &  $b=0$   & then &  $a$      \\
        &     & if &  $2\mid a, 2\mid b$   & then & $2(a/2,b/2)$       \\
        &     & if &  $2\mid a, 2\nmid b$   & then & $(a/2,b)$  \\
        &     & if &  $2\nmid a, 2\mid b$   & then & $(a,b/2)$     \\
        &     &   &      & else &  $((a-b)/2,b)$ \\
\hline
\end{tabular}
}
\end{tt}
\end{center}
\end{slide}



\begin{slide}
\heading{Binary GCD Algorithm}
$$\begin{array}{rrcl}
 1.  & (1547,560)&=&(1547,280) \\
 2.  & (1547,280)&=&(1547,140) \\
 3.  & (1547,140)&=&(1547,70)  \\
 4.  &  (1547,70)&=&(1547,35)   \\
 5.  &  (1547,35)&=&(756,35)   \\
 6.  &   (756,35)&=&(378,35)   \\
 7.  &   (378,35)&=&(189,35)   \\
 8.  &   (189,35)&=&(77,35)    \\
 9.  &    (77,35)&=&(35,21)    \\
 10. &    (35,21)&=&(7,21)     \\
 11. &     (21,7)&=&(7,7)      \\
 12. &      (7,7)&=&(7,0)=7      \\
 %13. &        (7,0)&=&7        \\
\end{array}$$
\end{slide}



\begin{slide}
\heading{Extended Euclidean Algorithm}

The EEA \fbox{\begin{tiny}$
\begin{array}{rclcr}
       a &    =     & b       \cdot     q_0 & +        & r_1 \\
       b &    =     & r_1     \cdot     q_1 & +        & r_2 \\
     r_1 &    =     & r_2     \cdot     q_2 & +        & r_3 \\
     r_2 &    =     & r_3     \cdot     q_3 & +        & r_4 \\
%     r_3 &    =     & r_4     \cdot     q_4 & +        & r_5 \\
         &  \vdots  &                  &      \vdots    &     \\
 r_{k-2} &    =     & r_{k-1} \cdot q_{k-1} & +        & r_k \\
 r_{k-1} &    =     & r_k     \cdot   q_{k} &          &     
\end{array}
$\end{tiny}} \\ produces quotients $q_0,\cdots,q_k$ and remainders $r_1,\ldots,r_k$. It is easy to check that
$\exists$ integers $\alpha_1,\ldots,\alpha_k$ and $\beta_1,\ldots,\beta_k$ such that $\forall i=1,\ldots,k$
$$r_i= \alpha_i\cdot a+ \beta_i \cdot b.$$
Since $(a,b)=r_k$, this shows the existence of the Bezout coefficients. The integers $(\alpha_i,\beta_i)$,
$i=0,\ldots,k$
are called \emph{partial Bezout coefficients}.
Furthermore the following recursive formulas hold:
\begin{equation*}\left\{
  \begin{array}{l}
    \alpha_0=0,
    \alpha_1=1\\
    \alpha_i=\alpha_{i-2}-q_{i-1}\cdot\alpha_{i-1}
  \end{array}\right.\hfill
\left\{
  \begin{array}{l}
    \beta_0=1,
    \beta_1=-q_0\\
    \beta_i=\beta_{i-2}-q_{i-1}\cdot\beta_{i-1}
  \end{array}\right.
\end{equation*}
that can be written in matrix form as:\pause
$$
\begin{pmatrix}
\alpha_0 &\alpha_1 \\
\beta_0 & \beta_1
\end{pmatrix}=
\begin{pmatrix}
0 & 1 \\
1 & -q_0
\end{pmatrix},\hspace{1cm}
\begin{pmatrix}
\alpha_i \\
\beta_i
\end{pmatrix}=
\begin{pmatrix}
\alpha_{i-2} &\alpha_{i-1} \\
\beta_{i-2} & \beta_{i-1}
\end{pmatrix}\begin{pmatrix}
1 \\
-q_{i-1}
\end{pmatrix}.
$$

\end{slide}

\begin{slide}
\noindent\textbf{Example.}  $(1547,560)=7$\\
\noindent EEC:
$$
\begin{array}{rclr}
 1547&=&2\cdot560+427 \\
 560&=&1\cdot427+133  \\
 427&=&3\cdot133+28   \\
 133&=&4\cdot28+21    \\
 28&=&1\cdot21+7 & \leftarrow \text{GCD}      \\
 21&=&3\cdot7             \\
\end{array}$$

So that $(q_0,q_1,q_2,q_3,q_4,q_5)=(2,1,3,4,1,3)$.

\end{slide}

\begin{slide}
\heading{Example: $(1547,560)=7$ continues.}
\begin{equation*}\left\{
  \begin{array}{l}
    \alpha_0=0,
    \alpha_1=1\\
    \alpha_i=\alpha_{i-2}-q_{i-1}\cdot\alpha_{i-1}
  \end{array}\right.\hfill
\left\{
  \begin{array}{l}
    \beta_0=1,
    \beta_1=-q_0\\
    \beta_i=\beta_{i-2}-q_{i-1}\cdot\beta_{i-1}
  \end{array}\right.
\end{equation*}

$$
\begin{array}{|c|crr|}
\hline
 i & q_i & \alpha_i & \beta_i \\
\hline
 0 &  2  &    0     &    1    \\
 1 &  1  &    1     &   -2    \\
 2 &  3  &    -1    &    3    \\
 3 &  4  &    4     &   -11   \\
 4 &  1  &   -17    &   47    \\
 5 &  3   &    21    &   -58   \\
\hline
\end{array}
\ \ \ \text{In fact:}\ 7=21\cdot1547-58\cdot 560.
$$
\end{slide}

\begin{slide}
\heading{Analysis of EEC on $a,b\in\N$}

Assume that $a>b$. We want to show that the number of iterations (i.e. the number of divisions needed) 
during the EEA is (in the worst case) $O(\log a)$.\pause

\textbf{Fibonacci Numbers:} $F_1=F_2=1$ and $F_n=F_{n-1}+F_{n-2}$.\pause
In the very special case when $a=F_n$ and $b=F_{n-1}$ then $r_1=F_{n-2}$, $r_2=F_{n-3}$,$\ldots$
$r_{n-2}=F_{1}=1$ and $r_{n-1}=0$.\vspace*{-3mm}\pause
From this we deduce that 
\begin{enumerate}
	\item $\gcd(F_n,F_{n-1})=1$
        \item The number of divisions required by EEA is $O(n)$.
\end{enumerate}

\textbf{Proposition.} \textit{Let $\theta=(\sqrt{5}+1)/2$. Then 
$$F_n=\frac{\theta^n+(1-\theta)^n}{\sqrt{5}}.$$
Hence $\log F_n\sim n\theta$ (so that $n=O(\log F_n)$).}\vspace*{-3mm}\pause
\textsc{Proof.} By induction.$\qquad\square$
\end{slide}


\begin{slide}
\heading{Analysis of EEC on $a,b\in\N$}

\textbf{Consequence.}\textit{ If  $a=F_n$ and $b=F_{n-1}$, then EEA requires $O(\log a)$ divisions!} 

\textbf{Proposition.} \textit{Assume that $a>b\ge1$. If the EEA to compute $\gcd(a,b)$ requires $k$ divisions, Then
$a\geq F_{k+2}$ and $b\ge F_{k+1}$.}\pause
\textsc{Proof.} Let us first show that $r_{k-j}\ge F_{j+1}$. 
Indeed by induction or $j$:\begin{itemize}
	\item $r_k=\gcd(a,b)\ge 1=F_1$, $r_{k-1}\ge 1=F_2$
\item  $r_{k-j}=q_{k-(j-1)}r_{k-(j-1)}+r_{k-(j-2)}\ge F_{j}+F_{j-1}=F_{j+1}.$
\end{itemize}\pause

Hence $b=r_0\geq F_{k+1}$ and $a=q_0b+r_1\ge F_{k+1}+F_{k}=F_{k+2}.\qquad\square$\pause

\textbf{Consequence.} \textit{The number of divisions $k=O(\log F_{k+2})=O(\log a) \forall a,b$.}\pause

A more careful analysis (the fact that the size of the integers decreases exponentially)
of EEA shows that the bit complexity is $O(\log^2a)$.
\end{slide}

\begin{slide}
\heading{Geometric GCD algorithm (probably the original one)}
\stepwise{\begin{itemize}
	\item \step{To compute $(a,b)$ with $a\ge b>0$, consider the rectangle with base $a$ and height $b$.}
        \item  \step{Remove from it a square of maximal area obtaining a rectangle of sizes $a$ and $a-b$.}
        \item \step{Reorder them (if needed) and then repeat the process of removing a square.}
        \item  \step{Keep on removing squares till it is left a square.}
        \item \step{The edge of the final square is the gcd.}
\end{itemize}}\pause

\textbf{Example. } $(1547,560)=(987,560)=(427,560)=(427,133)=(294,133)=(161,133)=(28,133)=(105,28)
=(77,28)=(49,28)=(21,28)=(21,7)=(14,7)=(7,7)=7$

\end{slide}

\begin{slide}
\heading{Extended GCD algorithm (EEA)}

\begin{center}\fbox{\textcolor{black}{
\begin{minipage}[c]{9cm}
\texttt{\noindent 
\textcolor{red}{Input:} $a,b\in\N$, $a>b$\\
\textcolor{blue}{Output:} $x,y,z$ where $z=\gcd(a,b)$ and $z=ax+by$\\
1. $(X,Y,Z)=(1,0,a)$\\
2. $(x,y,z)=(0,1,b)$\\
\hspace*{5mm} \quad While $Z>0$\\
\hspace*{10mm} \quad $q:=\lfloor Z/z\rfloor$\\
\hspace*{10mm} \quad $(X,Y,Z)=(x,y,z)$\\
\hspace*{10mm} \quad $(x,y,z)=(X-qx,Y-qy,Z-qz)$\\
\hspace*{5mm} \quad Output $X,Y,Z$}
\end{minipage}}}
\end{center}\pause

To show that it is correct it is enough to check that after one iteration
$(X_1,Y_1,Z_1)=(1,-q_0,r_1)$ and after $k$ iterations 
$$(X_k,Y_k,Z_k)=(X_{k-2}-q_{k-1}X_{k-1},Y_{k-2}-q_{k-2}Y_{k-2},Z_{k-2}-q_{k-1}Z_{k-1})=(\alpha_{k},\beta_{k},r_k).$$
\end{slide}

\begin{slide}
\heading{The Euler $\varphi$--function}

A first important application of EEA is to determine the inverses in $\Z/m\Z$ \pause
\textbf{Theorem.} \textit{Let $a\in\Z$ and $m\in\N$ with $m>1$. Then $a\mod m$ is invertible 
(i.e. $\exists b\in\Z/m\Z$ with $ab\equiv 1\bmod m$) iff $\gcd(a,m)=1$. Furthermore
the ``\emph{arithmetic inverse}'' $b$ can be computed with time $O(\log m^2)$.} \pause
\textbf{Proof.} If $\gcd(a,m)=1$ then in time $O(\log m^2)$ we can compute $x,y\in\Z$ such that
$1=xa+ym$. Hence $b=x\mod m$ has the required property. \\
Conversely if $ab\equiv 1\bmod m$, then $1=ab+km$ for an appropriate $k\in\Z$. This implies that
$\gcd(a,m)$ divides $1$ and finally $\gcd(a,m)=1\quad\square$.\pause
\textbf{Corollary.} \textit{The set $U(\Z/m\Z)$ of invertible elements of $\Z/m\Z$ coincides with
$$\{a\in\N\text{ s.t. }1\le a\le m,\gcd(a,m)=1\}.$$}\pause
We define the Euler $\varphi$ function as
$$\varphi(n)=\#U(\Z/m\Z)=\#\{a\in\N\text{ s.t. }1\le a\le m,\gcd(a,m)=1\}.$$

\end{slide}


\begin{slide}
\heading{The Euler $\varphi$--function continues}

\parstepwise
{
\begin{itemize}
\item \step
{$\varphi(1)=1,\quad\varphi(p)=p-1,\quad\varphi(p^\alpha)=p^{\alpha-1}(p-1)$}
\item \step
{$\varphi(mn)=\varphi(m)\varphi(n)$ if $\gcd(m,n)=1$.}\\
\step
{This is a consequence of the Chinese Remainder Theorem}\\ \step{(we shall meet it later).}
\item \step
{Hence if we can factor $n=p_1^{\alpha_1}\cdots p_r^{\alpha_r}$, then $\varphi(n)$ is easy to compute.} \\ 
\step
{it is enough to compute $n\prod_{p\mid n}1-1/p$.}
\item \step
{If we know that $k=\varphi(n)$ and that $n=q\times p$ then we can factor $n$}\\
\step
{ In fact $\{p,q\}=\left\{\frac{\varphi(n)-n-1\pm\sqrt{(\varphi(n)-n-1)^2-4n}}2\right\}.$}
\item \step
{An important \textbf{Theorem of Euler:}}\\ \step{\textit{ If $a\in U(\Z/m\Z)$ then $a^{\varphi(n)}\equiv1\bmod n$.}}
\end{itemize}}

The latter is crucial in RSA encryption and decryption 
\end{slide}


\end{document}
