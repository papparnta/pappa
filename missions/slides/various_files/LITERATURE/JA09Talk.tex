%%%%%%%%%%%%%%%%%%%%%%%%%%%%%%%%%%%%%%%%%%%%%%%%%%%%%%%%%%%%%%%%%%%%%%
%% Instructions for using the LaTeXSlide style package
%%
%% Top of document should look like:
%%
%%%%%%%%%%%%%%%%%%%%%%%%%%%%%%%%%%%%%%%%%%%%%%%%%%%%%%%%%%%%%%%%%%%%%%
%%  Macros available for formatting slides
%%
%% \SetSlideHead{XXX}  -  Heading to appear at top of each slide
%% \SectionTitlePage{XXX} - Title page for section
%% \SlideTitle{XXX}  -  Centered title for slide
%% \ITEM[X] - item in a list, labeled with X
%% \WIDEITEM[XXX] - item in a list, with a wide label XXX
%% \BULLETITEM - item in a list, labeled with a bullet
%%   NOTE: It is important to put a \par at the end of each ITEM
%% \HRULE - horizonal rule
%%
%% \DefTerm{XXX} - for highlighting a term being defined
%% \Declare{XXX} - for Theorem, Corollary, etc
%% \HighlightDisplay{math material} - math on a  colored background
%%     Example:  \[ \HighlightDisplay{X^n+Y^n=Z^n} \]
%% \TheoremBox{type}{statement} -  to state a theorem, proposition, lemma, etc.
%%     Example: \TheoremBox{Theorem}{(Tate 1960) Let $K$ be a ...}
%%
%% \FinalEq - use before displayed equations appearing just _before_  \EndPart
%% \InitialItem -  use before an \ITEM that appears just _after_  \EndPart
%%
%% The format for a slide with overlays is:
%%   \BeginSlide
%%     Material to initially print on the slide.
%%   \EndPart
%%     Additional material to overlay on the slide.
%%   \EndPart
%%     And yet more material to overlay on the slide.
%%   \EndSlide
%%%%%%%%%%%%%%%%%%%%%%%%%%%%%%%%%%%%%%%%%%%%%%%%%%%%%%%%%%%%%%%%%%%%%%
\documentclass[12pt]{article} 
\def\SplitSlides{F}  %% where X is T for true or F for false
% \def\SplitSlides{T}  %% where X is T for true or F for false

\usepackage{graphics}
\usepackage{c:/JHS/Conferences/LaTeXSlidePackage/LaTeXSlide}

\advance\voffset by-10pt
%%--------------------------------
\def\TalkTitle{%
Specialization Maps\\[10pt] 
and\\[10pt] 
Unlikely Intersections}
\def\TalkTitleOneLine{Specialization and Unlikely Intersections}
\def\Speaker{Joseph H. Silverman}
\def\Affiliation{Brown University}
\def\Event{26th Journ\'ees Arithm\'etique\\
Saint-\'Etienne, France}
\def\TalkDate{July 6--10, 2009}
%%%%%%%%%%%%%%%%%%%%%%%%%%%%%%%%%%%%%%%%%%%%%%%%%%%%%%%%%%%%%%%%%%%%%%
%% 
%%%%%%%%%%%%%%%%%%%%%%%%%%%%%%%%%%%%%%%%%%%%%%%%%%%%%%%%%%%%%%%%%%%%%%
%%--------------------------------
\cfoot{\fancyplain{}{\footnotesize\textrm{\TalkTitleOneLine}}} 
\def\SlideHead{\TalkTitleOneLine}
%%--------------------------------
%% Insert additional macro definitions here
\newcommand{\qedtag}{$\square$}
\renewcommand{\endproof}[1]{\hbox to\hsize{#1\hfill\qedtag\enspace}}
\newcommand{\Prob}{\operatorname{Prob}}
\renewcommand\pmod[1]{~(\text{mod}~#1)}
%% \newcommand{\oa}{{\circ a}}
\newcommand{\oa}{\textup{na}}
\newcommand{\Hcal}{\mathcal{H}}
\newcommand{\Gcal}{\mathcal{G}}

%%%%%%%%%%%%%%%%%%%%%%%%%%%%%%%%%%%%%%%%%%%%%%%%%%%%%%%%%%%%%%%%%%%%%%
%%--------------------------------
\begin{document}
\Huge
\TitlePage  %% delete if do not want a title page

%%%%%%%%%%%%%%%%%%%%%%%%%%%%%%%%%%%%%%%%%%%%%%%%%%%%%%%%%%%%%%%%%%%%%%
\SetSlideHead{Specialization for Families of Abelian Varieties}
\SectionTitlePage{Specialization for\\[10pt] Families of\\[10pt] Abelian Varieties}

%%\SectionTitlePage{XXXXXXXXXXXXXXXXXXX}
%% If need to break lines on a title page, use
%% \SectionTitlePage{XXXXXXXXXXXX\\[10pt]  and\\[10pt] YYYYYYYYYYYYYYYY}

%%%%%%%%%%%%%%%%%%%%%%%%%%%%%%%%%%%%%%%%%%%%%%%%%%%%%%%%%%%%%%%%%%%%%%
\BeginSlide
\SlideTitle{An Example}
We start with an example.
Consider the family of elliptic curves and family of points,
\[\HighlightDisplay{
  E:y^2=x^3+x+T^2,\qquad P=(0,T).
  }
\]
\EndPart
\vspace{-5pt}
We are interested in studying how frequently the \emph{specialized
point} $P_t=(0,t)$ is a point of finite order on the \emph{specialized
elliptic curve} $E_t$.
\EndPart
\vspace{10pt}
For example, setting $t=0$, we find that
\[\HighlightDisplay{
   \text{$P_0=(0,0)$ is a point of order~$2$ on~$E_0$.}
  }
\]
\EndPart
Similarly,
\[\HighlightDisplay{
  t=\frac{\sqrt[4]{\sqrt5-1}}{2}
  \quad\text{makes $P_t$ a point of order $5$ on $E_t$.}
  }
\]
\EndSlide

%%%%%%%%%%%%%%%%%%%%%%%%%%%%%%%%%%%%%%%%%%%%%%%%%%%%%%%%%%%%%%%%%%%%%%
\BeginSlide
\SlideTitle{The Specialization Map}
In general, given an elliptic curve~$E$ defined over $K(T)$, 
\[\HighlightDisplay{
  E:y^2 = x^3+A(T)x+B(T),
  }
\]
each
$t\in K$ defines a \DefTerm{Specialization Map}
\[\HighlightDisplay{
  \s_t : E\bigl(K(T)\bigr) \longrightarrow E_t(K).
  }
\]
\EndPart
Thus given a point $P=\bigl(x(T),y(T)\bigr)\in E\bigl(K(T)\bigr)$, we
compute $\s_t(P)$ by evaluating at~$T=t$:
\[\HighlightDisplay{
  \s_t(P)=\bigl(x(t),y(t)\bigr)\in E_t(K) : y^2=x^3+A(t)x+B(t).
  }
\]
\EndPart
It is natural to ask how frequently independent points in~$E\bigl(K(T)\bigr)$
remain independent when specialized. Equivalently, how large is
the \DefTerm{exceptional set}
\[\HighlightDisplay{
  \Ecal(E,K) \stackrel{\scriptscriptstyle\text{def}}{=}
  \bigl\{ t\in K : \text{$\s_t$ fails to be injective }\bigr\}?
  }
\]
\EndSlide
%%%%%%%%%%%%%%%%%%%%%%%%%%%%%%%%%%%%%%%%%%%%%%%%%%%%%%%%%%%%%%%%%%%%%%

%%%%%%%%%%%%%%%%%%%%%%%%%%%%%%%%%%%%%%%%%%%%%%%%%%%%%%%%%%%%%%%%%%%%%%
\BeginSlide
\SlideTitle{Specialization Theorems}
\TheoremBox{Theorem}{%
Let $K/\QQ$ be a number field and $E/K(T)$ an elliptic curve. Then
\[
%%  \{t\in K : \text{$\s_t$ fails to be injective} \}
  \Ecal(E,K) 
  \longrightarrow\begin{cases}
    \text{is small \textup(density $0$\textup) 
              \textup(N\'eron, 1952\textup),} \\[8pt]
    \text{is finite \textup(JS, 1983\textup).}\\
  \end{cases}
\]
}
\EndPart
More generally, we look at one-parameter families of abelian
varieties and consider specializations over $\Qbar$.
\TheoremBox{Theorem}{%
Let $T/\Qbar$ be a curve and let $A\to T$ be a family of abelian
varieties, defined over $\Qbar$, with no constant part.
%% \textup(i.e., with $\Qbar(T)/\Qbar$ trace zero\textup).  
Then
\[
  \Ecal(A,\Qbar) = 
  \bigl\{t\in T(\Qbar) :  \text{$\s_t$ fails to be injective} \}
\]
is a set of bounded height.}
\EndSlide
%%%%%%%%%%%%%%%%%%%%%%%%%%%%%%%%%%%%%%%%%%%%%%%%%%%%%%%%%%%%%%%%%%%%%%

%%%%%%%%%%%%%%%%%%%%%%%%%%%%%%%%%%%%%%%%%%%%%%%%%%%%%%%%%%%%%%%%%%%%%%
\BeginSlide
\SlideTitle{Heights and Specialization}
The theorem is proven using  a height specialization result. 
We fix an ample symmetric divisor~$D$ on~$A$ and consider 
three height functions:
\[\HighlightDisplay{
  \begin{aligned}
    \hhat_t : A_t(\Qbar)&\longrightarrow \RR,
      &&\text{canonical height wrt $D_t$,} \\
    \hhat : A\bigl(\Qbar(T)\bigr) &\longrightarrow \RR,
      &&\text{canonical height wrt divisor $D$,} \\
    h : T(\Qbar) &\longrightarrow\RR
      &&\text{height wrt degree 1 divisor.}
  \end{aligned}
  }
\]
\EndPart
These heights are related by a limit formula:
\TheoremBox{Theorem}{%
Let $P\in A\bigl(\Qbar(T)\bigr)$. Then
\[\HighlightDisplay{
  \lim_{\substack{t\in T(\Qbar)\\ h(t)\to\infty\\}}
  \frac{\hhat_t(P_t)}{h(t)} = \hhat(P).
  }
\]
}
\EndPart
The specialization theorem follows from nondegeneracy of $\hhat$ and
$\hhat_t$.
\EndSlide
%%%%%%%%%%%%%%%%%%%%%%%%%%%%%%%%%%%%%%%%%%%%%%%%%%%%%%%%%%%%%%%%%%%%%%

%%%%%%%%%%%%%%%%%%%%%%%%%%%%%%%%%%%%%%%%%%%%%%%%%%%%%%%%%%%%%%%%%%%%%%
\SetSlideHead{Specialization in the Multiplicative Group}
\SectionTitlePage{Specialization\\[10pt]  in the\\[10pt] Multiplicative Group}
%%%%%%%%%%%%%%%%%%%%%%%%%%%%%%%%%%%%%%%%%%%%%%%%%%%%%%%%%%%%%%%%%%%%%%

%%%%%%%%%%%%%%%%%%%%%%%%%%%%%%%%%%%%%%%%%%%%%%%%%%%%%%%%%%%%%%%%%%%%%%
\BeginSlide
\SlideTitle{Specialization in the Multiplicative Group}
\vspace{-5pt}
Somewhat surprisingly, specialization results on elliptic curves and abelian
varieties preceded study of the analogous
question for the multiplicative group.%
\EndPart
\vspace{5pt}
We again start with an example.
\[\HighlightDisplay{
  \begin{gathered}
    \text{For which $t\in\QQ$ are $t$ and $t-2$}\\
    \text{ multiplicatively dependent?}
  \end{gathered}
  }
\]
\EndPart
If either $t$ or $t-2$ equals~$1$ or~$-1$, they are dependent, so
that's $t\in\{-1,1,3\}$. Are there other $t\in\QQ$?
\EndPart
\vspace{-5pt}
\[\HighlightDisplay{
  \begin{gathered}
    \text{$t$ and $t-2$ are multiplicatively dependent} \\
    \hspace{7em}\Longleftrightarrow\quad
    t\in\{-2,-1,1,3,4\}.
  \end{gathered}
  }
\]
\EndPart 
Of course, if we allow $t\in\Qbar$, there are many exceptional
values.  Indeed, each equation $t^n(t-2)^m=1$ with $n,m\in\ZZ$,
$(m,n)\ne(0,0)$ gives finitely many $t\in\Qbar$ such that $t$ and
$t-2$ are multiplicatively dependent.
\EndSlide
%%%%%%%%%%%%%%%%%%%%%%%%%%%%%%%%%%%%%%%%%%%%%%%%%%%%%%%%%%%%%%%%%%%%%%

%%%%%%%%%%%%%%%%%%%%%%%%%%%%%%%%%%%%%%%%%%%%%%%%%%%%%%%%%%%%%%%%%%%%%%
\BeginSlide
\SlideTitle{Specialization in the Multiplicative Group}
Let 
\[\HighlightDisplay{
  f_1,\ldots,f_r\in\Qbar(T)^*
  }
\]
be rational functions that are multiplicatively independent modulo
$\Qbar^*$. The associated \DefTerm{exceptional set} is
\[\HighlightDisplay{
  \Ecal(f_1,\ldots,f_r) = \left\{t\in\Qbar : 
    \begin{tabular}{l}
      $f_1(t),\ldots,f_r(t)$ are\\ multiplicatively dependent\\
    \end{tabular}
  \right\}.
  }
\]
\EndPart
\TheoremBox{Theorem}{%
(Bombieri--Masser--Zannier, 1999)
\[
  \text{$\Ecal(f_1,\ldots,f_r)$ is a set of bounded height.}
\]
}
\EndSlide
%%%%%%%%%%%%%%%%%%%%%%%%%%%%%%%%%%%%%%%%%%%%%%%%%%%%%%%%%%%%%%%%%%%%%%

%%%%%%%%%%%%%%%%%%%%%%%%%%%%%%%%%%%%%%%%%%%%%%%%%%%%%%%%%%%%%%%%%%%%%%
\BeginSlide
\SlideTitle{Heights and Specialization}
\vspace{-4pt}
The BMZ proof relies on height estimates, but note that the
height is not a positive definite form on $\GG_m$.
\TheoremBox{Theorem}{%
(BMZ)
There exist $C_1,C_2>0$ such that for all $m_1,\ldots,m_r\in\ZZ$ and
all $t\in\Qbar$,
\[\HighlightDisplay{
    \begin{aligned}
      h\bigl(f_1(t)^{m_1}f_2(t)^{m_2}&\cdots f_r(t)^{m_r}\bigr)\\
      &\ge \bigl(\max_{1\le i\le r} |m_i|\bigr) \bigl(C_1h(t)-C_2).
    \end{aligned}
  }
\]
}
\EndPart
\emph{Proof Idea.}\enspace
\textbullet\enspace
$\deg\left(f_1^{m_1}\cdots f_r^{m_r}\right) \gg \max |m_i|$.
\EndPart
\textbullet\enspace
A general theorem says that
\[\HighlightDisplay{
    h\bigl(f(t)\bigr) \ge (\deg f)h(t) - c(f),
  }
\]
but that's no good, since $c(f)$ depends on $f$. It requires an
intricate argument to replace the constant $c(f)$ with
$\deg(f)c(f_1,\ldots,f_r)$ when~$f$ is in the group generated by
$f_1,\ldots,f_r$.
\EndSlide
%%%%%%%%%%%%%%%%%%%%%%%%%%%%%%%%%%%%%%%%%%%%%%%%%%%%%%%%%%%%%%%%%%%%%%

%%%%%%%%%%%%%%%%%%%%%%%%%%%%%%%%%%%%%%%%%%%%%%%%%%%%%%%%%%%%%%%%%%%%%%
\BeginSlide
\SlideTitle{An Unlikely Intersection}
\vspace{-5pt}
We can reformulate the BMZ result as follows. Consider the map
\[\HighlightDisplay{
  F = (f_1,\ldots,f_r) : T \longrightarrow \AA^r.
  }
\]
Since $T$ is a curve, we have
\[\HighlightDisplay{
    \begin{aligned}
      f_1(t),\ldots&,f_r(t)\text{ are multiplicatively dependent} \\
      &\Longleftrightarrow
      \text{$F(t)$ lies on some
	$\underbrace{X_1^{e_1}\cdots X_r^{e_r}=1}_{\text{subgroup 
	       of $\GG_m^r$}}$.} \\
    \end{aligned}
  }
\]
\EndPart
Thus the exceptional set is the intersection
\[\HighlightDisplay{
    \vrule height0pt depth70pt width0pt
    \hspace{30pt}
    \Ecal(f_1,\ldots,f_r) = \operatorname{Image}(F) \cap
	\left(\begin{tabular}{l}subgroups\\of $\GG_m^r$\\\end{tabular}\right).
    \hspace{30pt}
  }
\]
\EndPart
\vspace{-70pt}
\hspace{200pt}
$\underbrace{\phantom{XXXXXXXXXXXX}}_{\textbf{Unlikely Intersection}}$
\EndPart
\vspace{14pt}
The BMZ results says that this unlikley intersection is a set
of bounded height.
%% The term \DefTerm{Unlikely Intersection}  refers to
%% a body of work by
%% Pink, Zilber, Bombieri, Masser, Zannier,\dots\thinspace.%
\EndSlide
%%%%%%%%%%%%%%%%%%%%%%%%%%%%%%%%%%%%%%%%%%%%%%%%%%%%%%%%%%%%%%%%%%%%%%

%%%%%%%%%%%%%%%%%%%%%%%%%%%%%%%%%%%%%%%%%%%%%%%%%%%%%%%%%%%%%%%%%%%%%%
\SetSlideHead{Higher Dimensional Families}
\SectionTitlePage{Higher Dimensional\\[10pt] Families}
%%%%%%%%%%%%%%%%%%%%%%%%%%%%%%%%%%%%%%%%%%%%%%%%%%%%%%%%%%%%%%%%%%%%%%

%%%%%%%%%%%%%%%%%%%%%%%%%%%%%%%%%%%%%%%%%%%%%%%%%%%%%%%%%%%%%%%%%%%%%%
\BeginSlide
\SlideTitle{Higher Dimensional Families}
\vspace{-5pt}
Up to now we have been considering one-paramter families, i.e., 
$\dim T=1$. Things become much more complicated when $\dim T\ge2$.
Let
\[\HighlightDisplay{
  f_1,\ldots,f_r\in\Qbar(T_1,\ldots,T_n)^*
  }
\]
be multiplicatively independent rational functions of $n$
variables. Each relation
\[\HighlightDisplay{
  f_1(t)^{e_1}\cdots f_r(t)^{e_r}=1
  }
\]
cuts the dimension of the solution set by one, so each $n$ \emph{independent}
relations gives a finite set of points.
\EndPart
\vspace{-5pt}
\TheoremBox{Problem}{%
Describe the \emph{exceptional set}
\[
%%  \Ecal 
  \Ecal(f_1,\ldots,f_r)
  = \left\{ t\in \Qbar^n :
     \begin{tabular}{l}
	 $f_1(t),\ldots,f_r(t)$ satisfy\\
	 $n$ independent\\
	 multiplicative relations\\
      \end{tabular}
  \right\}.
\]
}
\EndSlide
%%%%%%%%%%%%%%%%%%%%%%%%%%%%%%%%%%%%%%%%%%%%%%%%%%%%%%%%%%%%%%%%%%%%%%

%%%%%%%%%%%%%%%%%%%%%%%%%%%%%%%%%%%%%%%%%%%%%%%%%%%%%%%%%%%%%%%%%%%%%%
\BeginSlide
\SlideTitle{Higher Dimensional Unlikely Intersections}
We again reformulate the problem by looking at the map
\[\HighlightDisplay{
  F = (f_1,\ldots,f_r) : \AA^n \longrightarrow \AA^r.
  }
\]
Then
\[\HighlightDisplay{
  \Ecal(F) = \text{(Image of $F$)} \quad\cap\quad
    \bigcup_{\substack{H\subset\GG_m^r\\
       \hidewidth\text{subgroup of codim $n$}\hidewidth\\
   }} H. \qquad
  }
\]
\EndPart
It is natural to guess that
\[\HighlightDisplay{
  \Ecal(F) \stackrel{?}{\subset} 
      \left(\begin{tabular}{l}
              proper Zariski\\closed set\\
           \end{tabular}\right)
      \cup
      \left(\begin{tabular}{l}
              set of bounded\\ height\\
           \end{tabular}\right).
  }
\]
\EndPart
This guess is correct for $r=n$,  which is the case
\[\HighlightDisplay{
  \Ecal(F) = \bigl\{t\in\Qbar^n : 
    \text{$f_1(t),\ldots,f_n(t)$ are roots of unity} \bigr\}.
  }
\]
However, in general it is \textbf{not correct}.
\EndSlide
%%%%%%%%%%%%%%%%%%%%%%%%%%%%%%%%%%%%%%%%%%%%%%%%%%%%%%%%%%%%%%%%%%%%%%

%%%%%%%%%%%%%%%%%%%%%%%%%%%%%%%%%%%%%%%%%%%%%%%%%%%%%%%%%%%%%%%%%%%%%%
\BeginSlide
\SlideTitle{A Counterexample}
\vspace{-10pt}
This example was shown to me by David Masser.
%% David sent it to me in answer to an email in which I had formulated
%% the incorrect conjecture.
%% He recently sent me a simpler (but less satisfying) example.
%% \[
%%   n=1,\quad r=2, \quad f_1=X, \quad f_2=2.
%% \]
%% (The image is $C=\GG_m\times\{2\}\subset\GG_m\times\GG_m$.) The 
%% points~$\{2^k\}$ have unbounded height, but $f_1(2^k)$ and $f_2(2^k)$
%% satisfy one relation, namely $f_1(2^k)=f_2(2^k)^k$. In this case
%% the non-anomalous subset is empty,~$C^\na=\emptyset$.
\[\HighlightDisplay{
  n=2,\quad r=3,\quad f_1=X,\quad f_2=Y,\quad f_3=X+Y.
  }
\]
\EndPart
\vspace{-17pt}
For $u,v\in\NN$, let $\b\in\Qbar$ be a root of
\[\HighlightDisplay{
  \b^u + \b^v = 1
  \quad\text{and specialize}\quad
  X = \b^u,\enspace Y=\b^v.
  }
\]
\EndPart
\vspace{-17pt}
Then the specialized values
\vspace{-5pt}
\[\HighlightDisplay{
  f_1 = \b^u,\qquad f_2=\b^v,\qquad f_3=1,
  }
\]
satisfy two independent relations
\[\HighlightDisplay{
  f_1^vf_2^u=1\qquad\text{and}\qquad f_3=1.
  }
\]
\EndPart
\vspace{-17pt}
But for any B, the set
\[\HighlightDisplay{
  \bigcup_{u,v\in\NN}
  \bigl\{(\b^u,\b^v) : \b^u+\b^v=1,\;h(\b^u)>B\bigr\}
  }
\]
is Zariski dense in~$\AA^2$, so $\Ecal(f_1,f_2)$ is not %contained
in the union of a set of bounded height and a proper %Zariski
closed subset.
\EndSlide
%%%%%%%%%%%%%%%%%%%%%%%%%%%%%%%%%%%%%%%%%%%%%%%%%%%%%%%%%%%%%%%%%%%%%%

%%%%%%%%%%%%%%%%%%%%%%%%%%%%%%%%%%%%%%%%%%%%%%%%%%%%%%%%%%%%%%%%%%%%%%
\SetSlideHead{Anomalous Subvarieties and the Bounded Height Theorem}
\SectionTitlePage{Anomalous Subvarieties\\[10pt]  and the\\[10pt]  Bounded Height Theorem}
%%%%%%%%%%%%%%%%%%%%%%%%%%%%%%%%%%%%%%%%%%%%%%%%%%%%%%%%%%%%%%%%%%%%%%

%%%%%%%%%%%%%%%%%%%%%%%%%%%%%%%%%%%%%%%%%%%%%%%%%%%%%%%%%%%%%%%%%%%%%%
\BeginSlide
\SlideTitle{Anomalous Subvarieties}
\Declare{Definition.}\enspace
Let
\vspace{-10pt}
\[\HighlightDisplay{
    Y\subset X\subset \GG_m^r.
  }
\]
$Y$ is \DefTerm{anomalous} (\DefTerm{for $\boldsymbol X$}) if $\dim Y\ge1$
and there exists a coset (translate of a subgroup) $K\subset\GG_m^r$
satisfying
\[\HighlightDisplay{
    \vrule height0pt depth50pt width0pt
    Y\subset K
    \quad\text{and}\quad
    \dim Y >
    \dim X+\dim K - r\strut.\hspace{40pt}
  }
\]
\EndPart
\vspace{-80pt}\hspace{245pt}
  $\underbrace{\phantom{\dim X+\dim K - r}}_{
     \text{expected dimension of $X\cap K$}}$
\EndPart
\vspace{20pt}
Since $Y\subset X\cap K$, this means that $X$ and $K$ contain
an \DefTerm{unlikey intersection}.
\EndPart
\vspace{5pt}
\Declare{Definition.}\enspace
The \DefTerm{non-anomalous part} of $X$ is
\[\HighlightDisplay{
    X^{\oa} = X \setminus (\text{all anomalous subvarieties}).
  }
\]
\EndPart
\vspace{-15pt}
\Declare{Example.}\enspace
$X = \bigl\{(x,y,x+y)\bigr\}\subset\GG_m^3 \Longrightarrow
X^{\oa}=\emptyset$.
%% In fact, in this case the larger set~$X^{\circ}$ is empty, where~$X^{\circ}$
%% is the complement of all cosets that are contained in~$X$. Here's why.
%% For any numbers~$a,b\in\CC^*$ with $a,b,a+b$ all nonzero, let~$K$
%% be the translate of the one-dimensional subgroup~$\{(t,t,t)\}$
%% by the point~$(a,b,a+b)$. Thus
%% \[
%%   K = \bigl\{ (at,bt,(a+b)t) : t\in\CC^* \bigr\}.
%% \]
%% Then take $Y=K$ in the definition of anomalous. We have $Y\subset X$,
%% and
%% \[
%%   1 = \dim Y > \dim X + \dim K - 3 = 2+1-3=0.
%% \]
%% Hence~$Y=K$ is $X$-anomalous. However, as~$a$ and~$b$ vary, the sets~$K$
%% completely cover~$X$, which proves that $X^\oa=\emptyset$.
\EndPart
\vspace{-10pt}
\TheoremBox{Theorem}{%
\text{(BMZ) $X^\oa$ is a Zariski open subset of~$X$.}}
\EndSlide
%%%%%%%%%%%%%%%%%%%%%%%%%%%%%%%%%%%%%%%%%%%%%%%%%%%%%%%%%%%%%%%%%%%%%%

%%%%%%%%%%%%%%%%%%%%%%%%%%%%%%%%%%%%%%%%%%%%%%%%%%%%%%%%%%%%%%%%%%%%%%
\BeginSlide
\SlideTitle{The Bounded Height Theorem}
\vspace{-8pt}
The following beautiful result was conjectured by Bombieri, Masser and
Zannier (2007) and proven by Habegger.  
\EndPart
\Declare{Notation.}\enspace
\[\HighlightDisplay{
    \Gcal^{[d]} = \left(\begin{tabular}{l}
	  union of all algebraic subgroups\\
	    of $\GG_m^r$ of codimension $d$\\
	\end{tabular}\right)
  }
\]
%% \[\HighlightDisplay{
%%     \Hcal_d = \left(\begin{tabular}{l}
%% 	  union of all algebraic subgroups\\
%% 	    of $\GG_m^r$ of dimension $d$\\
%% 	\end{tabular}\right)
%%   }
%% \]
\EndPart
\vspace{-10pt}
\TheoremBox{Theorem}{%
(Habegger 2009)
\[
  \text{$X^{\oa}\cap \Gcal^{[\dim X]}$ is a set of bounded height.}
\]
}
\EndPart
More generally, Habegger shows that the result is true for points that
are ``close to $\Gcal^{[\dim X]}$.''
\EndPart
\vspace{-6pt}
\TheoremBox{Theorem}{%
There exists an $\e>0$ such that
\[
  \!X^{\oa} \cap \left\{ xy :
      \begin{array}{l}
         x\in\Gcal^{[\dim X]},\; y\in\GG_m^r,\\
         \text{and }h(y)\le\e(h(x)+1) \\
     \end{array}
   \right\}
%%   \!X^{\oa} \cap \bigl\{ xy :
%%     x\in\Gcal^{[\dim X]},\, y\in\GG_m^r,\, h(y)\le\e(h(x)+1) \bigr\}
\]
is a set of bounded height.
}
\EndSlide
%%%%%%%%%%%%%%%%%%%%%%%%%%%%%%%%%%%%%%%%%%%%%%%%%%%%%%%%%%%%%%%%%%%%%%

%%%%%%%%%%%%%%%%%%%%%%%%%%%%%%%%%%%%%%%%%%%%%%%%%%%%%%%%%%%%%%%%%%%%%%
\BeginSlide
\SlideTitle{Sketch of Habegger's Proof}
Let $n=\dim X$.
Construct a set of algebraic quotient groups $\G_1,\ldots,\G_t$
of~$\GG_m^r$ such that every $P\in \GG_m^r$ lying in a subgroup of
codimension~$n$ is in the kernel of some~$\GG_m^r\to\G_i$. It thus suffices
to fix one
\[\HighlightDisplay{
    \psi:\GG_m^r\longrightarrow\G.
  }
\]
\EndPart
\vspace{-10pt}
In order to compare the heights of $P$ and $\psi(P)$, Habegger uses
the following theorem of Siu.
\TheoremBox{Theorem}{%
(Siu 1993)
Let $X$ be an irreducible projective variety of dimension $n\ge1$ defined
over $\CC$. Let $\Lcal$ and $\Mcal$ be \textup{nef} line bundles on $X$. If
\[
  c_1(\Lcal)^n[X]  > n \bigl(c_1(\Lcal)^{n-1}c_1(\Mcal)[X]\bigr),
\]
then there exists an integer~$k\ge1$ such that
\text{$(\Lcal\otimes\Mcal^{-1})^k$} has a nonzero global section.}
\EndSlide
%%%%%%%%%%%%%%%%%%%%%%%%%%%%%%%%%%%%%%%%%%%%%%%%%%%%%%%%%%%%%%%%%%%%%%

%%%%%%%%%%%%%%%%%%%%%%%%%%%%%%%%%%%%%%%%%%%%%%%%%%%%%%%%%%%%%%%%%%%%%%
\BeginSlide
\SlideTitle{Sketch of Habegger's Proof (continued)}
Habegger uses Siu's theorem to construct an effective divisor~$D$
satisfying
\[\HighlightDisplay{
  h_D\bigl(\psi(P)\bigr) \ge c(\psi)h_D(P)
  \quad\text{for $P\in \bigl(X\setminus|D|\bigr)(\Qbar)$,}
  }
\]
where
\[\HighlightDisplay{
  c(\psi) \ge 0\quad\text{is given as an intersection number.}
  }
\]
\EndPart
\vspace{-5pt}
In order to show that $c(\psi)>0$, Harbegger proves that it extends to
a continuous map
\[\HighlightDisplay{
  c : \operatorname{Hom}(\GG_m^r,\G)\otimes\RR \longrightarrow\RR
  }
\]
and uses properties of $X^{\oa}\cap(\text{subgroups})$ 
and a (special case of a) theorem of~Ax 
on analytic subgroups of $\GG_m^r(\CC)$ to prove positivity.
\EndPart
\vspace{5pt}
The proof also requires working on an appropriate compactification 
of the group law map~$\GG_m^r\times\GG_m^r\to\GG_m^r$.
%% Ax's Theorem:
%% Let $A$ be an analytic subgroup of $\GG_m^r(\CC)$, i.e. the image of a
%% $\CC$-vector space under the exponential map. Let $K$ be an
%% irreducible analytic subvariety of an open subset of $\GG_m^r(\CC)$
%% with $1\in K$ and $K\subset X\cap A$. Let $V$ be the Zariski closure
%% of $K$. Then there exists an algebraic subgroup $H\subset\GG_m^r$
%% containing $V$ and satisfying $\dim H\le \dim V+\dim H-\dim K$.
\EndSlide
%%%%%%%%%%%%%%%%%%%%%%%%%%%%%%%%%%%%%%%%%%%%%%%%%%%%%%%%%%%%%%%%%%%%%%

%%%%%%%%%%%%%%%%%%%%%%%%%%%%%%%%%%%%%%%%%%%%%%%%%%%%%%%%%%%%%%%%%%%%%%
\BeginSlide
\SlideTitle{The Case $r=n$}
We sketch an elementary proof of Habegger's theorem when $r=n$.
We thus have a rational map
\[\HighlightDisplay{
  F = (f_1,\ldots,f_n) : \GG_m^n \longrightarrow \GG_m^n.
  }
\]
\EndPart
\vspace{-10pt}
If $F$ is not dominant,  a result of Laurent gives the desired
result.
%% Laurent's Theorem
%% Let $X\subset\GG_m^n$ be a subvariety. Then the Zariski closure of
%% $X\cap\GG_m^n(\CC)_\tors$ is a \emph{finite} union of translates of
%% algebraic subgroups of~$\GG_m^n$ by torsion points.
Let
\[\HighlightDisplay{
  \f : V \dashrightarrow W
  }
\]
be a rational map of varieties. The triangle inequality gives an
elementary upper bound (on an open set)
\[\HighlightDisplay{
  h\bigl(\f(t)\bigr) \ll h(t).
  }
\]
\EndPart
\vspace{-22pt}
\TheoremBox{Proposition}{%
Assume that $\dim V=\dim W$ and that $\f$ is dominant. Then
\[
  h\bigl(\f(t)\bigr) \gg h(t)
  \quad\text{on a nonempty open subset of $V$.}
\]
}
\EndSlide
%%%%%%%%%%%%%%%%%%%%%%%%%%%%%%%%%%%%%%%%%%%%%%%%%%%%%%%%%%%%%%%%%%%%%%

%%%%%%%%%%%%%%%%%%%%%%%%%%%%%%%%%%%%%%%%%%%%%%%%%%%%%%%%%%%%%%%%%%%%%%
\BeginSlide
\SlideTitle{The Case $r=n$ (continued)}
\textit{Proof Sketch}.\enspace
\[\HighlightDisplay{
  \begin{aligned}
    \dim V&=\dim W\text{ and }
    \text{$\f$ dominant}\\
    &\Longrightarrow\quad
    \text{$k(V)/\f^*k(W)$ is a finite extension}.
  \end{aligned}
  }
\]
\EndPart
Let~$f\in k(V)$, so $f$ is a root of
\[\HighlightDisplay{
  X^d+A_1X^{d-1}+\cdots+A_d=0,
  \qquad A_i\in\f^*k(W).
  }
\]
There is an open subset~$U\subset V$ so that
for all $t\in U$,
\[\HighlightDisplay{
  f(t)~\text{is a root of}~
  X^d+A_1(t)X^{d-1}+\cdots+A_d(t)=0.
  }
\]
\EndPart
Standard estimates relating the height of the coefficients of a
polynomial to its roots gives
\[\HighlightDisplay{
  h\bigl(f(t)\bigr) \le h\bigl([1,A_1(t),\ldots,A_d(t)]\bigr) + O(1).
  }
  \qquad(*)
\]
\EndSlide
%%%%%%%%%%%%%%%%%%%%%%%%%%%%%%%%%%%%%%%%%%%%%%%%%%%%%%%%%%%%%%%%%%%%%%

%%%%%%%%%%%%%%%%%%%%%%%%%%%%%%%%%%%%%%%%%%%%%%%%%%%%%%%%%%%%%%%%%%%%%%
\BeginSlide
\SlideTitle{The Case $r=n$ (continued)}
\vspace{-8pt}
Write $A_i=\f^*B_i$ and define
\[\HighlightDisplay{
    \begin{aligned}
      \a &= [1,A_1,\ldots,A_d] : V \dashrightarrow \PP^d,\\
      \b &= [1,B_1,\ldots,B_d] : W \dashrightarrow \PP^d.
    \end{aligned}
  }
\]
Then $(*)$ says
\[\HighlightDisplay{
  h\bigl(f(t)\bigr) \le h\bigl(\a(t)\bigr) + O(1).
  }
\]
\EndPart
\vspace{-8pt}
The elementary height estimate gives
\[\HighlightDisplay{
  h\bigl(\b(x)\bigr) \ll h(x) + O(1).
  }
\]
\EndPart
\vspace{-18pt}
\[\HighlightDisplay{
    \begin{aligned}
      h\bigl(f(t)\bigr) &\le h\bigl(\a(t)\bigr) + O(1) \\
      &= h\bigl(\b\circ\f(t)\bigr) + O(1) &&\text{since $\a=\b\circ\f$,}\\
      &\ll h\bigl(\f(t)\bigr) + O(1).
    \end{aligned}
  }
\]
\EndPart
Now apply this to each of the coordinate functions of some embedding
$V\subset\PP^N$.
\hspace{250pt}$\square$
\EndSlide
%%%%%%%%%%%%%%%%%%%%%%%%%%%%%%%%%%%%%%%%%%%%%%%%%%%%%%%%%%%%%%%%%%%%%%

%%%%%%%%%%%%%%%%%%%%%%%%%%%%%%%%%%%%%%%%%%%%%%%%%%%%%%%%%%%%%%%%%%%%%%
\SetSlideHead{``Very'' Unlikely Intersections}
\SectionTitlePage{``Very''\\[10pt]  Unlikely\\[10pt]  Intersections}
%%%%%%%%%%%%%%%%%%%%%%%%%%%%%%%%%%%%%%%%%%%%%%%%%%%%%%%%%%%%%%%%%%%%%%

%%%%%%%%%%%%%%%%%%%%%%%%%%%%%%%%%%%%%%%%%%%%%%%%%%%%%%%%%%%%%%%%%%%%%%
\BeginSlide
\SlideTitle{Increasing the Codimension}
The codimension condition used to define 
unlikely intersections between subvarieties and subgroups
is set up so that each individual $X\cap H$ is finite.
But since there may be infinitely many~$H$, the exceptional
set consisting of all unlikely intersections
\[\HighlightDisplay{
  \Ecal = X \cap \bigcup_H H
  }
\]
is generally infinite. In this case, one hopes to prove that the points in
$\Ecal$  in a geometrically
described subset have bounded height.
\EndPart
\vspace{10pt}
If we increase the codimension condition by one, then most individual
intersections  $X\cap H$ should be empty, so we might expect
the full exceptional set to be finite.
\EndSlide
%%%%%%%%%%%%%%%%%%%%%%%%%%%%%%%%%%%%%%%%%%%%%%%%%%%%%%%%%%%%%%%%%%%%%%

%%%%%%%%%%%%%%%%%%%%%%%%%%%%%%%%%%%%%%%%%%%%%%%%%%%%%%%%%%%%%%%%%%%%%%
\BeginSlide
\SlideTitle{``Very'' Unlikely Intersections}
Conjectures for \DefTerm{Very Unlikely Intersections} of this sort
were originally formulated by Zilber (2002, for constant families) and
Pink (2005, in general).  Here is a general version.
\EndPart
\TheoremBox{Conjecture}{%
Let $\Gcal\to T$ be a semiabelian scheme over a base~$T$, all defined
over~$\CC$. For any~$d$, let $\Gcal^{[d]}$ be the union of the
semiabelian subschemes of~$\Gcal$ of codimension at least~$d$. Let $X$
be an irreducible closed subvariety of~$\Gcal$. Then
$X^\oa\cap\Gcal^{[\dim X+1]}$ is contained in a finite union of
semiabelian subschemes of~$\Gcal$ of positive codimension.}  
\EndPart
\vspace{10pt}
A number of people (Bombieri, Habegger, Masser, Maurin,
Ratazzi, Remond, Viada, Zannier,\dots) have made progress on this
conjecture in the last few years, especially for constant families.

\EndSlide
%%%%%%%%%%%%%%%%%%%%%%%%%%%%%%%%%%%%%%%%%%%%%%%%%%%%%%%%%%%%%%%%%%%%%%


%%%%%%%%%%%%%%%%%%%%%%%%%%%%%%%%%%%%%%%%%%%%%%%%%%%%%%%%%%%%%%%%%%%%%%
\BeginSlide
\SlideTitle{Very Unlikely Intersections in $\GG_m^r$}
For the constant group scheme
\[\HighlightDisplay{
  \Gcal = \GG_m^r
  }
\]
over a variety~$X$, 
Habegger's upper bound can be combined with:
\TheoremBox{Theorem}{%
(BMZ 2008) For all $B$,
\[
  \bigl\{ P\in X^\oa\cap \Gcal^{[\dim X+1]} : h(P) \le B \bigr\}
%%   \bigl\{ P\in X^\oa\cap \Hcal_{r-n-1} : h(P) \le B \bigr\}
  \quad\text{is finite.}
\]}
\EndPart
\vspace{10pt}
to prove finiteness:
\TheoremBox{Corollary}{%
\[\HighlightDisplay{
    X^\oa\cap\Gcal^{[\dim X+1]}\quad\text{is finite.}
%    X^\oa\cap\Hcal_{r-n-1}\quad\text{is finite.}
  }
\]
}
\EndPart
\vspace{10pt}
For non-constant families, very little  is known.
\EndSlide
%%%%%%%%%%%%%%%%%%%%%%%%%%%%%%%%%%%%%%%%%%%%%%%%%%%%%%%%%%%%%%%%%%%%%%


%%%%%%%%%%%%%%%%%%%%%%%%%%%%%%%%%%%%%%%%%%%%%%%%%%%%%%%%%%%%%%%%%%%%%%
\BeginSlide
\SlideTitle{Very Unlikely Intersections in a Non-Constant Family}
\vspace{-3pt}
Consider the elliptic curve
\[\HighlightDisplay{
  E : y^2 = x(x-1)(x-T)
  }
\]
and the two points
\[\HighlightDisplay{
  P = \bigl(2,\sqrt{4-2T}\,\bigr),\qquad
  Q = \bigl(3,\sqrt{18-6T}\,\bigr).
  }
\]
(We may view $E$ as an elliptic curve over the function field
$\QQ(T,U,V)$, where $U^2=4-2T$ and $V^2=18-6T$.)%
\EndPart
\vspace{2pt}
The original theorem that we discussed says that the set of~$t$ for
which $P_t$ is a torsion point is a set of bounded height, and
similarly for~$Q_t$. What happens if we require that~$P_t$
and~$Q_t$ simultaneously be torsion points?
\EndPart
\vspace{-3pt}
\TheoremBox{Theorem}{%
(Masser--Zannier, 2008)
\[
  \{ t \in \CC : \text{$P_t$ and $Q_t$ are both torsion points} \}
\]
is a finite set.
}
\EndSlide
%%%%%%%%%%%%%%%%%%%%%%%%%%%%%%%%%%%%%%%%%%%%%%%%%%%%%%%%%%%%%%%%%%%%%%

%% %%%%%%%%%%%%%%%%%%%%%%%%%%%%%%%%%%%%%%%%%%%%%%%%%%%%%%%%%%%%%%%%%%%%%%
%% \SetSlideHead{Additional Results and Directions}
%% \SectionTitlePage{Additional Results\\[10pt]  and\\[10pt]  Directions}
%% %%%%%%%%%%%%%%%%%%%%%%%%%%%%%%%%%%%%%%%%%%%%%%%%%%%%%%%%%%%%%%%%%%%%%%

%% %%%%%%%%%%%%%%%%%%%%%%%%%%%%%%%%%%%%%%%%%%%%%%%%%%%%%%%%%%%%%%%%%%%%%%
%% \BeginSlide
%% \SlideTitle{Some Additional Results and Directions}
%% \vspace{-25pt}
%% \BULLETITEM
%% Habegger has also proven results for unlikely intersections in an abelian
%% variety. His results may be reformulated as specialization theorems for
%% a constant family of abelian varieties.
%% \EndPart
%% \BULLETITEM
%% For non-constant families of abelian varieties $A\to T$, there is a
%% result of Masser~(1989) saying that the specialization map
%% $\s_t:A(T)\to A(\Qbar)$ is injective for most $t\in T(\Qbar)$ (in a
%% certain density-theoretic sense). 
%% \EndPart
%% \BULLETITEM
%% For non-constant families of abelian varieties~$A\to T$ over bases~$T$
%% of dimension at least two, we still lack general bounded height
%% results for unlikely specializations,  and
%% we lack finiteness results for very
%% unlikely specializations.
%% \EndSlide
%% %%%%%%%%%%%%%%%%%%%%%%%%%%%%%%%%%%%%%%%%%%%%%%%%%%%%%%%%%%%%%%%%%%%%%%

%%%%%%%%%%%%%%%%%%%%%%%%%%%%%%%%%%%%%%%%%%%%%%%%%%%%%%%%%%%%%%%%%%%%%%
\SetSlideHead{Selected References}
%%%%%%%%%%%%%%%%%%%%%%%%%%%%%%%%%%%%%%%%%%%%%%%%%%%%%%%%%%%%%%%%%%%%%%

%%%%%%%%%%%%%%%%%%%%%%%%%%%%%%%%%%%%%%%%%%%%%%%%%%%%%%%%%%%%%%%%%%%%%%
\BeginSlide
\SlideTitle{Selected References}
%% *** What about my specialization paper. And Neron's 1950s paper?
\vspace{-20pt}
{\Large
\begin{itemize}
\setlength{\itemsep}{-5pt}
\item
J. Ax, Some topics in differential algebraic geometry I: Analytic
subgroups of algebraic groups, \emph{Amer. J. Math.} 94 (1972),
1195--1204.
\item
E. Bombieri, D. Masser, and U. Zannier, Intersecting a curve with
algebraic subgroups of mult.\ groups,
%% algebraic subgroups of multiplicative groups,
\emph{Int. Math. Res. Not.} 20 (1999), 1119--1140.
\item
E. Bombieri, D. Masser, and U. Zannier, Anomalous Subvarieties -
Structure Theorems and Applications, \emph{Int. Math. Res. Not.}
(2007) 1--33.
\item
E. Bombieri, D. Masser, and U. Zannier, Intersecting a plane with
algebraic subgroups of multiplicative groups, \emph{Ann. Scuola Norm.
Super. Pisa Cl. Sci.} (5) 7 (2008), 51--80.
\item
P. Habegger,
Intersecting subvarieties of abelian varieties with algebraic
subgroups of complementary dimension, \emph{Invent. Math.}, to appear.
\item
P. Habegger, On the Bounded Height Conjecture, 
\emph{Int. Math. Res. Not.} (2009), 860--886.
%% \item
%% D. Masser, Specializations of finitely generated subgroups of abelian
%% varieties, \emph{Trans. Amer. Math. Soc.} 311 (1989), 413--424.
\item
D.~Masser and U.~Zannier,
Torsion anomalous points and families of elliptic curves,
\emph{C.~R.~Math. Acad. Sci. Paris} 346 (2008), 491--494.
%% There is also a longer preprint with the same title that gives
%% more details
\item
R.~Pink, A Common Generalization of the Conjectures of Andr\'e--Oort,
Manin--Mumford, and Mordell--Lang, Preprint (April 17, 2005).
\item
Y.-T. Siu, An effective Matsusaka big theorem, \emph{Ann. Inst. Fourier}
43 (1993), no. 5, 1387--1405. 
(See R. Lazarsfeld, \emph{Positivity in Algebraic Geometry I},
Springer, 2004, Theorem 2.2.15.) %% page 143
\item
B.~Zilber, Exponential sums equations and the Schanuel conjecture,
\emph{J. London Math. Soc.} (2) 65 (2002), no. 1, 27--44.
\end{itemize}
}
\EndSlide
%%%%%%%%%%%%%%%%%%%%%%%%%%%%%%%%%%%%%%%%%%%%%%%%%%%%%%%%%%%%%%%%%%%%%%


%%%%%%%%%%%%%%%%%%%%%%%%%%%%%%%%%%%%%%%%%%%%%%%%%%%%%%%%%%%%%%%%%%%%%%
\TitlePage

\end{document}

%%%%%%%%%%%%%%%%%%%%%%%%%%%%%%%%%%%%%%%%%%%%%%%%%%%%%%%%%%%%%%%%%%%%%%
\BeginSlide
\SlideTitle{}

\EndSlide
%%%%%%%%%%%%%%%%%%%%%%%%%%%%%%%%%%%%%%%%%%%%%%%%%%%%%%%%%%%%%%%%%%%%%%
