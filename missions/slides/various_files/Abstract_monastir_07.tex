\centerline{Words and primitive roots}\smallskip
\centerline{Francesco Pappalardi}\bigskip

\noindent\bf ABSTRACT: \rm The famous Artin Conjecture for primitive roots asserts that any non zero rational number $a\neq -1$ which
is not a perfect square, is a primitive root for infinitely many primes.

In 1957, A. Schinzel deduced Artin Conjecture from his Hypothesis H. Later in 1967, C. Hooley deduced a
strong form of Artin Conjecture from the Generalized Riemann Hypothesis. More recently in 1986, R. Heath--Brown,
elaborating on an idea of R. Gupta and R. Murty proved a \it quasi resolution \rm of the Artin Conjecture.
A special case of the Heath--Brown, Gupta, Murty Theorem is that one out of any three given primes is a primitive
root for infinitely many primes.

Over the last 40 years, several generalizations and analogies to the Artin Conjecture have been considered. 
We will review some of them including some recent ones.

The three methods of Schinzel, of Hooley and of Heath--Brown, Gupta, Murty will be outlined explaining their limits and 
disadvantages.

Next we will concentrate on two special problems that belong to the family of \it generalizations and analogies to the Artin Conjecture:\rm
\medskip
\item{1.} \bf The Schinzel--W\'ojcik Problem: \rm given non zero rational numbers $a_1,\ldots,a_r$, determine whether there
are infinitely many primes $p$ such that ord$_pa_1=\cdots=$ ord$_pa_r$.

\item{2.} \bf Trivially transposition invariant words. \rm This problem cames from the Theory of Languages and translates into the
problem of enumerating the primes $p$ for which $p+1$ admits a divisor which
is a primitive root modulo $p$.\medskip

We will analyze these problems proposing some new joint results with several authors.
\bye