\documentclass[a4paper,10pt]{article}
\usepackage{amsfonts,amsmath,amssymb}
\newcommand{\Q}{\mathbb Q}
\newcommand{\N}{\mathbb N}
\usepackage[cm]{fullpage}

\usepackage[utf8x]{inputenc}

\title{Mathematical Tour of West Africa}
\author{Alain Togbe}
\date{July 19 -- 25, 2014}

\pdfinfo{%
  /Title    (Mathematical Tour of West Africa)
  /Author   (Alain Togbe)
  /Creator  (FP)
  /Producer ()
  /Subject  (Number Theory)
  /Keywords ()
}

\begin{document}
\maketitle

\section{Introduction}

\begin{enumerate}
 \item Jorge Jimenez Urroz (Universitat Polytecnica de Catalunia, Barcelona, Espa\~na)
 \item Claude Levesque (Universit\'e Laval \`a Qu\'ebec, Canada)
 \item Francesco Pappalardi (Universit\`a Roma Tre, Italia)
 \item Adriana Salerno (Lewiston, USA)
 \item Alain Togbe (Westville, USA)
 \item Michel Waldschmidt (Universit\'e Pierre et Marie Curie, Paris, France)
\end{enumerate}


\section{Program – Lom\'e - 2014 (July 21, 2014)}

\begin{enumerate}
\item \textsc{Professeur Michel Waldschmidt (Paris, France)}\medskip

\textbf{Titre}: \textit{La constante d'Euler est-elle un nombre rationnel, un nombre alg\'ebrique irrationnel ou bien un nombre transcendant?}

\textbf{R\'esum\'e:} 
\textit{D\'eterminer la nature arithm\'etique de constantes de l'analyse est le plus souvent un probl\`eme difficile, 
tr\`es fr\'equemment on ne connait pas la r\'eponse - c'est le cas pour la constante d'Euler. 
N\'eanmoins, on connaît un certain nombre de propri\'et\'es de cette constante qui vaut approximativement
$$0, 577 215 664 901 532 860 606 512 090 082 402 431 042 1. $$
Nous en d\'ecrirons quelques unes.}\bigskip

\item \textsc{Professeur Francesco Pappalardi (Rome, Italie)} \medskip

\textbf{Titre}: \textit{Propri\'et\'es des r\'eductions de groupes de nombres rationnels}

\textbf{R\'esum\'e:} \textit{Soit $\Gamma$ un sous-groupe multiplicatif de $\mathbb Q^*$ et soit $p$ un nombre 
premier pour lequel la valuation $v_p(x) =  0$, pour tout $x$ dans $G$. Alors le groupe 
$\Gamma_p = \{x (\bmod p): x\in \Gamma\}$ est un bien-d\'efini sous-groupe de $\mathbb F_p^*$. 
Nous allons examiner les diff\'erentes propri\'et\'es de $\Gamma_p$ lorsque $p$ varie 
et nous proposons divers r\'esultats nouveaux dans analogie avec  l'ancienne conjecture de Artin pour les racines primitives.}\medskip

\textbf{Title:} \textit{Properties of reductions of groups of rational numbers}

\textbf{Abstract:} \textit{Let $\Gamma$ be a multiplicative subgroup of $\mathbb Q^*$ and let $p$ be a prime number for 
which the valuation $v_p(x)=0$ for every $x\in\Gamma$. Then the group $\Gamma_p =\{x (\bmod p) : x  \in \Gamma\}$ is a 
well--defined subgroup of $\mathbb F_p^*$. We will consider various properties of $\Gamma_p$ as $p$ 
varies and we will propose various new results in analogy with the old Artin Conjecture for Primitive roots.}
\bigskip

\item \textsc{Professeur Alain Togbe (Westville, USA)} \medskip

\textbf{Titre}: \textit{The $P$-integer conjecture of Pomerance.}

Abstract:  Let $k>1$ be an integer. Moreover, let $\varphi(k)$ denote Euler's totient function and $\omega(k)$ the number of distinct prime divisors of $k$. An integer $k$ is a $P$-integer if the first $\varphi(k)$ primes coprime to $k$ form a reduced residue system modulo $k$. In 1980, Pomerance proved the finiteness of the set of $P$-integers. Moreover, he proposed the following conjecture.
Conjecture: If $k$ is a $P$-integer, then $k\leq 30$.
  In this talk, we will discuss the proof of this conjecture.
Titre : La conjecture de P-entier de Pomerance.
R\'esum\'e: Soit $k>1$ un entier. En outre, soit $\varphi (k)$ la fonction indicatrice d'Euler et $\omega(k)$ le nombre de diviseurs premiers distincts de $k$. Un entier $k$ est un $P$-entier si les premiers $\varphi(k)$ entiers relativement premiers entre eux pour $k$ forment un syst\`eme r\'eduit de r\'esidus modulo $k$. En 1980, Pomerance a  prouv\'e la finitude de l'ensemble des $P$-entiers. En outre, il a propos\'e la conjecture suivante. 
Conjecture: Si $k$ est un $ P$-entier, alors $k \leq 30$. 
  	 Dans cet expos\'e, nous allons discuter du progr\`es jusqu'\`a la preuve de cette conjecture.


\item \textsc{Claude Levesque (Qu\'ebec, Canada)}\medskip

\textbf{Title}: \textit{Congruent numbers  and related topics}

\textbf{Abstract:}  
\textit{A positive natural number $n$ is called a congruent number if $n$ is the area of a right triangle with rational sides. 
In other words, $n$ is congruent if $n = ab/2$, with $a, b, c$ rational numbers verifying $a^2 + b^2 = c^2$. 
It turns out that $n$ is congruent if the elliptic curve
$$E: Y^2 = X^3 - n^2 X$$
over $\Q$ has at least one rational  solution $(x, y)$ with nonzero  rational numbers $x, y$.
We will give other equivalent conditions for n to be a congruent number. We will also  take this opportunity for  
introducing  elliptic curves and give some of their properties.} 
\medskip

\textbf{Titre}: \textit{Sur les nombres congruents  et sujets connexes.}

\textbf{R\'esum\'e}: 
\textit{Un  nombre entier  positif $n$ est dit  un   nombre congruent si $n$  est l'aire d'un 
triangle droit  avec des côt\'es   rationnels. En d'autres mots, $n$ est congruent si  $n = ab/2$,  o\`u  $a$, $b$, $c$  
sont des  nombres rationnels v\'erifiant $a^2 + b^2 = c^2$.  Il s'av\`ere que $n$ est un nombre  
congruent  si la  courbe elliptique sur $\Q$ d\'efinie par 
$$E: Y^2 = X^3 - n^2 X$$
poss\`ede au moins  une  solution $(x,y)$ avec des  nombres    rationnels   $x, y$ non nuls.
Nous donnerons d'autres conditions \'equivalentes  pour que $n$  soit un nombre  congruent. 
Nous profiterons de l'occasion  pour introduire  les  courbes  elliptiques et 
mentionner 
certaines de leurs propri\'et\'es.}\bigskip

\item \textsc{Jorge Jimenez Urroz (Barcelone, Espagne)}\medskip

\textbf{Title}: \textit{Malleability and the factorization of RSA numbers}

\textbf{Abstract:} \textit{Given a number $n$ product of two unknown primes, is it easier to factorize 
if we allow the attacket to factorize another number (coprime to $n$) at his will?}\bigskip

\item \textsc{Adriana Salerno (Lewiston, USA)}\medskip

\textbf{Title}: \textit{Effective computations in arithmetic mirror symmetry}

\textbf{Abstract:} \textit{In this talk, I will talk about computational approaches to the problem of arithmetic mirror symmetry. 
One of the biggest questions facing string theorists is the one of mirror symmetry. In arithmetic mirror symmetry, 
we approach the conjecture from a number theoretic point of view, namely by computing Zeta functions of mirror pairs. 
I will define all of these terms and then explain our work through a couple of examples of families of $K3$ surfaces. 
This is joint work with Xenia de la Ossa, Charles Doran, Tyler Kelly, Stephen Sperber, and Ursula Whitcher. }
\end{enumerate}


\section{Program – Abidjan - 2014 (July 24, 2014)}

\begin{enumerate}
 \item \textsc{Professeur Michel Waldschmidt (Paris, France)}\medskip

\textbf{Titre}: \textit{Introduction \`a la cryptographie}

\textbf{R\'esum\'e}: 
\textit{Apr\`es un rapide survol de l'histoire de la cryptographie, nous pr\'esenterons le cryptosyst\`eme RSA sous une forme tr\`es simplifi\'ee, 
faisant intervenir de nombres de trois chiffres (restes de la division par 1000). Nous terminerons par quelques compl\'ements sur les nombres premiers.} \bigskip

\item \textsc{Professeur Francesco Pappalardi (Rome, Italie)} \medskip

\textbf{Title}: \textit{Introduction to Elliptic Cryptosystems}

\textbf{Abstract}: 
\textit{We will introduce the notion of group of rational points of an elliptic curve defined over a finite field and 
we will review all the basic properties that they satisfy. We will also classify all the possible elliptic curves over the fields with 2 and 3 elements. 
Then we will illustrate the fundamental algorithmic problems related with the Theory and possible solutions. We will conclude with some more examples 
and records related to elliptic curves.}\medskip

\textbf{Titre}: \textit{Introduction aux syst\`emes Cryptoelliptiques}

\textbf{R\'esum\'e}: 
\textit{Nous allons introduire la notion de groupe de points  rationnels d'une courbe elliptique d\'efinie sur un corps 
fini et nous  allons passer en revue toutes les propri\'et\'es de base qu'ils satisfont.  
Nous allons aussi classifier toutes les possibles courbes elliptiques sur les corps de 2 et de 3 \'el\'ements. 
Ensuite, nous allons illustrer  les probl\`emes algorithmiques fondamentaux en rapport avec la th\'eorie  
et des solutions possibles. Nous allons terminer avec d'autres exemples et records li\'es aux  courbes elliptiques.}\bigskip

\item \textsc{Professeur Alain Togbe (Westville, USA)}\medskip

\textbf{Title:} \textit{On Diophantine $m$--tuples}

\textbf{Abstract:} 
\textit{A set of $m$ distinct positive integers $\{a_1, \dots,  a_m\}$ is called a Diophantine $m$-tuple  if $a_i a_j +1$ 
is a perfect square. In general, let $n$ be an integer, a set of $m$ positive integers $\{a_1, \dots,  a_m\}$ is called a 
Diophantine $m$-tuple with the property $D(n)$ or a $D(n)$-$m$-tuple (or a $P_n$-set of size $m$), if $a_i a_j +n$ is a
perfect square. Diophantus studied sets of positive rational numbers with the same property, particularly he found the 
set of four positive rational numbers $\left\{\frac{1}{16}, \frac{33}{16}, \frac{17}{4}, \frac{105}{16}\right\}$. 
But the first Diophantine quadruple was found by Fermat. That is the set $\{1, 3, 8, 120\}$. 
Moreover, Baker and Davenport proved that the set $\{1, 3, 8, 120\}$ cannot be extended to a Diophantine quintuple. 
The problem of extendibility of $P_n$-sets is of a big interest. 
In this talk, we will give a very quick survey of results obtained. 
Finally, we will discuss the conjectures on Diophantine $m$-tuples and the recent progress to solve them.}\medskip

\textbf{Titre:} \textit{Sur les $m$--tuplets diophantiens}

\textbf{R\'esum\'e}: 
\textit{Un ensemble de $m$ les nombres entiers positif distincts $\{a_1, $ $\dots,$  $a_m\}$ est appel\'e un $m$-tuplet diophantien 
si $a_i a_j + 1$ est un carr\'e parfait. En g\'en\'eral, soit $n$ un nombre entier, un ensemble de  $m$ nombres entiers 
positifs $\{a_1, \dots, a_m\}$ est appel\'e un m-tuplet diophantien avec la propri\'et\'e $D(n)$ ou un $D(n)$-$m$-tuplet 
si $a_i a_j + n$ est un carr\'e parfait. Diophantus a \'etudi\'e des ensembles de nombres rationnels positifs avec la m\^eme propri\'et\'e, 
notamment il a trouv\'e l’ensemble de quatre nombres rationnel positif $\left\{\frac{1}{16}, \frac{33}{16}, \frac{17}{4}, \frac{105}{16}\right\}$. 
Mais le premier quadruplet diophantien a \'et\'e trouv\'e par Fermat. Cela est l’ensemble $\{1, 3, 8, 120\}$. 
De plus, Baker et Davenport ont prouv\'e que l’ensemble $\{1, 3, 8, 120\}$ ne peut pas \^etre \'etendu \`a un quintuplet diophantien. 
Le probl\`eme d'extension de $P_n$-ensembles est d'un grand int\'er\^et. 
Dans cet expos\'e, nous allons donner un aperçu tr\`es rapide quelques r\'esultats obtenus. Enfin, nous allons discuter des 
conjectures sur $m$--tuplets diophantiens et les progr\`es r\'ecents pour les r\'esoudre.}\bigskip

\item \textsc{Claude Levesque (Qu\'ebec, Canada)}\medskip

\textbf{Title}: \textit{Congruent numbers  and related topics}

\textbf{Abstract:}  
\textit{A positive natural number $n$ is called a congruent number if $n$ is the area of a right triangle with rational sides. 
In other words, $n$ is congruent if $n = ab/2$, with $a, b, c$ rational numbers verifying $a^2 + b^2 = c^2$. 
It turns out that $n$ is congruent if the elliptic curve
$$E: Y^2 = X^3 - n^2 X$$
over $\Q$ has at least one rational  solution $(x, y)$ with nonzero  rational numbers $x, y$.
We will give other equivalent conditions for n to be a congruent number. We will also  take this opportunity for  
introducing  elliptic curves and give some of their properties.} 
\medskip

\textbf{Titre}: \textit{Sur les nombres congruents  et sujets connexes.}

\textbf{R\'esum\'e}: 
\textit{Un  nombre entier  positif $n$ est dit  un   nombre congruent si $n$  est l'aire d'un 
triangle droit  avec des côt\'es   rationnels. En d'autres mots, $n$ est congruent si  $n = ab/2$,  o\`u  $a$, $b$, $c$  
sont des  nombres rationnels v\'erifiant $a^2 + b^2 = c^2$.  Il s'av\`ere que $n$ est un nombre  
congruent  si la  courbe elliptique sur $\Q$ d\'efinie par 
$$E: Y^2 = X^3 - n^2 X$$
poss\`ede au moins  une  solution $(x,y)$ avec des  nombres    rationnels   $x, y$ non nuls.
Nous donnerons d'autres conditions \'equivalentes  pour que $n$  soit un nombre  congruent. 
Nous profiterons de l'occasion  pour introduire  les  courbes  elliptiques et 
mentionner 
certaines de leurs propri\'et\'es.}\bigskip

\item \textsc{Jorge Jimenez Urroz (Barcelone, Espagne)}\medskip

\textbf{Title}: \textit{Linear secret sharing schemes and algebraic curves}

\textbf{Abstract:} 
\textit{A linear secret sharing scheme is a way of sharing a secret amount a set 
of participants with different hierarchy. We will introduce the concept, 
talk about the Shamir threshold scheme, and then generalize it to elliptic 
and hyperelliptic curves (the last part depending on time.)}\bigskip\bigskip

\end{enumerate}

\end{document}
