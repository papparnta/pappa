\documentclass[a4paper,12pt]{article}
\usepackage[utf8]{inputenc}

%opening
\title{ALGEBRAIC NUMBER THEORY -- FIELDS \& GALOIS THEORY\\
Tribhuvan University}
\author{written test: 90 minutes -- October 31, 2014\\ Lectures by Francesco Pappalardi}

\begin{document}
\maketitle
\thispagestyle{empty}

Solve the maximun number of the following exercises:

\begin{enumerate}
 \item Choose one element $d$ in the set $\{3,6,7,10,11,14,15,19,22\}$ and show that the group 
 of the units $(\textbf{Z}[\sqrt{d}])^*$ is infinite.\bigskip
 \item Consider the stem field $\textbf{Q}[\alpha], \alpha^3=-2\alpha+1$. Compute $a, b, c\in\textbf{Q}$ such that
 $$\frac{1}{\alpha^2+1}=a+b\alpha+c\alpha^2$$\bigskip
 Can you produce infinitely many units in the ring $\textbf{Z}[\alpha]$?
 \item Write the definition of stem--field, write a stem for the field $\textbf{Q}[\sqrt{3}-\sqrt[3]{5}]$ 
 and prove that $\textbf{Q}[\sqrt{3},\sqrt[3]{5}]=\textbf{Q}[\sqrt{3}-\sqrt[3]{5}]$.\bigskip
 \item Prove that $\sqrt{-20}\in\textbf{Q}[\sqrt{5}+i]$\bigskip
 \item Give the definition of Galois field.
\end{enumerate}


\end{document}
