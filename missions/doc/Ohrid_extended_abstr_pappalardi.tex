\documentclass[a4paper,10pt]{article}
\usepackage[utf8]{inputenc}
\usepackage{amsmath,amssymb,amsfonts,amsthm}
\usepackage[cm]{fullpage}
\newcommand{\Z}{\mathbb{Z}}
\newcommand{\N}{\mathbb{N}}
\newcommand{\F}{\mathbb{F}}
\newcommand{\Q}{\mathbb{Q}}
\newtheorem*{theorem}{Theorem}
\newtheorem*{corollary}{Corollary}

%opening
\title{Galois representations on torsion points of elliptic curves\\ NATO ASI 2014 -- 
Arithmetic of Hyperelliptic Curves and Cryptography}
%, the former Yugoslav Republic of Macedonia
\author{Francesco Pappalardi}
\date{Ohrid, August 25 - September 5, 2014}
\begin{document}

\maketitle

%\begin{abstract}
%later today
%\end{abstract}

\section{Lecture 1 - Introduction }

Let $E/\Q$ be an elliptic curve. It is well known since the time of Jacobi that, for any field extension $K/\Q$, the set
$E(K)$ of projective $K$--rational points has a natural group structure. Furthermore, if $K/\Q$ is finite, $E(K)$, with respect to this group structure, 
is finitely generated.

For any integer $n$, we consider the kernel of the \emph{multiplication-by-$n$ map},
$$E[n]=\{P\in E(\overline{\Q}): nP=\infty\}$$
which is called the \emph{$n$--torsion subgroup}. It is part of the classical theory the fact that 
if $n$ is an odd integer and $P(x,y,1)\in E(\overline{\Q})$
is non zero, then $P\in E[n]$ if and only if $x$ is a root of the \emph{$n$--division polynomial} $\psi_n(X)\in\Z[X]$
which are defined by recursive formulas and are separable. Furthermore
$$\deg\psi_n=\frac{n^2-1}2.$$
This implies easily that if $n$ is odd, then 
\begin{equation}\label{uno}
E[n]\cong \Z/n\Z\times\Z/n\Z. 
\end{equation}
A similar argument allows to conclude that (\ref{uno}) holds also for $n$ even.
We set 
$$E[\infty]=\bigcup_{n\in\N}E[n]$$
which is the \emph{torsion subgroup} of $E(\overline{\Q})$.
In virtue of (\ref{uno}), we have that
$$\operatorname{Aut}(E[n])\cong\operatorname{GL}_2(\Z/n\Z).$$
So, there is a profinite group structure
$$\operatorname{Aut}(E[\infty])\cong\operatorname{GL}_2(\hat\Z)\quad\text{where}\quad\hat\Z=\lim_{\leftarrow}\Z/n\Z.$$

% For a prime $\ell$, the $\ell$--adic \emph{Tate module} of $E$ is defined by the inverse limit
%     $$T_\ell(E)=\underset{\longleftarrow}{\lim} E[\ell^n]$$
% over positive integers $n$ with transition morphisms given by the multiplication-by-$p$ map $E[\ell^{n+1}] \rightarrow E[\ell^n]$. 
% Thus, the Tate module encodes all the $\ell$-power torsion of $E$. So, we  can think at the elements of $T_\ell(E)$ as compatible sequences 
% $(P_n)_{n\in\Z}$ where $P_n\in E[\ell^n]$ and $\ell\cdot P_{n+1}=P_n$. Hence $T_\ell(E)$ is equipped with the structure of $\Z_\ell$--module 
% ($\Z_\ell$ is the ring of $\ell$--adic integers) via
% the multiplication 
% $$z\cdot(P_n)_{n\in\N}=((z \bmod \ell^n)P_n)_{n\in\N}.$$
% Furthermore $T_\ell(E)$ is a free $\Z_\ell$--module of rank $2$ So that 
% $$\operatorname{Aut}T_\ell(E)\cong\operatorname{GL}_2(\Z_\ell).$$
% However there is an extra very powerful structure that can be defined on $T_\ell(E)$.
    
The absolute Galois group 
$$G_\Q:=\operatorname{Gal}(\overline{\Q}/\Q)=\{\sigma:\overline{\Q}\rightarrow\overline{\Q},\text{ field automorphism}\}$$ 
is also a profinite group and if $K$ is any Galois extension of $\Q$, then
$$\operatorname{Gal}(K/\Q)\cong G_\Q/\{\sigma\in G_\Q: \sigma_{|_K}=\text{id}_K\}.$$
So $G_\Q$ admits as quotient any possible Galois Group of Galois extensions of $\Q$ and it is the projective limit of its finite quotients.
%Furthermore, any group homomorphism $\operatorname{Gal}(K/\Q)\rightarrow G$ can be extended to a group homomorphism
%$$G_\Q\rightarrow G$$
%in such a way that it is trivial if restricted to $\{\sigma: \sigma_{|_K}=\text{id}_K\}$

For every integer $n$, we consider the \emph{$n$--torsion field} $\Q(E[n])$ obtained by adjoining to $\Q$ all coordinates of
all non zero points in $E[n]$. Finally we use $G(n)$ to denote the Galois group $\operatorname{Gal}(\Q(E[n])/\Q)$. 

If $P=(x,y,1)\in E[n]$ and $\sigma\in G(n)$, then $\sigma P:=(\sigma x,\sigma y,1)\in E[n]$. This property and the fact that the operation in $E(\overline{\Q})$
is defined by $\Q$--rational functions, provides us with an inclusion
$$\rho_n: G(n)\hookrightarrow\operatorname{Aut}(E[n])\cong \operatorname{GL}_2(\Z/n\Z)$$
which can be extended to
$$\rho: G_\Q\longrightarrow \operatorname{Aut}(E[\infty])=\prod_{\ell\text{ prime}}\operatorname{Aut}(E[\ell^\infty])
\cong \prod_{\ell\text{ prime}}\operatorname{GL}_2(\Z_\ell).$$
where $E[\ell^\infty]=\cup_{m\in\N}E[\ell^m]$ and $\Z_\ell$ denoted the ring of \emph{$\ell$--adic integers}.
The above representation is an object of study during these three lectures. 

The main result of the Theory is

\begin{theorem}[Serre's Uniformity Theorem] If $E$ does not have complex multiplication (i.e. the only homomorphism
$E(\overline{\Q})\rightarrow E(\overline{\Q})$ which are defined by rational maps are the \emph{multiplication-by-$n$ maps}), 
then the index of $\rho_n(G(n))$ inside $\operatorname{Aut}(E[n])$ is bounded by a constant that depends
only on $E$. 
\end{theorem}

This statement has several striking consequences among which:
\begin{corollary} If $E$ does not have complex multiplication, then  for all $\ell$ large enough 
$$G(\ell)= \operatorname{Aut}(E[\ell])$$
\end{corollary}

One of the central problem of this theory is to establish explicit bounds for $\ell$ for which the conclusion of 
the above Corollary holds. It is believed that it holds for all $\ell>37$.

The group homomorphism
$$\rho_{\ell^\infty}: G_\Q\rightarrow\operatorname{Aut}(E[\ell^\infty])$$
obtained by composing $\rho$ with the projection on the $\ell$-th component,
is actually a continuous homomorphism of topological groups and it is called \emph{$\ell$--adic representation}.

The representation $\rho_{\ell^\infty}$ is \emph{unramified} at all primes $p\nmid \ell\Delta_E$ in the sense that for such primes
$\rho_\ell|_{I_\mathfrak p}=\operatorname{Id}_{\Z_\ell}$
where, for a fixed prime number $p$ and a fixed prime $\mathfrak p$ of $\bar{\Q}$ over $p$, one defines the \emph{inertia subgroup} 
$I_\mathfrak p\subset G_\Q$ as the set of those elements of $G_\Q$ such that
$$\sigma(x)\equiv x\bmod\mathfrak p,\quad \forall x\in\bar{\Z}.$$

Serre's Uniformity Theorem is equivalent to the conjunction of the following two statements:
\begin{itemize}
 \item For all primes $\ell$, $\rho_{\ell^\infty}(G_\Q)$ is an open subgroup with respect to the $\ell$--adic topology,
 \item For all but finitely many primes $\ell$, $\rho_{\ell^\infty}(G_\Q)=\operatorname{Aut}(E[\ell^\infty])$.
\end{itemize}

An important tool in the study of the above representations is the \emph{Frobenius element}. In general, in a Galois extension $K$
of $\Q$, for an unramified prime $p$, one defines the Frobenius element as any element in the conjugation class of the Galois Group 
$\operatorname{Gal}(K/\Q)$ which is determined by the lift of the Frobenius automorphism of the finite field 
$\mathcal O_K/\mathcal P$ obtained as a quotient of the ring of integers $\mathcal O$ by any 
prime ideal $\mathcal P$ over $p$. Sometimes one calls \emph{Artin symbol}, the conjugation class itself and denotes it
by $\left[\frac{K/\Q}p\right]$.

In the case of the division fields $\Q(E[n])$, the Artin symbol can be thought as a conjugation class of matrices in $\operatorname{GL}_2(\Z/n\Z)$.
The characteristic polynomial 
$\det(\left[\frac{\Q(E[n])/\Q}p\right]-T)$
turns out not to depend on $n$ in the sense that
$$\det\left(\left[\frac{\Q(E[n])/\Q}p\right]\right)\equiv p\bmod n,\qquad \operatorname{tr}\left(\left[\frac{\Q(E[n])/\Q}p\right]\right)\equiv a_E\bmod n$$
where $a_E=p-1-\#E(\F_p)$.

During the first lecture we will introduce the above notions and explain some of their properties.

\section{Lecture 2 - Serre's Open Mapping Theorem and its applications}

In most of Lecture 2 we will assume that $E$ has no complex multiplication. During this lecture we will introduce more tools and notions necessary
for later applications.

\subsection{Chebotarev Density Theorem}

If $K/\Q$ is a finite Galois extension and $\mathcal C\subset\operatorname{Gal}(K/\Q)$ is a union if conjugation classes of 
$\mathcal G=\operatorname{Gal}(K/\Q)$, then the \emph{Chebotarev Density Theorem} predicts that 
the density of the primes $p$ such that the Artin symbol
$\left[\frac{K/\Q}p\right]\subset \mathcal C$ equals $\frac{\#\mathcal C}{\#\mathcal G}.$
The Chebotarev Density Theorem has also a quantitative versions. Let
$$\pi_{\mathcal C/\mathcal G}(x):=\#\left\{p\le x: \left[\frac{K/\Q}p\right]\subset \mathcal C\right\}.$$
Then (see Serre \cite{S3} and Murty, Murty \& Saradha \cite{MMS}), assuming that the Dedekind zeta function of $K$ satisfies the \emph{Generalized Riemann Hypothesis}, 
$$\pi_{\mathcal C/\mathcal G}(x)=
\frac{\#\mathcal C}{\#\mathcal G}\int_2^x\frac{dt}{\log x}+O\left(\sqrt{\#\mathcal C}\sqrt{x}\log(xM\#\mathcal G)\right)$$
where $M$ is the product of primes numbers that ramify in $K/\Q$. An analogue version, independent on the Generalized Riemann Hypothesis can be
found in \cite{S3}.

We will apply it in the special case when $K=\Q(E[n])$ where we think at the element of $\mathcal G$ as 2 by 2 non singular matrices. For example
\begin{itemize}
 \item In the case when $\mathcal C=\{\operatorname{id}\}$, the condition $\left[\frac{\Q(E[n])/\Q}p\right]=\{\operatorname{id}\}$ is equivalent
to the property that 
$$E[n]\subset \bar{E}(\F_p)$$
where $\overline E(\F_p)$ is the group of $\F_p$-rational points on the reduced curve $\overline E$.
\item In the case when $\mathcal C=\mathcal G_{\text{tr}=r}=\{\sigma\in\mathcal G:\operatorname{tr}\sigma=t\}$, and $\ell$ is a sufficiently large prime so that $\operatorname{Gal}(\Q(E[\ell])/\Q)=
\operatorname{GL}_2(\F_\ell)$, then
$$\#\operatorname{GL}_2(\F_\ell)_{\text{tr}=r}=\begin{cases}
                   \ell^2(\ell-1) & \text{if } r=0\\
                   \ell(\ell^2-\ell-1) & \text{otherwise.}
                  \end{cases}
$$
\end{itemize}
These examples will be elaborated during Lecture~3.

\subsection{Classification of possible subgroups of $\operatorname{GL}_2(\F_\ell)$ that can appear as image of Galois}
Part of the work of Serre consists in classifying the possible images of $G(\ell)$. More precisely, Serre proved in \cite{S2} 
that $\rho_\ell(G_\Q)$ contains a subgroup of one of the 
following types:
\begin{enumerate}
 \item ``split half Cartan subgroup'': A cyclic subgroup of of $\ell-1$ which can be represented as
 $$\left\{\begin{pmatrix} a&0\\ 0&1            
           \end{pmatrix}: a\in\F_\ell^*\right\},$$
 \item ``half Borel subgroup'': A solvable group that can be represented as
 $$\left\{\begin{pmatrix} a&0\\ 0&b            
           \end{pmatrix}: a\in\F_\ell^*, b\in\F_\ell\right\},$$ 
 \item ``non split half Cartan subgroup'': A cyclic subgroup of of $\ell^2-1$.
\end{enumerate}
Furthermore if $\rho_\ell(G_\Q)\neq\operatorname{GL}_2(\F_\ell)$, then one of the following happens:

\begin{itemize}
 \item $\rho_\ell(G_\Q)$ is either contained in a Cartan subgroup or a Borel subgroup (upper
triangular matrices) of $\operatorname{GL}_2(\F_\ell)$,
 \item $\rho_\ell(G_\Q)$ is contained in the normalizer of a Cartan subgroup and it is not contained in the Cartan subgroup
 of $\operatorname{GL}_2(\F_\ell)$.
\end{itemize}
We will conclude with some explicit examples.
\subsection{The Definition of Serre's Curve}

It is in general not easy to compute the image $\rho(G_\Q)\subset\operatorname{GL}_2(\hat\Z).$ Actually, it was showed by Serre that
$\rho(G_\Q)$ is always contained in an index $2$ subgroup of $\operatorname{GL}_2(\hat\Z)$. 
Such subgroup is called the \emph{Serre's Subgroup} $\mathcal H_E$ and it 
is defined as 
$$\mathcal H_E=\pi_{m_E}^{-1}(H_{m_E})$$ where 
\begin{itemize}
 \item $\pi_m:\operatorname{GL}_2(\hat\Z)\rightarrow\operatorname{GL}_2(\Z/m\Z)$ is the natural projection,
 \item $m_E$ is the \emph{Serre number of $E$} defined as the least common multiple $[2,\operatorname{disc}(\Q(\sqrt{|\Delta_E|}))]$,
 \item and if $\varepsilon$ denotes the \emph{signature map} (i.e.
$\varepsilon:\operatorname{GL}_2(\Z/m\Z)\rightarrow\operatorname{GL}_2(\Z/2\Z)\cong S_3\rightarrow\{\pm1\}$), then
$$H_m=\left\{\sigma\in\operatorname{GL}_2(\Z/m\Z): \varepsilon(A)=\left(\frac{\Delta_E}{\det A}\right)\right\}.$$
 \end{itemize}
 
An elliptic curve $E/\Q$ is called \emph{a Serre curve} if $\rho(G_\Q)=\mathcal H_E$. These curves are quite common and will be considered in 
the third lecture.

\section{Lecture 3 - The Lang--Trotter Conjectures}

The third lecture is devoted to reviews of some applications of $\ell$--adic representations
to number Theory and in particular to the Lang--Trotter Conjectures. 

\subsection{Lang Trotter for primitive points}

E. Artin made a celebrated conjecture concerning the density of primes for which a given 
 integer is a primitive root. 
  In the first part of this lecture, we will discuss an analogous conjecture for elliptic curves.
   Let $E$ be an elliptic curve defined over $\Q$ with $P\in E(\Q)$ a $\Q$-rational point 
   of infinite order. $P$ is called \emph{primitive} for a prime $p$ if the reduction $\overline P$ of $P\bmod p$ 
   generates the entire group $\overline E(\F_p)$ of $\F_p$-rational points on the reduced curve $\overline E$.
   
We set $$\pi_{E,P}(x)=\#\{p\le x: p\nmid\Delta_E\text{ and } P\text{ is primitive for } p\}.$$

In 1976, Lang and Trotter in \cite{LT1} conjecture an asymptotic formula for $\pi_{E,P}(x)$ and consequently 
an expression for the density of primes for which $P$ is primitive.
More precisely, they conjecture that
$$\pi_{E,P}(x)\sim \delta_{E,P}\frac x{\log x}\qquad x\rightarrow\infty.$$
where 
$$\delta_{E,P}=\sum_{n=1}^\infty\mu(n)\frac{\#\mathcal C_{P,n}}{\#\operatorname{Gal}(\Q(E[n],n^{-1}P)/\Q)}$$
where $\Q(E[n],n^{-1}P)$ is the extension of $\Q(E[n])$ obtained with all the coordinates of the points
$Q\in E(\bar{\Q})$ such that $nQ=P$ and $\mathcal C_{P,n}$ are suitable defined union of conjugacy classes in 
$\operatorname{Gal}(\Q(E[n],n^{-1}P)/\Q)$.

The heuristic argument is based on the Chebotarev Density Theorem.
%   Let $K_\ell=\Q(E[\ell],\ell^{-1}P)$. The affine group equal to the extension of the 
%    translation group $E[\ell]$ by $\operatorname{GL}_2(\F_\ell)$ operates on $\ell^{-1}P$. 
%    
%    Fix $u_0\in \ell^{-1}P$ and represent $\sigma$ in this 
%    affine group by $(\gamma,\tau)$ with $\gamma\in\operatorname{GL}_2(\F_\ell)$ and $\tau\in E[\ell]$ 
%    such that $(\gamma,\tau)u=u_0+\gamma(u-u_0)+\tau$. 
%   
%   Assume $p\nmid\Delta_E\ell$ where $\Delta_E$ is the discriminant of $E$. 
%   Let $S_\ell=G_\ell-S_\ell{}'$ where $S_\ell{}'$ denotes the set of all $(\gamma,\tau)$ 
%    such that $\gamma$ has eigenvalue 1 and either $\gamma=1$ or 
%    $\operatorname{Ker}(\gamma-1)$ is cyclic and $\tau\in\operatorname{Im}(\gamma-1)$. 
%    The index of the subgroup generated by $\overline P$ is divisible by $\ell\leftrightarrow$ the Frobenius lies in $S_\ell{}'$.
%    For any set of primes $L$, let $S_L=\prod_{\ell\in L}S_\ell$, let $K_L$ be the compositum of all $K_\ell$ with $\ell\in L$, let 
%    $G_L=\operatorname{Gal}(K_L/\Q)$, let $P_{L,S}(x)$ be the set of primes $p\leq x$ such that $p\nmid\Delta_E$ and such that the Frobenius 
%    $(p,K_\ell/\Q)\in S_\ell$, $\forall \ell\in L$, $\ell\neq p$. 
%    Chebotarev's density theorem implies that 
%    $$\delta_L(S)=\lim_{x\rightarrow\infty}|P_{L,S}(x)|/\pi(x)=|S_L\cap G_L|/|G_L|$$ 
%    for $L$ finite. 
%    
%    As $L$ increases to include all primes, $\delta(S)=\lim_L\delta_L(S)$ exists. 
%    The conjecture is then that $\delta(S)$ is equal to the density of primes for which $P$ is primitive. 
%    A number of examples, numerical computations, and generalizations will be discussed.
   The Lang--Trotter conjecture for primite points is still not known in any case. The Generalized Riemann Hypothesis allows to deduce some analogue conjectures 
   for CM elliptic curves. We will discuss some of the known results and in particular those due to Gupta, Murty and Murty \cite{GM}
   
\subsection{Serre's Cyclicity Conjecture}
J. P. Serre has formulated a conjecture with a similar flavor. Let $E/\Q$ be an elliptic curve and let 
$$\pi_E^{\text{cyclic}}(x)=\#\{p\le x:\overline E(\F_p)\text{ is cyclic}\}.$$
The conjecture postulates the validity of the asymptotic formula:
$$\pi_E^{\text{cyclic}}(x)\sim\delta_E^{\text{cyclic}}\frac x{\log x}\qquad x\rightarrow\infty$$
where 
$$\delta_E^{\text{cyclic}}=\sum_{n=1}^\infty\frac{\mu(n)}{\operatorname{Gal}(\Q(E[n])/\Q)}.$$
Serre himself applied the Chebotarev Density Theorem, in analogy with the Hooley's work for Artin's Conjecture, 
and proved this conjecture as a consequence of the Generalized Riemann Hypothesis.
Furthermore, if $E$ has no CM,  $\delta_E^{\text{cyclic}}$ is a rational multiple of the quantity
$$\prod_\ell\left(1-\frac{1}{(\ell^2-\ell)(\ell^2-1)}\right).$$
We will discuss this result and several more due to Gupta and Murty \cite{GM2} and to A. Cojocaru \cite{C}.
 
\subsection{Lang Trotter for fixed trace of Frobenius}
In an earlier publication \cite{LT2}, Lang and Trotter considered, for  a fixed elliptic curve $E/\Q$ and an integer $r$, the function
$$\pi_E^r(x)=\#\{p\le x: p\nmid\Delta_E\text{ and } \#\overline E(\F_p)=p+1-r\}$$
and they conjecture that if either $r\ne0$ or if $E$ has no CM, then
$$\pi_E^r(x)\sim C_{E,r}\frac{\sqrt{x}}{\log x}\qquad x\rightarrow\infty$$
where $C_{E,r}$ is the so--called \emph{Lang--Trotter constant} which is defined as follows:
$$C_{E,r}=\frac2\pi\lim_{m\rightarrow\infty}\frac{K_m\#\operatorname{Gal}\Q(E[K_m])/\Q)_{\text{trace}=r}}{\#\operatorname{Gal}\Q(E[K_m])/\Q)}$$
where $K_m$ a sequence of integers with the property that every integer divides $K_m$ when $m$ is large enough. For example $k_m=m!$  
has this property.

In virtue of the Open Mapping Theorem, we know that there exists an integer $N_E$, called the \emph{torsion conductor} such that 
$$C_{E,r}=\frac2\pi\frac{N_E\#\operatorname{Gal}\Q(E[N_E])/\Q)_{\text{trace}=r}}{\#\operatorname{Gal}\Q(E[N_E])/\Q)}
\times\prod_{\ell\nmid N_E}\frac{\ell\#\operatorname{GL}_2(\F_\ell)_{\text{tr}=r}}{\#\operatorname{GL}_2(\F_\ell)} $$
 
 
As an application of the theory of $\ell$--adic representations and of the Chebotarev density Theorem, assuming the Generalized Riemann 
Hypoythesis, Serre in \cite{S3} showed that
$$\pi_E^r(x)\ll\begin{cases} x^{7/8}(\log x)^{-1/2}&\text{if}\ r\ne0\\ x^{3/4}&\text{if}\ r=0.\end{cases}$$  

These results were improved for $r\ne0$ by Murty, Murty and Sharadha \cite{MMS} that showed, assuming the Gereralized Riemann Hypothsis, that
$\pi_E^r(x)\ll x^{4/5}/(\log x).$

%    Let $K/{\bf Q}$ be an (infinite) Galois extension with Galois group of $G$, let
%     $\rho\colon G\rightarrow\times\text{GL}_2({\bf Z}_\ell)$ be representation 
%     and let $\rho_l$ be the projection on the $\ell$-th factor. 
%     From $\rho$ is required: 
%     \begin{enumerate}
%      \item  $\rho(G)$ is an open subset of $\times\text{GL}_2({\bf Z}_\ell)$ ($\ell$ prime);
%  \item There is an integer $\Delta$, such that for all primes $p$ with $p\nmid\Delta\cdot\ell$, $\rho_\ell$ is unramified at $p$
%     \end{enumerate}
%     Property (2) makes it possible to consider the Frobenius class $\sigma_p$ in 
%     $G/\text{Ker}\rho_l$. Let $X^2-t_pX+p$ be the characteristic polynomial $\rho_l(\sigma_p)$. 
%   It is well known that (3) $t_p$ is an integer, which is independent of $\ell$ ($t_p =$ trace of Frobenius), and that the roots
% of the characteristic polynomial have the absolute value less than $\sqrt{p}$ and are complex conjugate to each other.
% We denote by $\pi_p$ one such a root. 
% We can also consider the extension $K$ of $\Q$, obtained by adjoining the torsion points of an ellipti curve over $\Q$ with out
% complex multiplication \cite{S2}.
% Let $t_0\in{\bf Z}$, $k$ an imaginary quadratic number field, $x\in{\bf R}$. 
% Let $N_{t_{0,\rho}}(x)=$ number of primes $p\leq x$ with $t_p=t_0$ and 
% $N_{k,\rho}(x)=$ number of primes $p\leq x$ with $k={\bf Q}(\pi_p)$. Sia $\pi_{1/2}(x)=\sum_{p\leq x}1/(2\surd p)$..
% 
% Then the following conjecture is made: There is a constant $C(t_0,\rho)$ and a constant $C(K,\rho)$
% such that: $N_{t_{0,\rho}}(x)\sim C(t_{0,\rho})\cdot\pi_{1/2}(x)$ and $N_{k,\rho}(x)\sim C(k,\rho)\cdot\pi_{1/2}(x)$.
% 
% To justify these assumptions, the simplest possible probabilistic models with the sequences $\{t_p\}$ are considered as random sequences,
% o that almost all the episodes show an asymptotic behavior that is consistent with the density of sets of Chebotarev, Hecke 
% and the Sato-Tate Conjecture.
% 
% Example (Part I,$\S$ $\S$3, 4):): Let $M$ be a positive number, $G(M)$ the reduction of 
% $\rho(G)\,\text{mod}\,M$, $G(M)_t=\{\sigma\in G(M)$, $\text{tr}(\sigma)\equiv t\,\text{mod}\,M\}$. Let $F_M(t)=M\cdot|G(M)_t|/|G(M)|$.
% 
% Then, by Chebotarev $F_M(t)/M$ is the density of primes $p$ such that 
% $t_p\equiv t\text{mod}\,M$. It is further assumed that a density function $g(\xi)$ exists which is equal to 0 outside $[-1,1]$, 
% the integral of the density of primes $p$ measures the $\xi(t_p,p):= t_p/(2\surd p)$ is in a certain interval.
% If $\rho$ comes from an elliptic curve, we take $g(\xi)=(2/\pi)\surd(1-\xi^2)$.    
%       
% The main assumption with respect to the probability model is then that the function $f_M(t,p)=c_pg(\xi(t,p)) F_M(t)$ abbia misura
% $\mu_p$ in the fiber through $p$ of the probability space representing, wherein $c_p$ is determined so that 
% $\sum_tf_M(t,p)=1$.
% 
% It follows: $c_p\sim 1/(2\surd p)$.
% 
%   In $\S4$ it is shown that $\lim_MF_M(t_0):= F(t_0)$ exists for $p\rightarrow\infty$ 
%   goes $c_p\cdot g(\xi,t_0)$ against $(1/(2\surd p))\cdot g(0)$.
%   Substituting $C(t_{0,\rho})=g(0)\cdot F(t_0)$, one has for ``almost all'' sequences 
%   $\{t_p\}\colon N_{t_0}(x)\sim C(t_{0,\rho})\pi_{1/2}(x)$.
% 
% For further determination of $C(t_{0,\rho})$, $F(t_0)$ 
% is written as a product of local factors, the determination can be achieved $\rho$ 
% by an accurate investigation of the special presentation.
% 
% For the so-called "Serre curves'' (the image of $\rho$ is as large as possible) and for $X_0(11)$, 
% this is in \S \S 5-8, Part I, using and further development of techniques of J.-P. Serre \cite{S1} 
% and G. Shimura [J. Reine Angew Math 221 (1966), 209-220.;
% MR0188198 (\# 32 5637)] carried out. 
% 
% Much more complicated is the situation with the statement about imaginärquadratische field $k$ (Part II), 
% especially when the maximum abelian extension of $k$ with $K$ has an average which is truly greater than the bodies of all roots of unity.
%      
% To handle these cases, in Part III special ingenious calculations are performed. 
% The resulting constant $C(k,\rho)$ is inversely proportional to  $\surd|D|$ ($D=$ discriminant $k$ ) 
% is again the product of local factors is dependent almost all $\ell$ only of $D$ and $\ell$, and an ``infinite'' 
% factor resulting from the Sato Tate distribution.
%  
% The local factors can be represented as $\ell$-adic integrals of certain functions that are Harish transform ($\S$ $\S$7, 8, Part II). 
%     
% A numerical discussion on Serre curves is in the fourth part (for the first 5000 prime numbers) and $X_0(11)$  (with even further calculations), 
% where a satisfactory agreement between theory and practice can be determined.    

\subsection{Average Lang--Trotter}

For every integer $a,b$ such that $4a^3+27b^2\ne0$, we let
$$E(a,b):y^2=x^3+ax+b.$$
We will conclude the lecture with a discussion of the following statement which appeared in \cite{DP}.
 Let $r$ be an integer, $A,B>1$. For every $c > 0$ we have
$$\frac{1}{4AB}\sum_{|a|\le A,|b|\le B}
\pi^r_{E(a,b)}(x)=C_r\int_2^x\frac{dt}{\sqrt{t}\log t}+
O\left(\left(\frac1A+\frac1B\right)x^{3/2}+\frac{x^{5/2}}{AB}+\frac{\sqrt x}{\log^cx}
\right).$$
where
$$C_r=\frac2\pi\prod_{\ell}\frac{\#\operatorname{GL}_2(\F_\ell)_{\text{tr}=r}}{\#\operatorname{GL}_2(\F_\ell)}.$$

\begin{thebibliography}{2}
\bibitem{C}
\textsc{Cojocaru, Alina Carmen}, 
\textit{Cyclicity pf CM Elliptic Curves modulo $p$}
Trans. of the AMS \textbf{355}, 7, (2003) 2651--2662.

\bibitem{DP}  
\textsc{David, Chantal; Pappalardi, Francesco},
\textit{Average Frobenius Distribution of Elliptic Curves}, 
Internat. Math. Res. Notices \textbf{4} (1999) 165--183.

\bibitem{GM}
\textsc{Gupta, Rajiv; Murty M. Ram},
\textit{Primitive points on elliptic curves},
Compositio Mathematica \textbf{58}, n. 1 (1986), 13--44.

\bibitem{GM2}
\textsc{Gupta, Rajiv; Murty M. Ram},
\textit{Cyclicity and generation of points mod $p$ on elliptic curves},
Inventiones mathematicae \textbf{101} 1 (1990)  225--235

\bibitem{LT1}
\textsc{Lang, Serge; Trotter, Hale},
Frobenius distributions in ${\rm GL}_{2}$-extensions.
Lecture Notes in Mathematics, Vol. 504. \textit{Springer-Verlag}, Berlin--New York, 1976

\bibitem{LT2}
\textsc{Lang, Serge; Trotter, Hale},
\textit{Primitive points on elliptic curves}. Bull. Amer. Math. Soc. 
\textbf{83} (1977), no. 2, 289--292.

\bibitem{MMS}
\textsc{Murty, M. Ram; Murty, V. Kumar; Saradha, N.},
\textit{Modular Forms and the Chebotarev Density Theorem,}
American Journal of Mathematics, \textbf{110}, No. 2 (1988), 253--281

\bibitem{S1}
\textsc{Serre, Jean-Pierre}, 
Abelian $\ell$-adic representations and elliptic curves. 
With the collaboration of Willem Kuyk and John Labute. 
Second edition. Advanced Book Classics. Addison-Wesley Publishing Company, 
\textit{Advanced Book Program}, Redwood City, CA, 1989.

\bibitem{S2}
\textsc{Serre, Jean-Pierre},
\textit{Propri\'et\'es galoisiennes des points d'ordre fini des courbes elliptiques}. 
(French) Invent. Math. \textbf{15} (1972), no. 4, 259--331.
\rho
\bibitem{S3}
\textsc{Serre, Jean-Pierre},
\textit{Quelques applications du th\'eor\`eme de densit\'e de Chebotarev}. (French) 
Inst. Hautes \'Etudes Sci. Publ. Math. No. \textbf{54} (1981), 323--401. 
\end{thebibliography} 

\end{document}
